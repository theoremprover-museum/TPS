\chapter{How to define wffs, abbrevs, etc}
\label{wffs}

There are several ways to define wffs so that they persist and
can be reused, saving typing and allowing them to be used to
extend the logical system.  For examples of how to represent
wffs, see sections 1.6 and 5.1.3 of the {\ETPS} User's Manual.

Abbreviations must usually be capitalized; for example, one should write
`ASSOCIATIVE p(AAA)' rather than `associative p(AAA)'. (Specifically, abbreviations
must be capitalized unless the \indexother{FO-SINGLE-SYMBOL} property is T. See chapter \ref{library}
for more details.) For reasons
to do with the internal workings of {\TPS}, please avoid using superscripts
on variables in wffs, and do not use the variables h or w in wffs or
abbreviations. In general, it is better (and certainly more readable)
to use lowercase letters for variables and uppercase for constants.

The first type of wff is good for wffs which you think of as static,
and for which you just want a short name.  These are called weak labels,
and can be created by the editor command \indexedop{CW}.  They may then be
stored in a file with the editor command \indexedop{SAVE}.  Other editor
commands allow you to delete the labels, replacing the short name
with its definition.  Weak labels are really weak.  When you wish to
refer to a weak label inside another wff, you must put the weak label
in all capitals, so that the parser will know how to find it when trying
to parse the wff.  In addition, operations like $\lambda$-contraction will
cause weak labels to be deleted in favor of their definitions.  In
short, weak labels are just a convenient shorthand for wffs and you
can use them to save typing.  They may also be saved in library files;
see chapter \ref{library}. In the editor, command names which end in `*'
are recursive; they will find the outermost appropriate node(s) and apply
the operation there. Non-recursive commands generally only apply to the current
node.

Some new wffs are really extensions of the system.  These denote operators
such as {\tt SUBSET}, {\tt INTERSECT}, etc.  Such wffs are called
abbreviations, and they can be made polymorphic, so that they can be
used for any types.  These wffs are more persistent than weak labels.
The parser will recognize them even if in lower case letters, and they
will be instantiated with their definitions only when specifically
required.  They may be saved in library files; see chapter \ref{library}
for more information.

Other wffs you might wish to save are exercises and theorems.  These
behave much like weak labels, but can have more properties.  At the
present time, these must be placed in files manually.  See the source
files {\tt ml1-theorems.lisp} and {\tt ml2-theorems.lisp} for how they
are defined.

Within the editor, you can record wffs as they are created as follows.
(By creating weak labels for the wffs you generate in the editor, you can
also save them in the library, which is more useful in that it allows you
to read them back into {\TPS} at a later date.)
If \indexflag{PRINTEDTFLAG} is T, the editor will every once in a while
write the edwff into a file \indexflag{PRINTEDTFILE} (global variable,
initialised to \indexfile{edt.mss}).  The criterion can be set as a lisp function, but at
the moment it will write whenever you call an edop whose result replaces
the current edwff.  Moving wffs (like A,D,L,R etc), non-editor commands
and printing command do not cause the
new edwff to be written.  Just so you can keep track of when things
are written, the prompt in the editor is modified.  Whenever
\indexflag{PRINTEDTFLAG} is T, it will either be
{\tt <-Edn>} or {\tt <+Edn>}
where the former means nothing has been written and the latter means the wff
which is now current has just been written. The wffs are appended to the end of
the file, which is written in style SCRIBE; also see the help messages
for the commands \indexcommand{O} and \indexcommand{REM}.

If you change the flag \indexflag{PRINTVPDFLAG} to T, it will print a
vertical path diagram into the file given by \indexflag{VPD-FILENAME} whenever
i