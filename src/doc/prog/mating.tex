\chapter{Mating-Search}

The top level files for matingsearch are:
{\it mating-dir.lisp} for ms88, 
{\it ms90-3-top.lisp} for ms90-3,
{\it option-tree-search.lisp} for ms89 and ms90-9, {\it ms91-search.lisp} for
ms91-6 and ms91-7, and {\it ms92-9-top.lisp} for ms92-9 and {\it ms93-1.lisp} 
for ms93-1.  The lisp files with prefix {\it ms98} are those used by ms98-1.
The code for \indexcommand{GO} in {\it mating-top.lisp} shows 
what the main functions are.
Mating search with extensional expansion dags 
is different in many respects than mating search with expansion trees.
We delay this discussion until section~\ref{sect:edags}.

There are a lot of comments about the workings of the code embedded in the
lisp files; in particular there is an outline of ms90-3 at the top of
{\it ms90-3-top.lisp}.

\section{Data Structures}

See the section on flavors and labels (section ~\ref{labels}) for a discussion
of some relevant information about the data structures below. Among other things, 
that section has the definition of the flavor "etree".

\subsection{Expansion Tree}
\label{etrees}

The data structure \indexData{etree}, defined in 
\indexfile{etrees-labels.lisp}, has the following properties:

\begin{enumerate}
\item \indexother{name}: the name of the etree. We can use this attribute to identify
which kind of structure an etree is.

\item \indexother{components}: is a list which contains all children of the etree. 
The children of an etree are also etrees. We could use this attribute to check whether an 
etree is a leaf, true, or false. 

\item \indexother{positive}: 
tells us whether the formula which an etree represents (which is the formula 
given by \indexfunction{get-shallow}) appears positively or negatively in the 
whole formula. This will be used to compute the vpform of the whole formula.
(The vpform of a subformula may be not the same as the corresponding part of it 
in the whole formula because the "positive" property
of the subformula is dependent on the context.)

\item \indexother{junctive}: can be used for printing the vpform. 
This attribute is linked tightly with the "positive" attribute, and has to do with
whether the node acts as neutral, conjunction or disjunction.

\item \indexother{free-vars}: is a list, containing the free variables in whose scope 
the node appears. When you skolemize a formula, you should use this attribute.
 
\item \indexother{parent}: is the parent of this etree.

\item \indexother{predecessor}: this slot tells you the leaf name from which the current
etree was deepened. It is mainly used for handling skolem constants.

\item \indexother{status}: has little to do with the system as currently 
implemented, but you should be
careful when you are creating commands which will change the variable 
{\tt current-topnode}. You have two choices:
\begin{enumerate}
\item Change the value of {\tt *ignore-statuses*} to T. Then you need
not worry about this attribute. Of course,
what you are doing may then not be compatible with the future versions
of the system. This is highly discouraged.

\item When you want to create new nodes or change some nodes in the {\tt current-topnode}
make the corresponding changes in the attribute {\tt statuses} of {\tt current-eproof}, 
which is a hash-table. Don't forget this, otherwise your new commands won't work.
\end{enumerate}
\end{enumerate}

Actually, according to an old \indexother{email from Dan Nesmith about status},
the status of etrees
is {\it not} stored in the status slot of an etree.  Instead the status
of an etree is in a hash table associated with the current-eproof.
(So, update-statuses depends on the value of current-eproof.)

The same email contains information on the predecessor slot.
For reference, here is the email:\label{dans-mail}

\begin{verbatim}
To: Peter.Andrews@K.GP.CS.CMU.EDU
Cc: issar@K.GP.CS.CMU.EDU, hwxi@K.GP.CS.CMU.EDU, mbishop@K.GP.CS.CMU.EDU
Subject: Re: STATUS and PREDECESSOR 
In-Reply-To: Your message of "Fri, 25 Sep 92 16:05:44 EDT."
             <1992.9.25.20.0.41.Peter.Andrews@K.GP.CS.CMU.EDU> 
Date: Tue, 29 Sep 92 16:34:36 +0100
Message-Id: <29531.717780876@js-sfbslc10.cs.uni-sb.de>
From: "Dan Nesmith" <nesmith@cs.uni-sb.de>

Your message dated: Fri, 25 Sep 92 16:05:44 EDT
>	Can you please explain what the slots STATUS and PREDECESSOR
>in the structure current-topnode are for? (Of course, maybe we should
>have a meeting with Sunil to have a discussion about this.)

Sorry it took so long to reply.  At first glance, I thought this was
something to do with unification.

current-topnode is a variable whose value, while you are in
the mating-search top-level, is the node of the expansion tree that you
are currently looking at.  Commands like D (down), UP, ^ (move to root),
use and change the value of this variable (see etrees-wffops, mating-top, 
mating-move).  

Actually, the unification top-level uses this same variable name, of course
rebinding it during the duration of the top-level (so there's no real
conflict).

Now, STATUS and PREDECESSOR are actually slots in every expansion tree
node (defined in etrees-labels).  

Each etree node has a status, which is a nonnegative integer.  0 means
that the etree node should be ignored (as if it and its descendants
were not in the tree), while positive values indicate the node is active,
and (potentially) the higher the value, the more important it is.  
By the etree's status, you could rank certain expansions as more interesting
than others.  I don't think that is now being used anywhere.
Originally, this status was kept in a slot in each etree node.  I didn't 
really like this, because then you can't share etree nodes among different
"virtual" expansion trees.  For example, during the MS90-9 search procedures,
there is really just one expansion tree, which contains all of the 
expansions.  There are, however, many "virtual" expansion trees, that is,
expansion trees with the same root, but with different subsets of the
expansions "turned on".  Each one of these virtual trees is kept in a
separate eproof structure.  For this reason, the statuses are actually
kept in a hashtable in the eproof structure as well, so changing the status
of a node in one virtual tree doesn't affect its status in other trees. 
E.g.,
Assume we have a tree with root expansion node EXP0, and children LEAF1, LEAF2.
Then we have potentially 3 virtual trees: one where LEAF1 has positive status
and LEAF2 has 0 (is not there); one where LEAF2 positive status and LEAF1 has 0;
and one where both LEAF1 and LEAF2 have positive status (are thus both
considered in the proof process).  Functions that do things like create the
jform use the status to decide which nodes belong and which don't.

Because statuses are now kept separate from the nodes themselves, the STATUS 
slot is an anachronism, and actually can now be removed (delete the form
"(status 1)" from the file etrees-labels.lisp).

PREDECESSOR is related.  This is a symbol, the name of the etree node from
which this node originated.  For example, suppose we have a leaf node LEAF0.
If we deepen this node, then we will get something like EXP0 as a result.  Its
PREDECESSOR slot will be LEAF0. If we then change the status of all its 
expansions to 0, then this node is effectively a leaf node again, and it 
will be printed out with the name LEAF0 as before.  E.g.
<34>mate "exists y P y" 

DEEPEN (YESNO): Deepen? [Yes]>no

<Mate35>etd

LEAF0   EXISTS y(I) P(OI) y

<Mate36>dp

EXP0
<Mate37>etd

EXP0   LEAF1    y^0(I)
LEAF1   P(OI) y^0(I)

<Mate38>1

LEAF1
<Mate39>mod-status 0


<Mate40>up

LEAF0
<Mate41>etd

LEAF0   EXISTS y(I) P(OI) y


PREDECESSOR is also used in case a node's name is not found in the
statuses hashtable; so effectively a node can inherit the status of the
node from which it was created.

Dan
\end{verbatim}

The file \indexfile{etrees-debug} contains functions useful for debugging
code dealing with etrees.  
The function \indexfunction{check-etree-structure}
recursively checks structural properties of an etree,
and the function \indexfunction{check-etree-structure-break}
calls \indexfunction{check-etree-structure} and calls a break
if the etree fails the structural test.
The idea is that one can temporarily insert 
\begin{verbatim}
(check-etree-structure-break <etree>) 
\end{verbatim}
in suspicious parts of the code
to find out when an etree loses its integrity.  If the etree does not
have structural integrity, a break is called, sending the user (programmer)
to the debugger.  If one wants to insert this in several places in the code,
one may want to include a message as in
\begin{verbatim}
(check-etree-structure-break <etree> "unique identifying message")
\end{verbatim}
to identify which caused the break.

\subsection{The Expansion Proof}
In the mate toplevel, we have an \indexother{expansion proof} stored in the 
special variable
\indexother{current-eproof}, which is an \indexData{eproof}-structure. 
\indexother{current-eproof}
has a attribute \indexData{etree}, whose value is often used to update
variable \indexother{current-topnode}.

Actually, a whole formula is represented by a tree, each node of which is an 
\indexData{etree}.
At first, \indexother{current-topnode} is the root of the tree. 
Each node in the tree can be one of the following structures,
all of which are derived from the structure \indexData{etree}, described 
above. We note only the differences between these structures and etrees.

\begin{enumerate}
\item \indexother{econjunction} is just an etree without 
any additional new attributes. \indexother{components} is a list containing two 
elements, and \indexother{junctive} should be {\tt dis} or {\tt con}.

\item \indexother{edisjunction} is like \indexother{econjunction}.

\item \indexother{implication} is like \indexother{econjunction}

\item \indexother{negation} is just an etree. \indexother{components} contains one element
and \indexother{junctive} is {\tt neutral}.

\item \indexother{skolem} is an etree with two additional attributes:
\begin{enumerate}
\item \indexother{shallow}: contains the shallow formula that the attribute 
\indexother{skolem} represents.
Never forget to make the corresponding changes in it if you have changed some 
other parts of this node; otherwise the proof cannot be transformed into natural 
deduction style by \indexcommand{etree-nat}
since the function \indexfunction{get-shallow} would not work normally.

\item \indexother{terms}: is a \indexData{skolem-term} structure, containing
a term replacing the original variable, and something else.
\end{enumerate}

\item \indexother{selection} is also an etree with attributes \indexother{shallow}
and \indexother{terms}, just as \indexother{skolem} etree nodes.  Whether the
etree contains \indexother{selection} or \indexother{skolem} nodes depends on
the values of \indexflag{SKOLEM-DEFAULT}.

\item \indexother{expansion} is an etree with three additional properties:
\begin{enumerate}
\item \indexother{shallow}: is the same as in \indexother{skolem}.
\item \indexother{terms}: is an \indexData{exp-var} structure, containing the 
expansion variable for this expansion.
\item \indexother{prim-vars}
\end{enumerate}

\item \indexother{rewrite} is an etree that rewrites the wff in some way.
Rewrite nodes have the following four additional attributes:
\indexother{shallow}, \indexother{justification}, \indexother{ruleq-shallow},
and \indexother{reverse}.  The \indexother{justification} attribute is
a symbol which can currently be one of the following values (this list may not be exhaustive):
\begin{enumerate}
\item \indexother{EQUIVWFFS}:  Usually means there have been some definition expansions.
In case dual instantiation is being used, it may mean the wff has been rewritten to
a conjunction or disjunction of the wff and the instantiated form.
\item \indexother{LAMBDA}, \indexother{BETA}, \indexother{ETA}:  The wff is the result of the appropriate normalization.
\item \indexother{EQUIV-IMPLICS}, \indexother{EQUIV-DISJS}:  An equivalence was expanded.
\item \indexother{LEIBNIZ=}:  Rewrites an equational wff using the Leibniz definition of equality.
\item \indexother{EXT=}:  Rewrites an equational wff between terms of functional type using extensionality.
\item \indexother{REFL=}: Rewrites an equational wff of the form ``a = a'' to TRUTH.
\item \indexother{RULEQ}:  The only time this appears to be used is when \indexflag{MIN-QUANTIFIER-SCOPE}
is set to T, in which case the quantifiers in the wff are pushed in as far as possible.
\item \indexother{ADD-TRUTH}, \indexother{TRUTHP}:  May conjoin the wff with TRUTH, or disjoin the wff with (NOT . TRUTH).
See the flags \indexflag{ADD-TRUTH} and \indexflag{TRUTHVALUES-HACK}.
\end{enumerate}

\item \indexother{leaf} is an etree with the additional attribute \indexother{shallow},
as in \indexother{skolem}, above. The \indexother{components}, \indexother{junctive}
and \indexother{predecessor} attributes of \indexother{leaf} are all {\tt nil}.

\item \indexother{true}

\item \indexother{false}

\item \indexother{empty-dup-info}  is an etree used by the \indexcommand{NAT-ETREE}
translation code, not by the mating search.

\end{enumerate}

There is also an eproof stored in the global variable \indexother{master-eproof}.
In my experience, this has been set to the same value as \indexother{current-eproof}.
The only place I can find in the code where
it may have a different value is when using option sets (search procedures \indexother{MS91-6}
and \indexother{MS91-7},
see the files \indexfile{ms91-basic.lisp} and \indexfile{ms91-search.lisp}).
In particular, there are \indexother{option-set} structures which have an \indexother{eproof} slot.
These are set to {\it copies} of the eproof structure (as opposed to the identical
structure) in \indexother{master-eproof}.  Then, in \indexfunction{finish-up-option-search},
\indexother{current-eproof} is set to the value of such an \indexother{eproof} slot.

In addition to the etree slot,
there are numerous other slots associated with an eproof:
\begin{enumerate}
\item \indexother{jform}:  Contains the jform associated with the etree (see section~\ref{etree-to-jform}).

\item \indexother{all-banned}: A list of expansion
vars and a list of selected vars whose selection node occurs beneath the expansion term.
This is needed to check acyclicity condition.  The value is set by the function \indexfunction{fill-selected}.

\item \indexother{inst-exp-vars-params}:  An association list of expansion variables that occur instantiated in the etree
and the selected variables that occur in the term.  This is needed to check the acyclicity condition when
there are substitutions (e.g., set variable substitutions made in a preprocessing stage) that have
contain selected variables.  (See section ~\ref{acyclicity}.)

\item \indexother{dissolve}:  An alist of symbols representing connections between nodes in the etree
which we assume will be in the final solution.  The code for building jforms from etrees (\indexfunction{etree-to-jform},
\indexfunction{etree-to-prop-jform}) will use this, as well as the flag \indexflag{DISSOLVE}, to
dissolve vertical paths from the jform.  (Dissolution is described in~\cite{Murray93}.  The current
dissolution code dissolves one connection at a time, iterating the procedure for each connection.)

\item \indexother{free-vars-in-etree}:  This is an association list between free expansion variables
which occur in the eproof (i.e., those which have not be instantiated), and the corresponding expansion
node in which the variable was introduced.  Note that if expansion variables are \indexother{EXP-VAR}
structures.  An expansion variable is uninstantiated with the \indexother{VAR} slot is the same as
its \indexother{SUBST} slot.  When an expansion variable $p$ (introduced in expansion node $EXP_j$)
is instantiated with a term which introduces
new expansion variables $\{q_i\}$ (e.g., a PRIMSUB), the pair ($p$ . $EXP_j$) is removed from this slot
and the pairs ($q_i$ . $EXP_j$) is included in the list.  (Also, in such a case, the value of
the slots \indexother{substitution-list}, \indexother{inst-exp-vars-params}, 
\indexother{all-banned} may change to reflect this instantiation.)

\item \indexother{skolem-constants}:  An association list of skolem constants and integers representing their arities.
(Note: If \indexflag{SKOLEM-DEFAULT} is set to NIL, then all skolem constants will have arity 0.)

\item \indexother{substitution-list}:  A list of expansion variables which have been instantiated.

\item \indexother{leaf-list}:  A list of the leaf nodes occurring in the etree.

\item \indexother{skolem-method}:  Corresponds to the value of \indexflag{SKOLEM-DEFAULT}.

\item \indexother{max-cgraph-counter}

\item \indexother{bktrack-limit}

\item \indexother{connections-array}

\item \indexother{incomp-clists-wrt-etree}

\item \indexother{mating-list}

\item \indexother{incomp-clists}

\item \indexother{cgraph}

\item \indexother{skolem-node-list}:  A list of the skolem nodes which occur in the etree.

\item \indexother{stats}

\item \indexother{max-incomp-clists-wrt-etree}

\item \indexother{symmetry}:  A hash table with etree nodes as keys
and \indexother{symmetry-holder} structures as values.  This information
is built when the etree is deepened, but does not appear to be used
anywhere in the code.

\item \indexother{merged}:  This is true if the etree has been merged.

\item \indexother{statuses}:  A hashtable of nodes in the etree and their statuses.
See the discussion above, and Dan's email in section~\ref{dans-mail}.

\item \indexother{name}:  This symbol is the name of the eproof.

\item \indexother{lemmas}:  This slot contains information
about the lemma structure in the expansion proof.
See section~\ref{exp-pf-lemmas}.
\end{enumerate}

\subsection{Relevant Global Variables}

In addition to the eproof slots, there are some global variables 
which store information relevant to the current etree.

The following global variables are used by dual instantiation
(when
\indexflag{REWRITE-DEFNS} or \indexflag{REWRITE-EQUALITIES}
is set to \indexother{LAZY2} or, equivalently, \indexother{DUAL}).
\begin{enumerate}
\item \indexother{*instantiated-defs-list*}
The value is an association list of symbols and the shallow formula of the
rewrite in which the dual instantiation was performed.  For example,
the value might be
\begin{verbatim}
((#:G162733 ((SUBSET<O<OI><OI>> . c<OI>) . d<OI) 
(#:G162731 . ((=<OII> . a<I>) . b<I>)))
\end{verbatim}
The symbols have no apparent meaning, but are used internally as an identifier.
Note that this list contains both abbreviations and equations which have been
instantiated using dual instantiation.  The value of this global is built during
deepening (see section~\ref{deepening})
\item \indexother{*instantiated-eqs-list*}  
This global's value during deepening is an association list of symbols and the shallow
formula of the equation being rewritten (so its elements are a subset of the
elements of *instantiated-defs-list*).
However, the elements of this list are removed during deepening so that
the final value after deepening an etree to literals seems to always be NIL.
\item \indexother{*hacked-rewrites-list*}
Its value is a list of elements of the form 
\begin{verbatim}
(<rewrite node> . 
    (<instantiated wff-or-symbol> . <leaf with uninstantiated form>))
\end{verbatim}
\item \indexother{*banned-conns-list*}  This is an association list of leaves which are not to be mated, e.g.,
\begin{verbatim}
((L4 . L8) (L4 . L9) (L11 . L14) (L11 . L15))
\end{verbatim}
The value appears to be leaves which correspond to an uninstantiated definition and the leaves which
appear beneath the instantiated form.  Since these leaves share a vertical path, they could be mated.
Apparently, the intuition is that we never want to mate a wff with an uninstantiated defn with a subformula
of the instantiated form.  However, it is not clear that we {\it can} really rule out such connections,
since higher order quantifiers might cause a wff to be the negation of a subformula of itself.
However, we can legally ban these connections if only for the reason that we do not {\it need} to use
dual instantiation at all.
The value of this flag is used by \indexfunction{quick-unification-connection} to rule out some connections.
\item \indexother{*ho-banned-conns-list*}  Similar to *banned-conns-list*, but with a slightly different
representation.  Instead of pairs of leaves, each leaf corresponding to an uninstantiated definition
is associated with a list of the leaves that occur beneath the instantiated form.  For example,
\begin{verbatim}
((L4 . (L9 L8)) (L11 . (L15 L14)))
\end{verbatim}
\item \indexother{*unsubst-exp-vars*}  This is a list of the expansion variables in the etree
which are not instantiated.  This should usually be a list of the car's from the \indexother{current-eproof}
eproof slot \indexother{free-vars-in-etree}.  Note that this variable is only set if some node
was rewritten using dual instantiation.  Otherwise its value will be whatever it was the last time
an etree was deepened using dual instantiation.
\item \indexother{*rew-unsubst-exps*}  This is a list of expansion variables which occur free
in some leaf corresponding to an uninstantiated definition.
\end{enumerate}

These are some other global variables.
\begin{enumerate}
\item \indexother{*leibniz-var-list*}  An association list of variables and rewrite nodes.
The variables are expansion variables (actually, only
the symbol for the variable is used) introduced by the Leibniz definition of equality (i.e., the
$q$ in $\forall q \, . \, q \, A \, \supset \, q \, B$).  The rewrite node is the
rewrite node in which the equality was rewritten using the Leibniz definition.
Note that the $q$ is only an expansion variable if the rewrite node is positive.
\item \indexother{*after-primsub*}  This is just a toggle that is temporarily set to T
after a primsub has been done, so that it will be T while the new etree is deepened.
\end{enumerate}

There may be other global variables.  Needless to say, it is difficult to build an
expansion tree in any way other than using the deepening code that is already written
(see the file \indexfile{etrees-wffops.lisp} and section~\ref{deepening} of this chapter)
because all these global variables
and eproof slots need to be maintained.


\subsection{Functional Representation of Expansion Trees}\label{ftrees}

The implementation of expansion trees described above
is well-designed for mating
search on a single tree.  An expansion tree has a great deal of global information
associated with it such as what expansion variables and selected variables
occur in the tree.  Also, expansion trees contain circular references, since
each node is associated with both its children and its parent.
(In particular, since a node is associated with its parent, we
cannot coherently share a node in two expansion trees.)
Unfortunately, this makes it very difficult to build
and modify expansion trees using recursive algorithms.
One must constantly update global information.

There is an alternative representation called ``ftrees'' implemented
in the file \indexfile{ftrees}.  Ftrees are designed for functional
programming (the ``f'' is for ``functional'').  Operations on
ftrees are never destructive, and the information carried at
each node is minimal.  Functions \indexfunction{etree-to-ftree}
and \indexfunction{ftree-to-etree}
translate between the two representations.

Finally, I was convinced that we needed a different representation
designed for functional programming.  I have implemented this
alternative representation (``ftrees'') and translations between the two representations.

The new representation also allowed me to write code to save and restore
expansion trees.  The commands are
\indexcommand{SAVE-ETREE} and
\indexcommand{RESTORE-ETREE}.

\subsection{Other Structures}
\begin{itemize}
\item A \indexData{mating} has certain attributes:
\begin{enumerate}
\item A set of connections

\item A unification tree
\end{enumerate}

\item A \indexData{unification tree} is a tree of nodes.

\item A \indexData{uni-term} is an attribute of a node (which is a structure); it is 
a set of disagreement pairs.

\item A \indexData{failure record} is a hashtable. 
MS88 (and the other non-path-focused procedures) uses the failure 
record.  MS90-3 (and the other path-focused procedures) does not use it (this is one reason
why, when \TPS abandons a mating and later returns to the partially completed 
eproof, ms91-6 continues approximately where it left off and ms91-7 does not).
Links (which all occur in the connection graph) are represented as numbers,
so sets of links are just ordered lists of numbers, and one can
efficiently test for subsets. 
Given a new partial mating M, \TPS just
looks at all the entries in the failure record to see if any of them
are subsets of M. 

\end{itemize}

\section{Operations on Expansion Trees}

\subsection{Deepening}\label{deepening}

The code for deepening an expansion tree is in the file \indexfile{etrees-wffops.lisp}.
The idea behind deepening an expansion tree is to find the leaves,
then destructively replace the leaves with a new node depending on the
structure of the shallow formula and the setting of many, many flags.
The key function to try to understand is \indexfunction{deepen-leaf-node-real}.
Suppose $A$ is the shallow wff of the leaf in question.
Here is a quick outline of what this function does:
\begin{enumerate}
\item If $A$ can be $\lambda$-reduced, create a rewrite node with a leaf of
the reduced form as its child.
\item If $A$ is $\lnot B$, then create a negation node with a leaf of $B$ as its child
(except sometimes when \indexflag{ADD-TRUTH} is set to T).
\item If $A$ is $B \land C$, then create an econjunction node with leaves for $B$ and $C$ as children.
\item If $A$ is $B \lor C$, then create an edisjunction node with leaves for $B$ and $C$ as children.
\item If $A$ is $B \limplies C$, then create an implication node with leaves for $B$ and $C$ as children.
\item If $A$ is $B \equiv C$, then create a rewrite node, rewriting the equivalence either as a conjunction
of implications or disjunction of conjunctions (depending on the values of \indexflag{MIN-QUANTIFIER-SCOPE}
and \indexflag{REWRITE-EQUIVS}
and the parity of the leaf).
\item If $A$ is a positive existential formula or negative universal formula,
create a skolem node or selection node (depends on the value of \indexflag{SKOLEM-DEFAULT})
with a leaf of the scope of the quantifier as its child.
\item If $A$ is a positive universal formula or negative existential formula,
create an expansion node with a leaf of the scope of the quantifier as its child.
\item If $A$ is an equation of the form $t = t$, 
and \indexflag{REWRITE-EQUALITIES} is not set to \indexother{NONE}, then 
create a rewrite node with a leaf of TRUTH as its child.
\item If $A$ is an equation,
and \indexflag{REWRITE-EQUALITIES} is not set to \indexother{NONE}, then 
rewrite the equation (depending on \indexflag{REWRITE-EQUALITIES}) and
create an appropriate rewrite node.
\item If $A$ is a symbol introduced by dual instantiation of an equality, replace it
with the instantiated formula.
\item Finally, if there is a definition, then there is a very complicated
case which creates an appropriate rewrite node.  Anyone trying to figure out
this part of the code needs to pay close attention to the value of
\indexother{REWRITE-DEFNS}.  Ordinarily, \indexother{REWRITE-DEFNS} is a flag
whose value is a list of a form such as 
\begin{verbatim}
(DUAL (EAGER TRANSITIVE) (NONE INJECTIVE SURJECTIVE))
\end{verbatim}
However, at the beginning of \indexfunction{deepen-leaf-node-real},
the value of \indexother{REWRITE-DEFNS} is dynamically set to
a form
\begin{verbatim}
((DUAL SUBSET REFLEXIVE) (EAGER TRANSITIVE) (NONE INJECTIVE))
\end{verbatim}
where all the abbreviations appearing the $A$ are explicitly in the list,
and those not appearing in $A$ are removed.
To make matters more confusing, rewrite-defns is dynamically set in this
case to a simple list of abbreviations which may be rewritten, before
calling the function \indexfunction{contains-some-defn}.  This would be a value
such as
\begin{verbatim}
(SUBSET REFLEXIVE TRANSITIVE)
\end{verbatim}
\end{enumerate}
The deepening code also sets many global variables as well as eproof slots
in current-eproof.  The code really assumes we are deepening the etree
in the etree slot of \indexother{current-eproof}.

\section{Skolemization}

There are three \indexother{skolemization} procedures in \TPS; SK1, SK3 and NIL.
(Actually, the latter is not skolemization at all, but the selection nodes method from Miller's thesis. 
However, it still uses skolem constants internally.) 
The flag \indexflag{SKOLEM-DEFAULT} decides which one will be used in a proof, and
the help message for that flag explains the difference.

We assume familiarity with the way that SK1 is handled in TPS. SK3 is broadly 
similar; the only difference between the two is in the function \indexfunction{create-skolem-node}, where
the skolem variables are chosen differently.

NIL, the selection node method, is very different. Selections are represented as Skolem
constants with no arguments, and we now describe the additional machinery needed to make the search 
procedure work in this case.

During simplification (in unification), the requirement that a certain relation should be acyclic
is checked. The exact statement of this relation is given in Miller's thesis; we implement it
(roughly speaking) as a requirement that no substitution term for an expansion variable should 
contain any of the selections which occur below that variable.

Extra slots, called {\tt exp-var-selected} on expansion variables and {\tt universal-selected} on 
universal jforms, are used to record all of the 
selections below each quantifier. This is used in the unification check, and more crucially
in path-focused duplication (since skolem terms are stripped out of the jform, this is our only 
way to remember where they were).  See the section on selected variables for more information
about checking acyclicity of the dependency relation.

In SK1 and SK3, duplicating a quantifier above a skolem term produces a new skolem term 
consisting of the same skolem constant applied to different variables. In NIL, we obviously 
can't use the same skolem constant everywhere (consider EXISTS X FORALL Y . P X IMPLIES P Y ; 
if we persistently select the same Y every time we duplicate X, the proof will fail). This 
has two major consequences:

\begin{itemize}
\item Path-focused duplication has to be changed. We can no longer duplicate implicitly by changing the
name of the quantified variable; we must now make a copy of the entire scope of the 
quantifier and descend into it, renaming all the selections as we go. These copies are stored
in a chain using the universal-dup slot of the jform (so the universal-dup of the top jform
contains the first copy, whose universal-dup contains the second copy, and so on). These
duplications are preserved during backtracking, in case they are needed again later; we 
use universal-dup-mark to remember how many of them are "really" there.

\item The procedure for expanding the etree after ms90-3 finishes a search has to be completely
replaced. The old procedure relied on the fact that the names of skolem constants never changed,
and so it was possible to attach all of the expansion terms to the jform and then duplicate and
deepen the etree while applying the appropriate substitutions. The names of selections {\it do}
change; this makes the substitutions incorrect (because they will contain the names of old selections).
So we use ms90-3-mating and dup-record to duplicate the etree directly, and then add the
correct connections to it using ADD-CONN. This procedure is probably still buggy.
\end{itemize}

\section{Checking Acyclicity of the Dependency Relation}\label{acyclicity}

\subsection{The Dependency Relation}

An expansion proof is given by an expansion tree and a mating.
One of the conditions on the expansion tree is that the
so-called ``dependency relation'' is acyclic (irreflexive).
For an expansion tree $Q$, let $S_Q$ be the set of selected
variables in $Q$ and let $\Theta_Q$ be the set of occurrences of expansion terms.
The following definition can be found in Miller's thesis.

{\bf Definition.}  Let $Q$ be an expansion tree.  Let $<^0_Q$ be the binary
relation on $\Theta_Q$ such that $t<^0_Q s$ if there exists $y\in S_Q$
so that $y$ is selected in a node dominated by $t$ and $y$ is free in $s$.
The transitive closure
$<_Q$ of $<^0_Q$ is called the {\it dependency relation}.

While the relation above is defined in terms of expansion terms,
the way TPS actually searches for a proof is as follows.
Expansion variables are used in place of expansion terms,
and TPS finds instantiations for these variables.  These
instantiations arise from two sources: pre-processing (usually
giving set variable instantiations)
and unification.  The instantiations made during pre-processing may contain expansion
variables which will later be instantiated by unification.

% In procedures designed before 2000 . . .
During the mating search, unification constructs substitutions for
expansion variables, and checks the acyclicity condition for those 
% ******* clarify this, the acyclicity condition is on etree's 
substitutions.  To be sound, TPS should include the substitutions
made in pre-processing as well.  Until now, TPS did not include these
substitutions in the check.  This did not cause a problem with soundness
because the omitted substitutions (all PRIMSUBS/GENSUBS) did not
contain any selected variables.

The most obvious way to ensure soundness is to include all the
substitutions in the acyclicity check, not only those arising from
unification.  However, since the instantiations made before search
begins will not change during search, we should be able to find a more efficient
method.  The idea is to start with an expansion tree obtained after pre-processing, i.e.,
after instantiations have been made and the resulting tree has been deepened.

{\bf Definitions.}  Let $Q$ be a given expansion tree.  
\begin{itemize}
\item Let $\Sigma_Q\subseteq \Theta_Q$ be the set of occurrences of expansion terms in $Q$
which are {\it not} expansion variables.
\item Let $V_Q$ be the set of expansion variables
that occur in $Q$.  Note that $V_Q$ and $\Sigma_Q$ are disjoint sets.
\item For each expansion variable $v$, let $T_v \subseteq \Theta_Q$ be the set
of expansion terms in which $v$ occurs free.  Note that $T_v$ is always nonempty.
\item For each $t\in \Sigma_Q$, let $T_t = \{t\} \subseteq \Theta_Q$.
\item For each $q\in\Sigma_Q \cup V_Q$, let $B(q)$ be the set of selected variables
$y$ whose selection node is dominated by the arc corresponding to $t$ for some $t\in T_q$.  
\item For each $t\in\Sigma_Q$, let $S(t)$ be the set of selected variables which
occur free in $t$.
\end{itemize}

The set $B(t)$ is the set of ``banned'' selection variables for $t\in\Sigma_Q\cup V_Q$.

The following definitions depend upon a substitution $\theta$.  This corresponds
to the substitution found by unification during mating search.

{\bf Definitions.} Let $\theta$ be a given substitution for the expansion variables in $V_Q$.
(Note that we allow $\theta(v) = v$ for some $v\in V_Q$ in order to make $dom(\theta) = V_Q$.)
\begin{itemize}
\item For each $v\in V_Q$, let $S(v)$ be the set of selected variables which occur
free in $\theta(v)$.
\item Define a relation $<^0_{Q,\theta}$ on $\Sigma_Q\cup V_Q$ by
$q <^0_{Q,\theta} r$ iff there exists a $y\in B(q)\cap S(r)$.
\item Let $<_{Q,\theta}$ be the transitive closure of $<^0_{Q,\theta}$.
\item The substitution $\theta$ is acyclic with respect to $Q$ if
the relation $<_{Q,\theta}$ is acyclic
(irreflexive).
\end{itemize}

{\bf Remark 1.}  It is important to note that $\theta$ is a substitution which does not create
any new nodes in $Q$.  (This is in contrast to substituting and then deepening the tree.)
In particular, the set of selected variables is the same, i. e., $S_Q = S_{\theta(Q)}$.  
Also, for the same reason, for each $t\in\Sigma_Q$, $B(t)$ is the set of selected variables $y$
whose selection node is dominated in $\theta(Q)$ by the expansion term occurrence $\theta(t)$.  
Similarly, for $v\in V_Q$, $y\in B(v)$ means there is some expansion term $t\in\Theta_Q$
such that $v$ occurs free in $t$ and $y$ is dominated in $\theta(Q)$
by $\theta(t)$.

{\bf Remark 2.}  Note that we do not consider $\lambda$-terms equivalent up to $\lambda$-conversion.
The expansion tree must have explicit rewrite nodes to $\lambda$-normalize formulas.
For this reason, if $y$ is free in a term $t$ in $Q$, it will also be free in $\theta(t)$ in $\theta(Q)$.
(That is, we can never project the $y$ away because this would require not just substitution,
but also deepening $\theta(Q)$.)

{\bf Lemma 1.}  Let an expansion tree $Q$ and a substitution $\theta$ for $V_Q$ be given.
Suppose $q <_{Q,\theta} q'$.
Then there exists a $t\in T_q$ such that for every $t'\in T_{q'}$
$$\theta(t) <_{\theta(Q)} \theta(t').$$

{\bf Proof.}  By induction on the number of transitivity steps.

For the base case, suppose $q <^0_{Q,\theta} q'$.  Then, there is some
$y\in B(q) \cap S(q')$.  Since $y\in B(q)$, there is some $t\in T_q$
such that $y$ is dominated by $t$ in $Q$.  So, $y$ is dominated by $\theta(t)$
in $\theta(Q)$.  Now, suppose $t'\in T_{q'}$.  We consider two cases.

If $q'\in \Sigma_Q$, then $t' = q'$ and $y\in S(q')$.  So, $y$ is free in $t'$.
By Remark 2, $y$ is free in $\theta(t')$ and we are done.

If $q' \in V_Q$, then $q'$ is free in $t'$.
Also, $y\in S(q')$ implies $y$ is free in $\theta(q')$.
From this we have $y$ is free in $\theta(t')$ and we are done.

For the induction step, suppose we have $q <_{Q,\theta} q_1 <_{Q,\theta} q'$.
By induction, we have some $t \in T_q$ such that for any $t_0 \in T_{q_1}$, $\theta(t) <_{\theta(Q)} \theta(t_0)$.
Also, we have some $t_1 \in T_{q_1}$ such that for any $t' \in T_{q'}$, $\theta(t_1) <_{\theta(Q)} \theta(t')$.
In particular, we have $\theta(t) <_{\theta(Q)} \theta(t_1) <_{\theta(Q)} \theta(t')$
for any $t'\in T_{q'}$. $\Box$

{\bf Lemma 2.}  Let an expansion tree $Q$ and a substitution $\theta$ for $V_Q$ be given.
Suppose we have $t, t'\in \Theta_Q$ with $\theta(t) <_{\theta(Q)} \theta(t')$.
There is a $q'\in \Sigma_Q \cup V_Q$ with $t' \in T_{q'}$ such that for any 
$q\in \Sigma_Q \cup V_Q$ with $t \in T_{q}$ 
we have $q <_{Q,\theta} q'$.

{\bf Proof.}  By induction on the number of transitivity steps.

For the base case, suppose $\theta(t) <^0_{\theta(Q)} \theta(t')$.
Let $y$ be a selected variable dominated by $\theta(t)$ and free in $\theta(t')$.
Since $y$ is free in $\theta(t')$, we must either have $y$ free in $t'$
or $y$ free in $\theta(v)$ for some $v\in V_Q$ free in $t'$.
In the first case, let $q' = t'$.  In the second case, let $q'$ be some $v \in V_Q$
where $y$ is free in $\theta(v)$ and $v$ is free in $t'$.
So, we have $y\in S(q')$.  Suppose we have any $q$ with $t\in T_q$.
Since $y$ is dominated by $t$, we have $y\in B(q)$ and we are done.

For the induction step, suppose $\theta(t) <^0_{\theta(Q)} \theta(t_1) <^0_{\theta(Q)} \theta(t')$.
By induction there is a $q'$ with $t'\in T_{q'}$ such that for any $q_0$ with $t_1\in T_{q_0}$,
we have $q_0 <_{Q,\theta} q'$.
Also by induction there is a $q_1$ with $t_1\in T_{q_1}$ such that for any $q$ with $t\in T_q$,
we have $q <_{Q,\theta} q_1$.
Together we have $q <_{Q,\theta} q_1 <_{Q,\theta} q'$ for any $q$ with $t\in T_q$. $\Box$

{\bf Proposition.}  Given an expansion tree $Q$ and a substitution $\theta$ for $V_Q$,
the dependency relation for $\theta(Q)$ is acyclic iff $\theta$ is acyclic with respect to $Q$.

{\bf Proof.}  Suppose we have $q <_{Q,\theta} q$ for some $q\in \Sigma_Q \cup V_Q$.
By Lemma 1, there is a $t\in T_q$ such that for any $t'\in T_q$, $\theta(t) <_\theta(Q) \theta(t')$.
In particular, $\theta(t) <_\theta(Q) \theta(t)$.

Suppose we have $s <_{\theta(Q)} s$ for some expansion term $s$ in $\theta(Q)$.
Since $\theta(Q)$ is obtained from $Q$ by substitution (and no deepening), there
is a unique expansion term occurrence $t\in \Theta_Q$ such that $s = \theta(t)$.
By Lemma 2, there is a $q'\in \Sigma_Q \cup V_Q$ with $t\in T_{q'}$ such that for 
any $q\in \Sigma_Q \cup V_Q$ with $t\in T_q$ we have $q <_{Q,\theta} q'$.
In particular, $q' <_{Q,\theta} q'$. $\Box$

After pre-processing, we can compute the set $\Sigma(Q)$ and $V_Q$, as well as
the sets $B(t)$, $B(v)$, and $S(t)$ for each $t\in\Sigma(Q)$ and $v\in V_Q$.
So, to check that the
acyclicity condition is satisfied when unification generates
a substitution $\theta$, it suffices to compute $S(v)$ for each $v\in V_Q$
with respect to $\theta$, and check for a $<^0_{Q,\theta}$-cycle. 

{\bf Efficiency Refinement.}  Clearly, if for some $s\in \Sigma(Q)$, either $B(s)$ or
$S(s)$ is empty, then $s$ cannot be part of a cycle with respect to any substitution
$\theta$, so we may disregard any such term.

\section{Expansion Tree to Jform Conversion}\label{etree-to-jform}
In {\it ms90-3-node}, the jform is computed directly from the etree without using 
the jform which may be stored with the etree. (It is not clear where or whether
that jform is used; it might be part of the interactive system.)

\indexfunction{msearch} does the search. It returns 
{\tt (dup-record ms90-3-mating unif-prob)}.
{\tt unif-prob} represents the solution to the unification problem, perhaps
as a substitution stack.
This triple is then handed to the processes that translate things back to
an expansion proof, call merging, and then translate to 
a natural deduction proof.
\indexfunction{msearch} looks at the flag \indexflag{order-components}; read
the help message for this flag for more information.

Each literal is a jform. One of its attributes is a counter
which gets adjusted to count how many mates that literal has; this is
compared with \indexflag{max-mates}.
The current jform being worked on is
essentially represented as a stack which is passed around as an 
argument. Indices are associated
with outermost variables which are implicitly duplicated. These
indices are also associated with literals in the scope of these
quantifiers to keep track of what copy of the literal is being mated.
It is only when unification is called that these indices are actually
attached to the variables to construct the terms unification must work
on.  (The functions \indexfunction{check-conn} and \indexfunction{conn-unif-p} in 
the file {\it ms90-3-path-enum}, and
related functions in that file, may be relevant here.)

The original code only created literals for leaves of the etree.
However, it is possible to mate arbitrary nodes of an etree (that share a vertical path)
if we include literals corresponding to these nodes.
Currently, the user may use the flag \indexflag{ALLOW-NONLEAF-CONNS} to specify
which nodes to include in the jform.  The flag \indexflag{ALLOW-NONLEAF-CONNS} takes
a list of symbols as its value.  If this list contains the symbol \indexother{ALL},
then every node will have a literal in the jform.  If this list the symbol \indexflag{REWRITES}
is in the list, then every rewrite node will have a literal in the jform (giving a jform
similar to the one dual instantiation produces, though dual instantiation affects the
structure of the etree).  If the name of any particular etree node is in the list, then
that etree node will have a literal in the jform.
The code also uses the slot \indexother{allow-nonleaf-conns} in the \indexother{current-eproof}
to decide which nonleaf etree nodes to include as literals in the jform.

After converting an etree into a jform, \TPS will perform dissolution (see~\cite{Murray93}) iteratively
on each connection in the flag \indexflag{DISSOLVE} and in the \indexother{dissolve} slot
of the \indexother{current-eproof}.  The resulting jform will not have any vertical paths
that pass through these connections.

\section{Path-Enumerator}

\subsection{Duplication Order}
Along a path the procedure stops at the first eligible universal
jform. A slot {\it dup-t} in a universal jform tells whether it is
eligible. Then it starts from there to find the innermost
universal jform on the path under the currently picked one, and 
uses the innermost one as its candidate for next duplication. This
is fulfilled by calling function {\it find-next-dup}.

When testing, please set flag max-dup-paths to an appropriate value
so that you can suppress some unnecessary quantifier duplications.
It may save a lot of your searching time and make you aware if
you are on the right track. Always duplicating innermost quantifiers
has the following advantages.
1) producing shorter and clearer proofs, and
2) lowering the values of flags max-search-depth, max-mates, and num-of-dups,
sometimes.

\subsection{Backtracking}
When bactracking starts, the search procedure removes the
last added connection. A path attached to the connection tells
the procedure where it should pick up the search. This works
efficiently since the following claim is almost always true:
With the help of disjunction heuristic, the number of paths
used to block jform is often a very small fraction of the
whole paths in the jform. This means that it is not a big
burden to carry the paths around all the time during searching.
The advantage is that the procedure knows exactly where it is
without having to do heavy computation by using the information
given by the current mating. To make this work, also carried with a path
is an environment, which stores the indices and (partial) substitutions
for the variables in the path.

\section{Propositional Case}
In the file {\it mating-dir.lisp}, you can see that the function 
\indexfunction{ms-director}
checks whether there are any free variables in \indexother{current-eproof}
(seealso eproof, current-eproof)
in order to decide whether to call \indexfunction{ms} or 
\indexfunction{ms-propositional};
if there are no free variables in \indexother{current-eproof}, ms-propositional is
called.

\subsection{Sunil's Propositional Theorem Prover}

The original files for Sunil's fast propositional calculus theorem-prover
are in \\
{\it /home/theorem/project/tps-variants/si-prop/}. {\tt qload} these files, go into the
editor, and make the edwff the example you wish to run. Within the editor
{\tt (test)} runs the program using edwff as argument. When it is done,
{\tt (test1)} shows the mating it found.

Most of the code in the above directory is now a permanent part of \TPS;
the function \indexfunction{prop-msearch} can be called from the {\tt mate} top 
level, and will display a correct mating for propositional jforms. The 
way to call it is: {\tt (auto::prop-msearch (auto::cr-eproof-jform))} (the latter
function is the internal name for \indexcommand{CJFORM}). For some reason, 
the propositional theorem prover is never used, except to reconstruct the 
mating after a path-focused duplication procedure has found it.

\section{Control Structure and Interface to Unification}

The non-path-focused-duplication (npfd) search procedures have
a \indexother{connection graph}, but the pfd procedures do not; the latter just
apply simpl to decide whether literals may be mated.
Sunil's disjunction heuristic (see below) is implemented for pfd search
procedures, but not for npfd.

The non-path-focused-duplication search procedures break a
jform with top-level disjuncts into separate problems, but the
path-focused-duplication search procedures do not.

When searching for a way to span a path, \TPS runs down the path
from the top, and considers each literal. As a mate for that literal, it
considers each literal which precedes it on the path.

When \TPS considers adding an essentially ffpair (pair of
literals which each start with a variable when one ignores any
negations) to the mating, it simultaneously considers both
orientations (choices for which literal will be negative and which
positive) of the ffpair.  Roughly speaking, it does this by putting a
disagreement pair corresponding to the ffpair into the leaves of the
unification tree, and proceeding with the unification process.  If
this process encounters a disagreement pair of the form <{\bf A},
\verb+~+{\bf B}>, where {\bf A} starts with a constant but {\bf B} does not, it
replaces this pair with {\w <\verb+~+{\bf A}, {\bf B}>} and continues. In this
way it finds whichever substitution works in a very economical
fashion.  When a success node is found for a complete mating, the
associated substitution determines the orientation of the ffpair in
the mating.

Here is some more detail about how this is actually implemented.
When TPS decides to mate a pair {L, K} of literals (which it considers	
as an unordered pair), it seeks to unify $\verb+~+\;$L with K, where L 
occurred before K on the path. Whenever the unification process
encounters a double negation, it deletes it. (Thus, in the case of a
first-order problem, TPS quickly starts to unify the atoms of the mated
literals.)

When the unification process encounters a flexible-rigid pair
(which we designate by <...f... , ...$\verb+~+\;$H...>)
where the flexible term has head variable f and the rigid term has
a head of the form $\verb+~+\;$H, the following substitutions for f are
generated:

\begin{enumerate}
\item Projections

\item $\lambda w^{1}\cdots\lambda w^{k}. \verb+~+ f^{1} \ldots$, where information is attached
to $f^{1}$ which does not permit a substitution of this same type (i.e.,
introducing a negation) to be applied to $f^{1}$.

\item $\lambda w^{1}\cdots \lambda w^{k}. f^{2} \ldots$, where information is attached
to $f^{2}$ which does not permit the first two of these types of substitution
to be applied to it. 
\end{enumerate}

(The information is stored by putting the variables into the lists
neg-h-var-list and  imitation-h-var-list.)
The restrictions on $f^{2}$ assure that the dpair which is essentially
<...$f^{2}$... , ...\verb+~+ H...> can only be used to generate new substitutions
for $f^{2}$ if other substitutions reduce \verb+~+H to a form which does not
start with a negation. 

\subsection{Sunil's Disjunction Heuristic}

\begin{enumerate}
\item If a matrix contains $[A \lor B]$, and $A$ has no mate, then no
mate for $B$ will be sought.

\item If a matrix contains  $[A \lor B]$, and $A$ has a mate, but no
mate for $B$ can be found, then the search will backtrack,
throwing out the mate for A and all links which were subsequently
added to the mating.
\end{enumerate}

{\bf Remark:}  This heuristic is also used by Matt Bishop's search
procedure ms98-1.  See his thesis~\cite{Bishop99a} for more details.

\section{After a Mating is Found}

Here is the sequence of events for pfd:

\begin{enumerate}
\item An expansion proof has been found.
A record (probably called DUP-RECORD) of indices for duplicated
variables, leaves, and connections is maintained by the search process.
The unification tree associated with the mating has a record of the
substitutions for variables.

\item Construct jform with final substitutions applied.
This uses all copies of variables needed for the final mating.

\item Duplicate expansion tree from the jform

\item Attach expansion terms to the expansion tree

\item Call propositional search to reconstruct the mating

\item Reorganize mating in ms88 form

\item Merge expansion proof

\item Translate expansion proof
\end{enumerate}

\section{How MIN-QUANT-ETREE Works}

After a proof is found, \TPS constructs an expansion proof tree.
The implementation of flag \indexflag{MIN-QUANT-ETREE} consists of the
following steps.


\begin{enumerate}
\item \TPS searches through the expansion proof tree to find if
there are primsubs which are not in minimized-scope form.
If \TPS finds some, it goes to step (2).

\item First, \TPS transform all the primsubs into their minimized-scope 
forms. In order to make sure that the expansion proof
tree is still a correct one, \TPS has to modify it. This is
done by calling two functions, namely, \indexfunction{one-step-mqe-bd} and
\indexfunction{one-step-mqe-infix}. Now \TPS goes to step (3).

\item Since the expansion proof tree is still a correct one, \TPS
can use a propositional proof checker to search for a mating.
This mating will be used to construct a proof in natural
deduction style.
\end{enumerate}

There are still potential bugs in the procedure, since
various rewrite nodes in an expansion proof can interfere with
flag \indexflag{MIN-QUANT-ETREE}. This has to be dealt with case by case.

\section{Lemmas in Expansion Proofs}\label{exp-pf-lemmas}

There are facilities to allow an expansion proof of a theorem $A$
to depend on lemmas.  For the simplest case, if a theorem $A$ depends
on a lemma $B$, then the expansion tree has shallow formula
$B\land [B\limplies A]$.  So, a complete mating for this expansion
tree gives a proof of $B$ and a proof of $B\limplies A$.
If an expansion proof does contain lemmas, merging and translation
must take this into account.  In general, lemmas can themselves
depend on lemmas.  The value of \indexother{lemmas} slot is
a list structure containing symbols.  We can recursively
describe these values as
\begin{verbatim}
<LEMMAS>::((<SYMBOL> . <LEMMAS>) . . . (<SYMBOL> . <LEMMAS>))
\end{verbatim}
We can describe how these values correspond to lemmas
in the expansion tree inductively.  An expansion tree
proving $A$ with lemmas corresponding to the value
\begin{verbatim}
((<SYM1> . <LEMMAS1>) . . . (<SYMn> . <LEMMASn>))
\end{verbatim}
where $n>0$ is of the form
$$
\econj{}{}
{\econj{}{}{Q_1}
{\etra{\cdots}{\econj{}{}{Q_{n-1}}{Q_n}}}}
{\eimp{}{}
{\econj{}{}{P_1}
{\etra{\cdots}{\econj{}{}{P_{n-1}}{P_n}}}}{Q}}
$$
where $Q$ has shallow formula $A$,
each $P_i$ has shallow formula $B_i$,
and each $Q_i$ is an expansion tree proving $B_i$
with lemmas corresponding to the value \verb+LEMMASi+.
Note that if \verb+LEMMASi+ is NIL, then the shallow
formula of $Q_i$ will be $B_i$, the same as that of $P_i$.
Otherwise, the shallow formula of $Q_i$ will be
of the form $[C \land [D \limplies B_i]]$.

% Extensionality example.
We may use lemmas to handle some extensionality reasoning.
Consider the example 
$$ \,P_{\greeko(\greeko(\greeko\greeki))} [ \lambda \,x_{\greeko\greeki} . \,A_{\greeko(\greeko\greeki)} \,x \lor \bot ] \supset \,P \,A$$
The jform for this example is of the form
\begin{verbatim}
|              L1              |
|P [LAMBDA x .A x OR FALSEHOOD]|
|                              |
|              L2              |
|             ~P A             |
Number of vpaths: 1
\end{verbatim}
We would like to mate \verb+L1+ with \verb+L2+, but we cannot since
they are not unifiable.  We need to use the fact that $A$ and
$\lambda x . A x \lor \bot$ are extensionally equivalent.
The mate command \indexcommand{ADD-EXT-LEMMAS} finds pairs
of such propositional or set or relation terms embedded inside
literals and includes an extensionality lemma for any two
such terms (occurring in literals of opposite polarity).
In this example, there are two extensionality lemmas
added to the expansion tree:
\begin{itemize}
\item $ \forall \,x_{\greeko\greeki} [ \bot \equiv \,A_{\greeko(\greeko\greeki)} \,x ] \supset \lambda \,x \bot \,= \,A$
\item $ \forall \,x_{\greeko\greeki} [ \,A_{\greeko(\greeko\greeki)} \,x \lor \bot \equiv \,A \,x ] \supset \lambda \,x [ \,A \,x \lor \bot ] \,= \,A$
\end{itemize}
This second lemma can be used to prove the theorem.
Both lemmas have an easy proof by expanding equality
using extensionality.  For example, the part of the expansion
tree corresponding to the proof of the second lemma is of the form
$$\eimp{1}{ \forall \,x_{\greeko\greeki} [ \,A_{\greeko(\greeko\greeki)} \,x \lor \bot \equiv \,A \,x ] \supset \lambda \,x [ \,A \,x \lor \bot ] \,= \,A}
{\uexpnode{0}{\forall \, x [\,A_{\greeko(\greeko\greeki)} \,x \lor \bot \equiv \,A \,x]}{x^0}
{\eleaf{4}{\,A_{\greeko(\greeko\greeki)} \,x^0 \lor \bot \equiv \,A \,x^0}}}
{\erew{2}{EXT=}{\lambda \,x [ \,A \,x \lor \bot ] \,= \,A}
{\erew{1}{\lambda}{\forall \,x_{\greeko\greeki} . [ \lambda \,x . \,A_{\greeko(\greeko\greeki)} \,x \lor \bot ] \,x \,= \,A \,x}
{\sel{0}{ \forall \,x_{\greeko\greeki} . [ \,A_{\greeko(\greeko\greeki)} \,x \lor \bot ] \,= \,A \,x}{x^0}
{\erew{0}{EXT=}{\,A_{\greeko(\greeko\greeki)} \,x^0 \lor \bot = \,A \,x^0}
{\eleaf{3}{\,A_{\greeko(\greeko\greeki)} \,x^0 \lor \bot \equiv \,A \,x^0}}}}}}
$$
with complete mating \verb+((L3 . L4))+.
The full expansion tree with the two lemmas has the form
$$
\econj{9}{}
{\econj{7}{}{lemma_1^-}{lemma_2^-}}
{\eimp{18}{}
{\econj{8}{}{lemma_1^+}{lemma_2^+}}
{theorem}}
$$
The $lemma_i^-$ etrees correspond to the proofs of the lemmas,
and the $lemma_i^+$ etrees correspond to the lemmas that can be
used to show the theorem.
The value of the \verb+lemmas+ slot in the expansion proof has a value
such as \verb+((EXT-LEMMA-1) (EXT-LEMMA-2))+
indicating that there are two lemmas neither of which
depend on lemmas of their own.
The jform, after dissolving the connections corresponding to
the proofs of the lemmas, is
\begin{verbatim}
|             L14      FORALL q^9                                       |
|            A x^18 OR  |           L15                L16 |            |
|                       |~q^9 [LAMBDA x FALSEHOOD] OR q^9 A|            |
|                                                                       |
||  L5   |    |  L7   |                                                 |
||A x^15 |    |A x^15 |    FORALL q^7                                   |
||       | OR |       | OR  |               L9                     L10 ||
||  L6   |    |  L8   |     |~q^7 [LAMBDA x .A x OR FALSEHOOD] OR q^7 A||
||~A x^15|    |~A x^15|                                                 |
|                                                                       |
|                                  L1                                   |
|                    P [LAMBDA x .A x OR FALSEHOOD]                     |
|                                                                       |
|                                  L2                                   |
|                                 ~P A                                  |
\end{verbatim}
Here we can find a complete mating by connecting
\verb+((L1 . L9) (L2 . L10))+ (corresponding to the mating
of \verb+L1+ and \verb+L2+ that we wanted to make),
and \verb+((L5 . L6) (L7 . L8))+ (corresponding to the proof
that $\lambda x . A x \lor \bot$ and $A$ really are extensionally
the same).

Given this complete expansion proof, we can merge the proof
and translate to a natural deduction proof.  During the merging
process, the first two lemmas are recognized as unused.
The translation to natural deduction depends on the value of the flag
\indexflag{ASSERT-LEMMAS}.  If \indexflag{ASSERT-LEMMAS}
is set to T, then the expansion proof translates into two natural deduction
proofs.  The first is a proof of the lemma.  The second is a proof
of the theorem using the lemma.  The name of the natural deduction proof
of the lemma is given by the corresponding symbol in the value of
the \verb+lemmas+ slot.  For example, the name could be EXT-LEMMA-2.
In the proof of the theorem, there would be a line justified by
``Assert: EXT-LEMMA-2''.  If \indexflag{ASSERT-LEMMAS} is NIL,
the expansion proof translates to a single natural deduction proof.
First, the outline \\
\begin{tabular}{lll}
 & $\cdots\cdots$ & \\
(5) & $\vdash \forall \,x_{\greeko\greeki} [ \,A_{\greeko(\greeko\greeki)} \,x \lor \bot \equiv \,A \,x ] \supset \lambda \,x [ \,A \,x \lor \bot ] \,= \,A$ & PLAN4 \\
(6) & $\vdash \forall \,x_{\greeko\greeki} [ \,A_{\greeko(\greeko\greeki)} \,x \lor \bot \equiv \,A \,x ] \supset \lambda \,x [ \,A \,x \lor \bot ] \,= \,A$  & Same as: 5 \\
 & $\cdots\cdots$ & \\
(104) & $\vdash \,P_{\greeko(\greeko(\greeko\greeki))} [ \lambda \,x_{\greeko\greeki} . \,A_{\greeko(\greeko\greeki)} \,x \lor \bot ] \supset \,P \,A$ & PLAN2
\end{tabular} \\
is formed, then these gaps are filled in using the corresponding
parts of the expansion proof.

Extensionality examples can of this form can be proven
automatically using \indexcommand{DIY} if the flag \indexflag{USE-EXT-LEMMAS}
is set to T.

% Set constraint example. - see section

\section{Extensional Expansion Dags}\label{sect:edags}

Extensional expansion dags are a generalization of expansion trees which
represent proofs in extensional type theory (see Chad E. Brown's thesis~\cite{Brown2004a}).
There are four lisp structures used to represent extensional expansion dags:
ext-exp-dag, ext-exp-arc, ext-exp-open-dag and ext-exp-open-arc.
These structures are defined in \indexfile{ext-exp-dag-macros.lisp}.
The intention is that ext-exp-dag (ext-exp-arc) structures represent nodes (arcs) in ground dags with no
expansion variables and should only be constructively manipulated.
On the other hand, ext-exp-open-dag (ext-exp-open-arc) structures represent nodes (arcs) may contain expansion variables
and can be destructively manipulated (e.g., via substitution).
The global variable \indexother{ext-exp-dag-verbose} causes the structures to be printed with
a huge amount of verbosity and should be set to NIL unless debugging.
Similarly, setting the global variable \indexother{ext-exp-dag-debug} to T causes
a lot of extra sanity checking to aid debugging.

The \indexother{EXT-MATE} top level can be used to manipulate extensional expansion dags.
The code implementing the \indexother{EXT-MATE} is in the file \indexfile{ext-mate-top.lisp}.

The automatic search procedures ms03-7 and ms04-2 use extensional expansion dags.
The code for ms03-7 is in \indexfile{ext-search.lisp}.
The code for ms04-2 is in \indexfile{ms04-search.lisp}.

\section{Printing}
