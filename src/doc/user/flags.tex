\chapter{Setting and Varying Flags}
\label{flags}

%Some facilities which can be helpful in finding modes
%appropriate for proving particular theorems automatically are
%discussed in section \ref{modes}.  





The settings of flags are very important during automatic search.  For
example, flags like \indexflag{REWRITE-DEFNS} and
\indexflag{REWRITE-EQUALITIES} must be set to T (true) if you want
definitions and equalities, respectively, to be rewritten when the
expansion tree is created from the formula to be proved.  To see all
of the flags which will affect mating-search, check the facilities
guide.

Since there are many flags in {\TPS}, the flags which are most crucial
to mating search have been collected together under the subject
\indexother{IMPORTANT}, so that typing LIST IMPORTANT will produce a
list of the settings of the dozen or so most important flags for
mating search.

{\TPS} has a facility called \indexcommand{UNIFORM-SEARCH},
which attempts to find the correct flag settings
on its own; it is capable of finding correct settings for many of the theorems
which {\TPS} has proven to date. See section \ref{testtop} for more details.


\section{Review, flags, and modes}


The Review top-level is provided to make it easier to update flag-settings.
Enter the top-level with the command \indexcommand{REVIEW}.  The commands provided
in this top-level (for a listing, enter \indexcommand{?}) give information about the
various flags available.  Each flag is listed under one or more subjects.
You may use the command \indexcommand{SETFLAG} to change the value of a flag, or
you may simply type the name of the flag itself. When entering the argument for
a flag, you may type \indexcommand{??} for more information about the argument
type. In some cases, {\TPS} will warn the user that the flag is irrelevant, given
the settings of other flags.
File names and extensions should be entered as strings (i.e. surrounded
by double quotes); although in some circumstances one can omit the quotes, it
is never an error to include them.

All the flags for a subject can be listed by using the \indexcommand{LIST} command;
{\tt help list} will tell you more. You may search the help messages of flags by
using the \indexcommand{KEY} command; for example {\tt key `unification' all} will
find the flags whose help messages mention unification, in all subjects. A similar
command \indexcommand{SEARCH} allows you to search the help messages of all {\TPS}
objects.

To update many flags at once, use \indexcommand{UPDATE} or 
\indexcommand{UPDATE-RELEVANT}.
The \indexcommand{UPDATE} commmand 
allows the user to set all the flags in given subjects.
Since certain flag values render some other flags
irrelevant, there is a command called \indexcommand{UPDATE-RELEVANT} which
allows users to conveniently take into account the relationships
between flags and the decisions previously made while setting flags
interactively.  Since proof procedures are constantly being developed
or refined, when \indexcommand{UPDATE-RELEVANT} is called, the user is given the
option of using the current flag relevancy information in memory,
loading flag relevancy information which has been saved to a file, or
rebuilding flag relevancy information automatically by processing the Lisp
source files.  

Modes are collections of flags with specified settings.  By selecting a
mode, you simultaneously set all the included flags.  You can also
define your own modes and save them.  See the \indexcommand{INSERT} command,
in the library top level, for more information.

See section \ref{searchanalysis} for information on how to use
a known proof to suggest {\TPS} flag settings, and Chapter
\ref{ms-guide} for information about certain flags which affect the
search processes.

Currently defined flags and modes are listed in the facilities guide
\cite{AndrewsTPS88e}.

\section{Test: Multiple Searches on the Same Problem}\label{testtop}

The \indexcommand{TEST} top level is designed to allow a succession of searches with
varying flag settings to be performed without any interaction from the user.

One of the major uses of this top level is to find settings of the various
flags which will produce a proof of a theorem; this is done by the
\indexcommand{UNIFORM-SEARCH} command, which is explained later in
section \ref{uniform}.

The top level is entered with the \indexcommand{TEST} command; as with
the mate top level, this will prompt for a gwff and ask whether you
want to open a vpwindow.

There are three major groups of TEST commands; those which are for
specifying a list of flags to be varied, those which deal with the
associated library functions and those which actually perform the search.

There is one major new data structure: a searchlist is a list of search items;
each search item contains the name of a flag, a default (initial) value for the flag,
and a range of values over which it should be varied. The user can have
several searchlists in memory at once, and can switch between them
with the \indexcommand{NEW-SEARCHLIST} command. See below for an example of how to
construct a searchlist. Searchlists can be stored in the library with the
\indexcommand{INSERT} command (from the test top level), and retrieved with the
\indexcommand{FETCH} command (also from the test top level). There is an additional
(optional) function attached to each searchlist; this will be called on each iteration of
the search, and may do such things as setting flags which are not in the searchlist (so, for
example, \indexflag{MAX-SEARCH-DEPTH} and \indexflag{MAX-UTREE-DEPTH} can be kept equal to
each other by having one in the searchlist and the other set by the optional function).
The function UNIFORM-SEARCH-FUNCTION is used by the UNIFORM-SEARCH procedures.

The searchlist also contains (invisible to the user) internal flags which determine the current
position in the search; this means that it is possible to interrupt a search,
save the current searchlist in the library, and later on to reload that searchlist
(even into a different core image) and continue the search from the point where it was
interrupted.

Once a searchlist is constructed, the \indexcommand{GO} command will start a search
and the \indexcommand{CONTINUE} command will continue a search after an interruption.
The flags in the subject \indexsubject{test-top} control many of the parameters of the
search; type {\tt list test-top} for more information. The user has the option of opening
a test-window, analogous to the vpwindow, which will show a summary of the search so far.
If the search finds a proof, it will terminate; otherwise, it will increase the time limit
in the flag \indexflag{TEST-INITIAL-TIME-LIMIT} and try again. Once the search terminates,
it will define a new mode which can be saved in the library using \indexcommand{INSERT}
{\it from the test top level}. (The \indexcommand{INSERT} in the library top level is different;
when you try to save a mode, it asks you to specify all the flag settings. The \indexcommand{INSERT}
in the test top level merely asks for the name of the mode.)

\subsection{How to Build A Searchlist Without Any Effort}

The commands \indexcommand{VARY-MODE} and \indexcommand{QUICK-DEFINE} are
useful for defining searchlists quickly. The former takes an existing mode,
and steps through it flag by flag, building a searchlist by offering to
add each flag of the mode and an appropriate range to the new searchlist.
The latter uses a pre-defined list of flags, either those listed in the
\indexflag{TEST-FASTER-*} group or those listed in the \indexflag{TEST-EASIER-*}
group, to create a searchlist in which each of the flags has up to
\indexflag{TEST-MAX-SEARCH-VALUES} possible values.

Given a large searchlist, it is possible to trim it down using the \indexcommand{SCALE-UP}
and \indexcommand{SCALE-DOWN} commands; for each of the flags in the TEST-EASIER-* or TEST-FASTER-*
lists (respectively), these commands compare the range of the flags in the
searchlist with the current values of the flags, and remove all values that would not make the
search easier (in the case of SCALE-UP) or faster (in the case of SCALE-DOWN).

In some cases, it is not necessary to build a searchlist at all. The commands \indexcommand{PRESS-DOWN},
\indexcommand{PUSH-UP} and \indexcommand{FIND-BEST-MODE} all build their own searchlists
using the flags listed in the TEST-FASTER-* flags (for PRESS-DOWN and FIND-BEST-MODE)
and those listed in the TEST-EASIER-* flags (for PUSH-UP and FIND-BEST-MODE). Below are
examples of how to use these commands.

\subsection{Using TEST to Improve a Successful Mode}
\begin{tpsexample}
<18>test sample-theorem !
<test19>mode mode-sample-theorem
{\it mode-sample-theorem is a mode in which sample-theorem can be proven.}
<test20>list test-top

Subject: TEST-TOP
  TEST-FASTER-IF-HIGH:   (MIN-QUICK-DEPTH)
  TEST-FASTER-IF-LOW:    (MAX-SEARCH-DEPTH SEARCH-TIME-LIMIT MAX-UTREE-DEPTH
                          MAX-MATES MAX-SEARCH-LIMIT)
  TEST-FASTER-IF-NIL:    ()
  TEST-FASTER-IF-T:      (MIN-QUANTIFIER-SCOPE MS-SPLIT)
  TEST-FIX-UNIF-DEPTHS:  T
  TEST-INCREASE-TIME:    0
  TEST-INITIAL-TIME-LIMIT: 30
  TEST-MAX-SEARCH-VALUES: 10
  TEST-REDUCE-TIME:      T
{\it Some flags have been omitted. When we do PRESS-DOWN, it will
automatically create a searchlist which will vary each
of the flags listed in the TEST-FASTER-* flags above, and then
start searching with a maximum time of 30 seconds, decreasing the
time as it goes, and fixing the unification depths after the
first successful search.}
<test21>press-down
{\it The search is started. While it works, we can go away for a
cup of coffee (or a two-week vacation to Mexico, depending on how
difficult sample-theorem is).}
\end{tpsexample}

\subsection{Using TEST to Discover a Successful Mode}
\begin{tpsexample}
<18>test sample-theorem !
<test19>mode bad-mode
{\it bad-mode is a mode in which sample-theorem cannot be proven.}
<test20>list test-top

Subject: TEST-TOP
  TEST-EASIER-IF-HIGH:   (MAX-SEARCH-DEPTH SEARCH-TIME-LIMIT NUM-OF-DUPS
                          MAX-UTREE-DEPTH MAX-MATES MAX-SEARCH-LIMIT)
  TEST-EASIER-IF-LOW:    (MIN-QUICK-DEPTH)
  TEST-EASIER-IF-NIL:    ()
  TEST-EASIER-IF-T:      (ETA-RULE MIN-QUANTIFIER-SCOPE MS-SPLIT)
  TEST-INCREASE-TIME:    10
  TEST-INITIAL-TIME-LIMIT: 30
  TEST-MAX-SEARCH-VALUES: 10
{\it Some flags have been omitted. When we do PUSH-UP, it will
automatically create a searchlist which will vary each
of the flags listed in the TEST-EASIER-* flags above, and then
start searching with a maximum time of 30 seconds, increasing the
time by ten percent on each attempt.}
<test21>push-up
{\it The search is started. It will stop as soon as it finds a mode which works.
We can combine both PUSH-UP and PRESS-DOWN (first push-up to find a successful
mode, then press-down to find a better mode) by using the command FIND-BEST-MODE
instead of PUSH-UP.}
\end{tpsexample}

\subsection{Building A Searchlist with TEST}
\begin{tpsexample}
<12>list test-top
Subject: TEST-TOP
  TEST-INITIAL-TIME-LIMIT: 30
  TEST-NEXT-SEARCH-FN:   EXHAUSTIVE-SEARCH
  TEST-REDUCE-TIME:      T
  TEST-VERBOSE:          T
  TESTWIN-HEIGHT:        24
  TESTWIN-WIDTH:         80
{\it These are the flags available. Note the first three; the first says that
each search will be allowed 30 seconds, the second that the function
EXHAUSTIVE-SEARCH will be used to determine which flag settings are used next,
and the third that if a proof is found, the time allowed for the next proof
will be decreased.}
<13>test
GWFF (GWFF0-OR-EPROOF): Gwff or Eproof [No Default]>x5305
DEEPEN (YESNO): Deepen? [Yes]>
WINDOW (YESNO): Open Vpform Window? [No]>yes
File to send copy Vpform output to (`` '' to discard) [``vpwin.vpw'']>``test-top.vpw''

Use CLOSE-MATEVPW when you want to close the vpwindow.
<test14>?
Top Levels:    LEAVE
Mating search: CLOSE-TESTWIN CONTINUE GO OPEN-TESTWIN SEARCH-ORDER
Searchlists:   ADD-FLAG ADD-FLAG* ADD-SUBJECTS NEW-SEARCHLIST REM-FLAG
               REM-FLAG* SEARCHLISTS SHOW-SEARCHLIST
Library:       DELETE FETCH INSERT

<test15>new-searchlist
NAME (SYMBOL): Name of new searchlist [No Default]>x5305-search
{\it We begin to define a new searchlist, containing the flags to be varied.}
<test16>searchlists
X5305-SEARCH

{\it ...this is currently the only searchlist in memory...}
<test17>add-flag*
FLAG (TPSFLAG): Flag to be added [No Default]>max-search-depth
INIT (ANYTHING): Initial value of flag [No Default]>42
RANGE (ANYTHING-LIST): List of possible values to use [No Default]>4 8 10

Add another flag?  [Yes]>

FLAG (TPSFLAG): Flag to be added [No Default]>max-utree-depth
INIT (ANYTHING): Initial value of flag [No Default]>20
RANGE (ANYTHING-LIST): List of possible values to use [No Default]>4 8 10

Add another flag?  [Yes]>no

{\it We have added two flags to our searchlist, with four values for each.
Note also that there are commands to add a single flag (ADD-FLAG) or
every flag in a given list of subjects (ADD-SUBJECTS), as well as
commands to remove a flag or flags (REM-FLAG, REM-FLAG*).}
<test18>show-searchlist
NAME (SYMBOL): Name of searchlist [X5305-SEARCH]>
Searchlist X5305-SEARCH is as follows:
MAX-UTREE-DEPTH = 20, default is 20, range is [(20 4 8 10)]
MAX-SEARCH-DEPTH = 42, default is 42, range is [(42 4 8 10)]

{\it Next, we decide to check how many searches there will be, and in what
order they will be performed.}
<test19>search-order
NAME (SYMBOL): Name of searchlist [X5305-SEARCH]>
There are 16 possible settings of these flags.
Search : (MAX-UTREE-DEPTH = 20) (MAX-SEARCH-DEPTH = 42)
Search : (MAX-UTREE-DEPTH = 10) (MAX-SEARCH-DEPTH = 42)
Search : (MAX-UTREE-DEPTH = 8) (MAX-SEARCH-DEPTH = 42)
Search : (MAX-UTREE-DEPTH = 4) (MAX-SEARCH-DEPTH = 42)
Search : (MAX-UTREE-DEPTH = 20) (MAX-SEARCH-DEPTH = 10)
Search : (MAX-UTREE-DEPTH = 10) (MAX-SEARCH-DEPTH = 10)
Search : (MAX-UTREE-DEPTH = 8) (MAX-SEARCH-DEPTH = 10)
Search : (MAX-UTREE-DEPTH = 4) (MAX-SEARCH-DEPTH = 10)
Search : (MAX-UTREE-DEPTH = 20) (MAX-SEARCH-DEPTH = 8)
Search : (MAX-UTREE-DEPTH = 10) (MAX-SEARCH-DEPTH = 8)
Search : (MAX-UTREE-DEPTH = 8) (MAX-SEARCH-DEPTH = 8)
Search : (MAX-UTREE-DEPTH = 4) (MAX-SEARCH-DEPTH = 8)
Search : (MAX-UTREE-DEPTH = 20) (MAX-SEARCH-DEPTH = 4)
Search : (MAX-UTREE-DEPTH = 10) (MAX-SEARCH-DEPTH = 4)
Search : (MAX-UTREE-DEPTH = 8) (MAX-SEARCH-DEPTH = 4)
Search : (MAX-UTREE-DEPTH = 4) (MAX-SEARCH-DEPTH = 4)

{\it Because TEST-NEXT-SEARCH-FN is EXHAUSTIVE-SEARCH, the search will try
every possible combination of these flags, as shown above.}

<test20>go
MODENAME (SYMBOL): Name for optimal mode [TEST-BESTMODE]>x5305-bestmode
TESTWIN (YESNO): Open a window for test-top summary? [No]>yes
File to send test-top summary to (`` '' to discard) [``info.test'']>

Use CLOSE-TESTWIN when you want to close the testwindow.

Changing flag settings as follows:
(MAX-UTREE-DEPTH = 20) (MAX-SEARCH-DEPTH = 42)

{\it lots of output omitted...}

The time used in each process:
-----------------------------------------------------------------------------
Process Name         | Realtime | Internal-runtime |  GC-time   | I-GC-time
-----------------------------------------------------------------------------
                                (Interrupted)
Mating Search        |       77 |            52.13 |       0.00 |      52.13
                      (1.3 mins)
-----------------------------------------------------------------------------

Changed MAX-UTREE-DEPTH = 20, default is 20, range is [(20 4 8 10)]

Changed MAX-SEARCH-DEPTH = 42, default is 42, range is [(42 4 8 10)]
Search finished.Changing flag settings as follows:
(MAX-UTREE-DEPTH = 20) (MAX-SEARCH-DEPTH = 42)
Finished varying flags; succeeded in proof.
Best mode was ((MAX-SEARCH-DEPTH 42) (MAX-UTREE-DEPTH 8))
Best time was 28.667

Have defined a mode called X5305-BESTMODE; use INSERT to put this into the library.

{\it We ran the search; it terminated and defined a new mode for us.}
<test21>help X5305-BESTMODE
X5305-BESTMODE is a mode.
A mode resulting from a test-search.
Flags are set as follows:
   Flag                     Value in Mode         Current Value
   MAX-SEARCH-DEPTH         42                    42
   MAX-UTREE-DEPTH          8                     20

<test22>close-testwin
Closed test-window file : info.test
Shall I delete the output file info.test?  [No]>yes

\end{tpsexample}

\subsection{Uniform Search: Finding Successful Modes Automatically}\label{uniform}

There are two top-level commands in {\TPS} which search for a successful mode for
proving a theorem, without requiring the user to master the \indexcommand{TEST} top level first.
They are \indexcommand{UNIFORM-SEARCH} and \indexcommand{UNIFORM-SEARCH-L}.

\indexcommand{UNIFORM-SEARCH-L} is analogous to \indexcommand{DIY-L}; it attempts
to find a successful mode which will allow {\TPS} to fill in a subproof during the
interactive construction of a larger theorem. Apart from the fact that it generates
subproofs within a given range of lines, it works in exactly the same way as
\indexcommand{UNIFORM-SEARCH}, and so the rest of this section is devoted entirely to
\indexcommand{UNIFORM-SEARCH}.

\indexcommand{UNIFORM-SEARCH} takes three essential arguments: a gwff which is to be proven,
a mode and a searchlist. By default, the mode is called UNIFORM-SEARCH-MODE and the
searchlist is called UNIFORM-SEARCH-2; if these objects are not present in your library,
you should provide appropriate alternatives.

The idea is that {\TPS} will set all of the flags to the values given in UNIFORM-SEARCH-MODE,
and then vary the flags as prescribed by UNIFORM-SEARCH-2. So UNIFORM-SEARCH-MODE should
be a `neutral' mode which does not constrain the search very much, and which sets all of
the less important flags to reasonable values.

UNIFORM-SEARCH-2 might be defined as follows (for example):
\begin{tpsexample}
DEFAULT-MS = MS91-6, default is MS91-6, range is [(MS91-6 MS91-7)]
MAX-SUBSTS-VAR = 3, default is 3, range is [(3 5 7)]
MAX-MATES = 1, default is 1, range is [(1 2 3 4 6)]
NUM-OF-DUPS = 0, default is 0, range is [(0 1 2)]
REWRITE-EQUALITIES = ALL, default is ALL, range is [(ALL LAZY2 LEIBNIZ)]
REWRITE-DEFNS = (EAGER), default is (EAGER), range is [((EAGER) (LAZY2))]
SEARCH-TIME-LIMIT = 30, default is 30, range is [(30 120 240 920 1840 3600 7200)]
...plus the function UNIFORM-SEARCH-FUNCTION
\end{tpsexample}

As you can see, this varies the most important flags in {\TPS} over a reasonable range of values.
Once you have entered the TEST top level with UNIFORM-SEARCH, all that remains is to type GO,
and wait for a proof. Of course, proofs in UNIFORM-SEARCH usually take longer to be found than proofs in which
the correct mode is already known.

If a proof is found, a new mode will be defined, which can be stored in the library, and by merging the etree
and then calling NAT-ETREE this proof can be translated into a natural deduction proof, exactly
as in the MATE top level.

One final word about UNIFORM-SEARCH: it will also offer to modify the searchlist for you.
This will speed up the search, if possible, by removing those flags in the searchlist
which will have no effect on the proof of the given gwff. In particular, it will remove
unification depths for first-order gwffs, REWRITE-DEFNS for gwffs that contain no definitions,
REWRITE-EQUALITIES for those that contain no equalities, and so on. It will also change the
settings of DEFAULT-MS to MS88 and/or MS90-3 if it determines that there are no primitive
substitutions for the given gwff.


\section{Search Analysis: Facilities for Setting Flags and Tracing Automatic Search}\label{searchanalysis}

If one has a natural deduction proof of a theorem, one can use the
command \indexcommand{AUTO-SUGGEST} to obtain suggested settings for certain flags of
a mode with which that theorem can be proved automatically.
\indexcommand{AUTO-SUGGEST} will also show all of the instantiations of quantifiers
that are necessary for that proof.  The command \indexcommand{ETR-AUTO-SUGGEST} does
the same thing when given an expansion proof. Such an expansion proof
could be the result of translating a natural deduction proof into an
expansion proof using the command \indexcommand{NAT-ETREE}.
The commands \indexcommand{MS03-LIFT} and \indexcommand{MS04-LIFT} 
in the \indexother{EXT-MATE} top level
suggests flag settings for the corresponding extensional search
procedures when given an extensional expansion proof.

One can obtain an expansion proof for a gwff by several methods,
including constructing a mating by hand
in the \indexcommand{MATE} top level.
An easier way is to use \indexcommand{NAT-ETREE} (see section \ref{natetr})
to translate a natural deduction proof into an expansion proof.
Once we have an expansion proof, we can use this to suggest
flag values and to trace the \indexother{MS98-1} search procedure.

In this section we will consider two examples: THM12
and X2116.

\subsection{Example: Setting Flags for THM12}\label{THM12-NAT-ETREE}

Our first example concerns THM12:

$\forall R _{\greeko\greeki}   \forall S _{\greeko\greeki} .R = S \implies  \forall X _{\greeki} .S X \implies R X$

The following excerpts from a TPS session shows how we can
use \indexcommand{NAT-ETREE} to get an expansion proof
and then use this expansion proof to suggest flag settings.

\begin{tpsexample}
<401>prove THM12
. . . ; ***prove the theorem, perhaps interactively***
<431>pstatus
 No planned lines
<432>nat-etree
PREFIX (SYMBOL): Name of the Proof [THM12]>
Proof THM12-2 restored.

. . . ; ***nat-etree preprocesses the proof,
. . . ;    converts it to a sequent calculus derivation,
. . . ;    performs cut elimination
. . . ;    then translates to an expansion tree with a complete mating.***
|            L117             |
| \(\sim\)R^152 X^141 OR S^17 X^141  |
|                             |
|            L115             |
| \(\sim\)S^17 X^141 OR R^152 X^141  |
|                             |
|            L116             |
|\(\sim\)[\(\sim\)S^17 X^141 OR R^152 X^141]|
Number of vpaths: 1
((L115  . L116 ))    ; note this is a connection between nonatomic wffs

Adding new connection: (L115 . L116)
If you want to translate the expansion proof back to a natural deduction proof,
you must first merge the etree.  If you want to use the expansion proof to determine
flag settings for automatic search, you should not merge the etree.
Merge The Etree? [Yes]>n           ; ***we don't merge the tree***
The expansion proof Eproof:EPR122  can be used to trace MS98-1 search procedure.
The current natural deduction proof THM12-2 is a modified version
of the original natural deduction proof.

Use RECONSIDER THM12 to return to the original proof.

<433>mate
GWFF (GWFF0-OR-LABEL-OR-EPROOF): Gwff or Eproof [Eproof:EPR122 ]>
DEEPEN (YESNO): Deepen? [Yes]>n
REINIT (YESNO): Reinitialize Variable Names? [Yes]>n
WINDOW (YESNO): Open Vpform Window? [No]>
. . .
<Mate437>etr-info  ; ***etr-info lists the expansion terms***
Expansion Terms:
X^141(I)           ; ***only one (easy) expansion term in the proof***

<Mate438>etr-auto-suggest
. . .
MS98-INIT suggestion: 1
MAX-SUBSTS-VAR should be 1
NUM-OF-DUPS should be 0
MS98-NUM-OF-DUPS should be 1
MAX-MATES should be 1
Do you want to define a mode with these settings? [Yes]>
Name for mode?  [MODE-THM12-2-SUGGEST]>

<Mate439>leave
Merge the expansion tree? [Yes]>n   ; ***let's still not merge***


<440>mode MODE-THM12-2-SUGGEST
\end{tpsexample}

This suggested mode will prove the theorem.  In a later
section we will continue this example to see how we can
use the eproof to trace the \indexother{MS98-1} search procedure.

\subsection{Example: Setting Flags for X2116}\label{X2116-NAT-ETREE}

Suppose we have a natural deduction proof for X2116:

$\forall x _{\greeki}   \exists y _{\greeki}  [P _{\greeko\greeki}  x \implies R _{\greeko\greeki\greeki}  x [g _{\greeki\greeki}  .h _{\greeki\greeki}  y] \and P y] \and  \forall w _{\greeki}  [P w \implies P [g w] \and P .h w]
 \implies  \forall x .P x \implies  \exists y .R x y \and P y$

The following excerpts from a TPS session shows how we can
use \indexcommand{NAT-ETREE} to get an expansion proof
and then use this expansion proof to suggest flag settings.

\begin{tpsexample}
<405>nat-etree
PREFIX (SYMBOL): Name of the Proof [X2116]>
. . .
|              |        L90        |   |
|    L89       |R x^329 [g .h y^70]|   |
|  \(\sim\)P x^329 OR |                   |   |
|              |       L92         |   |
|              |      P y^70       |   |
|                                      |
|                   |    L102     |    |
|        L99        |P [g .h y^70]|    |
|    \(\sim\)P [h y^70] OR |             |    |
|                   |    L104     |    |
|                   |P [h .h y^70]|    |
|                                      |
|                  |   L100   |        |
|         L96      |P [g y^70]|        |
|       \(\sim\)P y^70 OR |          |        |
|                  |   L98    |        |
|                  |P [h y^70]|        |
|                                      |
|                 L88                  |
|               P x^329                |
|                                      |
|        L91                  L103     |
|\(\sim\)R x^329 [g .h y^70] OR \(\sim\)P [g .h y^70]|
Number of vpaths: 16
((L102  . L103 ) (L98  . L99 ) (L92  . L96 ) (L88  . L89 ) (L90  . L91 ) (L88  . L89 ))

. . .
If you want to translate the expansion proof back to a natural deduction proof,
you must first merge the etree.  If you want to use the expansion proof to determine
flag settings for automatic search, you should not merge the etree.
Merge The Etree? [Yes]>n
The expansion proof Eproof:EPR116  can be used to trace MS98-1 search procedure.
The current natural deduction proof X2116-1 is a modified version
of the original natural deduction proof.

Use RECONSIDER X2116 to return to the original proof.

<406>mate
GWFF (GWFF0-OR-LABEL-OR-EPROOF): Gwff or Eproof [Eproof:EPR116 ]>
DEEPEN (YESNO): Deepen? [Yes]>n
REINIT (YESNO): Reinitialize Variable Names? [Yes]>n
WINDOW (YESNO): Open Vpform Window? [No]>n
. . .
<Mate409>show-mating   ; ***note the mating has six connections***

Active mating:
(L88 . L89)  (L90 . L91)  (L88 . L89)
(L92 . L96)  (L98 . L96.1)  (L100.1 . L103)

<Mate410>etr-info  ; ***there are 4 expansion terms, with reasonable sizes***
Expansion Terms:
g(II).h(II) y^70(I)
h(II) y^70(I)
y^70(I)
x^329(I)

<Mate411>etr-auto-suggest ; ***to get suggested flag settings***
. . .
MS98-INIT suggestion: 1
MAX-SUBSTS-VAR should be 3
NUM-OF-DUPS should be 1
MS98-NUM-OF-DUPS should be 1
MAX-MATES should be 1
Do you want to define a mode with these settings? [Yes]>
Name for mode?  [MODE-X2116-1-SUGGEST]>


<Mate412>mode MODE-X2116-1-SUGGEST
\end{tpsexample}

These flag settings are sufficient for \indexother{MS98-1} to prove the
theorem.  Later we will continue this example to show how
to use the expansion proof to trace \indexother{MS98-1}.

\subsection{Tracing MS98-1}\label{ms98-trace}

Currently there are some facilities for tracing the automatic search
procedure \indexother{MS98-1} (see Matt Bishop's thesis for details
of this search procedure).  The flag \indexflag{MS98-VERBOSE} can
be set to {\tt T} to simply obtain more output.  If we have an expansion
proof already given, we can obtain a finer trace on MS98-1 to find
information about the search.  For example, we may be able to find
which connections failed to be added to the mating.

A background expansion proof may be the result of calling
\indexcommand{NAT-ETREE} as described above.
We may also construct a complete mating interactively
in the \indexcommand{MATE} top level, then use
\indexcommand{SET-BACKGROUND-EPROOF} to save this expansion
proof as the one to use for tracing.  Consider the following
example session showing how we might get a mating
for X2116 (instead of using \indexcommand{NAT-ETREE}
as in section \ref{X2116-NAT-ETREE}).

\begin{tpsexample}
<0>exercise x2116
Would you like to load a mode for this theorem? [No]>y
2 modes for this theorem are known:
1) MODE-X2116  1999-04-23  0 seconds  (read only)
2) MS98-FO-MODE  1999-04-23  1 seconds  (read only)
3) None of these.
Input a number between 1 and 3: [1]>
. . .

<1>mate 100 y n n
<Mate2>go
. . .
Eureka!  Proof complete..
. . .
<Mate3>show-mating

Active mating:
(L21.1 . L14.1)  (L20.1 . L9)  (L12.1 . L15)
(L7.1 . L10)  (L12 . L10)  (L7 . L17)
(L21 . L15)
is complete.

<Mate4>set-background-eproof
EPR (EPROOF): Eproof [Eproof:EPR0 ]>
\end{tpsexample}

Once there is a background expansion proof,
the information given by the mating in the background proof
can be transferred to a search expansion tree via `colors'.
For each connection, we create a `color' (really just
a generated symbol) which is associated with all nodes
in the search expansion tree which could correspond to the connected
nodes in the background tree.  The examples in subsections
\ref{THM12-NAT-ETREE} and \ref{X2116-NAT-ETREE}
should make the role of colors clearer.

The flag \indexflag{MS98-TRACE} can
be used to obtain information about how the search is performing
relative to the background eproof.  The value of this flag is a list of
values.  Ordinarily, when tracing is off, \indexflag{MS98-TRACE}
will have the value NIL.  Certain symbols have the following meanings
if they occur on the value of \indexflag{MS98-TRACE}:

\begin{description}
\item[] {\tt MATING} -- If the symbol MATING is on the list,
search as usual, printing when `good' connections and components
are found.
A `good' connection is one in which the two nodes share a common color.
A `good' component contains only good connections.
A search succeeds when it generates all the good connections, and combines
these into good components, merging these components until the complete mating
is generated.  Note that successfully generating connections and merging components
depends on unification, and in particular, on the value of \indexflag{MAX-SUBSTS-VAR}.

\item[] {\tt MATING-FILTER} -- This prints the same information as {\tt MATING}, but
only generates good connections and only builds good components.
This value is useful for finding if the unification bounds (\indexflag{MAX-SUBSTS-VAR})
are set in such a way that the search can possibly succeed.

\end{description}

After setting \indexflag{MS98-VERBOSE} and \indexflag{MS98-TRACE},
one can invoke the search procedure \indexother{MS98-1} using
the mate command \indexcommand{MS98-1} or the mate command
\indexcommand{GO} if \indexflag{DEFAULT-MS} is set to \indexother{MS98-1}.

The next two subsections show how the tracing works
in practice by continuing the examples in subsections
\ref{THM12-NAT-ETREE} and \ref{X2116-NAT-ETREE}.

\subsection{Example: Tracing THM12}\label{THM12-MS98-TRACE}

Suppose the background expansion proof is a proof for THM12,
and suppose we are in a mode such as the suggested mode
from \indexcommand{ETR-AUTO-SUGGEST} in section \ref{THM12-NAT-ETREE}.

First, examine the background expansion proof more closely.
The expansion proof and jform are shown in Figure \ref{thm12-vp}.

\begin{figure}
\begin{tpsexample}
<459>ptree*
                                        [SEL11]
                                      [ R^152(OI) ]
                                            |
                                            |
                                            |
                                        [SEL10]
                                      [ S^17(OI) ]
                                            |
                                            |
                                            |
                                        [IMP86]
            R^152(OI) = S^17(OI) IMPLIES FORALL X(I).S^17 X IMPLIES R^152 X
                                            |
                      /-------------------------------------------\
                      |                                           |
                  [REW10]                                      [SEL9]
                   [EXT=]                                   [ X^141(I) ]
                      |                                           |
                      |                                           |
                      |                                           |
                   [EXP7]                                      [L116]
                [ X^141(I) ]                  S^17(OI) X^141(I) IMPLIES R^152(OI) X^141
                      |                                           |
                      |
                      |
                   [REW9]
                   [EXT=]
                      |
                      |
                      |
                   [REW8]
              [EQUIV-IMPLICS]
                      |
                      |
                      |
                  [CONJ74]
                      *
                      |
           /---------------------\
           |                     |
        [L117]                [L115]
           * S^17(OI) X^141(I) IMPLIES R^152(OI) X^141

<460>vp ; ***JFORM***
|                       SEL11                        |
|EXISTS R EXISTS S [R = S AND EXISTS X .S X AND \(\sim\)R X]|
|                                                    |
|                        L117                        |
|             \(\sim\)R^152 X^141 OR S^17 X^141             |
|                                                    |
|                        L115                        |
|             \(\sim\)S^17 X^141 OR R^152 X^141             |
|                                                    |
|                       L116                         |
|           \(\sim\)[\(\sim\)S^17 X^141 OR R^152 X^141]            |
Number of vpaths: 1

<461>show-mating   ; ***Complete Mating***
(L115 . L116)
\end{tpsexample}
\caption{Expansion Tree and JForm for THM12}
\label{thm12-vp}
\end{figure}

There is only one connection, so this should correspond to a single
color.  This is an interesting example because the search expansion tree will
have a somewhat different form.  Since an EQUIV occurs in the formula,
the form of the tree will depend on the value of \indexflag{REWRITE-EQUIVS}.

First, suppose \indexflag{REWRITE-EQUIVS} is set to 1.  The search
expansion tree in this case is shown in Figure \ref{thm12-search-etree1}.

\begin{figure}
\begin{tpsexample}
<466>ptree*
                                        [SEL11]
                                      [ R^152(OI) ]
                                            |
                                            |
                                            |
                                        [SEL10]
                                      [ S^17(OI) ]
                                            |
                                            |
                                            |
                                        [IMP86]
                                            |
                                            |
                      /-------------------------------------------\
                      |                                           |
                  [REW10]                                      [SEL9]
                   [EXT=]                                   [ X^141(I) ]
                      |                                           |
                      |                                           |
                      |                                           |
                   [EXP7]                                      [L116]
                [ X^141(I) ]                                      |
                      |                                           |
                      |
                      |
                   [REW9]
                   [EXT=]
                      |
                      |
                      |
                   [REW8]
              [EQUIV-IMPLICS]
                      |
                      |
                      |
                  [CONJ74]
                      |
                      |
           /---------------------\
           |                     |
        [L117]                [L115]
           |                     |
\end{tpsexample}
\caption{Search Expansion Tree with REWRITE-EQUIVS 1}
\label{thm12-search-etree1}.
\end{figure}

To use tracing, set MS98-TRACE to an appropriate value,
such as (MATING) or (MATING MATING-FILTER).
Then execute the DIY command (or the MS98-1 command in
the mate top level), and {\TPS} will color the nodes
before performing mating search.

A new color corresponding to the (L115 . L116) connection is created.
{\TPS} needs to ensure all nodes corresponding to L115 and L116 are given this color.
When trying to find the nodes that correspond to L115 (in the background),
{\TPS} first runs into a problem at the correspondence between
REW8 (in the background) to REW2 (in the search expansion tree).  In REW8,
EQUIV is expanded as a conjunction of implications.  In the search
tree, EQUIV is expanded as a disjunction of conjunctions.
So, {\TPS} colors every node beneath REW2.  To find the nodes that correspond
to L116.  {\TPS} finds IMP1 corresponds structurally, so {\TPS} colors every node
beneath IMP1.  As a result, every leaf gets the single color in this case.
So, tracing is basically useless in this case.  Every two literals will
share the only color, and so will be considered `good'.

However, if REWRITE-EQUIVS is set to 4, the EQUIV in the search expansion tree
will be expanded as a conjunction of implications.  This will make
the search expansion tree correspond more closely to the background tree.
This is shown in Figure \ref{thm12-search-etree2}.

\begin{figure}
\begin{tpsexample}
<475>ptree*
                                         [SEL0]
                                      [ R^159(OI) ]
                                            |
                                            |
                                            |
                                         [SEL1]
                                      [ S^24(OI) ]
                                            |
                                            |
                                            |
                                         [IMP0]
                                            |
                                            |
                      /-------------------------------------------\
                      |                                           |
                   [REW0]                                      [SEL2]
                   [EXT=]                                   [ X^148(I) ]
                      |                                           |
                      |                                           |
                      |                                           |
                   [EXP0]                                      [IMP3]
                [ x^346(I) ]                                      |
                      |                                           |
                      |                                /---------------------\
                      |                                |                     |
                   [REW1]                            [L16]                 [L17]
                   [EXT=]                              |                     |
                      |                                |                     |
                      |
                      |
                   [REW2]
              [EQUIV-IMPLICS]
                      |
                      |
                      |
                  [CONJ0]
                      |
                      |
           /---------------------\
           |                     |
        [IMP1]                [IMP2]
           |                     |
           |                     |
      /---------\           /---------\
      |         |           |         |
    [L11]     [L12]       [L13]     [L14]
      |         |           |         |

\end{tpsexample}
\caption{Search Expansion Tree with REWRITE-EQUIVS 4}
\label{thm12-search-etree2}.
\end{figure}

In this case, we can see that L115 corresponds to the node IMP2.
So, {\TPS} colors IMP2, L13, and L14.  L116 corresponds to the node IMP3,
so {\TPS} colors IMP3, L16, and L17.
This time the leaves L11 and L12 are left uncolored,
so the trace should have an effect.

Suppose \indexflag{MS98-TRACE} is set to (MATING MATING-FILTER).

\begin{tpsexample}
<Mate500>ms98-1
Substitutions in this jform:
None.
Transfering mating information from background eproof
Eproof:EPR122
transfering conn (L115 . L116)
COLORS:
COLOR62 - L14 L13 L17 L16
. . .
Trying to Unify L17 with L14  ; ***both have COLOR62***
Component 1 is good:
. . .
Trying to Unify L16 with L13  ; ***both have COLOR62***
Component 2 is good:
. . .
Component 3 is good:
Success! The following is a complete mating:
((L17 . L14) (L16 . L13))
. . .
\end{tpsexample}

\subsection{Example: Tracing X2116}\label{X2116-MS98-TRACE}

Suppose the background expansion proof is a proof for X2116,
and suppose we are in a mode such as the suggested mode
from \indexcommand{ETR-AUTO-SUGGEST} in section \ref{THM12-NAT-ETREE}.

Since the mating in this example has six connections, there
will be six colors.

\begin{tpsexample}
<414>exercise x2116
. . .
<415>mate 100 y n n
. . .
<Mate417>ms98-trace
MS98-TRACE [(MATING MATING-FILTER)]>

<Mate418>ms98-1
Substitutions in this jform:
None.
Transfering mating information from background eproof
Eproof:EPR116
transfering conn (L88 . L89)
transfering conn (L90 . L91)
transfering conn (L88 . L89)
transfering conn (L92 . L96)
transfering conn (L98 . L99)
transfering conn (L102 . L103)
COLORS:
COLOR50 - L14 L14.1 L21 L21.1
COLOR49 - L15 L15.1 L12 L12.1
COLOR48 - L10 L10.1 L12 L12.1
COLOR47 - L17 L7 L7.1
COLOR46 - L9 L9.1 L20 L20.1
COLOR45 - L17 L7 L7.1
\end{tpsexample}

We can see in this example that the colors closely correspond
to the literals used in each connection.  Note also that
duplicate leaves get a common color, because any of the
children of an expansion node can correspond to any children
of the corresponding expansion node.  For example, L102
in the background tree could correspond to either
L14 or L14.1 in the search expansion tree.

In the session, {\TPS} continues searching, printing information about
`good' components, and filtering out those which are not `good'.

