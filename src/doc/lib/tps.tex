\def\endfake{\bigskip\bigskip}

\def\Alpha{{\rm A}}
\def\Beta{{\rm B}}
\def\Chi{{\rm X}}
\def\Epsilon{{\rm E}}
\def\Eta{{\rm H}}
\def\Iota{{\rm I}}
\def\Kappa{{\rm K}}
\def\Mu{{\rm M}}
\def\Nu{{\rm N}}
\def\Omicron{{\rm O}}
\def\omicron{{\rm o}}
\def\Rho{{\rm P}}
\def\Tau{{\rm T}}
\def\Upsilon{{\rm Y}}
\def\Zeta{{\rm Z}}

\def\bolda{{\bf A}}
\def\boldb{{\bf B}}
\def\boldc{{\bf C}}
\def\boldd{{\bf D}}
\def\bolde{{\bf E}}
\def\boldf{{\bf F}}
\def\boldg{{\bf G}}
\def\boldh{{\bf H}}
\def\boldi{{\bf I}}
\def\boldj{{\bf J}}
\def\boldk{{\bf K}}
\def\boldl{{\bf L}}
\def\boldm{{\bf M}}
\def\boldn{{\bf N}}
\def\boldo{{\bf O}}
\def\boldp{{\bf P}}
\def\boldq{{\bf Q}}
\def\boldr{{\bf R}}
\def\bolds{{\bf S}}
\def\boldt{{\bf T}}
\def\boldu{{\bf U}}
\def\boldv{{\bf V}}
\def\boldw{{\bf W}}
\def\boldx{{\bf X}}
\def\boldy{{\bf Y}}
\def\boldz{{\bf Z}}

%*-*-*-*-*-*-*-*-*
\def\greeka{\alpha}
\def\greekb{\beta}
\def\greekc{\xi}
\def\greekd{\delta}
\def\greeke{\epsilon}
\def\greekf{\phi}
\def\greekg{\gamma}
\def\greekh{\eta}
\def\greeki{\iota}
\def\greekj{j}
\def\greekk{\kappa}
\def\greekl{\lambda}
\def\greekm{\mu}
\def\greekn{\nu}
\def\greeko{o}
\def\greekp{\pi}
\def\greekq{\theta}
\def\greekr{\rho}
\def\greeks{\sigma}
\def\greekt{\tau}
\def\greeku{\upsilon}
\def\greekv{v}
\def\greekw{\omega}
\def\greekx{\chi}
\def\greeky{\psi}
\def\greekz{\zeta}

\def\greek #1{#1}
%*-*-*-*-*-*-*-*-*
%\land, \lor are defined correctly in TeX itself.
\def\bigbar{\vrule height8pt depth2pt width 1pt}

\def\turnstile{$\vdash$} %hx
\def\turnstyle{$\vdash$} %hx
\def\lbottom{\bot}
\def\powerset{{\cal P}} %hx
\def\leqp{E} %hx
\def\lone{\overline {\bf 1}} %hx
\def\lzero{\overline {\bf 0}} %hx
\def\percentage{\%} %hx
\def\cond{{\bf COND}}
\def\finite{{\bf FINITE}}
\def\funimage{\#}
\def\nat{{\bf NAT}}
\def\nc{{\bf NC}}
\def\recursion{{\bf recursion}}
\def\succ{{\bf SUCC}}
 
\def\arrowi{\mathord{\rightarrow{\rm I}}}
\def\arrowe{\mathord{\rightarrow{\rm E}}}
\def\botc{\perp_{C}}
\def\foralli{\mathord{\forall{\rm I}}}
\def\foralle{\mathord{\forall{\rm E}}}

\def\existse{\mathord{\exists{\rm E}}}
\def\ande{\mathord{\land{\rm E}}}
\def\ore{\mathord{\land{\rm E}}}
\def\eq{=}

%The following macros are used to print a square box. (hx 10/28/92)
\def\sqr#1#2{{\vcenter{\hrule height.#2pt
     \hbox{\vrule width.#2pt height#1pt \kern#1pt \vrule width.#2pt}
\hrule height.#2pt}}}
\def\square{\kern2pt\raise0.25pt\hbox{$\mathchoice\sqr56\sqr56\sqr{4.1}5\sqr{2.5}5$}\kern2.5pt}


\def\scripta{{\cal A}}

\def\scriptb{{\cal B}}

\def\scriptc{{\cal C}}

\def\scriptd{{\cal D}}

\def\scripte{{\cal E}}

\def\scriptf{{\cal F}}

\def\scriptg{{\cal G}}

\def\scripth{{\cal H}}

\def\scripti{{\cal I}}

\def\scriptj{{\cal J}}

\def\scriptk{{\cal K}}

\def\scriptl{{\cal L}}

\def\scriptm{{\cal M}}

\def\scriptn{{\cal N}}

\def\scripto{{\cal O}}

\def\scriptp{{\cal P}}

\def\scriptq{{\cal Q}}

\def\scriptr{{\cal R}}

\def\scripts{{\cal S}}

\def\scriptt{{\cal T}}

\def\scriptu{{\cal U}}

\def\scriptv{{\cal V}}

\def\scriptw{{\cal W}}

\def\scriptx{{\cal X}}

\def\scripty{{\cal Y}}

\def\scriptz{{\cal Z}}

\def\lintegral{\int}

\def\sub#1{\null_{#1}}

\def\sup#1{\null^{#1}}

\def\cut{\mathord{Cut}}

\def\mix{\mathord{Mix}}

\def\contr{\mathord{C}}

\def\defn{\mathord{defn}}

\def\ori{\mathord{\lor{\rm I}}}

\def\andi{\mathord{\land{\rm I}}}

\def\existsi{\mathord{\exists{\rm I}}}

\def\notnot{\mathord{\sim\sim}}

\def\ornot{\mathord{\lor\sim}}

\def\andnot{\mathord{\land\sim}}

\def\forallnot{\mathord{\forall\sim}}

\def\existsnot{\mathord{\exists\sim}}

\def\truth{\mathord{\top}}
\def\union{\cup}
\def\image{\mathord{\null^{>}}\kern 0pt}


\newbox\hypbox
\newdimen\turnstylelength
\newbox\formulabox
\newdimen\formulawd

\def\linenumbox#1{\hbox to 36pt{#1\hss}}
\def\emptyhypbox{\hbox to 72pt{\hss}}
\def\breakinhyp{\linenumbox{} \emptyhypbox}
\def\breakinformula{\linenumbox{} \emptyhypbox}
\def\hypnumbox#1{\setbox\hypbox=\hbox{#1} %
                 \ifdim \wd\hypbox > 72pt \box\hypbox\hfil\break\breakinhyp %
	                \else \hbox to 72pt{\unhbox\hypbox\hss}\fi}

\def\middleformula{middleformula}
\def\lastformula{lastformula}

\def\partformula#1#2{
   \setbox\formulabox =\hbox{#1}   
   \formulawd = \wd\formulabox 
   \advance\formulawd by 120pt
   \ifdim \formulawd > \hsize 
          \advance\formulawd by -130pt
          \setbox\formulabox =\hbox to \formulawd{\box\formulabox\hss}
          \advance\formulawd by 130pt
          \else \fi
   \box\formulabox 
   \ifx\middleformula #2
       \ifdim \formulawd > \hsize \hfil\penalty-1000 \else \hfil\break \fi
       \else \fi}

\def\judgelink#1{
   \ifdim \formulawd > \hsize\penalty-1000\rightline{#1}\else \setbox\formulabox = \hbox{#1}   
                \advance \formulawd by \wd\formulabox
                \advance \formulawd by 5pt
                \ifdim \formulawd >\hsize\hfil\break\rightline{#1}\else\hfill\box\formulabox\fi\fi}
               
\hfuzz = 20pt

% Added for alternative formatting of nd proofs in TeX - cebrown 3/8/03

\def\setubspaces#1{\begingroup\noexpandarg\StrSubstitute{#1}{ }{~}\endgroup}

\def\raggedleft{\leftskip=0pt plus \hsize \parfillskip=0pt\rightskip=0pt}

\def\cebproofline#1#2#3#4{
\setbox\formulabox = \hbox{(#1)\hskip 5mm #2 \turnstile #3}
\formulawd = \wd\formulabox
\setbox\formulabox = \hbox{#4}
\advance \formulawd by \wd\formulabox
\advance \formulawd by 1.5cm
\ifdim \formulawd > \hsize
  \vbox{
    \vbox{(#1)\hskip 5mm #2 \turnstile #3}\par
    \vskip 3pt plus 2pt minus 1pt
    \vbox{\raggedleft{#4}}\par
  }
\else
  \vbox{(#1)\hskip 5mm #2 \turnstile #3 \hfill #4}\par
\fi}

\def\cebprooflinea#1#2#3#4#5{
\setbox\formulabox = \hbox{(#1)\hskip 5mm #2}
\formulawd = \wd\formulabox
\ifdim #3 > \hsize
  \vbox{
    \vbox{(#1)\hskip 5mm #2}\hfill\break
    \vbox{\turnstile #4}\par
    \vskip 3pt plus 2pt minus 1pt
    \vbox{\raggedleft #5}\par
    \vskip 3pt plus 2pt minus 1pt
  }
\else
  \ifdim \formulawd > #3
    \cebproofline{#1}{#2}{#4}{#5}
  \else
    \formulawd = #3
    \setbox\formulabox = \hbox{#4}
    \advance \formulawd by \wd\formulabox
    \setbox\formulabox = \hbox{#5}
    \advance \formulawd by \wd\formulabox
    \advance \formulawd by 1.5cm
    \ifdim \formulawd > \hsize
      \vbox{
        \vbox{\indent\hbox{\hbox to #3{(#1)\hskip 5mm #2} \turnstile} #4}\par
        \vskip 3pt plus 2pt minus 1pt
        \vbox{\raggedleft #5}\par
        \vskip 3pt plus 2pt minus 1pt
      }
    \else
      \vbox{\indent\hbox{\hbox to #3{(#1)\hskip 5mm #2} \turnstile} #4 \hfill #5}\par
    \fi
  \fi
\fi
}

\def\ceblinejust#1{\par\vbox{\raggedleft{#1}}\par}


%---------------------------------------------------
%the following is used to display timing recording
\newbox\processname
\newbox\realtime
\newbox\inruntime
\newbox\gctime
\newbox\igctime

\newdimen\processnamewd
\newdimen\realtimewd
\newdimen\inruntimewd
\newdimen\gctimewd
\newdimen\igctimewd
\newdimen\diffwd
\newdimen\ndiffwd

\setbox\processname=\hbox{Proof transformation}
\setbox\realtime=\hbox{Real time}
\setbox\inruntime=\hbox{Internal-run time}
\setbox\gctime=\hbox{GC time}
\setbox\igctime=\hbox{I-GC time}

\processnamewd=\wd\processname
\setbox\processname=\hbox to \processnamewd{\hss Process name\hss}
\realtimewd=\wd\realtime
\inruntimewd=\wd\inruntime
\gctimewd=\wd\gctime
\igctimewd=\wd\igctime

\diffwd=\hsize
\advance \diffwd by -\processnamewd
\advance \diffwd by -\realtimewd
\advance \diffwd by -\inruntimewd
\advance \diffwd by -\gctimewd
\advance \diffwd by -\igctimewd
\divide \diffwd by 5

\advance \processnamewd by \diffwd 
\advance \realtimewd by \diffwd 
\advance \inruntimewd by \diffwd 
\advance \gctimewd by \diffwd 
\advance \igctimewd by \diffwd 

\ndiffwd=\hsize
\advance \ndiffwd by -\processnamewd

\divide \diffwd by 2

\setbox\processname=\hbox to \processnamewd{\hskip\diffwd Process name\hss}
\setbox\realtime=\hbox to \realtimewd{\hss Real time\hss}
\setbox\inruntime=\hbox to \inruntimewd{\hss Internal-run time\hss}
\setbox\gctime=\hbox to \gctimewd{\hss GC time\hss}
\setbox\igctime=\hbox to \igctimewd{\hss I-GC time\hss}

\def\timeline#1#2#3#4#5{
  \hbox{\strut
        \hbox to \processnamewd{\hskip\diffwd#1\hss}\vrule
        \hbox to \realtimewd{\hss#2\hskip\diffwd}\vrule
        \hbox to \inruntimewd{\hss#3\hskip\diffwd}\vrule
        \hbox to \gctimewd{\hss#4\hskip\diffwd}\vrule
        \hbox to \igctimewd{\hss#5\hskip\diffwd} } }


\def\ntimeline#1{\strut
\hbox to \processnamewd{\hskip\diffwd#1\hss}\vrule
\hbox to \ndiffwd{\hss not called in this proof\hss}}

%the above is used to display timing recording
%---------------------------------------------------
%The following is used to print character documentation
\def\today{
\ifcase\month\or January\or February\or
             March\or April\or May\or June\or July\or August\or
             September\or Octobor\or November\or December\fi
\space\number\day, \number\year}

\def\frontpageofchardoc{
\font\bigfont=cmbx10 at 14.4pt
\font\midfont=cmbx10 at 12pt 
\footline{\hfill}
\vbox to \vsize{\vfill
\centerline{{\bigfont TPS3 Character Sets}}
\vskip10pt
\centerline{{\midfont\today}}
\vskip10pt
\centerline{{\midfont Peter Andrews}}
\vskip5pt
\centerline{{\midfont Sunil Issar}}
\vskip5pt
\centerline{{\midfont Carl Klapper}}
\vskip5pt
\centerline{{\midfont Dan Nesmith}}
\vskip5pt
\centerline{{\midfont Hongwei Xi}}
\vskip100pt
\centerline{Working Edition}
\vfill}
\eject

\footline{\hfil\folio\hfil}
\pageno=1
\headline={\ifnum\pageno>1
                 {{\bf TPS3 Character Sets}\hfill{\bf Help for Special Characters}}
                 \else{}\fi}}


%---------------------------------------------------

\newbox\tempa
\newbox\tempb
\newdimen\tempc
\newbox\tempd

%\newdimen\halfsize

%\newdimen\halfbaselineskip

\def\inruleanhelp#1#2#3{\setbox\tempa=\hbox{$\displaystyle{\mathstrut #2}$}%
		        \setbox\tempb=\vbox{\halign{##\cr
	\mud{#1}\cr
	\noalign{\vskip\the\lineskip}%
	\noalign{\hrule height 0pt}%
	\rig{\vbox to 0pt{\vss\hbox to 0pt{${\; #3}$\hss}\vss}}\cr
	\noalign{\hrule}%
	\noalign{\vskip\the\lineskip}%
	\mud{\copy\tempa}\cr}}%
		      \tempc=\wd\tempb
		      \advance\tempc by \wd\tempa
		      \divide\tempc by 2 }

\def\inrulemhelp#1#2#3{\setbox\tempa=\hbox{$\displaystyle{\mathstrut #2}$}%
		        \setbox\tempb=\vbox{\halign{##\cr
	\mud{#1}\cr
	\noalign{\vskip\the\lineskip}%
	\noalign{\hrule}%
	\noalign{\vskip\the\lineskip}%
	\noalign{\hrule height 0pt}%
	\rig{\vbox to 0pt{\vss\hbox to 0pt{${\; #3}$\hss}\vss}}\cr
	\noalign{\hrule}%
	\noalign{\vskip\the\lineskip}%
	\mud{\copy\tempa}\cr}}%
		      \tempc=\wd\tempb
		      \advance\tempc by \wd\tempa
		      \divide\tempc by 2 }


\def\inrulemchelp#1#2#3{\setbox\tempa=\hbox{$\displaystyle{\mathstrut #2}$}%
			\setbox\tempd=\hbox{$\; #3$}%
		        \setbox\tempb=\vbox{\halign{##\cr
	\mud{#1}\cr
	\noalign{\vskip\the\lineskip}%
	\noalign{\hrule}%
	\noalign{\vskip\the\lineskip}%
	\noalign{\hrule height 0pt}%
	\rig{\vbox to 0pt{\vss\hbox to 0pt{${\; #3}$\hss}\vss}}\cr
	\noalign{\hrule}%
	\noalign{\vskip\the\lineskip}%
	\mud{\copy\tempa}\cr}}%
		      \tempc=\wd\tempb
		      \advance\tempc by \wd\tempa
		      \divide\tempc by 2 }

\def\inruleanchelp#1#2#3{\setbox\tempa=\hbox{$\displaystyle{\mathstrut #2}$}%
			\setbox\tempd=\hbox{$\; #3$}%
		        \setbox\tempb=\vbox{\halign{##\cr
	\mud{#1}\cr
	\noalign{\vskip\the\lineskip}%
	\noalign{\hrule height 0pt}%
	\rig{\vbox to 0pt{\vss\hbox to 0pt{\copy\tempd \hss}\vss}}\cr
	\noalign{\hrule}%
	\noalign{\vskip\the\lineskip}%
	\mud{\copy\tempa}\cr}}%
		      \tempc=\wd\tempb
		      \advance\tempc by \wd\tempa
		      \divide\tempc by 2 }

\def\inrulean#1#2#3{{\inruleanhelp{#1}{#2}{#3}%
		     \hbox to \wd\tempa{\hss \box\tempb \hss}}}

%%%
%%% Alternative in-text definition for deductions
%%%



\def\inruleanc#1#2#3{{\inruleanchelp{#1}{#2}{#3}%
		     %\halfbaselineskip=\baselineskip
		     %\divide\halfbaselineskip by 2
		     \raisebox{-0.5\baselineskip}{\box\tempb\hskip\wd\tempd}}}

\def\inrulemc#1#2#3{{\inrulemchelp{#1}{#2}{#3}%
		     %\halfbaselineskip=\baselineskip
		     %\divide\halfbaselineskip by 2
		     \raisebox{-0.5\baselineskip}{\box\tempb\hskip\wd\tempd}}}


\def\inrulem#1#2#3{{\inrulemhelp{#1}{#2}{#3}%
		    \hbox to \wd\tempa{\hss \box\tempb \hss}}}

\def\ginrulem#1#2#3{{\inrulemhelp{#1}{#2}{#3}%
		     %\halfsize=\linewidth
		     %\divide\halfsize by 2
		     \hbox to .5\linewidth{\hfil \box\tempb \hfil}}}

\def\ginruleml#1#2#3{{\inrulemchelp{#1}{#2}{#3}%
		     %\halfsize=\linewidth
		     %\divide\halfsize by 2
		     \hbox to .5\linewidth{\hfil \box\tempb \hskip\wd\tempd \hfil
		     }}}

\def\linrulem#1#2#3{{\inrulemhelp{#1}{#2}{#3}%
		      \hbox to \tempc{\hss \box\tempb}}}

\def\rinrulem#1#2#3{{\inrulemhelp{#1}{#2}{#3}%
		      \hbox to \tempc{\box\tempb \hss}}}


\def\ginruleanl#1#2#3{{\inruleanchelp{#1}{#2}{#3}%
		     %\halfsize=\linewidth
		     %\divide\halfsize by 2
		       \hbox to .5\linewidth{\hfil
		         \box\tempb\hskip\wd\tempd \hfil
		       }}}

\def\ginrulean#1#2#3{{\inruleanhelp{#1}{#2}{#3}%
		     %\halfsize=\linewidth
		     %\divide\halfsize by 2
		     \hbox to .5\linewidth{\hfil \box\tempb \hfil}}}

\def\linrulean#1#2#3{{\inruleanhelp{#1}{#2}{#3}%
		      \hbox to \tempc{\hss \box\tempb}}}

\def\rinrulean#1#2#3{{\inruleanhelp{#1}{#2}{#3}%
		      \hbox to \tempc{\box\tempb \hss}}}

\newbox\gappremises

\def\gap#1{\gapbuild#1{}\inrulegap{\box\gappremises}}

\def\gapbuild#1{\if *#1*\let\next=\relax\else
        \setbox\gappremises=\hbox{\ifvoid\gappremises\else
					 \unhbox\gappremises \qquad\fi
				  \mud{#1}}%
	\let\next=\gapbuild\fi\next}

\def\gaphelp#1#2#3{\setbox\tempa=\hbox{$\displaystyle{\mathstrut #2}$}%
		   \setbox\tempb=\vbox{\lineskip=2pt%
 \halign{##\cr
	\mud{#1}\cr
	%\noalign{\vskip\the\lineskip}%
	\noalign{\hrule height 0pt}%
	\mud{#3}\cr
	\noalign{\hrule height 0pt}%
	\noalign{\vskip\the\lineskip}%
	\mud{\copy\tempa}\cr}}%
		      \tempc=\wd\tempb
		      \advance\tempc by \wd\tempa
		      \divide\tempc by 2 }

\def\gapchelp#1#2#3{\setbox\tempa=\hbox{$\displaystyle{\mathstrut #2}$}%
		   \setbox\tempb=\hbox{$\vcenter{\lineskip=2pt%
 \halign{##\cr
	\mud{#1}\cr
	%\noalign{\vskip\the\lineskip}%
	\noalign{\hrule height 0pt}%
	\mud{#3}\cr
	\noalign{\hrule height 0pt}%
	\noalign{\vskip\the\lineskip}%
	\mud{\copy\tempa}\cr}}$}%
		      \tempc=\wd\tempb
		      \advance\tempc by \wd\tempa
		      \divide\tempc by 2 }



\def\inrulegap#1#2#3{{\gaphelp{#1}{#2}{#3}%
		      \hbox to \wd\tempa{\hss \box\tempb \hss}}}

\def\ginrulegap#1#2#3{{\gaphelp{#1}{#2}{#3}%
		       \hbox to 0.5\linewidth{\hfil \box\tempb \hfil}}}

\def\inrulegapc#1#2#3{{\gaphelp{#1}{#2}{#3}%
		     %\halfbaselineskip=\baselineskip
		     %\divide\halfbaselineskip by 2
		     \raisebox{-0.5\baselineskip}{\box\tempb}}}


\def\inrulebn#1#2#3#4{\inrulean{#1\quad\qquad #2}{#3}{#4}}
\def\inrulebnc#1#2#3#4{\inruleanc{#1\quad\qquad #2}{#3}{#4}}
\def\linrulebn#1#2#3#4{\linrulean{#1\quad\qquad #2}{#3}{#4}}
\def\ginrulebn#1#2#3#4{\ginrulean{#1\quad\qquad #2}{#3}{#4}}
\def\ginrulebnl#1#2#3#4{\ginruleanl{#1\quad\qquad #2}{#3}{#4}}
\def\rinrulebn#1#2#3#4{\rinrulean{#1\quad\qquad #2}{#3}{#4}}

\def\mud#1{\hfil $\displaystyle{#1}$\hfil}
\def\rig#1{\hfil $\displaystyle{#1}$}

\def\inruleax#1#2{\quad\vbox{\halign{\hfil $\displaystyle{##}$\hfil\cr
				     #1\cr
				     \noalign{\vskip\the\lineskip
				              \hrule
					      \vskip\the\lineskip}%
				     \cdots\cr
				     \noalign{\vskip\the\lineskip
				              \hrule
					      \vskip\the\lineskip}%
				     #2\cr}}\;}


\def\inruleadots#1#2{\quad\vbox{\halign{\hfil $\displaystyle{##}$\hfil\cr
				     #1\hskip 3em\cr
				     \noalign{\vskip\the\lineskip
				              \hrule
					      \vskip\the\lineskip}%
				     \ddots\cr
				     \noalign{\vskip\the\lineskip
				              \hrule
					      \vskip\the\lineskip}%
				     \hskip 3em #2\cr}}\;}


% cebrown - 11/5/00 - added macros for typesetting expansion trees.
\def\etra#1#2{{\begin{array}[t]{c}#1 \\ \begin{array}[t]{c} \downarrow \\ #2\end{array}\end{array}}}
\def\etrb#1#2#3{{\begin{array}[t]{c}#1 \\ \begin{array}[t]{cc} \swarrow & \searrow \\ #2 & #3 \end{array}\end{array}}}
\def\etraa#1#2#3{{\begin{array}[t]{c}#1 \\ \begin{array}[t]{c} \downarrow #2 \\ #3\end{array}\end{array}}}
\def\etrba#1#2#3#4#5{{\begin{array}[t]{c}#1 \\ \begin{array}[t]{cc} #2 \swarrow & \searrow #4 \\ #3 & #5 \end{array}\end{array}}}
\def\etrca#1#2#3#4#5#6#7{{\begin{array}[t]{c}#1 \\ \begin{array}[t]{ccc} #2 \swarrow & \downarrow #4 & \searrow #6 \\ #3 & #5 & #7 \end{array}\end{array}}}

\def\erew#1#2#3#4{{\etra{(#2)REW#1: #3}{#4}}}
\def\erewstar#1#2#3{{\etra{#1 REW^\ast:#2}{#3}}}
\def\erewstarnf#1#2{{\etra{#1 REW^\ast}{#2}}}
\def\eleaf#1#2{{LEAF#1: #2}}
\def\econj#1#2#3#4{{\etrb{CONJ#1}{#3}{#4}}}  % or CONJ#1: #2 to include fmla
\def\edisj#1#2#3#4{{\etrb{DISJ#1}{#3}{#4}}}  % or DISJ#1: #2 to include fmla
\def\eimp#1#2#3#4{{\etrb{IMP#1}{#3}{#4}}}  % or IMP#1: #2 to include fmla
\def\eneg#1#2#3{{\etra{NEG#1}{#3}}}  % or NEG#1: #2 to include fmla
\def\uexpnode#1#2#3#4{{\etraa{EXP#1: #2}{#3}{#4}}}
\def\bexpnode#1#2#3#4#5#6{{\etrba{EXP#1: #2}{#3}{#4}{#5}{#6}}}
\def\gexpnode#1#2#3#4#5#6{{\etrca{EXP#1: #2}{#3}{#4}{\ast}{\cdots}{#5}{#6}}}
\def\sel#1#2#3#4{{\etraa{SEL#1: #2}{#3}{#4}}}
\def\skol#1#2#3#4{{\etraa{SKOL#1: #2}{#3}{#4}}}
\def\truenode#1{{TRUE#1}}
\def\falsenode#1{{FALSE#1}}
\def\etrgap#1{{\begin{array}[t]{c}\cdots \\ \downarrow \\ #1\end{array}}}

