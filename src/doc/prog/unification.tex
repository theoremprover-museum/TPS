\chapter{Unification}

The relevant files are: {\it ms90-3-node.lisp}, {\it ms90-3-unif*.lisp}, 
{\it node.lisp}, {\it unif*.lisp}

\TPS has four unification algorithms, two for first-order logic and
two for the full type theory. Here we are mainly concerned with the two
type theory ones, which differ as follows:

\begin{itemize}
\item UN88 is called by those procedures which do not use
path-focused duplication, and by the \TPS UNIFY top level.
Each variable is a symbol. We use lazy reduction. Head normal form.
General implementation. Can use different strategies for searching the
unification tree. Default breadth-first. Requires storing almost the
entire tree.
When called from mating-search, we search for a success node or generate the
tree to a pre-determined maximum depth.

\item UN90 is called by those procedures which do use
path-focused duplication. No interactive interface exists now.
Each variable has the form (symbol . number)
Terms are reduced to $\lambda$-normal form as in Huet's paper.
Depth-first search. Stores only non-failure leaf node.
Does not store the entire unification tree.
When called from mating-search, we search for the first non-failure node
within the pre-determined maximum depth.  Search for a success node 
only when the mating is complete.
Major drawback: Needs modification to implement subsumption. 
\end{itemize}


\section{Data Structures}
\section{Computing Head Normal Form}
\section{Control Structure}
\section{First-Order Unification}
\section{Subsumption Checking}
There is a subsumption checker for UN88 which uses the slot {\tt subsumed} in 
each node of the unification tree; this is implemented in the file {\it unif-subs.lisp}.
The subsumption-checker is passed the new node and a list of other nodes which 
might subsume it. If \indexflag{SUBSUMPTION-CHECK} is NIL, it returns immediately.
Otherwise, it first checks the flags \indexflag{SUBSUMPTION-NODES} and \indexflag{SUBSUMPTION-DEPTH}
and eliminates all nodes from the list that do not fit the criteria established by these two flags 
(so it might, for example, pick out just those nodes at a depth of less than ten which lie 
either on the path to the new node or at the leaves of the current unification tree). Since
it is possible to add new disagreement pairs to the leaves of the tree under some conditions,
it also rejects any nodes that do not represent the same original set of disagreement pairs
as the new node.

Then it computes a hash function, somewhat similar to Goedel-numbering, by considering each wff
in the set of disagreement pairs at a node. The hash function has to ignore variables, because
we want to catch nodes that are the same up to a change in the h-variables that have been introduced.
These hash numbers are calculated once and then stored
in the {\tt subsumed} slot in the following format: for a dpairset 
%\begin{tpsexample}
\begin{verbatim}
((A1 . B1) (A2 . B2) ...)
\end{verbatim}
%\end{tpsexample}
we first calculate the hash numbers for each wff, and generate the following list:
%\begin{tpsexample}
\begin{verbatim}
(((#A1 . #B1) (A1 . B1)) ((#A2 . #B2) (A2 . B2)) ...)
\end{verbatim}
%\end{tpsexample}
Then, for each disagreement pair, if \#Bi < \#Ai we replace it with ((\#Bi . \#Ai) (Bi . Ai)).
Finally, we sort the list lexicographically by the pairs of hash numbers and store it in the
{\tt subsumed} slot. In future, if we return to this node, we can just read off the hash
function without recalculating it.

Now \TPS compares the dotted pairs of numbers from the hash functions of the new and old node.
If those for the new node are equal to, or a superset of, those for the old node, then we need
to do some more detailed checking. This is the point at which \TPS prints a "?", if \indexflag{UNIFY-VERBOSE}
is not SILENT. Otherwise we know there is no subsumption and proceed to the next node.

If there is still a possibility of subsumption, the next thing to do is to enumerate all the 
ways in which the old node might be considered a subset of the new one. If we are lucky, 
each dotted pair of numbers in a given node will be different from each other and from all other dotted pairs
at that node, and there will only be one way in which this could happen. If we aren't so lucky (if 
there are several disagreement pairs that get the same pair of hash numbers, or if there is a 
disagreement pair where the hash numbers for both wffs are the same), there may
be multiple ways to think about. For each possible way, we output two disagreement pair lists, which 
will be the entire old node and that subset of the new node to which it might correspond, ordered 
so that the nth element of one is supposed to compare to the nth element of the other, for all n.

Next, for each one of these possible ways, we take the two disagreement pair sets given, and begin
to rename the h-variables in them. We start at the left of both sets, and build up a substitution as we move rightwards,
comparing each term to the other symbol-by-symbol.
(Note that we are only replacing variables with other variables.) 
If we reach the end of the term without contradicting ourselves, we output a
"!" and the new node is subsumed. If we fail (because the substitution is inconsistent, or because we reach 
two different variables neither of which is an h-variable), we fail immediately
and go on to the next arrangement, if there is one.

Subsumption-checking can be very slow; set the flag \indexflag{SUBSUMPTION-DEPTH} with care.
Because of this, it was necessary to add time-checking to unification (it was previously only 
done between considering connections). The functions {\tt unify}, {\tt unify-ho-rec} and
{\tt subsumption-check} now check the time if they are called from within a procedure that 
uses time limits (and in order to implement this, many other unification functions have 
been given optional "start-time" and "time-limit" arguments that they do nothing 
with except passing them on to the next function).

\section{Notes}

The code that \TPS uses to handle double-negations is part of the unification 
code. See \indexfunction{imitation-eta} in the file \indexfile{unif-match.lisp}


