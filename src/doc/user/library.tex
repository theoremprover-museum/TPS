\chapter{Using the library}\label{library}
A library facility exists, and can be accessed by
entering the library top-level via the \indexcommand{LIB} command
or by entering the Unix-style library top-level via the \indexcommand{UNIXLIB}.
The library provides a way of storing various types of objects
(e.g., gwff, abbreviation, etc.) in files, which are collected in directories.
A library directory is a directory of the file system containing an index file whose
name is determined by the flag \indexflag{LIB-MASTERINDEX-FILE}.
(The default value for \indexflag{LIB-MASTERINDEX-FILE} is currently
{\tt `libindex.rec'}, so that a library directory is a directory containing
a file named {\tt `libindex.rec'}.)  This index file associates the objects
stored in the directory with the type of the object and the file in which
the object is stored.
{\TPS} automatically maintains the index file as objects are inserted, deleted, renamed, etc.
Also, index files are affected when library files are manipulated.
The objects in the library can be classified into classification schemes.
Classification schemes themselves are objects which can be stored in the
library.

The user has the option of using multiple libraries at the same time. The names
of these libraries are
stored in the flags \indexflag{DEFAULT-LIB-DIR} (for the libraries which the user can both read
and write) and \indexflag{BACKUP-LIB-DIR} (for the libraries which the user can read only).
It is a good idea to have a line such as
{\tt (set-flag default-lib-dir '(`/afs/cs/project/tps/tps/tpslib/andrews/'))}
in your {\it tps3.ini} file, to save having to type this every time {\TPS} is loaded.
Note the {\tt `/'} at the end of the string!
Changing the value of this flag will make {\TPS} reload the library index files.
(Note that including a directory in the value of these flags does not
create the library directory.  See the next paragraph.)
If \indexflag{ADD-SUBDIRECTORIES}
is T, then at the same time {\TPS} will look for any subdirectories of the library directories which are
in themselves library directories, and add them to \indexflag{DEFAULT-LIB-DIR} or \indexflag{BACKUP-LIB-DIR} as appropriate.
Note that \indexflag{DEFAULT-LIB-DIR} and \indexflag{BACKUP-LIB-DIR}
are flags.  As such their values can be explicitly changed by the user to any list
of library directories.

The TPS distribution includes a library directory called
`distributed'.  This directory contains a variety of library
objects which have been defined over the years.
To have access to this directory, the user should include
the {\tt `/whatever/tps/library/distributed'} directory in the value of
\indexflag{BACKUP-LIB-DIR}.

Library directories can be created using \indexcommand{CREATE-LIB-DIR}, and
library subdirectories can be created using \indexcommand{CREATE-LIB-SUBDIR}.
Library subdirectories may be used to allow overloading of names for library objects.
For example, one could have two different abbreviations {\tt GROUP} with different
definitions as long as they are stored in different library directories.
It is illegal to have two different library objects with the same name and type in the same
library directory.  (WARNING:  Not all TPS library commands currently enforce this.)
The intended use of BACKUP-LIB-DIR is, of course, to make it possible for a group of
users to have a common library of useful definitions in addition to their own private library
workspace. If there is no such common library, set BACKUP-LIB-DIR to NIL.

When {\TPS} starts up,
it creates a master index by loading the index files in the library directories listed
in DEFAULT-LIB-DIR and BACKUP-LIB-DIR.  This master index is recreated by certain
commands involving library directories (such as \indexcommand{CREATE-LIB-DIR}) or when either
DEFAULT-LIB-DIR
or BACKUP-LIB-DIR is changed.  The master index can be explicitly recreated by
the command \indexcommand{RESTORE-MASTERINDEX}.

Within the library top-level, the user can
\begin{itemize}
\item save and retrieve {\TPS} objects. {\tt HELP LIB-ARGTYPE} provides a list of types of objects
that can be saved in the library. When saving objects in the library, the user
will be prompted for any attributes that are required. This includes other
library objects that must be retrieved before the object that is currently
being saved is retrieved.

\item modify existing library objects (by using the \indexcommand{INSERT}
command with the name of an existing object)

\item display objects, parts of objects, or even entire files from the library, and produce lists of
all the files in the directory, all the objects in a file, or all the objects in a directory.

\item output this information in a format suitable for processing and printing by Scribe or TeX.

\item perform simple file maintenance tasks such as copying, deleting and moving objects
(or entire files or directories), or listing all the files in a directory.

\item reformat library files. For sake of efficiency, when modifying existing
objects in the library, no formatting is done for the objects that
follow or precede the object that is being modified. Hence, over a period
of time, the library files may be in a form that you may find difficult to
read. At such times you may want to reformat the files, so that they are
in a more readable form.
\end{itemize}

\section{Storing and Retrieving Objects}

All of the following types of object can be stored in the library: gwff, abbreviation, constant,
mode, mode1, rewrite rule, theory, searchlist and dpairset. Some of these require extra input from the
user, and are discussed in more detail here; the others are discussed elsewhere in this manual,
and once defined can be saved in the library without extra input. (For example, dpairsets are discussed
in the section about the unification top level, and searchlists in the section about the test top level.)

A list of keywords is stored in the library directory in the file specified by \indexflag{LIB-KEYWORD-FILE}.
An arbitrary list of keywords can be attached to each library object; these keywords can then be used for
searches using \indexcommand{KEY} or \indexcommand{SHOW-ALL-WFFS}. The keyword list is user-specifiable,
using the \indexcommand{ADD-KEYWORD} and \indexcommand{SHOW-KEYWORDS} commands.  For more information
on keywords, see section \ref{KEYWORDS}.

The library also stores a file {\it bestmodes.rec} which associates theorems to modes in which those theorems
can be proven automatically. All of the {\it bestmodes.rec} files in any library directory are available
simultaneously. The commands \indexcommand{FIND-MODE} and \indexcommand{SHOW-BESTMODE} search this
file, and new modes can be inserted either during a call to \indexcommand{DATEREC}, or by using the command
\indexcommand{ADD-BESTMODE}. The \indexcommand{MODEREC} command invokes ADD-BESTMODE with the name of the
most recently proven theorem and mode.

Assume that the user wishes to define his own abbreviation for an {\tt EMPTYSET}.
He should enter the library top-level, and use the command
\indexcommand{INSERT}. He'll then be prompted for various attributes that are
necessary. As usual, \indexcommand{?} will provide some help on the
appropriate responses.
In this session, and in all subsequent sessions {\TPS} will recognize
{\tt EMPTYSET} as a library object. When the user needs to use the
library object {\tt EMPTYSET}, he should first use the library command
\indexcommand{FETCH} to make this library object available within {\TPS}.
If objects in several directories have the same name, and \indexflag{SHOW-ALL-LIBOBJECTS} is set to T,
the user will be asked which one to FETCH. Otherwise, the directories are taken in the order given in
DEFAULT-LIB-DIR, and then BACKUP-LIB-DIR, and the first directory containing such an object will be
used.

A mode can be stored in the library. This is a list of flag settings (for convenience,
there is also an object called a \indexother{MODE1}; this consists of
all the flags belonging to a given list of subjects, and this may also be stored
in the library). Users who discover that a particular collection of flag settings
produces a fast proof of a theorem may find it useful to record these settings
as a mode in their library. See chapter \ref{flags} for more information about
modes.

When entering abbreviations, if the \indexother{FO-SINGLE-SYMBOL} property is
T, you can specify the abbreviation in uppercase, lowercase,  or any
combination of these. Otherwise it has to be specified as an uppercase
symbol only.

The \indexother{FACE} property controls how the abbreviation will be printed by {\TPS}.
It should be a list of symbols (which may include special printing
symbols such as POWERSET); type ? at the prompt for more information.

Also, the correct format for typelists in abbreviations entered in
the library is that illustrated by {\tt (`A' `B')}.  The user who types
{\tt ?} when prompted for the typelist will see the example. See the programmer's
manual for more information about defining abbreviations, logical constants, etc.

A library object is allowed to depend on other library objects, called needed-objects
(for example, one might define an abbreviation {\tt BIJECTIVE} in terms of
other abbreviations {\tt INJECTIVE} and {\tt SURJECTIVE}). {\TPS} will attempt
to type-check all objects it inserts into the library, and for this reason all definitions
must be made from the bottom up (in our example, {\tt INJECTIVE} and {\tt SURJECTIVE}
would have to already exist at the time when we attempt to define {\tt BIJECTIVE}).
When a definition is {\tt FETCH}ed, all the needed-objects will be retrieved with it;
an error will be signalled if they cannot be found.  Needed objects will be loaded
from the same directory as the original object if this is possible, and otherwise from the first
place in which they are found.
Definitions may be nested arbitrarily deeply, and needed-objects are assumed to
be abbreviations, unless no abbreviation of the right name can be
found in the library, in which case they're assumed to be gwffs.
The command \indexcommand{CHECK-NEEDED-OBJECTS} can be used to check if a library
directory is closed with respect to needed-objects.  The command
\indexcommand{IMPORT-NEEDED-OBJECTS}
can be used to copy any extra needed-objects into a directory (so the directory
will be closed with respect to needed-objects).

Gwffs stored in the library are also equipped with a PROVABILITY property, which
indicates the current state of attempts to prove them in {\TPS}. This property can only be changed by the user
({\TPS} will never automatically change it), using the commands \indexcommand{DATEREC} or
\indexcommand{CHANGE-PROVABILITY}. The command \indexcommand{FIND-PROVABLE} looks for all gwffs
with a specified provability status.

The command \indexcommand{RETRIEVE-FILE} will load all the objects in a given library file.

There are various commands in other top levels that affect the library. For example,
\indexcommand{DATEREC} stores timing information, and the \indexcommand{TEST} top level
has its own commands for saving and loading searchlists. Also the commands \indexcommand{BUG-SAVE},
\indexcommand{BUG-RESTORE}, and their associated commands, create a separate library of
bugs in the directory given by \indexflag{DEFAULT-BUG-DIR} (although these commands can also
save bugs to your personal library if you prefer).

\section{Displaying Objects}
The \indexcommand{SHOW} command displays a single object from the library.
Library objects can be very long (particularly if the object is a gwff, and
\indexcommand{DATEREC} has
been used to append information regarding attempts at proving the gwff), and
there are commands \indexcommand{SHOW-WFF}, \indexcommand{SHOW-MHELP} and
\indexcommand{SHOW-WFF\&HELP} which display only a part of a stored object.

The user can also display the names of all the objects in a file, using
\indexcommand{LIBOBJECTS-IN-FILE}, or all the wffs in a file with
\indexcommand{SHOW-WFFS-IN-FILE}.
The analogues of these commands for the entire directory, rather than just one
file, are \indexcommand{LIST-OF-LIBOBJECTS}
and \indexcommand{SHOW-ALL-WFFS}; the latter can take some time.

To find an object whose name you only partly remember, use the command \indexcommand{KEY}.
For example, {\tt KEY `THM135' !} will list all objects, of any type, in the current and
backup directories, whose names contain
the string `THM135'. The \indexcommand{SEARCH} and \indexcommand{SEARCH2} commands do similar
things, but search the entire text of library objects rather than just their names.

\section{File Maintenance}
Library directories and subdirectories can be created using \indexcommand{CREATE-LIB-DIR} and
\indexcommand{CREATE-LIB-SUBDIR}, deleted using
\indexcommand{DELETE-LIB-DIR}, copied using \indexcommand{COPY-LIBDIR}, and
renamed using \indexcommand{RENAME-LIBDIR}.  The command \indexcommand{CREATE-LIB-DIR}
will add the new library directory to DEFAULT-LIB-DIR.
Also, \indexcommand{DELETE-LIB-DIR}
will delete the directory from DEFAULT-LIB-DIR, if it was an entry in this list.
(All changes to DEFAULT-LIB-DIR only apply during the current {\TPS} session.)
The command \indexcommand{COPY-LIBDIR} can be used in two ways.  First, it can
be used to create a new library directory (which will be added to DEFAULT-LIB-DIR)
and copy the contents of an existing directory into this new directory.  Second,
it can be used to copy the contents of an existing directory into an existing directory.
In this second case, if an object with the same name and type exists already in both
the source and destination directories, the object will not be copied.  Instead, the
original object in the destination directory will be kept.  Also,
\indexcommand{COPY-LIBDIR} will copy the bestmodes and keywords information files
from the source directory to the target directory.  If the target directory
already contains a bestmodes or keywords information file, \indexcommand{COPY-LIBDIR}
will merge the information from both directories.

A `common' library directory containing a copy of all library objects
in every directory in \indexflag{DEFAULT-LIB-DIR} and \indexflag{BACKUP-LIB-DIR}
can be created and maintained using the command \indexcommand{UPDATE-LIBDIR}.
\indexcommand{UPDATE-LIBDIR} has the same effect as calling \indexcommand{COPY-LIBDIR}
on each directory in \indexflag{DEFAULT-LIB-DIR} and \indexflag{BACKUP-LIB-DIR}.

Library files can be created implicitly by \indexcommand{INSERT} when a new object is placed in
the file.  Library files can be deleted implicitly when the last object stored in
the file is deleted using \indexcommand{DELETE} or moved into a different file
using \indexcommand{MOVE-LIBOBJECT}.  Library files can also be deleted explicitly, along
with all the objects stored in the file, using \indexcommand{DELETE-LIBFILE}.  Files
can be renamed (within the same directory) using \indexcommand{RENAME-LIBFILE},
moved (into a different directory) using \indexcommand{MOVE-LIBFILE}, and
copied (into a different directory) using \indexcommand{COPY-LIBFILE}.

As discussed above, new objects can be inserted into the library using \indexcommand{INSERT}.
The \indexcommand{INSERT} command can also be used to modify an existing library object.
A library object can be deleted using \indexcommand{DELETE}, renamed (within the same library
file) using \indexcommand{RENAME-OBJECT}, and moved (into a new file and possibly new directory)
using \indexcommand{MOVE-LIBOBJECT}.  (In fact, \indexcommand{MOVE-LIBOBJECT} can be used
more generally to move several objects of the same type at once.)
Users are encouraged to have many small library files, rather than a few large
files, as this makes many of the library commands much faster. The command
\indexcommand{MOVE-LIBOBJECT} can be used to break up a large file into several smaller files,
if need be.
It is a (fairly) general principle that if \indexflag{SHOW-ALL-LIBOBJECTS} is set to T,
and more than one object of a given name (and type if the type is specified) is found,
then the user is prompted to choose one.

The command
\indexcommand{LIBFILES} lists all the files referred to in the directories
listed in DEFAULT-LIB-DIR, the directories listed in BACKUP-LIB-DIR, all the
directories listed in DEFAULT-LIB-DIR or BACKUP-LIB-DIR, or a single directory
chosen by the user.

The \indexcommand{REFORMAT} command reads in a whole file and writes it out again;
this is useful if you have manually edited it and it's become a bit messy.
The \indexcommand{SORT} command puts a file into alphabetical order.

Finally, the command \indexfile{SPRING-CLEAN} will do any or all of: delete non-library
files in your library directory, reindex all library files in that directory, reformat
all library files in that directory, and sort all library files in that directory.

\section{Printed Output}
The commands \indexcommand{SCRIBE-LIBFILE} and \indexcommand{TEX-LIBFILE}
print the contents of a library file (or a list of library files)
in a form suitable for Scribe or TeX.
The commands prompt for the required degree of verbosity; the various options
for this are described in the help messages for the commands.

\section{Expert Users}
Expert users (i.e. those with \indexflag{EXPERTFLAG} set to T)
are allowed to use library gwffs and
abbreviations in proofs, by using the \indexcommand{ASSERT}
command; they may also instantiate gwffs and
abbreviations while in the editor.

\section{Keywords}\label{KEYWORDS}
Library objects may have associated keywords.  Some keywords
are automatically generated when the objects is created.
Examples of such keywords are PROVEN, UNPROVEN,
WITH-EQUALITY, and WITHOUT-EQUALITY.  The user can create a
new keyword using \indexcommand{ADD-KEYWORD}.  The command
\indexcommand{SHOW-KEYWORDS}
shows a list of acceptable keywords
with help messages.
The keywords
associated with an object can be changed using
\indexcommand{CHANGE-KEYWORDS}.
The command \indexcommand{UPDATE-KEYWORDS}
updates the keywords field to include all of those
keywords that can be determined automatically (leaving
all other keywords untouched).

The keywords associated with an object are printed by
the commands \indexcommand{SHOW} and \indexcommand{SHOW-WFF\&HELP}.
Keywords of objects are also printed by the commands
\indexcommand{TEXLIBFILE}, \indexcommand{SCRIBELIBFILE},
\indexcommand{TEX-ALL-WFFS}, and \indexcommand{SCRIBE-ALL-WFFS},
as long as the verbosity setting is MED or MAX.  The commands
\indexcommand{TEX-ALL-WFFS} and \indexcommand{SCRIBE-ALL-WFFS}
also use keywords
as a filter so the user to select certain classes of gwffs.
The user can also use keywords to find certain objects
using the commands \indexcommand{SEARCH} and \indexcommand{SEARCH2}.

A file containing keywords and help messages for keywords is
stored in each library directory.  The name of this file is
determined by the flag \indexflag{LIB-KEYWORD-FILE}.

\section{Classification Schemes}

A `classification scheme' for the library is a directed acyclic
graph of `classes'.  The graph has a root class with the same
name as the \indexother{classification scheme}.  Each class other than the
root class has a primary parent.  There may also be other parents
of the class.  Each class may have child classes.  Each item in
the library can be associated with multiple classes.

More than one class can have the same name.  Any particular
path can only be uniquely identified by a full path from
the root.  Similarly, the name of a library item does not
uniquely identify the item.  It is, in fact, common practice
for different files in the library to contain separate copies
of theorems and abbreviations.  Usually the definitions of
the items are the same, but the `other-remarks' property
often differs.  Several library items with the same name
can be classified in the same class.  The user is asked to
disambiguate when necessary.

The value of the flag \indexflag{CLASS-SCHEME} is the name of
the current classification scheme.  Classification schemes
can be stored in and retrieved from the library in the usual
way (via the library commands \indexcommand{INSERT} and \indexcommand{FETCH})
as items of type CLASS-SCHEME.
To find out what classification schemes are in the library,
use the command \indexcommand{LIST-OF-LIBOBJECTS} with type CLASS-SCHEME
as an argument.
The top-level command \indexcommand{PSCHEMES} lists the classification
schemes currently in memory.

Once \indexflag{CLASS-SCHEME} is set, there is always a
`current class'.  One can change the current class using
the library commands \indexflag{GOTO-CLASS} and \indexflag{ROOT-CLASS}.

In the library top-level, the following commands can be used
to modify and use classification schemes.

\begin{itemize}
\item {\bf CREATE-CLASS-SCHEME}  Creates a new classification scheme
which can be the value of \indexcommand{CLASS-SCHEME}
and can be stored and fetched from the library.

\item {\bf CREATE-LIBCLASS}  Creates a new class in the classification scheme.

\item {\bf ROOT-CLASS}  Makes the root class the current class.

\item {\bf CLASSIFY-CLASS}  Classify one class as the child of another.

\item {\bf UNCLASSIFY-CLASS}  Remove a class as a child of another.

\item {\bf CLASSIFY-ITEM}  Classify a library item in a class.

\item {\bf UNCLASSIFY-ITEM}  Remove a library item from a class.

\item {\bf FETCH-LIBCLASS}  Fetches all the library items in a class.

\item {\bf FETCH-DOWN}  Fetches all the library items in a class
as well as those classified in descendent classes.

\item {\bf FETCH-UP}  Fetches all the library items in a class
as well as those classified in ancestor classes.

\item {\bf FETCH-LIBCLASS*}  This behaves as FETCH-UP or FETCH-DOWN
depending on the value of the flag \indexflag{CLASS-DIRECTION}.

\item {\bf PCLASS}  Prints information (parents, children,
and classified library items) about the current class.

\item {\bf PCLASS-SCHEME-TREE}  Print the classification scheme as a tree
starting from the root.

\item {\bf PCLASS-TREE}  Print the classification scheme as a tree starting
from the current class.
\end{itemize}

\section{The Unix-style Library Top Level}

The command \indexcommand{UNIXLIB} can be used to enter a top level
that uses a Unix-style interface to access the TPS library.
The classification scheme named by the value of the flag \indexflag{CLASS-SCHEME}
is used to create a virtual directory structure.  Many of the commands one
can use at this top level correspond to Unix commands:

\begin{itemize}
\item {\bf ls}  Lists the child classes (as subdirectories) of the current class.

\item {\bf cd}  Changes the current class.

\item {\bf pwd}  Prints the full path to the current class.
This full path is shown in the prompt if the flag \indexflag{UNIXLIB-SHOWPATH}
is set to T.

\item {\bf mkdir}  Creates a new class as a child of the current class.

\item {\bf ln}  Links a class to be a child of another class.  This is a
command that enables the user to create a class with several parents.

\item {\bf cp}  Copies library items classified in one class to be classified
under another class as well.  The \indexcommand{cp} command can be
used in two ways.  If the user specifies an item to copy, that
particular item is copied.  If the user specifies a class to copy from,
all the items classified in that class are copied.

\item {\bf rm}  Removes a child class or library item from the current class.

\end{itemize}

To specify a class or item, one can use the Unix-style path notation,
for example, /a/b, b/c, ../a/b, etc.

Some library commands (e.g., \indexcommand{fetch}, \indexcommand{show})
are also commands in the UNIXLIB top-level.  These commands
in the UNIXLIB top-level use the current class to determine where
to look for library items.

\section{Cautions}
The user should note a distinction between a library object and a {\TPS}
object. A {\TPS} object is represented in a form most congenial to internal
manipulation, while a library object is represented in a form that
is easy to specify, and these two forms may not be the same. The library
operation \indexcommand{FETCH} takes a library object, and makes it
available within {\TPS}{by converting it to the appropriate internal form.}

Once an object has been loaded from the library and turned into a {\TPS} object,
the \indexcommand{FETCH} command will query any attempt to load another library object
of the same name. A previously loaded object can be removed from {\TPS} by using the \indexcommand{DESTROY}
command.

\section{How to insert TPTP Problems into the \TPS~Library}

The TPTP Problem Library for Automated Theorem Proving can be found at
http://www.cs.miami.edu/~tptp/.
TPTP Problems can be converted into \TPS~library items thanks to the utility TPTP2X, available with the TPTP Library. You need to have a Prolog interpreter installed prior to using the TPTP2X
utility. SWIPL for instance.\\

Important note: versions prior to SWIPL 5.10.1 may cause some TPTP Problems
not to be processed by TPTP2X. SWIPL 5.8.2 could not handle the following
ones: CSR130\textasciicircum 1, CSR144\textasciicircum 1,
CSR153\textasciicircum 1 and SYN000\textasciicircum 2.

\begin{enumerate}

\item Download the latest version of TPTP Library, including the TPTP2X utility.

\item Once extracted, install TPTP2X, using the tptp2X\_install script (in
Prolog).Example:
\begin{alltt}
whatever/TPTP-v5.0.0/TPTP2X\% ./tptp2X\_install
\end{alltt}

Follow the instructions.

\item In the TPTP2X directory, replace the original {\it format.tps} by the one
avaibable in the {\it utilities} directory. Example:
\begin{alltt}
mv format.tps format.tps.old
mv /afs/andrew.cmu.edu/mcs/math/TPS/utilities/format.tps whatever/TPTP-v5.0.0/TPTP2X/
\end{alltt}

\item Use TPTP2X on THF problems, whose name have the following structure:
DOM***\textasciicircum *.p.
To convert a specific Problem, use the syntax:
\begin{alltt}
./tptp2X -f tps <the problem you need> \# e.g. ALG268\textasciicircum 4, whitout quotes
\end{alltt}
To convert every THF problem, you can use:
\begin{alltt}
./tptp2X -f tps "whatever/TPTP-v5.0.0/Problems/**/*\textasciicircum* .p" \# Note the quotes
\end{alltt}

The converted files will be in {\it whatever/TPTP-v5.0.0/TPTP2X/tps}.

\item Create the needed library (sub)directories to receive the converted
TPTP theorems. You can either use the CREATE-LIB-SUBDIR command, or manually
create the directories, an empty {\it libindex.rec} file and a copied
{\it keywords.rec} file.

\item In your {\it tps3.ini} file, or directly inside of a \TPS~instance, set the flag
\indexflag{AUTO-LIB-DIR} to the destination library directory. For instance, if you want
to convert the ALG Problems of TPTP, you may set the flag to
{\it your-lib-directory/ALG/}.\\
Example:
\begin{alltt}
(set-flag 'auto-lib-dir
"/afs/andrew.cmu.edu/mcs/math/TPS/tpslib/chretien/tptp/ALG/")
\end{alltt}
\item In \TPS , use the library command \indexcommand{INSERT-TPTP*} to automatically convert a
directory of .tps files (converted TPTP problems) into TPS library
items. Example:
\begin{alltt}
insert-tptp* "whatever/TPTP-v5.0.0/TPTP2X/tps/ALG"
\end{alltt}
You can also use the \indexcommand{INSERT-TPTP} command to insert only one problem at a
time. Example:
\begin{alltt}
insert-tptp "whatever/TPTP-v5.0.0/TPTP2X/tps/ALG/ALG268\textasciicircum 4.tps" "ALG2684.lib"
"ALG2684"
\end{alltt}
The last argument provides a suffix added to every abbreviation or theorem, in
order to prevent any conflit with buil-in \TPS~items.

\item The newly created items are usually called {\it con-<suffix>} (for conjecture)
or {\it thm-<suffix>}, the suffix being the last argument of INSERT-TPTP or the
name of the library file, if you used INSERT-TPTP*.
\end{enumerate}
