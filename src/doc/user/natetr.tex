\subsection{Translating from Natural Deduction to Expansion Proofs}\label{natetr}

The flag \indexflag{NAT-ETREE-VERSION} determines which
version of \indexcommand{NAT-ETREE} will be used.
The latest version (as of 2001) is \indexother{CEB}.
The basic steps of the \indexother{CEB} version of
\indexcommand{NAT-ETREE} are as follows:

\begin{description}
\item[] 

{\tt 1} -- The natural deduction
proof is preprocessed to eliminate applications of RuleP, Subst=,
and similar rules in favor of more basic inference rules.

{\tt 2} -- The natural deduction proof is translated into a different
structural form called a \indexother{natree}.  One can perform
this part of the translation in isolation using the command
\indexcommand{PFNAT}.  The current natree can be viewed using
the command \indexcommand{PNTR}.

{\tt 3} -- The natree representation is translated into a \indexother{sequent calculus}
derivation.  The sequent calculus derivations in memory can be listed
using the command \indexcommand{SEQLIST}.  A particular sequent calculus
derivation can be viewed using the commands \indexcommand{PSEQ}
or \indexcommand{PSEQL}.  The flag \indexflag{PSEQ-USE-LABELS} controls
whether formulas are abbreviated by showing a symbol associated with
the formula in a legend.

{\tt 4} -- \indexother{Cut elimination} is performed on the sequent calculus derivation.
This is not guaranteed to succeed or even terminate.  A common case
where cut elimination will fail is when a nontrivial
use of extensionality occurs.

{\tt 5} -- If cut elimination succeeds, the cut-free derivation is translated
to an expansion tree with a complete mating.  The user is given the
option of merging this expansion proof.  Merging is appropriate if
the user intends to translate this expansion proof back into a natural
deduction proof.  If the user is trying to use this expansion proof
to help determine flag settings to find the proof automatically, the
user should not merge the tree. 
\end{description}

The expansion proof can be viewed by entering the \indexcommand{MATE} top level.
Section \ref{searchanalysis} explains how this expansion proof can
be used to suggest flag settings and to trace automatic search.

The Programmers Guide has more information about \indexcommand{NAT-ETREE}.

