\chapter{MExpr's} \label{defmexprs}
\TPS provides its own top-level.  It allows for default arguments
and provides a way of giving arguments (e.g. wffs) in some external
representation which is converted before the "real" function is called.
All this is also available in an interactive mode, where the user is
prompted for arguments after he has been told what the defaults are
and which alternatives are open.  The way all this has been implemented
is through MExpr's, which constitute special functional objects analogous
to Expr's or FExpr's in LISP.  Every \TPS command should be an MExpr
so that the facilities of \TPS' top-level can be utilized. 

\section{Defining MExpr's} \label{mexprargs}
Mexprs are special functional objects that are
recognized by the top level of \tps.
They can be defined with the {\tt defmexpr} macro, which has
a number of optional arguments.
The general format is ({\tt {}} indicate optional arguments)
\begin{verbatim}
(defmexpr {\it name}
	 {(ArgTypes {\it type1} {\it type2} ...)}
	 {(ArgNames {\it name1} {\it name2} ...)}
	 {(ArgHelp {\it help1} {\it help2} ...)}
	 {(DefaultFns  {\it fnspec1} {\it fnspec2} ...)}
         {(EnterFns {\it fnspec1} {\it fnspec2} ...)}
	 {(MainFns {\it fnspec1} {\it fnspec2} ...)}
	 {(CloseFns {\it fnspec1} {\it fnspec2} ...)}
         {(Print-Command {\it boolean})}
         {(Dont-Restore {\it boolean})}
	 {(MHelp "{\it comment}")}
\end{verbatim}

There are actually two other possible entries, {\tt Wffop-Typelist} and {\tt WffArgTypes}; these
are only used in mexprs which are generated automatically by the Rules package.

In the following a {\it function specification} is either a symbol naming
a function, or an expression of the form {\tt (Lambda {\it arglist} . {\it body})}.
We also assume that the main function which is to perform the command has
{\it n} arguments.  Then the phrases in the above definition have the
following meaning.
\begin{description}
\item [{\it name}\index{{\it name}}]
 This is the name of the MExpr as called by the user.

\item [{\tt ArgTypes}] This is a list which must have as many elements as the
function arguments, i.e. {\it n}.  {\it type1}, {\it type2}, ..., {\it typen} have
to be valid types, which means that they
have to have a non-{\tt NIL} {\tt ArgType} property.  Each argument supplied by the
user on the command line will be processed first by the corresponding
{\tt GetFn}.  In case an FExpr is to be called, each element of the
argument list is presupposed to be of the same type.  This type is
specified in parentheses.  If {\tt ArgTypes} is omitted, the function has
no arguments.

\item [{\tt ArgHelp}] This has to be a list of length {\it n}.  Each element is
a string describing the argument, or {\tt NIL}.  These quick helps
for arguments can be accessed via the {\tt ?} when being prompted
for the argument value.  For an FExpr, there should be only one string.

\item [{\tt DefaultFns}]
The {\it fnspecs} declared in this place are being processed in a
left-to-right order, where the result of one {\it fnspec} is passed on
to next.  A {\it fnspec} can signal an error (a {\tt THROW} with a
{\tt FAIL} label) if the arguments seem to be contradictory (e.g.  if a
planned line and a term is supplied for a {\tt P}-rule, but the term does
not appear in the proof), but it can count on the arguments being of
the correct type and in internal representation. 

In detail, each default {\it fnspec} must be either a symbol denoting
a function of {\it n} arguments, where {\it n} is the number of mexpr
arguments, or else a lambda expression of {\it n} arguments.
Each {\it fnspec} must return a list of
length {\it n}.  This list will then be handed on and processed by the next
{\it fnspec} as if it were the list of arguments supplied by the user.
Any entry which is not a {\tt \$} should be left unchanged.  The function
is not allowed to have side-effects.
As a general convention, the arguments which are not used by
a {\it fnspec} are not written out with their name, but replaced by
{\tt \%}{\it i}.  This makes it easier to see at one glance which defaults
are filled in by a certain {\it defaultspec}. 

\item [{\tt EnterFns}] {\it fnspec1}, {\it fnspec2}, ... is an arbitrary list of function
specifications.  They are called in succession with the value list
returned by the last default {\it fnspec}, before the {\tt MainFns} are called.

\item [{\tt MainFns}] {\it fnspec1}, {\it fnspec2}, ... is an arbitrary list of function
specifications.  They are called in succession with the value list
returned by the last default {\it fnspec}.  If none are specified, it is assumed
that there is a function named {\it name}, which can be called.
Notice that at this stage, no defaulted arguments may be left.
{\tt ComDeCode} (the command processing function) will refuse to call
any function, unless all the defaults are determined.  This clearly
divides the responsibilities between {\tt GetFn}'s, {\tt DefaultFn}'s and
{\tt MainFn}'s. Any {\it fnspec} may abort with an error by doing a
{\tt THROW} with a {\tt FAIL} label.  A {\tt THROW} with a {\tt TryNext} label
will be handled like a normal return.  A {\tt THROW} with a {\tt CutShort} label
means that none of the remaining {\tt MainFn}'s will be executed and the 
value of the {\tt THROW} will be handed on to the {\tt CloseFn}'s.

\item [{\tt CloseFns}] {\it fnspec1}, {\it fnspec2}, ... is a list of function
specifications.  They are called in succession with the value returned
by the last {\tt MainFn}.  Even if the {\tt MainFn}'s were FExpr's, each
{\it fnspec} has to describe an Expr.

\item [{\tt Dont-Restore}] {\it boolean} determines whether or not this command will be
restored, if it is saved using SAVE-WORK. For example, commands like HELP and ? should not
be restored.

\item [{\tt Print-Command}] {\it boolean} is used by RESTORE-WORK and EXECUTE-FILE, which 
both ask "Execute Print-Commands?"; this is how they know which commands are print 
commands.

\item [{\tt MHelp}] This has to be a string and will be available through
{\tt UserHelp} and the {\tt ??} if no {\tt QuickHelp} is available.

\end{description}

\section{Argument Types}
At the top-level of \TPS
explicitly declared argument types are available.  Many of the more important ones are
all declared in the file {\tt argtyp.lisp}.  They can be recognized by their
{\tt ArgType} property value, which is {\tt T}.  Each of argument type
has at least three properties, {\tt GetFn}, {\tt TestFn}, and {\tt PrintFn}.
{\tt GetFn} is responsible for translating the user's value
into internal representation, {\tt TestFn} tests if some object is of the given
type, and {\tt PrintFn} makes the internal
representation intelligible to the user.  

The defining command for the category {\tt argtype} is actually {\tt deftype\%\%}, but 
all definitions of argtypes should be made through the secondary macro {\tt DefType\%}.
Its format is as follows
({\tt {}} enclose optional arguments):
\begin{verbatim}
(DefType% {\it name}
	 (GetFn {\it fnspec})
	 (TestFn {\it fnspec})
	 (PrintFn {\it fnspec})
	{(Short-Prompt {\it boolean})}
	{(MHelp "{\it comment}")}
	{({\it property1} {\it value1}) ({\it property2} {\it value2}) ...})
\end{verbatim}
In the above a {\it fnspec} is either the name of a one-argument function,
or a list of forms which are to be evaluated as an implicit progn.
In the latter case, {\it name} stands for the argument supplied.
\begin{description}
\item [{\it name}\index{{\it name}}] 
The name of the argument type.  It will get a property value of {\tt T}
for the property {\tt ArgType} when the {\tt DefType\%} has been executed.


\item [{\tt GetFn}] Here {\it fnspec} defines the function used to process the argument
as supplied by the user on the command line.  The value returned by it
is then handed on to the main function executing the command.  No {\tt GetFn}
will ever receive a {\tt \$}.  It is simply not called, if the corresponding
argument in the command line is defaulted.  A {\tt GetFn} should signal an
error if the argument is not
of the correct type.  This will be implemented (as it is right now)
as a {\tt THROW} with the label {\tt FAIL}.  A special case of {\it fnspec}
for a {\tt GetFn} is {\tt TestFn}.  This means the {\tt GetFn} will
test if the supplied argument is of the correct type.  If yes, the argument
will simply be returned, otherwise an error will be signaled.
A {\tt GetFn} may have side-effects, but this has to be declared
under {\tt Side-Effects}.

\item [{\tt PrintFn}] Here {\it fnspec} should print the external representation
of its only argument.  It can expect this argument to be of the correct
type.  The value returned is ignored.  A {\tt PrintFn} may signal an error
if printing is not possible (e.g. if the current style does not
have a representation of the given data type).

\item [{\tt TestFn}] Here {\it fnspec} should return {\tt NIL} if its argument is not of
type {\it name}, and return something not {\tt NIL} otherwise.

\item [{\tt Short-Prompt}] {\it boolean} is only used in {\it otl-typ.lisp}, but I can't 
work out what for.

% \begin{comment}
% This was removed...
% {\tt Side-Effects}\\ {\it value1}, {\it value2}, ... constitute a list of identifiers
% which cause the {\tt GetFn} to have side-effects.  {\w {\tt (Side-Effects T)}}
% means that any argument to {\tt GetFn} will lead to side-effects.  This is
% necessary for instance in the {\tt RWff} argument type, since {\tt RDC}, {\tt RD} ,or
% {\tt PAD} have side-effects and thus should not be called more than once.
% It is useful however, to call a {\tt GetFn} on the same argument more than
% once to figure out defaults as well as possible.
% \end{comment}
\item [{\tt MHelp}] This is an optional documentation and is accessed during {\tt MHelp}
or after a {\tt ?} while the user supplies command arguments interactively.

\item [({\it property} {\it value})] Pairs like this allow for more information about the type.
\footnote{More properties may become useful, so
the {\tt Deftype\%} macro allows arbitrary property names.  Possibilities here
include {\tt EdFn} (for editing this argument type) or {\tt OutputFn} (to be able to
read back a data object of the specified type.)}
\end{description}
For example
\begin{verbatim}
(deftype% anything
  (getfn (lambda (anything) anything))
  (testfn (lambda (anything) (declare (ignore anything)) t))
  (printfn princ)
  (mhelp "Any legal LISP object."))

(deftype% integer+
  (getfn testfn)
  (testfn (and (integerp integer+) (> integer+ -1)))
  (printfn princ)
  (mhelp "A nonnegative integer."))

(deftype% boolean
  (getfn (cond (boolean t) (t nil)))
  (testfn (or (eq boolean t) (eq boolean nil)))
  (printfn (if boolean (princ t) (princ nil)))
  (mhelp "A Boolean value (NIL for false, T for true)."))
\end{verbatim}

No {\tt TestFn} or {\tt PrintFn} is allowed to have any
side-effects, since they may be called arbitrarily often.  No {\tt GetFn}
needs to expect
{\tt \$} as an argument, since defaults are now figured out elsewhere.
This avoids conflicts between different defaults for the same argument
type in different functions.  Hence {\tt GetFn} never computes the
default. 

\subsection{List Types}

The macro {\tt deflisttype} defines a list from an existing type:

%\begin{lispcode}
\begin{verbatim}
(deflisttype filespeclist filespec)
\end{verbatim}
%\end{lispcode}

This takes an existing type, {\tt filespec}, and produces a type of lists of filespecs.
It is also possible to specify other properties (the same properties as for {\tt deftype\%}),
in which case these properties override those of the original type. This is typically 
used to give the list type a different help message from the original type.

\subsection{Consed Types}

The macro {\tt defconstype} defines a type as a cons of two existing types:

%\begin{lispcode}
\begin{verbatim}
(defconstype subst-pair gvar gwff
  (mhelp "Means substitute gwff for gvar."))
\end{verbatim}
%\end{lispcode}

This takes two existing types, {\tt gvar} and {\tt gwff}, and produces a type {\tt subst-pair} of 
consed pairs {\tt (gvar . gwff)}.
It is also possible to specify other properties (the same properties as for {\tt deftype\%}),
in which case these properties override those of the original type. This is typically 
used to give the cons type a different help message from the original type.






% \begin<comment>
% Which other properties of argument types are allowed is determined
% by the global variable {\tt ArgTyProps}.
% {\tt ArgTyProps} is very similar to {\tt GrinProps} and serves the same
% purpose.  It contains a list of pairs
% whose first element is the name of a property associated with the
% argument type and whose second element is either {\tt LIST} or {\tt CONS}.
% A value of {\tt LIST} indicates that the property to be stored is just
% a single element, and hence has to be written out using
% @w[{\tt (LIST {\it property} (GET {\it type} {\it property}))}].  {\tt CONS} on the other
% hand means that the property will be a list and thus has to be written
% using
% @w[{\tt (CONS {\it property} (GET {\it type} {\it property}))}].  
% As an example consider the following definition made in the file
% {\tt BASICS}.
% \begin{verbatim}
% (DV ARGTYPROPS
%     ((GETFN . LIST)
%      (TESTFN . LIST)
%      (PRINTFN . LIST)
%      (SIDE-EFFECTS . CONS)
%      (MHELP . LIST)))
% \end{verbatim}
% \end<comment>

% \begin<comment>
% \section{The Tactic mechanism}
% {\it Carl's comments}
% Tactics and tacticals, though inspired by proof methods, are implemented
% in a general way for \tps. Basically, anything which can described as
% a goal-satisfying process using some fixed kind of object or state, can
% form a base for tactics. We define four entities to implement tactics.
% 
% \begin{description}
% \item [Tacticals] These are functions which take formal arguments, either parameters,
% tactic functions, tactic expressions or lists of these, and with additional
% arguments
% indicating the goals, object and bindings for parameters, return the same
% values as tactics. The application of a tactical to its formal arguments
% is a tactic expression. For example, {\tt (or deduct same)} is a
% tactic expression, with {\tt or} as the tactical.
% 
% \item [Base] A base determines how a tactic expression consisting of a non-tactic
% atom is to be interpreted. Besides the recognition function ({\tt isfn})
% and the function to "run" the atom, producing the same output as a tactic,
% there are properties to ensure the correct operation of certain tacticals.
% For example, {\tt fork} needs a process function in order for the forking
% of processes to make sense in a particular base (and to hack the setting
% of global variables). Much of the work of developing tactics for, say,
% rules consists of defining the base correctly. This makes
% a great deal more sense than to change the general procedure to adapt to
% changes in the implementation of rules.
% 
% \item [Tactic functions] These are functions made into objects for tactics.
% This allows them to be documented, as would be necessary for functions
% accessing an expression tree, and limited, as teachers allowing rule
% tactics might wish.
% 
% \item [Tactics] A Tactic is an object with a tactic expression in its {\tt defn}
% property. If uncompiled, the tactic interpreter, {\tt run-tactic-exp},
% will interpret that tactic expression. If compiled, its {\tt compile}
% property should contain a function to achieve the same effect as
% when it is uncompiled. [Compilation has not been implemented.]
% A tactic has arguments, listed in it {\tt mouths} property, which the 
% {\tt feed} tactical may fill using a feed function for the base.
% This provides a more general kind of argument-handling.
% 
% \end{description}
% \end<comment>
