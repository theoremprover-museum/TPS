\chapter{Teaching Records}\label{Teach}

\section{Events in TPS3}

The primary purpose of events in \TPS is to collect information about
the usage of the system.  That includes support of features such as
automatic grading of exercises and keeping statistics on the application of
inference rules.

Events, once defined and initialized, can be signalled from anywhere in
\tps.  Settings of flags, ordinarily collected into modes, control if,
when, and where signalled events are recorded. Siganlling of events can 
be suppressed by changing the values of the flags in subject \indexother{EVENTS}.
Notice that this, of course, only suppresses the signalling, not the events
themselves!

In \ETPS, a basic set of events is predefined, and the events are signalled
automatically whenever appropriate.  If these events are then recorded
depends on your \ETPS profile.

There are some restrictions on events that should be respected, if
you plan to use {\tt REPORT} to extract statistics from the files recording
events.  Most importantly: {\bf No two events should be written to the
same file.}  If you would like to record different things into the
same file, make one event with one template and allow several kinds of
occurrences of the event.  For an example, see the event {\tt PROOF-ACTION}
below.

\subsection{Defining an Event}

If you are using \ETPS, it is unlikely that you need to define an event
yourself.  However, a lot of general information about events is given
in the following description.

Events are defined using the {\tt DEFEVENT} macro.
Its format is

\begin{verbatim}
(defevent <name>
  (event-args <arg1> ... <argn>)
  (template <list>)
  (template-names <list>)
  (signal-hook <hook-function>)
  (write-when <write-when>)
  (write-file <file-parameter>)
  (write-hook <hook-function>)
  (mhelp <help-string>))
\end{verbatim}

\begin{description}

\item [{\tt event-args}]  list of arguments passed on by {\tt SIGNAL-EVENT} for any event
	of this kind.

\item [{\tt template}]  constructs the list to be written.  
        It is not assumed that every event is
	time-stamped or has the user-id.  The template
        must only contain globally evaluable forms and the arguments
	of the particular event signalled.  It could be the source of
        subtle bugs, if some variables are not declared special.

\item [{\tt template-names}]  names for the individual entries in the template.
These names are used by the {\tt REPORT} facility.  As general conventions,
when the template form is a variable, use the same name for the
template name (e.g. {\tt DPROOF}).  If the template form is {\tt (STATUS {\it statusfn})}
use {\it statusfn} as the template name (e.g. {\tt DATE} for {\tt (STATUS DATE)} or
{\tt USERID} for {\tt (STATUS USERID)}).

\item [{\tt signal-hook}]  an optional function to be called whenever the
	the event is signalled.  This should {\bf not} to the writing of
	the information, but may be used to do something else.  If the
        function does a {\tt THROWFAIL}, the calling {\tt SIGNAL-EVENT} will
        return {\tt NIL}, which means failure of the event.  The arguments
        of the function should be the same as {\tt EVENT-ARGS}.

\item [{\tt write-when}]  one of {\tt IMMEDIATE}, {\tt NEVER}, or an integer {\it n}, which means
     to write after an implementation depended period of {\it n}.
     At the moment this will write, whenever the number of inputs = {\it n}
     * {\tt EVENT-CYCLE}, where {\tt EVENT-CYCLE} is a global variable, say 5.

\item [{\tt write-file}]  the name of the global {\tt FLAG} with the filename of the
     file for the message to be appended to.

\item [{\tt write-hook}]  an optional function to be called whenever a number
	(>0) of events are written.  Its first argument is the file it will
        write to, if the write-hook returns.  Its second argument is the
        list of evaluated templates to be written.  If an event is to be
        written immediately, this will always be a list of length 1.

\item [{\tt mhelp}]  The mhelp string for the event.
\end{description}

Remember that an event is ignored, until {\tt (INIT-EVENTS)} or {\tt (INIT-EVENT
{\it event})} has been called.

\subsection{Signalling Events}

\TPS provides a function {\tt SIGNAL-EVENT}, which takes a variable number
of arguments.  The first argument is the kind of event to be signalled,
the rest of the arguments are the event-args for this particular event.
{\tt SIGNAL-EVENT} will return {\tt T} or {\tt NIL}, depending on whether the action
to be taken in case of the event was successful or not.  Note that when
an event is disabled (see below), signalling the event will always be
successful.  There are basically three cases in which an event will be
considered unsuccessful: if the {\tt SIGNAL-HOOK} is specified and does a
{\tt THROWFAIL}, if {\tt WRITE-WHEN} is {\tt IMMEDIATE} and either the {\tt WRITE-HOOK}
(if specify) does a {\tt THROWFAIL}, or if for some reason the writing to
the file fails (if the file does not exists, or is not accessible
because it has the wrong protection, for example).

It is the caller's responsibility to make use of the returned value of
{\tt SIGNAL-EVENT}.  For example, the signalling of {\tt DONE-EXERCISE} below.

If {\tt WRITE-WHEN} is a number, the evaluated templates will be collected
into a list {\it event{\tt -LIST}}.  This list is periodically written out and
cleared.  The interval is determined by {\tt EVENT-CYCLE}, a global flag
(see description of {\tt WRITE-WHEN} above).  The list is also written out
when the function {\tt EXIT} is called, but not if the user exits \TPS with
{\tt $\hat{}$C}.  Note that if events have been signalled, the writing is done
without considering whether the event is disabled or not.  This ensures
that events signalled are always recorded, except for the {\tt $\hat{}$C} safety valve.

Events may be disabled, which means that signalling them will always
be successful, but will not lead to a recordable entry.  This is done
by setting or binding the flag {\it event{\tt -ENABLED}} to {\tt NIL} (initially
set to {\tt T}).  For example, the line {\tt (setq error-enabled nil)} 
in your {\tt .INI} file will make sure that no MacLisp error will be recorded.
For a maintainer using expert mode, this is probably a good idea.

\subsection{Examples}

Here are some examples take from the file {\tt ETPS-EVENTS}.  Interspersed
is also the code from the places where the events are signalled.

%\begin{tpsexample}
\begin{verbatim}

(defflag error-file
  (flagtype filespec)
  (default "etps3.error")
  (subjects events)
  (mhelp "The file recording the events of errors."))

(defevent error
  (event-args error-args)
  (template ((status-userid) error-args))
  (template-names (userid error-args))
  (write-when immediate)
  (write-file error-file)    ; a global variable, eg
			     ; `((tpsrec: *) etps error)
  (signal-hook count-errors) ; count errors to avoid infinite loops
  (mhelp "The event of a Lisp Error."))
\end{verbatim}
{\tt DT} is used to freeze the daytime upon invocation of {\tt DONE-EXC} so that
the code is computed correctly.  The code is computed by {\tt CODE-LIST},
implementing some ``trap-door function''.

\begin{verbatim}
(defvar computed-code 0)

(defvar dt '(0 0 0)) 

(defvar score-file)
(defflag score-file
  (flagtype filespec)
  (default "etps3.scores")
  (subjects events)
  (mhelp "The file recording completed exercises."))

(defevent done-exc
  (event-args numberoflines)
  (template ((status-userid) dproof numberoflines computed-code
			     (status-date) dt))
  (template-names (userid dproof numberoflines computed-code date daytime))
  (signal-hook done-exc-hook)
  (write-when immediate)
  (write-file score-file)
  (mhelp "The event of completing an exercise."))

(defun done-exc-hook (numberoflines)
  ;; The done-exc-hook will compute the code written to the file.
  ;; Freeze the time of day right now.
  (declare (special numberoflines))
  ;; because of the (eval `(list ..)) below.
  (setq dt (status-daytime))
  (setq computed-code 0)
  (setq computed-code (code-list (eval `(list ,@(get 'done-exc 'template))))))

(defflag proof-file
  (flagtype filespec)
  (default "etps3.proof")
  (subjects events)
  (mhelp "The file recording started and completed proofs."))

(defevent proof-action
  (event-args kind)
  (template ((status-userid) kind dproof (status-date) (status-daytime)))
  (template-names (userid kind dproof date daytime))
  (write-when immediate)
  (write-file proof-file)
  (mhelp "The event of completing any proof."))

(defflag advice-file
  (flagtype filespec)
  (default "etps3.advice")
  (subjects events)
  (mhelp "The file recording advice."))

(defevent advice-asked
  (event-args hint-p)
  (template ((status-userid) dproof hint-p))
  (template-names (userid dproof hint-p))
  (write-when 1)
  (write-file advice-file)
  (mhelp "Event of user asking for advice."))

\end{verbatim}
%\end{tpsexample}

Here is how the {\tt DONE-EXC} and {\tt PROOF-ACTION} are used in the code of
the {\tt DONE} command.  We don't care if the {\tt PROOF-ACTION} was successful
(it will usually be), but it's very important that the user knows
when a {\tt DONE-EXC} was unsuccessful, since it is used for automatic
grading.

%\begin{tpsexample}
\begin{verbatim}
(defun done ()
  ...
  (if (funcall (get 'exercise 'testfn) dproof)
      (do ()
	  ((signal-event 'done-exc (length (get dproof 'lines)))
	   (msgf "Score file updated."))
	(msgf "Could not write score file.  Trying again ... (abort with ^G)")
	(sleep 1/2))
      (msgf "You have completed the proof.  Since this is not an assigned exercise,"
	    t "the score file will not be updated."))
  (signal-event 'proof-action 'done))
\end{verbatim}
%\end{tpsexample}

\section{The Report Package}

The \indexother{REPORT} package in \TPS allows the processing of data
from EVENTS. Each report draws on a single event, reading
its data from the record-file of that event. The execution
of a report begins with its BEGIN-FN being run. Then 
the DO-FN is called repetitively on the value of the EVENTARGS
in each record from the record-file of the event, until that
file is exhausted or the special variable DO-STOP is given a non-NIL
value. Finally, the END-FN is called. The arguments
for the report command are given to the BEGIN-FN and END-FN.
The DO-FN can only access these values if they are assigned to
certain PASSED-ARGS, in the BEGIN-FN. Also, all updated values
which need to be used by later iterations of the DO-FN or by
the END-FN should be PASSED-ARGS initialized (if the default NIL
is not acceptable in the BEGIN-FN.

NOTE: The names of PASSED-ARGS should be different from
other arguments (ARGNAMES and EVENTARGS). Also, they should
be different from other variables in those functions where
you use them and from the variables which DEFREPORT always 
introduces into the function for the report: FILE, INP and DO-STOP.

The definition of the category of REPORTCMD, follows:

%\begin{LispCode}
\begin{verbatim}
(defcategory reportcmd
  (define defreport1)
  (properties 
   (source-event single)
   (eventargs multiple)   ;; selected variables in the var-template of event
   (argnames multiple)
   (argtypes multiple)
   (arghelp multiple)
   (passed-args multiple) ;; values needed by DO-FN (init in BEGIN-FN)
   (defaultfns multiplefns)
   (begin-fn singlefn)    ;; args = argnames    
   (do-fn singlefn)       ;; args = eventargs ;; special = passed-args
   (end-fn singlefn)      ;; args = argnames
   (mhelp single))
  (global-list global-reportlist)
  (mhelp-line "report")
  (mhelp-fn princ-mhelp)
  (cat-help "A task to be done by REPORT."))

\end{verbatim}
%\end{LispCode}

	The creation of a new report consists of a DEFREPORT statement
(\indexother{DEFREPORT} is a macro that invokes \indexother{DEFREPORT1})
and the definition of the BEGIN-FN, DO-FN and END-FN. Any PASSED-ARGS
used in these functions should be declared special. It is suggested
that most of the computation be done by general functions which are more
readily usable by other reports. In keeping with this philosophy,
the report EXER-TABLE uses the general function MAKE-TABLE. The latter
takes three arguments as input:  a list of column-indices, a list of
indexed entries (row-index, column-index, entry) and the maximum printing size
of row-indices. With these, it produces a table of the entries.
EXER-TABLE merely calls this on data it extracts from the record file
for the DONE-EXC event. The definition for EXER-TABLE follows:

%\begin{LispCode}
\begin{verbatim}
(defreport exer-table
  (source-event done-exc)
  (eventargs userid dproof numberoflines date)
  (argtypes date)
  (argnames since)
  (defaultfns (lambda (since)
		(cond ((eq since '$) (setq since since-default)))
		(list since-default)))
  (passed-args since1 bin exerlis maxnam)
  (begin-fn exertable-beg)
  (do-fn exertable-do)
  (end-fn exertable-end)
  (mhelp "Constructs table of student performance."))

(defun exertable-beg (since)
  (declare (special since1 maxnam))	;the only non-Nil passed-args
  (setq since1 since)
  (setq maxnam 1))

(defun exertable-do (userid dproof numberoflines date)
  (declare (special since1 bin exerlis maxnam))
  (if (greatdate date since1)
      (progn
       (setq bin (cons (list userid dproof numberoflines) bin))
       (setq exerlis 
	     (if (member dproof exerlis) exerlis (cons dproof exerlis)))    
       (setq maxnam (max (flatc userid) maxnam)))))

(defun exertable-end (since)
  (declare (special bin exerlis maxnam))
  (if bin 
      (progn
       (make-table exerlis bin maxnam)
       (msg t "On exercises completed since ")
       (write-date since)
       (msg "." t))
      (progn
       (msg t "No exercises completed since ")
       (write-date since)
       (msg "." t))))

\end{verbatim}
%\end{LispCode}

