\chapter{Notes on setting things up}\label{set-up}
\section{Compiling TPS and ETPS}

{\ETPS} is simply a subsystem of {\TPS}.  {\ETPS} lacks some
of the files used to build {\TPS}.
The procedure for building {\ETPS} is just like that for
building {\TPS}, except that one should type {\tt make etps}
rather than {\tt make tps}.  You can build both {\TPS} and {\ETPS}
in the same directory, but you must
make sure the bin directory is empty before building
each system because the compiled files for
{\TPS} are not compatible with those for {\ETPS}.
For simplicity in the discussion below,
we often refer to {\TPS} when it would be more precise
to refer to ``{\TPS} or {\ETPS}".

{\TPS} has been compiled in several versions of Common Lisp: Allegro
Common Lisp (version 3.1 or higher); Lucid Common Lisp; CMU Common
Lisp; Kyoto Common Lisp; Austin Kyoto Common Lisp; Ibuki Common Lisp,
a commercial version of Kyoto Common Lisp; and DEC-20 Common Lisp.
% \comment{Note: We do not recommend the use of DEC-20 Common Lisp, as
% there are many compiler bugs in it.}
  Several source files contain
compiler directives which are used to switch between the various
definitions required.

For the time being, we assume you are compiling on a Unix
or MS Windows operating system
using one of the versions of Lisp given above.
Otherwise, considerably more work will be required.  Some additional information
which may be helpful will be found as comments in the text file
{\it whatever/tps/doc/user/manual.mss} for this section (Compiling TPS) of
this User Manual.  These comments are not printed out when Scribe
processes the file.

\subsection{Compiling TPS under Unix}\label{set-up-unix}

To compile and build tps, proceed as follows:
\begin{enumerate}

\item Create a bin directory, if one does not exist.
The bin directory should be empty when you start this process.
If you have previously built tps or etps, start by removing
all files from the bin directory ({\tt rm -f bin/*}).

\item Read and follow the directions which are presented as comments in the
{\tt Makefile}. In general, this will just mean
changing the {\tt Makefile}
to show the correct pathname for your
version of Lisp (and possibly java), changing the \indexfile{Makefile} to show where remarks by
TPS users should be sent and which users are allowed privileges.

\item Issue the command `{\tt make tps}' or `{\tt make etps}'.
(Of course, this assumes you are using a Unix operating system.)
The Makefile will also try to compile the files for the Java interface.
If you do not have java compiler, this will fail, but only after
all the lisp files have been compiled.
A Java compiler is not necessary to install TPS.  The installation
will still create TPS and ETPS, but you will not be able to use the
Java interface.
Note that if you do not have a Java compiler, you can download
Java SDK (with a compiler) from http://java.sun.com/.

\item The script file tps-build-install-linux can be used to
build and install tps.  Look at it.

\item If you are using KCL or IBCL, you may get an error during compiling
which says something like `unable to allocate'.  This error indicates
that your C compiler cannot handle the size of the file that is
being compiled.  To fix this, split the offending file (e.g. {\it foo.lisp})
into smaller pieces (e.g., {\it foo1.lisp} and {\it foo2.lisp}) and replace
the occurrence of `foo' in the file \indexfile{defpck.lisp} with `foo1 foo2'.
If this doesn't work you may have to split the files again.

\item If you are using Allegro Common Lisp 5.0, the name of the core image should
end in .dxl; for example, {\tt tps3.dxl}. To achieve this, you can set tps-core-name
in the Makefile (in which case the new core image may overwrite the old one if
you rebuild), or just use the Unix {\tt mv} command to rename the core image
once it is built.

\item When {\TPS} starts up, it loads a file called \indexfile{tps3.patch} if one is
there; this contains fixes for bugs, new code which has been added
since {\TPS} was last built, etc. 
%[[This doesn't really work, so we delete this example]]
%For example, if you wish to change expert-list after tps has been built, just put the appropriate line
%into \indexfile{tps3.patch}, using the format of the example in \indexfile{tps3-save.lisp}.
After you build a new {\TPS}, you may wish to delete (or save in a
different file) the contents of the old \indexfile{tps3.patch} file. Keeping the
empty file there assures that it will be in the right place when you
need it again.

\item After loading the patch file, {\TPS} will look for a file called \indexfile{tps3.ini}
in the same directory as the patch file and (if the user is an expert)
for a file also called \indexfile{tps3.ini}
in the directory from which the user starts {\TPS}.  (These directories may or may not be the same).
Before using {\TPS}, you may want to change these initialization files.
\end{enumerate}

% \begin{comment}
% If you don't use the Makefile, you may have to do these things in addition
% to those mentioned above:
% 
% Reinstate the definition of {\tt expert-list} which is currently commented
% out of the file \indexfile{tps3-save.lisp} (modifying it as appropriate; it should contain
% a list of the user names of all those users who are to be allowed expert privileges
% while using {\TPS}).
% 
% Make any changes required to the file {\tt tps3.sys}.  This is a file which
% is loaded during compilation and each time {\TPS} is run which
% sets the values of certain system-specific variables like directories,
% file names, etc.  For {\ETPS}, change the file {\tt etps.sys}.
% 
% {\TPS} and {\ETPS} have a command called {\tt REMARK}, which allows the user
% to enter a string to be sent to the teacher/maintainer.
% If you have a Unix system and are running CMU Common Lisp, Lucid (or Sun)
% Common Lisp, KCL or IBCL, you can use email to send the remark.  To do
% this, set the variable {\tt MAIL-REMARKS} in the file {\tt \{TPS3,ETPS\}.SYS} to
% a string containing the mail addresses of those who should receive the
% remark.  If you do not desire this, set {\tt MAIL-REMARKS} to NIL and
% set {\tt REMARKS-FILE} to the name of the file that should store the
% remarks instead.
% 
% Examine the file {\tt tps-compile}.  This is a Unix script file which will
% automatically load and compile all the source files.
% Make any changes required.  For {\ETPS}, change the file {\tt etps-compile}.
% 
% At your system top-level, just type {\tt tps-compile}.  Compilation should
% take anywhere from 1.5 hours (CMU Common Lisp) to 5 hours (Kyoto Common Lisp),
% if all goes well.
% 
% Then type {\tt tps-build} to create a core image.
% 
% If you have most of the files already compiled and just want to build
% a new system, use the files {\tt tps-build} and {\tt tps-build.lisp}, or
% for {\ETPS}, {\tt etps-build} and {\tt etps-build.lisp}. {\tt tps-build}
% checks the dates on source files and their compiled versions, and
% compiles any files which have been changed since their compiled
% versions were created, or which were not already compiled. However, it
% may not recompile files containing macros which have been changed, or
% catch other basic changes, so when major changes have occurred, it is
% best to use {\tt tps-compile} (which recompiles all files)
% before using {\tt tps-build}.
% \end{comment}
If your lisp is not in the list above, you may need to change some of the
system-dependent functions. The features used by TPS are `tops-20', `lucid',
`:cmu', `kcl', `allegro' and `ibcl'.  System-dependent files include

\begin{description}
\item[{\tt SPECIAL.EXP}]	 Contains symbols which cannot be exported in some lisps. These
are found by trial and error.

\item[{\tt BOOT0.LISP, BOOT1.LISP}]	 Contain  some  lisp and operating-system dependent
functions and macros, like file manipulation.

\item[{\tt TOPS20.LISP}]	 Redefining the lisp top-level, saving a core image, exiting, etc.

\item[{\tt TPS3-SAVE.LISP}]	 Some I/O functions which should work for Unix lisps, also the
original definition of {\tt expert-list}.

\item[{\tt TPS3-ERROR.LISP}]	 Redefinitions of trapped error functions, as used in {\ETPS}.
\end{description}

\subsection{Compiling TPS under MS Windows}\label{set-up-windows}

When compiling {\TPS} under MS Windows,
there is a lisp file \indexfile{make-tps-windows.lisp}
which can be used instead of the Makefile.  The file
\indexfile{make-tps-windows.lisp} was designed to work
for Allegro Lisp version 5.0 or later, though some parts of
it should work for other versions of lisp.

To compile and build {\TPS} under MS Windows, perform the following steps:

\begin{enumerate}
\item Create a folder for TPS, e.g., open `My Computer',
%then `C:', then `Program Files', and choose `New > Folder'
then `C:', then `Program Files', and choose `New $>$ Folder'
from the file menu.  Then rename the created folder to
%TPS.  Now you have a folder C:$\setminus$Program Files$\setminus$TPS$\setminus$.
TPS.  Now you have a folder C:$\setminus$Program Files$\setminus$TPS$\setminus$.

\item Download the gzip'd tps tar file and unzip it (using,
for example, NetZip or WinZip) into the C:$\setminus$Program Files$\setminus$TPS$\setminus$
folder.

\item Create a C:$\setminus$Program Files$\setminus$TPS$\setminus$bin folder, if one does not exist.
Also, create a top level folder C:$\setminus$dev.  This is so TPS can
send output to $\setminus$dev$\setminus$null.

\item Determine if you have a Java compiler.  You can do this by running
`Find' or `Search' (on `Files and Folders') (probably available
through the `Start' menu) and searching for a file named `javac.exe'.
A Java compiler is not necessary to install TPS.  The installation
will still create TPS and ETPS, but you will not be able to use the
Java interface.  Note that if you do not have a Java compiler, you can
download Java SDK (with a compiler) from http://java.sun.com/.

\item Lisp (preferably Allegro 5.0 or greater) will probably be in
`Programs' under the `Start' menu.  Start Lisp (by choosing it from
there) and do the following:
\begin{alltt}
(load `C:\(\setminus\setminus\)Program Files\(\setminus\setminus\)TPS\(\setminus\setminus\)make-tps-windows.lisp')
\end{alltt}
This should prompt you for information used to compile and build TPS,
as well as compiling the Java files (if you have a Java compiler).  It
will also create executable batch files, e.g., C:\(\setminus\)Program Files\(\setminus\)TPS\(\setminus\)tps3.bat
which you can use to start {\TPS} after it has been built.

\item After Lisp says `FINISHED', enter {\tt (exit}).
\end{enumerate}

If for some reason \indexfile{make-tps-windows.lisp} fails to compile and build
{\TPS}{and {\ETPS}}, you can look at \indexfile{make-tps-windows.lisp} to try to figure
out how to build it by hand.  The remaining steps are an outline
of what is needed.

\begin{enumerate}
\item If \indexfile{make-tps-windows.lisp} did not create the files
\indexfile{tps3.sys} and \indexfile{etps.sys},
rename tps3.sys.windows.example to tps3.sys,
and rename etps.sys.windows.example to etps.sys.
You may want to edit the value of the constant
expert-list to include your user name.  In Windows, this is often
`ABSOLUTE', which is already included on the list.
If the {\TPS} directory is something other than
`C:\(\setminus\)Program Files\(\setminus\)TPS\(\setminus\)', then you will need to edit tps3.sys
and etps.sys by replacing each
`C:\(\setminus\setminus\)Program Files\(\setminus\setminus\)TPS\(\setminus\setminus\)' with `whatever\(\setminus\setminus\)'.
Also, you will need to edit the
files \indexfile{tps-compile-windows.lisp}, \indexfile{tps-build-windows.lisp}
(and \indexfile{etps-compile-windows.lisp} and \indexfile{etps-build-windows.lisp}
if you intend to use etps) by replacing the line
\begin{alltt}
  (setq tps-dir `C:\(\setminus\setminus\)Program Files\(\setminus\setminus\)TPS\(\setminus\setminus\)'))
\end{alltt}
by
\begin{alltt}
  (setq tps-dir `whatever\(\setminus\setminus\)'))
\end{alltt}

\item Make sure the bin directory is empty.
If you have previously built tps or etps, start by
sending all files from the bin directory to the Recycle Bin.

\item Run Lisp.  Load the tps-compile-windows.lisp
file from C:\(\setminus\)Program Files\(\setminus\)TPS\(\setminus\) as follows:
\begin{alltt}
(load `C:\(\setminus\setminus\)Program Files\(\setminus\setminus\)TPS\(\setminus\setminus\)tps-compile-windows.lisp')
\end{alltt}
This will compile the lisp source files in the C:\(\setminus\)Program Files\(\setminus\)TPS\(\setminus\)lisp
folder into the C:\(\setminus\)Program Files\(\setminus\)TPS\(\setminus\)bin folder.

\item Exit and restart lisp.  Load the tps-build-windows.lisp
%file from C:\Program Files\TPS\ as follows:
file from C:\(\setminus\setminus\)Program Files\(\setminus\setminus\)TPS\(\setminus\setminus\) as follows:
\begin{alltt}
(load `C:\(\setminus\setminus\)Program Files\(\setminus\setminus\)TPS\(\setminus\setminus\)tps-build-windows.lisp')
\end{alltt}
If you try to load tps-build-windows.lisp after loading
tps-compile-windows.lisp without restarting Lisp, you will probably
get an error because packages are being redefined.  So, it
is important to exit and start a new Lisp session before
loading tps-build-windows.lisp.
The end of tps-build-windows.lisp calls tps3-save, which
saves the image file.  Under Allegro, this should be tps3.dxl.
(The name and location of the image file is determined by
the values of sys-dir and save-file in tps3.sys.)

\item Repeat the previous steps using etps-compile-windows.lisp and etps-build-windows.lisp
to compile and build {\ETPS}.

\item If you have a Java compiler, use it to compile the java files in
C:\(\setminus\)Program Files\(\setminus\)TPS\(\setminus\)java\(\setminus\)tps
and then
C:\(\setminus\)Program Files\(\setminus\)TPS\(\setminus\)java\(\setminus\)
(see section \ref{compiling-java})

\item If \indexfile{make-tps-windows.lisp} did not create the batch files
tps3.bat and etps.bat,
create the batch file tps3.bat containing something like
\begin{alltt}
jecho off
call `C:\(\setminus\)<lisppath>\(\setminus\)alisp.exe' -I `C:\(\setminus\)Program Files\(\setminus\)TPS\(\setminus\)tps3.dxl'
\end{alltt}
and the batch file etps.bat containing something like
\begin{alltt}
jecho off
call `C:\(\setminus\)<lisppath>\(\setminus\)alisp.exe' -I `C:\(\setminus\)Program Files\(\setminus\)TPS\(\setminus\)etps.dxl'
\end{alltt}
You need the quotes because Windows easily gets confused about spaces
in pathnames.  You should be able to double click on tps3.bat to
start {\TPS}.

\item If \indexfile{make-tps-windows.lisp} did not create the batch files
for starting {\TPS} and {\ETPS} with the Java interface, then you
can create files like \indexfile{tps-java.bat} containing something like
\begin{alltt}
jecho off
call `C:\(\setminus\)<lisppath>\(\setminus\)alisp.exe' -I `C:\(\setminus\)Program Files\(\setminus\)TPS\(\setminus\)tps3.dxl' -- -javainterface java -classpath `C:\(\setminus\)Program Files\(\setminus\)TPS\(\setminus\)java' TpsStart
\end{alltt}
(See section \ref{using-java} for more command line options associated with the Java interface.)

\end{enumerate}

Double clicking on the batch files tps3.bat and etps.bat should start
{\TPS} and {\ETPS}, respectively.  Also, if you had \indexfile{make-tps-windows.lisp}
compile the code for
the Java interface and build the batch files for starting
{\TPS} with the Java interface (or you have done this manually),
then there should be several batch
files with names like \indexfile{tps-java.bat} and \indexfile{etps-java-big.bat}.
Executing these should start {\TPS} or {\ETPS} with the Java interface.

An alternative to using batch files to start {\TPS} using Allegro Lisp is as follows:

\begin{enumerate}
\item Put a copy of the Lisp executable (such as lisp.exe or alisp8.exe) into the
C:\(\setminus\)Program Files\(\setminus\)TPS\(\setminus\) folder, and rename it tps3.exe.
(You may only need to explicitly change `lisp' to `tps3'
in order to rename lisp.exe to tps3.exe.)

\item Copy acl*.epll or acl*.pll (or similarly named files)
from the Allegro Lisp directory to the TPS directory.
You may also need to copy a license file *.lic
from the Allegro Lisp directory to the TPS directory.

\item Double-click on tps3.exe to start up TPS.  This will automatically
find tps3.dxl as the image (since it is in the same directory and has the
same root name).  If Allegro complains that some file isn't found,
look for that file under the Allegro Lisp directory and copy it to
the TPS directory.

\end{enumerate}

\subsection{Compiling the Java Interface}\label{compiling-java}

There is a Java interface for {\TPS} supporting menus and pop-up
windows.
To use this interface, {\TPS} must be able to use sockets
and multiprocessing.
Currently it seems that these features are both
implemented only in
Allegro Lisp (version 5.0 or later).

To compile the java code
under Unix,
simply cd to the directory `whatever/tps/java/tps' and call
\begin{alltt}
javac *.java
\end{alltt}
This should create a collection of .class files
in the java/tps directory.  Then cd to `whatever/tps/java'
and call
\begin{alltt}
javac *.java
\end{alltt}
This should create a collection of .class files
in the java directory.

Compiling the java code under Windows is a bit
more complicated.  There is a Lisp file
\indexfile{make-tps-windows.lisp} provided
with the distribution which should be able
to compile the Java files if you load
\indexfile{make-tps-windows.lisp} in
Allegro Lisp.  (See section \ref{set-up-windows}.)

If you must compile the java code under Windows
manually, the following hints may help.
If the version of Windows allows
the user to bring up a DOS shell, you should be
able to chdir to `whatever\(\setminus\)TPS\(\setminus\)java\(\setminus\)tps' and call
\begin{alltt}
javac *.java
\end{alltt}
Then do the same under `whatever\(\setminus\)TPS\(\setminus\)java'.
Otherwise, you might be able to create a batch file
temp.bat containing the following code:
\begin{alltt}
chdir whatever\(\setminus\)TPS\(\setminus\)java\(\setminus\)tps
javac *.java
chdir whatever\(\setminus\)TPS\(\setminus\)java\(\setminus\)
javac *.java
\end{alltt}
Then you can double click on the icon for the batch file
to get Windows to execute it.
If Windows cannot find the executable `javac',
then you can either write the full path (`C:\(\setminus\setminus\)whatever\(\setminus\setminus\)javac')
or include the appropriate directory in the PATH environment variable.
In Windows XP, the PATH environment
variable can be changed by opening the Control Panel, then System,
then choosing Advanced and Environment Variables.

\section{Initialization}\label{Initialization}

\subsection{Initializing {\TPS}}

There can be one \indexfile{tps3.ini} file in the
directory where tps is built which will be loaded for all
users, and each \indexother{expert} user can have an individual \indexfile{tps3.ini} file in
the directory from which he calls {\TPS}.
For nonexperts, the common \indexfile{tps3.ini} file will be loaded quietly
(without any indication this is being done).

For TeX files generated by {\TPS} to work correctly,
you should set your Unix environment variable TEXINPUTS appropriately, so that TeX
can find the \indexfile{tps.sty} file.  The \indexfile{tps.sty} file
can be found in the {\it whatever/doc/lib/} directory.

After loading this common \indexfile{tps3.ini} file, {\TPS} then looks for an individual's
\indexfile{tps3.ini} file in the directory from which the individual starts {\TPS}.
This should be used by an individual
user, for tailoring the system to the needs of a particular person (or a
particular computer). For example, user1's \indexfile{tps3.ini} file might contain
appropriate settings for garbage collection flags in several variants of Lisp, as well
as a preferred default for DEFAULT-MS, and so on.
Also, user1 might wish to have in his \indexfile{tps3.ini} file the line
\begin{alltt}
(set-flag 'default-lib-dir '(`/whatever/tps/library/user1/'))
\end{alltt}
to specify his library directory (see section \ref{library} for more details).

Also in the \indexfile{tps3.ini} file, you can define aliases. For example, it may be useful to
have several different settings for the \indexflag{TEST-THEOREMS} flag used by \indexcommand{TPS-TEST},
and you can define aliases to switch between them as follows:

\begin{tpsexample}
(alias test-long `(set-flag 'test-theorems '((user::thm1  user::mode1) ..etc..))')

(alias test-default `(set-flag 'test-theorems '((user::thm2  user::mode2) ..etc..))')

(alias test-short `(set-flag 'test-theorems '((user::thm1  user::mode1) ..etc..))')
\end{tpsexample}

The last line of your \indexfile{tps3.ini} file should be {\tt (set-flag 'last-mode-name `')}, so that
the flag \indexflag{LAST-MODE-NAME} will start off empty. Also, you should set the value of
\indexflag{RECORDFLAGS} to include \indexflag{LAST-MODE-NAME}, so that \indexcommand{DATEREC}
will properly record what mode you were using at the time.

The flags \indexflag{INIT-DIALOGUE} and \indexflag{INIT-DIALOGUE-FN} should be mentioned
here; if the former is T, then after loading the two {\it .ini} files, {\TPS} will call the
function named by INIT-DIALOGUE-FN. My \indexfile{tps3.ini} file sets these flags to T and
INIT-DEFINE-MY-DEFAULT-MODE respectively, so that on startup I have a new mode
MY-DEFAULT-MODE which contains my default settings of all the flags. See the help messages
of these flags for more information.

\subsection{Initializing {\ETPS}}

There is a common \indexfile{etps.ini} file which is loaded when a user starts {\ETPS}.
This can be especially useful if students will be using {\ETPS} for a class.
The \indexfile{etps.ini} file can be used to limit what students can
do while using {\ETPS}.

One thing you may wish to do is to prevent students from being able to
access Lisp directly.
First, the flag \indexflag{EXPERTFLAG}, which, if false, does not allow the user to
enter arbitrary forms for evaluation.
For this purpose, the flag \indexflag{EXPERTFLAG} should be set to NIL
in the \indexfile{etps.ini} file.

A list of \indexother{expert}s containing
the user id's of persons allowed to change the expertflag to true (e.g.,
maintainers) is given in the \indexfile{Makefile}.  You should change this list
before building {\ETPS}.

The second way to keep students out of {\TPS} internals is to trap all
errors, and prevent students from entering the break loop.
There is a command
\begin{alltt}
(setq *trap-errors* t)
\end{alltt}
in the distributed \indexfile{etps.ini} file which
does this for Allegro Lisp.
The file \indexfile{tps3-error.lisp} has this
set up properly for DEC-20, Kyoto Common Lisp and Ibuki Common Lisp,
but you may have to do some work on this if you are using some other lisp.
Basically, the idea is that if the debugger is called, an immediate
throw back to the top level is performed.


\section{Starting {\TPS}}

Look at the \indexfile{aliases-dist} file and the run-* script files
for examples of how to start tps.

In some lisps, tps will be an executable file which can be
executed directly.

If you are using CMULISP, instead of the above use the command
\begin{alltt}
cmulisp -core tps \&
\end{alltt}
where cmulisp is the name by which you call CMULISP.

If you are using Allegro Common Lisp 5.0 or greater, you can use the command
\begin{alltt}
lisp -I tps3.dxl \&
\end{alltt}
where tps3.dxl is the file that was created when {\TPS} was built.

If you are using a version of Allegro Common Lisp prior to 5.0,
then an executable file should have been created by the Makefile.
You can simply call this executable to start {\TPS}.  For example,
\begin{alltt}
tps3 \&
\end{alltt}

There are several command line switches that control different
options for starting {\TPS}.  For more information about these
options, in {\TPS} one can execute HELP \indexother{COMMAND-LINE-SWITCHES}.

\section{Using {\TPS} with the X window system}\label{X}

{\TPS} can be run under the X window system (X10R4, X11R3 or X11R4),
with nice output including mathematical symbols, by doing the
following.

\begin{enumerate}
\item For X10R4: Make sure that
the font directory {\tt fonts}
is in your {\tt XFONTPATH}.

\item For X11R3 or X11R4: Add the fonts directory to your font path by a
\begin{alltt}
{\tt xset +fp whatever/tps/fonts}
\end{alltt}
The {\tt +fp} adds the font to the start
of your font path, so the {\TPS} fonts will override any other fonts of
the same name in your font path. You may wish to put this {\tt xset}
command in the .Xclients or .xinitrc file in your home directory,
or add this command to the `Startup Programs' on your computer.
\end{enumerate}

Then start
{\TPS} by
\begin{alltt}
{\tt \%xterm -fn vtsingle -fb vtsymbold -e tps}
\end{alltt}
where {\tt tps} is the complete name of the
executable file, and, of course, you can add fancy
things like geometry, side-bar, etc.
If you are using CMULISP, instead of the above use the command
\begin{alltt}
{\tt \%xterm -fn vtsingle -fb vtsymbold -e cmulisp -core tps \&}
\end{alltt}
where cmulisp is the name by which you call CMULISP.

If you are using Allegro Common Lisp 5.0, you can use the command
\begin{alltt}
{\tt \%xterm -geometry 80x48+4+16 '\#+963+651' -fn vtsingle -fb vtsymbold  -n Tps3jCOMPUTERNAME -T Tps3jCOMPUTERNAME -sb -e lisp -I tps3.dxl \&}
\end{alltt}
where tps3.dxl is the executable file.
(Here COMPUTERNAME is the name of the computer on which you are running;
this feature is optional, of course.)


  Demonstrations are easier to see if you use the X10 fonts gallant.r.19.onx
and galsymbold.onx, which are included with this distribution, in place of
vtsingle and vtsymbold.  These fonts are very large.

Thus, to start up tps using Allegro Common Lisp 5.0  in an X window
with large fonts, you can use the command
\begin{alltt}
{\tt \%xterm -geometry 82x33+0+0 '\#+963+651' -fn gallant.r.19 -fb galsymbold  -n Tps3jCOMPUTERNAME -T Tps3jCOMPUTERNAME -sb -e lisp -I tps3.dxl \&}
\end{alltt}

The fonts vtsingle.bdf, vtsymbold.bdf, gallant.r.19.bdf and galsymbold.bdf
are provided for use with X11.

When {\TPS} starts, switch to style
{\tt XTERM} as follows:

\begin{alltt}
{\tt <0>style xterm}
\end{alltt}

Also, if you see blinking text instead of special symbols,
then try changing the value of the flag \indexflag{XTERM-ANSI-BOLD}
to 49 as follows:\footnote{In 2005 (and previously), a value of 53
worked for XTERM-ANSI-BOLD at
CMU while using xterm version XFree86 4.2.0(165), and a value of 49
worked at Saarbrucken while using xterm version X.Org 6.7.0(192).}

\begin{alltt}
{\tt <0>xterm-ansi-bold 49}
\end{alltt}

If the TPS fonts are not being displayed properly on your screen, the
reason might be that many recent Linux systems are using a UTF-8 
\index{UTF-8} locale,
while the TPS fonts seem to work only in the traditional POSIX
\index{POSIX}
locale. To get the standard POSIX behavior, unset the environment
variable LC\_ALL.\index{LC\_ALL} This can be accomplished by executing the Linux command
\begin{alltt}
{\tt unset LC\_ALL}
\end{alltt}
or
\begin{alltt}
{\tt setenv LC\_ALL C.}
\end{alltt}
If LC\_ALL is unset, all the other LC\_* environment variables are ignored.

If you resize the X window, you should change the setting of the flag
{\tt RIGHTMARGIN}.

\section{Using {\TPS} with the \indexother{Java Interface}}\label{using-java}

There is a Java interface for {\TPS} supporting menus and pop-up
windows.
To use this interface, {\TPS} must be able to use sockets
and multiprocessing.
Currently it seems that these features are both
implemented only in
Allegro Lisp (version 5.0 or later).
In order for the Java windows to work, the
TCP IP driver on your computer must be activated.  Therefore, if the
Java interface does not work on your computer, you may be able to
remedy the problem by starting up internet connections.

The Java code for the interface is distributed under
the `java' and `java/tps' directories.  The code
in the `java' directory is used only to start the java
{\TPS} interface.  The actual code for running the interface
is in a `tps' package under `java/tps'.

To start TPS with the Java interface, you must supply
appropriate command line arguments.  For example, under Unix,
\begin{alltt}
lisp -I whatever/tps/tps3.dxl
     -- -javainterface cd whatever/tps/java \(\setminus\); java TpsStart
\end{alltt}
The command line argument `-javainterface' tells TPS that it should
run with the Java interface.  The command line arguments that follow
should form a shell command which cd's to the directory where the Java
code is, then calls java on the TpsStart class file.  (Note that the
shell command separator `;' needs to be quoted to `\(\setminus\);'.)

You may also
want to redirect the \TPS output to /dev/null, i.e., call
\begin{alltt}
lisp -I whatever/tps/tps3.dxl
     -- -javainterface cd whatever/tps/java \(\setminus\); java TpsStart > /dev/null
\end{alltt}
since the output the user needs to see shows up in the Java window.
Furthermore, if you want to continue to use the shell from which you
started {\TPS}, use \& to start run it in the background:
\begin{alltt}
lisp -I whatever/tps/tps3.dxl
     -- -javainterface cd whatever/tps/java \(\setminus\); java TpsStart > /dev/null \&
\end{alltt}

There are other command line arguments which can be sent to TpsStart.
These must be preceeded by the command line argument -other so
that {\TPS} can distinguish these from the shell command used to
start the java interface.

Some command line arguments control the size of fonts.
For example,
\begin{alltt}
lisp -I whatever/tps/tps3.dxl
     -- -javainterface cd whatever/tps/java \(\setminus\); java TpsStart -other -big > /dev/null \&
\end{alltt}
tells Java to use the bigger sized fonts.  The command line argument
-x2 tells Java to multiply the font size by two.
The command line argument -x4 tells Java to multiply the font size by four.

The command line argument `-nopopups' will make the Java interface
act more like the X window interface.  First, there will be a Text Field
at the bottom of the Java window used to enter commands.  Second, the
user is prompted for input using this TextField instead of prompting
the user via a popup window.  For example,
\begin{alltt}
lisp -I whatever/tps/tps3.dxl
     -- -javainterface cd whatever/tps/java \(\setminus\); java TpsStart -other -nopopups > /dev/null \&
\end{alltt}
Note that to enter commands into the TextField, the user may need
to focus on the TextField by clicking on it.

Finally, there are command line arguments `-maxChars', `-rightOffset',
`-bottomOffset', `-screenx', and `-screeny'.  Each of these should be
immediately followed by a non-negative integer.  For example,
\begin{alltt}
lisp -I whatever/tps/tps3.dxl
     -- -javainterface cd whatever/tps/java \(\setminus\); java TpsStart
     -other -maxChars 50000 -rightOffset 20 -bottomOffset 30 -screenx 900 -screeny 500 > /dev/null \&
\end{alltt}
starts the Java interface with a buffer size big enough to hold 50000 characters,
20 pixels of extra room at the right of the output window, 30 pixels of extra
room at the bottom of the output window, and an initial window size of 900 by 500 pixels.
These command line arguments are useful since the optimal default values may vary with
different machines and different operating systems.

Another way to run {\TPS} using the java interface is to start
{\TPS} without the `-javainterface' command line, then
use the {\TPS} command \indexcommand{JAVAWIN} to start the Java interface.
Once the command \indexcommand{JAVAWIN} is executed, all interaction
with this {\TPS} must be conducted via the Java interface.
For \indexcommand{JAVAWIN} to work, the flag \indexflag{JAVA-COMM}
must be set appropriately in the file {\tt tps3.sys} (or {\tt etps.sys}
for {\ETPS}) before the image file for {\TPS}{or {\ETPS}} is built.
For example, in Unix, {\tt tps3.sys} should contain a line like
\begin{alltt}
(defvar java-comm `cd whatever/tps/java ; java TpsStart')
\end{alltt}
In Windows, {\tt tps3.sys} should contain a line like
\begin{alltt}
(defvar java-comm `java -classpath C:\\whatever\\TPS\\java TpsStart')
\end{alltt}

Note: Resizing the main Java window for {\TPS} will automatically
adjust the value of the flag \indexflag{RIGHTMARGIN}.

\section{Using {\TPS} within Gnu Emacs}

The following will produce output within Emacs, which may be useful as an
alternative to creating a script file.

First start up Emacs, then type {\tt M-x shell} (where M-x is meta-x, or more
likely escape-x), then {\tt tps3} (or whatever alias you have defined to start
up {\TPS}). It is advisable to make your first commands {\tt style generic} and
{\tt save-flags-and-work {\it filename}}, so that the output will be readable
and a work file will be written.

Then you can use {\TPS} as normal (except that where you would normally type
{\tt \^G <Return>} to abort a command, you must now type {\tt \^Q\^G<return>}).

At the end of your session, you can rename the buffer with {\tt M-x rename-buffer}
and save it with {\tt \^X\^S}. Then type {\tt exit} twice: once to leave {\TPS} and once
to leave the shell.

Conversely, depending on the particular local configuration of your
version of Lisp, you may be able to run Emacs from within {\TPS}, using
the \indexmexpr{LEDIT} command.


\section{Running {\TPS} in Batch Mode or from \indexother{Omega}}

There are two methods of batch processing in {\TPS}: work files and UNIFORM-SEARCH. Both of them are invoked
by command line switches when {\TPS} is first started.

When in batch mode, {\TPS} will write to a file {\tt tps-batch-jobs} in your home directory
to confirm that the job has begun, and again to confirm that it has ended.

The point of these switches is that they can be used to run {\TPS} with the Unix commands {\tt at}, {\tt batch} and {\tt cron},
without requiring interaction from the user.

Note: it is possible that your Lisp handles switches on the command line differently. For example, Allegro Lisp uses {\tt --}
to separate switches for Lisp itself and switches for the core image, so in Allegro all of the examples below should begin
{\tt tps -- } rather than {\tt tps }.

\subsection{Batch Processing Work Files}

{\TPS} has a command line switch \indexother{-batch} which allows the user to run a work file.
Assuming that {\tt tps} is the command which starts {\TPS} on your system, and that you have a work
file {\tt foo.work} in your home directory, the command
\begin{tpsexample}
tps -batch foo
\end{tpsexample}
is equivalent to starting {\TPS}, typing {\tt execute-file foo}, and then exiting {\TPS} when the work file finishes.

To redirect the output of this process to a file {\tt bar}, use
\begin{tpsexample}
tps -batch foo -outfile bar
\end{tpsexample}
This will redirect the lisp streams {\tt *standard-output*}, {\tt *debug-io*}, {\tt *terminal-io*} and {\tt *error-output*}
to the file {\tt bar}. To redirect absolutely everything, use the Unix redirection commands > and >> instead. The file
{\tt /dev/null/} is a valid output file.

Examples:
\begin{tpsexample}
tps -batch thm266
{\it runs thm266.work through tps3, showing the output on the terminal.}
tps -batch thm266 -outfile thm266.script
{\it does the same but directs the output to thm266.script.}
tps -batch thm266 -outfile /dev/null
{\it does the same but discards the output.}
\end{tpsexample}

\subsection{Interactive/Omega Batch Processing}

The \indexother{-omega} switch allows the user to start TPS, run a work file, interact with TPS and then save the
resulting proof on exiting TPS. As the name suggests, this switch is used by the Omega system
(see \cite{Benzmuller97} and \cite{Benzmuller98b}).
When this switch is present, the -outfile switch is used to name the resulting proof file.
The saved proof will be the current version of whichever proof was active at the end of the work file.
If -outfile is omitted, TPS will use `<proofname>-result.prf' as the filename.
If there is no dproof created by the workfile, the saved proof will be
whichever proof is current when the user types EXIT, and it will be named
`tps-omega.prf'

Example:
\begin{tpsexample}
tps -omega -batch thm266 -outfile thm266
{\it starts TPS, runs thm266.work and leaves the TPS window open for the user to interact;
when the user exits, TPS will write a file thm266.prf}
\end{tpsexample}

\subsection{Batch Processing With UNIFORM-SEARCH}

The command line switch \indexother{-problem} tells {\TPS} to run \indexcommand{UNIFORM-SEARCH} on the given problem (which
must be the name of a gwff either in the library or internal to {\TPS}). The user can also specify the mode and
searchlist to be used, with the \indexother{-mode} and \indexother{-slist} switches. If either of these is omitted,
the mode UNIFORM-SEARCH-MODE and the searchlist UNIFORM-SEARCH-2 will be used by default.

The switch {\tt -outfile} may be used to redirect output, as in the example above. Other output may be generated by
specifying the \indexother{-record} switch, which takes no arguments, and which forces {\TPS} to call DATEREC and SAVEPROOF
after the proof is finished. {\tt -record} will also insert into the library the mode which is generated by
UNIFORM-SEARCH.

Examples:
\begin{tpsexample}
tps -problem x2138
{\it Search for a proof of X2138 using the default mode and searchlist; send output to the standard output}
tps -problem x2138 -mode mode-x2138 -record -outfile x2138.script
{\it Search as above, but use mode-x2138 to set all the flags that are not set by the default searchlist.
Send the output to x2138.script, and when the search is finished call daterec, save x2138.prf and insert
the new mode x2138-test-mode into the library}
tps -problem difficult-problem -mode my-mode -slist my-slist -record >> /dev/null/
{\it Search for a proof of difficult-problem, fixing all of the flags in my-mode and varying the flags in my-slist
as specified by that searchlist. If a proof is found, record it, the time taken, and the successful mode, but throw
away all other output.}
\end{tpsexample}

\section{Calling {\TPS} from Other Programs}

The command line switches
\indexother{-service} and \indexother{-lservice}
start {\TPS} in a way that accepts requests to prove theorems automatically for
other programs.  Both command line switches connect with other programs
via sockets used to communicate requests and responses.
Descriptions of programming with sockets can be found on the web.
Performing a search for `sockets' via Google (for example) yields millions of results.
We will assume some familiarity with communication via sockets here.

With respect to {\TPS} started with \indexother{-service} and \indexother{-lservice}
there are two relevant sockets: {\tt inout} and {\tt serv}.
The socket {\tt serv} is intended to connect {\TPS} with \indexother{MathWeb}
(see the website {\tt http://www.mathweb.org/})
but is practically unused by {\TPS} at the moment.
All the information described here
is communicated via the {\tt inout} socket.
The only difference between the command line switches
\indexother{-service} and \indexother{-lservice}
involve how these sockets are initialized.

For the purposes of this description, we refer
to the program calling {\TPS} as the `client'.
We assume the client is running on a machine called {\tt clienthost}.

\subsection{Establishing Connections}

Assume the client has two passive sockets {\tt clientio} and {\tt clientserv}
at port numbers {\tt clientioport} and {\tt clientservport}, respectively.
Start {\TPS} with \indexother{-service} as follows:
\begin{alltt}
tps -service <clienthost> <clientioport> <clientservport>
\end{alltt}
{\TPS} will connect to the client via the given hostname and port numbers
establishing the {\TPS} sockets {\tt inout} and {\tt serv}.

To use the command line switch \indexother{-lservice}
we must assume the client and {\TPS} are running on the same machine.
(The `l' in \indexother{-lservice} stands for `local'.)
Assume the client has a passive socket
with port number {\tt clientport1}.  On the same machine start {\TPS} with
\indexother{-lservice} as follows:
\begin{alltt}
tps -lservice <clientport1>
\end{alltt}
After initialization {\TPS} will open two new ports {\tt inout} and {\tt serv}
and send the port numbers to the client via {\tt clientport1}.
These port number are sent as a character string of the form
`{\tt (inoutport servport})'.
The client should take these values and connect to the sockets {\tt inout} and {\tt serv}.
At this point the communication between the client and {\TPS} works the same
way as with \indexother{-service}.

\subsection{Socket Communication}

The client and {\TPS} is communicate messages via the established sockets
using a sequence of bytes (ASCII characters).  The special byte {\tt 128}
(character {\tt \%null}) indicates the end of a message.

\subsection{Ping-Pong Protocol}

Before sending requests to {\TPS} the client must first
follow a ping-pong protocol.  The client sends a message
{\tt (PING <clientname>}) via the {\tt inout} socket.
{\TPS} should respond with
{\tt (<clientname> PONG <TPSname>}).
The client reads this and begins sending requests to
{\TPS} using the identifier {\tt <TPSname>}.

\subsection{Requests}

Every request is of the form
{\tt (<TPSname> <request> . . .})
and is sent to {\TPS} via the {\tt inout} socket.
The requests the client can send to {\TPS} are as follows:
\begin{description}
\item[{\tt DIY}]	 Try to prove a theorem.

\item[{\tt BASIC-DIY}]	 Try to prove a theorem without using special rules (like RULEP).

\item[{\tt KILL}]	 Kill a {\TPS} process.

\item[{\tt ADJUSTTIME}]	  Adjust the time remaining for a {\TPS} process.
\end{description}

The format for {\tt DIY} and {\tt BASIC-DIY} requests are as follows:
\begin{alltt}
(<TPSname> [DIY|BASIC-DIY] <procname> <servcomms> <proofoutline> <TPSmode> <DEFAULT-MS> <timeout>)
\end{alltt}
The name {\tt <procname>} is a string the client chooses to identify this particular request.
Assume {\tt <servcomms>} is NIL (in general it could be a list of commands to send to MathWeb).
The {\tt <proofoutline>} is in the form of a `defsavedproof'.  In general you can
get such a form by obtaining the proof outline in {\TPS}, performing \indexcommand{saveproof}
into a file {\tt foo.prf} and then examining the file {\tt foo.prf}.
A simple example is
\begin{alltt}
 & (defsavedproof FOTR
 &  (4 2 29)
 &  (assertion `TRUTH')
 &  (nextplan-no 2)
 &  (plans ((100)))
 &  (lines (100 NIL `TRUTH' PLAN1 () NIL))
 &  0 () (comment `') (locked NIL))
\end{alltt}
which represents the proof outline
\begin{alltt}

               ...
(100) \( \assert \truth \) & PLAN1
\end{alltt}
The value of {\tt <TPSmode>} should be the name of a mode in {\TPS}.
The value of {\tt <DEFAULT-MS>} can be used to override the value of
the flag \indexflag{DEFAULT-MS} in the mode {\tt <TPSmode>}.  If {\tt <DEFAULT-MS>}
is NIL, then the value of the flag \indexflag{DEFAULT-MS} is set by {\tt <TPSmode>}.
{\tt <timeout>} is the number of seconds {\TPS} should try to search for a proof
before giving up.

After the client sends a {\tt DIY} or {\tt BASIC-DIY} request with name {\tt <procname>},
the client can later kill the request or allow the request more time to succeed.
The {\tt KILL} request has the format
\begin{alltt}
(<TPSname> KILL <procname>)
\end{alltt}
The {\tt ADJUSTTIME} request has the format
\begin{alltt}
(<TPSname> ADJUSTTIME <procname> <seconds>)
\end{alltt}
This {\tt ADJUSTTIME} request will add the value {\tt <seconds>}
to the time remaining for the process {\tt <procname>}.  (Note that
{\tt <seconds>} may be negative.)

When a {\tt DIY} or {\tt BASIC-DIY} request has finished (either due to success, failure or timeout),
then the message returned via the {\tt inout} socket will be
\begin{alltt}
(<procname> <proof> <printedproof>)
\end{alltt}
where {\tt <proof>} is the proof in the `defsavedproof' format (with no remaining planned lines if the request succeeded)
and {\tt <printedproof>} is the result of the {\TPS} command \indexcommand{PALL}.

\subsection{Example}

Consider a quick example where the client is running under Allegro Lisp
on the host jtps.math.cmu.edu.

Client:
\begin{alltt}
>(setq clientio (acl-socket:make-socket :connect :passive))
\#<MULTIVALENT stream socket waiting for connection at */34032 \ \#x722d43aa>
>(setq clientserv (acl-socket:make-socket :connect :passive))
\#<MULTIVALENT stream socket waiting for connection at */34033 \ \#x722d58ea>
\end{alltt}

Start {\TPS} on jtps.math.cmu.edu:
\begin{alltt}
tps -service jtps.math.cmu.edu 34032 34033
\end{alltt}

Client:
\begin{alltt}
>(setq inout (acl-socket:accept-connection clientio))
\#<MULTIVALENT stream socket connected from jtps.math.cmu.edu/34032 to
  jtps.math.cmu.edu/34034 \ \#x722d820a>
>(setq serv (acl-socket:accept-connection clientserv))
\#<MULTIVALENT stream socket connected from jtps.math.cmu.edu/34033 to
  jtps.math.cmu.edu/34035 \ \#x7231f6ba>
> (defun send-info (s)
    (format inout `\(\sim\)S' s)
    (write-char \#\(\setminus\)%null inout)
    (force-output inout))
SEND-INFO
> (defun read-msg ()
     (let ((ret `'))
       (do ((z (read-char inout nil nil) (read-char inout nil nil)))
           ((eq z \#\(\setminus\)%null) ret)
         (setq ret (format nil `\(\sim\)d\(\sim\)d' ret z)))))
READ-MSG
> (send-info '(PING CLIENTNAME))
NIL
> (setq rets (read-msg))
`(CLIENTNAME PONG |TPSjjtps.math.cmu.edu-3287332390|)
'
> (setq ret (read-from-string rets))
(CLIENTNAME PONG |TPSjjtps.math.cmu.edu-3287332390|)
> (setq tpsname (caddr ret))
|TPSjjtps.math.cmu.edu-3287332390|
> (send-info (list tpsname 'DIY `TRUEPROCNAME' nil '(defsavedproof FOTR
  (4 3 3)
  (assertion `TRUTH')
  (nextplan-no 2)
  (plans ((100)))
  (lines
    (100 NIL `TRUTH' PLAN1 () NIL)
  ) 0
  ( )
    (comment `')
    (locked NIL)
    NIL
  )
  'MS98-HO-MODE NIL 5))
NIL
> (setq rets (read-msg))
``(\"TRUEPROCNAME\(\setminus\)'' \(\setminus\)''(defsavedproof FOTR
  (4 3 3)
  (assertion \(\setminus\setminus\setminus\)''TRUTH\(\setminus\setminus\setminus\)'')
  (nextplan-no 3)
  (plans NIL)
  (lines
    (1 NIL \(\setminus\setminus\setminus\)''TRUTH\(\setminus\setminus\setminus\)'' \(\setminus\setminus\setminus\)''Truth\(\setminus\setminus\setminus\)'' () NIL)
) 0
( )
  (comment \(\setminus\setminus\setminus\)''\(\setminus\setminus\setminus\)'')
  (locked NIL)
  NIL
)
\(\setminus\)'' \(\setminus\)''
(1)   !  TRUTH                                               Truth\(\setminus\)'')
'
> (setq ret (read-from-string rets))
(`TRUEPROCNAME' `(defsavedproof FOTR
  (4 3 3)
  (assertion \(\setminus\)''TRUTH\(\setminus\)'')
  (nextplan-no 3)
  (plans NIL)
  (lines
    (1 NIL \(\setminus\)"TRUTH\" \(\setminus\)"Truth\(\setminus\)" () NIL)
) 0
( )
  (comment \(\setminus\)"\(\setminus\)")
  (locked NIL)
  NIL
)
' `
(1)   !  TRUTH                                               Truth')
> (send-info (list tpsname 'DIY `FALSEPROCNAME' nil '(defsavedproof FOFA
    (4 3 3)
    (assertion `FALSEHOOD')
    (nextplan-no 2)
    (plans ((100)))
    (lines
      (100 NIL `FALSEHOOD' PLAN1 () NIL)
    ) 0
  ( )
    (comment `` ``)
    (locked NIL)
    NIL
  )
  'MS98-HO-MODE NIL 5))
NIL
> (setq rets (read-msg))
``(\(\setminus\)"FALSEPROCNAME\(\setminus\)" \(\setminus\)"(defsavedproof FOFA
  (4 3 3)
  (assertion \(\setminus\setminus\setminus\)''FALSEHOOD\(\setminus\setminus\setminus\)'')
  (nextplan-no 2)
  (plans ((100)))
  (lines
    (100 NIL \(\setminus\setminus\setminus\)''FALSEHOOD\(\setminus\setminus\setminus\)'' PLAN1 () NIL)
) 0
( )
  (comment \(\setminus\setminus\setminus\)''\(\setminus\setminus\setminus\)'')
  (locked NIL)
  NIL
)
\(\setminus\)" \(\setminus\)"
               ...
(100) !  FALSEHOOD                                            PLAN1\(\setminus\)")
"
> (setq ret (read-from-string rets))
("FALSEPROCNAME" "(defsavedproof FOFA
  (4 3 3)
  (assertion \(\setminus\)"FALSEHOOD\(\setminus\)")
  (nextplan-no 2)
  (plans ((100)))
  (lines
    (100 NIL \(\setminus\)"FALSEHOOD\(\setminus\)" PLAN1 () NIL)
) 0
( )
  (comment \(\setminus\)"\(\setminus\)")
  (locked NIL)
  NIL
)
" "
               ...
(100) !  FALSEHOOD                                            PLAN1")
\end{alltt}

\section{Starting {\TPS} as an Online Server}\label{tps-server}\index{server}\index{web server}

Under Allegro Lisp 5.0 or greater,
{\TPS}{and {\ETPS}} can be started as a web server for use online.
Once everything is set up and the server is started,
remote users will be able to access {\TPS}
via a browser (possibly using
a user id and password) and communicate with {\TPS} via
a Java interface.

\subsection{Setting up the Online Server}

The following steps are necessary to set up the {\TPS} server
in a Unix or Linux operating system.  Analogous steps
are necessary for setting up the {\TPS} server for MS Windows.

\begin{enumerate}
\item We start by assuming {\ETPS}, {\TPS} and the Java files
have already been compiled.

\item Create a directory for the server, e.g.
{\tt /home/theorem/tpsonline}
Move to this directory.

\item To set up the user id's and passwords, start {\TPS}
in the new directory.

\item The id and password information will be saved in the file named by
the flag \indexflag{USER-PASSWD-FILE}.  The default value
is \indexother{user-passwd}.  It is recommended that
you not change the value of this flag from its default.

\item From within {\TPS}, run the command \indexcommand{SETUP-ONLINE-ACCESS}.
\begin{alltt}
<1>setup-online-access
\end{alltt}
The command \indexcommand{SETUP-ONLINE-ACCESS} will ask you for
a series of user id's and passwords for remote access
to {\TPS} and {\ETPS}.  It will also ask if you
wish to allow anonymous users to be able to remotely run {\TPS} or {\ETPS}.
The user id's and passwords
are unrelated to the user id's and passwords created and used by the operating
system.  The following is an example where two students are given
id's and passwords for {\ETPS} and anonymous users are allowed to remotely
run {\TPS}.
\begin{alltt}
<0>setup-online-access
Allow ETPS Anonymous Access To Everyone? [No]>no
Add a userid?  [Yes]>

User Id  [`']>`student1'
Password  [`']>`password1'
Added user student1
Add another userid?  [Yes]>

User Id  [`']>`student2'
Password  [`']>`password2'
Added user student2
Add another userid?  [Yes]>n
Allow TPS Anonymous Access To Everyone? [No]>y
Although anyone can run TPS
You may still wish to add specific users which will be allowed to save files
in a directory.
Add a userid?  [Yes]>n
\end{alltt}
A file named \indexfile{user-passwd} (or, in general,
a name given by the value of \indexflag{USER-PASSWD-FILE})
should have been created.

\item Exit {\TPS}.

\item Copy or link the java directory (with the compiled Java class files)
into the tpsonline directory.
\end{enumerate}

\subsection{Starting or Restarting the Online Server}

To start the {\TPS} server, make sure you are in the directory
with the user-password file.
You may be able to start the {\TPS} server by using
the shell script \indexfile{run-tpsserver}, which is included
in the distribution.
In general, start {\TPS}{or {\ETPS}} as a server using the following
pattern:
\begin{alltt}
<lisp> -I <tpsdir>/tps3.dxl -- -server <tpsdir>/tps3.dxl <tpsdir>/etps.dxl
\end{alltt}
<lisp> should be the name of the lisp executable (e.g., lisp or alisp8).
<tpsdir> should be the directory where the {\TPS} and {\ETPS} image files are located.
If the server starts successfully, a directory named `logs' for log files
should be created.  One can explicitly give a different name for the log directory
using the -logdir command line switch as follows:
\begin{alltt}
<lisp> -I <tpsdir>/tps3.dxl -- -server <tpsdir>/tps3.dxl <tpsdir>/etps.dxl -logdir <logdirname>
\end{alltt}

Once the server is started, it can be accessed on the web using the URL
`http://<machine-name>:29090/'.  The number `29090' is the default
port number used by the {\TPS} server.  If this port number is not free,
then the {\TPS} server will fail to start.  In this case, you can
explicitly provide a different port number using the -port command line
switch as follows:
\begin{alltt}
<lisp> -I <tpsdir>/tps3.dxl -- -server <tpsdir>/tps3.dxl <tpsdir>/etps.dxl -port <portnum>
\end{alltt}
In this case, the {\TPS} web server can be accessed via a browser using
the URL `http://<machine-name>:<portnum>/'.  If you wish to link to
the web server from an HTML file on another web site, use an href anchor
as follows:
\begin{alltt}
<A HREF=`http://<matchine-name>:<portnum>/'>Click here to run TPS or ETPS online</A>
\end{alltt}

When a remote user accesses {\TPS} or {\ETPS} online via the {\TPS} web server,
a directory for that user is created (or a directory named `anonymous' if it is being
run without a user id and password).  Files may be saved by the remote user
in this directory.

It should be noted that an anonymous remote user is allowed to do less.
For example, an anonymous user cannot start {\TPS} or {\ETPS} using a command
line prompt.  Instead, they are forced to rely on menus to execute allowed
commands and popup prompts to enter other information.  This prevents
an anonymous user from executing arbitrary commands.

One possible use of the {\TPS} server is to allow students in a class
to use {\ETPS} to complete class assignments without having
{\ETPS} installed on their computer.  This is discussed further in
section \ref{etps-class-server}.

\section{Preparing ETPS for classroom use}

Building {\ETPS} is just like building {\TPS}, except that one should type {\tt make etps}
rather than {\tt make tps}.  Before calling {\tt make etps}
make sure the bin directory is empty because the compiled files for
tps won't work right for etps.
The modules of {\ETPS} are just a subset of those for {\TPS}.

If you wish to use {\ETPS} for a class, there are some things you might
want to change.

To determine for which exercises students may use
\indexcommand{ADVICE} and commands such as {\tt RULEP}, set the
\indexother{allowed-cmd-p} attributes appropriately. For example, in
the definitions in \indexfile{ml2-theorems.lisp}, we find the \indexother{allowed-cmd-p}
attributes set to \indexother{ALLOW-ALL}, so for these exercises the
students may use all the facilities of {\ETPS}. On the other hand, in
\indexfile{ml1-theorems.lisp} they are set to \indexother{ALLOW-RULEP}, which
allows everything except advice.

If you desire different inference rules or exercises, see chapter \ref{rules} for
tips on defining and compiling new ones.  Examine the {\it .rules} files
which have been used to define the current rules. Then put your new files into
a new {\TPS} package, and load that package when building or compiling
{\ETPS}.

\subsection{Starting ETPS as an Online Server for a Class}\label{etps-class-server}

Following the instructions in section \ref{tps-server}, a teacher
can start {\TPS} as a web server.  Using the command \indexcommand{SETUP-ONLINE-ACCESS}
as described in section \ref{tps-server}, the teacher can enter
a list of student user id's and passwords.  This allows students to
log in through the {\TPS} web server and start {\ETPS}.  This will create
a directory on the server machine for this student.  Files relevant to
recording scores for this student are saved in this directory.

\subsection{Grades}


When a student completes an
exercise and executes the \indexcommand{DONE} command,
a message recording that fact can be appended to the end of a
file to which students have write access.
The path and name of this file
is given by the value of the flag \indexflag{score-file}.
The flag \indexflag{score-file} should be set in
the common initialization file \indexfile{etps.ini}
which is loaded by every user of {\ETPS}.
The distributed \indexfile{etps.ini} file contains the
following line:
\begin{alltt}
;*; (setq score-file `/afs/andrew/mcs/math/etps/records/etps-spring03.scores')
\end{alltt}
If you want \indexfile{etps.ini} to set \indexflag{score-file},
then you should remove ;*;
from the beginning of this line to uncomment it.
This (if uncommented) will set the flag \indexflag{score-file}
to the same value for every user of {\ETPS}.
Appropriate adjustments in the pathname should be made.

If you prefer to have students with different userid's
to have different score files, you can use
the following option instead.
The distributed \indexfile{etps.ini} file also contains these lines:
\begin{alltt}
;*; (setq score-file (concatenate 'string
;*;     `/afs/andrew/mcs/math/etps/records/' (string (status-userid))
;*;     `/etps-spring03.scores'))
\end{alltt}
If the user's userid is pa01, when this is read {\ETPS}
will set the flag \indexflag{score-file} to
\begin{alltt}
`/afs/andrew/mcs/math/etps/records/pa01/etps-spring04.scores'
\end{alltt}
If you use this option, you will need to use a utility
to combine the score files for the different students in
the class.

The {\tt GRADER} program (for which there is a separate manual)
can be used to process the
grades in a file which is the value of the
{\tt GRADER} flag \indexflag{etps-file}.
This should be the same as the value of
the {\ETPS} flag \indexflag{score-file}
if all the students write to the same file.
Otherwise, it can be a file into which all the
students' files have been collected.
The sample line in the \indexfile{grader.ini}
file setting \indexflag{etps-file} should be edited
by changing the pathname appropriately.

\subsection{Security}

Note: On a Unix system, you can use {\ETPS} as a setuid program to allow
students to write to their score files, i.e., so that any process
running {\ETPS} has full access to the files, while other processes do
not.  However, this may be an excessive precaution, since each message
issued by the {\ETPS} \indexcommand{DONE} command has a special
encryption number used to ensure security.  Thus a student cannot edit
the score file to make it appear that he or she has completed an
exercise.  The routines in the {\tt GRADER} package check the encryption
number to make sure the information in that line of the score file is
valid.

Checksums are generated for all saved proofs
when the \indexflag{EXPERTFLAG} flag is set to NIL, but not when it is set to T.
This means that students should be unable to manually edit a saved partial proof
and fool {\ETPS} into thinking that it's complete; it also means that proofs
saved in {\TPS} with \indexflag{EXPERTFLAG} T cannot be reloaded into {\ETPS}
with \indexflag{EXPERTFLAG} NIL.


\subsection{Diagnosing Problems in ETPS}

{\ETPS} catches errors so that when there are problems one does
not get an error message, and is not thrown to the debugger.
To change this, run {\ETPS} as a user on the expertlist (which is in the Makefile),
set \indexflag{EXPERTFLAG} to T, and set the variable \indexother{*trap-errors*} to
nil as follows:
\begin{alltt}
(setq *trap-errors* nil)
\end{alltt}

\section{Interruptions and Emergencies}

This section consists mostly of implementation-dependent information,
although some of the following will work in most situations.
The following control characters will work in most circumstances:

\begin{description}
\item[] {\tt \^G <Return>} (i.e. type one followed by the other) will abort the current process.

\item[] {\tt <Return>} will stop a mating search and drop into the mate top level. When you
leave the mate top level, the mating search will attempt to continue (of course,
if you've made drastic changes, it may fail).

\item[] {\tt \^Z} will suspend lisp; you can then kill the job if necessary, or put it into
the background with {\tt bg}

\item[] {\tt \^C} will interrupt and throw you into the lisp break package
\end{description}

You can save a core image with the \indexcommand{TPS3-SAVE} command, as follows:

\begin{tpsexample}
<24>setflag save-file `mycore.exe'
<25>tps3-save
\end{tpsexample}

You should also save the flag settings, since when you restart {\TPS} with
this core image it will re-read the \indexfile{tps3.ini} file and may reset some flags.

In Allegro Common lisp, if you get the {\tt <cl>} prompt, the following are
some of the possible responses:

\begin{description}
\item[] {\tt :help} to see all the options

\item[] {\tt :cont} to attempt to continue

\item[] {\tt :out} to get back to top level

\item[] {\tt :res} to get back to top level

\item[] {\tt (exit)} to kill {\TPS}
\end{description}

In CMU common lisp, if you get the {\tt 0]} debugger prompt, the command
{\tt q} will get you back to the top level, and the command {\tt h} will
list all the other options available.

If {\TPS} crashes, or you discover a bug, use the \indexcommand{BUG-SAVE} command
to save the current state. Give your bug a name, and describe it (possibly
cut and paste the error message that was produced into the `comments' field).
This will save all the flag settings, the timing information, the history and
the current proof, in such a way that one can use \indexcommand{BUG-RESTORE}
to return to the same error at another time. Bugs are, by default, saved to
\indexflag{DEFAULT-BUG-DIR}, although they can be saved to \indexflag{DEFAULT-LIB-DIR}
by setting flag \indexflag{USE-DEFAULT-BUG-DIR} to NIL. There are also commands to list,
delete and examine the comments field of bugs (\indexcommand{BUG-LIST}, \indexcommand{BUG-DELETE}
and \indexcommand{BUG-HELP}, respectively); these correspond to library commands \indexcommand{LIST-OF-LIBOBJECTS},
\indexcommand{DELETE} and \indexcommand{SHOW-HELP}.

\section{How to produce manuals}
At the present time, to produce printed manuals, you must either have the
Scribe text-processing system and a Postscript printer or the \LaTeX ~system.


\subsection{Scribe manuals}
Enter the directory which corresponds to the manual you wish to make,
 then run Scribe on the file
{\tt manual}.  For example, if you wish to make the manual for {\ETPS}, do

\begin{alltt}
\% cd doc/etps
\% scribe scribe-manual
\end{alltt}

If you are using {\ETPS} as part of a course, you may wish to modify
the files in that directory to tailor it toward the inference rules
of your system.

To produce the facilities guide, which lists all commands, flags, modes,
etc., you can use the {\TPS} command \indexcommand{SCRIBE-DOC}.
% \comment{or \indexcommand{SCRIBE-CATS}.}
However, it may be easiest to use the
files in the directory {\it whatever/tps/doc/facilities}
(i.e., something like {\it /usr/tps/doc/facilities}). To do this,
you will use the file \indexfile{scribe-facilities.lisp} for a long and pretty comprehensive
manual, or \indexfile{scribe-facilities-short.lisp} for a shorter version which excludes some of the
obscurer TPS objects.
Use \indexfile{scribe-facilities-cmd.lisp} for the shortest manual of all, which contains only commands
and flags (i.e. the short facilities guide
without information on tactics, tacticals, binders, abbreviations, types,
subjects, modes, events, styles, grader commands etc). It also prints with
narrower page margins. To produce this manual, replace `-short'
with `-cmd' in the following. At the time of writing, manual-cmd.mss ran to 90 pages,
manual-short.mss was 156 pages, and manual.mss was 246 pages.

If you want a very short manual containing just a little information,
(such as a summary of a new search procedure) use scribe-facilities-temp. Edit
facilities-temp.lisp to contain just the categories you wish, and
FLAG. In TPS
tload `whatever/tps/doc/facilities/scribe-facilities-temp.lisp'
Then edit whatever/tps/doc/facilities/scribe-facilities-temp.mss
to eliminate all the flags you do not want in this manual, and
scribe the file.

Part of the lisp function in the file specifies the output file; EDIT
THAT PATHNAME to put the file into the facilities directory.
Then proceed as follows (to make the short manual)
\begin{alltt}
\% tps3
<2>tload `whatever/tps/doc/facilities/scribe-facilities-short.lisp'
Written file whatever/tps/doc/facilities/scribe-facilities-short.mss
T
<3>exit
\% cd whatever/tps/doc/facilities
\% scribe scribe-manual-short
\end{alltt}

If you were making the full manual, use the files \indexfile{scribe-facilities.lisp},
\indexfile{scribe-facilities.mss}, and \indexfile{scribe-manual.mss} in place of \indexfile{scribe-facilities-short.lisp},
\indexfile{scribe-facilities-short.mss}, and \indexfile{scribe-manual-short.mss}, respectively. Similarly,
for the very short manual, use \indexfile{scribe-facilities-cmd.lisp},
\indexfile{scribe-facilities-cmd.mss}, and \indexfile{scribe-manual-cmd.mss}.

Note: If you use a {\TPS} core image into which you have already loaded
certain wffs from your library, these will show up in the facilities
guide.

This information is also in the file \indexfile{doc/facilities/README}.

\subsection{\LaTeX ~manuals}

Similarly to the Scribe manuals, enter the directory which corresponds to the manual you wish to make,
 then run \LaTeX on the file
{\tt manual}.  For example, if you wish to make the manual for {\ETPS}, do

\begin{alltt}
\% cd doc/etps
\% latex latex-manual
\end{alltt}

You may have to compile it several times (up to three) in order to get the cross-references and index right. You will also need to have the \indexfile{tps.tex} and \indexfile{tpsdoc.tex} files containing the \LaTeX macros used in these manuals.

If you are using {\ETPS} as part of a course, you may wish to modify
the files in that directory to tailor it toward the inference rules
of your system.

To produce the facilities guide, which lists all commands, flags, modes,
etc., you can use the {\TPS} command \indexcommand{LATEX-DOC}. Note that this command generates the content of the manual but still need to be compiled with the \indexfile{latex-manual.tex} file which contains the title page, the preamble and the style directives.
% \comment{or \indexcommand{LATEX-CATS}.}
However, it may be easiest to use the
files in the directory {\it whatever/tps/doc/facilities}
(i.e., something like {\it /usr/tps/doc/facilities}). To do this,
you will use the file \indexfile{latex-facilities.lisp} for a long and pretty comprehensive
manual, or \indexfile{latex-facilities-short.lisp} for a shorter version which excludes some of the
obscurer TPS objects.
Use \indexfile{llatex-facilities-cmd.lisp} for the shortest manual of all, which contains only commands
and flags (i.e. the short facilities guide
without information on tactics, tacticals, binders, abbreviations, types,
subjects, modes, events, styles, grader commands etc). It also prints with
narrower page margins. To produce this manual, replace `-short'
with `-cmd' in the following. 

If you want a very short manual containing just a little information,
(such as a summary of a new search procedure) use facilities-temp. Edit
latex-facilities-temp.lisp to contain just the categories you wish, and
FLAG. In TPS
tload `whatever/tps/doc/facilities/latex-facilities-temp.lisp'
Then edit whatever/tps/doc/facilities/facilities-temp.tex
to eliminate all the flags you do not want in this manual, and
scribe the file.

Part of the lisp function in the file specifies the output file; EDIT
THAT PATHNAME to put the file into the facilities directory.
Then proceed as follows (to make the short manual)
\begin{alltt}
\% tps3
<2>tload `whatever/tps/doc/facilities/latex-facilities-short.lisp'
Written file whatever/tps/doc/facilities/latex-facilities-short.tex
T
<3>exit
\% cd whatever/tps/doc/facilities
\% scribe manual-short
\end{alltt}

If you were making the full manual, use the files \indexfile{latex-facilities.lisp},
\indexfile{latex-facilities.tex}, and \indexfile{latex-manual.tex} in place of \indexfile{latex-facilities-short.lisp},
\indexfile{latex-facilities-short.tex}, and \indexfile{latex-manual-short.tex}, respectively. Similarly,
for the very short manual, use \indexfile{latex-facilities-cmd.lisp},
\indexfile{latex-facilities-cmd.tex}, and \indexfile{latex-manual-cmd.tex}.

Note: If you use a {\TPS} core image into which you have already loaded
certain wffs from your library, these will show up in the facilities
guide.

This information is also in the file \indexfile{doc/facilities/README}.


\subsection{HTML manuals}

The information in the facilities guide can also be output in a rudimentary HTML format
by using the command \indexcommand{HTML-DOC}. You will need to provide TPS with the name
of an empty directory which has about 10MB of free space; the main page of the manual
will be the file \indexfile{index.html} in that directory.

Also, HTML documentation for {\ETPS} can be generated by using the command
\indexcommand{HTML-DOC} from within {\ETPS}.

\section{Miscellaneous Information}

The common \indexfile{tps3.ini} and \indexfile{etps.ini} files are used to set the default values
of some flags for a particular site.
The setting of LATEX-PREAMBLE
refers to input files tps.sty, tps.tex and vpd.tex, which are part of the {\TPS} distribution.
This is currently set using the value of sys-dir.
You could change the settings of these flags in  the ini files, but you probably won't need to.
The following commands from LATEX-PREAMBLE should not be changed, since
TPS relies on them:
\begin{tpsexample}
\(\setminus\setminus\)def\(\setminus\setminus\)endf\{\(\setminus\setminus\)end\{document\}\}
\(\setminus\setminus\)newcommand\{\(\setminus\setminus\)markhack\}[1]\{\(\setminus\setminus\)vspace*\{-0.6in\}\{\#1\}\(\setminus\setminus\)vspace*\{0.35in\}\(\setminus\setminus\)markright\{\{\#1\}\}\}
\end{tpsexample}


If you have access to the Scribe text-processing system,
you can change the documentation for {\ETPS} as appropriate (found in
the directory {\it doc/etps} and give it to students.  Note that in
its present form there are some CMU-specific assumptions made,
and it contains a listing of the inference rules as used in classes here.

You might want to employ a different scheme for grading.  The file
\indexfile{etps-events.lisp} defines how the results of each exercise are
output to the score file.  See Chapter \ref{events} for information
on how to define events.
