\documentclass[10pt,twoside]{book}
%\documentclass[11pt,twoside]{book}
%\documentclass[11pt,twoside]{report}
%\documentclass[11pt,dvips,twoside]{report}
%TPS User's Manual

\usepackage{amssymb}
\usepackage{latexsym}
\usepackage{makeidx}

%the following package permits us to use the comment environment
\usepackage{verbatim}

%the following package permits us to use the alltt environment
%which is like verbatim but has certain technical advantages
\usepackage{alltt}

\input /home/theorem/tps/doc/lib/tps.tex
\input /home/theorem/tps/doc/lib/tpsdoc.tex

\def\tpscomfont{}  % Temporary command for a special font for commands
                   %replace by \indexcommand{}

\setlength{\textwidth}{6.7in}
\setlength{\oddsidemargin}{0.1in}
\setlength{\evensidemargin}{0.1in}
%\setlength{\topmargin}{-0.5in}
%\setlength{\footskip}{-0.5in}
\setlength{\textheight}{9.0in}

\makeindex % generate index data

\begin{document}

%th next command starts page numbering using roman numberals
\frontmatter                           %used only for books


\begin{titlepage}
\begin{center}
{\bf

\vspace{1.00in}
{\Large TPS User's Manual}

\vspace{0.50in}


Peter B. Andrews\\
Chad E. Brown\\
Matthew Bishop\\
Sunil Issar\\
Dan Nesmith\\
Frank Pfenning\\
Hongwei Xi\\
Mark Kaminski\\
R\'{e}my Chr\'{e}tien\\
\vspace{0.50in}

\today 
}
\end{center}

\vspace{3.20in}

%If you want to add the copyright, adjust the vspace above and choose the
%appropriate one of the two commands below.
%\noindent{\copyright-and-research-credit}

\noindent{\research-credit}

\end{titlepage}

\tableofcontents

\mainmatter\pagestyle{headings}         %used only for books

\chapter{Introduction}

Welcome to {\TPS}, a theorem-proving environment developed at
Carnegie Mellon University for
both research and education.  {\TPS} is based on the typed
$\lambda$-calculus, and supports automatic, semi-automatic, and
interactive proofs of theorems of first- and higher-order logic.

For details on setting {\TPS} up on your system, see chapter \ref{set-up}.

There are two subsystems of {\TPS} which you may wish to use.  The
first, {\ETPS}, is an educational version which is used in logic courses
to prove theorems in purely interactive mode.  It contains none of the
automatic features of {\TPS}.  The second, Grader, is a set of functions
which are useful in keeping class grades, allowing, in particular, the
automatic processing of the record files created by {\ETPS}.  There are
separate manuals for each of these systems.  Grader is a part of {\TPS};
you can enter the Grader top level via the command
\indexcommand{DO-GRADES}.


\section{Guide to documentation}

At the present time, the {\TPS} manuals are far from finished.  Some areas
are covered in detail, while others are very sketchy.  We hope, however,
that these manuals 
will allow you to get started with {\TPS}.

To first learn the system, read \cite{AndrewsTPS88b}.  This will introduce
you to the interaction style of the program, tell you how to get
help, and show you how to enter wffs and use the inference rules.
{\ETPS} is used for educational purposes, however, so it contains none
of the automatic facilities of {\TPS} such as mating-search.

For a full list of all commands, flags, etc., see \cite{AndrewsTPS88d}.  Those
relating to mating-search are in a separate chapter, and other commands
and flags can be found under the heading {\tt mating-search} in other
chapters.

There is also a programmer's manual. 

%Unlike other manuals which are in {\bf Scribe}, this is in {\bf LaTeX}.

\cite{AndrewsTPS88c} covers the commands which are particular to the
Grader subsystem.

Additional references which you may wish to consult are listed in the
bibliography. \cite{Andrews95b} and \cite{AndrewsBrown2005} provide
general information about {\TPS}. Some of the material in 
the latter paper was taken from this manual.



\section{The {\TPS} User Interface}\label{user-int}

%Note: when we refer to a top level, "top" is an adjective, so there is
%no hyphen. If we say "top-level commmand", then "top-level" is an
%adjective, and should be hyphenated.

{\TPS} has various top levels.  There is a main top level, and
there are sub-top levels for specific
purposes. Note that the same command may be defined in more than one top level,
and have different effects. In each top level, the ? command will list
all the commands available at that top level.

% proofwindows, special characters, javawin, printed proofs
% editor

There are three interfaces for {\TPS}: text-based,
xterm-based, and Java-based.  The first
is purely text-based and behaves as a lisp interpreter.
The user types in text commands and {\TPS} outputs
a text response. % new paragraph?
To use the xterm-based interface, one starts
{\TPS} inside a new xterm with special character fonts.
The user also enters commands as text, but the output
can include special symbols (e.g., for logical operators
and quantifiers). % new paragraph?
The newest interface is a
menu-based Java interface.
With the Java interface
the user can either type in a command or choose
the command from a menu.  The output in the Java interface
can also display special symbols such as logical operators. 

In any of these interfaces {\TPS} can open new windows
for special purposes.  A common example is the use
of proofwindows for displaying different portions of the current
proof.  Using the command {\indexcommand BEGIN-PRFW}, the user opens
three new (xterm or Java) windows: the ``Complete Proof'' window,
the ``Current Subproof'' window and the ``Current Subproof and Line Numbers''
window.

The ``Complete Proof'' window displays the entire proof, and is useful
when the proof is either short or one wants a global view of the
current state of a proof.  At each stage in the construction of a
natural deduction proof, one unproved (or {\it planned}) proof line is
the current goal, and certain lines, which must be processed to derive
it, are designated as support lines for that goal. The current goal
and its supporting lines are displayed in the ``Current Subproof''
window.  The choice of these lines can be adjusted with the 
SUBPROOF, SPONSOR, and UNSPONSOR commands.
When the user applies a rule or tactic, the proofwindows
are automatically updated (under certain flag settings).

Another common use of auxiliary windows is in the
\indexother{EDITOR} top level.
This top level is used to manipulate formulas.
When the user enters this top level, a particular
formula is given (either explicitly or implicitly,
e.g., by giving the corresponding line number in
the current natural deduction proof).
{\TPS} opens two new windows: the ``Top Edwff''
window and the ``Current Edwff'' window.
The user can issue commands to point to different
subformulas (which changes the contents of the ``Current Edwff''
window).  If the user issues a command to change the
formula (e.g., {\indexcommand INSTALL} to instantiate abbreviations)
the effect is to change the formula in the ``Top Edwff'' window
and the corresponding subformula in the ``Current Edwff'' window.
In the \indexother{EDITOR} top level, one can also give names (``weak labels'')
to certain formulas which can later be used (in the same {\TPS} session) to refer to
the formula.  To save a formula permanently, one uses
the library facilities (see Chapter~\ref{library}).

{\TPS} is also capable of creating {\TeX} output.
For example, the command {\indexcommand TEXPROOF} creates
a {\TeX} file containing the current natural deduction
proof (which may be partial or complete).

The user interface for {\TPS} is the same as that for {\ETPS}. More 
information about it can be found in \cite{Andrews2004b}.




\nocite{Andrews2002a}
\nocite{Andrews00c}
\nocite{Andrews00a}
\nocite{Andrews95b}
\nocite{Andrews89}
\nocite{AndrewsTPS88b}
\nocite{AndrewsTPS88c}
\nocite{AndrewsTPS88d}
\nocite{Andrews87}
\nocite{Andrews84}
\nocite{Andrews81}
\nocite{Andrews80}
\nocite{Andrews76}
\nocite{Andrews71}
\nocite{Benzmuller98b}
\nocite{Benzmuller97}
\nocite{Bishop99a}
\nocite{Bishop99}
\nocite{Bishop98}
\nocite{Brown2002}
\nocite{Felty86}
\nocite{GORDON79}
\nocite{huet75}
\nocite{Miller87}
\nocite{Miller84}
\nocite{Miller83}
\nocite{Miller82}
\nocite{Pfenning86}
\nocite{Pfenning84}

\chapter{Proving theorems}
\label{proving}

\section{Introduction}


{\TPS} can be used to prove theorems automatically, interactively, or by
using various combinations of automatic and interactive facilities.
Even if one is primarily interested in using it in automatic mode, one
should consult the {\bf ETPS User's Manual} \cite{AndrewsTPS88b} to
learn the basics of interacting with both {\ETPS} and {\TPS}.  Even if one
is proving theorems purely interactively, one should probably use {\TPS}
rather than {\ETPS} so that one can use the library facilities.

To start a proof in {\TPS}, use the \indexcommand{PROVE} command. One can
then develop the proof in natural deduction style
interactively, semi-automatically, or
automatically.

To develop a proof in semi-automatic mode, 
one can alternate between applying rules of inference interactively,
using a command called \indexcommand{GO} to apply rules suggested by
{\TPS}, using a command called \indexcommand{GO2} to call a number of
tactics which quickly apply mundane rules of inference, and using the
automatic facilities of {\TPS} to prove lemmas and to derive certain
lines of the proof from other specified lines.  One can develop parts
of the proof in whatever chronological order is most convenient. For
example, one could start by inserting into the proof the lines which
represent the basic outline of a plan for the proof, and then work on
filling in various parts of this outline.

One can invoke the facilities for finding or completing the proof
automatically with the \indexcommand{DIY} (``do it yourself'')
or \indexcommand{DIY-L} command. (The latter is used to fill in part of
a proof, such as a proof of a lemma.)
Before doing this one should set various flags which control the
search procedures. These flags play very important roles.  See chapter
\ref{flags} for some information on how to set flags.
The command \indexcommand{PIY} (``prove it yourself'') combines
the commands \indexcommand{PROVE} and \indexcommand{DIY}.

A set of flags and values for these flags is called a {\it
mode}. 
If one does not know an appropriate
mode when one wishes to invoke {\TPS}'s automatic procedures, one can
use commands which systematically try a variety of modes.  A {\it
bestmode} for a theorem is a mode which can be used to prove that
theorem automatically (and which will, in general, produce a proof
more quickly than other modes).  The {\TPS} library contains certain sets of
modes called {\it goodmodes} such that each of the theorems which
{\TPS} can currently prove automatically can be proven using at least
one of the goodmodes in the set.  For example, GOODMODES1 is a list of
68 modes, and each of the 639 theorems for which
bestmodes are currently saved can be proven automatically by at least
one of the modes in GOODMODES1.  Using the command \indexcommand{PIY2}
(``Prove It Yourself, version 2''), \indexcommand{DIY2}, or 
\indexcommand{DIY2-L}, one can direct {\TPS} to 
apply its proof procedures with each of these 68 modes in turn for a specified
time, then increment the time limit and repeat the process, and
continue in this way until a proof is found or space or patience is
exhausted.  Since {\TPS} can prove many theorems of moderate
difficulty within a few seconds (see 
\cite{Bishop98,Bishop99,Bishop99a,Andrews00a,Brown2002,Brown2004a} for some
examples), this makes {\TPS} extremely
convenient to use for filling in gaps of moderate difficulty while one
is constructing a major proof semi-interactively.


\section{Automatic mode}
%::::::::::::::
%auto-mode.tex
%::::::::::::::

Automatic proof in {\TPS} is based on the `mating-search' paradigm described
in \cite{Andrews2002a}, \cite{Andrews81},  and \cite{Andrews89}, or
enhancements of it described in \cite{Bishop99,Bishop99a,Brown2002,
Brown2004a}.
{\TPS} starts by searching for an expansion proof
\cite{Miller84,Miller87,Andrews89}
or an extensional expansion proof \cite{Brown2004a}, and
then translates this into a natural deduction proof.

There are several ways in which one can use the automatic facilities of \TPS.


The command \indexcommand{DIY} calls the mating-search facility
(see Chapter \ref{ms-guide}), and if that succeeds in finding a proof, applies a
tactic (such as {\tt COMPLETE-TRANSFORM-TAC} or {\tt PFENNING-TAC}) to
translate the expansion tree proof  into a natural
deduction proof. Using  \indexcommand{DIY} is the simplest way to
prove a theorem automatically.

The nature of the search initiated by \indexcommand{DIY} or by
\indexcommand{GO} will be heavily influenced by which setting you have
for the flag \indexflag{DEFAULT-MS}. If you do {\it setflag default-ms} and
then respond with ? when asked for an argument, you'll see the
available options.  Help messages will tell you a little about them, and
there is some more information in Chapter \ref{ms-guide}.


If you want to see more of what is going on in the search for a proof,
you may want to start proving a theorem by entering
the  MATE top-level with a particular wff and
experimenting with the mating-search commands.  To do this, use the
\indexcommand{MATE} command. See section \ref{ms-guide} for information
about this top-level.

After a mating has been found in the mating-search top-level and the
expansion proof constructed, you can apply the proper substitutions to
the expansion tree with the \indexcommand{MERGE-TREE} command, which
also eliminates redundant branches from the tree.  Or just use the
\indexcommand{LEAVE} command, which will apply the merge procedure and
then return you to the main top-level.  To build a natural deduction
proof using the information in the expansion proof you just
constructed, use the command \indexcommand{ETREE-NAT}.  You may also
enter mating-search with the most recent expansion proof created.
This is stored in the variable {\tt \indexother{CURRENT-EPROOF}},
and is the default
value for the {\tt MATE} command.  When you leave mating-search, this
variable is set to the current expansion proof, so you may reenter
with the same proof if desired.  The value of this variable is also
the expansion proof used when translating into a natural deduction
proof with {\tt ETREE-NAT}.  Another variable, {\tt LAST-EPROOF}, stores
the value of the expansion proof created before {\tt CURRENT-EPROOF}.
Either of these symbols may be entered when prompted by the {\tt MATE}
command for a wff or eproof.

Details of how to save output from the mating search are in Chapter
\ref{output}.

Let us discuss here exactly what the effects of the various mating
search and translation commands are.
\begin{itemize}
\item The command \indexcommand{MATE}, when it returns to the top level,
alters the value of
the variable {\tt CURRENT-EPROOF} to be the expansion proof constructed.

\item The command \indexcommand{ETREE-NAT} takes the expansion proof stored
in {\tt CURRENT-EPROOF}, places the information it contains on the specified
line of the natural deduction proof, then calls the specified tactic on the
natural deduction proof.

\item The command \indexcommand{DIY} will construct an expansion proof for
the specified planned line of the current natural deduction proof,
then, like {\tt ETREE-NAT}, will place the information it contains on
the specified line(s) of the natural deduction proof and call the
specified tactic.  The value of {\tt CURRENT-EPROOF} is not changed, however.
\end{itemize}

Note that once the information from the expansion proof has been
placed into the natural deduction proof, as by {\tt DIY} and
{\tt ETREE-NAT}, the value of {\tt CURRENT-EPROOF} is unnecessary to
the translation process, and one can use the command {\tt USE-TACTIC} to
apply translation tactics.
But since the {\tt MATE} command does not transfer the information fromn
{\tt CURRENT-EPROOF} into the natural deduction proof,
the {\tt ETREE-NAT} command must be applied after {\tt MATE} in
order to start the translation process.

It is possible to translate natural deduction proofs to expansion proofs.
This is accomplished by the command {\tt NAT-ETREE}.  The expansion proof
created will be stored in the variable {\tt CURRENT-EPROOF}.  At present,
this procedure
is not complete and will fail on some natural deduction proofs, including:
\begin{itemize}
\item Those which are not cut-free, i.e., do not satisfy the subformula property

\item Those which use substitution of equality rules

\item Those which use the assertion of axioms, like reflexivity of equality
\end{itemize}







\subsection{An Example Using DIY}

\begin{tpsexample}
{\it Comments are in italics.}
<1>save-work workfile
{\it We create a file called workfile.work which will record the commands used.
This is optional.}
<2>prove theorem-name
{\it We could also use the `exercise' command if the theorem to be proved is
an exercise in ETPS.}
<3>diy
\end{tpsexample}

\subsection{An Example Using MATE and ETREE-NAT}


\begin{tpsexample}
{\it Alternatively, we may proceed as follows:}
<1>save-work workfile
<2>prove theorem-name
<3>mate
<4>go
<5>merge-tree
{\it You will be prompted for this when leaving the mate top level.}
<6>etree-nat
{\it To convert the proof to natural deduction style.}
<7>daterec
{\it To store the timing information in the library.}
{\it You may also want to go into the library to store a mode recording the
current flags.}
<8>texproof
{\it To produce printable output.}
\end{tpsexample}


\subsection{A  Longer Example Using MATE and ETREE-NAT}
\label{MATE-then-ETREE-NAT}

To make the formulas easier to read, we have left off the type information.
Note that `\% f x' denotes the image of the set `x' under the function `f'.

\begin{tpsexample}
<8>exercise x5203
(100)        !  \% f [x INTERSECT y] SUBSET \% f x INTERSECT \% f y           PLAN1

{\it We call mating-search directly.}

<9>mate
GWFF (GWFF0): gwff [No Default]>x5203
POSITIVE (YESNO): Positive mode [No]>!

{\it We call the automatic proof search.}
<Mate1>go
...
Displaying VP diagram ...

|                    LEAF7                     |
|                    x T0                      |
|                                              |
|                    LEAF8                     |
|                    y T0                      |
|                                              |
|                    LEAF6                     |
|                  X0 = f T0                   |
|                                              |
|LEAF12      LEAF13       LEAF15      LEAF16   |
|\(\sim\)x t21 OR \(\sim\)X0 = f t21 OR \(\sim\)y t22 OR \(\sim\)X0 = f t22| 
..*.+1.*.+2.*.+3.*.+4..
Trying to unify mating:(4 3 2 1)
Substitution Stack:

t21   ->   T0
t22   ->   T0..
{\it Return to the main top-level.}
<Mate2>leave
Merging the expansion tree.  Please stand by.
****
T
{\it Begin the translation process using the expansion proof just constructed
in the mating-search top-level.}
<9>etree-nat
PREFIX (SYMBOL): Name of the Proof [X5203]>
NUM (LINE): Line Number for Theorem [100]>
TAC (TACTIC-EXP): Tactic to be Used [COMPLETE-TRANSFORM-TAC]>
MODE (TACTIC-MODE): Tactic Mode [AUTO]>

{\it We elide the output from the translation.}

<0>pall

(1)   1      !  EXISTS t18 .x t18 AND y t18 AND x4 = f t18                   Hyp
(2)   1,2    !  x t18 AND y t18 AND x4 = f t18                       Choose: t18
(3)   1,2    !  x t18                                                   RuleP: 2
(4)   1,2    !  y t18                                                   RuleP: 2
(5)   1,2    !  x4 = f t18                                              RuleP: 2
(88)  1,2    !  x t18 AND x4 = f t18                                  RuleP: 3 5
(89)  1,2    !  EXISTS t19 .x t19 AND x4 = f t19                    EGen: t18 88
(93)  1,2    !  y t18 AND x4 = f t18                                  RuleP: 4 5
(94)  1,2    !  EXISTS t20 .y t20 AND x4 = f t20                    EGen: t18 93
(95)  1,2    !      EXISTS t19 [x t19 AND x4 = f t19]
                 AND EXISTS t20 .y t20 AND x4 = f t20               RuleP: 89 94
(96)  1      !      EXISTS t19 [x t19 AND x4 = f t19]
                 AND EXISTS t20 .y t20 AND x4 = f t20                RuleC: 1 95
(97)         !          EXISTS t18 [x t18 AND y t18 AND x4 = f t18]
                 IMPLIES     EXISTS t19 [x t19 AND x4 = f t19]
                         AND EXISTS t20 .y t20 AND x4 = f t20         Deduct: 96
(98)         !  FORALL x4 .        EXISTS t18 [x t18 AND y t18 AND x4 = f t18]
                            IMPLIES     EXISTS t19 [x t19 AND x4 = f t19]
                                    AND EXISTS t20 .y t20 AND x4 = f t20
                                                                     UGen: x4 97
(99)         !  FORALL x4 .        EXISTS t18 [x t18 AND y t18 AND x4 = f t18]
                            IMPLIES     EXISTS t19 [x t19 AND x4 = f t19]
                                    AND EXISTS t20 .y t20 AND x4 = f t20
                                                                    Equality: 98
(100)        !  \% f [x INTERSECT y] SUBSET \% f x INTERSECT \% f y   EquivWffs: 99
\end{tpsexample}


\subsection{Automatically Produced Proofs with Lemmas}

Some search procedures (e.g., using \indexcommand{DIY} when \indexflag{DEFAULT-MS} is \indexother{MS98-1} and \indexflag{DELAY-SETVARS} is T)
will generate proofs which include lemmas which are asserted in the proofs.  The proofs of these lemmas are also generated
and can be examined by using \indexcommand{RECONSIDER}.  The name of the lemma is included with the
Assert justification.  In general, \indexcommand{PROOFLIST} lists the names of all the natural deduction proofs
in memory.  An example is the proof of THM2 produced using the mode EASY-SV-MODE.


\subsection{Interrupts: \^C, <CR>, \^G<CR>, M<CR> and T<CR>}\label{interrupt}

During mating search, if the value of the flag \indexflag{INTERRUPT-ENABLE} is
T, you can interrupt the search by typing {\tt <Return>}.  You will get
a new mating-search (or matingstree, as appropriate) top-level, where you can inspect
and/or alter the current expansion tree and mating.  Type \indexcommand{LEAVE} to return to mating-search.

Similarly, you can type {\tt \^G<Return>} (i.e. Control-G, then Return) to abandon the
search for good and return to the current top level. It is not possible to restart a
search after doing this.

Lastly, you can type {\tt M<Return>} to see the current mating, or {\tt T<Return>} to see a
printout of the time taken so far by the search. The search will not be interrupted at all if you do this.

Of course, there is one, more drastic, way to interrupt a search: press {\tt \^C}.
This will work regardless of the setting of \indexflag{INTERRUPT-ENABLE}, and will throw you into
the Lisp debugger. Leaving the debugger with a restart should return you to either the {\TPS} top level or the
current sub-toplevel; leaving it with a continue should carry on the search.

For Allegro Common Lisp (debuggers are not standardized, so this will be different in other lisps),
this works as follows:
\begin{alltt}
\^C                             (control C)
Error: Received signal number 2 (Keyboard interrupt)
Restart actions (select using :continue):
 0: continue computation
[1c] <cl>		
	{\it Here one can use debugger commands,
        or get into TPS on another level as follows:}
[1c] <cl> (secondary-top-main)
	{\it Now do what you want in TPS to examine things.
        To get back to the original TPS level:}
<73>\^C
Error: Received signal number 2 (Keyboard interrupt)
Restart actions (select using :continue):
 0: continue computation
 1: continue computation
[2c] <cl> :cont 1
{\it Now the original TPS process continues}
{\it Typing :res would have returned us to the TPS top level}
\end{alltt}


\section{Interactive Mode}

\subsection{Natural Deduction}

There are several examples showing how to construct natural deduction
proofs completely interactively in Chapter 4 of the {\bf ETPS User's
Manual} \cite{AndrewsTPS88b}.  Everything that can be done in {\ETPS}
can also be done in {\TPS}.

To prepare a demonstration of how to construct a proof
interactively, use \indexcommand{SAVE-WORK}.  To give the demonstration, use
\indexcommand{BEGIN-PRFW} with the flags \indexflag{PROOFW-ACTIVE},
\indexflag{PROOFW-ACTIVE+NOS} or
\indexflag{PROOFW-ALL} set to T, then
use \indexcommand{EXECUTE-FILE} to give the demonstration;
respond `yes' to the `STEPPING?' prompt.
The audience will see the proof being constructed step-by-step in the
proofwindows. (Note: Stepping will only stop between each command, so it will not stop
(for example) between each step of GO2. Also, if you change top level in a work file,
stepping will be turned off until you return to the original top level. It is possible to force
a stop in these situations by inserting a PAUSE command into the workfile; see the help message
for \indexcommand{PAUSE} for more details.)

Alternatively, you may wish to go into \indexcommand{MATE} to prove the theorem
automatically, then call \indexcommand{ETREE-NAT} in interactive mode to
demonstrate (with the aid of the proofwindows) how the natural deduction proof
can be constructed step-by-step. Make sure that the setting of \indexflag{ETREE-NAT-VERBOSE}
is appropriate before doing this!
(The user should note that if there is a command in the
work file that changes the top level - for example, \indexcommand{BEGIN-PRFW}
or \indexcommand{MATE} - then this command will also turn off stepping.
One can get around this limitation by splitting the work file into two at
the point where the change of top level occurs.)

While translating expansion proofs to natural deduction proofs, you
may wish to set the flag \indexflag{TACMODE} to {\tt INTERACTIVE}.  See Section
\ref{tactics} for information on the effect of this flag. If you are using
\indexcommand{ETREE-NAT} it is best to call \indexcommand{MERGE-TREE} first;
in fact, the mate top level will automatically prompt you for this if you
attempt to leave with a completed mating.

Equalities in expansion proofs can be translated into equational
proofs as laid out in \cite{Pfenning86}.  This does not include the
work on extensionality.  In order to have the proper expansion
proof transformation done during merging, one should have the flag
\indexflag{REMOVE-LEIBNIZ} set to {\tt T}, which is the default.  It is also best to
have the flags \indexflag{REWRITE-EQUAL-EXT}, \indexflag{REWRITE-EQUAL-EAGER}, and
\indexflag{REWRITE-ONLY-EXT} set to {\tt NIL}, and \indexflag{REWRITE-EQUALITIES} set
to {\tt T}.

From a {\TPS} prompt (this could be either a top level prompt or the prompt produced by a
command which requires you to input some arguments) you can type \indexother{PUSH} to suspend
what you're doing and start a new top level. The command \indexother{POP} will return from this
top level to the point where you typed \indexother{PUSH}. For example, you could suspend
an interactive session with {\tt ETREE-NAT} in order to print out the proof at various stages of
development.

It is also possible to interrupt an automatic search, change some flags and then continue with the
search; see section \ref{interrupt} for details.

\subsection{Extensional Sequent Calculus}

There is a top level \indexcommand{EXT-SEQ} which provides
an environment for constructing derivations of (one-sided) sequents
in the extensional sequent calculus described in \cite{Brown2004a}.
The command \indexcommand{?} will list all the available rules.
Some rules are basic while others are derived rules which can be
expanded later.  The cut rule is included, but certain commands
(namely, \indexcommand{CUTFREE-TO-EDAG}) will only work when given
cut-free derivations.

The command \indexcommand{GO2} in the top level EXT-SEQ will
automatically suggest rules which are applicable.  This can
make interactive construction of a sequent calculus derivation
much easier.

Sequent derivations can be saved and restored using \indexcommand{SAVEPROOF}
and \indexcommand{RESTOREPROOF}.

\subsection{Manipulating Proofs}\label{manip-pfs}

{\TPS} has many facilities for manipulating proofs.  There can be many
proofs in memory at the same time, and the command \indexcommand{PROOFLIST} lists the
names of all the natural deduction proofs or extensional sequent
derivations currently in memory, along with the theorems they
prove. One proof is designated as the current proof, and one can
change this with the \indexcommand{RECONSIDER} command.  Proofs can be saved as files
by \indexcommand{SAVEPROOF}, and restored to memory by \indexcommand{RESTOREPROOF}.

\indexcommand{CREATE-SUBPROOF} creates a new proof consisting of specified lines of
the current proof, plus all the lines on which they depend.
\indexcommand{MERGE-PROOFS} merges all of the lines of a subproof into the current
proof.  \indexcommand{TRANSFER-LINES} copies specified lines of a subproof, and all
lines on which they depend, into the current proof.

Various commands (\indexcommand{MOVE}, \indexcommand{DELETE}, 
\indexcommand{INTRODUCE-GAP}, \indexcommand{MODIFY-GAPS},
\indexcommand{RENUMBERALL}, \indexcommand{SQUEEZE}) are 
available for deleting or moving portions of
a proof, changing the gaps in the numbers between lines, and
renumbering the lines.  The \indexcommand{CLEANUP} command will delete all lines of a
completed proof which are not actually needed to prove the final line.

Sometimes one wishes to look at the main steps in a natural deduction
proof without looking at all the intermediate steps.  The command
\indexcommand{BUILD-PROOF-HIERARCHY} builds dependency information into a proof so
that the proof can be viewed as a hierarchy of subproofs.  The command
\indexcommand{PBRIEF} displays the proof lines included in the top levels of this
hierarchy to a depth specified by the user.  When one asks {\TPS} to
\indexcommand{EXPLAIN} a specified line in a proof, it displays in the same way the
lines of the proof which are used to prove the specified line.
\indexcommand{PRINT-PROOF-STRUCTURE} displays the hierarchy itself in terms of the
numbers of the proof lines.


\section{Combining Interactive and Automatic Searches}

The command \indexcommand{GO} will apply inference rules based
upon the structure of the formulas in the current proof structure --
breaking up conjunctions, applying the deduction rule, instantiating
definitions, etc.  This facility is rather shallow, and requires the user to
provide any terms for universal instantiation or existential generalization.
Thus while it may be useful in getting a proof started, it will
eventually fail.  The {\tt GO} facility is fairly static as well; to change
the priority of the rules and/or keep some rules from being applied requires
some programming (see the file {\tt ml2-prior.lisp}).

Tactics can also be used to do the same job.  In this case, the user can
build a tactic (see section \ref{tactics}) which will apply inference rules
in whatever order is desired.  Tactics allow the user to experiment with
different proof strategies and express his or her own creative spirit.
Tactics for applying most of the current inference rules are already defined.
See section \ref{usetac} for more information on commands which invoke
the most commonly used tactics
such as \indexmexpr{MONSTRO} and \indexmexpr{GO2}.

If a proof is being constructed interactively by natural deduction
commands, it is possible to use the command \indexcommand{DIY-L} to automatically
complete some of the subproofs. This calls DIY to prove a
lemma, adding lines to the proof within a specified range. This is useful
for quickly filling in the trivial parts of more difficult theorems which
you are proving interactively. See the help message for \indexcommand{DIY-L}
for details. To keep the proof short and readable, automatically-produced subproofs
need not be translated completely; see the help message of the flag \indexflag{USE-DIY}
for more information about this.

The tactic {\tt DIY-TAC} basically just
calls the  \indexcommand{DIY} command, and thus can be used in tactics which first do
some manipulation of the proof based upon the structure of the
formulas, then call mating-search when instantiations of quantifiers
must be found. (Note: if the flag \indexflag{USE-DIY} is set,
the translation may simply consist of justifying the goal line as `Automatic'
from the support lines. This is useful for keeping the proof short.)

We now give several examples of how to start a proof interactively
and continue automatically. Note that if you modify the expansion tree
interactively, you should use the command \indexcommand{CJFORM} before
attempting to construct a mating interactively.

\subsection{An Example Using Go to Start the Proof }

\begin{tpsexample}
<3>exercise x5203
(100)        !  \% f [x INTERSECT y] SUBSET \% f x INTERSECT \% f y           PLAN1
{\it We use the GO command to start the proof.}
<4>go
Considering planned line 100.
  IDEF 100
Command [(IDEF 100)]>
{\it Instantiate the definition of SUBSET.}
(99)         !  FORALL x .        \% f [x INTERSECT y] x
                           IMPLIES [\% f x INTERSECT \% f y] x               PLAN2
Considering planned line 99.
{\it The first two occurrences of x have different types, but they are not shown.}
  UGEN 99
Command [(UGEN 99)]>
(98)         !  \% f [x INTERSECT y] x IMPLIES [\% f x INTERSECT \% f y] x    PLAN3
Considering planned line 98.
  DEDUCT 98
Command [(DEDUCT 98)]>
(1)   1      !  \% f [x INTERSECT y] x                                        Hyp
(97)  1      !  [\% f x INTERSECT \% f y] x                                  PLAN4
Considering planned line 97.
  EDEF 1
  IDEF 97
Command [(EDEF 1)]>
(2)   1      !  EXISTS t .[x INTERSECT y] t AND x = f t                  Defn: 1
Considering planned line 97.
  RULEC 97 2
Command [(RULEC 97 2)]>
Some defaults could not be determined.
y (GWFF): Chosen Variable Name [No Default]>''t''
(3)   1,3    !  [x INTERSECT y] t AND x = f t                          Choose: t
(96)  1,3    !  [\% f x INTERSECT \% f y] x                                  PLAN7
Considering planned line 96.
  ECONJ 3
Command [(ECONJ 3)]>
(4)   1,3    !  [x INTERSECT y] t                                        Conj: 3
(5)   1,3    !  x = f t                                                  Conj: 3
Considering planned line 96.
  SUBST=L 5
  SUBST=R 5
  EDEF 4
  IDEF 96
Command [(SUBST=L 5)]>\^G
`[Command aborted.]'
{\it We abort the command because the advice doesn't seem very helpful.
Here are the current active lines of the proof.}
<5>\^p
(4)   1,3    !  [x INTERSECT y] t                                        Conj: 3
(5)   1,3    !  x = f t                                                  Conj: 3
               ...
(96)  1,3    !  [\% f x INTERSECT \% f y] x                                  PLAN7
T
{\it Call mating-search on the partial proof.}
<6>diy
GOAL (PLINE): Planned Line [96]>
SUPPORT (EXISTING-LINELIST): Support Lines [(4 5)]>
...
Displaying VP diagram ...

|                   LEAF88                   |
|                    x t                     |
|                                            |
|                   LEAF89                   |
|                    y t                     |
|                                            |
|                  LEAF90                    |
|                  x = f t                   |
|                                            |
|LEAF96      LEAF97      LEAF99     LEAF100  |
|\(\sim\)x t16 OR \(\sim\)x = f t16 OR \(\sim\)y t17 OR \(\sim\)x = f t17|
..*.+1.*.+2.*.+3.*.+4..
Trying to unify mating:(4 3 2 1)
Substitution Stack:

t16   ->   t
t17   ->   t.
Eureka!  Proof complete..
|                 LEAF88                 |
|                  x t                   |
|                                        |
|                 LEAF89                 |
|                  y t                   |
|                                        |
|                LEAF90                  |
|                x = f t                 |
|                                        |
|LEAF96     LEAF97     LEAF99    LEAF100 |
| \(\sim\)x t  OR \(\sim\)x = f t OR  \(\sim\)y t  OR \(\sim\)x = f t|
****
{\it A proof was found and now will be translated back to natural deduction.}
What tactic should be used for translation? [COMPLETE-TRANSFORM-TAC]>
Tactic mode? [AUTO]>

{\it For brevity we have elided the output from the translation process.}

\end{tpsexample}

{\it Here's the complete proof. Note that it is slightly different from
that shown in section \ref{MATE-then-ETREE-NAT}, where all the
definitions were instantiated at one time.}

\begin{tpsexample}

<7>pall
(1)   1      !  \% f [x INTERSECT y] x                                        Hyp
(2)   1      !  EXISTS t .[x INTERSECT y] t AND x = f t                  Defn: 1
(3)   1,3    !  [x INTERSECT y] t AND x = f t                          Choose: t
(4)   1,3    !  [x INTERSECT y] t                                        Conj: 3
(5)   1,3    !  x = f t                                                  Conj: 3
(6)   1,3    !  x t AND y t                                         EquivWffs: 4
(7)   1,3    !  x t                                                     RuleP: 6
(8)   1,3    !  y t                                                     RuleP: 6
(88)  1,3    !  x t AND x = f t                                       RuleP: 5 7
(89)  1,3    !  EXISTS t14 .x t14 AND x = f t14                       EGen: t 88
(93)  1,3    !  y t AND x = f t                                       RuleP: 5 8
(94)  1,3    !  EXISTS t15 .y t15 AND x = f t15                       EGen: t 93
(95)  1,3    !      EXISTS t14 [x t14 AND x = f t14]
                 AND EXISTS t15 .y t15 AND x = f t15                RuleP: 89 94
(96)  1,3    !  [\% f x INTERSECT \% f y] x                          EquivWffs: 95
(97)  1      !  [\% f x INTERSECT \% f y] x                            RuleC: 2 96
(98)         !  \% f [x INTERSECT y] x IMPLIES [\% f x INTERSECT \% f y] x
                                                                      Deduct: 97
(99)         !  FORALL x .        \% f [x INTERSECT y] x
                           IMPLIES [\% f x INTERSECT \% f y] x          UGen: x 98
(100)        !  \% f [x INTERSECT y] SUBSET \% f x INTERSECT \% f y        Defn: 99

\end{tpsexample}


\subsection{Duplicating Interactively and then Running Matingsearch}

\begin{tpsexample}
<11>mate
GWFF (GWFF0-OR-EPROOF): Gwff or Eproof [No Default]>thm-name
DEEPEN (YESNO): Deepen? [Yes]>
WINDOW (YESNO): Open Vpform Window? [No]>

<Mate12>vp
<Mate13>goto leaf58
{\it This should be the name of the literal you wish to duplicate.}
<Mate14>up
{\it Go to the appropriate expansion node.}
EXP8
<Mate18>vp
<Mate19>dup-var
<Mate21>dp*
<Mate23>\(\wedge\)
<Mate24>vp
{\it To check we got it right...}
<Mate27>go
{\it To start matingsearch.}
\end{tpsexample}

Users who are planning to both duplicate and Skolemize interactively should
note that duplication and Skolemization are not interchangeable. If
you duplicate first and then Skolemize you will get different Skolem
functions, whereas if you Skolemize and then duplicate you will get
the same Skolem function twice.

\subsection{Applying Primsubs Interactively and then Running Matingsearch}

\begin{tpsexample}
<23>mate
GWFF (GWFF0-OR-EPROOF): Gwff or Eproof [No Default]>thm-name
DEEPEN (YESNO): Deepen? [Yes]>
WINDOW (YESNO): Open Vpform Window? [No]>

<Mate24>name-prim
{\it We see that PRIM1 is the term we want to use.}
<Mate25>vp
{\it This will show the variable name we want. Note that pdeep and psh may
not show the right variable; things get renamed!}
<Mate26>prim-single
TERM (GWFF):  [No Default]>prim1
VAR (GWFF):  [No Default]>`R^3(OII)'
<Mate27>vp	
{\it Just to check...}
<Mate28>go
\end{tpsexample}

\subsection{Mating Interactively and then Unifying}

\begin{tpsexample}
<Mate24>cjform
<Mate25>vp
{\it We do this because the leaf names may have changed.}
<Mate26>add-conn*
{\it Now we type in a whole list of connections; we could also have used add-conn.}
{\it Suppose we now think we have a complete mating...}
<Mate27>complete-p
Mating is complete.
<Mate28>unify
{\it We enter the unification top level; this only works for higher-order problems.}
<Unif29>go

In case TPS halts with the message
MORE
you can proceed as follows:
<Unif30>MAX-UTREE-DEPTH
MAX-UTREE-DEPTH [20]>30
<Unif31>go
\end{tpsexample}

\subsection{Duplicating and Mating Interactively and then Converting to ND}

\begin{tpsexample}
<66>mate
<Mate67>vp
<Mate68>goto leaf58
<Mate69>up
<Mate73>vp
<Mate74>dup-var
<Mate76>dp*
<Mate78>^
<Mate79>vp
<Mate80>add-conn leaf29 leaf58
{\it ...as often as needed to get the mating...}
<Mate88>complete-p
<Mate89>show-mating
<Mate90>show-substs
<Mate91>vp
<Mate92>leave
Merge the expansion tree? [Yes]>
<93>etree-nat
\end{tpsexample}

\subsection{Using Prim-Single and Dup-Var Interactively}

\begin{tpsexample}
<8>mate
{\it We should check some flags before proceeding:}
<Mate10> PRIM-BDTYPES
<Mate11>MIN-PRIM-DEPTH
<Mate12>MAX-PRIM-DEPTH
<Mate13> PRIM-QUANTIFIER
<Mate16>vp
<Mate17>NAME-PRIM
<Mate24>PRIM-SINGLE
SUBST (GWFF):  [No Default]>prim14
VAR (GWFF):  [No Default]>`M^0(O(II))'
<Mate25>vp
<Mate30>etd
<Mate31>goto
NODE (SYMBOL):  [No Default]>exp2
<Mate33>dup-var
<Mate35>dp*
<Mate37>^
<Mate38>vp
<Mate39>add-conn*
<Mate40>complete-p
{\it We should check some flags before proceeding:}
<Mate42>MAX-SEARCH-DEPTH
<Mate43>MAX-UTREE-DEPTH
<Mate44>unify
<Unif45>go
<Unif47>leave
<Mate48>leave
Merge the expansion tree? [Yes]>
<50>etree-nat
<51>pall
{\it ...to see the final proof.}
\end{tpsexample}


\chapter{Flags}

Here is an example of how to add the flag {\it neg-prim-sub} to \TPS:

Insert into the file prim.lisp the code:
\begin{verbatim}
(defflag neg-prim-sub
  (flagtype boolean)
  (default nil)
  (subjects primsubs)
  (mhelp "When T, one of the primitive substitutions will introduce negation."))

\end{verbatim}

Actually, that code almost worked, but changing the flag did not have the
desired effect. The primsubs were stored in a hashtable which, once computed,
was never changed again, so the code had to be  
replaced by:

\begin{verbatim}
(defflag neg-prim-sub
  (flagtype boolean)
  (default nil)
  (change-fn (lambda (a b c)
	       (declare (ignore a b c))
	       (ini-prim-hashtable)))
  (subjects primsubs)
  (mhelp "When T, one of the primitive substitutions will introduce negation."))

\end{verbatim}

Also put into the file auto.exp the line
\begin{verbatim}
(export '(neg-prim-sub))
\end{verbatim}

There are two ways to update flags. One is to do it manually. This is
supported by function update-flag. The other way is to set flags
automatically. For example, you may have to do this in your .ini
file. If XXX is a flag and you want to set it to YYY, then you can add
a line (set-flag 'XXX YYY) in your .ini file. Sometimes, you may use
(setq XXX YYY), but this is highly discouraged because XXX may have a
"change-fn" associated with it, which should be called whenever you
set XXX. Flag \indexflag{HISTORY-SIZE} is such an example.
(Note that if the variable being set is just a variable, and not a \TPS
flag, then the setq form is correct.)

\section{Symbols as Flag Arguments}

If your new flag accepts symbols as arguments, and only certain symbols 
are acceptable (as in, for example, \indexflag{PRIMSUB-METHOD} or \indexflag{APPLY-MATCH}),
the symbols which can be used should have help messages attached somehow. This can either
be done by defining a new category for the arguments, such as ORDERCOM or DEV-STYLE, or it can
be done using the \indexother{definfo} command: {\tt (definfo foo (mhelp "Help text."))} attaches
the given text to the symbol {\tt foo}.

\section{Synonyms}

It is possible to define two flags with different names which are 
synonymous to each other, using the \indexcommand{defsynonym} macro.
The advantage of this is that it allows the name of a flag to be changed
(from the user's point of view) without requiring either a change in the
code or extensive editing of all the modes saved in the library.

For example:

%\begin{tpsexample}
\begin{verbatim}
(defsynonym SUBNAME
            (synonym TRUENAME)
            (replace-old T)
            (mhelp "SUBNAME is a synonym for the flag TRUENAME."))
\end{verbatim}
%\end{tpsexample}

defines a new synonym for {\tt TRUENAME}. The {\it replace-old} property 
determines whether or not the new synonym is to be regarded as the
"new name" of the flag, if replace-old is {\tt T} 
(and so to be recorded in the library, etc.) 
or merely as an alias, if replace-old is {\tt NIL}.

\section{Relevancy Relationships Between Flags}

When defining a new flag, one can specify relevancy relationships
between the flag and the values of other flags.  For example,
if the flag {\bf DEFAULT-MS} is set to {\bf MS90-3}, then
the flag {\bf MS98-NUM-OF-DUPS} is irrelevant.  On the other
hand, if {\bf DEFAULT-MS} is set to {\bf MS98-1}, then the
flag {\bf MS98-NUM-OF-DUPS} is relevant.  Since we expect the
relevancy information to be incomplete at any point in time,
it makes sense to explicitly record relevancy and irrelevancy
information separately.

The slots used to record these relationships are
\begin{itemize}
\item irrelevancy-preconditions
\item relevancy-preconditions
\item irrelevant-kids
\item relevant-kids
\end{itemize}

If we want to record the irrelevancy relationship in
the {\bf DEFAULT-MS}/{\bf MS98-NUM-OF-DUPS} 
example above, there are two ways.  The first is to
record this in the definition of {\bf DEFAULT-MS} using the
{\it irrelevant-kids} slot as shown below.
\begin{verbatim}
(defflag default-ms
  (flagtype searchtype)
  (default ms90-3)
  (subjects mating-search . . .)
  (change-fn (lambda (flag value pvalue)
	       (when (neq value pvalue) (update-otherdefs value))))
  (irrelevant-kids ((neq default-ms 'ms98-1) '(ms98-num-of-dups)))
  (mhelp . . . ))
\end{verbatim}
The format for this slot is a list of elements of the form
\verb+(<pred> <sexpr>)+
where \verb+<pred>+ is a condition on the value of the flag
being defined
and \verb+<sexpr>+ is an s-expression
which should evaluate to a list of flags.

Alternatively, we can specify the relationship when
we define {\bf MS98-NUM-OF-DUPS} using the slot {\it irrelevancy-preconditions}
\begin{verbatim}
(defflag ms98-num-of-dups
  (default nil)
  (flagtype null-or-posinteger)
  (subjects ms98-1)
  (irrelevancy-preconditions (default-ms (neq default-ms 'ms98-1)))
  (mhelp . . .))
If some positive integer n, we reject any component using more than n 
of the duplications."))
\end{verbatim}
The format for this slot is a list of pairs \verb+(<flag> <pred>)+
where \verb+<flag>+ is a flag and \verb+<pred>+
is a condition on the value of the flag \verb+<flag>+.

Relevancy relationships can be specified in an analogous way.
\begin{verbatim}
(defflag default-ms
  (flagtype searchtype)
  (default ms90-3)
  (subjects mating-search . . .)
  (change-fn (lambda (flag value pvalue)
	       (when (neq value pvalue) (update-otherdefs value))))
  (irrelevant-kids ((neq default-ms 'ms98-1) '(ms98-num-of-dups)))
  (relevant-kids ((eq default-ms 'ms98-1) '(ms98-num-of-dups)))
  (mhelp . . . ))
\end{verbatim}
or
\begin{verbatim}
(defflag ms98-num-of-dups
  (default nil)
  (flagtype null-or-posinteger)
  (subjects ms98-1)
  (irrelevancy-preconditions (default-ms (neq default-ms 'ms98-1)))
  (relevancy-preconditions (default-ms (eq default-ms 'ms98-1)))
  (mhelp . . .))
\end{verbatim}

The conditions given are compiled into a relevancy graph and an irrelevancy graph.
The graphs are labelled directed graphs, where the nodes are flags and the
arcs are labelled by the conditions.
These graphs are currently used in the following two ways.
\begin{enumerate}
\item The relevancy graph is used by the {\bf \indexcommand{UPDATE-RELEVANT}} command.
The user specifies a flag to update.  Based on the value given for that flag,
the user is then asked to specify values for the target flags for which the condition
on the arc is true.  
For example, consider the following session:
\begin{verbatim}
<0>update-relevant default-ms
DEFAULT-MS [MS98-1]>ms98-1
. . . 
MS98-NUM-OF-DUPS [NIL]>2
. . .
<1>update-relevant default-ms
DEFAULT-MS [MS98-1]>ms90-3
. . .
<2>
\end{verbatim}
\item The irrelevancy graph is used to warn the user when an irrelevant flag
is being set.  A flag $F_0$ is {\it never irrelevant}
if there are no arcs with $F_0$ as the target.
A flag $F_1$ is {\it irrelevant} when there is a path
from a flag $F_0$ to $F_1$, where $F_0$ is never irrelevant and
at least one of the conditions on the path evaluates to true.  
Consider the following session.
\begin{verbatim}
<2>default-ms
DEFAULT-MS [MS90-3]>

<3>ms98-num-of-dups
MS98-NUM-OF-DUPS [NIL]>3
WARNING: The setting of the flag DEFAULT-MS makes
     the value of the flag MS98-NUM-OF-DUPS irrelevant.

<4>default-ms
DEFAULT-MS [MS90-3]>ms98-1

<5>ms98-num-of-dups
MS98-NUM-OF-DUPS [3]>2

<6>
\end{verbatim}
\end{enumerate}

\subsection{Automatically Generating Flag Relevancy}

In addition to the relevancy information directly specified
in the \verb+defflag+ declarations in the code, there
is now code (in \indexfile{flag-deps.lisp}) to read
and analyze the code in the lisp files to determine
flag relevancy.  This code first reads the lisp files
and records all \verb+defun+ and \verb+defmacro+ definitions.
Then, it computes easy flag conditions in which flags
and calls to other functions occur.  At present an {\it easy
flag condition} is built from atoms
of the form {\it easy-flag-term} {\it easy-operator} {\it easy-flag-term}
using boolean operations and \verb+IF+.
An {\it easy-operator} is one of =, <, >, <=, >=, eq, eql, equal, or neq.
An {\it easy-flag-term} is a flag, NIL, T, a number, or any quoted term.
The important property these conditions should satisfy is that their
values are static, i.e., they do not depend on the dynamic environment.
(Note that some flags, such as \indexflag{FIRST-ORDER-MODE-MS} are
often dynamically set by the code.  This flag, for example, is removed
from the list of flags, along with any flag, such as \indexflag{RIGHTMARGIN},
that is never relevant for automatic search.  The code does attempt to recognize
when flags are dynamically bound in a certain context and take this into
consideration.)

A user is given the option of computing this flag relevancy information
when calling \indexcommand{UPDATE-RELEVANT} or \indexcommand{SHOW-RELEVANCE-PATHS}.
Another option is to load the relevance information from a file.
Such a file could have been created in a previous \TPS session
using \indexcommand{SAVE-FLAG-RELEVANCY-INFO}.









\chapter{How to define wffs, abbrevs, etc}
\label{wffs}

There are several ways to define wffs so that they persist and
can be reused, saving typing and allowing them to be used to
extend the logical system.  For examples of how to represent
wffs, see sections 1.6 and 5.1.3 of the {\ETPS} User's Manual.

Abbreviations must usually be capitalized; for example, one should write
`ASSOCIATIVE p(AAA)' rather than `associative p(AAA)'. (Specifically, abbreviations
must be capitalized unless the \indexother{FO-SINGLE-SYMBOL} property is T. See chapter \ref{library}
for more details.) For reasons
to do with the internal workings of {\TPS}, please avoid using superscripts
on variables in wffs, and do not use the variables h or w in wffs or
abbreviations. In general, it is better (and certainly more readable)
to use lowercase letters for variables and uppercase for constants.

The first type of wff is good for wffs which you think of as static,
and for which you just want a short name.  These are called weak labels,
and can be created by the editor command \indexedop{CW}.  They may then be
stored in a file with the editor command \indexedop{SAVE}.  Other editor
commands allow you to delete the labels, replacing the short name
with its definition.  Weak labels are really weak.  When you wish to
refer to a weak label inside another wff, you must put the weak label
in all capitals, so that the parser will know how to find it when trying
to parse the wff.  In addition, operations like $\lambda$-contraction will
cause weak labels to be deleted in favor of their definitions.  In
short, weak labels are just a convenient shorthand for wffs and you
can use them to save typing.  They may also be saved in library files;
see chapter \ref{library}. In the editor, command names which end in `*'
are recursive; they will find the outermost appropriate node(s) and apply
the operation there. Non-recursive commands generally only apply to the current
node.

Some new wffs are really extensions of the system.  These denote operators
such as {\tt SUBSET}, {\tt INTERSECT}, etc.  Such wffs are called
abbreviations, and they can be made polymorphic, so that they can be
used for any types.  These wffs are more persistent than weak labels.
The parser will recognize them even if in lower case letters, and they
will be instantiated with their definitions only when specifically
required.  They may be saved in library files; see chapter \ref{library}
for more information.

Other wffs you might wish to save are exercises and theorems.  These
behave much like weak labels, but can have more properties.  At the
present time, these must be placed in files manually.  See the source
files {\tt ml1-theorems.lisp} and {\tt ml2-theorems.lisp} for how they
are defined.

Within the editor, you can record wffs as they are created as follows.
(By creating weak labels for the wffs you generate in the editor, you can
also save them in the library, which is more useful in that it allows you
to read them back into {\TPS} at a later date.)
If \indexflag{PRINTEDTFLAG} is T, the editor will every once in a while
write the edwff into a file \indexflag{PRINTEDTFILE} (global variable,
initialised to \indexfile{edt.mss}).  The criterion can be set as a lisp function, but at
the moment it will write whenever you call an edop whose result replaces
the current edwff.  Moving wffs (like A,D,L,R etc), non-editor commands
and printing command do not cause the
new edwff to be written.  Just so you can keep track of when things
are written, the prompt in the editor is modified.  Whenever
\indexflag{PRINTEDTFLAG} is T, it will either be
{\tt <-Edn>} or {\tt <+Edn>}
where the former means nothing has been written and the latter means the wff
which is now current has just been written. The wffs are appended to the end of
the file, which is written in style SCRIBE; also see the help messages
for the commands \indexcommand{O} and \indexcommand{REM}.

If you change the flag \indexflag{PRINTVPDFLAG} to T, it will print a
vertical path diagram into the file given by \indexflag{VPD-FILENAME} whenever
i
\chapter{Using the library}\label{library}
A library facility exists, and can be accessed by
entering the library top-level via the \indexcommand{LIB} command
or by entering the Unix-style library top-level via the \indexcommand{UNIXLIB}.
The library provides a way of storing various types of objects
(e.g., gwff, abbreviation, etc.) in files, which are collected in directories.
A library directory is a directory of the file system containing an index file whose
name is determined by the flag \indexflag{LIB-MASTERINDEX-FILE}.
(The default value for \indexflag{LIB-MASTERINDEX-FILE} is currently
{\tt `libindex.rec'}, so that a library directory is a directory containing
a file named {\tt `libindex.rec'}.)  This index file associates the objects
stored in the directory with the type of the object and the file in which
the object is stored.
{\TPS} automatically maintains the index file as objects are inserted, deleted, renamed, etc.
Also, index files are affected when library files are manipulated.
The objects in the library can be classified into classification schemes.
Classification schemes themselves are objects which can be stored in the
library.

The user has the option of using multiple libraries at the same time. The names
of these libraries are
stored in the flags \indexflag{DEFAULT-LIB-DIR} (for the libraries which the user can both read
and write) and \indexflag{BACKUP-LIB-DIR} (for the libraries which the user can read only).
It is a good idea to have a line such as
{\tt (set-flag default-lib-dir '(`/afs/cs/project/tps/tps/tpslib/andrews/'))}
in your {\it tps3.ini} file, to save having to type this every time {\TPS} is loaded.
Note the {\tt `/'} at the end of the string!
Changing the value of this flag will make {\TPS} reload the library index files.
(Note that including a directory in the value of these flags does not
create the library directory.  See the next paragraph.)
If \indexflag{ADD-SUBDIRECTORIES}
is T, then at the same time {\TPS} will look for any subdirectories of the library directories which are
in themselves library directories, and add them to \indexflag{DEFAULT-LIB-DIR} or \indexflag{BACKUP-LIB-DIR} as appropriate.
Note that \indexflag{DEFAULT-LIB-DIR} and \indexflag{BACKUP-LIB-DIR}
are flags.  As such their values can be explicitly changed by the user to any list
of library directories.

The TPS distribution includes a library directory called
`distributed'.  This directory contains a variety of library
objects which have been defined over the years.
To have access to this directory, the user should include
the {\tt `/whatever/tps/library/distributed'} directory in the value of
\indexflag{BACKUP-LIB-DIR}.

Library directories can be created using \indexcommand{CREATE-LIB-DIR}, and
library subdirectories can be created using \indexcommand{CREATE-LIB-SUBDIR}.
Library subdirectories may be used to allow overloading of names for library objects.
For example, one could have two different abbreviations {\tt GROUP} with different
definitions as long as they are stored in different library directories.
It is illegal to have two different library objects with the same name and type in the same
library directory.  (WARNING:  Not all TPS library commands currently enforce this.)
The intended use of BACKUP-LIB-DIR is, of course, to make it possible for a group of
users to have a common library of useful definitions in addition to their own private library
workspace. If there is no such common library, set BACKUP-LIB-DIR to NIL.

When {\TPS} starts up,
it creates a master index by loading the index files in the library directories listed
in DEFAULT-LIB-DIR and BACKUP-LIB-DIR.  This master index is recreated by certain
commands involving library directories (such as \indexcommand{CREATE-LIB-DIR}) or when either
DEFAULT-LIB-DIR
or BACKUP-LIB-DIR is changed.  The master index can be explicitly recreated by
the command \indexcommand{RESTORE-MASTERINDEX}.

Within the library top-level, the user can
\begin{itemize}
\item save and retrieve {\TPS} objects. {\tt HELP LIB-ARGTYPE} provides a list of types of objects
that can be saved in the library. When saving objects in the library, the user
will be prompted for any attributes that are required. This includes other
library objects that must be retrieved before the object that is currently
being saved is retrieved.

\item modify existing library objects (by using the \indexcommand{INSERT}
command with the name of an existing object)

\item display objects, parts of objects, or even entire files from the library, and produce lists of
all the files in the directory, all the objects in a file, or all the objects in a directory.

\item output this information in a format suitable for processing and printing by Scribe or TeX.

\item perform simple file maintenance tasks such as copying, deleting and moving objects
(or entire files or directories), or listing all the files in a directory.

\item reformat library files. For sake of efficiency, when modifying existing
objects in the library, no formatting is done for the objects that
follow or precede the object that is being modified. Hence, over a period
of time, the library files may be in a form that you may find difficult to
read. At such times you may want to reformat the files, so that they are
in a more readable form.
\end{itemize}

\section{Storing and Retrieving Objects}

All of the following types of object can be stored in the library: gwff, abbreviation, constant,
mode, mode1, rewrite rule, theory, searchlist and dpairset. Some of these require extra input from the
user, and are discussed in more detail here; the others are discussed elsewhere in this manual,
and once defined can be saved in the library without extra input. (For example, dpairsets are discussed
in the section about the unification top level, and searchlists in the section about the test top level.)

A list of keywords is stored in the library directory in the file specified by \indexflag{LIB-KEYWORD-FILE}.
An arbitrary list of keywords can be attached to each library object; these keywords can then be used for
searches using \indexcommand{KEY} or \indexcommand{SHOW-ALL-WFFS}. The keyword list is user-specifiable,
using the \indexcommand{ADD-KEYWORD} and \indexcommand{SHOW-KEYWORDS} commands.  For more information
on keywords, see section \ref{KEYWORDS}.

The library also stores a file {\it bestmodes.rec} which associates theorems to modes in which those theorems
can be proven automatically. All of the {\it bestmodes.rec} files in any library directory are available
simultaneously. The commands \indexcommand{FIND-MODE} and \indexcommand{SHOW-BESTMODE} search this
file, and new modes can be inserted either during a call to \indexcommand{DATEREC}, or by using the command
\indexcommand{ADD-BESTMODE}. The \indexcommand{MODEREC} command invokes ADD-BESTMODE with the name of the
most recently proven theorem and mode.

Assume that the user wishes to define his own abbreviation for an {\tt EMPTYSET}.
He should enter the library top-level, and use the command
\indexcommand{INSERT}. He'll then be prompted for various attributes that are
necessary. As usual, \indexcommand{?} will provide some help on the
appropriate responses.
In this session, and in all subsequent sessions {\TPS} will recognize
{\tt EMPTYSET} as a library object. When the user needs to use the
library object {\tt EMPTYSET}, he should first use the library command
\indexcommand{FETCH} to make this library object available within {\TPS}.
If objects in several directories have the same name, and \indexflag{SHOW-ALL-LIBOBJECTS} is set to T,
the user will be asked which one to FETCH. Otherwise, the directories are taken in the order given in
DEFAULT-LIB-DIR, and then BACKUP-LIB-DIR, and the first directory containing such an object will be
used.

A mode can be stored in the library. This is a list of flag settings (for convenience,
there is also an object called a \indexother{MODE1}; this consists of
all the flags belonging to a given list of subjects, and this may also be stored
in the library). Users who discover that a particular collection of flag settings
produces a fast proof of a theorem may find it useful to record these settings
as a mode in their library. See chapter \ref{flags} for more information about
modes.

When entering abbreviations, if the \indexother{FO-SINGLE-SYMBOL} property is
T, you can specify the abbreviation in uppercase, lowercase,  or any
combination of these. Otherwise it has to be specified as an uppercase
symbol only.

The \indexother{FACE} property controls how the abbreviation will be printed by {\TPS}.
It should be a list of symbols (which may include special printing
symbols such as POWERSET); type ? at the prompt for more information.

Also, the correct format for typelists in abbreviations entered in
the library is that illustrated by {\tt (`A' `B')}.  The user who types
{\tt ?} when prompted for the typelist will see the example. See the programmer's
manual for more information about defining abbreviations, logical constants, etc.

A library object is allowed to depend on other library objects, called needed-objects
(for example, one might define an abbreviation {\tt BIJECTIVE} in terms of
other abbreviations {\tt INJECTIVE} and {\tt SURJECTIVE}). {\TPS} will attempt
to type-check all objects it inserts into the library, and for this reason all definitions
must be made from the bottom up (in our example, {\tt INJECTIVE} and {\tt SURJECTIVE}
would have to already exist at the time when we attempt to define {\tt BIJECTIVE}).
When a definition is {\tt FETCH}ed, all the needed-objects will be retrieved with it;
an error will be signalled if they cannot be found.  Needed objects will be loaded
from the same directory as the original object if this is possible, and otherwise from the first
place in which they are found.
Definitions may be nested arbitrarily deeply, and needed-objects are assumed to
be abbreviations, unless no abbreviation of the right name can be
found in the library, in which case they're assumed to be gwffs.
The command \indexcommand{CHECK-NEEDED-OBJECTS} can be used to check if a library
directory is closed with respect to needed-objects.  The command
\indexcommand{IMPORT-NEEDED-OBJECTS}
can be used to copy any extra needed-objects into a directory (so the directory
will be closed with respect to needed-objects).

Gwffs stored in the library are also equipped with a PROVABILITY property, which
indicates the current state of attempts to prove them in {\TPS}. This property can only be changed by the user
({\TPS} will never automatically change it), using the commands \indexcommand{DATEREC} or
\indexcommand{CHANGE-PROVABILITY}. The command \indexcommand{FIND-PROVABLE} looks for all gwffs
with a specified provability status.

The command \indexcommand{RETRIEVE-FILE} will load all the objects in a given library file.

There are various commands in other top levels that affect the library. For example,
\indexcommand{DATEREC} stores timing information, and the \indexcommand{TEST} top level
has its own commands for saving and loading searchlists. Also the commands \indexcommand{BUG-SAVE},
\indexcommand{BUG-RESTORE}, and their associated commands, create a separate library of
bugs in the directory given by \indexflag{DEFAULT-BUG-DIR} (although these commands can also
save bugs to your personal library if you prefer).

\section{Displaying Objects}
The \indexcommand{SHOW} command displays a single object from the library.
Library objects can be very long (particularly if the object is a gwff, and
\indexcommand{DATEREC} has
been used to append information regarding attempts at proving the gwff), and
there are commands \indexcommand{SHOW-WFF}, \indexcommand{SHOW-MHELP} and
\indexcommand{SHOW-WFF\&HELP} which display only a part of a stored object.

The user can also display the names of all the objects in a file, using
\indexcommand{LIBOBJECTS-IN-FILE}, or all the wffs in a file with
\indexcommand{SHOW-WFFS-IN-FILE}.
The analogues of these commands for the entire directory, rather than just one
file, are \indexcommand{LIST-OF-LIBOBJECTS}
and \indexcommand{SHOW-ALL-WFFS}; the latter can take some time.

To find an object whose name you only partly remember, use the command \indexcommand{KEY}.
For example, {\tt KEY `THM135' !} will list all objects, of any type, in the current and
backup directories, whose names contain
the string `THM135'. The \indexcommand{SEARCH} and \indexcommand{SEARCH2} commands do similar
things, but search the entire text of library objects rather than just their names.

\section{File Maintenance}
Library directories and subdirectories can be created using \indexcommand{CREATE-LIB-DIR} and
\indexcommand{CREATE-LIB-SUBDIR}, deleted using
\indexcommand{DELETE-LIB-DIR}, copied using \indexcommand{COPY-LIBDIR}, and
renamed using \indexcommand{RENAME-LIBDIR}.  The command \indexcommand{CREATE-LIB-DIR}
will add the new library directory to DEFAULT-LIB-DIR.
Also, \indexcommand{DELETE-LIB-DIR}
will delete the directory from DEFAULT-LIB-DIR, if it was an entry in this list.
(All changes to DEFAULT-LIB-DIR only apply during the current {\TPS} session.)
The command \indexcommand{COPY-LIBDIR} can be used in two ways.  First, it can
be used to create a new library directory (which will be added to DEFAULT-LIB-DIR)
and copy the contents of an existing directory into this new directory.  Second,
it can be used to copy the contents of an existing directory into an existing directory.
In this second case, if an object with the same name and type exists already in both
the source and destination directories, the object will not be copied.  Instead, the
original object in the destination directory will be kept.  Also,
\indexcommand{COPY-LIBDIR} will copy the bestmodes and keywords information files
from the source directory to the target directory.  If the target directory
already contains a bestmodes or keywords information file, \indexcommand{COPY-LIBDIR}
will merge the information from both directories.

A `common' library directory containing a copy of all library objects
in every directory in \indexflag{DEFAULT-LIB-DIR} and \indexflag{BACKUP-LIB-DIR}
can be created and maintained using the command \indexcommand{UPDATE-LIBDIR}.
\indexcommand{UPDATE-LIBDIR} has the same effect as calling \indexcommand{COPY-LIBDIR}
on each directory in \indexflag{DEFAULT-LIB-DIR} and \indexflag{BACKUP-LIB-DIR}.

Library files can be created implicitly by \indexcommand{INSERT} when a new object is placed in
the file.  Library files can be deleted implicitly when the last object stored in
the file is deleted using \indexcommand{DELETE} or moved into a different file
using \indexcommand{MOVE-LIBOBJECT}.  Library files can also be deleted explicitly, along
with all the objects stored in the file, using \indexcommand{DELETE-LIBFILE}.  Files
can be renamed (within the same directory) using \indexcommand{RENAME-LIBFILE},
moved (into a different directory) using \indexcommand{MOVE-LIBFILE}, and
copied (into a different directory) using \indexcommand{COPY-LIBFILE}.

As discussed above, new objects can be inserted into the library using \indexcommand{INSERT}.
The \indexcommand{INSERT} command can also be used to modify an existing library object.
A library object can be deleted using \indexcommand{DELETE}, renamed (within the same library
file) using \indexcommand{RENAME-OBJECT}, and moved (into a new file and possibly new directory)
using \indexcommand{MOVE-LIBOBJECT}.  (In fact, \indexcommand{MOVE-LIBOBJECT} can be used
more generally to move several objects of the same type at once.)
Users are encouraged to have many small library files, rather than a few large
files, as this makes many of the library commands much faster. The command
\indexcommand{MOVE-LIBOBJECT} can be used to break up a large file into several smaller files,
if need be.
It is a (fairly) general principle that if \indexflag{SHOW-ALL-LIBOBJECTS} is set to T,
and more than one object of a given name (and type if the type is specified) is found,
then the user is prompted to choose one.

The command
\indexcommand{LIBFILES} lists all the files referred to in the directories
listed in DEFAULT-LIB-DIR, the directories listed in BACKUP-LIB-DIR, all the
directories listed in DEFAULT-LIB-DIR or BACKUP-LIB-DIR, or a single directory
chosen by the user.

The \indexcommand{REFORMAT} command reads in a whole file and writes it out again;
this is useful if you have manually edited it and it's become a bit messy.
The \indexcommand{SORT} command puts a file into alphabetical order.

Finally, the command \indexfile{SPRING-CLEAN} will do any or all of: delete non-library
files in your library directory, reindex all library files in that directory, reformat
all library files in that directory, and sort all library files in that directory.

\section{Printed Output}
The commands \indexcommand{SCRIBE-LIBFILE} and \indexcommand{TEX-LIBFILE}
print the contents of a library file (or a list of library files)
in a form suitable for Scribe or TeX.
The commands prompt for the required degree of verbosity; the various options
for this are described in the help messages for the commands.

\section{Expert Users}
Expert users (i.e. those with \indexflag{EXPERTFLAG} set to T)
are allowed to use library gwffs and
abbreviations in proofs, by using the \indexcommand{ASSERT}
command; they may also instantiate gwffs and
abbreviations while in the editor.

\section{Keywords}\label{KEYWORDS}
Library objects may have associated keywords.  Some keywords
are automatically generated when the objects is created.
Examples of such keywords are PROVEN, UNPROVEN,
WITH-EQUALITY, and WITHOUT-EQUALITY.  The user can create a
new keyword using \indexcommand{ADD-KEYWORD}.  The command
\indexcommand{SHOW-KEYWORDS}
shows a list of acceptable keywords
with help messages.
The keywords
associated with an object can be changed using
\indexcommand{CHANGE-KEYWORDS}.
The command \indexcommand{UPDATE-KEYWORDS}
updates the keywords field to include all of those
keywords that can be determined automatically (leaving
all other keywords untouched).

The keywords associated with an object are printed by
the commands \indexcommand{SHOW} and \indexcommand{SHOW-WFF\&HELP}.
Keywords of objects are also printed by the commands
\indexcommand{TEXLIBFILE}, \indexcommand{SCRIBELIBFILE},
\indexcommand{TEX-ALL-WFFS}, and \indexcommand{SCRIBE-ALL-WFFS},
as long as the verbosity setting is MED or MAX.  The commands
\indexcommand{TEX-ALL-WFFS} and \indexcommand{SCRIBE-ALL-WFFS}
also use keywords
as a filter so the user to select certain classes of gwffs.
The user can also use keywords to find certain objects
using the commands \indexcommand{SEARCH} and \indexcommand{SEARCH2}.

A file containing keywords and help messages for keywords is
stored in each library directory.  The name of this file is
determined by the flag \indexflag{LIB-KEYWORD-FILE}.

\section{Classification Schemes}

A `classification scheme' for the library is a directed acyclic
graph of `classes'.  The graph has a root class with the same
name as the \indexother{classification scheme}.  Each class other than the
root class has a primary parent.  There may also be other parents
of the class.  Each class may have child classes.  Each item in
the library can be associated with multiple classes.

More than one class can have the same name.  Any particular
path can only be uniquely identified by a full path from
the root.  Similarly, the name of a library item does not
uniquely identify the item.  It is, in fact, common practice
for different files in the library to contain separate copies
of theorems and abbreviations.  Usually the definitions of
the items are the same, but the `other-remarks' property
often differs.  Several library items with the same name
can be classified in the same class.  The user is asked to
disambiguate when necessary.

The value of the flag \indexflag{CLASS-SCHEME} is the name of
the current classification scheme.  Classification schemes
can be stored in and retrieved from the library in the usual
way (via the library commands \indexcommand{INSERT} and \indexcommand{FETCH})
as items of type CLASS-SCHEME.
To find out what classification schemes are in the library,
use the command \indexcommand{LIST-OF-LIBOBJECTS} with type CLASS-SCHEME
as an argument.
The top-level command \indexcommand{PSCHEMES} lists the classification
schemes currently in memory.

Once \indexflag{CLASS-SCHEME} is set, there is always a
`current class'.  One can change the current class using
the library commands \indexflag{GOTO-CLASS} and \indexflag{ROOT-CLASS}.

In the library top-level, the following commands can be used
to modify and use classification schemes.

\begin{itemize}
\item {\bf CREATE-CLASS-SCHEME}  Creates a new classification scheme
which can be the value of \indexcommand{CLASS-SCHEME}
and can be stored and fetched from the library.

\item {\bf CREATE-LIBCLASS}  Creates a new class in the classification scheme.

\item {\bf ROOT-CLASS}  Makes the root class the current class.

\item {\bf CLASSIFY-CLASS}  Classify one class as the child of another.

\item {\bf UNCLASSIFY-CLASS}  Remove a class as a child of another.

\item {\bf CLASSIFY-ITEM}  Classify a library item in a class.

\item {\bf UNCLASSIFY-ITEM}  Remove a library item from a class.

\item {\bf FETCH-LIBCLASS}  Fetches all the library items in a class.

\item {\bf FETCH-DOWN}  Fetches all the library items in a class
as well as those classified in descendent classes.

\item {\bf FETCH-UP}  Fetches all the library items in a class
as well as those classified in ancestor classes.

\item {\bf FETCH-LIBCLASS*}  This behaves as FETCH-UP or FETCH-DOWN
depending on the value of the flag \indexflag{CLASS-DIRECTION}.

\item {\bf PCLASS}  Prints information (parents, children,
and classified library items) about the current class.

\item {\bf PCLASS-SCHEME-TREE}  Print the classification scheme as a tree
starting from the root.

\item {\bf PCLASS-TREE}  Print the classification scheme as a tree starting
from the current class.
\end{itemize}

\section{The Unix-style Library Top Level}

The command \indexcommand{UNIXLIB} can be used to enter a top level
that uses a Unix-style interface to access the TPS library.
The classification scheme named by the value of the flag \indexflag{CLASS-SCHEME}
is used to create a virtual directory structure.  Many of the commands one
can use at this top level correspond to Unix commands:

\begin{itemize}
\item {\bf ls}  Lists the child classes (as subdirectories) of the current class.

\item {\bf cd}  Changes the current class.

\item {\bf pwd}  Prints the full path to the current class.
This full path is shown in the prompt if the flag \indexflag{UNIXLIB-SHOWPATH}
is set to T.

\item {\bf mkdir}  Creates a new class as a child of the current class.

\item {\bf ln}  Links a class to be a child of another class.  This is a
command that enables the user to create a class with several parents.

\item {\bf cp}  Copies library items classified in one class to be classified
under another class as well.  The \indexcommand{cp} command can be
used in two ways.  If the user specifies an item to copy, that
particular item is copied.  If the user specifies a class to copy from,
all the items classified in that class are copied.

\item {\bf rm}  Removes a child class or library item from the current class.

\end{itemize}

To specify a class or item, one can use the Unix-style path notation,
for example, /a/b, b/c, ../a/b, etc.

Some library commands (e.g., \indexcommand{fetch}, \indexcommand{show})
are also commands in the UNIXLIB top-level.  These commands
in the UNIXLIB top-level use the current class to determine where
to look for library items.

\section{Cautions}
The user should note a distinction between a library object and a {\TPS}
object. A {\TPS} object is represented in a form most congenial to internal
manipulation, while a library object is represented in a form that
is easy to specify, and these two forms may not be the same. The library
operation \indexcommand{FETCH} takes a library object, and makes it
available within {\TPS}{by converting it to the appropriate internal form.}

Once an object has been loaded from the library and turned into a {\TPS} object,
the \indexcommand{FETCH} command will query any attempt to load another library object
of the same name. A previously loaded object can be removed from {\TPS} by using the \indexcommand{DESTROY}
command.

\section{How to insert TPTP Problems into the \TPS~Library}

The TPTP Problem Library for Automated Theorem Proving can be found at
http://www.cs.miami.edu/~tptp/.
TPTP Problems can be converted into \TPS~library items thanks to the utility TPTP2X, available with the TPTP Library. You need to have a Prolog interpreter installed prior to using the TPTP2X
utility. SWIPL for instance.\\

Important note: versions prior to SWIPL 5.10.1 may cause some TPTP Problems
not to be processed by TPTP2X. SWIPL 5.8.2 could not handle the following
ones: CSR130\textasciicircum 1, CSR144\textasciicircum 1,
CSR153\textasciicircum 1 and SYN000\textasciicircum 2.

\begin{enumerate}

\item Download the latest version of TPTP Library, including the TPTP2X utility.

\item Once extracted, install TPTP2X, using the tptp2X\_install script (in
Prolog).Example:
\begin{alltt}
whatever/TPTP-v5.0.0/TPTP2X\% ./tptp2X\_install
\end{alltt}

Follow the instructions.

\item In the TPTP2X directory, replace the original {\it format.tps} by the one
avaibable in the {\it utilities} directory. Example:
\begin{alltt}
mv format.tps format.tps.old
mv /afs/andrew.cmu.edu/mcs/math/TPS/utilities/format.tps whatever/TPTP-v5.0.0/TPTP2X/
\end{alltt}

\item Use TPTP2X on THF problems, whose name have the following structure:
DOM***\textasciicircum *.p.
To convert a specific Problem, use the syntax:
\begin{alltt}
./tptp2X -f tps <the problem you need> \# e.g. ALG268\textasciicircum 4, whitout quotes
\end{alltt}
To convert every THF problem, you can use:
\begin{alltt}
./tptp2X -f tps "whatever/TPTP-v5.0.0/Problems/**/*\textasciicircum* .p" \# Note the quotes
\end{alltt}

The converted files will be in {\it whatever/TPTP-v5.0.0/TPTP2X/tps}.

\item Create the needed library (sub)directories to receive the converted
TPTP theorems. You can either use the CREATE-LIB-SUBDIR command, or manually
create the directories, an empty {\it libindex.rec} file and a copied
{\it keywords.rec} file.

\item In your {\it tps3.ini} file, or directly inside of a \TPS~instance, set the flag
\indexflag{AUTO-LIB-DIR} to the destination library directory. For instance, if you want
to convert the ALG Problems of TPTP, you may set the flag to
{\it your-lib-directory/ALG/}.\\
Example:
\begin{alltt}
(set-flag 'auto-lib-dir
"/afs/andrew.cmu.edu/mcs/math/TPS/tpslib/chretien/tptp/ALG/")
\end{alltt}
\item In \TPS , use the library command \indexcommand{INSERT-TPTP*} to automatically convert a
directory of .tps files (converted TPTP problems) into TPS library
items. Example:
\begin{alltt}
insert-tptp* "whatever/TPTP-v5.0.0/TPTP2X/tps/ALG"
\end{alltt}
You can also use the \indexcommand{INSERT-TPTP} command to insert only one problem at a
time. Example:
\begin{alltt}
insert-tptp "whatever/TPTP-v5.0.0/TPTP2X/tps/ALG/ALG268\textasciicircum 4.tps" "ALG2684.lib"
"ALG2684"
\end{alltt}
The last argument provides a suffix added to every abbreviation or theorem, in
order to prevent any conflit with buil-in \TPS~items.

\item The newly created items are usually called {\it con-<suffix>} (for conjecture)
or {\it thm-<suffix>}, the suffix being the last argument of INSERT-TPTP or the
name of the library file, if you used INSERT-TPTP*.
\end{enumerate}

\chapter{Mating Searches}\label{ms-guide}

\section{Expansion trees and how they grow}
See the theses and papers by Miller and Pfenning.

\section{The MATE Top-Level}

Inside the mating-search top-level, you
may examine the expansion tree, apply substitutions for variables,
etc.  If you want to actually search for a mating, use the
\indexcommand{GO} mating-search command (Note that this is a different
{\tt GO} command from the one in the main top-level. This one works only from
the mate top-level.). You can also search for a mating by typing the name
of the mating-search you wish to use.



The algorithm used by {\TPS} to search for matings can be altered by changing
the setting of the flag \indexflag{DEFAULT-MS}. At present, there are eight
possible settings for this flag, not including the matingstree procedures described in
chapter \ref{mtree}:

\begin{description}
\item[] \indexother{MS88} is Andrews' original
search procedure, as detailed in \cite{Andrews81}, which exhausts all paths through
a jform before duplicating the outermost quantifiers and trying again. MS88 will
apply primitive substitutions to a very limited extent.

\item[] \indexother{MS89} is like MS88, but will apply a variety of primitive substitutions
and duplications, and will work on several variants of
the jform simultaneously in the `option tree' style.

\item[] \indexother{MS90-3} uses Issar's `path-focused duplication' procedure to search
a single jform. See \cite{issar90} for further details.

\item[] \indexother{MS90-9} is like MS90-3, but will apply a variety of primitive substitutions
and duplications, and will work on several variants of the jform simultaneously in
the `option tree' style.

\item[] \indexother{MS91-6} is a variant of MS89, using `option sets' rather than `option trees',
which allows the user more control over the
order in which the variant jforms are considered. See section \ref{ms91} for further details.

\item[] \indexother{MS91-7} is a variant of MS90-9, using `option sets' rather than `option trees',
which allows the user more control over the
order in which the variant jforms are considered. See section \ref{ms91} for further details.

\item[] \indexother{MS92-9} is a simulation of MS88, using the code from MS90-3.

\item[] \indexother{MS93-1} is a simulation of MS89, using the code from MS90-9.

\item[] \indexother{MS98-1} is Matt Bishop's Search Procedure, Component Search.
See Matt Bishop's thesis for details.

\end{description}

Each of these flag settings is also a command in its own right, and in the mating-search
top level any of these mating searches may be invoked by simply entering its name.
When using \indexcommand{DIY} or \indexcommand{GO}, however, the mating search that
will be used is that given by \indexflag{DEFAULT-MS}.

More information on (for example) the MS88 procedure may be obtained by entering
{\it help ms88}.
Information comparing the different searches can be found in the help message for
\indexflag{DEFAULT-MS}.

Each of these procedures is governed by a number of other flags; for example, in
some of the procedures the flag \indexflag{NUM-OF-DUPS} controls the maximum number
of duplications allowed. Each of the above flag settings is also a subject heading,
and hence a full list of all the flags associated with (for example) MS88 may be obtained
by typing {\tt list ms88}.

The user may also wish to examine the help message for the flag
\indexflag{QUERY-USER}; this flag allows some interactive control over the
automatic search procedures, by letting the user specify which vpforms to ignore
and which to search on, when to duplicate quantifiers, and so on.

\section{Primitive Substitutions}

There are two distinct procedures relating to primitive substitutions in {\TPS}, and they
interact. The first of these is the actual generation of the substitutions; this is
principally governed by the \indexflag{PRIMSUB-METHOD} flag and its related flags;
the second is the way in which they are chosen and combined in the various mating searches.

Primsub generation is dealt with in the next section; it is done in exactly the same manner
for every mating search, and also for the \indexcommand{NAME-PRIM} commands in the mate and
editor top levels. Once a hashtable of primsubs has been generated, it will remain in memory
until something is done that would change the substitutions available (for example, starting
a new mating search, or changing one of the flags in the subject PRIMSUBS)

Primsub use is considerably more complex, and is dealt with differently by different mating searches.
Given the hashtable of primsubs, which will be the same for all mating searches, it is up to the
individual mating search to decide how to use this information. In general, there are three approaches
to this: do nothing, use option trees, and use option sets.

MS88 does nothing. It has a few very basic primitive substitutions built-in, and uses these instead of
the table of primsubs. MS90-3 also does nothing.

MS89, MS90-9 and MS93-1 all use option trees. In this case, a tree of substitutions is built;
the root is empty, and at each node one branch is added for each possible substitution (it helps at
this point to consider a quantifier duplication as a vacuous substitution). The tree is of course
constructed as required, since it will probably be of infinite depth. The mating search then has a fixed
amount of time to consider the vpform corresponding to each node in the tree, which it does breadth first.
So, for example, it will start by considering the vpform with no substitutions, and then proceed to
the vpform with the first available substitution, then the second, and so on. When it runs out of substitutions,
it may then go on to consider the vpform with two copies of the first substitution, then with one of the first
and one of the second, and so on. This will continue until a proof is found; since there are always more
combinations of substitutions, there is no hope of halting without a proof.

MS91-6 and MS91-7 use option sets, which are multisets of substitutions generated from the original table of primsubs.
Again, since there is no {\it a priori} limit on the size of such a multiset, there are potentially infinitely
many of them, so only a few are generated at any time. In comparison to the option tree method, which gives the
user very little control over
the order in which substitutions are considered, option sets allow the user to specify how to go about
generating the sets, and in what order to consider them. First each substitution, and then each set,
is assigned a `weight'; the search proceeds
similarly to the option-tree searches, except that instead of searching a tree breadth-first, {\TPS} instead
considers the available sets of substitutions in order from `lightest' to `heaviest'.
Although the initial hashtable of substitutions is exactly the same as for all the other mating searches,
by giving certain substitutions or combinations of substitutions `infinite' weight the user can
effectively prevent their ever being considered. Option sets are rather more difficult to use than
option trees, because the user has to specify how this weighting is to be done; section \ref{ms91}
discusses this in some detail.

\subsection{How Primsubs are Generated}

In most mating-search procedures, {\TPS} will attempt to make some substitutions
where appropriate. There are three ways in which these substitutions can be generated;
which is used will depend on the setting of the \indexflag{PRIMSUB-METHOD} (MS88 is an exception
to this general rule; it has an extremely limited set of predefined primitive substitutions,
and does not use \indexflag{PRIMSUB-METHOD}. The results will
be as follows (examples are given for X5305 with \indexflag{NEG-PRIMSUBS} NIL and
\indexflag{PRIM-BDTYPES} (`I')):

\indexother{PR89} is the original method of generating primsubs, first written for MS89.
It generates first basic substitutions: a conjunction of two literals and a disjunction
of two literals. It will also generate a projection, if the types are appropriate, and a
negation, if \indexflag{NEG-PRIMSUB} is T. Finally it generates the simplest possible
quantified substitutions: a universally quantified single literal and and existentially
quantified single literal. The types of the quantified variables, will be determined
by \indexflag{PRIM-BDTYPES}, and two such substitutions will be generated for each
bound type listed in \indexflag{PRIM-BDTYPES}.
The setting of \indexflag{PRIM-BDTYPES} may itself be determined by the setting of
\indexflag{PRIM-BDTYPES-AUTO}; read the help message for this flag for more details. Example:

\begin{tpsexample}
Var:   \(f\sp{{2}}\sb{o\alpha}\)

     PRIM1   LogConst     \(\lambda w \sp{{1}}\sb{\greeka} .f \sp{{1}}\sb{\greeko\greeka}  w \sp{{1}}  \and f \sp{{2}}\sb{\greeko\greeka}  w \sp{{1}}\)
     PRIM2   LogConst     \(\lambda w \sp{{1}}\sb{\greeka} .f \sp{{3}}\sb{\greeko\greeka}  w \sp{{1}}  \lor f \sp{{4}}\sb{\greeko\greeka}  w \sp{{1}}\)
     PRIM3   PrimQuant   \(\lambda w \sp{{1}}\sb{\greeka}   \exists w \sp{{2}}\sb{\greeki}  f \sp{{5}}\sb{\greeko\greeki\greeka}  w \sp{{1}}  w \sp{{2}}\)
     PRIM4   PrimQuant   \(\lambda w \sp{{1}}\sb{\greeka}   \forall w \sp{{2}}\sb{\greeki}  f \sp{{6}}\sb{\greeko\greeki\greeka}  w \sp{{1}}  w \sp{{2}}\)
\end{tpsexample}

\indexother{PR93} extends the above method to more general substitutions.
All of the non-quantified substitutions are generated as before, and then {\TPS}
consults the flags \indexflag{MIN-PRIM-DEPTH} and \indexflag{MAX-PRIM-DEPTH}, which
contain integers, and generates substitutions for each depth in the given range.
At depth 1, quantified substitutions are generated as in PR89.
At depth N>1, a substitution containing (N-1) quantifiers ranging over (N-1) conjunctions
(respectively, disjunctions) of (N-2) disjunctions (respectively, conjunctions). Again, these are generated
at every bound type listed in \indexflag{PRIM-BDTYPES}. Example (\indexflag{MAX-PRIM-DEPTH} 2,
\indexflag{MIN-PRIM-DEPTH} 1):
\begin{tpsexample}
 Var:   \(f\sp{{2}}\sb{\greeko\greeka}\)

     PRIM5   LogConst     \(\lambda w\sp{{3}}\sb{\greeka} .f \sp{{7}}\sb{\greeko\greeka}  w \sp{{3}}  \and f \sp{{8}}\sb{\greeko\greeka}  w \sp{{3}}\)
     PRIM6   LogConst     \(\lambda w \sp{{3}}\sb{\greeka} .f \sp{{9}}\sb{\greeko\greeka}  w \sp{{3}}  \lor f \sp{{10}}\sb{\greeko\greeka}  w \sp{{3}}\)
     PRIM7   PrimQuant   \(\lambda w \sp{{3}}\sb{\greeka}   \exists w \sp{{4}}\sb{\greeki}  f \sp{{11}}\sb{\greeko\greeki\greeka}  w \sp{{3}}  w \sp{{4}}\)
     PRIM8   PrimQuant   \(\lambda w \sp{{3}}\sb{\greeka}   \forall w \sp{{4}}\sb{\greeki}  f \sp{{12}}\sb{\greeko\greeki\greeka}  w \sp{{3}}  w \sp{{4}}\)
     PRIM9   GenSub2   \(\lambda w \sp{{3}}\sb{\greeka}   \exists w \sp{{5}}\sb{\greeki} .f \sp{{13}}\sb{\greeko\greeki\greeka}  w \sp{{3}}  w \sp{{5}}  \lor f \sp{{14}}\sb{\greeko\greeki\greeka}  w \sp{{3}}  w \sp{{5}}\)
     PRIM10  GenSub2   \(\lambda w \sp{{3}}\sb{\greeka}   \exists w \sp{{5}}\sb{\greeki} .f \sp{{15}}\sb{\greeko\greeki\greeka}  w \sp{{3}}  w \sp{{5}}  \and f \sp{{16}}\sb{\greeko\greeki\greeka}  w \sp{{3}}  w \sp{{5}}\)
     PRIM11  GenSub2   \(\lambda w \sp{{3}}\sb{\greeka}   \forall w \sp{{5}}\sb{\greeki} .f \sp{{17}}\sb{\greeko\greeki\greeka}  w \sp{{3}}  w \sp{{5}}  \lor f \sp{{18}}\sb{\greeko\greeki\greeka}  w \sp{{3}}  w \sp{{5}}\)
     PRIM12  GenSub2   \( \lambda w \sp{{3}}\sb{\greeka}   \forall w \sp{{5}}\sb{\greeki} .f \sp{{19}}\sb{\greeko\greeki\greeka}  w \sp{{3}}  w \sp{{5}}  \and f \sp{{20}}\sb{\greeko\greeki\greeka}  w \sp{{3}}  w \sp{{5}}\)
\end{tpsexample}

\indexother{PR95} generates more substitutions than PR93, but is more economical
in terms of the number of literals in the resulting jform. It works as for PR93,
except that at depth N>1 it then examines the flags \indexflag{MIN-PRIM-LITS}
and \indexflag{MAX-PRIM-LITS}; for each M in the range given by these two flags,
it generates all possible arrangements of M literals separated by conjunctions and
disjunctions, and then quantifies each of them with (N-1) quantifiers. The bound
types are determined by \indexflag{PRIM-BDTYPES}. Example (\indexflag{MIN-PRIM-LITS} 2,
\indexflag{MAX-PRIM-LITS} 3):
\begin{tpsexample}
     PRIM13   LogConst     \(\lambda w \sp{{6}}\sb{\greeka} .f \sp{{21}}\sb{\greeko\greeka}  w \sp{{6}}  \and f \sp{{22}}\sb{\greeko\greeka}  w \sp{{6}}\)
     PRIM14   LogConst     \(\lambda w \sp{{6}}\sb{\greeka} .f \sp{{23}}\sb{\greeko\greeka}  w \sp{{6}}  \lor f \sp{{24}}\sb{\greeko\greeka}  w \sp{{6}}\)
     PRIM15   PrimQuant   \(\lambda w \sp{{6}}\sb{\greeka}   \exists w \sp{{7}}\sb{\greeki}  f \sp{{25}}\sb{\greeko\greeki\greeka}  w \sp{{6}}  w \sp{{7}}\)
     PRIM16   PrimQuant   \(\lambda w \sp{{6}}\sb{\greeka}   \forall w \sp{{7}}\sb{\greeki}  f \sp{{26}}\sb{\greeko\greeki\greeka}  w \sp{{6}}  w \sp{{7}}\)
     PRIM17   GenSub2   \(\lambda w \sp{{6}}\sb{\greeka}   \exists w \sp{{8}}\sb{\greeki} .f \sp{{27}}\sb{\greeko\greeki\greeka}  w \sp{{6}}  w \sp{{8}}  \lor f \sp{{28}}\sb{\greeko\greeki\greeka}  w \sp{{6}}  w \sp{{8}}\)
     PRIM18   GenSub2   \(\lambda w \sp{{6}}\sb{\greeka}   \exists w \sp{{8}}\sb{\greeki} .f \sp{{29}}\sb{\greeko\greeki\greeka}  w \sp{{6}}  w \sp{{8}}  \and f \sp{{30}}\sb{\greeko\greeki\greeka}  w \sp{{6}}  w \sp{{8}}\)
     PRIM19   GenSub2   \(\lambda w \sp{{6}}\sb{\greeka}   \exists w \sp{{8}}\sb{\greeki} .f \sp{{31}}\sb{\greeko\greeki\greeka}  w \sp{{6}}  w \sp{{8}}  \lor f \sp{{32}}\sb{\greeko\greeki\greeka}  w \sp{{6}}  w \sp{{8}}  \)
               \(\lor f \sp{{33}}\sb{\greeko\greeki\greeka}  w \sp{{6}}  w \sp{{8}}\)
     PRIM20   GenSub2   \(\lambda w \sp{{6}}\sb{\greeka}   \exists w \sp{{8}}\sb{\greeki} .[f \sp{{34}}\sb{\greeko\greeki\greeka}  w \sp{{6}}  w \sp{{8}}  \lor f \sp{{35}}\sb{\greeko\greeki\greeka}  w \sp{{6}}  w \sp{{8}} ] \)
               \(\land f \sp{{36}}\sb{\greeko\greeki\greeka}  w \sp{{6}}  w \sp{{8}}\)
     PRIM21   GenSub2   \(\lambda w \sp{{6}}\sb{\greeka}   \exists w \sp{{8}}\sb{\greeki} .f \sp{{37}}\sb{\greeko\greeki\greeka}  w \sp{{6}}  w \sp{{8}}  \and f \sp{{38}}\sb{\greeko\greeki\greeka}  w \sp{{6}}  w \sp{{8}}  \)
               \(\lor f \sp{{39}}\sb{\greeko\greeki\greeka}  w \sp{{6}}  w \sp{{8}}\)
     PRIM22   GenSub2   \(\lambda w \sp{{6}}\sb{\greeka}   \exists w \sp{{8}}\sb{\greeki} .f \sp{{40}}\sb{\greeko\greeki\greeka}  w \sp{{6}}  w \sp{{8}}  \and f \sp{{41}}\sb{\greeko\greeki\greeka}  w \sp{{6}}  w \sp{{8}}  \)
               \(\land f \sp{{42}}\sb{\greeko\greeki\greeka}  w \sp{{6}}  w \sp{{8}}\)
     PRIM23   GenSub2   \(\lambda w \sp{{6}}\sb{\greeka}   \forall w \sp{{8}}\sb{\greeki} .f \sp{{43}}\sb{\greeko\greeki\greeka}  w \sp{{6}}  w \sp{{8}}  \lor f \sp{{44}}\sb{\greeko\greeki\greeka}  w \sp{{6}}  w \sp{{8}}\)
     PRIM24   GenSub2   \(\lambda w \sp{{6}}\sb{\greeka}   \forall w \sp{{8}}\sb{\greeki} .f \sp{{45}}\sb{\greeko\greeki\greeka}  w \sp{{6}}  w \sp{{8}}  \and f \sp{{46}}\sb{\greeko\greeki\greeka}  w \sp{{6}}  w \sp{{8}}\)
     PRIM25   GenSub2   \(\lambda w \sp{{6}}\sb{\greeka}   \forall w \sp{{8}}\sb{\greeki} .f \sp{{47}}\sb{\greeko\greeki\greeka}  w \sp{{6}}  w \sp{{8}}  \lor f \sp{{48}}\sb{\greeko\greeki\greeka}  w \sp{{6}}  w \sp{{8}}  \)
               \(\lor f \sp{{49}}\sb{\greeko\greeki\greeka}  w \sp{{6}}  w \sp{{8}}\)
     PRIM26   GenSub2   \(\lambda w \sp{{6}}\sb{\greeka}   \forall w \sp{{8}}\sb{\greeki} .[f \sp{{50}}\sb{\greeko\greeki\greeka}  w \sp{{6}}  w \sp{{8}}  \lor f \sp{{51}}\sb{\greeko\greeki\greeka}  w \sp{{6}}  w \sp{{8}} ] \)
               \(\land f \sp{{52}}\sb{\greeko\greeki\greeka}  w \sp{{6}}  w \sp{{8}}\)
     PRIM27   GenSub2   \(\lambda w \sp{{6}}\sb{\greeka}   \forall w \sp{{8}}\sb{\greeki} .f \sp{{53}}\sb{\greeko\greeki\greeka}  w \sp{{6}}  w \sp{{8}}  \and f \sp{{54}}\sb{\greeko\greeki\greeka}  w \sp{{6}}  w \sp{{8}}  \)
               \(\lor f \sp{{55}}\sb{\greeko\greeki\greeka}  w \sp{{6}}  w \sp{{8}}\)
     PRIM28   GenSub2   \(\lambda w \sp{{6}}\sb{\greeka}   \forall w \sp{{8}}\sb{\greeki} .f \sp{{56}}\sb{\greeko\greeki\greeka}  w \sp{{6}}  w \sp{{8}}  \and f \sp{{57}}\sb{\greeko\greeki\greeka}  w \sp{{6}}  w \sp{{8}}  \)
               \(\land f \sp{{58}}\sb{\greeko\greeki\greeka}  w \sp{{6}}  w \sp{{8}}\)
\end{tpsexample}

\indexother{PR97} usually generates even more substitutions than PR95.
At depth 1, it produces substitutions which are appropriately generalized versions of
the subformulas of the current gwff, each having between \indexflag{MIN-PRIM-LITS}
and \indexflag{MAX-PRIM-LITS} literals.
At depth N>1 it adds (N-1) quantifiers in front of each of the substitutions generated at
depth 1. The types of the quantified variables are determined by \indexflag{PRIM-BDTYPES}.
Example (\indexflag{MIN-PRIM-LITS} 2, \indexflag{MAX-PRIM-LITS} 3, \indexflag{MAX-PRIM-DEPTH} 1):

\begin{tpsexample}
The current gwff is THM15B:
\(\forall f \sb{\greeki\greeki} . \exists g \sb{\greeki\greeki}  [ITERATE+ f g \and  \exists x \sb{\greeki} .g x = x \and  \forall z \sb{\greeki} .g z = z \implies z = x] \)
               \(\implies  \exists y \sb{\greeki} .f y = y\)

     PRIM15   LogConst     \(\lambda w \sp{{70}}\sb{\greeki\greeki} .p \sp{{60}}\sb{\greeko(\greeki\greeki)}  w \sp{{70}}  \and p \sp{{61}}\sb{\greeko(\greeki\greeki)}  w \sp{{70}}\)
     PRIM16   LogConst     \(\lambda w \sp{{70}}\sb{\greeki\greeki} .p \sp{{62}}\sb{\greeko(\greeki\greeki)}  w \sp{{70}}  \lor p \sp{{63}}\sb{\greeko(\greeki\greeki)}  w \sp{{70}}\)
     PRIM17   SubFmSub1   \(\lambda w \sp{{70}}\sb{\greeki\greeki}   \exists w \sp{{71}}\sb{\greeki} .p \sp{{64}}\sb{\greeki\greeki(\greeki\greeki)}  w \sp{{70}}  w \sp{{71}}  = p \sp{{65}}\sb{\greeki\greeki(\greeki\greeki)}  w \sp{{70}}  w \sp{{71}}\)
     PRIM18   SubFmSub1   \(\lambda w \sp{{70}}\sb{\greeki\greeki}   \forall w \sp{{72}}\sb{\greeko\greeki} .p \sp{{66}}\sb{\greeko(\greeko\greeki)(\greeki\greeki)}  w \sp{{70}}  w \sp{{72}}  \)
               \(\implies p \sp{{67}}\sb{\greeko(\greeko\greeki)(\greeki\greeki)}  w \sp{{70}}  w \sp{{72}}\)
     PRIM19   SubFmSub1   \(\lambda w \sp{{70}}\sb{\greeki\greeki} .p \sp{{68}}\sb{\greeki(\greeki\greeki)}  w \sp{{70}}  = p \sp{{69}}\sb{\greeki(\greeki\greeki)}  w \sp{{70}}\)
     PRIM20   SubFmSub1   \(\lambda w \sp{{70}}\sb{\greeki\greeki}   \forall w \sp{{73}}\sb{\greeki\greeki} .p \sp{{70}}\sb{\greeko(\greeki\greeki)(\greeki\greeki)}  w \sp{{70}}  w \sp{{73}}  \)
               \(\implies p \sp{{71}}\sb{\greeko(\greeki\greeki)(\greeki\greeki)}  w \sp{{70}}  w \sp{{73}}\)
     PRIM21   SubFmSub1   \(\lambda w \sp{{70}}\sb{\greeki\greeki} .p \sp{{72}}\sb{\greeko(\greeki\greeki)}  w \sp{{70}}  \and  \forall w \sp{{74}}\sb{\greeki\greeki}  p \sp{{73}}\sb{\greeko(\greeki\greeki)(\greeki\greeki)}  w \sp{{70}}  w \sp{{74}}\)
     PRIM22   SubFmSub1   \(\lambda w \sp{{70}}\sb{\greeki\greeki} .p \sp{{74}}\sb{\greeko(\greeki\greeki)}  w \sp{{70}}  \implies p \sp{{75}}\sb{\greeko(\greeki\greeki)}  w \sp{{70}}\)
     PRIM23   SubFmSub1   \(\lambda w \sp{{70}}\sb{\greeki\greeki}   \forall w \sp{{75}}\sb{\greeko(\greeki\greeki)} .p \sp{{76}}\sb{\greeko(\greeko(\greeki\greeki))(\greeki\greeki)}  w \sp{{70}}  w \sp{{75}}  \)
               \(\implies p \sp{{77}}\sb{\greeko(\greeko(\greeki\greeki))(\greeki\greeki)}  w \sp{{70}}  w \sp{{75}}\)
     PRIM24   SubFmSub1   \(\lambda w \sp{{70}}\sb{\greeki\greeki}  ITERATE+ [p \sp{{78}}\sb{\greeki\greeki(\greeki\greeki)}  w \sp{{70}} ].p \sp{{79}}\sb{\greeki\greeki(\greeki\greeki)}  w \sp{{70}}\)
     PRIM25   SubFmSub1   \(\lambda w \sp{{70}}\sb{\greeki\greeki} .p \sp{{80}}\sb{\greeko(\greeki\greeki)}  w \sp{{70}}  \)
               \(\implies p \sp{{81}}\sb{\greeki(\greeki\greeki)}  w \sp{{70}}  = p \sp{{82}}\sb{\greeki(\greeki\greeki)}  w \sp{{70}}\)
     PRIM26   SubFmSub1   \(\lambda w \sp{{70}}\sb{\greeki\greeki} .p \sp{{83}}\sb{\greeki(\greeki\greeki)}  w \sp{{70}}  = p \sp{{84}}\sb{\greeki(\greeki\greeki)}  w \sp{{70}}  \)
               \(\implies p \sp{{85}}\sb{\greeko(\greeki\greeki)}  w \sp{{70}}\)
     PRIM27   SubFmSub1   \(\lambda w \sp{{70}}\sb{\greeki\greeki}   \forall w \sp{{76}}\sb{\greeki} .p \sp{{86}}\sb{\greeko\greeki(\greeki\greeki)}  w \sp{{70}}  w \sp{{76}}  \)
               \(\implies p \sp{{87}}\sb{\greeki\greeki(\greeki\greeki)}  w \sp{{70}}  w \sp{{76}}  = p \sp{{88}}\sb{\greeki\greeki(\greeki\greeki)}  w \sp{{70}}  w \sp{{76}}\)
     PRIM28   SubFmSub1   \(\lambda w \sp{{70}}\sb{\greeki\greeki}   \forall w \sp{{77}}\sb{\greeki} .p \sp{{89}}\sb{\greeki\greeki(\greeki\greeki)}  w \sp{{70}}  w \sp{{77}}  = p \sp{{90}}\sb{\greeki\greeki(\greeki\greeki)}  w \sp{{70}}  w \sp{{77}}  \)
               \(\implies p \sp{{91}}\sb{\greeko\greeki(\greeki\greeki)}  w \sp{{70}}  w \sp{{77}}\)
     PRIM29   SubFmSub1   \(\lambda w \sp{{70}}\sb{\greeki\greeki} .p \sp{{92}}\sb{\greeki(\greeki\greeki)}  w \sp{{70}}  = p \sp{{93}}\sb{\greeki(\greeki\greeki)}  w \sp{{70}}  \)
               \(\land  \forall w \sp{{78}}\sb{\greeki}  p \sp{{94}}\sb{\greeko\greeki(\greeki\greeki)}  w \sp{{70}}  w \sp{{78}}\)
     PRIM30   SubFmSub1   \(\lambda w \sp{{70}}\sb{\greeki\greeki}   \exists w \sp{{79}}\sb{\greeki} .p \sp{{95}}\sb{\greeki\greeki(\greeki\greeki)}  w \sp{{70}}  w \sp{{79}}  \)
               \(= p \sp{{96}}\sb{\greeki\greeki(\greeki\greeki)}  w \sp{{70}}  w \sp{{79}}  \and  \forall w \sp{{80}}\sb{\greeki}  p \sp{{97}}\sb{\greeko\greeki\greeki(\greeki\greeki)}  w \sp{{70}}  w \sp{{79}}  w \sp{{80}}\)
     PRIM31   SubFmSub1   \(\lambda w \sp{{70}}\sb{\greeki\greeki} .p \sp{{98}}\sb{\greeko(\greeki\greeki)}  w \sp{{70}}  \and  \forall w \sp{{81}}\sb{\greeki\greeki} .p \sp{{99}}\sb{\greeko(\greeki\greeki)(\greeki\greeki)}  w \sp{{70}}  w \sp{{81}}  \)
               \(\implies p \sp{{100}}\sb{\greeko(\greeki\greeki)(\greeki\greeki)}  w \sp{{70}}  w \sp{{81}}\)
     PRIM32   SubFmSub1   \(\lambda w \sp{{70}}\sb{\greeki\greeki} .p \sp{{101}}\sb{\greeko(\greeki\greeki)}  w \sp{{70}}  \and  \forall w \sp{{82}}\sb{\greeki\greeki}  p \sp{{102}}\sb{\greeko(\greeki\greeki)(\greeki\greeki)}  w \sp{{70}}  w \sp{{82}}  \)
               \(\implies p \sp{{103}}\sb{\greeko(\greeki\greeki)}  w \sp{{70}}\)
     PRIM33   SubFmSub1   \(\lambda w \sp{{70}}\sb{\greeki\greeki}   \forall w \sp{{83}}\sb{\greeko(\greeki\greeki)} .p \sp{{104}}\sb{\greeko(\greeko(\greeki\greeki))(\greeki\greeki)}  w \sp{{70}}  w \sp{{83}}  \)
               \(\land  \forall w \sp{{84}}\sb{\greeki\greeki}  p \sp{{105}}\sb{\greeko(\greeki\greeki)(\greeko(\greeki\greeki))(\greeki\greeki)}  w \sp{{70}}  w \sp{{83}}  w \sp{{84}}  \implies p \sp{{106}}\sb{\greeko(\greeko(\greeki\greeki))(\greeki\greeki)}  w \sp{{70}}  w \sp{{83}}\)
     PRIM34   SubFmSub1   \(\lambda w \sp{{70}}\sb{\greeki\greeki} .ITERATE+ [p \sp{{107}}\sb{\greeki\greeki(\greeki\greeki)}  w \sp{{70}} ] [p \sp{{108}}\sb{\greeki\greeki(\greeki\greeki)}  w \sp{{70}} ] \)
               \(\land  \exists w \sp{{85}}\sb{\greeki}  p \sp{{109}}\sb{\greeko\greeki(\greeki\greeki)}  w \sp{{70}}  w \sp{{85}}\)
     PRIM35   SubFmSub1   \(\lambda w \sp{{70}}\sb{\greeki\greeki}   \exists w \sp{{86}}\sb{\greeki\greeki} .ITERATE+ [p \sp{{110}}\sb{\greeki\greeki(\greeki\greeki)(\greeki\greeki)}  w \sp{{70}}  w \sp{{86}} ] \)
               \([p \sp{{111}}\sb{\greeki\greeki(\greeki\greeki)(\greeki\greeki)}  w \sp{{70}}  w \sp{{86}} ] \and  \exists w \sp{{87}}\sb{\greeki}  p \sp{{112}}\sb{\greeko\greeki(\greeki\greeki)(\greeki\greeki)}  w \sp{{70}}  w \sp{{86}}  w \sp{{87}}\)
     PRIM36   SubFmSub1   \(\lambda w \sp{{70}}\sb{\greeki\greeki} . \exists w \sp{{88}}\sb{\greeki\greeki}  p \sp{{113}}\sb{\greeko(\greeki\greeki)(\greeki\greeki)}  w \sp{{70}}  w \sp{{88}}  \)
               \(\implies  \exists w \sp{{89}}\sb{\greeki} .p \sp{{114}}\sb{\greeki\greeki(\greeki\greeki)}  w \sp{{70}}  w \sp{{89}}  = p \sp{{115}}\sb{\greeki\greeki(\greeki\greeki)}  w \sp{{70}}  w \sp{{89}}\)
     PRIM37   SubFmSub1   \(\lambda w \sp{{70}}\sb{\greeki\greeki}   \forall w \sp{{90}}\sb{\greeki\greeki} . \exists w \sp{{91}}\sb{\greeki\greeki}  p \sp{{116}}\sb{\greeko(\greeki\greeki)(\greeki\greeki)(\greeki\greeki)}  w \sp{{70}}  w \sp{{90}}  w \sp{{91}}  \)
               \(\implies  \exists w \sp{{92}}\sb{\greeki} .p \sp{{117}}\sb{\greeki\greeki(\greeki\greeki)(\greeki\greeki)}  w \sp{{70}}  w \sp{{90}}  w \sp{{92}}  = p \sp{{118}}\sb{\greeki\greeki(\greeki\greeki)(\greeki\greeki)}  w \sp{{70}}  w \sp{{90}}  w \sp{{92}}\)
\end{tpsexample}

In the mating-search and editor top levels, the command
\indexcommand{NAME-PRIM} will show you all such substitutions; you can get more
(or fewer) by adjusting the settings of the flags given in the subject \indexother{PRIMSUBS}
(type {\tt list primsubs} for a list of these flags). In the mating-search top level,
you can interactively apply a primitive substitution with one of the commands
\indexcommand{PRIM-ALL}, \indexcommand{PRIM-OUTER}, \indexcommand{PRIM-SINGLE}
or \indexcommand{PRIM-SUB} (these are all slightly different; see the help messages
for more information).

Primitve substitutions are generated for all the head variables of appropriate
type in a given wff.
Substitutions that are generated may or may not respect the
value of \indexflag{PRIM-BDTYPES}, depending on the setting of the
flag \indexflag{PRIM-BDTYPES-AUTO}. Please read the help messages for these
flags for more information.

\subsection{The MS91 Procedures}
\label{ms91}

The two procedures MS91-6 and MS91-7 take the original jform and apply, respectively,
MS88 and MS90-3 to variants of it that are produced by applying primitive substitutions
and duplications. They are very similar to MS89 and MS90-9, but allow the user more
control over the substitutions that will be tried; because of this, they have many more
flags to be set than do any of the other procedures.

The MS91 procedures first generate a list of expansion options
(an expansion option is a quantifier duplication or an
instantiation of a predicate variable), and
then generate combinations (finite sets) of these options. Since there
are an infinite number of these finite sets (because any option may be
used more than once), we generate the option sets a few at a time.
The maximum number of sets generated at any one time is governed
by the flag \indexflag{NEW-OPTION-SET-LIMIT}.

The basic idea of the enumeration scheme is that there is a
weight assigned to each set of options. Starting with the weight
of the initial problem (with the empty set of options), we then
look for variants whose weight is within an acceptable range (given by
\indexflag{MS91-WEIGHT-LIMIT-RANGE}) of some target value. These weights
may correspond to new option sets, or they may correspond to old option
sets that are being reconsidered. If we fail
to find any such variants, we increase the acceptable value and try again.
So our objective is to ensure that the simplest substitutions are assigned
the smallest weights. It is possible for a set to be assigned an infinite
weight, in which case it will never be considered.

The weight is the sum of three
separate weights, which we denote \indexother{weight-a}, \indexother{weight-b}
and \indexother{weight-c}. The user is given control over the
weights by means of the settings of the flags \indexflag{WEIGHT-A-FN},
\indexflag{WEIGHT-B-FN}, \indexflag{WEIGHT-C-FN}, \indexflag{WEIGHT-A-COEFFICIENT},
\indexflag{WEIGHT-B-COEFFICIENT} and \indexflag{WEIGHT-C-COEFFICIENT}.
The total weight of each option set is initially the sum of each of the three
functions evaluated on the set, multiplied by each of the three
coefficients. Asking for help on these flags will give you a list of
the possible functions for each flag; asking for help on MS91-6 or MS91-7
will give you an summary of this section of the manual.

Weight-a should in some way measure
the complexity of an individual option (and so the weight-a of the
option set is the sum of the weight-a's of each of its elements).
So one possible value for \indexflag{WEIGHT-A-FN} is \indexother{EXPANSION-LEVEL-WEIGHT-A},
which assigns to each option its expansion level (i.e. how late in the list of options
it is).

Weight-b should calculate the weight of the entire option set, by
considering the number of duplications and substitutions it contains.
So one possible value for \indexflag{WEIGHT-B-FN} is \indexother{SIMPLE-WEIGHT-B-FN},
which sums up the  various penalties for substitutions and duplications given
in the flags \indexflag{PENALTY-FOR-EACH-PRIMSUB}, \indexflag{PENALTY-FOR-MULTIPLE-PRIMSUBS}
and \indexflag{PENALTY-FOR-MULTIPLE-SUBS}. Another possible value is
\indexother{SIMPLEST-WEIGHT-B-FN}, which adds up the number of substitutions and
duplications and returns that number plus one.
Yet a third possible value is \indexother{ALL-PENALTIES-FN}, which evaluates
\indexother{SIMPLE-WEIGHT-B-FN} and then adds the penalty given by the flag
\indexflag{PENALTY-FOR-ORDINARY-DUP}.

Weight-c should measure the complexity of the resulting jform.
So two possible values for \indexflag{WEIGHT-C-FN} are \indexother{OPTION-SET-NUM-VPATHS}
and \indexother{OPTION-SET-NUM-LEAVES}, which count up (respectively) the number of paths
or leaves in the resulting jform.

The procedures will work on each jform for a limited time (given by the
flag \indexflag{SEARCH-TIME-LIMIT}), before assigning a new weight to the option
set and proceeding to the next jform. This means that the procedure will only
give up searching on a particular jform if it has discovered that there can be no
successful mating, or if the new weight that is assigned is INFINITY.
The user can control the value of the new weight by changing the setting
of the flag \indexflag{RECONSIDER-FN}.

So the structure of the MS91 procedures looks like this:
\begin{enumerate}
\item Work on the original problem for \indexflag{SEARCH-TIME-LIMIT}
seconds using either MS88 or MS90-3 as appropriate.

\item If that fails, change the weight of the old problem (using \indexflag{RECONSIDER-FN}).

\item Generate new options (which will be a list of possible duplications and
primitive substitutions for the expansion variables of the problem).

\item Using the list of options, attempt to generate \indexflag{NEW-OPTION-SET-LIMIT} many
new option sets whose weights lie within \indexflag{MS91-WEIGHT-LIMIT-RANGE} of the
weight of the last option set used. If we fail to do this because no more option
sets can be generated from the current option list, give up and go on to the next step.
If we fail because all the option sets we generate are not in our desired weight range,
increase the lower limit of this range and try again.

\item Apply either MS88 or MS90-3 to each of the new option sets, in order of their
weight (may also reconsider old option sets; an option set expires permanently after \indexflag{MAX-SEARCH-LIMIT}
seconds in total have been spent on it). Each set gets considered for
\indexflag{SEARCH-TIME-LIMIT} seconds at a time.

\item When all of the new option sets have been used we call
the function given by \indexflag{OPTIONS-GENERATE-FN}, which will return T if new options should be
generated. Depending on the setting of this flag, this might be because we have generated all
possible new option sets from the current list of options, or because the current option sets
have become too complicated in some sense. If \indexflag{OPTIONS-GENERATE-FN} returns NIL
then we go back to step 4; otherwise
return to step 3 (note that the problem will usually now contain new expansion variables
because of the primitive substitutions).
\end{enumerate}

\section{Some Important Flags in ms90-3 Search}
\begin{itemize}
\item \indexflag{NUM-OF-DUPS}
The flag determines the maximum number of duplications allowed
on a path. In the extreme case, i.e., when num-of-dup is 0, a
universal jform can still appear several times on different paths
as long as it appears at most once on each path. It is exceedingly
crucial to keep the flag low while you are dealing with higher
problems. This could be achieved by reorganizing jforms, sometimes.

\item \indexflag{MAX-MATES} The flag determines the maximum number of times
a literal could occur in a mating. Please always remember that
literals generated by duplications on a path are regarded as a new
one, which bears no relations with the old ones. Hence, both the new
ones and the old ones can occur in a mating up to
\indexflag{MAX-MATES} times. You may often increment
\indexflag{MAX-MATES} so that you can decrement
\indexflag{NUM-OF-DUPS}, or vice versa. In the higher order case, it
is usually worth augmenting \indexflag{MAX-MATES} to keep
\indexflag{NUM-OF-DUP} low.

\item \indexflag{ORDER-COMPONENTS} The flag gives you a few choices to
reorganize the jform through which you want to search. Setting it to T
often reduces the minimum value of \indexflag{NUM-OF-DUPS} we need to
find a mating. When the jforms under predicate variables are a big
disjunction, it is recommended that you set \indexflag{ORDER-COMPONENTS}
to T so that you might have a chance to set \indexflag{NUM-OF-DUPS} to 0.
This expedites the whole search process dramatically.
\end{itemize}

\section{Helpful Hints for MS91}
\begin{itemize}
\item Setting \indexflag{MS91-WEIGHT-LIMIT-RANGE} to INFINITY means that the next option
set that is generated will always be accepted (unless it has weight INFINITY); this
will spped things up, at the possible expense of generating heavier option sets
before lighter ones.

\item Setting \indexflag{NEW-OPTION-SET-LIMIT} to 1 will minimize the delay between each search.
However, it's quicker to generate a batch of option sets all at once rather than individually,
so this may make the total time for the search slightly longer.

\item Setting \indexflag{RECONSIDER-FN} to INF-WEIGHT will prevent any option set from being
considered more than once.

\item Try not to use everything all at once; there's probably no need to use
weights a, b and c all together. Try setting one or two of their coefficients to 0.

\item Judicious use of INFINITY as a penalty will get you to your goal a lot
faster; so (for example) \indexflag{PENALTY-FOR-ORDINARY-DUPS} INFINITY will ensure
that you never get a vpform with any unused duplications, or
\indexflag{PENALTY-FOR-MULTIPLE-SUBS} INFINITY will ensure that each variable gets
at most one primsub. (However, generating sets with weight INFINITY still
takes time; if it seems to pause for a while before producing a new option
set, it may well be producing and ignoring a batch with weight INFINITY...)

\item Generating new options takes a considerable time (because the jform will have gone from
having one or two expansion variables to having ten or twenty of them).
If your search is likely to reach
the stage of generating new options, then:
\begin{itemize}
\item You could try increasing the number of primsubs produced in the first round of options,
by increasing the value of \indexflag{MAX-PRIM-DEPTH}; this might allow the search to
finish without generating new options. (In general, the set of primsubs produced by having
\indexflag{MAX-PRIM-DEPTH} greater than 1 is a subset of those produced by repeatedly
generating new options with \indexflag{MAX-PRIM-DEPTH} equal to 1.)

\item You should consider making weight-c or weight-a the dominant weight function,
as opposed to weight-b.
(This is because the new options will appear to be single primitive substitutions, just like
the old ones, and thus weight-b will be unable to distinguish them).
\end{itemize}

\item Use the \indexcommand{SEARCH-ORDER} command to see in what order the option sets will
be generated.

\item Compare your flag settings to the four preset modes \indexother{MS91-SIMPLEST}, \indexother{MS91-ORIGINAL},
\indexother{MS91-NODUPS} and \indexother{MS91-DEEP}; these modes are four of the most common
ways in which these flags can usefully be set.
\end{itemize}

% \begin{comment} (This is the old description, prior to  Mar 1994.)
% The two procedures MS91-6 and MS91-7 have a considerable number of flags which
% govern the order in which the various primitive substitutions are applied to the
% jform. The substitutions that are used are limited to those which are shown
% when you type \indexcommand{NAME-PRIM} (in particular, you should check that
% the settings of \indexflag{MAX-PRIM-DEPTH} and \indexflag{MIN-PRIM-DEPTH} are
% appropriate before beginning an automatic search with these procedures).
% 
% An expansion option is a quantifier duplication or an
% instantiation of a predicate variable. In general, we shall take care
% of duplications as part of the search process. Thus, an expansion
% option is essentially a wff to be substituted for a particular
% predicate variable; it may contain additional predicate variables.
% In principle, one may need to use certain options more than once,
% though in simple examples this rarely happens.
% 
% The MS91 procedures first generate a list of expansion options, and
% then generate combinations (finite sets) of these options. Since there
% are an infinite number of these finite sets (because any option may be
% used more than once), we generate the option sets a few at a time.
% The maximum number of sets generated at any one time is governed
% by the flag \indexflag{NEW-OPTION-SET-LIMIT}.
% 
% The basic idea of the enumeration scheme is that there is a
% weight assigned to each set of options. Starting with the weight
% of the initial problem (with the empty set of options), we then
% look for variants whose weight is within an acceptable range (given by
% \indexflag{MS91-WEIGHT-LIMIT-RANGE}) of some target value. These weights
% may correspond to new option sets, or they may correspond to old option
% sets that are being reconsidered. If we fail
% to find any such variants, we increase the acceptable value and try again.
% So our objective is to ensure that the simplest substitutions are assigned
% the smallest weights. It is possible for a set to be assigned an infinite
% weight, in which case it will never be considered.
% 
% The weight is the sum of three
% separate weights, which we denote \indexother{weight-a}, \indexother{weight-b}
% and \indexother{weight-c}. The user is given control over the
% weights by means of the settings of the flags \indexflag{WEIGHT-A-FN},
% \indexflag{WEIGHT-B-FN}, \indexflag{WEIGHT-C-FN}, \indexflag{WEIGHT-A-COEFFICIENT},
% \indexflag{WEIGHT-B-COEFFICIENT} and \indexflag{WEIGHT-C-COEFFICIENT}.
% The total weight of each option set is initially the sum of each of the three
% functions evaluated on the set, multiplied by each of the three
% coefficients. Asking for help on these flags will give you a list of
% the possible functions for each flag; asking for help on MS91-6 or MS91-7
% will give you an summary of this section of the manual.
% 
% Weight-a should in some way measure
% the complexity of an individual option (and so the weight-a of the
% option set is the sum of the weight-a's of each of its elements).
% So one possible value for \indexflag{WEIGHT-A-FN} is \indexother{EXPANSION-LEVEL-WEIGHT-A},
% which assigns to each option its expansion level (i.e. how late in the list of options
% it is).
% 
% Weight-b should calculate the weight of the entire option set, by
% considering the number of duplications and substitutions it contains.
% So one possible value for \indexflag{WEIGHT-B-FN} is \indexother{SIMPLE-WEIGHT-B-FN},
% which sums up the  various penalties for substitutions and duplications given
% in the flags \indexflag{PENALTY-FOR-EACH-PRIMSUB}, \indexflag{PENALTY-FOR-MULTIPLE-PRIMSUBS}
% and \indexflag{PENALTY-FOR-MULTIPLE-SUBS}. Another possible value is
% \indexother{SIMPLEST-WEIGHT-B-FN}, which adds up the number of substitutions and
% duplications and returns that number plus one.
% 
% Weight-c should measure the complexity of the resulting jform.
% So two possible values for \indexflag{WEIGHT-C-FN} are \indexother{OPTION-SET-NUM-VPATHS}
% and \indexother{OPTION-SET-NUM-LEAVES}, which count up (respectively) the number of paths
% or leaves in the resulting jform.
% 
% The procedures will work on each jform for a limited time (given by the
% flag \indexflag{SEARCH-TIME-LIMIT}), before assigning a new weight to the option
% set and proceeding to the next jform. This means that the procedure will only
% give up searching on a particular jform if it has discovered that there can be no
% successful mating, or if the new weight that is assigned is INFINITY.
% The user can control the value of the new weight by changing the setting
% of the flag \indexflag{RECONSIDER-FN}.
% 
% 
% \subsection{Examples}
% 
% To enumerate the variant jforms in as simple a manner as possible, set WEIGHT-A-COEFFICIENT
% and WEIGHT-C-COEFFICIENT to 0, set WEIGHT-B-COEFFICIENT to 10 and set
% WEIGHT-B-FN to SIMPLEST-WEIGHT-B-FN.
% 
% Another setting that works reasonably well is to set all the coefficients to 1,
% WEIGHT-A-FN to EXPANSION-LEVEL-WEIGHT-A, WEIGHT-B-FN to SIMPLE-WEIGHT-B-FN and
% WEIGHT-C-FN to OPTION-SET-NUM-LEAVES, and then set all the various PENALTY- flags
% to 3.
% \end{comment}
\section{The Matingstree Procedure}\label{mtree}

Note: The ideas and resources described in this chapter are
experimental and not well developed. Most users of TPS should
probably ignore them.

\subsection{A Brief Overview}

The matingsearch procedures already described take a vpform, enumerate all possible
paths through it, and attempt to block them all one by one. This is not the way that
a human operator, attempting the same problem, would proceed, and the matingstree
top level is an attempt to approach the problem in a more natural way. For example,
suppose that a vpform contains a conjunct (B OR C) and a single literal A. Then if
A is mated with B on some path, this closes all extensions of that path through B, but
creates an obligation to `do something' about C - which is to say, to block off
all variants of the path that go through C instead of through B. The matingstree
procedure is an attempt to formalize this notion of `obligation' and so to direct the proof
search.


\subsection{A Detailed Plan of the Matingstree Top Level}

	While TPS currently has a number of search procedures, they
all construct matings by searching paths through the formula in a very
systematic way.  This allows TPS to keep track of where it is in its
search process in a very economical way, so that TPS can run for weeks
without running out of memory. However, this very economical use of
space limits our ability to implement various search heuristics within
the context of current search procedures. In particular, we would like
to develop methods of searching for expansion proofs which closely
mimic attempts to construct natural deduction proofs.

	We shall describe a much more general way of organizing the
search process which, at the cost of using additional space, will
enable us to choose links to add to matings in arbitrary order, to
keep track of where we are in the search process, and to
simultaneously build up many matings (thus using a mixture of
breadth-first and depth-first search).  Information will be stored
which will enable TPS to know when it has spanned all paths without
actually searching through them all.  We need to avoid searching all
paths (since there are generally too many), and we need to construct a
mating in such a way that at a certain point we know without further
checking that we have an acceptable mating.  We ignore parts of the
wff (or potentially relevant lemmas which are not part of the wff)
until we explicitly make them part of the search process.

	We would like to avoid backtracking, which is troublesome and
time-consuming. By keeping more information, we can just abandon certain
parts of the search and turn our attention to others.

	We would like expansions of the jform (quantifier duplications
and applications of primitive projections and substitutions, plus
subformula substitutions for predicate variables) to be motivated by
the needs of the matingsearch process. Thus, this procedure should
incorporate some of the key advantages of path-focused duplication
(but will be `literal-focused' duplication rather than `path-focused'
duplication).

	The emphasis will be on being smart rather than fast. We will
try to compute whatever information is necessary to limit the size of
the search space, and give first priority to exploring those parts of
the search space which seem (according to our heuristics) to be most
likely to yield success. However, we will try to do this computation
efficiently. Sometimes a `partial computation' will suffice to make it
clear whether the complete computation is worth doing.

	We will try to manage our use of space effectively. When we
reach a stage where we are ready to discard certain structures, we
will do so in a way that will permit the space to be reclaimed by the
garbage collector. Some structures could be regenerated if necessary.
It would be nice to assign priorities to structures, and have a
mechanism for discarding those of low priority when more space is
needed.


	We shall sometimes speak loosely and call any set of connections
a mating, although officially a mating must have an associated
substitution which makes mated pairs complementary.

	As the search process progresses, we build up a tree of
matings, or matingstree (MST). By making use of the tree structure, we
can hope to store the information economically.  Each node in this
tree represents a step in the search process. In general, a node
represents a mating which was obtained by adding a link to the mating
of its parent node.  Each node has a number of attributes:
\begin{itemize}
\item its mating;

\item a set of obligations (see below);

\item auxiliary information about items such as:

\item free variables,

\item the unification tree for the mating,

\item potential mates for various literals.
\end{itemize}

Definition: A literal L is on some extension of the path P in the
jform W iff no disjunction of W contains L in one of its scopes (left
or right) and some literal which is on P in the other scope (right or
left).

	An obligation consists of a path and a node of the jform on
that path in the jform. When all the obligations for a node of the MST
are fulfilled, the mating will be complete (i.e., span every path).
Thus, we seek to construct a node of the MST which has an empty set of
obligations.


We shall have to consider how obligations can be represented
economically.  It seems that the nodes of the path of an obligation
are all literals in links of the mating associated with the node of
the mst where the obligation first appeared. Perhaps such paths can
be represented implicitly. (For the case where the wff is in cnf and
in fol, compare with the model-elimination procedure.)


	The initial node of the MST has an empty mating, and one
obligation: the node is the entire wff to be refuted, and the path
contains only that node.

	We express the process of constructing a mating in terms of
`blocking' a node N of an expansion tree relative to a path P
containing that node. We imagine an abstract little creature called a
`spoiler' trying to run through the expansion tree (or the jform
corresponding to it), and we must block all its attempts.

A node can be blocked in various ways (and
heuristics will help us decide which tasks to work on first):

\begin{itemize}
\item To block a conjunction, block any of its conjuncts.
One way to do this is to mate two of its conjuncts with each other.

\item To block an expansion node, block any of its expansions.

\item To block a disjunction, block all of its disjuncts.

\item To block a definition or selection node, block its
unique child in the tree.

\item To block a literal N, mate it with some literal of that path.

\item To block a literal N, mate it with some literal L not yet on the
path P but on some extension of it
and establish the set of subtasks
$\{<M, P \union{M}> | M is a literal on some horizontal path through L\}$.
(This amounts to blocking the disjunction node D  containing L
relative to the path $P \union{L}$.) The literal L may be generated
by making a new instantiation of some expansion node.
This might be a quantifier duplication or a primitive substitution
or a sequence of primitive substitutions.

\item A contradictory node (which might arise by instantiating a
predicate variable with FALSEHOOD) is blocked.

\item Another way to block a node is to use equations to mate it
with another node as above, or to reduce it to $ \not{ C = C}$.
Thus we implement equational matings. (Of course, our present
translation command etree-nat won't work on such matings.)

\end{itemize}

We visualize the root of the matingstree (i.e., the empty mating)
as being at the top of the tree.

CGRAPHS

In principle, there should be a connection graph (cgraph)
associated with each node of the matingstree which is obtained from
cgraphs of nodes higher in the matingstree by deleting links which are
not compatible with the mating for that node. Thus, the unifying
substitution (to non-branching depth) associated with the mating
should be used in computing the cgraph.  We don't actually compute
these cgraphs completely; we just put into them information which is
computed as it is needed. Also, instead of computing certain cgraphs
we may just use cgraphs associated with higher nodes in the tree.
In addition, we may associate with each node of the tree
a set of links which are incompatible with the mating at that node.
We call these links `negative'.
Thus, one obtains the cgraph for that node by deleting from the
original cgraph (the cgraph for the root node) all the `negative
links' associated with nodes at or above that node in the matingstree.
(As one adds links to the mating, one also accumulates more negative
links.)

When we try to add a link to a mating and find that it is
incompatible with it, we add that link to the set of negative links
for that node of the matingstree, and thus delete that link from the
positive cgraph for that node.

Eventually, we should compute and represent these cgraphs in as
sophisticated way as possible. Instead of blindly checking all links
which occur in the parent cgraph to see whether they can be deleted,
we should consider which variables are actually constrained by the
relevant links of the mating, and which literals these variables
actually occur in. Review for relevant ideas the paper
Robert Kowalski, `A Proof Procedure Using
 Connection Graphs', Journal of the ACM 22 (1975), 572-595.

We find how we can satisfy obligations by looking in the cgraph.

A node (mating) fails if:
it is found not to be unifiable, or
it is found that there is no way to satisfy one of its obligations.
When a node fails, we delete the link that created it from the
cgraph of the node above it. If this eliminates the last possible
way of fulfilling some obligation of that node, it will also fail.

The search process consists of performing tasks. A task consists of
creating a new node N2 of the MST which is a son of an existing node
N1, and fulfills an obligation of N1. N2 inherits all the obligations
of N1 except for the one it fulfilled, and it may also have additional
obligations.

Each task has a priority (which may change as the search progresses?),
and the priorities determine in what order tasks are performed. If
many processors are available, they each choose the next available task of
of highest priority.

At each stage we can ask questions:
What node (mating) should I work on?
What obligation should I work on?
How shall I try to satisfy that obligation?
Heuristics can be used to answer each of these questions.
If one has even very naive ways to answer all of these questions,
one has an automatic search procedure.

Search heuristics are generally expressed as ways of setting the priorities.
One heuristic: when we do things which create obligations, check quickly
whether those obligations can be simultaneously satisfied.
Another: concentrate on satisfying obligations which can be
satisfied only in a small number of ways (ideally one).

We may need some way to prevent the same mating from being generated on
different nodes of the MST. This is analogous to the subsumption problem
in resolution, and we may have to deal with it explicitly. Perhaps it will help
to order the obligations of a node, have this ordering be inherited by
descendant nodes, and insist that obligations be filled in this order.

Compare this procedure with model-elimination (etc.) when it is applied
to wffs of fol in cnf.
This procedure seems to incorporate the key idea in the resolution
set-of-support strategy.

        This procedure should be designed to deal with the following
problem:

	TPS can't prove THM120F because it doesn't apply apply
unifying substitutions to the jform as it proceeds. Thus projections
for head variables which which could produce contradictions on a path
don't actually get applied during the matingsearch process. Of course,
pure projections eventually get generated and applied as part of
mating-search, but for THM120F we have a literal r[p v $\sim$p]Q, and for
r we substitute [lambda u. lambda v .$\sim$u] in order to deal with other
pairs of literals. If this were actually applied to the jform, it would
yield a contradiction, but it is not applied.

	Note that this is a complicated issue which is not easy to
deal with, since at each stage we have a whole unification tree
associated with the current mating, and we do not have any one
current substitution to apply.

	We need to consider how to economically keeping track of what
we've tried with matingstree procedure. Ideas for possible solutions:
\begin{itemize}
\item Attach heuristic weights to various things we might try, and order
them by weight.

\item Whatever algorithm generates possibilities essentially enumerates
them. Just keep a list of numbers (under this enumeration) of
possibilities already explored.
\end{itemize}

\subsection{How to Use the Mtree Top Level}

The mtree top level is entered with the command \indexmexpr{MTREE}. This is
deliberately designed to be as similar to the {\tt MATE} top level as possible,
in order not to be too confusing. The \indexcommand{LEAVE} command leaves the top level
again, and will prompt the user about merging the expansion tree if it detects that the
proof has been finished. Everything uses the MS88 notation and MS88-style unification.

For brevity, we will refer to the matingstree as `the mtree', and an obligation tree
as `an otree'. There are many otrees (one for each node of the mtree); if we refer to `the otree'
this is to be interpreted as `the obligation tree associated to the current node of the
matingstree'. An obligation which is satisfied is also said to be `closed'; obligations
which are not closed are, naturally, referred to as `open'. A node of the mtree is said
to be `closed' if it has no open obligations left in its otree.

There are a wide variety of printing commands, since we have a lot of structures present.
\indexcommand{PM-NODE} prints the current node of the mtree in full detail; \indexcommand{POB-NODE}
does the same for the current obligation of the otree (there are potentially many open obligations
in an otree; the `current' one will be defined by the setting of the flag \indexflag{DEFAULT-OB}).
\indexcommand{POB} prints the current obligation in minimal detail.

\indexcommand{PMTR}, \indexcommand{PMTR*} and \indexcommand{PMTR-FLAT} print the matingstree, starting from
the current node (by default; they can also all take a node as an optional argument and start from that node instead)
and working downwards towards the leaves, in varying formats and degrees of detail.

\indexcommand{POTR}, \indexcommand{POTR-FLAT} and \indexcommand{POTR*-FLAT} do the same for the otree.

\indexcommand{PPATH} and \indexcommand{PPATH*} print all the obligations from the given (leaf) obligation
to the root of the obligation tree.

\indexcommand{CONNS-ADDED} shows which connections have already been added to this mtree node, if any.
\indexcommand{LIVE-LEAVES} shows which leaves of the tree are not marked as dead (the search procedure will mark a
node as dead if it has open obligations but they cannot be satisfied, or if it has decided to give up on it for
some other reason).
\indexcommand{SHOW-MATING} shows the mating associated with the current mtree node and \indexcommand{SHOW-SUBSTS}
shows the substitution stack at the current node (built up as the connections are unified).

There are also a collection of commands for moving about in the mtree; \indexcommand{UP} and \indexcommand{DOWN}
are fairly clear; \indexcommand{SIB} goes to the next sibling of the current mtree node, and \indexcommand{GOTO}
goes to a node by name. \indexcommand{INIT} starts a new mtree, with a single root node and nothing else.
\indexcommand{KILL} marks a node as dead; \indexcommand{RESURRECT} unmarks it. \indexcommand{PRUNE} removes
all dead nodes below the current node; \indexcommand{REM-NODE} removes a single node.

\indexcommand{ADD-CONN} adds a connection, as in the {\tt MATE} top level; this will generate a new mtree node,
with a new otree; this mtree node will become the current node.

Each time a new connection is added, new copies of all the expansion nodes involved are made, and the connection
is made between these new copies rather than between the original literals. This allows the whole mtree to refer to
a unique master expansion tree, rather than lots of smaller expansion trees. It also means that, at any given node, most
of the master expansion tree will be irrelevant; when a closed
mtree node is reached, the \indexcommand{CHOOSE-BRANCH} command discards all other branches of the tree and
trims the expansion tree down to just those parts that are relevant to the closed node. This allows us to use the same
merging functions as in the {\tt MATE} top level.

\subsection{Automatic Searches with the Mtree Top Level}

First, we have some semi-automatic commands: \indexcommand{QRY} takes a literal and an obligation as arguments,
and returns all possible mates for them. \indexcommand{ADD-ALL-LIT} uses {\tt QRY} to simply add all these possible
mates as sons of the current mtree node. \indexcommand{ADD-ALL-OB} does {\tt ADD-ALL-LIT} for every literal
in a given obligation, and \indexcommand{EXPAND-LEAVES} does {\tt ADD-ALL-OB} at every leaf node of the mtree.

It is clear that, time and space being no object, \indexcommand{EXPAND-LEAVES} is complete, in the sense
that if it possible to arrive at a proof in the mtree top level at all, then repeating {\tt EXPAND-LEAVES} will eventually
get you to a closed mtree node. (Notice that the mtree top level does not currently support primitive substitutions,
so not everything that is provable in {\tt MATE} is provable in {\tt MTREE}.)
The automatic procedure MT94-11 does exactly this. (As in {\tt MATE}, you can call this either directly
with \indexcommand{MT94-11}, or indirectly by setting \indexflag{DEFAULT-MS} to {\tt MT94-11} and calling
\indexcommand{GO}. The same applies to the other searches.)

\indexcommand{MT94-12} is a restricted version of \indexcommand{MT94-11}; for each leaf node of the tree,
using the integer flag
\indexflag{MT94-12-TRIGGER}, it decides whether the current obligation has more or fewer literals than
the current setting of this flag. If more, it just applies \indexcommand{ADD-ALL-LIT} to whichever single
literal has the fewest possible mates. If fewer, it applies \indexcommand{ADD-ALL-OB}.

\indexcommand{MT95-1} is still further restricted; it chooses the mtree node with the fewest open obligations,
and then applies {\tt MT94-12} to that node only, and repeats the process.

\subsection{The Mtree Subsumption Checker}

There are a number of approaches to subsumption checking in the mtree top level; the possible benefits
are very great, since not only the number of leaves in
the mtree but also the size of the expansion tree can be kept down by killing off mtree
nodes that are effectively duplicates of existing nodes.

The subsumption checker is governed by the flag \indexflag{MT-SUBSUMPTION-CHECK}. Setting this flag to {\tt NIL}
will turn off subsumption checking altogether.

The weakest subsumption check is {\tt SAME-TAG}; this uses the flags \indexflag{TAG-CONN-FN} and \indexflag{TAG-LIST-FN}
to generate a number corresponding to the connections in the mating at the current node. A new node is
then rejected if its tag is the same as that of some existing node. This method is very quick, it is also
obviously unsound unless one can guarantee that no two different nodes can get the same tag. Nonetheless, it
will probably be correct almost all the time...

The next strongest is {\tt SAME-CONNS}; this first checks as in {\tt SAME-TAG}, but then if the tags match, it actually
checks the matings to make sure they really are the same. This is sound, but possibly not restrictive enough,
since (e.g.) connection (leaf3 . leaf7.1) may effectively be the same as (leaf3 . leaf7.2), depending on
what other literals are contained in the same expansions as leaf 7.1 or 7.2 (which are copies of each other,
in standard {\TPS} notation), and whether any of them are mated to anything else.
The more accurate subsumption check, which takes this into account, has yet to be written. Meanwhile, {\tt SAME-CONNS}
will never wrongly reject a node, but may accept nodes that could have been rejected.

The strongest of all is {\tt SUBSET-CONNS}; this checks whether the new node contains a subset of the
connections at some other node, and rejects it if it does. This is much too strong for any search except {\tt MT94-11},
because only {\tt MT94-11} generates all possible successors to a given node all at once. (Example: Suppose the
first connection we added was A, and we are now at a stage where
we have (somehow) got a node that contains connections ABC, and we are about to generate one containing just AB, and the
correct mating is in fact ABD. Then rejecting the new node would seem to be wrong; it's only all right in MT94-11
because we know that node AD has already been generated elsewhere in the tree.)

\subsection{An Interactive Session in the Mtree Top Level}

\begin{tpsexample}
<4>mtree x2106 !
{\it We choose a simple example; X2106 says that, for all x, if (Rx implies Px} and ((not Qx) implies Rx),
then for all x either Px or Qx is true.)
<Mtree:5>pob

OB0

|FORALL x^2         |
| |LEAF6     LEAF7| |
| |\(\sim\)R x^2 OR P x^2| |
|                   |
|FORALL x^3         |
| |LEAF11    LEAF10||
| |Q x^3  OR R x^3 ||
|                   |
|      LEAF13       |
|      \(\sim\)P X^1       |
|                   |
|      LEAF14       |
|      \(\sim\)Q X^1       |

{\it This command prints the current obligation at the current node of the
matingstree. Each node of the mtree has an associated obligation tree,
which may have many open obligations; which of these is the `current'
obligation is determined by the flag DEFAULT-OB. At this point, the
mtree has exactly one node, and the otree at that node contains exactly
one obligation, which is the entire formula}

<Mtree:6>add-conn
LITERAL1 (LEAFTYPE):  [No Default]>6
OBLIG1 (SYMBOL-OR-INTEGER):  [OB0]>
LITERAL2 (LEAFTYPE):  [No Default]>10
OBLIG2 (SYMBOL-OR-INTEGER):  [OB0]>
Adding new connection: (LEAF6.1 . LEAF10.1)
MSTREE1

{\it We have added a connection between leaf6 and leaf10, both in obligation
number 0. This has created a new mtree node, MSTREE1. We have also moved
down the tree to this new node, which has become the current node.}

<Mtree:7>potr

Numbers in round brackets are open obligations; those in square
brackets are closed. Branches with *'s denote nodes that are
being omitted for lack of space.


                                     MSTREE1
                                      (OB0)
                                        |
                    /--------------------------------------\
                    |                                      |
                  (OB1)                                  (OB2)
                    |                                      |
          /-------------------\
          |                   |
        (OB3)               [OB4]

{\it Here is the obligation tree at the new node MSTREE1. We started from
the obligation OB0, above. Adding that one connection first broke up the
disjunction of leaf6 and leaf7, to form OB1 and OB2. We then connected leaf6
to leaf10, so the disjunction of leaf10 and leaf11 was also broken up to
create OB3 and OB4 on the path going through leaf6; we satisfied OB4 by
connecting leaf10 to leaf6, but OB3 is still open.}

<Mtree:8>pob

OB3
LEAF11.1
 Q x^5

{\it Our current obligation is OB3. However, this doesn't show what we can
mate it to, so we try a different command.}

<Mtree:9>ppath*

OB3
LEAF11.1
 Q x^5

OB1
LEAF6.1
\(\sim\)R x^4

OB0
|FORALL x^2         |
| |LEAF6     LEAF7| |
| |\(\sim\)R x^2 OR P x^2| |
|                   |
|FORALL x^3         |
| |LEAF11    LEAF10||
| |Q x^3  OR R x^3 ||
|                   |
|      LEAF13       |
|      \(\sim\)P X^1       |
|                   |
|      LEAF14       |
|      \(\sim\)Q X^1       |

{\it This shows the branch of the obligation tree that leads to our current obligation.
It seems clear that we can connect the single literal of OB3 to leaf14 in OB0 to close
off this path, so that's what we'll do.}

<Mtree:10>add-conn
LITERAL1 (LEAFTYPE):  [No Default]>11.1
OBLIG1 (SYMBOL-OR-INTEGER):  [OB3]>
LITERAL2 (LEAFTYPE):  [No Default]>14
OBLIG2 (SYMBOL-OR-INTEGER):  [OB0]>
Adding new connection: (LEAF11.1 . LEAF14)
MSTREE2

{\it Again, we have created a new node of the mtree. Let's look at the
obligation tree again and see what's left to do.}

<Mtree:11>potr
                                     MSTREE2
                                      (OB0)
                                        |
                    /--------------------------------------\
                    |                                      |
                  [OB1]                                  (OB2)
                    |                                      |
          /-------------------\
          |                   |
        [OB3]               [OB4]

{\it Notice that OB3 is now closed, and hence so is OB1. We are left with OB2
as our current, and only, obligation.}

<Mtree:12>ppath*

OB2
LEAF7.1
 P x^4

OB0
|FORALL x^2         |
| |LEAF6     LEAF7| |
| |\(\sim\)R x^2 OR P x^2| |
|                   |
|FORALL x^3         |
| |LEAF11    LEAF10||
| |Q x^3  OR R x^3 ||
|                   |
|      LEAF13       |
|      \(\sim\)P X^1       |
|                   |
|      LEAF14       |
|      \(\sim\)Q X^1       |

<Mtree:13>add-conn 7.1
OBLIG1 (SYMBOL-OR-INTEGER):  [OB2]>
LITERAL2 (LEAFTYPE):  [No Default]>13
OBLIG2 (SYMBOL-OR-INTEGER):  [OB0]>
Adding new connection: (LEAF7.1 . LEAF13)
MSTREE3

<Mtree:14>potr
                                     MSTREE3
                                      [OB0]
                                        |
                    /--------------------------------------\
                    |                                      |
                  [OB1]                                  [OB2]
                    |                                      |
          /-------------------\
          |                   |
        [OB3]               [OB4]

{\it All the obligations are closed. Let's look at the matingstree itself.}

<Mtree:15>pmtr* 0

Numbers in round brackets are open obligations. If the brackets
end in `..', there are too many open obligations to fit
under the mstree label. Leaves underlined with \(\wedge\)'s are
closed matingstrees. Matingstrees enclosed in curly brackets are
marked as dead. Branches with *'s denote nodes that are being
omitted for lack of space.


                                    [MSTREE0]
                                      (OB0)
                                        |
                                        |
                                        |
                                    [MSTREE1]
                                    (3 2 1 0)
                                        |
                                        |
                                        |
                                    [MSTREE2]
                                      (2 0)
                                        |
                                        |
                                        |
                                    [MSTREE3]
                                   \(\wedge\wedge\wedge\wedge\wedge\wedge\wedge\wedge\wedge\)

{\it A very boring matingstree, not branching at all. This is because each literal
had exactly one possible mate; if there had been several mates for a literal, the
tree might have branched. Now we're ready to translate our proof into ND form.}

<Mtree:16>leave
Choose branch? [Yes]>
Merge the expansion tree? [Yes]>

{\it `Choose branch?' allows you to prune the expansion tree to contain only those
expansions which were actually part of the branch leading to the final node. In an
mtree with many branches, this is an essential part of preprocessing before the
standard merge routines are called.}

<17>etree-nat
PREFIX (SYMBOL): Name of the Proof [*****]>x2106
NUM (LINE): Line Number for Theorem [100]>
TAC (TACTIC-EXP): Tactic to be Used [COMPLETE-TRANSFORM-TAC]>
MODE (TACTIC-MODE): Tactic Mode [AUTO]>

{\it Finally, we translate the proof into natural deduction form.}
\end{tpsexample}


\chapter{Unification}
There's a separate top-level for unification. In this
chapter we assume familiarity with Huet's paper on higher-order unification.
We'll follow the notation used in that paper.

The main data structure associated with the unification top-level is the
{\tt unification tree}. The set\footnote{Actually this should be a multiset as
no attempt is made to eliminate elements which are repeated. But for notational
convenience, we'll assume that we have a set. This does not affect the
unifiers, but just adds a certain amount of redundancy in the computation.}
of disagreement pairs at the root node of this
tree is the unification problem associated with the tree. A leaf node is a node
with no sons, while a terminal node is a node which is either a success node or
a failure node.

This top-level can be used for building unification trees in interactive mode,
automatic mode and combinations of these modes. It provides
facilities for moving through, building, and displaying the unification tree.
It also allows the user to specify substitutions at non terminal leaf nodes.
There are commands for building disagreement sets, and starting new
unification problems. The user is also allowed to add disagreement sets
at arbitrary leaf nodes. Although this modifies the initial unification
problem, it allows one to study the unification problems where new
constraints are added to the initial unification problem in an incremental
way.

One can enter this top-level by using the command
\indexmexpr{UNIFY}. However, if this top-level is entered from
mating-search, it is assumed that the user intends to look at the
unification problem associated with the active mating, and the unification
tree for this top-level is initialized to the unification tree associated with
the active mating. Note that any changes to this tree will affect the solution
to the unification problem for the active mating. Note also that the unification
top level is designed to work with unification trees generated by the MS88
procedures using depth bounds (see below); the path-focused procedures and the
\indexflag{MAX-SUBSTS} procedures (see below) use a slightly  different structure
for their unification trees.

\section{A Few Comments About Higher-Order Unification}
\begin{itemize}
\item In first-order unification substitutions are generated on the basis of the
disagreement set. In higher-order logic on the other hand,
the substitutions are formed in a generate and test fashion using very simple
substitutions.

\item Need to identify mgu's whenever possible.

\item Variables of type higher than type of any variable in the original problem are
introduced.

\item Types of elements in the disagreement pair rise during the process of
generating new disagreement pairs in the call to {\it simpl}. Some redundancy
is introduced by having elements in the binder which are not free in either
element of the disagreement pair.
\end{itemize}

\section{Bounds on Higher-Order Unification}
Since higher-order unification cannot be guaranteed to terminate, it is necessary
to have a way to decide when to abandon the search for a unifier. {\TPS} has two
basic methods for doing this: by the depth of the tree or by the complexity of
the substitutions that are being generated.

\subsection{Depth Bounds}
This method involves choosing a depth below which unification trees will not be
allowed to grow. This depth is governed by the flags \indexflag{MAX-SEARCH-DEPTH} and
\indexflag{MAX-UTREE-DEPTH}. If these depths are set too high (particularly if rigid path
checking is not in use; see the flag \indexflag{RIGID-PATH-CK} for details), then {\TPS} will
waste a lot of time running down useless branches of the unification tree. If they are set too low,
then {\TPS} will not find the unifier at all.

Complicating the depth-bound method (in MS88 only) is the flag \indexflag{MIN-QUICK-DEPTH}. As each
new potential connection is considered, {\TPS} first tries to check whether the connection
itself is unifiable, before adding it to the mating. This is called `quick unification'.
Attempting to unify a single connection all the way down to \indexflag{MAX-SEARCH-DEPTH}
can potentially waste a lot of time, and so the
unification tree is only generated until it branches at a depth below \indexflag{MIN-QUICK-DEPTH}.
(That is to say, it is generated to the depth \indexflag{MIN-QUICK-DEPTH}, if possible,
and then unification continues until the matching routine returns more than one substitution at a
given node, at which point that node is marked as `possibly unifiable' and not unified any further.)
Quick unification of a connection will mark that connection as acceptable if any of the leaves of
the resulting tree are either possibly or definitely unifiable.

The drawbacks of the depth-bound method are that small dpairsets are given as much time and space for
unification as large dpairsets, and that a single connection is very rarely rejected; it would have to
fail outright, or if \indexflag{MIN-QUICK-DEPTH} were equal to \indexflag{MAX-SEARCH-DEPTH} it could be
rejected if it contained no success nodes below \indexflag{MAX-SEARCH-DEPTH}. Of course, merely having a
success node down near \indexflag{MAX-SEARCH-DEPTH} isn't enough; we also need there to be enough space to
unify all the other dpairs of the eventual mating.

The advantages are that depth bounds are more precise than substitution bounds. In the case where a complete
mating has many dpairs, one requiring a large number of substitutions, and many others which can be unified to
arbitrary depth but in fact only need be unified to minimal depth for this problem, it is possible for
the unification tree generated by depth bounds to be much smaller than that generated by substitution bounds.
It seems that there are not many `naturally occurring' cases like this, however.

\subsection{Substitution Bounds}
This method involves setting a maximum number of matching substitutions which can be applied to
a given variable in the initial problem. This is governed by the flag \indexflag{MAX-SUBSTS-VAR}.
For example, if the variable {\it x} occurs in the initial problem, and we make a matching substitution
for it, we will introduce a number of {\it h}-variables {\it h1,...,hn} for which we may then make further substitutions.
We then consider Substs({\it x}), the number of substitutions made for {\it x} by the time we reach a particular
node with a given substitution stack, to be 0 if we never make such a substitution, and
otherwise 1 + Substs({\it h1}) + ... + Substs({\it hn}).

The flag \indexflag{MAX-SUBSTS-VAR} is the maximum value which Substs({\it x}) is allowed to take.
The flag \indexflag{MAX-SUBSTS-QUICK} is the maximum value which it may take during quick unification (see the
previous subsection). Nodes that are about to exceed \indexflag{MAX-SUBSTS-QUICK} but do not exceed
\indexflag{MAX-SUBSTS-VAR} are marked as `possibly unifiable' as above.

Notice that this gives us several advantages over the depth-bound method. Firstly, small dpairsets are given less
space and time than large dpairsets. Secondly, if \indexflag{MAX-SUBSTS-QUICK} equals \indexflag{MAX-SUBSTS-VAR}
then during quick unification we will never get a `possibly unifiable' node; all connections are known to be
either unifiable or not (taken by themselves, of course; a unifiable connection may still never be unifiable
in the context of a complete mating). This allows us to reject individual connections more often, without unifying them
all the way to a large maximum depth bound.

Furthermore, a slightly modified version of a theorem is likely to require the same settings for substitution bounds
as the original problem (on the assumption that the proof is likely to be similar), whereas it will very probably
require different unification bounds. Lastly, substitution bounds take a smaller possible range of values
than depth bounds (we have yet to find a {\TPS} theorem which requires a substitution depth of more than 6).

There are also flags \indexflag{MAX-SUBSTS-PROJ} and \indexflag{MAX-SUBSTS-PROJ-TOTAL}, which restrict the number
of projection substitutions allowed for each variable and the entire problem, respectively. It seems that these
flags have very little effect on the speed of the proof, and may as well remain NIL.

\subsection{Combining the Above}

Any of the flags above may be set to NIL. (Of course, enough of them should be non-NIL that there is no risk of
a unification problem never terminating.) In particular, you can opt to use just depth bounds, or just substitution
bounds; the commands \indexcommand{UNIF-DEPTHS} and \indexcommand{UNIF-NODEPTHS} set the flags for these two cases.
In general, substitution bounds are faster, although there are some examples where a depth bound is useful.

The only two flags above that cannot work together are \indexflag{MIN-QUICK-DEPTH} and \indexflag{MAX-SUBSTS-QUICK}.
In this case, \indexflag{MAX-SUBSTS-QUICK} overrides \indexflag{MIN-QUICK-DEPTH}. In fact,
\indexflag{MAX-SUBSTS-QUICK} overrides a number of the less effective unification flags: see the help message for
more details.

Notice that \indexflag{MAX-SUBSTS-QUICK} has an eccentric value, 0, which means `unify until either the tree
branches or we exceed \indexflag{MAX-SUBSTS-VAR}'. This is occasionally useful, and comparable to
\indexflag{MIN-QUICK-DEPTH} being 1, although it doesn't fit the general description of \indexflag{MAX-SUBSTS-QUICK}
given above.

MS88 unification using \indexflag{MAX-SUBSTS-QUICK} is significantly different to that without it, in
that it is quicker and uses less space, by neglecting to store all the irrelevant parts of the tree. (Unification
for the path-focused procedures does something similar.) The user
will not notice the difference unless he or she enters the Unification top level after an MS88, MS89 or MS91-6
search involving \indexflag{MAX-SUBSTS-QUICK}. To recover a usable unification tree under these circumstances,
enter the unification top level and type \indexcommand{EPROOF-UTREE} followed by MAX-SUBSTS-QUICK NIL and then
GO. A new unification tree will be generated.

\section{Support Facilities}

\subsection{Review}
The subject {\it unification} lists flags that affect the behavior of the
unification commands. See the chapter on {\tt review} commands for details on how
to modify these flags.

\subsection{Saving Disagreement sets}
Disagreement sets can be saved using the facilities provided by the
library top-level. The library object type {\it dpairset} represents set of
disagreement pairs. See chapter \ref{library} for details on how to
save and retrieve library objects.

\section{Unification Tree}
Each node in an MS88 unification tree has the following attributes\footnote{It has
certain other attributes which are convenient for implementation purposes.}:

\begin{description}
\item[dpairs]	 The set of disagreement pairs. Each element of these disagreement pairs
is in head normal form.

\item[print-name]	 A name given to the node for identification purposes.

\item[subst-stack]	 A stack of unit substitutions.

\item[free-vars]	 List of free variables in the disagreement pairs at this node.

\item[sons]	 List of descendents of this node.

\item[depth]	 The depth of this node in the tree. The root is at depth 1.

\item[measure]	 A measure associated with this node. This controls the search strategy
that is being used to find the next node in the unification tree that should
be considered.
\end{description}

In the unification procedures we are discussing, which are associated with MS88, the
entire tree is stored and the nodes of the tree have over 15 attributes, of which
those listed above are the most important.
By contrast, in the unification procedures for path-focused duplication,
the dpairs, sons, substitutions and depth are the only major attributes of a unification
node, and all that is stored is the root node and the currently active leaf nodes, which
are all considered the immediate sons of the root.

\subsection{Node Names}
The root node of this tree is named `0'. The sons of the node with name
`N' are named depending on the number m of sons.

\begin{itemize}
\item If m = 0, then the unique son is named `N-0'

\item else the m sons are named `N-1', ..., `N-m'
\end{itemize}

\subsection{Substitution Stack}
The substitution at any node is maintained as a stack of simpler
substitutions. The composition of the simpler substitutions of this stack
is the substitution associated with that node. Note that there are no
cycles in this stack. Although the user has the option of adding
more substitutions to a stack, the system will not let the user add
substitutions which make this stack cyclic. For example:

\begin{tpsexample}
Assume that the substitution stack has a single element $(x . y)$. The
user will not be allowed to add the substitution \(y \leftarrow x\).
\end{tpsexample}

\section{Simpl}
The command {\tt simpl} is a call to the function {\it simpl} in Huet's algorithm.
Some additions are:

\begin{enumerate}
\item In the presence of the ETA-RULE, we modify the binders and arguments of the
elements of rigid-rigid disagreement pairs so that the binders have the same
length. This way we obviate the necessity of finding eta head normal form,
keep the binder reduced to a certain extent, and do not form some disagreement
pairs which would have been immediately eliminated anyway.

\item As noted by Huet, if \(x \notin FV(e)\) then \(x \leftarrow e\) is a mgu
for this pair of terms, where \(x\) is a variable, \(e\) is a term
of the same type, and \(FV(e)\) denotes the set of variables that are free
in \(e\). We will find substitutions of this kind. We intend to
implement the rigid path check to identify non-unifiable disagreement sets here.

\item When {\it simpl} generates new disagreement pairs, we delete any element in the
binder that does not occur free in one of the elements of the disagreement
pair that is being formed. This allows us to reduce the types of the
disagreement pairs that are generated without any loss of unifiers.
\end{enumerate}


\section{Match}
As noted by Dale Miller, the substitutions generated by {\it match} in the presence
of eta-rule depend only on the {\bf rigid head} and {\bf type of flexible head}.
Hence, the same substitutions can be used at different nodes in the tree.
This may require some renaming of h-vars to assure that the h-vars in these
substitutions do not occur free in a substitution at any ancestor node of the
current node. We generalize this result slightly to handle the case where
eta-rule is not available.

Although our flexible-rigid pairs may not be in eta head normal form,
in the presence of eta-rule
the substitutions generated for these pairs will be exactly those that
would be generated if the pairs were in eta head normal form. This is possible
due to Miller's observation mentioned above.

\section{Comments}
The following modifications aid to identify certain substitutions:

\begin{enumerate}
\item The modifications in {\it simpl} and {\it match} mentioned above
obviate the need to compute eta head normal form, and it suffices to find
the head normal form of elements in the disagreement pairs.
For example, consider the disagreement pair:
$$ (f_{{ii}} , \lambda y_{{i}}  G_{{ii}}^{{n}}  y).$$ 
We can straight away
find the mgu for this dpair. If, however, we convert this to eta head normal
form, then we have to call the unification algorithm to find the unifiers.

\item Reducing the binder during generation of new disagreement pairs. For example,
consider the disagreement pair:
$$(\lambda x_{{i}} f_{{ii}}  , \lambda z_{{i}}\lambda y_{{i}}  G_{{ii}}^{{n}}  y).$$
This reduces to the following disagreement pair:
$$(f_{{ii}}  ,  \lambda y_{{i}}  G_{{ii}}^{{n}}  y)$$
\end{enumerate}

\section{A Session in Unification Top-Level}
\begin{alltt}
<0>{\tt unify}

<Unif0>{\tt ?}
Top Levels:    LEAVE
Unification:   0 APPLY-SUBST GO GOTO MATCH MATCH-PAIR NTH-SON P PALL
               PP SIMPLIFY SUBST-STACK UTREE \(\wedge\) \(\wedge\)\(\wedge\)
Dpairs:        ADD-DPAIR ADD-DPAIRS-TO-NODE ADD-DPAIRS-TO-UTREE
               RM-DPAIR SHOW-DPAIRSET UNIF-PROBLEM

<Unif1>{\tt add-dpairset}
NAME (SYMBOL): Name of set containing dpair [No Default]>{\tt foo}
ELT1 (GWFF): First element [No Default]>{\tt `f x'}
ELT2 (GWFF): Second Element [No Default]>{\tt `A'}

<Unif2>{\tt add-dpairset}
NAME (SYMBOL): Name of set containing dpair [FOO]>
ELT1 (GWFF): First element [No Default]>{\tt `f y'}
ELT2 (GWFF): Second Element [No Default]>{\tt `B'}

<Unif3>{\tt show-dpairset}
NAME (SYMBOL): Name of disagreement set. [FOO]>
f y  .  B
f x  .  A

<Unif4>{\tt unif-problem}
NAME (SYMBOL): Name of disagreement set [FOO]>
FREE-VARS (GVARLIST): List of free variables. [()]>{\tt `f(II)' `x' `y'}

<Unif5>{\tt go}
0  0-1  0-2
Substitution Stack:

\(f   ->    \lambda w^{{0}}  w^{{0}}\)
y   ->   B
x   ->   A

<Unif6> {\tt leave}
\end{alltt}

\chapter{Rewrite Rules and Theories}
\label{rewrite}

{\TPS} allows the user to define rewrite rules, and to apply them in interactive and automatic proofs.

Rewrite rules may be polymorphic and/or bidirectional; there may be a function attached to test for
the applicability of the rule (the default is `always applicable'), and there may also be another function
to be applied after the rule is applied - for example, lambda-normalization
(the default is that there is no such function).

Bidirectional rules are considered to `normally' work left-to-right, but
the user will be prompted if there is any ambiguity. (So, for example,
\indexcommand{APPLY-RRULE} and \indexcommand{UNAPPLY-RRULE} are not
distinguishable for a bidirectional rule, unless it has an associated function.)

Theories are basically collections of rewrite rules and gwffs (and possibly some other previously defined subtheories).
They have %only two uses in {\TPS}.
several uses in {\TPS}.
%Firstly,
They can be saved into the library, so the user can define a theory containing all of
the required axioms and rewrite rules and then load it with a single FETCH command.
%Secondly,
The command \indexcommand{USE-THEORY} activates all of the rewrite rules of the
given theory and deactivates all other rewrite rules in memory, allowing the user to switch
easily between different theories.
One can include the currently active rewrite rules as additional premises into
proofs by setting the flag \indexflag{ASSERT-RRULES} (see
\ref{RRulesInAutoProofs}).
Within the rewriting top level (see Section \ref{RewritingTopLevel}), theories
are used to describe rewrite relations. Theories designed solely for use within
the rewriting top level will typically contain no axioms.
When a theory is loaded from the library, {\TPS} also creates an abbreviation
which is the conjunction of the axioms, the universal
closure of the rewrite rules and the abbreviations representing any subtheories
given by the user. This allows the user to have sentences like
`PA IMPLIES [[ONE PLUS ONE] = TWO]', if PA is a theory.

%The top-level commands for manipulating rewrite rules are as follows:

\section{Top-Level Commands for Manipulating Rewrite Rules}
\begin{description}
\item[] \indexcommand{FETCH} gets an rrule from the library.

\item[] \indexcommand{DELETE-RRULE} deletes an rrule from {\TPS}.

\item[] \indexcommand{LIST-RRULES} lists the rrules currently in memory.

\item[] \indexcommand{MAKE-ABBREV-RRULE} creates an rrule from an abbreviation (eg EQUIVS, above).

\item[] \indexcommand{MAKE-INVERSE-RRULE} creates a new rule which is the inverse of an old rule.

\item[] \indexcommand{MAKE-THEORY} creates a new theory, which can be saved in the library.

\item[] \indexcommand{PERMUTE-RRULES} reorders the list of rrules in {\TPS}. (By default, {\TPS} will try to apply the
active rules in the order in which they are listed.)

\item[] \indexcommand{REWRITE-SUPP1} does one step of rewriting in an ND proof (see APPLY-ANY-RRULE).

\item[] \indexcommand{REWRITE-SUPP*} does many steps in an ND proof (see APPLY-ANY-RRULE*).

\item[] \indexcommand{UNREWRITE-PLAN*} and \indexcommand{UNREWRITE-PLAN1} are the same but in the other direction.
 Because the last four can get confusing, we also have:

\item[] \indexcommand{SIMPLIFY-PLAN}, \indexcommand{SIMPLIFY-PLAN*}, \indexcommand{SIMPLIFY-SUPP} and \indexcommand{SIMPLIFY-SUPP*}
 which attempt to simplify a plan or support line assuming that the left-to-right
  direction is `simplification', and that the higher-numbered lines should
  always be rewrite instances of the lower-numbered lines.
  Now that we have bidirectional rules, this makes sense.
  All of the ND commands use active rules only. All rules are active when
  first loaded/defined, and remain active until deactivated by the user.

\item[] \indexcommand{ACTIVATE-RULES} and \indexcommand{DEACTIVATE-RULES} allow you to turn rules on and off.

\item[] \indexcommand{USE-THEORY} activates all the rewrite rules in the given theory and deactivates
  all other rules. If the theory is unknown, {\TPS} will attempt to load it from the library.

\item[] \indexcommand{USE-RRULES}, and its associated wffop \indexother{INSTANCE-OF-REWRITING}, allow you to deduce
 any wff B(O) from a wff A(O) provided that A rewrites to B *without* using any
 overlapping (nested) rewrite rules. In particular, you can generate one wff from
 the other in the editor, rewriting only those parts you want to rewrite.
\end{description}

%There are also a number of editor operations dealing with rewrite rules:

\section{Editor Operations Dealing with Rewrite Rules}
\begin{description}
\item[] \indexcommand{ARR} is the editor command which applies the first applicable rrule.

\item[] \indexcommand{ARR*} does ARR until it terminates.
  (both of the above use active rules only)

\item[] \indexcommand{ARR1} applies a particular rrule once.

\item[] \indexcommand{ARR1*} applies the same rrule until it terminates.
  (both of the above can use either active or inactive rules).

\item[] \indexcommand{MAKE-RRULE} makes an rrule whose lhs is the current edwff.

\item[] \indexcommand{UNARR}, \indexcommand{UNARR*}, \indexcommand{UNARR1}, \indexcommand{UNARR1*} are the
  editor commands which apply rrules in the reverse direction.
\end{description}
%
%An example of how to use these commands will be given later.

\section{An Example of Rewrite Rules in Interactive Use}

\begin{tpsexample}
<lib25>fetch theo2
TYPE (LIB-ARGTYPE-OR-NIL): argtype or nil [NIL]>
THEO2
{\it We retrieve the theory THEO2 from the library.}
<lib26>help theo2
THEO2 is a theory and a logical abbreviation.
     -----
As a theory:
THEO2 is an extension of (THEO1)
Rewrite rules are: (ONE)
     -----
As a logical abbreviation:

THEO1 \(\and\) \(\one\) = SUCC\(\sb{\greeks\greeks}\) ZERO\(\sb{\greeks}\)
{\it THEO2 contains THEO1 as a subtheory; let's look at that.}
<lib27>help theo1
THEO1 is a theory and a logical abbreviation.
     -----
As a theory:
theory of arithmetic, sort of
Rewrite rules are: (ADD-A ADD-B)
     -----
As a logical abbreviation:
theory of arithmetic, sort of
\(\forall\)x\(\sb{\greeks}\) [x + ZERO\(\sb{\greeks}\) = x] \(\and\) \(\forall\)y\(\sb{\greeks}\) \(\forall\)x.x + SUCC\(\sb{\greeks\greeks}\) y = SUCC.x + y
{\it THEO1 is the usual theory of ZERO, SUCC and +}
<lib28>leave
<29>list-rrules
Currently defined rewrites are: (ONE ADD-B ADD-A)
These are defined as follows:
ONE : \(\one\) <--> SUCC\(\sb{\greeks\greeks}\) ZERO\(\sb{\greeks}\)
ADD-B : x\(\sb{\greeks}\) + SUCC\(\sb{\greeks\greeks}\) y\(\sb{\greeks}\) <--> SUCC\(\sb{\greeks\greeks}\).x\(\sb{\greeks}\) + y\(\sb{\greeks}\)
ADD-A : x\(\sb{\greeks}\) + ZERO\(\sb{\greeks}\) <--> x\(\sb{\greeks}\)

Of these, (ONE ADD-B ADD-A) are active.
{\it We see that there are now three rewrite rules, all active and all bidirectional.}
<30>ed sum1
<Ed31>p
SUCC\(\sb{\greeks\greeks}\) ZERO\(\sb{\greeks}\) + ZERO +.SUCC [SUCC.SUCC ZERO] + SUCC [SUCC.SUCC ZERO] + ZERO
{\it Now let's try to reduce this expression.}
<Ed32>arr*
Apply bidirectional rules in the forward direction? [Yes]>
SUCC\(\sb{\greeks\greeks}\).SUCC.SUCC.SUCC.SUCC.SUCC.SUCC ZERO\(\sb{\greeks}\)
{\it That was perhaps a little quick; let's do it again more slowly.}
<Ed33>sub sum1
SUCC\(\sb{\greeks\greeks}\) ZERO\(\sb{\greeks}\) + ZERO +.SUCC [SUCC.SUCC ZERO] + SUCC [SUCC.SUCC ZERO] + ZERO
<Ed34>arr
Apply bidirectional rules in the forward direction? [Yes]>
SUCC\(\sb{\greeks\greeks}\) ZERO\(\sb{\greeks}\) + ZERO +.SUCC [SUCC [SUCC.SUCC ZERO] + SUCC.SUCC ZERO] + ZERO
{\it Here we took the first applicable rewrite rule.}
<Ed35>arr1*
RULE (SYMBOL): name of rule [No Default]>add-b
Apply rule in the forward direction? [Yes]>
SUCC\(\sb{\greeks\greeks}\) ZERO\(\sb{\greeks}\) + ZERO +.SUCC [SUCC.SUCC.SUCC [SUCC.SUCC ZERO] + ZERO] + ZERO
{\it Then we apply ADD-B as much as possible.}
<Ed36>arr1*
RULE (SYMBOL): name of rule [No Default]>add-a
Apply rule in the forward direction? [Yes]>
SUCC\(\sb{\greeks\greeks}\) ZERO\(\sb{\greeks}\) + SUCC.SUCC.SUCC.SUCC.SUCC.SUCC ZERO
{\it Then ADD-A...}
<Ed37>arr1*
RULE (SYMBOL): name of rule [No Default]>add-b
Apply rule in the forward direction? [Yes]>
SUCC\(\sb{\greeks\greeks}\).SUCC.SUCC.SUCC.SUCC.SUCC.SUCC ZERO\(\sb{\greeks}\) + ZERO
{\it Then ADD-B, and clearly one more ADD-A would do it}
<Ed38>arr1*
RULE (SYMBOL): name of rule [No Default]>ONE
Apply rule in the forward direction? [Yes]>no
SUCC\(\sb{\greeks\greeks}\).SUCC.SUCC.SUCC.SUCC.SUCC.\(\one\) + ZERO\(\sb{\greeks}\)
{\it ...but instead we opt to apply ONE from right to left.}
<Ed39>ok
<40>prove sum3
SUM3 is not a known gwff. Search for it in library? [Yes]>
PREFIX (SYMBOL): Name of the Proof [No Default]>sum3
NUM (LINE): Line Number for Theorem [100]>
(100)  \(\assert\)   \(\one\) + \(\one\) + [ZERO\(\sb{\greeks}\) + SUCC\(\sb{\greeks\greeks}\) \(\one\)] + SUCC [ZERO + \(\one\)]
          = SUCC .SUCC .SUCC .SUCC .SUCC .SUCC ZERO & PLAN1
{\it Now let's try to prove a theorem with these rewrite rules.}
<41>simplify-plan*
P2 (LINE): Line after rewriting (higher-numbered) [100]>
P1 (LINE): Line before rewriting (lower-numbered) [99]>
(99)   \(\assert\)   SUCC\(\sb{\greeks\greeks}\) [SUCC .SUCC .SUCC .SUCC .SUCC ZERO\(\sb{\greeks}\)]
          = SUCC .SUCC .SUCC .SUCC .SUCC .SUCC ZERO & PLAN2
{\it All we need do now is ASSERT REFL=. Instead, we'll delete this line and prove it more slowly.}
<42>delete 99
<43>simplify-plan
P2 (LINE): Line after rewriting (higher-numbered) [100]>
P1 (LINE): Line before rewriting (lower-numbered) [99]>
(99)   \(\assert\)   SUCC\(\sb{\greeks\greeks}\) ZERO\(\sb{\greeks}\) + \(\one\) + [ZERO + SUCC \(\one\)] + SUCC [ZERO + \(\one\)]
          = SUCC .SUCC .SUCC .SUCC .SUCC .SUCC ZERO & PLAN4
<44>simplify-plan
P2 (LINE): Line after rewriting (higher-numbered) [99]>
P1 (LINE): Line before rewriting (lower-numbered) [98]>
(98)   \(\assert\)   SUCC\(\sb{\greeks\greeks}\) ZERO\(\sb{\greeks}\) + SUCC ZERO + [ZERO + SUCC \(\one\)] + SUCC [ZERO + \(\one\)]
          = SUCC .SUCC .SUCC .SUCC .SUCC .SUCC ZERO & PLAN5
{\it We could continue in this way, or we can try restricting the range of rewrite rules to be applied, as follows:}
<45>deactivate-rules
RLIST (RRULELIST): Rules to deactivate [(ONE ADD-B ADD-A)]>ADD-B ADD-A
{\it The only rule left to apply is now ONE.}
<46>simplify-plan*
P2 (LINE): Line after rewriting (higher-numbered) [98]>
P1 (LINE): Line before rewriting (lower-numbered) [97]>
(97)   \(\assert\)     SUCC\(\sb{\greeks\greeks}\) ZERO\(\sb{\greeks}\) + SUCC ZERO + [ZERO + SUCC .SUCC ZERO]
            + SUCC [ZERO + SUCC ZERO]
          = SUCC .SUCC .SUCC .SUCC .SUCC .SUCC ZERO & PLAN6
{\it Every instance if ONE is now rewritten to SUCC ZERO}
<47>activate-rules
RLIST (RRULELIST): Rules to activate [(ONE ADD-B ADD-A)]>ADD-B
{\it We reactivate ADD-B, which is the rule taking (x+(SUCC y)) to SUCC(x+y).}
<48>simplify-plan*
P2 (LINE): Line after rewriting (higher-numbered) [97]>
P1 (LINE): Line before rewriting (lower-numbered) [96]>
(96)   \(\assert\)   SUCC\(\sb{\greeks\greeks}\) [SUCC .  SUCC [SUCC .SUCC [SUCC ZERO\(\sb{\greeks}\) + ZERO] + .ZERO + ZERO]
                          + .ZERO + ZERO]
          = SUCC .SUCC .SUCC .SUCC .SUCC .SUCC ZERO & PLAN7
<49>activate-rules
RLIST (RRULELIST): Rules to activate [(ONE ADD-B ADD-A)]>
{\it Finally, we reactivate ADD-A, which takes (x+ZERO) to x.}
<50>simplify-plan*
P2 (LINE): Line after rewriting (higher-numbered) [96]>
P1 (LINE): Line before rewriting (lower-numbered) [95]>
(95)   \(\assert\)   SUCC\(\sb{\greeks\greeks}\) [SUCC .SUCC .SUCC .SUCC .SUCC ZERO\(\sb{\greeks}\)]
          = SUCC .SUCC .SUCC .SUCC .SUCC .SUCC ZERO & PLAN8
<51>assert refl=
LINE (LINE): Line with Theorem Instance [No Default]>95
{\it ...and the proof is done.}
\end{tpsexample}

\section{Using Rewrite Rules in Automatic Proof Search}
\label{RRulesInAutoProofs}

The \indexother{MS98-1} search procedure can extract rewrite rules from wffs
and use them for proof search. To use this feature within \index{MS98-1}, set
the flag \indexflag{MS98-REWRITES}. You may also need to set the flag
\indexflag{MAX-SUBSTS-VAR} to some positive value.

By default the procedure will not use any rewrite rules apart from those it
can extract during proof search. In some cases, though, it may be beneficial
to allow the procedure using additional rewrite rules. This can be done in two
ways.

The recommended way to add rewrite rules is adding them as equational premises
to the main assertion. To add the active rewrite rules as premises to the
assertion of a proof, set the flag \indexflag{ASSERT-RRULES} before beginning
a new proof using \indexcommand{PROVE}. Assume, for instance, you have two
active rewrite rules (of the form \mbox{\texttt{$l_i$ <--> $r_i$}} for
$i\in\{1,2\}$). After setting \indexflag{ASSERT-RRULES}, you start a proof by
entering \texttt{PROVE "\textit{main-assertion}"}. The resulting assertion will
be of the form \texttt{$l_2$ = $r_2$ $\implies$ .$l_i$ = $r_i$ $\implies$ \textit{main-assertion}}.

\indexother{MS98-1} may fail to recognize some of the rules passed by the above
method. This usually indicates that the rule is too complex to be used by the
procedure.
If you want to enforce that all active rewrite rules are indeed used by the
search procedure, you can set the flag \indexflag{MS98-EXTERNAL-REWRITES}.
This flag works independently from the setting of \indexflag{ASSERT-RRULES}.
Normally, i.e. when the flag is not set, \indexother{MS98-1} temporarily
deactivates all active rewrite rules which were not extracted from the
assertion to prove. When \indexflag{MS98-EXTERNAL-REWRITES} is set, the
globally active rewrite rules remain active and are used in addition to the
rules extracted from the proof assertion.

\section{The Rewriting Top Level} \label{RewritingTopLevel}

In many branches of mathematics and computer science relational theories are an
important matter of study. The formal proof technique which often appears most
natural when dealing with equations or other transitive relations (e.g. order
relations) is known as rewriting. As an example of a proof by rewriting,
consider how one would typically prove the uniqueness of inverses in a group
$\langle G,\cdot\rangle$:

Let $a,a',a''\in G$, $a'$ and $a''$ both inverses of $a$. Then
\[\begin{array}{rcl@{\qquad}r}
a' & = & a'\cdot e & (e\textup{ identity of }\langle G,\cdot\rangle)\\
   & = & a'\cdot(a\cdot a'') & (a''\textup{ inverse of }a)\\
   & = & (a'\cdot a)\cdot a'' & (associative law)\\
   & = & e\cdot a'' & (a'\textup{ inverse of }a)\\
   & = & a'' & (e\textup{ identity of }\langle G,\cdot\rangle)
\end{array}\]

The rewriting top level can be used to create and manipulate
complex rewriting derivations in a convenient way.

\subsection{Interacting with the Main Top Level}

There are two ways of entering the rewriting top level from the main
top level. One is by calling \indexcommand{REWRITING}. In case you had been
working on a rewrite derivation before you left the rewriting top level for
the last time, it will return to this derivation. Otherwise no derivation
will be active, so you may want to start a new one or to restore a saved one
from a file. If you want to use the rewriting top level to justify a line
of a natural deduction proof from a preceding line in a way similar to the
application of \indexcommand{USE-RRULES}, you can enter the top level by
calling \indexcommand{REWRITE} or \indexcommand{REWRITE-IN} (See example in
Subsection \ref{RewritingTopLevelExample}). After you
have found a justification, exit the rewriting top level using
\indexcommand{OK} to modify the natural deduction proof accordingly.

Relations derived in the rewriting top level may also be inserted as lemmas
into the main top level using \indexcommand{ASSERT-TOP}.

It is sometimes useful to be able to access lines of the current natural
deduction proof from the rewriting top level and vice versa. This is possible
by using the commands \indexcommand{TOP} and \indexcommand{REW}. Whenever a
wff needs to be supplied to a command within the rewriting top level, you can
type \texttt{(TOP \textit{linenum})} to access the line
\texttt{\textit{linenum}} of the
current natural deduction proof. In exactly the same way \indexcommand{REW}
allows one to access lines of the current rewriting derivation from the main
top level. Of course, both commands can be combined with \indexcommand{ED}.

\subsection{Rewrite Rules, Theories and Derivations}

For technical reasons, the left-hand side and the right-hand side of a rewrite
rule are not allowed to be identical. Since all rewriting commands work modulo
alphabetic change of bound variables, rules with alpha-equivalent sides are
also prohibited. If you use any rules of the prohibited form, some commands
will not work as expected. To assert reflexivity use the command
\indexcommand{SAME} instead.

Once a relation between two wffs has been established, it is possible to
define a derived rewrite rule from the two wffs using the
\indexcommand{DERIVE-RRULE} command. A rule can be derived in more than one
theory at the same time. Unless one decides to change a theory to include the
derived rule, the rule is not part of any theory in which it was derived.
However, when a derived rule is loaded from the library, it is added to the
run-time representation of those theories in memory in which it was derived.
So, after a derived rule has been loaded, it can be used from any theory which
has been loaded before the rule and in which it is derivable in the same way as
if it was part of the core theory.

Commands of the rewriting top level make certain assumptions about the
structure of rewrite theories.
\begin{enumerate}
\item Rewrite theories describe transitive relations.
\item All subtheories of a theory which share their relation sign with the main
  theory are reflexive iff the main theory is reflexive.
\item All subtheories of a theory are ``compatible'' with the main theory, in
  the following sense: If $R$ is the relation described by the main theory
  and $r$ the one described by a subtheory, then
  \begin{enumerate}
  \item $A~R~B$ and $B~r~C$ imply $A~R~C$,
  \item $A~r~B$ and $B~R~C$ imply $A~R~C$.
  \end{enumerate}
\end{enumerate}

The rewriting top level allows having multiple derivations in memory at the
same time, some of which may use different rewrite theories. To avoid
confusion, every rewrite derivation can be associated with a rewrite theory.
The theory associated with the current rewrite derivation is called the
``current theory''. This notion is to be distinguished from the ``active
theory'', i.e. the theory to be associated with new derivations.

At any point in time there may be at most one active theory. To display
the currently active theory use \indexcommand{ACTIVE-THEORY}. The commands
\indexcommand{USE-THEORY} and \indexcommand{DEACTIVATE-THEORY} can be used
to change the active theory.
When a derivation in the rewriting top level is started by calling
\indexcommand{PROVE} or \indexcommand{DERIVE} from the rewriting top level, or
using the \indexcommand{REWRITE} command from the main top level, and an
active theory is defined, it
becomes associated with the new derivation. Applying any rules which are
not part of the current theory, i.e. the theory associated with the current
derivation, will raise an error. On the other hand, it is possible to use
\indexcommand{APP} and similar commands with rules not in the active theory as
long as they are in the current theory.
To display the current theory use \indexcommand{CURRENT-THEORY}.
Changing the active theory, which can be done at any time both from the main
top level and from the rewriting top level, %by calling
%\indexcommand{USE-THEORY} or \indexcommand{DEACTIVATE-THEORY}
%once a derivation is in progress
will never affect the theory associated with an existing derivation.

If one wants to start a derivation in a theory different from the currently
active theory, one can use \indexcommand{PROVE-IN} or \indexcommand{DERIVE-IN}
from the rewriting top level or \indexcommand{REWRITE-IN} from the main top
level. These commands expect the theory to associate with the newly created
derivation as an additional parameter. If there is a currently active theory,
i.e. if \indexcommand{ACTIVE-THEORY} does not return NIL, it will not be
affected by these commands. If there is no currently active theory, the theory
passed to \indexcommand{PROVE-IN} or \indexcommand{REWRITE-IN} will become
active.

%If a theory makes use of library constants, abbreviations or other objects
%which are not loaded by default when \TPS starts up, \indexcommand{PROVE-IN}
%and \indexcommand{DERIVE-IN} may not work as expected.

\subsection{Automatic Search} \label{RewritingTopLevelAuto}
The command \indexcommand{AUTO} can be used to search for a rewrite sequence
between two specified lines automatically. Dependent on the setting of the
flag \indexflag{REWRITING-AUTO-SEARCH-TYPE}, one of the following search
algorithms is used:
\begin{description}
\item [\texttt{SIMPLE}:] Iterative deepening starting from the source wff (i.e.
  the wff in the lower-numbered line). The procedure uses a hash table, called
  `search table', for cycle and dead-end detection.
\item [\texttt{BIDIR}:] Bidirectional search using iterative deepening,
  starting from the source and the target wff. Two search tables of equal
  maximum size are used.
\item [\texttt{BIDIR-SORTED}:] As \texttt{BIDIR}, but rewriting shorter wffs
  first. This procedure is used by default because it is most likely to find a
  result within reasonable time.
\end{description}
Automatic search uses active rewrite rules from the current theory, or all
active rewrite rules if the current derivation is not associated with any
theory.

See the description of flags starting with \texttt{REWRITING-AUTO} in
Subsection \ref{RewritingTopLevelFlags} to learn how to tune different
parameters of the search.

%\subsection{Rewriting Top Level Commands and Flags}

%The rewriting top level commands are as follows:
%\begin{description}
%\item[]
\subsection{Commands for Entering and Leaving the Rewriting Top Level}
  \begin{description}
  \item[] \indexcommand{REWRITING} Enter the rewriting top level without
    starting a new rewriting derivation.
  \item[] \indexcommand{REWRITE} Enter the rewriting top level to search for a
    rewrite sequence justifying a step in the current natural deduction proof.
    The new derivation will use the currently active theory.
  \item[] \indexcommand{REWRITE-IN} Same as \indexcommand{REWRITE}, but
    prompting for the rewrite theory to use with the new derivation.
  \item[] \indexcommand{ASSERT-TOP} Leave the rewriting top level, inserting
    the derived relation as a lemma into the current natural deduction proof.
    See also \indexcommand{ASSERT} and \indexcommand{ASSERT2}.
  \item[] \indexcommand{BEGIN-PRFW} Enter the \texttt{REW-PRFW} top level to
    open proofwindows.
    Alternatively, one can enter the \texttt{PRFW} top level from the main
    top-level and switch to the rewriting top level afterwards.
  \item[] \indexcommand{END-PRFW} Leave the \texttt{REW-PRFW} top level.
  \item[] \indexcommand{LEAVE} Leave the rewriting top level.
  \item[] \indexcommand{OK} Leave the rewriting top level, passing the proven
    relation as a justification of a rewrite step in the current natural
    deduction proof. This command is only applicable if the current derivation
    was started from the main top level using \indexcommand{REWRITE} or
    \indexcommand{REWRITE-IN}.
  \end{description}

%\item[]
\subsection{Commands for Starting, Finishing and Printing Rewrite Derivations}
  \begin{description}
  \item[] \indexcommand{DERIVE} Begin a new rewrite derivation without a fixed
    target wff, associating with it the currently active theory.
  \item[] \indexcommand{DERIVE-IN} Same as \indexcommand{DERIVE}, but prompting
    for the rewrite theory to use with the new derivation.
  \item[] \indexcommand{DONE} Check whether the current derivation is complete.
  \item[] \indexcommand{PALL} Works the same as in the main top level.
  \item[] \indexcommand{PROOFLIST} Print a list of all rewrite derivations
    currently in memory. For proofs, the corresponding proof assertions are
    printed. For general derivations, the corresponding initial lines are
    printed.
  \item[] \indexcommand{PROVE} Start a new rewriting proof of a given
    relational assertion, using the currently active rewrite theory.
  \item[] \indexcommand{PROVE-IN} Same as \indexcommand{PROVE}, but prompting
    for the rewrite theory to use.
  \item[] \indexcommand{RECONSIDER} Switch the current rewrite derivation.
    Works the same as in the main top level.
  \item[] \indexcommand{RESTOREPROOF} Load a rewriting derivation from a file
    and make it the current derivation. If the derivation was obtained in a
    theory which is not yet in the memory, the command will offer to load the
    theory from the library.
  \item[] \indexcommand{SAVEPROOF} Save the current derivation to a file.
    Works the same as in the main top level.
  \item[] \indexcommand{TEXPROOF} Works the same as in the main top level.
  \end{description}
%\item[] Printing commands:
%\indexcommand{PALL}, \indexcommand{TEXPROOF}. Work
%  the same as in the main top level.

%\item[]
\subsection{Commands for Applying Rewrite Rules}
  \begin{description}
  \item[] \indexcommand{ANY} Try to apply any active rewrite rule from the
    current theory and all its subtheories. If there is no current theory, all
    active rules will be tried.
  \item[] \indexcommand{ANY*}/\indexcommand{UNANY*} Justify a line by a
    sequence of applications of any active rewrite rules from the current
    theory in the forward/backward direction, starting from a preceding line.
    In most cases, this command will apply rewrite rules in the corresponding
    direction as often as possible or until a specified target wff is obtained.
    If the wff after rewriting is specified but the one before rewriting is set
    to NIL, rewrite rules will be applied in the corresponding reverse
    direction, starting from the target formula. The command will add no
    intermediate lines to the derivation outline.\\
    The commands may not terminate if \indexflag{APP*-REWRITE-DEPTH} is set to
    NIL.
  \item[] \indexcommand{ANY*-IN}/\indexcommand{UNANY*-IN} Same as
    \indexcommand{ANY*}/\indexcommand{UNANY*}, but will only try rules from
    the specified subtheory of the current theory.
  \item[] \indexcommand{APP} Apply the specified rewrite rule from the current
    theory or any of its subtheories. The rule need not be active.\\
    Note: Entering the name of a rewrite rule \texttt{\textit{rrule}} at the
    top-level command prompt is considered to be an abbreviation for
    \texttt{APP \textit{rrule}}.
  \item[] \indexcommand{APP*}/\indexcommand{UNAPP*} Same as
    \indexcommand{ANY*}/\indexcommand{UNANY*}, but using the specified single
    rule, which need not be active.
  \item[] \indexcommand{AUTO} Search for a rewrite sequence between two lines
    using any active rewrite rules from the current theory. The exact behaviour
    is affected by following flags: \indexflag{REWRITING-AUTO-DEPTH},
    \indexflag{REWRITING-AUTO-MIN-DEPTH},
    \indexflag{REWRITING-AUTO-TABLE-SIZE},
    \indexflag{REWRITING-AUTO-MAX-WFF-SIZE}, \indexflag{REWRITING-AUTO-SUBSTS}.
    See Subsection \ref{RewritingTopLevelAuto}.
  \item[] \indexcommand{SAME} Use reflexivity of equality. Works almost the
    same as in the main top level, but allows alphabetic change of bound
    variables.
  \end{description}

%\item[]
\subsection{Commands for Rearranging the Derivation}
  \begin{description}
  \item[] \indexcommand{CLEANUP}, \indexcommand{DELETE},
    \indexcommand{INTRODUCE-GAP}, \indexcommand{MOVE}, \indexcommand{SQUEEZE}.
    Work almost the same as in the main top level. \indexcommand{DELETE}
    cannot be used to delete the initial or the target line of a derivation.
    Neither \indexcommand{INTRODUCE-GAP} nor \indexcommand{MOVE} will move the
    initial line of a derivation.
  \item[] \indexcommand{CONNECT} Given two identical lines, delete the lower
    one, rearranging the derivation appropriately. With symmetric relations,
    the command will also rearrange the lines from which the higher-numbered
    line was obtained to follow from the lower-numbered line.

    The main motivation behind \indexcommand{CONNECT} is a situation which is
    quite common when doing equational reasoning. Starting from the source and
    from the target wff, you obtain two derivations which have a line in
    common. By symmetry and transitivity of equality, you can combine the
    two subderivations into a single formal proof. \indexcommand{CONNECT}
    provides an efficient way of doing this transformation.

    More precisely, the behaviour of \indexcommand{CONNECT} can be described
    as follows:\\
    Assume \indexcommand{CONNECT} is applied to the lower-numbered line $p_1$
    and the higher-numbered $p_2$.
    \begin{itemize}
    \item If $p_2$ is the fixed target line of a derivation, or if the first
      line of the rewrite sequence justifying $p_2$ occurs before $p_1$,
      nothing will be done.
    \item If $p_2$ is justified by a rewrite sequence involving $p_1$, or if
      the derivation of $p_2$ involves directed rewrite rules, $p_2$ will be
      deleted. Lines which are justified by $p_2$ will be changed to refer to
      $p_1$ instead.
    \item If $p_1$ is not part of the derivation of $p_2$ and the derivation of
      $p_2$ was obtained using bidirectional rules only, the same as in the
      previous case will happen, but additionally the derivation of $p_2$ will
      be reversed, i.e. the sequence of lines $p_1\ldots d_1\ldots d_n\ldots
      p_2$ will be changed to $p_1\ldots d_n\ldots d_1$. Since in the former
      sequence $d_1$ has to be a planned line, this transformation will result
      in a derivation outline which has one planned line less than the initial
      outline.
    \end{itemize}
  \end{description}

%\item[]
\subsection{Lambda Conversion Commands}
  \begin{description}
  \item[] \indexcommand{BETA-EQ}, \indexcommand{ETA-EQ},
    \indexcommand{LAMBDA-EQ}. Assert the corresponding equivalence between two
    lines.
  \item[] \indexcommand{BETA-NF}, \indexcommand{ETA-NF},
    \indexcommand{LAMBDA-NF}, \indexcommand{LONG-ETA-NF}. Compute the
    corresponding normal form of a line.
  \end{description}

%\item[]
\subsection{Commands Concerned with Rewrite Rules and Theories}
  \begin{description}
  \item[] \indexcommand{CURRENT-THEORY} Show the theory associated with current
    rewrite derivation.
  \item[] \indexcommand{DERIVE-RRULE} Create a derived rewrite rule from two
    provably related lines. If the relation was proven using bidirectional
    rules only, the derived rule may be made bidirectional.
  \item[] \indexcommand{MAKE-RRULE} Create a new rewrite rule in memory.
  \item[] \indexcommand{SAVE-RRULE} Save an existing rewrite rule into the
    library. This is the only way of saving a derived rule such that the rule
    preserves its derived status in the library.
  \end{description}
%\end{description}

\subsection{Applicable Commands from the Main Top Level}
%In addition to the above commands, the following commands from the main
%top level are available:\\
\indexcommand{ACTIVATE-RULES}, \indexcommand{ACTIVE-THEORY},
\indexcommand{DEACTIVATE-RULES}, \indexcommand{DEACTIVATE-THEORY},\\
\indexcommand{DELETE-RRULE}, \indexcommand{LIST-RRULES},
\indexcommand{MAKE-ABBREV-RRULE}, \indexcommand{MAKE-INVERSE-RRULE},
\indexcommand{MAKE-THEORY}, \indexcommand{PERMUTE-RRULES},
\indexcommand{USE-THEORY}, \indexcommand{REVIEW}, \indexcommand{LIB},
\indexcommand{EXIT}.\smallskip

\subsection{Flags} \label{RewritingTopLevelFlags}
%The following flags affect the behaviour of rewriting top level commands:
\begin{description}
\item[] \indexflag{APP*-REWRITE-DEPTH} Determines the maximal rewrite depth of
  an \indexcommand{APP*} application. Used in the same way by
  \indexcommand{UNAPP*}, \indexcommand{ANY*}, \indexcommand{UNANY*},
  \indexcommand{ANY*-IN} and \indexcommand{UNANY*-IN}.
\item[] \indexflag{REWRITING-AUTO-DEPTH} The maximal depth of a search tree
  when applying \indexcommand{AUTO}. For the \texttt{SIMPLE} search procedure,
  the number corresponds to the maximal rewrite depth, whereas for
  \texttt{BIDIR} and \linebreak \texttt{BIDIR-SORTED} the maximal search depth is
  twice the specified number.
\item[] \indexflag{REWRITING-AUTO-GLOBAL-SORT} When NIL, \texttt{BIDIR-SORTED}
  will choose the next wff to be rewritten from the successors of the current
  wff. When T, it will choose the next wff from all unexplored wffs obtained
  so far from the initial or the target wff, respectively. See the flag
  \indexflag{REWRITING-AUTO-SEARCH-TYPE}.
\item[] \indexflag{REWRITING-AUTO-MAX-WFF-SIZE} The maximal size of a wff to be
  rewritten when applying \indexcommand{AUTO}.
\item[] \indexflag{REWRITING-AUTO-MIN-DEPTH} The minimal depth of a search tree
  to be used by AUTO to find a derivation. The value should be less or equal to
  that of REWRITING-AUTO-DEPTH, otherwise no search will be performed.
\item[] \indexflag{REWRITING-AUTO-SEARCH-TYPE} Determines the search algorithm
  used by \indexflag{AUTO}. Currently defined are \texttt{SIMPLE},
  \texttt{BIDIR} and \texttt{BIDIR-SORTED}. \texttt{BIDIR-SORTED} will try to
  rewrite shorter wffs first. When this is not needed, use \texttt{BIDIR}. The
  precise behaviour of \texttt{BIDIR-SORTED} depends on the flag
  \indexflag{REWRITING-AUTO-GLOBAL-SORT}.
\item[] \indexflag{REWRITING-AUTO-SUBSTS} The list of terms to substitute for
  any free variables which may be introduced during rewriting by
  \indexcommand{AUTO}. If NIL, the list will be generated automatically from
  atomic subwffs of the source and the target wff.
\item[] \indexflag{REWRITING-AUTO-TABLE-SIZE} Specifies the maximal size of a
  search table used by \indexcommand{AUTO}. Note that while the \texttt{SIMPLE}
  search procedure uses only one table of the specified size, \texttt{BIDIR}
  and \texttt{BIDIR-SORTED} use two.
\item[] \indexflag{REWRITING-RELATION-SYMBOL} Contains the symbol that is
  printed between lines obtained by rewriting from immediately preceding lines.
  Normally this symbol is the relation sign of the current theory.
  If set explicitly, the symbol will only be used if there is no current
  theory or if the theory has no relation sign associated with it. Also, when
  switching between different rewrite derivations, the value of this flag will
  be changed. The value of the flag is also used by \indexcommand{PROVE} to
  determine the left and the right part of a relational assertion.
\item[] \indexflag{VERBOSE-REWRITE-JUSTIFICATION} When set to T, justification
  of ND proof lines by rewriting in the rewriting top level will indicate the
  rewrite theory used to obtain the justification.
\end{description}

%The rules \indexcommand{USE-THEORY}, \indexcommand{ACTIVE-THEORY} and
%\indexcommand{DEACTIVATE-THEORY} have special

\subsection{Example} \label{RewritingTopLevelExample}

%"FA [LAMBDA x(O).x] = FALSEHOOD"
\begin{tpsexample}
<0>prove "f FALSEHOOD AND f TRUTH IMPLIES f FALSEHOOD AND f TRUTH AND f x"
PREFIX (SYMBOL): Name of the Proof [No Default]>example
NUM (LINE): Line Number for Theorem [100]>
(100) \(\assert\)  f\(\sb{\greeko\greeko}\) \(\bot\) \(\land\) f \(\top\) \(\implies\) f \(\bot\) \(\land\) f \(\top\) \(\land\) f x\(\sb{\greeko}\)                                   PLAN1

<1>deduct !
(1)   1 \(\assert\)  f\(\sb{\greeko\greeko}\) \(\bot\) \(\land\) f \(\top\)                                                      Hyp
(99)  1 \(\assert\)  f\(\sb{\greeko\greeko}\) \(\bot\) \(\land\) f \(\top\) \(\land\) f x\(\sb{\greeko}\)                                             PLAN2

{\it Lines 1 and 99 can be proven equal in the rewrite theory HOL.}

<2>rewrite-in hol
P2 (LINE): Line after rewriting (higher-numbered) [99]>
P1 (LINE): Line before rewriting (lower-numbered) [98]>1
<REWRITING3>pall

(1)      f\(\sb{\greeko\greeko}\) \(\bot\) \(\land\) f \(\top\)
               ...
(100)    f\(\sb{\greeko\greeko}\) \(\bot\) \(\land\) f \(\top\) \(\land\) f x\(\sb{\greeko}\)                                               PLAN1

{\it Let us apply the rule Bin from HOL. It is sufficient to type in the name of the rule.}

<REWRITING4>bin
P1 (LINE): Line before rewriting  (lower-numbered) [1]>
P2 (LINE): Line after rewriting  (higher-numbered) [2]>
B (GWFF-OR-SELECTION): Wff after rewriting
 1)  \(\forall\) f\(\sb{\greeko\greeko}\)

 [1]>
(2)   =  \(\forall\) f\(\sb{\greeko\greeko}\)                                                           Bin: 1

{\it What to do next? We can use ANY to look at some alternatives.}

<REWRITING5>any
P1 (LINE): Line before rewriting  (lower-numbered) [2]>
P2 (LINE): Line after rewriting  (higher-numbered) [3]>
B (GWFF-OR-SELECTION): Wff after rewriting
 1)  [\(\lambda\)\(\sb{\greeko\greeko}\) \(\forall\) u] f\(\sb{\greeko\greeko}\)  (ETA)
 2)  \(\forall\).\(\lambda\)u\(\sb{\greeko}\) f\(\sb{\greeko\greeko}\) u  (ETA)
 3)  \(\forall\) f\(\sb{\greeko\greeko}\) \(\land\) \(\top\)  (AND-ID)
 4)  \(\forall\) f\(\sb{\greeko\greeko}\) \(\lor\) \(\bot\)  (OR-ID)
 5)  f\(\sb{\greeko\greeko}\) = \(\lambda\)u\(\sb{\greeko}\) \(\top\)  (FA-D)
 6)  f\(\sb{\greeko\greeko}\) \(\bot\) \(\land\) f \(\top\)  (BIN)

 [1]>5
(3)   =  f\(\sb{\greeko\greeko}\) = \(\lambda\)u\(\sb{\greeko}\) \(\top\)                                                    Fa-D: 2

<REWRITING6>pall

(1)      f\(\sb{\greeko\greeko}\) \(\bot\) \(\land\) f \(\top\)
(2)   =  \(\forall\) f\(\sb{\greeko\greeko}\)                                                           Bin: 1
(3)   =  f\(\sb{\greeko\greeko}\) = \(\lambda\)u\(\sb{\greeko}\) \(\top\)                                                    Fa-D: 2
               ...
(100)    f\(\sb{\greeko\greeko}\) \(\bot\) \(\land\) f \(\top\) \(\land\) f x\(\sb{\greeko}\)                                               PLAN1

<REWRITING7>and-id
P1 (LINE): Line before rewriting  (lower-numbered) [3]>
P2 (LINE): Line after rewriting  (higher-numbered) [4]>
B (GWFF-OR-SELECTION): Wff after rewriting
 1)  f\(\sb{\greeko\greeko}\) = \(\lambda\)u\(\sb{\greeko}\) \(\top\) \(\land\) \(\top\)
 2)  f\(\sb{\greeko\greeko}\) = \(\lambda\)u\(\sb{\greeko}\).\(\top\) \(\land\) \(\top\)

 [1]>
(4)   =  f\(\sb{\greeko\greeko}\) = \(\lambda\)u\(\sb{\greeko}\) \(\top\) \(\land\) \(\top\)                                              And-Id: 3

{\it Now we want to \(\beta\)-expand the last occurence of \(\top\) into \([\lambda f\sb{\greeko\greeko}.f x].\lambda x\sb{\greeko}.\top\). To save some time, we just specify the desired wff, using AUTO to fill in the gap in the two-step expansion.}

<REWRITING8>auto
P1 (LINE): Line before rewriting  (lower-numbered) [4]>
P2 (LINE): Line after rewriting  (higher-numbered) [100]>6
B (GWFF): Wff after rewriting [No Default]>ed 4
<Ed9>d

\(\top\)
<Ed10>sub "[LAMBDA f.f x]. LAMBDA x(O).TRUTH"

[\(\lambda\)f\(\sb{\greeko\greeko}\) f x\(\sb{\greeko}\)] .\(\lambda\)x \(\top\)
<Ed11>^

f\(\sb{\greeko\greeko}\) = \(\lambda\)u\(\sb{\greeko}\) \(\top\) \(\land\) [\(\lambda\)f f x\(\sb{\greeko}\)] .\(\lambda\)x \(\top\)
<Ed12>ok
Search in progress. Please wait...
Success.
Real time: 0.864435 sec.
Run time: 0.818894 sec.
Space: 6488652 Bytes
GC: 3, GC time: 0.200997 sec.

<REWRITING13>pall

(1)      f\(\sb{\greeko\greeko}\) \(\bot\) \(\land\) f \(\top\)
(2)   =  \(\forall\) f\(\sb{\greeko\greeko}\)                                                           Bin: 1
(3)   =  f\(\sb{\greeko\greeko}\) = \(\lambda\)u\(\sb{\greeko}\) \(\top\)                                                    Fa-D: 2
(4)   =  f\(\sb{\greeko\greeko}\) = \(\lambda\)u\(\sb{\greeko}\) \(\top\) \(\land\) \(\top\)                                              And-Id: 3
(5)   =  f\(\sb{\greeko\greeko}\) = \(\lambda\)u\(\sb{\greeko}\) \(\top\) \(\land\) [\(\lambda\)x\(\sb{\greeko}\) \(\top\)] x                                        Beta: 4
(6)   =  f\(\sb{\greeko\greeko}\) = \(\lambda\)u\(\sb{\greeko}\) \(\top\) \(\land\) [\(\lambda\)f f x\(\sb{\greeko}\)].\(\lambda\)x \(\top\)                                   Beta: 5
               ...
(100)    f\(\sb{\greeko\greeko}\) \(\bot\) \(\land\) f \(\top\) \(\land\) f x\(\sb{\greeko}\)                                               PLAN1

{\it Specifying an appropriate \(\beta\)-expansion was the trickiest part of the derivation. The rest can be done automatically.}

<REWRITING14>auto
P1 (LINE): Line before rewriting  (lower-numbered) [6]>
P2 (LINE): Line after rewriting  (higher-numbered) [100]>
Search in progress. Please wait...
Success.
Real time: 1.682173 sec.
Run time: 1.606185 sec.
Space: 12504376 Bytes
GC: 6, GC time: 0.377588 sec.

<REWRITING15>pall

(1)      f\(\sb{\greeko\greeko}\) \(\bot\) \(\land\) f \(\top\)
(2)   =  \(\forall\) f\(\sb{\greeko\greeko}\)                                                           Bin: 1
(3)   =  f\(\sb{\greeko\greeko}\) = \(\lambda\)u\(\sb{\greeko}\) \(\top\)                                                    Fa-D: 2
(4)   =  f\(\sb{\greeko\greeko}\) = \(\lambda\)u\(\sb{\greeko}\) \(\top\) \(\land\) \(\top\)                                              And-Id: 3
(5)   =  f\(\sb{\greeko\greeko}\) = \(\lambda\)u\(\sb{\greeko}\) \(\top\) \(\land\) [\(\lambda\)x\(\sb{\greeko}\) \(\top\)] x                                        Beta: 4
(6)   =  f\(\sb{\greeko\greeko}\) = \(\lambda\)u\(\sb{\greeko}\) \(\top\) \(\land\) [\(\lambda\)f f x\(\sb{\greeko}\)].\(\lambda\)x \(\top\)                                   Beta: 5
(7)   =  f\(\sb{\greeko\greeko}\) = \(\lambda\)u\(\sb{\greeko}\) \(\top\) \(\land\) [\(\lambda\)f f x\(\sb{\greeko}\)] f                                       Rep: 6
(8)   =  f\(\sb{\greeko\greeko}\) = \(\lambda\)u\(\sb{\greeko}\) \(\top\) \(\land\) f x\(\sb{\greeko}\)                                             Beta: 7
(9)   =  \(\forall\) f\(\sb{\greeko\greeko}\) \(\land\) f x\(\sb{\greeko}\)                                                   Fa-D: 8
(100) =  f\(\sb{\greeko\greeko}\) \(\bot\) \(\land\) f \(\top\) \(\land\) f x\(\sb{\greeko}\)                                              Bin: 9

{\it DONE can be used to check whether a derivation is complete, or to display the reason why the assertion in question has not yet been proved.}

<REWRITING16>done
Derivation complete.

{\it We are done. To exit the rewriting top level, updating the main proof, we use OK.}

<REWRITING17>ok


<18>pall

(1)   1 \(\assert\)  f\(\sb{\greeko\greeko}\) \(\bot\) \(\land\) f \(\top\)                                                      Hyp
(99)  1 \(\assert\)  f\(\sb{\greeko\greeko}\) \(\bot\) \(\land\) f \(\top\) \(\land\) f x\(\sb{\greeko}\)                                   Rewrite(HOL): 1
(100) \(\assert\)  f\(\sb{\greeko\greeko}\) \(\bot\) \(\land\) f \(\top\) \(\implies\) f \(\bot\) \(\land\) f \(\top\) \(\land\) f x\(\sb{\greeko}\)                              Deduct: 99
\end{tpsexample}

\subsection{Semantics of Rewrite Rules}

In order to create own rewrite rules, and to use existing ones efficiently, it
is useful to know how rewrite rules are interpreted by top-level procedures.
Therefore, let us give a short formal account of what it means for a pair of
wffs to be an instance of a rewrite rule.

For the simplicity of presentation, in the following we
consider monomorphic rewrite rules only. Rewrite rules with polymorphic types
can be thought of as being appropriately instantiated before the matching takes
place.

Let the structure of wffs and wff schemata be defined by the following grammar:

\[\begin{array}{rcl}
c & ::= & \top\ |\ \bot\ |\ \land\ |\ \lor\ |\ \ldots\\
Q & ::= & \lambda\ |\ \forall\ |\ \exists \\
A,B,C,D & ::= & x,y\ |\ c\ |\ A\,B\ |\ Qx.A
\end{array}\]

We define the notion of a context to stand for a partial function. The notation
$\Gamma,a{:=}b$ stands for the function
$\lambda x.\textup{if }x=a\textup{ then }b\textup{ else }\Gamma\,x$.

We define the matching relation $\Sigma;\Gamma\vdash B\rhd A;\Sigma'$ between
the initial global context $\Sigma$, the lexical context $\Gamma$, the wff $B$,
the wff schema $A$ and the final global context $\Sigma'$ by induction on the
structure of $A$:
\[
\frac{\Gamma\,x_{\greeka}=y_{\greeka}}
     {\Sigma;\Gamma\vdash y_{\greeka}\rhd x_{\greeka};\Sigma}
\qquad
\frac{x_{\greeka}\notin\textup{dom}\,\Gamma\qquad\Sigma\,x_{\greeka}=A_{\greeka}}
     {\Sigma;\Gamma\vdash A_{\greeka}\rhd x_{\greeka};\Sigma}
\qquad
\frac{x_{\greeka}\notin\textup{dom}\,\Gamma\qquad x_{\greeka}\notin\textup{dom}\,\Sigma}
     {\Sigma;\Gamma\vdash A_{\greeka}\rhd x_{\greeka};\Sigma,x_{\greeka}{:=}A_{\greeka}}
\qquad
\frac{~}{\Sigma;\Gamma\vdash c_{\greeka}\rhd c_{\greeka};\Sigma}
\]\[
\frac{\Sigma;\Gamma\vdash C\rhd A;\Sigma''\qquad
      \Sigma'';\Gamma\vdash D\rhd B;\Sigma'}
     {\Sigma;\Gamma\vdash C\,D\rhd A\,B;\Sigma'}
\qquad
\frac{\Sigma;\Gamma,x_{\greeka}{:=}y_{\greeka}\vdash B\rhd A;\Sigma'}
     {\Sigma;\Gamma\vdash Qy_{\greeka}.B\rhd Qx_{\greeka}.A;\Sigma'}
\]

Two wffs $C,D$ form an instance of a rewrite rule $A\longrightarrow B$
iff there exist global contexts $\Sigma,\Sigma'$ such that
$\varnothing;\varnothing\vdash C\rhd A;\Sigma'$ and
$\Sigma';\varnothing\vdash D\rhd B;\Sigma$.

The applicability test procedure \texttt{S-EQN-AXIOM-APPFN}, always to be used
with the rewriting procedure \texttt{S-EQN-AXIOM-REWFN}, enforces that both
sides of a rewrite rule are interpreted as wffs, not as wff schemata. So,
the only instances accepted by a rewrite rule using
\texttt{S-EQN-AXIOM-APPFN}/\texttt{S-EQN-AXIOM-REWFN} are, modulo alphabetic
change of bound variables, substitution instances of the two wffs forming the
rule. Formally this can be reflected by modifying the second and the third
matching rule as follows:
\[
\frac{x_{\greeka}\notin\textup{dom}\,\Gamma\qquad\Sigma\,x_{\greeka}=A_{\greeka}\qquad
      \textup{FV}\,A_{\greeka}\cup\textup{ran}\,\Gamma=\varnothing}
     {\Sigma;\Gamma\vdash A_{\greeka}\rhd x_{\greeka};\Sigma}
\qquad
\frac{x_{\greeka}\notin\textup{dom}\,\Gamma\qquad x_{\greeka}\notin\textup{dom}\,\Sigma\qquad
      \textup{FV}\,A_{\greeka}\cup\textup{ran}\,\Gamma=\varnothing}
     {\Sigma;\Gamma\vdash A_{\greeka}\rhd x_{\greeka};\Sigma,x_{\greeka}{:=}A_{\greeka}}
\]
$\textup{FV}\,A$ denotes the set of variables which occur free in $A$.

The applicability test \texttt{INFERENCE-SCHEME-APPFN}, to be used with
\texttt{INFERENCE-MATCH-BINDERS-REWFN}, enforces that whenever two binders in
the definition of a rewrite rule bind the same variable, this has to be
reflected in the rule instance. As a counterexample, consider a rule of the
form $\forall x_{\greeka}\forall x_{\greeka}.A\longleftrightarrow
\forall x_{\greeka}.A$. One can easily check that, if no
applicability tests are performed,
$\forall y_{\greeka}\forall x_{\greeka}.f\,x\,y\longleftrightarrow
\forall x_{\greeka}.f\,x\,y$ is an instance of the above rule.

To describe the semantics of
\texttt{INFERENCE-SCHEME-APPFN}/\texttt{INFERENCE-MATCH-BINDERS-REWFN}, we
define a relation $\Delta\vdash B\star A;\Delta'$ between the initial binder
matching context $\Delta$, the wff $B$, the wff schema $A$ and the final
binder matching context $\Delta'$ by induction on the structure of $A$:
\[
\frac{~}{\Delta\vdash A\star x_{\greeka};\Delta}
\qquad
\frac{~}{\Delta\vdash A\star c_{\greeka};\Delta}
\qquad
\frac{\Delta\vdash C\star A;\Delta''\qquad\Delta''\vdash D\star B;\Delta'}
     {\Delta\vdash C\,D\star A\,B;\Delta'}
\]\[
\frac{\Delta\,x_{\greeka}=y_{\greekb}}
     {\Delta\vdash Q'y_{\greekb}.B\star Qx_{\greeka}.A;\Delta}
\qquad
\frac{x_{\greeka}\notin\textup{dom}\,\Delta}
     {\Delta\vdash Q'y_{\greekb}.B\star Qx_{\greeka}.A;
      \Delta,x_{\greeka}{:=}y_{\greekb}}
\]

Two wffs $C,D$ form an instance of a rewrite rule $A\longrightarrow B$, using
the applicability test procedure \linebreak \texttt{INFERENCE-SCHEME-APPFN}
and the rewriting procedure \texttt{INFERENCE-MATCH-BINDERS-REWFN}, iff they
are an instance of $A\longrightarrow B$ and there exist binder matching
contexts $\Delta,\Delta'$ such that $\varnothing\vdash C\star A;\Delta'$ and
$\Delta'\vdash D\star B;\Delta$.

\section{How Rewrite Rules and Theories Are Stored in the Library}
Rewrite rules are stored as library objects of type
rrule, whose description is a dotted pair of gwffs with extra attributes
`typelist' (the list of all polymorphic type symbols in the rule),
`bidirectional' (T if the rule can be applied in both directions),
`appfn' (the name of a function to test applicability of the rule, or NIL),
`function' (an optional extra function which is applied, after the
rewriting, to the subformula that was rewritten),
`variables' (the list of universally quantified free variables; from the
technical point of view, only variables which do not appear in both sides
of the rewrite rule need to appear in this list) and
`derived-in' (the list of theories in which the rule is derivable).

%`appfn', if not NIL, should be a function from gwffs to booleans;
%`function', if not NIL, should be a function from gwffs to gwffs.

% mkaminski -- The above is no longer true.

`appfn', if not NIL, should be a function which expects four arguments.
The first argument will always be the gwff which is to be rewritten. The
following two arguments will contain both sides of the rewrite rule, passed
in the order in which the rule is to be applied. So, for instance, if the rule
is to be applied from right to left, the right-hand side of the rule will be
passed as the second argument, followed by the left-hand side. The fourth
argument is the list of polymorphic types of the rewrite rule.
`appfn' should return a boolean, indicating whether the gwff can be rewritten
by applying the rewrite rule in the specified direction.
`function', if not NIL, should expect the subformula after rewriting as its
first argument, and, just like `appfn', both sides of the rewrite rule as
additional arguments. Unlike `appfn' it is not passed the list of polymorphic
types. `function' should return the modified subformula.

Theories are
stored as the name of the theory with all rewrite rules, subtheories and other
needed objects attached as needed-objects. In addition to this, theories may
have the extra attributes `relation-sign' (a symbol representing the relation
which is assumed to hold between the two sides of a rewrite rule),
`reflexive' (T if the relation represented by the attached rewrite rules is
reflexive, NIL otherwise), `congruent' (T if the relation represented by
the rewrite rules is a congruence on lambda-terms, NIL if the rewrite rules
should only be applicable to a wff as a whole but not to its subwffs),
`derived-appfn' (the value which will be assigned to the `appfn' attribute of
rewrite rules derived from this theory in the rewriting top-level) and
`derived-rewfn' (the same as `derived-appfn' but for the `rewfn' attribute).


\chapter{Proof Translations and Tactics}

\section{Translation between proof formats; tactics}\label{tactics}
After a complete mating has been found, and the expansion tree correctly
instantiated with terms for variables, the expansion proof can be
translated into a natural deduction proof using the system's inference
rules.  This translation process is carried out by tactics.  The
following sections describe what tactics are and how to define them.  The
current tactics and tacticals of the system are listed in the facilities guide;
you can also get a complete listing of each by typing {\tt HELP TACTIC} or {\tt HELP TACTICAL}.

The \indexcommand{NAT-ETREE} can be used to
translate natural deduction proofs into expansion proofs.
See the section \ref{natetr} for more information about
\indexcommand{NAT-ETREE}.


%\chapter{Tactics and Tacticals}

%{\bf Modify \(\sim\)theorem/exp/nesmith/tactics.mss}

\subsection{Overview}

Ordinarily in {\TPS}, the user proceeds by performing a series of
atomic actions, each one specified directly.  For example, in constructing
a proof, she may first apply the deduct rule, then the rule of cases, then
the deduct rule again, etc..  These actions are related temporally, but
not necessarily in any other way; the goal which is attacked by
one action may result in several new goals, yet there is no distinction
between goals produced by one action and those produced by another.
In addition, this use of small steps prohibits the user from outlining
a general procedure
to be followed.  A complex strategy cannot be expressed in these terms,
and thus
the user must resign herself to proceeding by plodding along, using simple
(often trivial and tedious) applications of rules.

Tactics offer a way to encode strategies into new commands, using a
goal-oriented approach.  With the use of tacticals, more complex tactics
(and hence strategies) may be built.  Tactics and tacticals are, in essence,
a programming language in which one may specify techniques for solving
goals.

Tactics are called  partial subgoaling methods by
\cite{GORDON79}.  What this means is that a tactic is a
function which, given a goal to be accomplished,
will return a list of new goals, along with a procedure by which the
original goal can be achieved given that the new goals are first
achieved.  Tactics also may fail, that is, they may not be applicable
to the goal with which they are invoked.

Tacticals operate upon tactics in much the same way that functionals
operate upon functions.  By the use of tacticals, one may create
a tactic that repeatedly carries out a single tactic, or composes
two or more tactics.  This allows one to combine many small tactics
into a large tactic which represents a general strategy for solving goals.

As implemented in {\TPS}, a tactic is a function which takes a goal
as an argument and returns four values: a list of new goals, a message
which tells what the tactic did (or didn't do), a token indicating what
the tactic did, and a validation, which is a lambda expression which takes
as many arguments as the number of new goals, and which, given solutions
for the new goals, combines the solutions into a solution for the original
goal.

Consider this example.  Suppose we are trying to define tactics which
will convert an arithmetic expression in infix form to one in
prefix form and evaluate it.  One tactic might,
if given a goal of the form `A / B', where A and B are themselves
arithmetic expressions in infix form, return the list (`A' `B'),
some message, the token `succeed', and the
validation {\tt (lambda (x y) (/ x y))}.  If now we solve the new goals `A'
and `B' (i.e., find their prefix forms and evaluate them),
and apply the validation as a function to their solutions, we get
a solution to the original goal `A / B'.

When we use a tactic, we must know for what purpose the tactic is being
invoked.  We call this purpose the {\it use} of the tactic.  Uses include
{\tt nat-ded} for carrying out natural deduction proofs and
{\tt etree-nat}
for translating expansion proofs to natural deductions.  A single tactic may
have definitions for each of these uses.  In contrast to tactics, tacticals
are defined independent of any specific tactic use; some of the auxiliary
functions they use, however, such as copying the current goal, may depend
upon the current tactic use.  For this purpose, the current tactic use
is determined by the flag \indexflag{TACUSE}.  Resetting this flag resets
the default tactic use. Though a tactic can be called
with only a single use, that tactic can call other tactics with different
uses.  See the examples in the section `Using Tactics'.

Another important parameter used by a tactic is the {\it mode}.  There are
two tactic modes, {\tt auto} and {\tt interactive}.  The definition of
a tactic may make a distinction between these two modes; the current
mode is determined by the flag (\indexflag{TACMODE}, and resetting this
flag resets the default tactic mode.  Ideally, a tactic
operating in {\tt auto} mode should require no input from the user, while
a tactic in {\tt interactive} mode may request that the user make some decisions,
e.g., that the tactic actually be carried out.  It may be desirable, however,
that some tactics ignore the mode, compound tactics (those tactics created
by the use of tacticals and other tactics) among them.

One may wish to have tactics print informative messages as they operate;
the flag \indexflag{TACTIC-VERBOSE} can be set to {\tt MAX, MED} or {\tt MIN},
and then corresponding amounts of output will be printed when a call to
the function \indexfunction{tactic-output} is made by the tactic.

\subsection{Syntax for Tactics and Tacticals}
This section is of interest mainly to use who want to define new
tactics and tacticals.

The {\TPS} category for tactics is called {\tt tactic}.  The defining
function for tactics is {\tt deftactic}.  Each tactic definition has
the following form:

\begin{lispcode}
(deftactic tactic
  \{({\it tactic-use} {\it tactic-defn} [{\it help-string}])\}$^{\hbox{+}}$
)
\end{lispcode}

with components defined below:

\index{tactic-use!Syntactic Object}
\index{tactic-mode!Syntactic Object}      
\index{tactic-defn!Syntactic Object}      
\index{primitive-tactic!Syntactic Object} 
\index{compound-tactic!Syntactic Object}  
\index{tactic-exp!Syntactic Object}       
\index{goal!Syntactic Object}             
\index{goal-list!Syntactic Object}        
\index{msg!Syntactic Object}              
\index{token!Syntactic Object}            

\begin{tabular}{ll}
%\tabclear\\
%\tabset{1.5 inches}\\
%\indexsyntax{tactic-use} ::= &  {\tt nat-ded | etree-nat}\\
tactic-use ::= &  {\tt nat-ded | etree-nat}\\
tactic-mode ::= &  {\tt auto | interactive}\\
tactic-defn ::= &  {\it primitive-tactic}  {\tt  |} {\it compound-tactic}\\
primitive-tactic ::= &  {\tt (lambda (goal) \{{\it form}\}*)}\\
 &  This lambda expression should return four values of the form:\\
 &  {\it goal-list msg token validation}.\\
compound-tactic ::= &  ({\it tactical} \{{\it tactic-exp}\}*)\\
tactic-exp ::= &  {\tt tactic}  a tactic which is already defined\\
 &  | ({\it tactic} {\tt [:use {\it tactic-use}] [:mode {\it tactic-mode}] [:goal {\it goal}]})\\
 &  | {\it compound-tactic}\\
 &  | ({\tt call} {\it command}) ; where {\it command} is a command which could be\\
 &  given at the {\TPS} top level\\
goal ::= &  a goal, which depends on the tactic's use,\\
 &  e.g., a planned line when the tactic-use is {\tt nat-ded}.\\
goal-list ::= &  (\{{\it goal}\}*)\\
msg ::= &  {\it string}\\
token ::= &  {\tt complete}  meaning that all goals have been exhausted\\
 &  {\tt | succeed}  meaning that the tactic has succeeded\\
 &  {\tt | nil}  meaning that the tactic was called only for side effect\\
 &  {\tt | fail}  meaning that the tactic was not applicable\\
 &  {\tt | abort}  meaning that something has gone wrong, such as an undefined\\
 &  tactic\\
\end{tabular}

Tacticals are kept in the {\TPS} category {\tt tactical}, with defining
function {\tt deftactical}.  Their definition has the following form:

\begin{lispcode}
(deftactical tactical
  (defn {\it tacl-defn})
  (mhelp {\it string}))
\end{lispcode}

\index{tacl-defn!Syntactic Object}
\index{primitive-tacl-defn!Syntactic Object}      
\index{compound-tacl-defn!Syntactic Object}      

with

\begin{tabular}{ll}
%\tabclear\\
%\tabset{1.5 inches}\\
%tacl-defn ::= &  {\it primitive-tacl-defn | compound-tacl-defn}\\
tacl-defn ::= &  {\it primitive-tacl-defn}  {\tt |} {\it compound-tacl-defn}\\
primitive-tacl-defn ::= &  {\tt (lambda (goal tac-list) \{{\it form}\}*)}\\
 &  This lambda-expression, where {\tt tac-list} stands for a possibly\\
 &  empty list of tactic-exp's, should be independent of the tactic's\\
 &  use and current mode.  It should return values like those returned\\
 &  by a {\it primitive-tac-defn}.\\
compound-tacl-defn ::= &  {\tt (tac-lambda (\{{\it symbol}\}*) {\it tactic-exp})}\\
 &  Here the tactic-exp should use the symbols in the\\
 &  tac-lambda-list as dummy variables.\\
\end{tabular}

Here is an example of a definition of a primitive tactic.
\begin{lispcode}
(deftactic finished-p
 (nat-ded
  (lambda (goal)
    (if (proof-plans dproof)
	(progn
	 (tactic-output `Proof not complete.' nil)
	 (values nil `Proof not complete.' 'fail))
	(progn
	 (tactic-output `Proof complete.' t)
	 (values nil `Proof complete.' 'succeed))))
  `Returns success if all goals have been met, otherwise
returns failure.'))
\end{lispcode}

This tactic is defined for just one use, namely {\tt nat-ded}, or natural
deduction.  It merely checks to see whether there are any planned lines
in the current proof, returning failure if any remain, otherwise
returning success.  This tactic is used only as a predicate, so the
goal-list it returns is nil, as is the validation.  The function
\indexfunction{tactic-output} is called with a string to be printed and
whether the tactic succeeded or failed.  What will be printed will depend
on the current value of \indexflag{tactic-verbose}.

As an example of a compound tactic, we have
\begin{lispcode}
(deftactic make-nice
  (nat-ded
   (sequence (call cleanup) (call squeeze) (call pall))
   `Calls commands to clean up the proof, squeeze the line
numbers, and then print the result.'))
\end{lispcode}

Again, this tactic is defined only for the use {\tt nat-ded}.  {\tt sequence} is
a tactical which calls the tactic expressions given it as arguments
in succession.

Here is an example of a primitive tactical.
\begin{lispcode}
(deftactical idtac
  (defn
    (lambda (goal tac-list)
      (values (if goal (list goal)) `IDTAC' 'succeed
	      '(lambda (x) x))))
  (mhelp `Tactical which always succeeds, returns its goal
unchanged.'))
\end{lispcode}

The following is an example of a compound tactical.  {\tt then} and {\tt orelse} are tacticals.

\begin{lispcode}
(deftactical then*
  (defn
    (tac-lambda (tac1 tac2)
      (then tac1 (then (orelse tac2 (idtac)) (idtac)))))
  (mhelp `(THEN* tactic1 tactic2) will first apply tactic1; if it
fails then failure is returned, otherwise tactic2 is applied to
each resulting goal.  If tactic2 fails on any of these goals,
then the new goals obtained as a result of applying tactic1 are
returned, otherwise the new goals obtained as the result of
applying both tactic1 and tactic2 are returned.'))
\end{lispcode}

\subsection{Tacticals}
There are several tacticals available.  Many of them are taken directly from
\cite{GORDON79}.  After the name of each tactical is given
an example of how it is used, followed by a description of the behavior
of the tactical
when called with {\tt goal} as its goal.  The newgoals and validation returned
are described only when the tactical succeeds.


\begin{enumerate}
\item %\tabclear
%\tabset{1 inch, 1.5 inches, 2 inches, 2.5 inches}
{\tt \indexother{IDTAC}: (idtac)}\newline{}
Returns {\tt (goal), (lambda (x) x)}.

\item {\tt \indexother{FAILTAC}: (failtac)}\newline{}
Returns failure

\item {\tt \indexother{CALL}: (call command)}\newline{}
Executes command as if it were entered at top level of {\TPS}.  This is used
only for side-effects.  Returns {\tt (goal), (lambda (x) x)}.

\item {\tt \indexother{ORELSE}: (orelse tactic1 tactic2 ... tacticN)}\newline{}
If N=0 return failure, else apply {\tt tactic1} to {\tt goal}. If this fails, call {\tt (orelse tactic2 tactic3 ... tacticN)} on {\tt goal}, else return the result of applying {\tt tactic1} to {\tt goal}.

\item {\tt \indexother{THEN}: (then tactic1 tactic2)}\newline{}
Apply {\tt tactic1} to {\tt goal}. If this fails, return failure, else apply {\tt tactic2} to each of the subgoals generated by {\tt tactic1}.\newline{}
If this fails on any subgoal, return failure, else return the list of new subgoals returned from the calls to {\tt tactic2}, and the lambda-expression representing the combination of applying {\tt tactic1} followed by {\tt tactic2}.\newline{}
Note that if {\tt tactic1} returns no subgoals, {\tt tactic2} will not be called.

\item {\tt \indexother{REPEAT}: (repeat tactic)}\newline{}
Behaves like {\tt (orelse (then tactic (repeat tactic)) (idtac))}.

\item {\tt \indexother{THEN*}: (then* tactic1 tactic2)}\newline{}
Defined by:\newline{}
{\tt (then tactic1 (then (orelse tactic2 (idtac)) (idtac)))}.  This tactical is taken from \cite{Felty86}.

\item {\tt \indexother{THEN**}: (then** tactic1 tactic2)}\newline{}
Acts like {\tt then}, except that no copying of the goal or related structures will be done.

\item {\tt \indexother{IFTHEN}: (ifthen test tactic1)} or \newline{}
        {\tt (ifthen test tactic1 tactic2)}\newline{}
First evaluates {\tt test}, which may be either a tactic or (if user is an expert) an arbitrary LISP expression.  If test is a tactic and does not fail, or is an arbitrary LISP expression that does not evaluate to nil, then {\tt tactic1} will be called on {\tt goal} and its results returned. Otherwise, if {\tt tactic2} is present, the results of calling {\tt tactic2} on {\tt goal} will be returned, else failure is returned.  {\tt test} should be some kind of predicate; any new subgoals it returns will be ignored by {\tt ifthen}.

\item {\tt \indexother{SEQUENCE}: (sequence tactic1 tactic2 ... tacticN)}\newline{}
Applies {\tt tactic1, ... , tacticN} in succession regardless of their success or failure.  Their results are composed.

\item {\tt \indexother{COMPOSE}: (compose tactic1 ... tacticN)}\newline{}
Applies {\tt tactic1, ..., tacticN} in succession, composing their results until one of them fails.  Defined by:\newline{}
{\tt (idtac)} if {\tt N}=0\newline{}
{\tt (then* tactic1 (compose tactic2 ... tacticN))} if {\tt N} > 0.

\item {\tt \indexother{TRY}: (try tactic)}\newline{}
Defined by: {\tt (then tactic (failtac))}.  Succeeds only if tactic returns no new subgoals, in which case it returns the results from applying {\tt tactic}.

\item {\tt \indexother{NO-GOAL}: (no-goal)}\newline{}
Succeeds iff goal is nil.
\end{enumerate}


\subsection{Using Tactics}
\label{usetac}

To use a tactic from the top level, the command \indexmexpr{use-tactic} has
been defined.  Use-tactic takes three arguments: a {\it tactic-exp}, a
{\it tactic-use},
and a {\it tactic-mode}.  The last two arguments default to the values of
\indexflag{TACUSE} and \indexflag{TACMODE}, respectively.
Remember that a {\it tactic-exp} can be either the name of
a tactic or a compound tactic.  Here are some examples:

\begin{alltt}
<1> use-tactic propositional nat-ded auto

<2> use-tactic (repeat (orelse same-tac deduct-tac))
               \$ interactive

<3> use-tactic (sequence (call pall) (call cleanup) (call pall)) !

<4> use-tactic (sequence (foo :use nat-etree :mode auto)
                         (bar :use nat-ded :mode interactive)) !
\end{alltt}
Note that in the fourth example, the default use and mode are overridden
by the keyword specifications in the tactic-exp itself.  Thus during the
execution of this compound tactic, {\tt foo} will be called for one use and
in one mode, then {\tt bar} will be called with a different use and mode.

Remember, the value of the flag \indexflag{TACTIC-VERBOSE}  will
affect the amount of detail which is printed when the tactics execute.
Other flags also have effect on tactics, most noticeably \indexflag{USE-RULEP}
and \indexflag{LAMBDA-CONV}; look at the help messages of these flags, and at
the use of \indexother{USE-RULEP-TAC} in \indexother{COMPLETE-TRANSFORM*-TAC}
and of \indexother{LEXPD*-VARY-TAC} in \indexother{GO2-TAC} for examples.

Two of the most useful tactics have been given their own commands:
\indexmexpr{GO2} is equivalent to {\tt use-tactic go2-tac nat-ded}, and
\indexmexpr{MONSTRO} is equivalent to {\tt use-tactic monstro-tac nat-ded}.
Both of these tactics call a function \indexfunction{print-routines}, which
sends output to the screen and/or proofwindows, as specified by the flag
\indexflag{ETREE-NAT-VERBOSE}.


\subsection{Translating from Natural Deduction to Expansion Proofs}\label{natetr}

The flag \indexflag{NAT-ETREE-VERSION} determines which
version of \indexcommand{NAT-ETREE} will be used.
The latest version (as of 2001) is \indexother{CEB}.
The basic steps of the \indexother{CEB} version of
\indexcommand{NAT-ETREE} are as follows:

\begin{description}
\item[] 

{\tt 1} -- The natural deduction
proof is preprocessed to eliminate applications of RuleP, Subst=,
and similar rules in favor of more basic inference rules.

{\tt 2} -- The natural deduction proof is translated into a different
structural form called a \indexother{natree}.  One can perform
this part of the translation in isolation using the command
\indexcommand{PFNAT}.  The current natree can be viewed using
the command \indexcommand{PNTR}.

{\tt 3} -- The natree representation is translated into a \indexother{sequent calculus}
derivation.  The sequent calculus derivations in memory can be listed
using the command \indexcommand{SEQLIST}.  A particular sequent calculus
derivation can be viewed using the commands \indexcommand{PSEQ}
or \indexcommand{PSEQL}.  The flag \indexflag{PSEQ-USE-LABELS} controls
whether formulas are abbreviated by showing a symbol associated with
the formula in a legend.

{\tt 4} -- \indexother{Cut elimination} is performed on the sequent calculus derivation.
This is not guaranteed to succeed or even terminate.  A common case
where cut elimination will fail is when a nontrivial
use of extensionality occurs.

{\tt 5} -- If cut elimination succeeds, the cut-free derivation is translated
to an expansion tree with a complete mating.  The user is given the
option of merging this expansion proof.  Merging is appropriate if
the user intends to translate this expansion proof back into a natural
deduction proof.  If the user is trying to use this expansion proof
to help determine flag settings to find the proof automatically, the
user should not merge the tree. 
\end{description}

The expansion proof can be viewed by entering the \indexcommand{MATE} top level.
Section \ref{searchanalysis} explains how this expansion proof can
be used to suggest flag settings and to trace automatic search.

The Programmers Guide has more information about \indexcommand{NAT-ETREE}.



\chapter{Testing for Satisfiability}\label{models}

{\TPS} has a top level called \indexcommand{MODELS} which can be used to compute the
semantic value of a formula in small finite standard models of type
theory in which the domains of all types have cardinalities which are
powers of 2.  (The assumption that domains are a power of 2 is used
for an efficient representation of functions as binary expansions of numbers.)
This top level also features a \indexcommand{SOLVE} command that will solve
for values of  ``output" variables in terms of values of ``input"
variables which will make a given formula true.

We can use the MODELS top level to investigate the satisfiability
of variants of {THM616}.  First, we can simply
ask {\TPS} to interpret {THM616} (in the standard model
with 2 individuals)
using the \indexcommand{INTERPRET} command in the  MODELS top level.
The formula {THM616} has one free variable, $OPEN_{\greeko(\greeko\greeki)}$.
We can use the command \indexcommand{ASSIGN-VAR} to assign this variable to
an element of type $(\greeko(\greeko\greeki))$.  The 16 elements
of this type are represented by numbers between 0 and 15.
Once the variable is assigned a value, \indexcommand{INTERPRET} will
evaluate {THM616} and determine that it is true (represented by 1 in type $\greeko$).
We can also determine that {THM616} is true for any value of $OPEN_{\greeko(\greeko\greeki)}$
by universally quantifying over $OPEN$ in the prefix of {THM616}.

Of course, we already knew {THM616} must be true in every (extensional)
model since {THM616} is a theorem.
A better use of the MODELS top level is to establish that some formula
is not a theorem.  For example, we can remove parts of {THM616}
and question whether the simpler formula is a theorem.
For example, can we prove {THM616} without using the closure
of $OPEN$ under subsets?  The corresponding formula 
$$\forall \,x_{\greeki} [ \,B_{\greeko\greeki} \,x \supset \exists \,D_{\greeko\greeki} . \,OPEN_{\greeko(\greeko\greeki)} \,D \land \,D \,x \land \,D \subseteq \,B] \supset \,OPEN \,B$$
has two free variables $B_{\greeko\greeki}$ and $OPEN_{\greeko(\greeko\greeki)}$.
Of course, as mathematicians we know this is not a theorem.  The goal is
to find a counterexample.  The question is whether there is a counterexample
within the standard model with 2 individuals.  We can ask {\TPS} to try to
solve for such a counterexample by invoking the \indexcommand{SOLVE} command
using the negation of the formula above
with no input variables and the two output variables $B$ and $OPEN$.
{\TPS} returns 10 possible interpretations of the pair of variables 
satisfying the negation.  The first and simplest corresponds to
choosing $B$ to be the empty set of individuals and $OPEN$ to be the empty set of sets.

Similarly, \indexcommand{SOLVE} can solve for values of $B$ and $OPEN$
satisfying the negation of 
$$\forall \,G_{\greeko(\greeko\greeki)} [ \,G \subseteq \,OPEN_{\greeko(\greeko\greeki)} \supset \,OPEN . \bigcup \,G] \supset \,OPEN \,B_{\greeko\greeki}.$$
{\TPS} returns 11 solutions; the first solution corresponds to choosing $OPEN$
to be the set containing the empty set (which is indeed closed under arbitrary unions)
and choosing $B$ to be a singleton.

Of course, it is always possible that the hypotheses of {THM616} were inconsistent.
That is, we might be able to strengthen {THM616} to state
$$
\begin{array}{c}
\sim .  \forall \,G_{\greeko(\greeko\greeki)} [ \,G \subseteq \,OPEN_{\greeko(\greeko\greeki)} \supset \,OPEN . \bigcup \,G] \\
\land \forall \,x_{\greeki} . \,B_{\greeko\greeki} \,x \supset \exists \,D_{\greeko\greeki} . \,OPEN \,D \land \,D \,x \land \,D \subseteq \,B
\end{array}
$$
\indexcommand{SOLVE} finds 17 solutions for the negation of this formula.
Four of the solutions interpret $OPEN$ as the set of all sets of individuals
(so that the interpretation of $B$ is irrelevant).

\chapter{Output: Symbols, Files and Styles}
\label{output}

\section{Proofwindows}

If proofwindows have been opened (with the \indexcommand{BEGIN-PRFW}
command), then proofs or parts of proofs which have been produced
automatically may not automatically appear in the proofwindows.
The command \indexcommand{PSTATUS} can then be
used to update these windows appropriately.  There are also three
associated flags, \indexflag{PROOFW-ALL}, \indexflag{PROOFW-ACTIVE+NOS} and
\indexflag{PROOFW-ACTIVE}, which control the output to the `Complete
Proof', `Current Subproof and Line Numbers'
and `Current Subproof' windows respectively; when one of these
flags is NIL, no output goes to the relevant window, and when the flag
is T then output resumes. All these flags are set to T by default.

To make proofwindows work properly, the unix DISPLAY variable must have
been properly initialized. (For example, if you are working on
btps.tps.cs.cmu.edu, but proofwindows do not appear when they should,
try doing {\it setenv DISPLAY btps.tps.cs.cmu.edu:0.0} before starting up TPS.)


\section{Interpreting the Output from Mating Search}

\subsection{Symbols Printed by Mating Search}
Mating search outputs a number of special symbols; it isn't necessary to know what they mean,
but this section is provided for those who are curious.
Some of these symbols are replaced by more informative messages if \indexflag{MATING-VERBOSE}
is set to {\tt MAX} or {\tt MED}; conversely, if \indexflag{MATING-VERBOSE} is {\tt SILENT} they
may not be printed at all. Furthermore, those which are labelled (EVENTS) are generated by the events package,
and will not be printed unless events are switched on.

Unification also outputs special symbols (if \indexflag{UNIFY-VERBOSE} is set high enough);
Again, some of these symbols are replaced by more informative messages if \indexflag{UNIFY-VERBOSE}
is set to {\tt MAX} or {\tt MED}; conversely, if \indexflag{UNIFY-VERBOSE} is {\tt SILENT} they
may not be printed at all.

The symbols are as follows:
\begin{itemize}
\item {\bf *} means that the mating search is considering a connection (EVENTS).

\item {\bf +} means that a connection is being added (EVENTS).

\item {\bf -} means that a connection is being removed (EVENTS).

\item {\bf 2} means that a quantifier (or every quantifier) is being duplicated (EVENTS).

\item {\bf P} means that primitive substitutions are being applied (EVENTS).

\item {\bf \%} means that the current mating is not unifiable (EVENTS).

\item {\bf MST} means that a mating is being tested for subsumption (EVENTS).

\item {\bf MSS} means that a mating is subsumed by an incompatible or inextensible mating (EVENTS).

\item {\bf UST} means that the disagreement pairs are being tested for subsumption (EVENTS).

\item {\bf USS} means that the disagreement pairs are subsumed by a previous set of dpairs (EVENTS).

\item {\bf M} means that a connection has been rejected because \indexflag{MAX-MATES} is too low.

\item {\bf B} means that a connection has been rejected because it is banned (currently, this can only happen
if a gwff has been rewritten in two different ways: connections between leaves of the two rewrites are
marked as `banned').

\item {\bf .} means we could interrupt at this point; see section \ref{interrupt} for details.

\item {\bf C} means that a complete (but not necessarily unifiable) mating has been found.

\item {\bf s} means that unification is giving up on a branch of the unification tree because
\indexflag{MAX-SEARCH-DEPTH} has been exceeded.

\item {\bf u} denotes the same thing for \indexflag{MAX-UTREE-DEPTH}.

\item {\bf S} denotes the same thing for \indexflag{MAX-SUBSTS-VAR}.

\item {\bf F} means that unification has failed, and the search will now backtrack.

\item {\bf R} means that a rigid path check has succeeded.

\item {\bf ?} means that unification subsumption check has found a possible subsumed node and is checking further.

\item {\bf !} means that unification subsumption check has eliminated a subsumed node.

\item {\bf 0-1-0-2} or a similar string of digits separated by hyphens denote new unification
nodes being created. This particular example would be a new node which is the second son of the
only son of the first son of the root of the unification tree.
\end{itemize}

Other messages you may see (apart from the self-explanatory ones) are as follows:

In non-path-focused searches, the current path is often shown; this is simply the
leftmost open path, which {\TPS} is currently trying to block.

In path-focused duplication, the current mating is shown on occasion (the
actual frequency is determined by the flag \indexflag{PRINT-MATING-COUNTER}).
In these matings, the connection: LEAF19 . LEAF17    4 . 3
is between copy 4 of the innermost universal quantifier with LEAF 19 in its scope,
and copy 3 of the innermost universal quantifier with LEAF 17 in its scope.
A copy number of -1 indicates the sole occurrence of a literal which is not
in the scope of any quantifier.

Also in path-focused duplication, one may see timing statistics like:
\begin{alltt}
Timing statistics for mating-search:
Evaluation took:
  2.64 seconds of real time,			A
  0.96875 seconds of user run time,		B
  0.1875 seconds of system run time,		C
\end{alltt}
In these, B is the significant number, C is the amount of time taken up by paging,
etc, and A is at least the sum of B and C. These timing figures are usually
noticeably lower than the `official' figures produced by
\indexcommand{DISPLAY-TIME}.

\chapter{The Monitor}
\label{monitor}

The \indexother{monitor} is designed to be called during automatic proof searches; its basic
operation is described in the User Manual. There are three basic steps required to 
write a new monitor function, which are described below, using the monitor function 
\indexother{monitor-check} as an example. More examples are in the file {\it monitor.lisp}.

\section{The Defmonitor Command}

The command \indexcommand{defmonitor} behaves just like {\tt defmexpr}, the only difference being
that the function it defines does not appear in the list when the user types {\tt ?}. This command
will be called by the user before the search is begun, and should be able to accept any required 
parameters (or to calculate them from globally accessible variables at the time the command is
called).

So, for example, the {\tt defmonitor} part of \indexother{monitor-check} looks like this:

%\begin{tpsexample}
\begin{verbatim}
(defmonitor monitor-check
  (argtypes string)
  (argnames prefix)
  (arghelp "Marker string")
  (mainfns monitor-chk)
  (mhelp "Prints out the given string every time the monitor is called, 
followed by the place from which it was called."))

(defun monitor-chk (string)
  (setq *current-monitorfn* 'monitor-check)
  (setq *current-monitorfn-params* string)
  (setq *monitorfn-params-print* 'msg))
\end{verbatim}
%\end{tpsexample}

Note that this accepts a marker string as input from the user (other monitor functions may 
look for a list of connections, or flags, or the name of an option set; it may be necessary 
to define a new data type to accommodate the desired input). It then calls a secondary 
function, which in this case needs to do very little further processing in order to 
establish the three parameters which are {\it required} for every such function: {\tt *current-monitorfn*}
contains a symbol corresponding to the name of the monitor function, {\tt *current-monitorfn-params*} 
contains the user-supplied parameters (in any form you like, since your function will be the only 
place where they are used) and {\tt *monitorfn-params-print*} contains the name of a function that can 
print out {\tt *current-monitorfn-params*} in a readable way, for use by the commands \indexcommand{monitor}
and \indexcommand{nomonitor}. The latter should be set to {\tt nil} if you can't be bothered to write such 
a function.

\section{The Breakpoints}

In the relevant parts of the mating search code, you should insert breakpoints of the form:

%\begin{tpsexample}
\begin{verbatim}
(if monitorflag 
    (funcall (symbol-function *current-monitorfn-params*) 
             <place> <alist>))
\end{verbatim}
%\end{tpsexample}

The value of {\it place} should reflect what part of the code the breakpoint is at. So, for example,
it might be {\tt 'new-mating}, {\tt 'added-conn} or {\tt 'duplicating}.

The value of {\it alist} should be an association list of local variables and things that your monitor
function will need. For example, {\it alist} might be {\tt (('mating . active-mating) ('pfd . nil))}; it might 
equally well be just {\tt nil}.

All breakpoints should have exactly this pattern. By typing {\it grep "(if monitorflag (funcall" *.lisp} in
the {\it tpslisp} directory, you can get a listing of all the currently defined breakpoints.

\section{The Actual Function}

This is the function which will actually be called during mating search. By convention, it has the
same name as the {\tt defmonitor} function. Normally, it will first check the value of {\it place}, to
see if it has been called from the correct place; it can then use the {\tt assoc} command to retrieve the
relevant entries from {\it alist}. Theoretically, it should be completely non-destructive so as to ensure 
that the mating search continues properly; of course, you may be as destructive as you like, provided 
you understand what you're doing...

The function for {\tt monitor-check} is as follows; notice that this does not check {\it place} since it 
is intended to act at every single breakpoint.

%\begin{tpsexample}
\begin{verbatim}
(defun monitor-check (place alist)
  (declare (ignore alist))
  (msg *current-monitorfn-params* place t)) 
\end{verbatim}
%\end{tpsexample}




\section{Output files}

{\TPS} writes files into a directory determined by the value of the Lisp
function user-homedir-pathname; this function (presumably) gets the
value from the \$HOME environment variable.

See the {\ETPS} manual \cite{AndrewsTPS88b} for basic information on using the
commands \indexcommand{TEXPROOF}, \indexcommand{SCRIBEPROOF} and
\indexcommand{PRINTPROOF} to print proofs into files. {\TPS} has internal
modes called \indexother{SCRIBE-OTL}, \indexother{TEX-OTL} and
\indexother{TEX-1-OTL} which it uses by default for printing proofs into
Scribe and TeX files. These modes are generally good enough for the job, although
it is possible to turn them off (in order to use your own flag settings)
by setting \indexflag{USE-INTERNAL-PRINT-MODE} to NIL.

Printed output of wffs is designed so that a wff in a proof never extends
further left than the turnstile preceding it. If \indexflag{TURNSTILE-INDENT-AUTO}
is MIN, COMPRESS or VARY, and some lines have very long hypotheses, the
turnstile can end up moving a long way to the right, and this will mean
that some wffs have to be crammed into only a few columns. This can produce
very strange-looking output, so if you have many hypotheses and long wffs, it's probably best to set
\indexflag{TURNSTILE-INDENT} to, say, 5 and \indexflag{TURNSTILE-INDENT-AUTO} to FIX.
This will force the turnstile to always be in column 5, by inserting a newline if necessary.
You may also want to set \indexflag{USE-INTERNAL-PRINT-MODE} to NIL,
\indexflag{FILLINEFLAG} to T, \indexflag{FLUSHLEFTFLAG} to T,
and \indexflag{PPWFFLAG} to NIL in this case.

The editor commands \indexedop{VPF} and \indexedop{VPT} allow you to save vertical
path diagrams in generic and TeX styles, respectively.

\section{Output styles}
The value of the flag \indexflag{STYLE} determines how wffs are to be printed.
The styles which are most useful for printing on a terminal are
\indexstyle{GENERIC}, \indexstyle{CONCEPT-S} and \indexstyle{XTERM}.
The first of these uses no special characters.
The other two are useful only when using {\TPS} on a Concept terminal or inside an
xterm window under the X window system, respectively.  When these styles are used,
special characters for
logical connectives, constants, and type symbols are printed on the screen,
making output more readable.  See section \ref{X} for how to use {\TPS}
with the X window system.

The styles \indexstyle{SCRIBE} and \indexstyle{TEX} are used for printing output
in a form which is acceptable to those typesetting systems. Most {\TPS} output
can be produced in either of these two styles, although vertical path diagrams
in Scribe are not pretty at all. Documentation produced by commands like
\indexcommand{QUICK-REF} is always in Scribe format.


\section{Saving Output from Mating Search}

{\TPS} has several available methods of storing output from the mating-search.
The \indexmexpr{SCRIPT} command will record a transcript of the session
(this, of course, will record everything you do, not just the mating-search).
If you are creating a script file, you may wish to do {\tt style generic} in
order to make the output more readable; if you are working on a
Concept keyboard, change this to {\tt style concept-s}. The \indexcommand{SCRIPT} command
will add Scribe or TeX headers to the resulting file, if the \indexflag{STYLE} flag
is either {\tt SCRIBE} or {\tt TEX}.
The \indexmexpr{UNSCRIPT} command
will close the script file. The \indexmexpr{SCRIPT} command is intended
to mimic the Unix command {\it script}, and one can also call the Unix
{\it script} before running {\TPS} (this will even save bug messages which come from
Lisp, which the {\TPS} command SCRIPT will not do; however, one should be careful
to ensure that the Unix command is recording the
output from the correct window, if one is running X windows!).
The script file can be viewed using the Unix {\it cat} command.

A list of all the events (duplication, etc) which occur during mating-search
can be saved to a file by setting the flags \indexflag{REC-MS-FILE} to T,
\indexflag{REC-MS-FILENAME} to a file name and \indexflag{INTERRUPT-ENABLE} to T.
This only works for non-path-focused procedures.

More usefully (if you are running {\TPS} under X-windows or using the Java interface),
when using \indexcommand{DIY} or when entering the
\indexcommand{MATE} top level you will be offered the chance to open a vpform
window. You can also open this window manually with the \indexcommand{OPEN-MATEVPW}
command, and you can close it with \indexcommand{CLOSE-MATEVPW}. The output of
the window may be discarded, or may be saved to a file for future reference.
If the vpform window is open, then every time {\TPS} prints a vpform to the main
window, it will send a copy to the vpform window. In the mating-search, it will
also send copies of the associated substitutions to the vpform window, and if
the mating-search terminates then the complete mating will also be sent there.

Thus the vpform window and/or file will contain a record of the entire search;
since this window is just an xterm, one can scroll about in it while the search is
still proceeding without risking any information being lost.
Scribe or TeX headers will be added to the resulting file automatically, if the
\indexflag{STYLE} flag is either {\tt SCRIBE} or {\tt TEX}.

The file can be viewed again by using the \indexcommand{DISPLAYFILE} command
within {\TPS}, or by using the \indexother{vpshow} utility from a Unix prompt.
The vpshow utility is provided as part of the {\TPS} system; look for the directory
{\it tps/utilities}, which should contain the files \indexfile{vpshow} and
\indexfile{vpshow-big}; just type {\tt vpshow {\it filename}} to view the file.

\section{Interrupting TPS for Occasional Output}

The \indexcommand{PUSH} and \indexcommand{POP} commands are very useful here. By setting
the flags so that {\TPS} pauses for input during a mating search or translation, one can
interrupt the program for long enough to print out a proof or vertical path diagram
before continuing. For example, setting \indexflag{QUERY-USER} to {\tt QUERY-JFORMS} makes
{\TPS} ask whether to search on each new jform it generates during a mating search. Instead
of replying yes or no, reply {\tt PUSH}; this will start a new top level from which you can
print the vpform (for example) before typing {\tt POP} to continue the search.
The monitor function \indexcommand{PUSH-MATING} is also useful in this context; see the help
message for more details.

Similarly, by using \indexcommand{ETREE-NAT} in {\tt INTERACTIVE} mode, you can use
{\tt PUSH} and {\tt POP} to produce `snapshots' of a natural deduction proof as it is
constructed from an expansion proof. (You
can also use a suitable setting of \indexflag{ETREE-NAT-VERBOSE} and edit the resulting
output for much the same effect.)

\section{Output for Slides}

\indexcommand{SLIDEPROOF} is like \indexcommand{SCRIBEPROOF}, but prints
proofs in the vpstyle \indexstyle{SCRIBE-SLIDES}. You may need to adjust the flags
\indexflag{SLIDES-TURNSTYLE-INDENT} (set it to 6 if there is at most one
hypothesis per line), \indexflag{PRINTEDTFLAG-SLIDES},
\indexflag{FILLINEFLAG} and \indexflag{PPWFFLAG}.

You can make slides for Scribe of a session with {\TPS} as follows:

\begin{tpsexample}
<0>save-work file1
{\it Do what you want to do in tps}
<100>stop-save
<101>setup-slide-style		
{\it This sets rightmargin and sets style to scribe
Actually you may need to set rightmargin to 45 instead of 51.
Try setting the margins in the scribe heading below smaller.
If there are just a few lines that are too long, edit them.
Also perhaps change the settings of PPWFFLAG and FILLINEFLAG.}
<102>execute-file
COMFIL (FILESPEC): SAVE-WORK file [`work.work']file1
EXECPRINT (YESNO): Execute Print-Commands? [No]>y
OUTFIL (FILESPEC): Output file (`NUL:' to discard) [`TTY']>`file2.mss'
{\it then leave tps}
\end{tpsexample}

You should then edit the resulting Scribe file, adding the following
header:

\begin{alltt}
jmake(slides)
juse(Database `/afs/cs/project/tps/pmax/doc/lib')
jmodify(verbatim, spacing 2, linewidth 51)
jstyle(rightmargin = .25in)
jstyle(leftmargin = .25in)
jlibraryfile(tps18)
jPageFooting(Immediate,  Center  <\value{Page}>)
jBegin(Verbatim)
\end{alltt}

and add {\tt jend(verbatim}) at the end. You can, of course, edit the body
of the file as necessary.

\section{Record files}
To make a script of a session with TPS which you can examine
later, proceed as follows:
\begin{alltt}
\%script {\it filename}		
\%tps				enter TPS
<0>setflag style {\it style}
				do what you wish in TPS
<9>exit				exits TPS
\%exit				exits the script session
\end{alltt}

Notice that the above uses the Unix \indexcommand{script} command, which records all the
output from a particular shell; this will not work if you are using an xterm window
(i.e. if you have aliased the `tps' command so that it starts a new shell in a new window
and runs {\TPS} there, then nothing will be recorded to the script file).

Alternatively (and this will work in all cases), use the command \indexmexpr{SCRIPT}
in {\TPS} to start a script file, as follows:
\begin{alltt}
<0>setup-slide-style
{\it if you want output for overhead projector slides; otherwise omit this.}
<1>style {\it style}
{\it choose your output style for the main window.}
<2)window-style {\it style}
{\it choose your output style for the vpwindow and other windows.}
<3>script {\it filename}
{\it start an output file for the main window.}
{\it ....
do what you wish in TPS (you can opt to save output from vpwindows when they are opened}.)
<9>unscript
{\it close the output file for the main window.}
<10>close-matevpw
{\it if you haven't already closed it.}
<11>exit
{\it exits TPS.}
\end{alltt}

The command \indexmexpr{UNSCRIPT} will close the most recently opened script file.
(Warning: At the time of writing, there was a bug in the {\tt SCRIPT} command
in that if a script file is started from a sub-top-level such as {\tt mate}, the
file will be closed without warning when you leave. So always use {\tt SCRIPT}
from the main top level.)

The value of the flag {\tt STYLE} should be set ({\it before} you issue the \indexmexpr{SCRIPT} command,
so that the file will get the correct header on it) to either {\tt GENERIC}, {\tt TEX}, {\tt SCRIBE},
{\tt CONCEPT-S}, or {\tt XTERM}, depending upon what use you plan for the
file:
\begin{description}
\item[] If you need to print out a copy of your session, use
{\tt GENERIC}, {\tt TEX} or {\tt SCRIBE}. To produce output for
use as overhead projector slides, use the command \indexmexpr{SETUP-SLIDE-STYLE} before starting the
script file;
this produces output in {\tt SCRIBE} format, so vpforms will be badly formatted, but
everything else will be correct. The style {\tt SCRIBE} can also be used to produce
regular printable output in {\tt SCRIBE} format. If you are using the automatic
procedures, it may be best to set \indexflag{STYLE} to {\tt TEX} (and possibly also opt to save the vpwindow
output to a separate file). This works best because the only style
in which vpforms are printed correctly is {\tt TEX}. Note that the mating procedures output
some characters that will confuse TeX (notably <, > and \#), and so the resulting files will still
need a certain amount of editing before they are entirely correct.
Better yet, you can set the \indexflag{WINDOW-STYLE} flag to {\tt TEX} and the \indexflag{STYLE}
flag to {\tt SCRIBE} for the best of both worlds; output to the vpwindow will be saved in
style {\tt TEX}, in a separate file to the main output which will be saved in style {\tt SCRIBE};
this way you get everything formatted correctly all at once.

\item[] Output files in style {\tt GENERIC} will be printable immediately, without any editing,
although they may be a little difficult to read if you are printing a lot of wffs.

\item[] The other two settings ({\tt CONCEPT-S} and {\tt XTERM}) should be used if you wish to `play back'
the file as a demonstration on a Concept with special characters or
in an xterm window with the special boldface font vtsymbold.
The resulting script file will not be human-readable in its raw form, but you can play it back
using the Unix command `more {\it filename}',
or by using the {\it vpshow} command (from a shell), or with the \indexmexpr{DISPLAYFILE} command
from within {\TPS}.
\end{description}



\chapter{Events}
\label{events}


% \comment{\chapter{Teaching Records}\label{Teach}}
% {\bf Make revisions to \$progdoc/teach.mss}
% % \comment{EVENT-SIGNAL EVENTS REPORT}

\section{Events in TPS}


The primary purpose of events in {\TPS} is to collect information about
the usage of the system.  That includes support of features such as
automatic grading of exercises, keeping statistics on the application of
inference rules, being informed about bugs in the system,
and recording remarks made by the users of {\TPS}.  Other events are
used to print informative messages to the user during mating search.

Events, once defined and initialized, can be signalled from anywhere in
{\TPS}.  Settings of flags, ordinarily collected into modes, control if,
when, and where signalled events are recorded.

In {\ETPS}, a basic set of events is predefined, and the events are signalled
automatically whenever appropriate.  Whether these events are then recorded
depends on your {\ETPS} profile.

There are some restrictions on events that should be respected, if
you plan to use {\tt REPORT} to extract statistics from the files recording
events.  Most importantly: {\bf No two events should be written to the
same file}.  If you would like to record different things into the
same file, make one event with one template and allow several kinds of
occurrences of the event.  For an example, see the event {\tt PROOF-ACTION}
below.

\subsection{Defining an Event}

If you are using {\ETPS}, it is unlikely that you need to define an event
yourself.  However, a lot of general information about events is given
in the following description.

Events are defined using the {\tt DEFEVENT} macro.
Its format is

\begin{tpsexample}
(defevent {\it name}
  (event-args {\it arg1} ... {\it argn})
  (template {\it list})
  (template-names {\it list})
  (signal-hook {\it hook-function})
  (write-when {\it write-when})
  (write-file {\it file-parameter})
  (write-hook {\it hook-function})
  (mhelp {\it help-string}))
\end{tpsexample}

\begin{description}
\item[{\tt event-args} ]	  list of arguments passed on by {\tt SIGNAL-EVENT} for any event
	of this kind.

\item[{\tt template} ]	  constructs the list to be written.
        It is not assumed that every event is
	time-stamped or has the user-id.  The template
        must only contain globally evaluable forms and the arguments
	of the particular event signalled.  It could be the source of
        subtle bugs, if some variables are not declared special.

\item[{\tt template-names} ]	  names for the individual entries in the template.
These names are used by the {\tt REPORT} facility.  As general conventions,
when the template form is a variable, use the same name for the
template name (e.g. {\tt DPROOF}).  If the template form is {\tt (STATUS {\it statusfn})}
use {\it statusfn} as the template name (e.g. {\tt DATE} for {\tt (STATUS DATE)} or
{\tt USERID} for {\tt (STATUS USERID)}).

\item[{\tt signal-hook} ]	  an optional function to be called whenever the
	the event is signalled.  This should {\bf not} do the writing of
	the information, but may be used to do something else.  If the
        function does a {\tt THROWFAIL}, the calling {\tt SIGNAL-EVENT} will
        return {\tt NIL}, which means failure of the event.  The arguments
        of the function should be the same as {\tt EVENT-ARGS}.

\item[{\tt write-when} ]	  one of {\tt IMMEDIATE}, {\tt NEVER}, or an integer {\it n}, which means
     to write after an implementation dependent period of {\it n}.
     At the moment this will write, whenever the number of inputs = {\it n}
     * {\tt EVENT-CYCLE}, where {\tt EVENT-CYCLE} is a global variable, say 5.

\item[{\tt write-file} ]	  the name of the global {\tt FLAG} with the filename of the
     file for the message to be appended to.

\item[{\tt write-hook} ]	  an optional function to be called whenever a number
	(>0) of events are written.  Its first argument is the file it will
        write to, if the write-hook returns.  Its second argument is the
        list of evaluated templates to be written.  If an event is to be
        written immediately, this will always be a list of length 1.

\item[{\tt mhelp} ]	  The mhelp string for the event.
\end{description}

Remember that an event is ignored, until {\tt (INIT-EVENTS)} or {\tt (INIT-EVENT
{\it event})} has been called.

\subsection{Signalling Events}

{\TPS} provides a function \indexfunction{signal-event}, which takes a variable number
of arguments.  The first argument is the kind of event to be signalled,
the rest of the arguments are the event-args for this particular event.
\indexfunction{signal-event} will return {\tt T} or {\tt NIL}, depending on whether the action
to be taken in case of the event was successful or not.  Note that when
an event is disabled (see below), signalling the event will always be
successful.  There are basically three cases in which an event will be
considered unsuccessful: if the {\tt SIGNAL-HOOK} is specified and does a
{\tt THROWFAIL}, if {\tt WRITE-WHEN} is {\tt IMMEDIATE} and either the {\tt WRITE-HOOK}
(if specified) does a {\tt THROWFAIL}, or if for some reason the writing to
the file fails (if the file does not exists, or is not accessible
because it has the wrong protection, for example).

It is the caller's responsibility to make use of the returned value of
\indexfunction{signal-event}.  For example, the signalling of {\tt DONE-EXERCISE} below.

If {\tt WRITE-WHEN} is a number, the evaluated templates will be collected
into a list {\it event{\tt -LIST}}.  This list is periodically written out and
cleared.  The interval is determined by \indexflag{EVENT-CYCLE}, a global flag
(see description of {\tt WRITE-WHEN} above).  The list is also written out
when the function {\tt EXIT} is called, but not if the user exits {\TPS} with
{\tt \^C}.  Note that if events have been signalled, the writing is done
without considering whether the event is disabled or not.  This ensures
that events signalled are always recorded, except for the {\tt \^C} safety valve.

Events may be disabled, which means that signalling them will always
be successful, but will not lead to a recordable entry.  This is done
by setting or binding the flag {\it event{\tt -ENABLED}} to {\tt NIL} (initially
set to {\tt T}).  For example, the line {\tt (setq error-enabled nil)}
in your {\tt .INI} file will make sure that no Lisp error will be recorded.
For a maintainer using expert mode, this is probably a good idea.

\subsection{Examples}

Here are some examples taken from the file {\tt ETPS-EVENTS}.  Interspersed
is also the code from the places where the events are signalled.

\begin{tpsexample}

(defflag error-file
  (flagtype filespec)
  (default ((tpsrec *) etps error))
  (subjects events)
  (mhelp `The file recording the events of errors.'))

(defevent error
  (event-args error-args)
  (template ((status userid) (status subsys) error-args))
  (template-names (userid subsys error-args))
  (write-when immediate)
  (write-file error-file)    ; a global variable, eg
			     ; `((tpsrec: *) etps error)
  (signal-hook count-errors) ; count errors to avoid infinite loops
  (mhelp `The event of a MacLisp Error.'))

{\rm 
{\tt DT} is used to freeze the daytime upon invocation of {\tt DONE-EXC} so that the code is computed correctly.
The code is computed by {\tt CODE-LIST}, implementing some ``trap-door function''.}

(defvar computed-code 0)

(defvar dt '(0 0 0))

(defflag score-file
  (flagtype filespec)
  (default ((tpsrec *) tps scores))
  (subjects events)
  (mhelp `The file recording completed exercises.'))

(defevent done-exc
  (event-args numberoflines)
  (template ((status userid) dproof numberoflines computed-code
			     (status date) dt))
  (template-names (userid dproof numberoflines computed-code date daytime))
  (signal-hook done-exc-hook)
  (write-when immediate)
  (write-file score-file)
  (mhelp `The event of completing an exercise.'))

(defun done-exc-hook (numberoflines)
  {\tt  The} done-exc-hook will compute the code written to the file.
  (declare (special numberoflines))
  {\tt  because} of the (eval `(list ..)) below.
  (setq dt (status daytime))
  {\tt  Freeze} the time of day right now.
  (setq computed-code 0)
  (setq computed-code (code-list (eval `(list ,j(get 'done-exc 'template))))))

(defflag proof-file
  (flagtype filespec)
  (default ((tpsrec *) tps proof))
  (subjects events)
  (mhelp `The file recording started and completed proofs.'))

(defevent proof-action
  (event-args kind)
  (template ((status userid) kind dproof
			     (status date) (status daytime)))
  (template-names (userid kind dproof date daytime))
  (write-when immediate)
  (write-file proof-file)
  (mhelp `The event of completing any proof.'))

(defflag remarks-file
  (flagtype filespec)
  (default ((tpsrec *) tps remarks))
  (subjects events)
  (mhelp `The file recording remarks.'))

(defevent remark
  (event-args remark-string)
  (template ((status userid) dproof remark-string
			     (status date) (status daytime)))
  (template-names (userid dproof remark-string date time))
  (write-when immediate)
  (write-file remarks-file)
  (mhelp `The event of a remark by the user.'))

\end{tpsexample}

Here is how the {\tt DONE-EXC} and {\tt PROOF-ACTION} are used in the code of
the {\tt DONE} command.  We don't care if the {\tt PROOF-ACTION} was successful
(it will usually be), but it's very important that the user knows
when a {\tt DONE-EXC} was unsuccessful, since it is used for automatic
grading.

\begin{tpsexample}
(defun done ()
  ...
  (when (funcall (get 'exercise 'testfn) dproof)
	{\tt  here} we have an assigned exercise.
	(do ()
	    ((signal-event 'done-exc (length (get dproof 'lines)))
	     (msgf `Score file updated.'))
	  (msgf `Could not write score file.  Trying again... (abort with \^G)')
	  {\tt  wait} for a bit, in case the problem was simultaneous access.
	  (sleep 0.5)))
  (signal-event 'proof-action 'done)
  ...
  )
\end{tpsexample}

\section{More on Events}

Each command (mexpr) may have associated with
it an EVENT-TYPE.  EVENTs could be PRINTING, INFERENCE, SYSTEM, FILEOP,
ADVICE, STARTED-PROOF, DONE-PROOF, and perhaps more.  One may define for
each event, how much information about the event is saved, and when, and
if the operation is legal, if the information could not be saved.
An event could be signalled whenever a command is executed (signal the
associated event), or from within a function (say for an error) with
the (event arg1 ... argn) LEXPR.
For example:
\begin{tpsexample}
(defevent advice
  (append-file advice-file)		;advice-file is a global var.
  (append-when immediate)		;could be IMMEDIATE, NEVER, PERIODICAL.
  (append-failed abort)			;could be ABORT, RETRY-LATER.
					;ABORT means that operation must be
					;aborted if recording of event failed.
  (append-info '(time user exercise))	;a template
  (mhelp `Event that occurs when advice is asked.'))

(defevent inference
  (append-file inference-file)
  (append-when periodical)
  (append-failed retry-later)
  (append-info '(time user exercise legal-p wrong-defaults-count))
					;legal-p and wrong-defaults-count
					; are two args supplied to EVENT
					; inside COMDECODE.
  (mhelp `Event that occurs when an inference rule is applied.'))

(defevent maclisp-error
  (append-file error-file)
  (append-when immediate)
  (append-failed retry-later)
  (append-info '(time current-command err-message))
  (mhelp `An uncaught error condition.'))

\end{tpsexample}
The {\it .proof} and {\it .scores} files, run through REPORT,
can be used to get a distribution of when people did their work, how long
the average proof was etc.

The report called SECURITY checks for lists
with the wrong security code in a source file (tps:etps.rec). To run
it:
\begin{tpsexample}
<\#>lload tn:report
<\#>report
MODE:....[nil]>             ;;hit return
<rep1>security
OUTFIL....[tty:]>
SINCE....[(85 1 1)]>
WHO......[(t *)]>
WHAT.....[(t *)]>           ;;the t says the case fold is on. * matches
                            ;;strings of any length (including zero).
                            ;;? matches a single character.
T
<rep2>on security
<rep3>run
....
<rep4>exitrep
<\#>
\end{tpsexample}
REMARK, GRADE-FILE and EXER-TABLE also work in a similar manner.

Remarks are sent to the file {\it etps.remarks} (in addition to
{\it etps.rec})

\begin{enumerate}
\item A category of EVENT.  Every event has a few properties:
\begin{description}
\item[MHELP]	 the obvious

\item[EVENT-ARGS]	 list of arguments passed on by SIGNAL for any event
	of this kind.

\item[TEMPLATE]	 constructs the list to be written.  Contrary to what
	we had before, we will not assume that every event is
	time-stamped or has the user-id.  The template
        must only contain globally evaluable forms and the arguments
	of the particular event signalled.

\item[WRITE-WHEN]	 one of IMMEDIATE, NEVER, or an integer n, which means
    	write after a period of n inputs.

\item[WRITE-FILE]	 the filename of the file for the message to be
        appended to.

\item[SIGNAL-HOOK]	 an optional function to be called whenever the
	the event is signalled.  This should NOT to the writing of
	the information, but may be used to do something else.

\item[WRITE-HOOK]	 an optional function to be called whenever a number
	(>0) of events are written.
\end{description}

\item A macro or function SIGNAL-EVENT, whose first argument is the
    kind of even to be signalled, the rest of the arguments are the
    event-args for this particular event.  SIGNAL-EVENT will return
    T or NIL, depending on whether the action to be taken in case of
    the even was successful or not.  It is the caller's responsibility
    to act accordingly.  E.g. if (SIGNAL-EVENT COSTLY-ADVICE 'X2106)
    returns NIL, the advice should not be given (of course at the moment
    we don't charge for advice).
\end{enumerate}
Examples:
\begin{tpsexample}
(defevent maclisp-error
  (event-args error-args)
  (template ((status userid) dproof current-command error-args))
  (write-when 5)
  (write-file error-file)    ; a global variable, eg
			     ; `((tpsrec: *) etps error)
  (signal-hook count-errors) ; count errors to avoid infinite loops
  (mhelp `The event of a MacLisp Error.'))

(defevent tps-complain
  (event-args complain-msglist)
  (template ((status userid) complain-msglist))
  (write-when 10)
  (write-file complain-file)
  (mhelp `The event of an error message given by TPS.'))

{\rm The event tps-complain could be ``hard-wired'' into the COMPLAIN macro, so that every time COMPLAIN is
executed, the event is signalled.}

(defevent advice-asked
  (event-args)
  (template ((status userid) dproof)
  (write-when immediate)
  (write-file advice-file)
  (mhelp `Event of user asking for advice.'))

\end{tpsexample}

The definition of EVENTS now includes TEMPLATE-NAMES, which is
a list for the entries in the events.  some general conventions:
(status userid) => userid
(status date)   => date
(status daytime)=> daytime
If an event-arg appears directly in the template, use that same name as
the TEMPLATE-NAME.

% \begin{comment}
% Is there a way of disabling certain events, that is, to stop recording
% the information which an event would otherwise record, without having
% to rebuild the image (and leaving the undesired events undefined)?
% Do we have a variable indicating which events are enabled?
% [This may be another property of a process (instance of a top-level)]
% \end{comment}
\section{The Report Package}

The REPORT package in {\TPS} allows the processing of data
from EVENTS. Each report draws on a single event, reading
its data from the record-file of that event. The execution
of a report begins with its BEGIN-FN being run. Then
the DO-FN is called repetitively on the value of the EVENTARGS
in each record from the record-file of the event, until that
file is exhausted or the special variable DO-STOP is given a non-NIL
value. Finally, the END-FN is called. The arguments
for the report command are given to the BEGIN-FN and END-FN.
The DO-FN can only access these values if they are assigned to
certain PASSED-ARGS, in the BEGIN-FN. Also, all updated values
which need to be used by later iterations of the DO-FN or by
the END-FN should be PASSED-ARGS initialized (if the default NIL
is not acceptable in the BEGIN-FN.

NOTE: The names of PASSED-ARGS should be different from
other arguments (ARGNAMES and EVENTARGS). Also, they should
be different from other variables in those functions where
you use them and from the variables which DEFREPORT2 always
introduces into the function for the report: FILE, INP and DO-STOP.

The definition of the category of REPORTCMD, follows:

\begin{tpsexample}

(defcategory reportcmd
  (define defreport1)	      ; DEFREPORT defines a function and a command
  (properties                 ; (MEXPR), as well as a REPORTCMD.
   (source-event single)
   (eventargs multiple)   ;; selected variables in the var-template of event
   (argnames multiple)
   (argtypes multiple)
   (arghelp multiple)
   (passed-args multiple) ;; values needed by DO-FN (init in BEGIN-FN)
   (defaultfns multiple)
   (begin-fn single)    ;; args = argnames
   (do-fn single)       ;; args = eventargs
   (end-fn single)      ;; args = argnames
   (mhelp single))
  (global-list global-reportlist)
  (mhelp-line `report')
  (mhelp-fn princ-mhelp)
  (cat-help `A task to be done by REPORT.'))

\end{tpsexample}

	The creation of a new report consists of a DEFREPORT statement
and the definition of the BEGIN-FN, DO-FN and END-FN. Any PASSED-ARGS
used in these functions should be declared special. It is suggested
that most of the computation be done by general functions which are more
readily usable by other reports. In keeping with this philosophy,
the report EXER-TABLE uses the general function MAKE-TABLE. The latter
takes three arguments as input:  a list of column-indices, a list of
indexed entries (row-index, column-index, entry) and the maximum printing size
of row-indices. With these, it produces a table of the entries.
EXER-TABLE merely calls this on data it extracts from the record file
for the DONE-EXC event. The definition for EXER-TABLE follows:

\begin{tpsexample}

(defreport exer-table
  (source-event done-exc)
  (eventargs userid dproof numberoflines date)
  (argtypes date)
  (argnames since)
  (passed-args since1 bin exerlis maxnam)
  (begin-fn exertable-beg)
  (do-fn exertable-do)
  (end-fn exertable-end)
  (mhelp `Constructs table of student performance.'))

(defun exertable-beg (since)
  (declare (special since1 maxnam))	; the only passed-args initialized
  (setq since1 since)                   ; to non-Nil values
  (setq maxnam 1))

(defun exertable-do (userid dproof numberoflines date)
  (declare (special since1 bin exerlis maxnam))
  (if (greatdate date since1)
      (progn
       (setq bin (cons (list userid dproof numberoflines) bin))
       (setq exerlis
	     (if (member dproof exerlis) exerlis (cons dproof exerlis)))
       (setq maxnam (max (flatc userid) maxnam)))))

(defun exertable-end (since)
  (declare (special bin exerlis maxnam))
  (if bin
      (progn
       (make-table exerlis bin maxnam)         ;; exerlis --> column headers
       (msg t `On exercises completed since ') ;; bin --> row headers, entries
       (write-date since)                      ;; maxnam =
       (msg `.' t))                            ;; max \{size x : x a row header\}
      (progn
       (msg t `No exercises completed since ')
       (write-date since)                      ;; prints date in English
       (msg `.' t))))

\end{tpsexample}
% \begin{comment}
% \section{More on Report}
% 
% The new efficient REPORT is ready. The passing of arguments is handled
% through the PASSED-ARGS property.
% NOTE: The only arguments to the DO-FN are the EVENTARGS. If you wish
% to use the arguments from the MEXPR (i.e. the ARGNAMES), you must
% make PASSED-ARGS with different names and initialize them
% in the BEGIN-FN. Any PASSED-ARGS used should be declared special.
% Of course, this means that no PASSED-ARGS should bear the same name
% as any of the other arguments (ARGNAMES and EVENTARGS).
% 
% REPORT now uses WITH-OPEN-FILE, which is defined for MacLisp in MACSYS.
% Also, REPORT has a new special variable DO-STOP. If it is set to
% a non-NIL value, it will terminate the reading from the events file.
% 
% There is a command file and a work file which in conjunction makes it
% very easy to evaluate the students progress on TF, while logged in
% on the C.  Here is how to use it:
% 
% \begin{text}
% 
%  jFTP TF
%  ..>LOGIN user-name
%  ..>TAKE TPS:TF.CMD
% \end{text}
% 
%  Next run ETPS and do
% 
% \begin{text}
% 
%  <1>EXECUTE-FILE TPS:TF NO NUL:
%  <2>REPORT
%  <Rep1>
% \end{text}
% 
%  then type a ? for a list of available reports.  Among them is
%  EXER-TABLE and a way to get a list of remarks.
% 
% In the SECURITY report, set
% the code to 0 and include it in the list, then compute the code.
% There is a flag SINCE-DEFAULT for the DEFAULTFN
% of all the reports which take a SINCE argument.  The default date
% is (80 1 1).
% 
% Changed filename HACK to
% EVENT-SIGNAL-UTILS and made it part of a new package EVENT-SIGNAL.
% Both OTLNL and REPORT have the latter as a needed-package.
% 
%    `STATUS ' is now `STATUS-' in file TL:ETPS-EVENTS.Lsp. The 3
% macros (STATUS-USERID), (STATUS-DATE), (STATUS-DAYTIME) are defined in
% file MACSYS-3.
% 
% You changed the file TPS:ETPS-EVENTS.LSP
% and left the file TCL:ETPS-EVENTS.CLISP unchanged.  Therefore the new
% ETPS core image I built turned out to be rotten, since STATUS-USERID (etc)
% were undefined functions!
% Anyway, I changed them to be functions in TPS3 - no reason for the duplication
% of code everywhere it is called.  Also, it returns a string - the old
% function returned a list (eg. (10 30 86)), but that was an easy fix.
% 
% Should have built events into grader.
% 
% ;;; This file contains all the REMARKS sent to the instructor.
% ;;; Format: (userid dproof remark-string date daytime)
% (TK06 X5207 `string?' (86 3 14) (10 22 52))
% (PB99 ***** `just checking the remarks facility on tf.' (86 3 15) (0 26 23))
% (TK06 X5207 `
% 
% \section{Mail Messages}
% 
% I have made a command file and a work file which in conjunction make it
% very easy to evaluate the students progress on TF, while logged in
% on the C.  Here is how to use it:
% 
% \begin{tpsexample}
% jFTP TF
%  ..>LOGIN user-name
%  ..>TAKE TPS:TF.CMD
% \end{tpsexample}
% 
%  Next run ETPS and do
% 
% \begin{tpsexample}
%  <1>EXECUTE-FILE TPS:TF NO NUL:
%  <2>REPORT
%  <Rep1>
% \end{tpsexample}
% 
%  then type a ? for a list of available reports.  Among them is Carl's
%  EXER-TABLE and a way to get a list of remarks.
% 
% \heading{The Old Report Package}
% 
% 	The REPORT package provides a general framework for the running
% of reports. It is accessed through the REPORT command in BTPS. This brings
% the user into the REPORT top-level.
% 	Once in the REPORT top-level, the user establishes the parameters
% for his reports by running:
% 
% SOURCE      :sets the list of input file(s)
% 
% <reportname>:sets the parameters for <reportname>
% 
% RUN         :runs the active reports on the SOURCE file(s)
% 
% 	Reports may be disabled by using the OFF command and enabled using
% the ON command.
% 	The current command strings for each report are saved in their
% Environ properties. These may be labelled at any time using the SETMODE
% command. These modes are accessible during the remainder of the session.
% When the user EXITs REPORT, the modes and the time of the last run of
% each report are saved in the REPORT.INI file on the users account. When
% report is run again, this file is loaded and the modes are available.
% 	The full syntax of the REPORT command is:
% 	REPORT <modename>
% Since SETMODE also allows commands to be added to the ones which establish
% the mode's report-environment, REPORT can be called without any visible
% interaction with its top-level, viz. by adding RUN and EXIT to the mode.
% 
% 
% \end{comment}
re accessible during the remainder of the session.
% When the user EXITs REPORT, the modes and the time of the last run of
% each report are saved in the REPORT.INI file on the users account. When
% report is run again, this file is loaded and the modes are available.
% 	The full syntax of the REPORT command is:
% 	REPORT {\tt\char`\<}modename{\tt\char`\>}
% Since SETMODE also allows commands to be added to the ones which establish
% the mode's report-environment, REPORT can be called without any visible
% interaction with its top-level, viz. by adding RUN and EXIT to the mode.
% 
% 
% \end{comment}

\chapter{The Rules Module}
\label{rules}


% \begin{comment}
% \chapter{Inference Rules}\label{Rules}
% 
% {\bf Make revisions to \$progdoc/rules.mss}
% {\bf Someone  familiar with the rules module needs to look at this chapter,
% and update it appropriately.}
% 
% \section{Etc}
% 
% In TPS3, the Rules Module is part of the TPS3 core image, rather than
% a separate core-image.
% 
% Individual Rule files are assembled using ASSEMBLE-FILE. Modules of
% rules are assembled using ASSEMBLE-MOD.
% 
% \section{Implementing Tactics and Tacticals}
% 
% \subsection{Questions and Difficulties}
% 
% The method of tactics and tacticals allows one to expand a collection of
% rules so that many rules may be used or tested in a single step. Specifically,
% this method formalizes certain ways of using existing rules, singly or
% in combination, to create new rules or advisors. The primitive ways are called
% tacticals, and the rules/advisors formed from them are called tactics.
% 
% This brings us quickly to a major problem in implementing tactics and
% tacticals. Since the ways in which the original rules are used are not uniform,
% the notion of combining them becomes vague. If we repeat universal
% instantiation, do we mean to ask for the substituted term, as in the full
% interactive use of the rule, or to take the standard defaults, or to consult
% the expansion tree or some exogenous heuristic (one which is not a tactic
% or otherwise based on the `endogenous' formalism of {\TPS} : inference rules,
% proof outlines, etc.) for the appropriate term?
% When does the iteration end: when all universal quantifiers have been
% instantiated at least once, or when a particular wff is generated, or when
% the user says stop?
% 
% It would seem that we would want to make the operation of
% tacticals dependent on the mode of proof, but even more on the strategy
% adopted. For whatever formalism we choose to direct our proof, we would
% like to interrupt that direction occasionally, maybe seeing only which
% path it plans to take and then giving our own choice.
% Thus, the {\tt GO} and {\tt SUGGEST} facilities, as well as the interactive
% mode should be found in each strategy.
% 
% Besides aesthetic questions as to the printing and keeping of lines and
% the style of the justifications, this dependence affects the efficiency
% of tactics. It may be possible, for a given heuristic, to generate
% sequences of substitution terms more quickly than to determine each term
% when it is called for. In any event, we would like each tactic to be optimally
% efficient for the proof-mode and strategy in which it is called.
% 
% Finally, we want our treatment of tacticals to adequately handle those
% rules which are not merely forwards or backwards. This we would expect
% from the dependence on strategies. The problem of applying, testing
% or trying rules with many potential directions seems to be one of fitting
% those rules into a strategy of proof. Of course, some strategies will
% be too open-ended (say the default strategy for {\ETPS}) to determine
% the appropriate application of some rule. In such cases, {\tt GO} or
% {\tt SUGGEST} may be little help. If we allow strategies to be more open-ended
% for {\tt GO} than for {\tt SUGGEST}, we then have the desired behavior for {\ETPS}.
% 
% \section{Random}
% \end{comment}
Inference rules in {\TPS} are created by typing rule definitions into a .rules
file and then building (or assembling) the rules in that file.
One result of building a rule is the creation of a command which
calls it. The same command may be
used to apply a rule in both its forward or backward directions, that is,
from the top (hypotheses and their consequents, called `support lines') or
the bottom (the conjectures, called `plan lines') of the proof. In our own
rules, we have adopted the convention of naming them as if they were to be
applied only in the forward direction. Thus `ICONJ' (Introduce CONJunction)
takes two
support lines and derives their conjunction (forward) or a plan line asserting
a conjunction and creates two new plan lines, one for each conjunct (backward).

\section{Defining Inference Rules}

The following definition of the inference rule {\tt ABU} provides
a good example of how such rules are defined.

\begin{lispcode}

(defirule abu
  (lines (p1 (h) () `forall y(A). `(S y x(A) A)')
	 (p2 (h) () `forall x(A). A' (`AB' (`x(A)') (p1))))
  (restrictions (free-for `y(A)' `x(A)' `A(O)')
		(not-free-in `y(A)' `A(O)'))
  (support-transformation ((p2 'ss)) ((p1 'ss)))
  (itemshelp (p1 `Lower Universally Quantified Line')
	     (p2 `Higher Universally Quantified Line')
	     (`x(A)' `Universally Quantified Variable in Higher Line')
	     (`A(O)' `Scope of Quantifier in Higher Line')
	     (`y(A)' `Universally Quantified Variable in Lower Line')
	     (S `Scope of Quantifer in Lower Line'))
  (mhelp `Rule to change a top level occurrence of a universally quantified
 variable.'))

\end{lispcode}

The defining macro is DEFIRULE. Next follows the name of the rule
being defined, in this case ABU for alphabetic change of a universally
bound variable. Then comes a list of lists setting the values of
several properties; the property being set is the first item of its list.

The first property we set is the {\tt LINES} property. This establishes the kind
of lines the rule will act on. In our example, {\tt P1} is a line with an
arbitrary hypothesis set {\tt h}, no new hypotheses (the empty list following
the list containing {\tt h}) and a wff matching a quoted expression of some
complexity. The first part of the expression seems clear enough: {\tt P1}
will be a universally quantified wff. But what is {\tt `(S y x(A) A)} ?
Just our way of denoting wff-transformations within an expression
similar to a wff. The backquote means `evaluate this Lisp form', in our case
a call to the substitution function {\tt S}, replacing free occurrences
of {\tt y} with {\tt x} in a wff which we call {\tt A}. See the Facilities Guide for
a list of the functions which can be used like {\tt S} (these are called
{\tt WFFOP}s). {\tt x} is of a type
called {\tt A} which just happens to have the same name as the wff we are
substituting into, but this causes TPS no confusion; when a primitive symbol is
followed by an expression in parentheses, that expression is a type expression
and not a wff.

The second line, {\tt P2}, looks similar. It has the same set of hypotheses,
{\tt h}, and it also introduces no new hypotheses. The quoted expression,
though, is much simpler; it indicates that the wff asserted by {\tt P2}
is universally quantified by the variable we substituted into {\tt A}
in {\tt P1} and quantifies over that same {\tt A}. An actual use of the rule
may have {\tt P2} bound by {\tt y} and {\tt P1} by {\tt x}, and this is fine as
long as they correspond in the way that {\tt x} and {\tt y} do in the DEFIRULE.
That is, the variables in a DEFIRULE are not actual variables in the logical
system, but part of a pattern-matching device for the rule. The quoted
expression is followed by a list indicating the justification for the line.
The first item is the name of the justification, in our case `AB' for
alphabetic change of bound variable. The second item is a list of parameters
(excluding lines) which figure in the justification; here we indicate
that the bound variable has been changed to the variable matched by {\tt x}.
The quotes around {\tt x} are necessary.
The last item is a list of lines from which the line {\tt P2} is derived,
in this case {\tt P1}.

The next property, {\tt RESTRICTIONS}, is optional, depending on whether
or not the rule can only be applied if certain conditions are met.
In this example, the variable matching {\tt y} must be free for the variable
matching {\tt x} in the wff matching {\tt A} and similarly for the `not free in'
restriction. Note that each argument to the restrictions is typed.
In restrictions, you must give each wff variable a type (or make sure it can be
inferred). Otherwise, the default type will be used, giving an undefined
symbol as an argument to the restriction function.

The next property, {\tt SUPPORT-TRANSFORMATION}, tells TPS how this rule
will change the proof structure. In our example, the support lines for
{\tt P2}, indicated by {\tt 'ss}, will be assigned to {\tt P1}, if the rule
is applied backwards.  In other rules, the abbreviation {\tt pp} may be
seen as the first member of a support-transformation list.  {\tt pp} will
match any planned line with the specified lines as supports; e.g., if
{\tt pp P1} appeared as the left hand side of a support transformation,
the transformation would be applied to every planned line which had
{\tt P1} as a support.

The {\tt ITEMSHELP} property specifies the help the command for the rule
will give on each argument. Arguments include all lines defined in the
{\tt LINES} property, all matching variables (except types) and the
name of functions ({\tt WFFOP}s) called from within the quoted expressions
(in case not all of its arguments are specified in time).

The {\tt MHELP} property, as always, provides a short description of the rule
for the {\tt HELP} command and for documentation.


\section{Assembling the Rules}

Once you have typed your {\tt DEFIRULE}s into a .rules file, the next step
is to assemble the rules. Assembling creates Lisp-code files which can
be loaded and/or compiled. You may assemble individual rules files
with {\tt ASSEMBLE-FILE} or whole modules (collections of files) with
{\tt ASSEMBLE-MOD}. The latter is preferable, not only because
it combines many steps in one, but because the initialization for the package
will be called less frequently. {\tt ASSEMBLE-FILE}
finds the proper initialization, not very cleverly, by asking for the package
to which the file belongs.


Before assembling your rules, the correct mode should be loaded: Call {\tt REVIEW}, entering its toplevel.
Call {\tt MODE} with % \comment{either {\tt \indexother{MATH-LOGIC1-MODE}} or
% {\tt \indexother{MATH-LOGIC-2-MODE}}}
 {\tt \indexother{RULES}}
as an argument.

% \comment{Directories or pathnames should not be specified when you are asked
% for the .RULES file.}
After assembling, you need only compile (if desired) and load to make
your rules available in that session, e.g., via the command {\tt CLOAD}.
Before compiling and loading your rules, you should go into {\tt REVIEW} and
set the mode to {\tt RULES}, so that the wffops which appear in your
rules will be properly interpreted.
To make your rules more permanently
available, create a package (in the {\tt DEFPCK} file) containing
the name of your assembled rules file (the same as the .rules file
but with a lisp extension) and load that package when building {\TPS} or
{\ETPS}.

\subsection{An example}

We added a new rule definition to the file {\it ml2-logic7a.rules}, making it
necessary to reassemble and recompile {\it ml2-logic7a.lisp}. This was done
as follows:

\begin{alltt}
<123>mode rules
<124>assemble-file
RULE-FILE (FILESPEC): Rule source file to be assembled [No Default]>ml2-logic7a
PART-OF (SYMBOL): Module the file is part of [OTLSUGGEST]>math-logic-2-rules
<125>mode rules
<125>cload ml2-logic7a
\end{alltt}

\subsection{Customizing {\ETPS} or {\TPS} with your own rules}

Suppose that we wanted to set up a logical system with just one rule,
{\it modus ponens}.  Here's how we would go about it.

First we find the definition of the rule MP, which is located in one
of the files with the extension {\tt .rules}.  Let's make some minor
modifications to it for our new system. This is what it looks like:
\begin{alltt}
(defirule modpon
  (lines (p1 (h) () `A')
	 (d2 (h) () `A implies B')
	 (d3 (h) () `B' (`Modus Ponens' () (p1 d2))))
  (support-transformation (('pp d2 'ss)) ((p1 'ss) ('pp d3 'ss p1)))
  (itemshelp (p1 `Line with Antecedent of Implication')
	     (d2 `Line with Implication')
	     (d3 `Line with Succedent of Implication')
	     (`A(O)' `Antecedent of Implication')
	     (`B(O)' `Succedent of Implication'))
  (mhelp `The rule of Modus Ponens.'))
\end{alltt}

We place the rule definition in a new file, say {\tt mp.rules}.  Now we
need to generate the Lisp functions that carry out the rule.  At the
{\ETPS} top level, we do the following:
\begin{alltt}
;;; Set up flags to read the new rule properly
<0> mode rules
<1>assemble-file
RULE-FILE (FILESPEC): Rule source file to be assembled [`.rules']>mp
PART-OF (SYMBOL): Package the file is part of [OTLSUGGEST]>newpackage
MODPON
Written file /afs/cs.cmu.edu/user/nesmith/tps/mp.lisp.
\end{alltt}

In order to put this new rule into a new {\lisp} package, we add the following
lines to the beginning of the file {\tt mp.lisp}:
\begin{alltt}
(unless (find-package `MY-RULES')
  (make-package `MY-RULES' :use (package-use-list (find-package `ML'))))
(in-package 'my-rules)
\end{alltt}

We also should put the new rule file into a separate {\TPS} package, by
adding the following to the file {\tt defpck.lisp}:
\begin{alltt}
(def-lisp-package my-rules
  (needed-lisp-packages core auto)
  (mhelp `My rules.'))

(defmodule my-rules
  (needed-modules math-logic-2-wffs theorems replace)
  (lisp-pack my-rules)
  (files mp)
  (mhelp `Defines my rules.'))
\end{alltt}

Now we load these changes into {\ETPS}:
\begin{alltt}
<2> qload defpck
Loading stuff from \#<File stream `/afs/cs.cmu.edu/user/nesmith/tps/defpck.lisp'>.
<3> sys-load my-rules
<4> use-package 'my-rules
\end{alltt}

Now the rules in the package my-rules are available for use.  We can have them
loaded each time {\ETPS} starts up by adding the following lines to the
file {\tt etps.patch}:
\begin{alltt}
(qload `defpck')
(sys-load-package 'my-rules)
(use-package 'my-rules)
(unuse-package 'ml)
\end{alltt}

The last form above will make the current rules unavailable to the user.

We could also build a version of {\ETPS} which uses these rules by loading
my-rules instead of math-logic-2-rules.  This change can be made by
modifying the file {\tt etps-build.lisp} in the appropriate place.

Notice that if two rules in different packages (or even different modules)
have the same name, and both are loaded into the same core image, the last one
loaded will be the one that is available. (In particular, in the standard {\TPS},
the modules {\tt math-logic-1} and {\tt math-logic-2} conflict in this way, although
{\tt math-logic-1} is normally not loaded.)

\subsection{Creating Exercises}

In general, you may use \indexfunction{DEFTHEOREM} to define theorems from the book
the student may want to use \indexother{BOOK-THEOREM}, practice exercises \indexother{PRACTICE},
exercises \indexother{EXERCISE} and test problems \indexother{TEST-PROBLEM}. This is indicated
in the value of the \indexother{THM-TYPE} property. There is one other value of \indexother{THM-TYPE},
which is \indexother{LIBRARY}, indicating that the theorem was loaded from the library.

You may specify the amount of advice given and the rules excluded by writing
the appropriate Lisp function and making its name the value of the
\indexother{ALLOWED-CMD-P} property. Some functions are already defined, and are described below.

By a similar device with the \indexother{ALLOWED-LEMMA-P} property,
you may specify which theorems may be asserted legally and used with
{\tt SUBST-WFF} to help in the proof.

The \indexother{ASSERTION} property should be given a wff in quotes, the assertion
of the theorem. There are other properties \indexother{SCORE}, \indexother{REMARKS},
\indexother{FOL} and \indexother{MHELP}. The first two are useless, but are meant to contain
the maximum score for the exercise (GRADER currently ignores this) and any remarks from
the teacher. The third should be T if the theorem is first-order, and the last is the usual
help message.

Here are some examples:

\begin{tpsexample}
(deftheorem X6004
  (assertion
   ` eqp [= x(B)] [= y(A)] ')
  (thm-type exercise)
  (allowed-cmd-p allow-all)
  (allowed-lemma-p allow-no-lemmas))

(deftheorem X5206
  (assertion
   ` \% f(AB) [x union y] = . [\% f x] union [\% f y] ')
  (thm-type exercise)
  (allowed-cmd-p allow-all)
  (required-lemmas (x5200 x5204))
  (allowed-lemma-p only-required-lemmas))
\end{tpsexample}

The functions for \indexother{allowed-cmd-p} are as
follows. Notice that for practice theorems (i.e. those
with THM-TYPE PRACTICE) these generally behave like ALLOW-ALL.

\begin{description}
\item[\indexfunction{ALLOW-ALL} ]	 allows all commands

\item[] \indexfunction{ALLOW-RULEP} 
allows all commands except \indexcommand{ADVICE}.

\item[\indexfunction{DISALLOW-RULEP} ]	 allows all commands except
\indexcommand{ADVICE} and \indexcommand{RULEP}.
\end{description}

Those for \indexother{allowed-lemma-p} are:
\begin{description}
\item[\indexfunction{ALLOW-NO-LEMMAS} ]	 allows no theorems to be asserted
in proving the exercise.

\item[\indexfunction{ONLY-REQUIRED-LEMMAS} ]	 allows only
theorems listed under \indexother{REQUIRED-LEMMAS} to be used.

\item[\indexfunction{THEOREM-NO} {\it nnn} ]	 allows only theorem {\it nnn}
to be asserted.

\item[\indexfunction{ALLOW-LOWER-NOS} ]	 allows any theorem with a lower number
to be asserted. (This obviously requires your theorems to be numbered,
as the default theorems from Andrews' book are.)
\end{description}

The definitions of exercises can be stored in files and loaded into {\TPS}
when needed. In order to use the safeguards of the {\TPS} package loading
functions, these are given a package of their own ({\tt ML} in our system),
and we have defined modules for these files and other files
defining the logical system for {\ETPS} in {\tt defpck}. The main module
for a system will have some mnemonic name, such as {\tt math-logic-2},
and the module for the exercises will have the suffix {\tt -exercises}.
Similarly, the module names for wffs (constants, abbreviations, etc.)
and rules bear the suffices {\tt -wffs} and {\tt -rules}, respectively.

If you just want to put an exercise into {\ETPS} temporarily, add the
necessary information to the file etps.patch as in the following
example:

\begin{tpsexample}
(export '(x8030a))

(context ml2-exercises)

(deftheorem x8030a
  (assertion
"
[g(OO) TRUTH AND g FALSEHOOD] = FORALL x(O) g x
  ")
  (thm-type exercise)
  (allowed-cmd-p allow-all)
  (allowed-lemma-p allow-no-lemmas))


\end{tpsexample}


\begin{comment}

% % \comment{Old documentation}
\section{Specifications of Inference Rules}
This section describes how the {\TPS} user can define his own
inference rules.  Sets of these inference rules can then be made available
as packages
which can be added to the basic {\TPS} core image.  The functions discussed
below are part of the package {\tt RULES}\index{Rules}, which is necessary
for defining inference rules.

It may be helpful to think of the {\tt RULES} package as a compiler.  It can read
source files or rule descriptions in various forms and compiles them into
{\tt LISP} code.  This {\tt LISP} code is stored in files
which can be retrieved from within {\TPS}.

Inside the {\tt RULES} package it is always assumed that the user is working
on a particular set of inference rules which he is trying to expand,
modify, or define. Every such set of rules has a name. {\tt S4}, {\tt CLASSIC},
or {\tt INTUITION} are examples of such names.
Source file name(s) and the rule file name are derived from the name of the
set of rules as described below.  The `rule file' is the file containing
the {\tt LISP} code produced from the rule descriptions, which typically come
from the `source file'.  Both can be read and written directly by {\tt LISP}
which also keeps track of different versions of the same
file as well as changes made to a file during a editing session etc.
The {\tt LISP} functions {\tt DSKIN}, {\tt DSKOUT}, and {\tt CHANGES} are sufficient for
handling all the necessary file operations.  If you are using \VIDI,
a `\VIDI file' containing rule descriptions, possibly with special characters,
may also be present.

The {\tt RULES} package produces two functions for every inference rule
specification.  One is prefixed by {\tt D-} and allows the application
of the rule in the forward direction, i.e. it can be used to infer new
lines in the partial proof from lines that have been proven already.
The other function is prefixed by {\tt P-} and allows the application
of the inference rule in the backward direction.  This function inserts
new lines into the incomplete proof which would justify previously unproven
lines ({\it planned lines}).  These new lines are now planned lines.
Inference rules are allowed to contain primitive operators, like
\LAMBDA-contraction, \LAMBDA-expansion, etc.  Their definition and
their use is laid down in a file {\tt PRIMOP}, which can be extended
by the user.  Primitive operators typically take wffs as arguments
and return wffs.  Primitive operators which have no simple inverse
(like \LAMBDA-normalization) often make it impossible to construct
both {\tt D-} and {\tt P-} versions for a given inference rule. In this case
a warning will be given.


\subsection{Available commands}
The following is a complete list of commands which are available to the
user of the {\tt RULES} package.
{\bf This section should be deleted as this can be generated automatically.
However, the MHELP strings should first be updated to reflect the information
here.}
\begin{description}
\item[] \indexmexpr{RuleFile {\it FileName}}
defines {\it FileName} to be the current rule file.  The associated
current source file will have a {\tt Z} appended to {\it FileName}. ({\tt Z} replaces
the sixth character if {\it FileName} consists of six or more characters.)
If these files exist in the user's or any library directory, they will
be read for modification.  Otherwise, the appropriate data
structures will be set up, and the files will be ready for {\tt DSKOUT}.
{\it FileName} follows the {\tt LISP} conventions and may include an extension,
as well as a PPN and a device specification.  The global variables
\indexparameter{RuleFileFNS} and \indexparameter{SourceFileFNS} are set
to identifiers which have the table of contents for the rule file or
the source file, respectively, as their value.  \indexparameter{Rule-File}
is another global variable which will be bound to {\it FileName}.

\item[] \indexmexpr{Define-Rule \{{\it RuleName}\} \{{\it RuleDef}\}}
This parses {\it RuleDef} and creates two {\tt LISP} function definitions, namely
{\tt D-{\it RuleName}} and {\tt P-{\it Rulename}}.  The rule definition will be printed
into every selected {\TPS} channel after it is parsed.  {\it RuleDef} defaults
to the definition specified in the source file.  The current rule and source
files are updated appropriately, so that any changes can be detected with
{\tt CHANGES} and saved with {\tt DSKOUT}.  Notice that {\it RuleName} will be appended
to the current source file, if it is not already present.  In case {\it RuleDef}
cannot be parsed correctly, an error message will be printed and neither
the rule file nor the source file will be updated.  For a description of
the argument type {\it RuleDef}, see page \pageref{ruledef}.  If {\it RuleName}
is left unspecified, {\it RuleDef} must start with `{\tt RULE: }{\it RuleName}'.

\item[] \indexmexpr{Compile-File \{{\it RuleFile}\}}
compiles every inference rule specification in the source file associated
with {\it RuleFile}.  During compilation every rule will be stated.
{\it RuleFile} defaults to the current rule file as specified in the last
{\tt RuleFile} command.

\item[] \indexmexpr{State-Rule {\it RuleName}}
prints {\it RuleName} into every open channel.  If {\it RuleName} had not been
parsed before, either with {\tt Compile-File} or {\tt Define-Rule}, but is
defined in the source file, {\TPS} will try to parse it first and the print it
from its internal representation.  This is necessary in order to make
the way the rule is stated independent of the way it was specified to the
system, which could be a string, a {\tt RdClist}, etc.

\item[] \indexmexpr{List-Rules \{{\it RuleFile}\} \{{\it ListFile}\} \{{\it Style}\} \{{\it Channel}\}}
prints every rule in {\it RuleFile} into {\it ListFile} and all currently selected
{\TPS} channels.  {\it RuleFile} defaults to the current rule file.  If no
{\it ListFile} is specified, the rules will just be printed into every selected
{\TPS} channel.  {\it Style} and {\it Channel} are as in the {\tt OPEN} command and
apply to the {\it ListFile}.
\end{description}

Any form of a rule description is first converted into a \itt{LexList}, a
data structure used in an intermediate step in parsing formulas.  This
{\tt LexList} is what the preprocessing functions will actually give to
{\tt Define-Rule} as an argument.
The argument type {\it RuleDef} can be any of the following.
\begin{description}
\item[] \label{Ruledef}\index{{\it RuleDef}}
\itt{\$} [defaulted]
see {\tt Define-Rule}.

\item[] a string
the string representation of a rule definition as used in the source file.
The string may not contain any special characters. {\tt !} stands for \ASSERT.

\item[] \itt{RD}
The user will be prompted in the next line for a rule definition
without special characters.  The input must be terminated with
{\tt <esc>}\index{{\tt <esc>}} and can be aborted with \itt{\^G}.
{\bf obsolete -SI}

\item[] \itt{RDC}
The user will be prompted for a rule definition which may include
special characters.  The input must be terminated with
{\tt <esc><esc>}\index{{\tt <esc><esc>}} and can be aborted with \itt{\^G}.
See page \pageref{rdc} for a description of the input format for wffs.
{\bf obsolete -SI}

\item[] \itt{PAD}
enters the {\tt PAD} and puts the terminal (Concept) into local mode.
The restrictions on the rule definition are the same as for {\tt RD}.  You can
transmit the contents of the {\tt PAD} using the SEND key.  {\bf obsolete -SI}

\item[] \itt{VIDI}
refers to the rule definition as created with the most recent \itt{\$r} inside
\VIDI. {\bf obsolete -SI}

\item[] an identifier
can mean a string, if the identifier is bound to a string, a \itt{RdCList}, if
the identifier is bound to a {\tt RdCList}, or a \itt{LexList}, if the identifier has
a \itt{RDEF} property which is a
{\tt LexList}.  A {\tt RdCList} is created during a {\tt RdC}, a {\tt LexList} is
the result of an {\tt \$r} command in \VIDI.
\end{description}

\subsection{Some sample specifications of inference rules}
This section contains some sample inference rules which demonstrate some
features of the syntax and the rule compiler.  A more precise and complete
grammar is given in section \ref{rulegrammar}.
\begin{alltt}
\tabdivide{2}
{\tt 
D 1 }{\it H}{\tt  }\assert {\tt  A}\(\sb{\greeko}${\tt  \and  B}$\sb{\greeko}${\tt ;} &  {\rm The declaration of \ScriptH as a {\tt WFFSET} has}
{\tt P 2 }{\it H}{\tt  }\assert {\tt  A; AndE:1;} &  {\rm been omitted, since \ScriptH and \ScriptG are}
{\tt P 3 }{\it H}{\tt  }\assert {\tt  B; AndE:1;} &  {\rm predeclared to be {\tt WFFSET}s.}

{\tt D 1 }{\it H}{\tt  }\assert {\tt  A}\(\sb{\greeko}${\tt ;} &  {\rm Notice that here \ScriptH appears twice.}
{\tt D 2 }{\it H}{\tt  }\assert {\tt  B}\(\sb{\greeko}${\tt ;} &  {\rm It will be the union of the hypotheses of lines 1 }
{\tt P 3 }{\it H}{\tt  }\assert {\tt  A \and  B; AndI:1,2;} &  {\rm and 2.}

{\tt D 1 }{\it H}{\tt ,A}\(\sb{\greeko}${\tt  }\assert {\tt  B}$\sb{\greeko}${\tt ;} &  {\rm Here one of the arguments of the {\tt D}-rule will}
{\tt P 2 }{\it H}{\tt  }\assert {\tt  A }\implies{\tt  B; ImpI:1;} &  {\rm be {\tt HYP-A<O>}, the number of a line asserting}
 &  {\rm {\tt A\Subomicron}.  One of the optional arguments}
 &  {\rm of the {\tt P}-rule will be {\tt HYP-A<O>}, the number}
 &  {\rm of the line which will be {\tt A\subomicron \Assert A\subomicron}.}

{\tt D 1 }{\it H}{\tt  }\assert {\tt  A}\(\sb{\greeko}${\tt  }\implies{\tt  B}$\sb{\greeko}${\tt ;
P 2 }{\it H}{\tt ,A }\assert {\tt  B; ImpE:1;

D 1 }{\it H}{\tt  }\assert {\tt  A}\(\sb{\greeko}${\tt ;
D 2 }{\it H}{\tt  }\assert {\tt  A }\implies{\tt  B}$\sb{\greeko}${\tt ;
P 3 }{\it H}{\tt  }\assert {\tt  B; MP:1,2;

D 1 }{\it H}{\tt  }\assert {\tt  \(\forall$ x}$\sb{\greeka}${\tt  . A}$\sb{\greeko}${\tt ;} &  {\rm In this rule, {\tt D} and {\tt P} lines have the same number.}
{\tt P 1 }{\it H}{\tt  }\assert {\tt  `LCONTR` [}$\lambda${\tt x . A] B}$\sb{\greeka}${\tt ; UnivE: 1;}
{\tt B free for x in A;} &  {\rm This is legal since no conflict can arise.}
 &  {\rm The keywords {\tt free}, {\tt for} and {\tt in} must be}
 &  {\rm  lower case.}
{\tt 
D 1 }{\it H}{\tt  }\assert {\tt  `LCONTR` [}$\lambda${\tt x}$\sb{\greeka}${\tt  . A}$\sb{\greeko}${\tt ]B}$\sb{\greeka}${\tt ;} &  {\rm In the {\tt D}-rule compiled from this definition, }
{\tt P 1 }{\it H}{\tt  }\assert {\tt  $\forall$ x . A; UnivI: 1;} &  {\rm the user will have to supply {\tt x\subalpha}, {\tt B\subalpha}, and}
{\tt B is variable;} &  {\rm a list of occurrences of {\tt B\subalpha} in the assertion}
{\tt B not free in }{\it H}{\tt ;} &  {\rm of line {\tt D1}, so that {\TPS} can reconstruct {\tt A\subomicron}.}

{\tt D 1 }{\it H}{\tt  }\assert {\tt  `LCONTR` [}$\lambda${\tt x}$\sb{\greeka}${\tt  . A}$\sb{\greeko}${\tt ]B}$\sb{\greeka}${\tt ;
P 1 }{\it H}{\tt  }\assert {\tt  $\exists$ x . A; ExistI: 1;

D 1 }{\it H}{\tt  }\assert {\tt  $\exists$ x}$\sb{\greeka}${\tt  . A}$\sb{\greeko}${\tt ;
P 1 }{\it H}{\tt  }\assert {\tt  `LCONTR` [}$\lambda${\tt x.A]B}$\sb{\greeka}${\tt ; ExistE: 1;
B is variable;
B free for x in A;
B not free in }{\it H}{\tt ;

D 1 }{\it H}{\tt  }\assert {\tt  $\exists$ x}$\sb{\greeka}${\tt .A}$\sb{\greeko}${\tt ;} &  {\rm Here in {\tt RuleC} we see the use of abbreviations.}
{\tt D 2 *D2* }\assert {\tt  `LCONTR` [}$\lambda${\tt x.A]B}$\sb{\greeka}${\tt ;} &  {\rm {\tt *D2*} stands for the assertion of line {\tt D2}, which}
{\tt D 3 }{\it G}{\tt ,*D2* }\assert {\tt  C}$\sb{\greeko}${\tt ;} &  {\rm otherwise could not appear as a hypothesis.}
{\tt P 1 }{\it G}{\tt  }\assert {\tt  C}$\sb{\greeko}${\tt ; RuleC: 1,2,3;
B is variable;
B free for x in A;
B not free in }{\it H}{\tt ;
B not free in C;
B not free in $\exists${}x.A;

A 1 `LCONTR` [$\lambda${\tt x.A]B}$\sb{\greeka}${\tt ;} &  {\rm This is an equivalent way of stating the first part}
{\tt D 1 }{\it H}{\tt  }\assert {\tt  $\exists$ x}$\sb{\greeka}${\tt .A}$\sb{\greeko}${\tt ;} &  {\rm of the previous rule, now using an {\it abbwff}}
{\tt D 2 *A1* }\assert {\tt *A1*;} &  {\rm construct. {\tt *A1*} stands for the wff}
{\tt D 3 }{\it G}{\tt ,*A1* }\assert {\tt  C}$\sb{\greeko}${\tt ;} &  {\rm following {\tt A 1}.}
{\tt P 1 }{\it G}{\tt  }\assert {\tt  C}$\sb{\greeko}${\tt ; RuleC: 1,2,3;}
\end{alltt}

\subsection{The order of arguments}\label{argorder}
The following convention has been adopted to decide the order of
arguments in the {\tt D}-rules created by the rule compiler.  In
general the name of the argument will be connected to
its meaning.  This is particularly helpful if the new way of
supplying arguments to {\tt MExpr}'s interactively is utilized.

\begin{enumerate}
\item {\it dlines} in the order they appear in the rule definition.  The
argument type is {\tt LineNumber}.  The name of the argument for
the line started with `{\tt D }{\it n}' will be {\tt D}{\it n}.

\item {\it wffs} in the order new wffs appear in the rule definition.
The argument type will be {\tt RWff}, the name of the argument
for the wff `{\tt A\subomicron}' will be {\tt A<O>}.
The {\it wffs} include special arguments that stem from inverting
primitive operators.

\item {\it hypotheses} in the order unknown hypotheses appear in the rule
definition.  The argument type is {\tt LineNumber}.  The name of the argument
for the hypothesis `{\tt B\subalpha}' will be {\tt HYP-B<A>}.

\item {\it prargs}, i.e. arguments of primitive operators as they appear in
the rule definition.  The argument type depends on the types of the
arguments of the primitive operator as specified in the file {\tt PRIMOP}.

\item {\it new hypotheses}, i.e. line numbers at which to insert new lines
of the form {\obeyspaces `{\tt A\Subomicron \Assert A\Subomicron}'}.  These can
be defaulted.  The argument type is {\tt NUM}, their naming
convention is identical to those for hypotheses.

\item {\it plines} in the order they appear in the rule definition.  Their
argument type is {\tt NUM} and they can be defaulted.  The name of the
argument for the line starting with `{\tt P }{\it n}' is {\tt P}{\it n}.
\end{enumerate}
The order of arguments for the {\tt P}-rules is strictly symmetric to
the argument order of {\tt D}-rules.
\begin{enumerate}
\item {\it plines} in the order they appear in the rule definition.  The
argument type is {\tt LineNumber}.  The name of the argument for
the line started with `{\tt P }{\it n}' will be {\tt P}{\it n}.

\item {\it wffs} in the order new wffs appear in the rule definition.
The argument type will be {\tt RWff}, the name of the argument
for the wff `{\tt A\subomicron}' will be {\tt A<O>}.
The {\it wffs} include special arguments that stem from inverting
primitive operators.

\item {\it hypotheses} in the order unknown hypotheses appear in the rule
definition.  The argument type is {\tt LineNumber}.  The name of the argument
for the hypothesis `{\tt B\subalpha}' will be {\tt HYP-B<A>}.

\item {\it prargs}, i.e. arguments of primitive operators as they appear in
the rule definition.  The argument type depends on the types of the
arguments of the primitive operator as specified in the file {\tt PRIMOP}.

\item {\it new hypotheses}, i.e. line numbers at which to insert new lines
of the form {\obeyspaces `{\tt A\Subomicron \Assert A\Subomicron}'}.  These can
be defaulted.  The argument type is {\tt NUM}, their naming
convention is identical to those for hypotheses.

\item {\it dlines} in the order they appear in the rule definition.  Their
argument type is {\tt NUM} and they can be defaulted.  The name of the
argument for the line starting with `{\tt D }{\it n}' is {\tt D}{\it n}.
\end{enumerate}

\section{A grammar for specifying inference rules}\label{rulegrammar}
In the following grammar in BNF style terminal symbols are underlined.
[{\it token}]\0inf means that {\it token} can be repeated 0 or more times.
<{\it name}> means that {\it name} can be any {\tt LISP} object which is a {\it name}.
\{{\it field}\} indicates that {\it field} is optional.  Note that the case of
characters matters, i.e. capital letters have to be capital, and lower
case letters have to be lower case.  Spaces are critical only where they
are needed to separate identifiers, just as in formulas.  The symbols
`{\tt [ ] ( ) . ; , : <return> <tab>}' separate identifiers and thus need
not to be surrounded by spaces.
\begin{description}
\item[rule ::=]	 \{\uxt{RULE:} <identifier>
\uxt{;}\}\{\uxt{COMMENT:} comment\uxt{;}\}[declaration]\0inf{adiline}\0inf{pline}\1inf{restriction}\0inf

\item[adiline ::=]	 abbwff | dline | iline

\item[abbwff ::=]	  \uxt{A} <number> wff\uxt{;}

\item[iline ::=]	  \uxt{I} <number> \{hypotheses\} \uxt{\assert} assertion\uxt{;}

\item[dline ::=]	  \uxt{D} <number> \{hypotheses\} \uxt{\assert} assertion\uxt{;}

\item[pline ::=]	  \uxt{P} <number> \{hypotheses\} \uxt{\assert} assertion\uxt{;} justification\uxt{;}

\item[declaration::=]	 \uxt{CONSTANT:} wff [\uxt{,}wff]\0inf\uxt{;}
| \uxt{WFFSET:} wffset [\uxt{,}wffset]\0inf\uxt{;}

\item[hypotheses ::=]	 hyp [\uxt{,}hyp]\0inf

\item[hyp ::=]	 wff\subomicron | wffset

\item[assertion ::=]	 wff\subomicron | \uxt{`}primop\uxt{`} prarg [\uxt{,}prarg]\0inf

\item[primop ::=]	 \uxt{LCONTR} | \uxt{LEXPD} | {\it others as defined by the user}

\item[prarg ::=]	 wff | occlist | {\it others as defined by the user}

\item[occlist ::=]	 \uxt{OCC}<number> | \uxt{[}<number>[\uxt{,}<number>]\0inf\uxt{]}

\item[wffset ::=]	 <identifier>

\item[justification ::=]	 <identifier> \{\uxt{:} wff [\uxt{,}wff]\0inf\}
\{\uxt{:} lineno [\uxt{,}lineno]\0inf\}

\item[restriction ::=]	   wff \{\uxt{not}\} \uxt{free} \uxt{in} wff\uxt{;}
| wff \{\uxt{not}\} \uxt{free} \uxt{in} wffset\uxt{;}
% \comment{\{@!$^{\hbox{wff}}$@/$_{\hbox{wffset}}$\}\uxt{;}}
| wff \{\uxt{not}\} \uxt{free} \uxt{for} wff \uxt{in} wff\uxt{;}
| wff \{\uxt{not}\} \uxt{free} \uxt{for} wff \uxt{in} wffset\uxt{;}
% \comment{\{@!$^{\hbox{wff}}$@/$_{\hbox{wffset}}$\}\uxt{;}}
| wff \uxt{is} \uxt{variable}\uxt{;}
| {\it others as defined by the user}

\item[comment ::=]	 {\it any sequence of characters not containing ` ( ) or ;}

\item[wff ::=]	  {\it any wff legal in {\TPS}}
\end{description}
Note that wffs may contain the following identifiers which stand for
assertions of lines or abbreviations which have been previously defined:
\begin{description}
\item[{\tt *A}{\it n}{\tt *}]	 stands for the wff of the {\tt abbwff} with number {\it n}.

\item[{\tt *D}{\it n}{\tt *}]	 stands for the assertion of the {\tt dline} with number {\it n}.

\item[{\tt *I}{\it n}{\tt *}]	 stands for the assertion of the {\tt iline} with number {\it n}.

\item[{\tt *P}{\it n}{\tt *}]	 stands for the assertion of the {\tt pline} with number {\it n}.
\end{description}
While the first can have any type, the last three have to be of type {\tt O}.
These special identifiers do not have to, but may be typed.

% \comment{\string{KsetSize=10}
% %\input{lib:ksets.mss}
% %\input{lib:symb10.mss}}
\commandstring{IN=`\member1{)

\section{Multiple Rules of Inference}

The plain {\tt RULES} package lacks facilities to specify relatively simple
inference rules, such as a rule which would infer every conjunct in
a large conjunction.  Instead of trying to improve the syntax of rule
descriptions to include constructions such as {\tt A$_{\hbox{1}}$ \AND ... \AND A$_{\hbox{n}}$},
the solution proposed here considers simple inference rules as
indivisible steps in a ``programming language''.  The ``programming language''
is what we are concerned with here.  We would like to have simple
and easy to use facilities to build {\it multiple inference rules}
from other ones.

One can certainly think of many such {\it rule construction languages}.
Lisp itself is certainly one choice that comes to mind.  We could write
commands which would invoke other inference rules as subroutines
appropriately and so achieve the desired outline transformation.
We do not want to exclude that possibility, since there will certainly
be rules to complicated to build up any other way.  Two drawbacks of
this method have to be noted, however:  Firstly, we have to decide
in advance which inference rules we would like to have available
as multiple rules, and, secondly, one will certainly want to write many
different multiple inference rules for different logical systems, which
requires a lot of special purpose programming.

Here we propose an alternative, which will allow us to treat many cases
of multiple inference rules at the non-programmer level.

\subsection{Regular Expressions}

Regular expressions R are frequently used in computer science and have
many nice properties.  Let us define regular expressions abstractly
first.  The definition is by induction.

\begin{enumerate}
\item u\in\CapSigma is a regular expression.  \CapSigma is the {\it underlying set},
often also called the underlying {\it alphabet}.

\item If u,v\in R then u+v\in R.  This can be interpreted as union, alternation,
or disjunction.

\item If u,v\in R then uv\in R.  This can be interpreted as concatenation.

\item If u\in R then u$^{\hbox{*}}$\in R.  This is the {\it Kleene star} and represents
potentially infinite repetition.
\end{enumerate}

\subsection{Regular Expressions as Rule Constructions}
We can now exploit the simple constructive nature of regular expressions
to build our rule construction language.  Let \CapSigma be the set of
primitive rules, presumable defined by the {\tt RULES} package.  We then define
the extended set of rules R by the same kind of induction.

\begin{enumerate}
If r\in\CapSigma then r\in R

If r,s\in R then r+s\in R.  r+s stands for the alternation of the two rules:
apply either r or s, whichever is possible (i.e. matches the given input line).
There must be a restriction on the uniformity of the rules, e.g. they must
take the same number of {\it dlines} into the same number of {\it plines}.  We may
loosen the analogy with regular expressions by postulating that the
elements in a sum are tested for a match from left to right.  The rules
do not have to be exclusive.

If r,s\in R then r\&s\in R.  r\&s stands for the successive application of the
two rules.  First apply r, than apply s to the results of r.  Again, some
restriction on the number of arguments of type {\it line} will have to be imposed.

If r\in R then r$^{\hbox{*}}$\in R.  r$^{\hbox{*}}$ stands for the repeated application of r.
First r is applied to the arguments, then to the results of the first operation etc
etc until we have no possible match left.
\end{enumerate}

\subsection{Some Examples}
Here are some examples.  Overlook some problems with the actual syntax
of definitions of multiple rules; this will have to be decided later.

\begin{alltt}\tabdivide{2}
CONJ* := <CONJ>*

Rule: CONJ &  {\it then {\tt CONJ}* looks like}
D 1 H \Assert A \AND B; &  D 1 H \Assert A \AND ... \AND Z;
P 2 H \Assert A; RuleP:1; &  P 2 H \Assert A; RuleP:1;
P 3 H \Assert B; Rulep:1; &  ...
 &  P n H \Assert Z; RuleP:1;

PUSH := <PUSHU>+<PUSHE>
PUSH* := <PUSH>*

Rule: PUSHU
D 1 H \Assert $\forall$x.A \AND B;
P 1 H \Assert $\forall$x A \AND $\forall$x B; RuleQ:1;

Rule: PUSHE
D 1 H \Assert $\exists$x.A \lor B;
P 1 H \Assert $\exists$x A \lor $\exists$x B; RuleQ:1;

{\it Then {\tt PUSH} can be applied to a line of either variety and {\tt PUSH*}
will distribute a quantifier over a multiple disjunction or conjunction}

DISTU := <PUSH*>\&<CONJ*>

{\it {\tt DISTU} will push in universal quantifiers over conjunctions, then
assert the conjunctions in separate lines.}

UI* := <UI>*

{\it {\tt UI*} allows to instantiate a whole series of quantifiers}
\end{alltt}

\comment{End old documentation}

\end{comment}

\chapter{Notes on setting things up}\label{set-up}
\section{Compiling TPS and ETPS}

{\ETPS} is simply a subsystem of {\TPS}.  {\ETPS} lacks some
of the files used to build {\TPS}.
The procedure for building {\ETPS} is just like that for
building {\TPS}, except that one should type {\tt make etps}
rather than {\tt make tps}.  You can build both {\TPS} and {\ETPS}
in the same directory, but you must
make sure the bin directory is empty before building
each system because the compiled files for
{\TPS} are not compatible with those for {\ETPS}.
For simplicity in the discussion below,
we often refer to {\TPS} when it would be more precise
to refer to ``{\TPS} or {\ETPS}".

{\TPS} has been compiled in several versions of Common Lisp: Allegro
Common Lisp (version 3.1 or higher); Lucid Common Lisp; CMU Common
Lisp; Kyoto Common Lisp; Austin Kyoto Common Lisp; Ibuki Common Lisp,
a commercial version of Kyoto Common Lisp; and DEC-20 Common Lisp.
% \comment{Note: We do not recommend the use of DEC-20 Common Lisp, as
% there are many compiler bugs in it.}
  Several source files contain
compiler directives which are used to switch between the various
definitions required.

For the time being, we assume you are compiling on a Unix
or MS Windows operating system
using one of the versions of Lisp given above.
Otherwise, considerably more work will be required.  Some additional information
which may be helpful will be found as comments in the text file
{\it whatever/tps/doc/user/manual.mss} for this section (Compiling TPS) of
this User Manual.  These comments are not printed out when Scribe
processes the file.

\subsection{Compiling TPS under Unix}\label{set-up-unix}

To compile and build tps, proceed as follows:
\begin{enumerate}

\item Create a bin directory, if one does not exist.
The bin directory should be empty when you start this process.
If you have previously built tps or etps, start by removing
all files from the bin directory ({\tt rm -f bin/*}).

\item Read and follow the directions which are presented as comments in the
{\tt Makefile}. In general, this will just mean
changing the {\tt Makefile}
to show the correct pathname for your
version of Lisp (and possibly java), changing the \indexfile{Makefile} to show where remarks by
TPS users should be sent and which users are allowed privileges.

\item Issue the command `{\tt make tps}' or `{\tt make etps}'.
(Of course, this assumes you are using a Unix operating system.)
The Makefile will also try to compile the files for the Java interface.
If you do not have java compiler, this will fail, but only after
all the lisp files have been compiled.
A Java compiler is not necessary to install TPS.  The installation
will still create TPS and ETPS, but you will not be able to use the
Java interface.
Note that if you do not have a Java compiler, you can download
Java SDK (with a compiler) from http://java.sun.com/.

\item The script file tps-build-install-linux can be used to
build and install tps.  Look at it.

\item If you are using KCL or IBCL, you may get an error during compiling
which says something like `unable to allocate'.  This error indicates
that your C compiler cannot handle the size of the file that is
being compiled.  To fix this, split the offending file (e.g. {\it foo.lisp})
into smaller pieces (e.g., {\it foo1.lisp} and {\it foo2.lisp}) and replace
the occurrence of `foo' in the file \indexfile{defpck.lisp} with `foo1 foo2'.
If this doesn't work you may have to split the files again.

\item If you are using Allegro Common Lisp 5.0, the name of the core image should
end in .dxl; for example, {\tt tps3.dxl}. To achieve this, you can set tps-core-name
in the Makefile (in which case the new core image may overwrite the old one if
you rebuild), or just use the Unix {\tt mv} command to rename the core image
once it is built.

\item When {\TPS} starts up, it loads a file called \indexfile{tps3.patch} if one is
there; this contains fixes for bugs, new code which has been added
since {\TPS} was last built, etc. 
%[[This doesn't really work, so we delete this example]]
%For example, if you wish to change expert-list after tps has been built, just put the appropriate line
%into \indexfile{tps3.patch}, using the format of the example in \indexfile{tps3-save.lisp}.
After you build a new {\TPS}, you may wish to delete (or save in a
different file) the contents of the old \indexfile{tps3.patch} file. Keeping the
empty file there assures that it will be in the right place when you
need it again.

\item After loading the patch file, {\TPS} will look for a file called \indexfile{tps3.ini}
in the same directory as the patch file and (if the user is an expert)
for a file also called \indexfile{tps3.ini}
in the directory from which the user starts {\TPS}.  (These directories may or may not be the same).
Before using {\TPS}, you may want to change these initialization files.
\end{enumerate}

% \begin{comment}
% If you don't use the Makefile, you may have to do these things in addition
% to those mentioned above:
% 
% Reinstate the definition of {\tt expert-list} which is currently commented
% out of the file \indexfile{tps3-save.lisp} (modifying it as appropriate; it should contain
% a list of the user names of all those users who are to be allowed expert privileges
% while using {\TPS}).
% 
% Make any changes required to the file {\tt tps3.sys}.  This is a file which
% is loaded during compilation and each time {\TPS} is run which
% sets the values of certain system-specific variables like directories,
% file names, etc.  For {\ETPS}, change the file {\tt etps.sys}.
% 
% {\TPS} and {\ETPS} have a command called {\tt REMARK}, which allows the user
% to enter a string to be sent to the teacher/maintainer.
% If you have a Unix system and are running CMU Common Lisp, Lucid (or Sun)
% Common Lisp, KCL or IBCL, you can use email to send the remark.  To do
% this, set the variable {\tt MAIL-REMARKS} in the file {\tt \{TPS3,ETPS\}.SYS} to
% a string containing the mail addresses of those who should receive the
% remark.  If you do not desire this, set {\tt MAIL-REMARKS} to NIL and
% set {\tt REMARKS-FILE} to the name of the file that should store the
% remarks instead.
% 
% Examine the file {\tt tps-compile}.  This is a Unix script file which will
% automatically load and compile all the source files.
% Make any changes required.  For {\ETPS}, change the file {\tt etps-compile}.
% 
% At your system top-level, just type {\tt tps-compile}.  Compilation should
% take anywhere from 1.5 hours (CMU Common Lisp) to 5 hours (Kyoto Common Lisp),
% if all goes well.
% 
% Then type {\tt tps-build} to create a core image.
% 
% If you have most of the files already compiled and just want to build
% a new system, use the files {\tt tps-build} and {\tt tps-build.lisp}, or
% for {\ETPS}, {\tt etps-build} and {\tt etps-build.lisp}. {\tt tps-build}
% checks the dates on source files and their compiled versions, and
% compiles any files which have been changed since their compiled
% versions were created, or which were not already compiled. However, it
% may not recompile files containing macros which have been changed, or
% catch other basic changes, so when major changes have occurred, it is
% best to use {\tt tps-compile} (which recompiles all files)
% before using {\tt tps-build}.
% \end{comment}
If your lisp is not in the list above, you may need to change some of the
system-dependent functions. The features used by TPS are `tops-20', `lucid',
`:cmu', `kcl', `allegro' and `ibcl'.  System-dependent files include

\begin{description}
\item[{\tt SPECIAL.EXP}]	 Contains symbols which cannot be exported in some lisps. These
are found by trial and error.

\item[{\tt BOOT0.LISP, BOOT1.LISP}]	 Contain  some  lisp and operating-system dependent
functions and macros, like file manipulation.

\item[{\tt TOPS20.LISP}]	 Redefining the lisp top-level, saving a core image, exiting, etc.

\item[{\tt TPS3-SAVE.LISP}]	 Some I/O functions which should work for Unix lisps, also the
original definition of {\tt expert-list}.

\item[{\tt TPS3-ERROR.LISP}]	 Redefinitions of trapped error functions, as used in {\ETPS}.
\end{description}

\subsection{Compiling TPS under MS Windows}\label{set-up-windows}

When compiling {\TPS} under MS Windows,
there is a lisp file \indexfile{make-tps-windows.lisp}
which can be used instead of the Makefile.  The file
\indexfile{make-tps-windows.lisp} was designed to work
for Allegro Lisp version 5.0 or later, though some parts of
it should work for other versions of lisp.

To compile and build {\TPS} under MS Windows, perform the following steps:

\begin{enumerate}
\item Create a folder for TPS, e.g., open `My Computer',
%then `C:', then `Program Files', and choose `New > Folder'
then `C:', then `Program Files', and choose `New $>$ Folder'
from the file menu.  Then rename the created folder to
%TPS.  Now you have a folder C:$\setminus$Program Files$\setminus$TPS$\setminus$.
TPS.  Now you have a folder C:$\setminus$Program Files$\setminus$TPS$\setminus$.

\item Download the gzip'd tps tar file and unzip it (using,
for example, NetZip or WinZip) into the C:$\setminus$Program Files$\setminus$TPS$\setminus$
folder.

\item Create a C:$\setminus$Program Files$\setminus$TPS$\setminus$bin folder, if one does not exist.
Also, create a top level folder C:$\setminus$dev.  This is so TPS can
send output to $\setminus$dev$\setminus$null.

\item Determine if you have a Java compiler.  You can do this by running
`Find' or `Search' (on `Files and Folders') (probably available
through the `Start' menu) and searching for a file named `javac.exe'.
A Java compiler is not necessary to install TPS.  The installation
will still create TPS and ETPS, but you will not be able to use the
Java interface.  Note that if you do not have a Java compiler, you can
download Java SDK (with a compiler) from http://java.sun.com/.

\item Lisp (preferably Allegro 5.0 or greater) will probably be in
`Programs' under the `Start' menu.  Start Lisp (by choosing it from
there) and do the following:
\begin{alltt}
(load `C:\(\setminus\setminus\)Program Files\(\setminus\setminus\)TPS\(\setminus\setminus\)make-tps-windows.lisp')
\end{alltt}
This should prompt you for information used to compile and build TPS,
as well as compiling the Java files (if you have a Java compiler).  It
will also create executable batch files, e.g., C:\(\setminus\)Program Files\(\setminus\)TPS\(\setminus\)tps3.bat
which you can use to start {\TPS} after it has been built.

\item After Lisp says `FINISHED', enter {\tt (exit}).
\end{enumerate}

If for some reason \indexfile{make-tps-windows.lisp} fails to compile and build
{\TPS}{and {\ETPS}}, you can look at \indexfile{make-tps-windows.lisp} to try to figure
out how to build it by hand.  The remaining steps are an outline
of what is needed.

\begin{enumerate}
\item If \indexfile{make-tps-windows.lisp} did not create the files
\indexfile{tps3.sys} and \indexfile{etps.sys},
rename tps3.sys.windows.example to tps3.sys,
and rename etps.sys.windows.example to etps.sys.
You may want to edit the value of the constant
expert-list to include your user name.  In Windows, this is often
`ABSOLUTE', which is already included on the list.
If the {\TPS} directory is something other than
`C:\(\setminus\)Program Files\(\setminus\)TPS\(\setminus\)', then you will need to edit tps3.sys
and etps.sys by replacing each
`C:\(\setminus\setminus\)Program Files\(\setminus\setminus\)TPS\(\setminus\setminus\)' with `whatever\(\setminus\setminus\)'.
Also, you will need to edit the
files \indexfile{tps-compile-windows.lisp}, \indexfile{tps-build-windows.lisp}
(and \indexfile{etps-compile-windows.lisp} and \indexfile{etps-build-windows.lisp}
if you intend to use etps) by replacing the line
\begin{alltt}
  (setq tps-dir `C:\(\setminus\setminus\)Program Files\(\setminus\setminus\)TPS\(\setminus\setminus\)'))
\end{alltt}
by
\begin{alltt}
  (setq tps-dir `whatever\(\setminus\setminus\)'))
\end{alltt}

\item Make sure the bin directory is empty.
If you have previously built tps or etps, start by
sending all files from the bin directory to the Recycle Bin.

\item Run Lisp.  Load the tps-compile-windows.lisp
file from C:\(\setminus\)Program Files\(\setminus\)TPS\(\setminus\) as follows:
\begin{alltt}
(load `C:\(\setminus\setminus\)Program Files\(\setminus\setminus\)TPS\(\setminus\setminus\)tps-compile-windows.lisp')
\end{alltt}
This will compile the lisp source files in the C:\(\setminus\)Program Files\(\setminus\)TPS\(\setminus\)lisp
folder into the C:\(\setminus\)Program Files\(\setminus\)TPS\(\setminus\)bin folder.

\item Exit and restart lisp.  Load the tps-build-windows.lisp
%file from C:\Program Files\TPS\ as follows:
file from C:\(\setminus\setminus\)Program Files\(\setminus\setminus\)TPS\(\setminus\setminus\) as follows:
\begin{alltt}
(load `C:\(\setminus\setminus\)Program Files\(\setminus\setminus\)TPS\(\setminus\setminus\)tps-build-windows.lisp')
\end{alltt}
If you try to load tps-build-windows.lisp after loading
tps-compile-windows.lisp without restarting Lisp, you will probably
get an error because packages are being redefined.  So, it
is important to exit and start a new Lisp session before
loading tps-build-windows.lisp.
The end of tps-build-windows.lisp calls tps3-save, which
saves the image file.  Under Allegro, this should be tps3.dxl.
(The name and location of the image file is determined by
the values of sys-dir and save-file in tps3.sys.)

\item Repeat the previous steps using etps-compile-windows.lisp and etps-build-windows.lisp
to compile and build {\ETPS}.

\item If you have a Java compiler, use it to compile the java files in
C:\(\setminus\)Program Files\(\setminus\)TPS\(\setminus\)java\(\setminus\)tps
and then
C:\(\setminus\)Program Files\(\setminus\)TPS\(\setminus\)java\(\setminus\)
(see section \ref{compiling-java})

\item If \indexfile{make-tps-windows.lisp} did not create the batch files
tps3.bat and etps.bat,
create the batch file tps3.bat containing something like
\begin{alltt}
jecho off
call `C:\(\setminus\)<lisppath>\(\setminus\)alisp.exe' -I `C:\(\setminus\)Program Files\(\setminus\)TPS\(\setminus\)tps3.dxl'
\end{alltt}
and the batch file etps.bat containing something like
\begin{alltt}
jecho off
call `C:\(\setminus\)<lisppath>\(\setminus\)alisp.exe' -I `C:\(\setminus\)Program Files\(\setminus\)TPS\(\setminus\)etps.dxl'
\end{alltt}
You need the quotes because Windows easily gets confused about spaces
in pathnames.  You should be able to double click on tps3.bat to
start {\TPS}.

\item If \indexfile{make-tps-windows.lisp} did not create the batch files
for starting {\TPS} and {\ETPS} with the Java interface, then you
can create files like \indexfile{tps-java.bat} containing something like
\begin{alltt}
jecho off
call `C:\(\setminus\)<lisppath>\(\setminus\)alisp.exe' -I `C:\(\setminus\)Program Files\(\setminus\)TPS\(\setminus\)tps3.dxl' -- -javainterface java -classpath `C:\(\setminus\)Program Files\(\setminus\)TPS\(\setminus\)java' TpsStart
\end{alltt}
(See section \ref{using-java} for more command line options associated with the Java interface.)

\end{enumerate}

Double clicking on the batch files tps3.bat and etps.bat should start
{\TPS} and {\ETPS}, respectively.  Also, if you had \indexfile{make-tps-windows.lisp}
compile the code for
the Java interface and build the batch files for starting
{\TPS} with the Java interface (or you have done this manually),
then there should be several batch
files with names like \indexfile{tps-java.bat} and \indexfile{etps-java-big.bat}.
Executing these should start {\TPS} or {\ETPS} with the Java interface.

An alternative to using batch files to start {\TPS} using Allegro Lisp is as follows:

\begin{enumerate}
\item Put a copy of the Lisp executable (such as lisp.exe or alisp8.exe) into the
C:\(\setminus\)Program Files\(\setminus\)TPS\(\setminus\) folder, and rename it tps3.exe.
(You may only need to explicitly change `lisp' to `tps3'
in order to rename lisp.exe to tps3.exe.)

\item Copy acl*.epll or acl*.pll (or similarly named files)
from the Allegro Lisp directory to the TPS directory.
You may also need to copy a license file *.lic
from the Allegro Lisp directory to the TPS directory.

\item Double-click on tps3.exe to start up TPS.  This will automatically
find tps3.dxl as the image (since it is in the same directory and has the
same root name).  If Allegro complains that some file isn't found,
look for that file under the Allegro Lisp directory and copy it to
the TPS directory.

\end{enumerate}

\subsection{Compiling the Java Interface}\label{compiling-java}

There is a Java interface for {\TPS} supporting menus and pop-up
windows.
To use this interface, {\TPS} must be able to use sockets
and multiprocessing.
Currently it seems that these features are both
implemented only in
Allegro Lisp (version 5.0 or later).

To compile the java code
under Unix,
simply cd to the directory `whatever/tps/java/tps' and call
\begin{alltt}
javac *.java
\end{alltt}
This should create a collection of .class files
in the java/tps directory.  Then cd to `whatever/tps/java'
and call
\begin{alltt}
javac *.java
\end{alltt}
This should create a collection of .class files
in the java directory.

Compiling the java code under Windows is a bit
more complicated.  There is a Lisp file
\indexfile{make-tps-windows.lisp} provided
with the distribution which should be able
to compile the Java files if you load
\indexfile{make-tps-windows.lisp} in
Allegro Lisp.  (See section \ref{set-up-windows}.)

If you must compile the java code under Windows
manually, the following hints may help.
If the version of Windows allows
the user to bring up a DOS shell, you should be
able to chdir to `whatever\(\setminus\)TPS\(\setminus\)java\(\setminus\)tps' and call
\begin{alltt}
javac *.java
\end{alltt}
Then do the same under `whatever\(\setminus\)TPS\(\setminus\)java'.
Otherwise, you might be able to create a batch file
temp.bat containing the following code:
\begin{alltt}
chdir whatever\(\setminus\)TPS\(\setminus\)java\(\setminus\)tps
javac *.java
chdir whatever\(\setminus\)TPS\(\setminus\)java\(\setminus\)
javac *.java
\end{alltt}
Then you can double click on the icon for the batch file
to get Windows to execute it.
If Windows cannot find the executable `javac',
then you can either write the full path (`C:\(\setminus\setminus\)whatever\(\setminus\setminus\)javac')
or include the appropriate directory in the PATH environment variable.
In Windows XP, the PATH environment
variable can be changed by opening the Control Panel, then System,
then choosing Advanced and Environment Variables.

\section{Initialization}\label{Initialization}

\subsection{Initializing {\TPS}}

There can be one \indexfile{tps3.ini} file in the
directory where tps is built which will be loaded for all
users, and each \indexother{expert} user can have an individual \indexfile{tps3.ini} file in
the directory from which he calls {\TPS}.
For nonexperts, the common \indexfile{tps3.ini} file will be loaded quietly
(without any indication this is being done).

For TeX files generated by {\TPS} to work correctly,
you should set your Unix environment variable TEXINPUTS appropriately, so that TeX
can find the \indexfile{tps.sty} file.  The \indexfile{tps.sty} file
can be found in the {\it whatever/doc/lib/} directory.

After loading this common \indexfile{tps3.ini} file, {\TPS} then looks for an individual's
\indexfile{tps3.ini} file in the directory from which the individual starts {\TPS}.
This should be used by an individual
user, for tailoring the system to the needs of a particular person (or a
particular computer). For example, user1's \indexfile{tps3.ini} file might contain
appropriate settings for garbage collection flags in several variants of Lisp, as well
as a preferred default for DEFAULT-MS, and so on.
Also, user1 might wish to have in his \indexfile{tps3.ini} file the line
\begin{alltt}
(set-flag 'default-lib-dir '(`/whatever/tps/library/user1/'))
\end{alltt}
to specify his library directory (see section \ref{library} for more details).

Also in the \indexfile{tps3.ini} file, you can define aliases. For example, it may be useful to
have several different settings for the \indexflag{TEST-THEOREMS} flag used by \indexcommand{TPS-TEST},
and you can define aliases to switch between them as follows:

\begin{tpsexample}
(alias test-long `(set-flag 'test-theorems '((user::thm1  user::mode1) ..etc..))')

(alias test-default `(set-flag 'test-theorems '((user::thm2  user::mode2) ..etc..))')

(alias test-short `(set-flag 'test-theorems '((user::thm1  user::mode1) ..etc..))')
\end{tpsexample}

The last line of your \indexfile{tps3.ini} file should be {\tt (set-flag 'last-mode-name `')}, so that
the flag \indexflag{LAST-MODE-NAME} will start off empty. Also, you should set the value of
\indexflag{RECORDFLAGS} to include \indexflag{LAST-MODE-NAME}, so that \indexcommand{DATEREC}
will properly record what mode you were using at the time.

The flags \indexflag{INIT-DIALOGUE} and \indexflag{INIT-DIALOGUE-FN} should be mentioned
here; if the former is T, then after loading the two {\it .ini} files, {\TPS} will call the
function named by INIT-DIALOGUE-FN. My \indexfile{tps3.ini} file sets these flags to T and
INIT-DEFINE-MY-DEFAULT-MODE respectively, so that on startup I have a new mode
MY-DEFAULT-MODE which contains my default settings of all the flags. See the help messages
of these flags for more information.

\subsection{Initializing {\ETPS}}

There is a common \indexfile{etps.ini} file which is loaded when a user starts {\ETPS}.
This can be especially useful if students will be using {\ETPS} for a class.
The \indexfile{etps.ini} file can be used to limit what students can
do while using {\ETPS}.

One thing you may wish to do is to prevent students from being able to
access Lisp directly.
First, the flag \indexflag{EXPERTFLAG}, which, if false, does not allow the user to
enter arbitrary forms for evaluation.
For this purpose, the flag \indexflag{EXPERTFLAG} should be set to NIL
in the \indexfile{etps.ini} file.

A list of \indexother{expert}s containing
the user id's of persons allowed to change the expertflag to true (e.g.,
maintainers) is given in the \indexfile{Makefile}.  You should change this list
before building {\ETPS}.

The second way to keep students out of {\TPS} internals is to trap all
errors, and prevent students from entering the break loop.
There is a command
\begin{alltt}
(setq *trap-errors* t)
\end{alltt}
in the distributed \indexfile{etps.ini} file which
does this for Allegro Lisp.
The file \indexfile{tps3-error.lisp} has this
set up properly for DEC-20, Kyoto Common Lisp and Ibuki Common Lisp,
but you may have to do some work on this if you are using some other lisp.
Basically, the idea is that if the debugger is called, an immediate
throw back to the top level is performed.


\section{Starting {\TPS}}

Look at the \indexfile{aliases-dist} file and the run-* script files
for examples of how to start tps.

In some lisps, tps will be an executable file which can be
executed directly.

If you are using CMULISP, instead of the above use the command
\begin{alltt}
cmulisp -core tps \&
\end{alltt}
where cmulisp is the name by which you call CMULISP.

If you are using Allegro Common Lisp 5.0 or greater, you can use the command
\begin{alltt}
lisp -I tps3.dxl \&
\end{alltt}
where tps3.dxl is the file that was created when {\TPS} was built.

If you are using a version of Allegro Common Lisp prior to 5.0,
then an executable file should have been created by the Makefile.
You can simply call this executable to start {\TPS}.  For example,
\begin{alltt}
tps3 \&
\end{alltt}

There are several command line switches that control different
options for starting {\TPS}.  For more information about these
options, in {\TPS} one can execute HELP \indexother{COMMAND-LINE-SWITCHES}.

\section{Using {\TPS} with the X window system}\label{X}

{\TPS} can be run under the X window system (X10R4, X11R3 or X11R4),
with nice output including mathematical symbols, by doing the
following.

\begin{enumerate}
\item For X10R4: Make sure that
the font directory {\tt fonts}
is in your {\tt XFONTPATH}.

\item For X11R3 or X11R4: Add the fonts directory to your font path by a
\begin{alltt}
{\tt xset +fp whatever/tps/fonts}
\end{alltt}
The {\tt +fp} adds the font to the start
of your font path, so the {\TPS} fonts will override any other fonts of
the same name in your font path. You may wish to put this {\tt xset}
command in the .Xclients or .xinitrc file in your home directory,
or add this command to the `Startup Programs' on your computer.
\end{enumerate}

Then start
{\TPS} by
\begin{alltt}
{\tt \%xterm -fn vtsingle -fb vtsymbold -e tps}
\end{alltt}
where {\tt tps} is the complete name of the
executable file, and, of course, you can add fancy
things like geometry, side-bar, etc.
If you are using CMULISP, instead of the above use the command
\begin{alltt}
{\tt \%xterm -fn vtsingle -fb vtsymbold -e cmulisp -core tps \&}
\end{alltt}
where cmulisp is the name by which you call CMULISP.

If you are using Allegro Common Lisp 5.0, you can use the command
\begin{alltt}
{\tt \%xterm -geometry 80x48+4+16 '\#+963+651' -fn vtsingle -fb vtsymbold  -n Tps3jCOMPUTERNAME -T Tps3jCOMPUTERNAME -sb -e lisp -I tps3.dxl \&}
\end{alltt}
where tps3.dxl is the executable file.
(Here COMPUTERNAME is the name of the computer on which you are running;
this feature is optional, of course.)


  Demonstrations are easier to see if you use the X10 fonts gallant.r.19.onx
and galsymbold.onx, which are included with this distribution, in place of
vtsingle and vtsymbold.  These fonts are very large.

Thus, to start up tps using Allegro Common Lisp 5.0  in an X window
with large fonts, you can use the command
\begin{alltt}
{\tt \%xterm -geometry 82x33+0+0 '\#+963+651' -fn gallant.r.19 -fb galsymbold  -n Tps3jCOMPUTERNAME -T Tps3jCOMPUTERNAME -sb -e lisp -I tps3.dxl \&}
\end{alltt}

The fonts vtsingle.bdf, vtsymbold.bdf, gallant.r.19.bdf and galsymbold.bdf
are provided for use with X11.

When {\TPS} starts, switch to style
{\tt XTERM} as follows:

\begin{alltt}
{\tt <0>style xterm}
\end{alltt}

Also, if you see blinking text instead of special symbols,
then try changing the value of the flag \indexflag{XTERM-ANSI-BOLD}
to 49 as follows:\footnote{In 2005 (and previously), a value of 53
worked for XTERM-ANSI-BOLD at
CMU while using xterm version XFree86 4.2.0(165), and a value of 49
worked at Saarbrucken while using xterm version X.Org 6.7.0(192).}

\begin{alltt}
{\tt <0>xterm-ansi-bold 49}
\end{alltt}

If the TPS fonts are not being displayed properly on your screen, the
reason might be that many recent Linux systems are using a UTF-8 
\index{UTF-8} locale,
while the TPS fonts seem to work only in the traditional POSIX
\index{POSIX}
locale. To get the standard POSIX behavior, unset the environment
variable LC\_ALL.\index{LC\_ALL} This can be accomplished by executing the Linux command
\begin{alltt}
{\tt unset LC\_ALL}
\end{alltt}
or
\begin{alltt}
{\tt setenv LC\_ALL C.}
\end{alltt}
If LC\_ALL is unset, all the other LC\_* environment variables are ignored.

If you resize the X window, you should change the setting of the flag
{\tt RIGHTMARGIN}.

\section{Using {\TPS} with the \indexother{Java Interface}}\label{using-java}

There is a Java interface for {\TPS} supporting menus and pop-up
windows.
To use this interface, {\TPS} must be able to use sockets
and multiprocessing.
Currently it seems that these features are both
implemented only in
Allegro Lisp (version 5.0 or later).
In order for the Java windows to work, the
TCP IP driver on your computer must be activated.  Therefore, if the
Java interface does not work on your computer, you may be able to
remedy the problem by starting up internet connections.

The Java code for the interface is distributed under
the `java' and `java/tps' directories.  The code
in the `java' directory is used only to start the java
{\TPS} interface.  The actual code for running the interface
is in a `tps' package under `java/tps'.

To start TPS with the Java interface, you must supply
appropriate command line arguments.  For example, under Unix,
\begin{alltt}
lisp -I whatever/tps/tps3.dxl
     -- -javainterface cd whatever/tps/java \(\setminus\); java TpsStart
\end{alltt}
The command line argument `-javainterface' tells TPS that it should
run with the Java interface.  The command line arguments that follow
should form a shell command which cd's to the directory where the Java
code is, then calls java on the TpsStart class file.  (Note that the
shell command separator `;' needs to be quoted to `\(\setminus\);'.)

You may also
want to redirect the \TPS output to /dev/null, i.e., call
\begin{alltt}
lisp -I whatever/tps/tps3.dxl
     -- -javainterface cd whatever/tps/java \(\setminus\); java TpsStart > /dev/null
\end{alltt}
since the output the user needs to see shows up in the Java window.
Furthermore, if you want to continue to use the shell from which you
started {\TPS}, use \& to start run it in the background:
\begin{alltt}
lisp -I whatever/tps/tps3.dxl
     -- -javainterface cd whatever/tps/java \(\setminus\); java TpsStart > /dev/null \&
\end{alltt}

There are other command line arguments which can be sent to TpsStart.
These must be preceeded by the command line argument -other so
that {\TPS} can distinguish these from the shell command used to
start the java interface.

Some command line arguments control the size of fonts.
For example,
\begin{alltt}
lisp -I whatever/tps/tps3.dxl
     -- -javainterface cd whatever/tps/java \(\setminus\); java TpsStart -other -big > /dev/null \&
\end{alltt}
tells Java to use the bigger sized fonts.  The command line argument
-x2 tells Java to multiply the font size by two.
The command line argument -x4 tells Java to multiply the font size by four.

The command line argument `-nopopups' will make the Java interface
act more like the X window interface.  First, there will be a Text Field
at the bottom of the Java window used to enter commands.  Second, the
user is prompted for input using this TextField instead of prompting
the user via a popup window.  For example,
\begin{alltt}
lisp -I whatever/tps/tps3.dxl
     -- -javainterface cd whatever/tps/java \(\setminus\); java TpsStart -other -nopopups > /dev/null \&
\end{alltt}
Note that to enter commands into the TextField, the user may need
to focus on the TextField by clicking on it.

Finally, there are command line arguments `-maxChars', `-rightOffset',
`-bottomOffset', `-screenx', and `-screeny'.  Each of these should be
immediately followed by a non-negative integer.  For example,
\begin{alltt}
lisp -I whatever/tps/tps3.dxl
     -- -javainterface cd whatever/tps/java \(\setminus\); java TpsStart
     -other -maxChars 50000 -rightOffset 20 -bottomOffset 30 -screenx 900 -screeny 500 > /dev/null \&
\end{alltt}
starts the Java interface with a buffer size big enough to hold 50000 characters,
20 pixels of extra room at the right of the output window, 30 pixels of extra
room at the bottom of the output window, and an initial window size of 900 by 500 pixels.
These command line arguments are useful since the optimal default values may vary with
different machines and different operating systems.

Another way to run {\TPS} using the java interface is to start
{\TPS} without the `-javainterface' command line, then
use the {\TPS} command \indexcommand{JAVAWIN} to start the Java interface.
Once the command \indexcommand{JAVAWIN} is executed, all interaction
with this {\TPS} must be conducted via the Java interface.
For \indexcommand{JAVAWIN} to work, the flag \indexflag{JAVA-COMM}
must be set appropriately in the file {\tt tps3.sys} (or {\tt etps.sys}
for {\ETPS}) before the image file for {\TPS}{or {\ETPS}} is built.
For example, in Unix, {\tt tps3.sys} should contain a line like
\begin{alltt}
(defvar java-comm `cd whatever/tps/java ; java TpsStart')
\end{alltt}
In Windows, {\tt tps3.sys} should contain a line like
\begin{alltt}
(defvar java-comm `java -classpath C:\\whatever\\TPS\\java TpsStart')
\end{alltt}

Note: Resizing the main Java window for {\TPS} will automatically
adjust the value of the flag \indexflag{RIGHTMARGIN}.

\section{Using {\TPS} within Gnu Emacs}

The following will produce output within Emacs, which may be useful as an
alternative to creating a script file.

First start up Emacs, then type {\tt M-x shell} (where M-x is meta-x, or more
likely escape-x), then {\tt tps3} (or whatever alias you have defined to start
up {\TPS}). It is advisable to make your first commands {\tt style generic} and
{\tt save-flags-and-work {\it filename}}, so that the output will be readable
and a work file will be written.

Then you can use {\TPS} as normal (except that where you would normally type
{\tt \^G <Return>} to abort a command, you must now type {\tt \^Q\^G<return>}).

At the end of your session, you can rename the buffer with {\tt M-x rename-buffer}
and save it with {\tt \^X\^S}. Then type {\tt exit} twice: once to leave {\TPS} and once
to leave the shell.

Conversely, depending on the particular local configuration of your
version of Lisp, you may be able to run Emacs from within {\TPS}, using
the \indexmexpr{LEDIT} command.


\section{Running {\TPS} in Batch Mode or from \indexother{Omega}}

There are two methods of batch processing in {\TPS}: work files and UNIFORM-SEARCH. Both of them are invoked
by command line switches when {\TPS} is first started.

When in batch mode, {\TPS} will write to a file {\tt tps-batch-jobs} in your home directory
to confirm that the job has begun, and again to confirm that it has ended.

The point of these switches is that they can be used to run {\TPS} with the Unix commands {\tt at}, {\tt batch} and {\tt cron},
without requiring interaction from the user.

Note: it is possible that your Lisp handles switches on the command line differently. For example, Allegro Lisp uses {\tt --}
to separate switches for Lisp itself and switches for the core image, so in Allegro all of the examples below should begin
{\tt tps -- } rather than {\tt tps }.

\subsection{Batch Processing Work Files}

{\TPS} has a command line switch \indexother{-batch} which allows the user to run a work file.
Assuming that {\tt tps} is the command which starts {\TPS} on your system, and that you have a work
file {\tt foo.work} in your home directory, the command
\begin{tpsexample}
tps -batch foo
\end{tpsexample}
is equivalent to starting {\TPS}, typing {\tt execute-file foo}, and then exiting {\TPS} when the work file finishes.

To redirect the output of this process to a file {\tt bar}, use
\begin{tpsexample}
tps -batch foo -outfile bar
\end{tpsexample}
This will redirect the lisp streams {\tt *standard-output*}, {\tt *debug-io*}, {\tt *terminal-io*} and {\tt *error-output*}
to the file {\tt bar}. To redirect absolutely everything, use the Unix redirection commands > and >> instead. The file
{\tt /dev/null/} is a valid output file.

Examples:
\begin{tpsexample}
tps -batch thm266
{\it runs thm266.work through tps3, showing the output on the terminal.}
tps -batch thm266 -outfile thm266.script
{\it does the same but directs the output to thm266.script.}
tps -batch thm266 -outfile /dev/null
{\it does the same but discards the output.}
\end{tpsexample}

\subsection{Interactive/Omega Batch Processing}

The \indexother{-omega} switch allows the user to start TPS, run a work file, interact with TPS and then save the
resulting proof on exiting TPS. As the name suggests, this switch is used by the Omega system
(see \cite{Benzmuller97} and \cite{Benzmuller98b}).
When this switch is present, the -outfile switch is used to name the resulting proof file.
The saved proof will be the current version of whichever proof was active at the end of the work file.
If -outfile is omitted, TPS will use `<proofname>-result.prf' as the filename.
If there is no dproof created by the workfile, the saved proof will be
whichever proof is current when the user types EXIT, and it will be named
`tps-omega.prf'

Example:
\begin{tpsexample}
tps -omega -batch thm266 -outfile thm266
{\it starts TPS, runs thm266.work and leaves the TPS window open for the user to interact;
when the user exits, TPS will write a file thm266.prf}
\end{tpsexample}

\subsection{Batch Processing With UNIFORM-SEARCH}

The command line switch \indexother{-problem} tells {\TPS} to run \indexcommand{UNIFORM-SEARCH} on the given problem (which
must be the name of a gwff either in the library or internal to {\TPS}). The user can also specify the mode and
searchlist to be used, with the \indexother{-mode} and \indexother{-slist} switches. If either of these is omitted,
the mode UNIFORM-SEARCH-MODE and the searchlist UNIFORM-SEARCH-2 will be used by default.

The switch {\tt -outfile} may be used to redirect output, as in the example above. Other output may be generated by
specifying the \indexother{-record} switch, which takes no arguments, and which forces {\TPS} to call DATEREC and SAVEPROOF
after the proof is finished. {\tt -record} will also insert into the library the mode which is generated by
UNIFORM-SEARCH.

Examples:
\begin{tpsexample}
tps -problem x2138
{\it Search for a proof of X2138 using the default mode and searchlist; send output to the standard output}
tps -problem x2138 -mode mode-x2138 -record -outfile x2138.script
{\it Search as above, but use mode-x2138 to set all the flags that are not set by the default searchlist.
Send the output to x2138.script, and when the search is finished call daterec, save x2138.prf and insert
the new mode x2138-test-mode into the library}
tps -problem difficult-problem -mode my-mode -slist my-slist -record >> /dev/null/
{\it Search for a proof of difficult-problem, fixing all of the flags in my-mode and varying the flags in my-slist
as specified by that searchlist. If a proof is found, record it, the time taken, and the successful mode, but throw
away all other output.}
\end{tpsexample}

\section{Calling {\TPS} from Other Programs}

The command line switches
\indexother{-service} and \indexother{-lservice}
start {\TPS} in a way that accepts requests to prove theorems automatically for
other programs.  Both command line switches connect with other programs
via sockets used to communicate requests and responses.
Descriptions of programming with sockets can be found on the web.
Performing a search for `sockets' via Google (for example) yields millions of results.
We will assume some familiarity with communication via sockets here.

With respect to {\TPS} started with \indexother{-service} and \indexother{-lservice}
there are two relevant sockets: {\tt inout} and {\tt serv}.
The socket {\tt serv} is intended to connect {\TPS} with \indexother{MathWeb}
(see the website {\tt http://www.mathweb.org/})
but is practically unused by {\TPS} at the moment.
All the information described here
is communicated via the {\tt inout} socket.
The only difference between the command line switches
\indexother{-service} and \indexother{-lservice}
involve how these sockets are initialized.

For the purposes of this description, we refer
to the program calling {\TPS} as the `client'.
We assume the client is running on a machine called {\tt clienthost}.

\subsection{Establishing Connections}

Assume the client has two passive sockets {\tt clientio} and {\tt clientserv}
at port numbers {\tt clientioport} and {\tt clientservport}, respectively.
Start {\TPS} with \indexother{-service} as follows:
\begin{alltt}
tps -service <clienthost> <clientioport> <clientservport>
\end{alltt}
{\TPS} will connect to the client via the given hostname and port numbers
establishing the {\TPS} sockets {\tt inout} and {\tt serv}.

To use the command line switch \indexother{-lservice}
we must assume the client and {\TPS} are running on the same machine.
(The `l' in \indexother{-lservice} stands for `local'.)
Assume the client has a passive socket
with port number {\tt clientport1}.  On the same machine start {\TPS} with
\indexother{-lservice} as follows:
\begin{alltt}
tps -lservice <clientport1>
\end{alltt}
After initialization {\TPS} will open two new ports {\tt inout} and {\tt serv}
and send the port numbers to the client via {\tt clientport1}.
These port number are sent as a character string of the form
`{\tt (inoutport servport})'.
The client should take these values and connect to the sockets {\tt inout} and {\tt serv}.
At this point the communication between the client and {\TPS} works the same
way as with \indexother{-service}.

\subsection{Socket Communication}

The client and {\TPS} is communicate messages via the established sockets
using a sequence of bytes (ASCII characters).  The special byte {\tt 128}
(character {\tt \%null}) indicates the end of a message.

\subsection{Ping-Pong Protocol}

Before sending requests to {\TPS} the client must first
follow a ping-pong protocol.  The client sends a message
{\tt (PING <clientname>}) via the {\tt inout} socket.
{\TPS} should respond with
{\tt (<clientname> PONG <TPSname>}).
The client reads this and begins sending requests to
{\TPS} using the identifier {\tt <TPSname>}.

\subsection{Requests}

Every request is of the form
{\tt (<TPSname> <request> . . .})
and is sent to {\TPS} via the {\tt inout} socket.
The requests the client can send to {\TPS} are as follows:
\begin{description}
\item[{\tt DIY}]	 Try to prove a theorem.

\item[{\tt BASIC-DIY}]	 Try to prove a theorem without using special rules (like RULEP).

\item[{\tt KILL}]	 Kill a {\TPS} process.

\item[{\tt ADJUSTTIME}]	  Adjust the time remaining for a {\TPS} process.
\end{description}

The format for {\tt DIY} and {\tt BASIC-DIY} requests are as follows:
\begin{alltt}
(<TPSname> [DIY|BASIC-DIY] <procname> <servcomms> <proofoutline> <TPSmode> <DEFAULT-MS> <timeout>)
\end{alltt}
The name {\tt <procname>} is a string the client chooses to identify this particular request.
Assume {\tt <servcomms>} is NIL (in general it could be a list of commands to send to MathWeb).
The {\tt <proofoutline>} is in the form of a `defsavedproof'.  In general you can
get such a form by obtaining the proof outline in {\TPS}, performing \indexcommand{saveproof}
into a file {\tt foo.prf} and then examining the file {\tt foo.prf}.
A simple example is
\begin{alltt}
 & (defsavedproof FOTR
 &  (4 2 29)
 &  (assertion `TRUTH')
 &  (nextplan-no 2)
 &  (plans ((100)))
 &  (lines (100 NIL `TRUTH' PLAN1 () NIL))
 &  0 () (comment `') (locked NIL))
\end{alltt}
which represents the proof outline
\begin{alltt}

               ...
(100) \( \assert \truth \) & PLAN1
\end{alltt}
The value of {\tt <TPSmode>} should be the name of a mode in {\TPS}.
The value of {\tt <DEFAULT-MS>} can be used to override the value of
the flag \indexflag{DEFAULT-MS} in the mode {\tt <TPSmode>}.  If {\tt <DEFAULT-MS>}
is NIL, then the value of the flag \indexflag{DEFAULT-MS} is set by {\tt <TPSmode>}.
{\tt <timeout>} is the number of seconds {\TPS} should try to search for a proof
before giving up.

After the client sends a {\tt DIY} or {\tt BASIC-DIY} request with name {\tt <procname>},
the client can later kill the request or allow the request more time to succeed.
The {\tt KILL} request has the format
\begin{alltt}
(<TPSname> KILL <procname>)
\end{alltt}
The {\tt ADJUSTTIME} request has the format
\begin{alltt}
(<TPSname> ADJUSTTIME <procname> <seconds>)
\end{alltt}
This {\tt ADJUSTTIME} request will add the value {\tt <seconds>}
to the time remaining for the process {\tt <procname>}.  (Note that
{\tt <seconds>} may be negative.)

When a {\tt DIY} or {\tt BASIC-DIY} request has finished (either due to success, failure or timeout),
then the message returned via the {\tt inout} socket will be
\begin{alltt}
(<procname> <proof> <printedproof>)
\end{alltt}
where {\tt <proof>} is the proof in the `defsavedproof' format (with no remaining planned lines if the request succeeded)
and {\tt <printedproof>} is the result of the {\TPS} command \indexcommand{PALL}.

\subsection{Example}

Consider a quick example where the client is running under Allegro Lisp
on the host jtps.math.cmu.edu.

Client:
\begin{alltt}
>(setq clientio (acl-socket:make-socket :connect :passive))
\#<MULTIVALENT stream socket waiting for connection at */34032 \ \#x722d43aa>
>(setq clientserv (acl-socket:make-socket :connect :passive))
\#<MULTIVALENT stream socket waiting for connection at */34033 \ \#x722d58ea>
\end{alltt}

Start {\TPS} on jtps.math.cmu.edu:
\begin{alltt}
tps -service jtps.math.cmu.edu 34032 34033
\end{alltt}

Client:
\begin{alltt}
>(setq inout (acl-socket:accept-connection clientio))
\#<MULTIVALENT stream socket connected from jtps.math.cmu.edu/34032 to
  jtps.math.cmu.edu/34034 \ \#x722d820a>
>(setq serv (acl-socket:accept-connection clientserv))
\#<MULTIVALENT stream socket connected from jtps.math.cmu.edu/34033 to
  jtps.math.cmu.edu/34035 \ \#x7231f6ba>
> (defun send-info (s)
    (format inout `\(\sim\)S' s)
    (write-char \#\(\setminus\)%null inout)
    (force-output inout))
SEND-INFO
> (defun read-msg ()
     (let ((ret `'))
       (do ((z (read-char inout nil nil) (read-char inout nil nil)))
           ((eq z \#\(\setminus\)%null) ret)
         (setq ret (format nil `\(\sim\)d\(\sim\)d' ret z)))))
READ-MSG
> (send-info '(PING CLIENTNAME))
NIL
> (setq rets (read-msg))
`(CLIENTNAME PONG |TPSjjtps.math.cmu.edu-3287332390|)
'
> (setq ret (read-from-string rets))
(CLIENTNAME PONG |TPSjjtps.math.cmu.edu-3287332390|)
> (setq tpsname (caddr ret))
|TPSjjtps.math.cmu.edu-3287332390|
> (send-info (list tpsname 'DIY `TRUEPROCNAME' nil '(defsavedproof FOTR
  (4 3 3)
  (assertion `TRUTH')
  (nextplan-no 2)
  (plans ((100)))
  (lines
    (100 NIL `TRUTH' PLAN1 () NIL)
  ) 0
  ( )
    (comment `')
    (locked NIL)
    NIL
  )
  'MS98-HO-MODE NIL 5))
NIL
> (setq rets (read-msg))
``(\"TRUEPROCNAME\(\setminus\)'' \(\setminus\)''(defsavedproof FOTR
  (4 3 3)
  (assertion \(\setminus\setminus\setminus\)''TRUTH\(\setminus\setminus\setminus\)'')
  (nextplan-no 3)
  (plans NIL)
  (lines
    (1 NIL \(\setminus\setminus\setminus\)''TRUTH\(\setminus\setminus\setminus\)'' \(\setminus\setminus\setminus\)''Truth\(\setminus\setminus\setminus\)'' () NIL)
) 0
( )
  (comment \(\setminus\setminus\setminus\)''\(\setminus\setminus\setminus\)'')
  (locked NIL)
  NIL
)
\(\setminus\)'' \(\setminus\)''
(1)   !  TRUTH                                               Truth\(\setminus\)'')
'
> (setq ret (read-from-string rets))
(`TRUEPROCNAME' `(defsavedproof FOTR
  (4 3 3)
  (assertion \(\setminus\)''TRUTH\(\setminus\)'')
  (nextplan-no 3)
  (plans NIL)
  (lines
    (1 NIL \(\setminus\)"TRUTH\" \(\setminus\)"Truth\(\setminus\)" () NIL)
) 0
( )
  (comment \(\setminus\)"\(\setminus\)")
  (locked NIL)
  NIL
)
' `
(1)   !  TRUTH                                               Truth')
> (send-info (list tpsname 'DIY `FALSEPROCNAME' nil '(defsavedproof FOFA
    (4 3 3)
    (assertion `FALSEHOOD')
    (nextplan-no 2)
    (plans ((100)))
    (lines
      (100 NIL `FALSEHOOD' PLAN1 () NIL)
    ) 0
  ( )
    (comment `` ``)
    (locked NIL)
    NIL
  )
  'MS98-HO-MODE NIL 5))
NIL
> (setq rets (read-msg))
``(\(\setminus\)"FALSEPROCNAME\(\setminus\)" \(\setminus\)"(defsavedproof FOFA
  (4 3 3)
  (assertion \(\setminus\setminus\setminus\)''FALSEHOOD\(\setminus\setminus\setminus\)'')
  (nextplan-no 2)
  (plans ((100)))
  (lines
    (100 NIL \(\setminus\setminus\setminus\)''FALSEHOOD\(\setminus\setminus\setminus\)'' PLAN1 () NIL)
) 0
( )
  (comment \(\setminus\setminus\setminus\)''\(\setminus\setminus\setminus\)'')
  (locked NIL)
  NIL
)
\(\setminus\)" \(\setminus\)"
               ...
(100) !  FALSEHOOD                                            PLAN1\(\setminus\)")
"
> (setq ret (read-from-string rets))
("FALSEPROCNAME" "(defsavedproof FOFA
  (4 3 3)
  (assertion \(\setminus\)"FALSEHOOD\(\setminus\)")
  (nextplan-no 2)
  (plans ((100)))
  (lines
    (100 NIL \(\setminus\)"FALSEHOOD\(\setminus\)" PLAN1 () NIL)
) 0
( )
  (comment \(\setminus\)"\(\setminus\)")
  (locked NIL)
  NIL
)
" "
               ...
(100) !  FALSEHOOD                                            PLAN1")
\end{alltt}

\section{Starting {\TPS} as an Online Server}\label{tps-server}\index{server}\index{web server}

Under Allegro Lisp 5.0 or greater,
{\TPS}{and {\ETPS}} can be started as a web server for use online.
Once everything is set up and the server is started,
remote users will be able to access {\TPS}
via a browser (possibly using
a user id and password) and communicate with {\TPS} via
a Java interface.

\subsection{Setting up the Online Server}

The following steps are necessary to set up the {\TPS} server
in a Unix or Linux operating system.  Analogous steps
are necessary for setting up the {\TPS} server for MS Windows.

\begin{enumerate}
\item We start by assuming {\ETPS}, {\TPS} and the Java files
have already been compiled.

\item Create a directory for the server, e.g.
{\tt /home/theorem/tpsonline}
Move to this directory.

\item To set up the user id's and passwords, start {\TPS}
in the new directory.

\item The id and password information will be saved in the file named by
the flag \indexflag{USER-PASSWD-FILE}.  The default value
is \indexother{user-passwd}.  It is recommended that
you not change the value of this flag from its default.

\item From within {\TPS}, run the command \indexcommand{SETUP-ONLINE-ACCESS}.
\begin{alltt}
<1>setup-online-access
\end{alltt}
The command \indexcommand{SETUP-ONLINE-ACCESS} will ask you for
a series of user id's and passwords for remote access
to {\TPS} and {\ETPS}.  It will also ask if you
wish to allow anonymous users to be able to remotely run {\TPS} or {\ETPS}.
The user id's and passwords
are unrelated to the user id's and passwords created and used by the operating
system.  The following is an example where two students are given
id's and passwords for {\ETPS} and anonymous users are allowed to remotely
run {\TPS}.
\begin{alltt}
<0>setup-online-access
Allow ETPS Anonymous Access To Everyone? [No]>no
Add a userid?  [Yes]>

User Id  [`']>`student1'
Password  [`']>`password1'
Added user student1
Add another userid?  [Yes]>

User Id  [`']>`student2'
Password  [`']>`password2'
Added user student2
Add another userid?  [Yes]>n
Allow TPS Anonymous Access To Everyone? [No]>y
Although anyone can run TPS
You may still wish to add specific users which will be allowed to save files
in a directory.
Add a userid?  [Yes]>n
\end{alltt}
A file named \indexfile{user-passwd} (or, in general,
a name given by the value of \indexflag{USER-PASSWD-FILE})
should have been created.

\item Exit {\TPS}.

\item Copy or link the java directory (with the compiled Java class files)
into the tpsonline directory.
\end{enumerate}

\subsection{Starting or Restarting the Online Server}

To start the {\TPS} server, make sure you are in the directory
with the user-password file.
You may be able to start the {\TPS} server by using
the shell script \indexfile{run-tpsserver}, which is included
in the distribution.
In general, start {\TPS}{or {\ETPS}} as a server using the following
pattern:
\begin{alltt}
<lisp> -I <tpsdir>/tps3.dxl -- -server <tpsdir>/tps3.dxl <tpsdir>/etps.dxl
\end{alltt}
<lisp> should be the name of the lisp executable (e.g., lisp or alisp8).
<tpsdir> should be the directory where the {\TPS} and {\ETPS} image files are located.
If the server starts successfully, a directory named `logs' for log files
should be created.  One can explicitly give a different name for the log directory
using the -logdir command line switch as follows:
\begin{alltt}
<lisp> -I <tpsdir>/tps3.dxl -- -server <tpsdir>/tps3.dxl <tpsdir>/etps.dxl -logdir <logdirname>
\end{alltt}

Once the server is started, it can be accessed on the web using the URL
`http://<machine-name>:29090/'.  The number `29090' is the default
port number used by the {\TPS} server.  If this port number is not free,
then the {\TPS} server will fail to start.  In this case, you can
explicitly provide a different port number using the -port command line
switch as follows:
\begin{alltt}
<lisp> -I <tpsdir>/tps3.dxl -- -server <tpsdir>/tps3.dxl <tpsdir>/etps.dxl -port <portnum>
\end{alltt}
In this case, the {\TPS} web server can be accessed via a browser using
the URL `http://<machine-name>:<portnum>/'.  If you wish to link to
the web server from an HTML file on another web site, use an href anchor
as follows:
\begin{alltt}
<A HREF=`http://<matchine-name>:<portnum>/'>Click here to run TPS or ETPS online</A>
\end{alltt}

When a remote user accesses {\TPS} or {\ETPS} online via the {\TPS} web server,
a directory for that user is created (or a directory named `anonymous' if it is being
run without a user id and password).  Files may be saved by the remote user
in this directory.

It should be noted that an anonymous remote user is allowed to do less.
For example, an anonymous user cannot start {\TPS} or {\ETPS} using a command
line prompt.  Instead, they are forced to rely on menus to execute allowed
commands and popup prompts to enter other information.  This prevents
an anonymous user from executing arbitrary commands.

One possible use of the {\TPS} server is to allow students in a class
to use {\ETPS} to complete class assignments without having
{\ETPS} installed on their computer.  This is discussed further in
section \ref{etps-class-server}.

\section{Preparing ETPS for classroom use}

Building {\ETPS} is just like building {\TPS}, except that one should type {\tt make etps}
rather than {\tt make tps}.  Before calling {\tt make etps}
make sure the bin directory is empty because the compiled files for
tps won't work right for etps.
The modules of {\ETPS} are just a subset of those for {\TPS}.

If you wish to use {\ETPS} for a class, there are some things you might
want to change.

To determine for which exercises students may use
\indexcommand{ADVICE} and commands such as {\tt RULEP}, set the
\indexother{allowed-cmd-p} attributes appropriately. For example, in
the definitions in \indexfile{ml2-theorems.lisp}, we find the \indexother{allowed-cmd-p}
attributes set to \indexother{ALLOW-ALL}, so for these exercises the
students may use all the facilities of {\ETPS}. On the other hand, in
\indexfile{ml1-theorems.lisp} they are set to \indexother{ALLOW-RULEP}, which
allows everything except advice.

If you desire different inference rules or exercises, see chapter \ref{rules} for
tips on defining and compiling new ones.  Examine the {\it .rules} files
which have been used to define the current rules. Then put your new files into
a new {\TPS} package, and load that package when building or compiling
{\ETPS}.

\subsection{Starting ETPS as an Online Server for a Class}\label{etps-class-server}

Following the instructions in section \ref{tps-server}, a teacher
can start {\TPS} as a web server.  Using the command \indexcommand{SETUP-ONLINE-ACCESS}
as described in section \ref{tps-server}, the teacher can enter
a list of student user id's and passwords.  This allows students to
log in through the {\TPS} web server and start {\ETPS}.  This will create
a directory on the server machine for this student.  Files relevant to
recording scores for this student are saved in this directory.

\subsection{Grades}


When a student completes an
exercise and executes the \indexcommand{DONE} command,
a message recording that fact can be appended to the end of a
file to which students have write access.
The path and name of this file
is given by the value of the flag \indexflag{score-file}.
The flag \indexflag{score-file} should be set in
the common initialization file \indexfile{etps.ini}
which is loaded by every user of {\ETPS}.
The distributed \indexfile{etps.ini} file contains the
following line:
\begin{alltt}
;*; (setq score-file `/afs/andrew/mcs/math/etps/records/etps-spring03.scores')
\end{alltt}
If you want \indexfile{etps.ini} to set \indexflag{score-file},
then you should remove ;*;
from the beginning of this line to uncomment it.
This (if uncommented) will set the flag \indexflag{score-file}
to the same value for every user of {\ETPS}.
Appropriate adjustments in the pathname should be made.

If you prefer to have students with different userid's
to have different score files, you can use
the following option instead.
The distributed \indexfile{etps.ini} file also contains these lines:
\begin{alltt}
;*; (setq score-file (concatenate 'string
;*;     `/afs/andrew/mcs/math/etps/records/' (string (status-userid))
;*;     `/etps-spring03.scores'))
\end{alltt}
If the user's userid is pa01, when this is read {\ETPS}
will set the flag \indexflag{score-file} to
\begin{alltt}
`/afs/andrew/mcs/math/etps/records/pa01/etps-spring04.scores'
\end{alltt}
If you use this option, you will need to use a utility
to combine the score files for the different students in
the class.

The {\tt GRADER} program (for which there is a separate manual)
can be used to process the
grades in a file which is the value of the
{\tt GRADER} flag \indexflag{etps-file}.
This should be the same as the value of
the {\ETPS} flag \indexflag{score-file}
if all the students write to the same file.
Otherwise, it can be a file into which all the
students' files have been collected.
The sample line in the \indexfile{grader.ini}
file setting \indexflag{etps-file} should be edited
by changing the pathname appropriately.

\subsection{Security}

Note: On a Unix system, you can use {\ETPS} as a setuid program to allow
students to write to their score files, i.e., so that any process
running {\ETPS} has full access to the files, while other processes do
not.  However, this may be an excessive precaution, since each message
issued by the {\ETPS} \indexcommand{DONE} command has a special
encryption number used to ensure security.  Thus a student cannot edit
the score file to make it appear that he or she has completed an
exercise.  The routines in the {\tt GRADER} package check the encryption
number to make sure the information in that line of the score file is
valid.

Checksums are generated for all saved proofs
when the \indexflag{EXPERTFLAG} flag is set to NIL, but not when it is set to T.
This means that students should be unable to manually edit a saved partial proof
and fool {\ETPS} into thinking that it's complete; it also means that proofs
saved in {\TPS} with \indexflag{EXPERTFLAG} T cannot be reloaded into {\ETPS}
with \indexflag{EXPERTFLAG} NIL.


\subsection{Diagnosing Problems in ETPS}

{\ETPS} catches errors so that when there are problems one does
not get an error message, and is not thrown to the debugger.
To change this, run {\ETPS} as a user on the expertlist (which is in the Makefile),
set \indexflag{EXPERTFLAG} to T, and set the variable \indexother{*trap-errors*} to
nil as follows:
\begin{alltt}
(setq *trap-errors* nil)
\end{alltt}

\section{Interruptions and Emergencies}

This section consists mostly of implementation-dependent information,
although some of the following will work in most situations.
The following control characters will work in most circumstances:

\begin{description}
\item[] {\tt \^G <Return>} (i.e. type one followed by the other) will abort the current process.

\item[] {\tt <Return>} will stop a mating search and drop into the mate top level. When you
leave the mate top level, the mating search will attempt to continue (of course,
if you've made drastic changes, it may fail).

\item[] {\tt \^Z} will suspend lisp; you can then kill the job if necessary, or put it into
the background with {\tt bg}

\item[] {\tt \^C} will interrupt and throw you into the lisp break package
\end{description}

You can save a core image with the \indexcommand{TPS3-SAVE} command, as follows:

\begin{tpsexample}
<24>setflag save-file `mycore.exe'
<25>tps3-save
\end{tpsexample}

You should also save the flag settings, since when you restart {\TPS} with
this core image it will re-read the \indexfile{tps3.ini} file and may reset some flags.

In Allegro Common lisp, if you get the {\tt <cl>} prompt, the following are
some of the possible responses:

\begin{description}
\item[] {\tt :help} to see all the options

\item[] {\tt :cont} to attempt to continue

\item[] {\tt :out} to get back to top level

\item[] {\tt :res} to get back to top level

\item[] {\tt (exit)} to kill {\TPS}
\end{description}

In CMU common lisp, if you get the {\tt 0]} debugger prompt, the command
{\tt q} will get you back to the top level, and the command {\tt h} will
list all the other options available.

If {\TPS} crashes, or you discover a bug, use the \indexcommand{BUG-SAVE} command
to save the current state. Give your bug a name, and describe it (possibly
cut and paste the error message that was produced into the `comments' field).
This will save all the flag settings, the timing information, the history and
the current proof, in such a way that one can use \indexcommand{BUG-RESTORE}
to return to the same error at another time. Bugs are, by default, saved to
\indexflag{DEFAULT-BUG-DIR}, although they can be saved to \indexflag{DEFAULT-LIB-DIR}
by setting flag \indexflag{USE-DEFAULT-BUG-DIR} to NIL. There are also commands to list,
delete and examine the comments field of bugs (\indexcommand{BUG-LIST}, \indexcommand{BUG-DELETE}
and \indexcommand{BUG-HELP}, respectively); these correspond to library commands \indexcommand{LIST-OF-LIBOBJECTS},
\indexcommand{DELETE} and \indexcommand{SHOW-HELP}.

\section{How to produce manuals}
At the present time, to produce printed manuals, you must either have the
Scribe text-processing system and a Postscript printer or the \LaTeX ~system.


\subsection{Scribe manuals}
Enter the directory which corresponds to the manual you wish to make,
 then run Scribe on the file
{\tt manual}.  For example, if you wish to make the manual for {\ETPS}, do

\begin{alltt}
\% cd doc/etps
\% scribe scribe-manual
\end{alltt}

If you are using {\ETPS} as part of a course, you may wish to modify
the files in that directory to tailor it toward the inference rules
of your system.

To produce the facilities guide, which lists all commands, flags, modes,
etc., you can use the {\TPS} command \indexcommand{SCRIBE-DOC}.
% \comment{or \indexcommand{SCRIBE-CATS}.}
However, it may be easiest to use the
files in the directory {\it whatever/tps/doc/facilities}
(i.e., something like {\it /usr/tps/doc/facilities}). To do this,
you will use the file \indexfile{scribe-facilities.lisp} for a long and pretty comprehensive
manual, or \indexfile{scribe-facilities-short.lisp} for a shorter version which excludes some of the
obscurer TPS objects.
Use \indexfile{scribe-facilities-cmd.lisp} for the shortest manual of all, which contains only commands
and flags (i.e. the short facilities guide
without information on tactics, tacticals, binders, abbreviations, types,
subjects, modes, events, styles, grader commands etc). It also prints with
narrower page margins. To produce this manual, replace `-short'
with `-cmd' in the following. At the time of writing, manual-cmd.mss ran to 90 pages,
manual-short.mss was 156 pages, and manual.mss was 246 pages.

If you want a very short manual containing just a little information,
(such as a summary of a new search procedure) use scribe-facilities-temp. Edit
facilities-temp.lisp to contain just the categories you wish, and
FLAG. In TPS
tload `whatever/tps/doc/facilities/scribe-facilities-temp.lisp'
Then edit whatever/tps/doc/facilities/scribe-facilities-temp.mss
to eliminate all the flags you do not want in this manual, and
scribe the file.

Part of the lisp function in the file specifies the output file; EDIT
THAT PATHNAME to put the file into the facilities directory.
Then proceed as follows (to make the short manual)
\begin{alltt}
\% tps3
<2>tload `whatever/tps/doc/facilities/scribe-facilities-short.lisp'
Written file whatever/tps/doc/facilities/scribe-facilities-short.mss
T
<3>exit
\% cd whatever/tps/doc/facilities
\% scribe scribe-manual-short
\end{alltt}

If you were making the full manual, use the files \indexfile{scribe-facilities.lisp},
\indexfile{scribe-facilities.mss}, and \indexfile{scribe-manual.mss} in place of \indexfile{scribe-facilities-short.lisp},
\indexfile{scribe-facilities-short.mss}, and \indexfile{scribe-manual-short.mss}, respectively. Similarly,
for the very short manual, use \indexfile{scribe-facilities-cmd.lisp},
\indexfile{scribe-facilities-cmd.mss}, and \indexfile{scribe-manual-cmd.mss}.

Note: If you use a {\TPS} core image into which you have already loaded
certain wffs from your library, these will show up in the facilities
guide.

This information is also in the file \indexfile{doc/facilities/README}.

\subsection{\LaTeX ~manuals}

Similarly to the Scribe manuals, enter the directory which corresponds to the manual you wish to make,
 then run \LaTeX on the file
{\tt manual}.  For example, if you wish to make the manual for {\ETPS}, do

\begin{alltt}
\% cd doc/etps
\% latex latex-manual
\end{alltt}

You may have to compile it several times (up to three) in order to get the cross-references and index right. You will also need to have the \indexfile{tps.tex} and \indexfile{tpsdoc.tex} files containing the \LaTeX macros used in these manuals.

If you are using {\ETPS} as part of a course, you may wish to modify
the files in that directory to tailor it toward the inference rules
of your system.

To produce the facilities guide, which lists all commands, flags, modes,
etc., you can use the {\TPS} command \indexcommand{LATEX-DOC}. Note that this command generates the content of the manual but still need to be compiled with the \indexfile{latex-manual.tex} file which contains the title page, the preamble and the style directives.
% \comment{or \indexcommand{LATEX-CATS}.}
However, it may be easiest to use the
files in the directory {\it whatever/tps/doc/facilities}
(i.e., something like {\it /usr/tps/doc/facilities}). To do this,
you will use the file \indexfile{latex-facilities.lisp} for a long and pretty comprehensive
manual, or \indexfile{latex-facilities-short.lisp} for a shorter version which excludes some of the
obscurer TPS objects.
Use \indexfile{llatex-facilities-cmd.lisp} for the shortest manual of all, which contains only commands
and flags (i.e. the short facilities guide
without information on tactics, tacticals, binders, abbreviations, types,
subjects, modes, events, styles, grader commands etc). It also prints with
narrower page margins. To produce this manual, replace `-short'
with `-cmd' in the following. 

If you want a very short manual containing just a little information,
(such as a summary of a new search procedure) use facilities-temp. Edit
latex-facilities-temp.lisp to contain just the categories you wish, and
FLAG. In TPS
tload `whatever/tps/doc/facilities/latex-facilities-temp.lisp'
Then edit whatever/tps/doc/facilities/facilities-temp.tex
to eliminate all the flags you do not want in this manual, and
scribe the file.

Part of the lisp function in the file specifies the output file; EDIT
THAT PATHNAME to put the file into the facilities directory.
Then proceed as follows (to make the short manual)
\begin{alltt}
\% tps3
<2>tload `whatever/tps/doc/facilities/latex-facilities-short.lisp'
Written file whatever/tps/doc/facilities/latex-facilities-short.tex
T
<3>exit
\% cd whatever/tps/doc/facilities
\% scribe manual-short
\end{alltt}

If you were making the full manual, use the files \indexfile{latex-facilities.lisp},
\indexfile{latex-facilities.tex}, and \indexfile{latex-manual.tex} in place of \indexfile{latex-facilities-short.lisp},
\indexfile{latex-facilities-short.tex}, and \indexfile{latex-manual-short.tex}, respectively. Similarly,
for the very short manual, use \indexfile{latex-facilities-cmd.lisp},
\indexfile{latex-facilities-cmd.tex}, and \indexfile{latex-manual-cmd.tex}.

Note: If you use a {\TPS} core image into which you have already loaded
certain wffs from your library, these will show up in the facilities
guide.

This information is also in the file \indexfile{doc/facilities/README}.


\subsection{HTML manuals}

The information in the facilities guide can also be output in a rudimentary HTML format
by using the command \indexcommand{HTML-DOC}. You will need to provide TPS with the name
of an empty directory which has about 10MB of free space; the main page of the manual
will be the file \indexfile{index.html} in that directory.

Also, HTML documentation for {\ETPS} can be generated by using the command
\indexcommand{HTML-DOC} from within {\ETPS}.

\section{Miscellaneous Information}

The common \indexfile{tps3.ini} and \indexfile{etps.ini} files are used to set the default values
of some flags for a particular site.
The setting of LATEX-PREAMBLE
refers to input files tps.sty, tps.tex and vpd.tex, which are part of the {\TPS} distribution.
This is currently set using the value of sys-dir.
You could change the settings of these flags in  the ini files, but you probably won't need to.
The following commands from LATEX-PREAMBLE should not be changed, since
TPS relies on them:
\begin{tpsexample}
\(\setminus\setminus\)def\(\setminus\setminus\)endf\{\(\setminus\setminus\)end\{document\}\}
\(\setminus\setminus\)newcommand\{\(\setminus\setminus\)markhack\}[1]\{\(\setminus\setminus\)vspace*\{-0.6in\}\{\#1\}\(\setminus\setminus\)vspace*\{0.35in\}\(\setminus\setminus\)markright\{\{\#1\}\}\}
\end{tpsexample}


If you have access to the Scribe text-processing system,
you can change the documentation for {\ETPS} as appropriate (found in
the directory {\it doc/etps} and give it to students.  Note that in
its present form there are some CMU-specific assumptions made,
and it contains a listing of the inference rules as used in classes here.

You might want to employ a different scheme for grading.  The file
\indexfile{etps-events.lisp} defines how the results of each exercise are
output to the score file.  See Chapter \ref{events} for information
on how to define events.


%\begin{comment}
%\end{comment}

\addtocontents{toc}{\contentsline {chapter}{\numberline {}Bibliography}{\arabic{page}}}

\bibliography{logictex}
\bibliographystyle{plain-casefix}

\newpage
\addtocontents{toc}{\contentsline {chapter}{\numberline {}Index}{\arabic{page}}}
\printindex  

\end{document}
