\section{Flavors and Labels of Gwffs}
\label{defflavors}
\label{labels}

It is sometimes desirable to be able to endow a gwff with additional
properties.  For example, one may wish to be able to refer to a
gwff by a short tag, or to specify that a particular gwff is actually
a node in an expansion tree.  For this purpose, \TPS provides the
facility of {\it labels} and {\it flavors} (see page \pageref{flavors}).  
A {\it label} is an object which, as far as \TPS is concerned, is merely
a special case of gwff. Labels thus stand for gwffs, but may have
additional properties and distinct representations.

{\it Flavors} are the classes into which labels are divided.  The definition
of a flavor specifies some common properties of a class
of labels, usually the behavior of wffops and predicates.  Also, a 
flavor's definition should specify what attributes each label of that 
flavor should have, and how it should be printed.

\subsection{Representation}

Each flavor is represented in \TPS by a Lisp structure of type {\tt flavor}, 
which has the following slots: {\tt wffop-hash-table}, which stores the
properties common to each instance of the flavor, in particular, how
wffops are to behave; {\tt constructor-fun}, which
is the name of the function to be called when a new label of the flavor
is to be created; {\tt name}, the flavor's name; and {\tt mhelp}, a description
of the flavor.  The values of these slots are automatically computed when
\TPS reads a {\tt defflavor} declaration.
The flavor structures are stored in a central hash table, called
{\tt *flavor-hash-table*}, keyed on the flavor names.  This also is 
updated automatically whenever a flavor is defined (or redefined).

There are two ways to represent labels (instances of flavors), and the
choice is made during the definition of the flavor.  The first, and more
traditional, way is to have each label be a Lisp symbol, with the attributes
of the label being kept on the symbol's property list.  
The second way is to make each label a Lisp structure.  The type of the
structure is the name of the flavor; thus an object's type can be used to
determine that it is a label of a certain flavor.

If one wishes to have labels be symbols, nothing must be done; this is the
default.  A flavor's labels will be structures only if one of two things
is declared in the {\tt defflavor}.  The first is that the property
{\tt structured} appears.  The second is if another flavor whose instances are
structures is specified to be {\it included} in the new flavor.

When a flavor's labels are to be structures, one will usually wish to
specify the {\tt printfn} property so that the labels will be printed
in a nice way.  This function must be one which is acceptable for use
in a {\tt defstruct}.  It is also required that one specify the slots, or
attributes, the
structures are to have, by including a list of the form
\begin{verbatim}
(instance-attributes (slot1 default1) ... (slotN defaultN))
\end{verbatim}
in the flavor definition.

\subsection{Using Labels}

The function {\tt define-label} is a generic way to create new labels of a
specified flavor.  The function call {\tt (define-label sym flavor-name)}
will do one of two things.  If {\tt flavor-name} is a flavor  whose
labels are symbols, then the property list of {\tt sym} will be updated
with property {\tt FLAVOR} having value {\tt flavor-name}.  If on the 
other hand, {\tt flavor-name} is a flavor having structures for labels,
then {\tt sym} will be setq'd to the value of the result of calling
the constructor function for {\tt flavor-name}, which will create
a structure of type {\tt flavor-name}.

To access the attributes of a label which is a symbol,  use {\tt get},
since all attributes will be on the symbol's property list.  
The attributes of a label which is a structure of type {\tt flavor-name}
can be accessed by using the standard Lisp accessing functions for
structures.  Thus, if one of the label's attributes is {\tt represents},
the attribute can be accessed by calling the function 
{\tt flavor-name-represents}. 

Flavors can be redefined or modified at any time.  This may be done if,
for example, one wished to extend a flavor's definition into a Lisp
package which was not always loaded.  Merely put another {\tt defflavor}
statement into the code.  You need only put the new or changed 
properties in the redefinition.  If, however, you wish to change the
attributes of a flavor which is a structure, you should put in all
of the attributes you desire, not just the new ones, and be sure
to declare any included flavor as well.  Note: it is possible to
change a flavor which uses symbols as labels into one which uses
structures, but if you fail to redefine code which depends on 
property lists, the program will be smashed to flinders.


\subsection{Inheritance and Subflavors}

Some flavors may be similar in many ways; in fact, some flavors may be
more specialized versions of other flavors.  One may wish a new flavor's
labels to be operated upon by most wffops in the same way as an existing
flavor's labels; this we will call inheritance of properties.  In addition,
one may wish a new flavor to actually be a subtype (in Lisp terms)
of an existing flavor, and have the attributes of the existing flavor's 
labels be included in the attributes of the new flavor's labels; this we
will call inclusion of attributes.   The {\tt defflavor} form allows either
or both types of sharing to be used.

Inheritance of properties is signalled in the {\tt defflavor} by a form such
as {\tt (inherit-properties {\it existing-flavor1} ... {\it existing-flavorN})}.
This will cause the properties in the {\tt wffop-hash-table} of the 
existing flavors to be placed into the {\tt wffop-hash-table} of the new
flavor. If any conflict of properties occurs, e.g., if {\it existing-flavorI}
and  {\it existing-flavorJ}, I < J, have a property with the same name, then
the value which {\it existing-flavorJ} has for that property will be 
the one inherited by the new flavor. A new flavor may inherit properties 
from any number of existing flavors.

In contrast, attributes may be included from only one other flavor.  This
can be done by using the form {\tt (include {\it existing-flavor})}. The 
existing flavor must be a flavor whose instances are structures, and the
new flavor's instances will also be structures whose slots include the
attributes of the existing flavor.  Thus the same accessing functions for
those slots will work on labels of both flavors.  To define default
values for those slots, add them to the {\tt include} form as if it were
an {\tt :include} specifier to a {\tt defstruct}; e.g., 
{\tt (include {\it existing-flavor} ({\it slot1 default1}))}.

\subsection{Examples}

Here are some examples of flavor definitions.

%\begin{lispcode}
\begin{verbatim}
(defflavor etree
  (mhelp "Defines common properties of expansion tree nodes.")
  (structured t)
  (instance-attributes
   (name '|| :type symbol)
   components				; a node's children
   (positive nil )			; true if node is positive in the
					;formula
   (junctive nil :type symbol)		; whether node acts as neutral,
					;conjunction, or disjunction
   free-vars				; expansion variables in whose scope
					; node occurs,
					; used for skolemizing
   parent				; parent of the node
   ;;to keep track of nodes from which this node originated when copying a
   ;;subtree
   (predecessor nil)
   (status 1))
   (printfn print-etree)
   (printwff (lambda (wff bracket depth)
	       (if print-nodenames (pp-symbol-space (etree-name wff))
		 (printwff 
		  (if print-deep (get-deep wff)
		    (get-shallow wff))
		  bracket depth))))
   ...many more properties...)
\end{verbatim}
%\end{lispcode}

{\tt \indexData{Etree}} labels will be structures, with several attributes.
The function used to print them will be {\tt print-etree}.

%\begin{lispcode}
\begin{verbatim}
(defflavor leaf
  (mhelp "A leaf label stands for a leaf node of an etree.")
  (inherit-properties etree)
  (instance-attributes
   shallow)
  (include etree (name (intern-str (create-namestring leaf-name))))))
\end{verbatim}
%\end{lispcode}

{\tt \indexData{Leaf}} labels will also be structures, with attributes including
those of {\tt etree}, as well as a new one called {\tt shallow}.  Note that
the {\tt name} attribute is given a default in the {\tt include} form.
{\tt Leaf} inherits all of the properties of {\tt etree}, including, for
example, its print function, unless they are explicitly redefined in
the definition of {\tt leaf}.
