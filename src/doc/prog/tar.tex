\subsubsection{Making a tar file for Distribution} 

You can execute
{\tt /afs/andrew/mcs/math/TPS/admin/tps-dist/make-tar.exe}
from {\tt /home/theorem/project/dist}
(The date is computed automatically.)
The tar file is placed in /home/ftp/pub.

\begin{verbatim}
			KEY STEPS:
     Login to gtps as root (so that you can write a file in /home/ftp/pub).
su
     (You may need to do a klog, e.g., 
klog pa01 -c andrew.cmu.edu
      or 
klog cebrown -c andrew.cmu.edu
         so you appropriate permissions.)
cd /home/theorem/project/dist
/afs/andrew/mcs/math/TPS/admin/tps-dist/make-tar.exe

    [This does
     tar cvhf /home/ftp/pub/tps3-date.tar .
     gzip /home/ftp/pub/tps3-date.tar
     ln -sf /home/ftp/pub/tps3-date.tar.gz /home/httpd/html/tps3.tar.gz
      ]
    (It's important that the tar file is not put into the same directory
      as is being tarred, or it will try to work on itself.)

cd /home/ftp/pub
move old tar file to the subdirectory old. Delete the older one.

\end{verbatim}



%% description of what make-tps.exe does.
% \subsubsection{Distribution of \TPS via ftp} \\
%
%To make a tar file for distribution via ftp:
%\begin{enumerate}
%
%\item Login to gtps.math.cmu.edu.
%Delete any old tar files in /home/ftp/pub, or move them to
%/home/ftp/pub/old.
%
%\item {\tt cd /home/theorem/project/dist}.
%
%\item Clean up /home/theorem/project/dist and its subdirectories by deleting
%each backup, postscript, or dvi file.
%% @comment[See /afs/cs/usr/andrews/clean.hlp if you want to automate this.]
%
%\item While in the directory /home/theorem/project/dist,
%execute the following command at the Unix prompt:
%{\tt tar cvhf /home/ftp/pub/tps3-date.tar .}
%(where {\tt date} is the current date).
%A list of file names should pass by on the screen.  Just watch until
%you get a new Unix prompt.
%
%\item {\tt cd /home/ftp/pub}.
%
%\item If you wish to check the tar file, enter
%{\tt tar tvf tps3-date.tar}.
%The same list of file names should pass by.  These are the names of
%the files which are now in the tar file.
%
%\item {\tt compress tps3-date.tar}. This will create the file
%tps3-date.tar.Z, which is ready for people to transfer using ftp.
%
%\end{enumerate}
%
%     To see what files are in the tar file do:
%      gunzip -c tps3-date.tar.gz | tar tvf -

\subsubsection{Distribution of \TPS via http}

There is a perl script {\tt /home/httpd/cgi-bin/tpsdist.pl}
which is used to distribute \TPS via the gtps web site.
This perl script displays the distribution agreement
and asks for information from the remote user (name, email, etc.).
Once this information is given, the perl script updates
the log file {\verb+/home/theorem/tps-dist-logs/tpsdist_log+}.
(For this to work, {\tt apache} should be the owner of this log file.)
Then, the perl script outputs instructions and a link to the tar file.


\subsubsection{Obsolete Information about Making tar tapes of \TPS}

To make a tar archive onto a big round mag tape (the kind that is seen
in science fiction movies of the sixties and seventies, always
spinning aimlessly, supposedly to suggest the immense computing power
of some behemoth machine):
\begin{enumerate}
\item Go to the CS operator's room on the third floor, at the end of the
Wean Hall 3600 hallway.  

\item Tell the operator that you wish to write a tar tape from the machine
K.GP.  Give her the tape and go back to the terminal room. 

\item Log in on the K.

\item At the Unix prompt, enter {\tt assign mt}.  This gives you control of
the mag tape units.  

\item Enter {\tt cd /home/theorem/project/dist}.  This puts you in the proper
directory.

\item Clean up /home/theorem/project/dist and its subdirectories by deleting
each backup, postscript, or dvi file.

\item Determine which device {\it devname} you wish to use.  This depends on the 
density which you wish to write.  For 6250 bpi, let {\it devname} be
{\tt /dev/rmt16}, for 1600 bpi, let {\it devname} be {\tt /dev/rmt8}, and
for 800 bpi, let {\it devname} be {\tt /dev/rmt0}.  Generally, you can go
for 6250 bpi unless the intended recipient has indicated otherwise.

\item Execute the following command at the Unix prompt:
{\tt tar rvhf {\it devname} .}

\item A list of file names should pass by on the screen.  Just watch until
you get a new Unix prompt.

\item To check the tape, enter
{\tt tar tvf {\it devname}}.
The same list of file names should pass by.  These are the names of
the files which are now on the tape.  If there were already files on
the tape, you will see all of them listed as well.

\item If all is well, call the operator and tell them that you are done with
the tape and that they can dismount it.  Then execute {\tt exit} at the
Unix prompt, to give up control of the tape drive, and log out as usual.

\item Go back to the operator's room and pick up the tape.

\end{enumerate}

To make a tar archive onto a Sun cartridge tape:
\begin{enumerate}
\item Take the tape to the CS operator, and ask her to put it on the machine
O.GP.  That is a Sun3 running Mach.  Go back to a terminal and log in on
any machine.  Again, you want to be in the directory /home/theorem/project/dist.

\item Now, you need to make sure that you can write to the O's tape drive.
You want to check the owner of the file /../o/dev/rst8, and make sure it's
{\tt rfsd}.  If not, call the CS operator and ask them to assign it to rfsd so
that you can make the tar tape.  

\item Now, execute the following command at the Unix prompt:
{\tt tar cvhf /../o/dev/rst8 .}

\item A list of file names should pass by on the screen.  It will be very slow.

\item After you get a new Unix prompt, wait a few seconds for the tape to rewind, 
then check it by entering
{\tt tar tvf /../o/dev/rst8}.
The same list of file names should pass by.  These are the names of
the files which are now on the tape.  

\item Go back to the operator and ask for the tape.

\end{enumerate}
