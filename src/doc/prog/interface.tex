\chapter{Running TPS With An Interface}

There is an interface for \TPS written in Java.
Running \TPS through such an interface is similar to running
\TPS within an xterm window, except the Java interface
supports menus and popup prompts.  The \TPS lisp code
now includes general facilities for communicating with
such an interface (when running under Allegro).

To start \TPS with the java interface, one can use the command line
argument 
\indexother{-javainterface} along with other relevant information
as shown below:
\begin{verbatim}
lisp -I tps3.dxl -- -javainterface cd javafiles \;
     /usr/bin/java TpsStart
\end{verbatim}
The command line arguments following -javainterface
should form a shell command which run the interface.
In this case, the shell command would be ``cd javafiles; /usr/bin/java TpsStart''.
Other command line arguments which have meaning for the ``java TpsStart'' command
are listed below.
\begin{description}
\item[\indexother{-big}]  Use the bigger sized fonts.
\item[\indexother{-x2}]  Multiply the font size by 2.
\item[\indexother{-x4}]  Multiply the font size by 4.
\item[\indexother{-nopopups}]  Do not use ``popup'' style prompts.
  Instead, the Java window should behave more like the x-window interface.
\end{description}
The remaining command line arguments should be followed by a non-negative integer.
\begin{description}
\item[\indexother{-screenx}]  The initial horizontal size of the Java window.
\item[\indexother{-screeny}]  The initial vertical size of the Java window.
\item[\indexother{-rightOffset}]  The amount of extra room given to the right margin.
\item[\indexother{-bottomOffset}]  The amount of extra room given to the bottom margin.
\item[\indexother{-maxChars}]  The maximum number of characters to hold in the buffer.
  This should be large enough that you can scroll back and see previous \TPS output.
  The default value of 20000 should usually be enough.
\end{description}

These other arguments should be preceeded by a command line ``-other''.
This tells TPS that the remaining command line information should be passed
to the call to ``java TpsStart''.  For example,
\begin{verbatim}
lisp -I tps3.dxl -- -javainterface cd javafiles \;
   /usr/bin/java TpsStart -other -big -rightOffset 10 -nopopups
\end{verbatim}
which tells
To send these command line arguments to ``java TpsStart'', they should
For example, ``java TpsStart -big'' instructs Java to use the big fonts,
and ``-x2'' and ``-x4'' instruct Java to multiply the size of the fonts
by 2 or 4, respectively.  The extra argument ``-nopopups'' will
provide an alternative to popup prompts.

Another way to use the java interface is to start \TPS as usual,
then use the command \indexcommand{JAVAWIN}.  This requires the
flag \indexflag{JAVA-COMM} to be set appropriately.

When \TPS is started in -javaservice mode, it uses the rest of the
command line arguments to start the Java interface and creates
sockets connecting \TPS to the Java interface.  Two processes
are spawned, one to receive input from the Java interface
(either from a prompt or from a menu item selection), and
another to actually run \TPS commands.

The rest of the description does not particularly depend on
the Java interface, so I will simply refer to ``the interface''
and attempt to emphasize that such an interface could be
implemented in a variety of ways.

The code to receive input from the interface is written
in \indexcommand{external-interface.lisp}.  It listens to the
socket stream and collects characters into strings
separated by null characters (ASCII 0).  
There are a few possibilities.
\begin{enumerate}
\item If the string ''COMMAND'' is received,
the next string is a command \TPS should run
(or the response to a prompt if popups are disabled).
All the input does with this command string is 
attach it to the \indexother{COMMAND} symbol
as the property \indexother{RESPONSE}.
The main process will accept this string as input
from linereadp since, when running through the interface,
a function \indexfunction{read-line-sym} will wait
for this \indexother{RESPONSE} property to be set.
An exceptional case is when the string after
``COMMAND'' is ``INTERRUPT''.  In this case,
the main process is killed and a new process
with a top level prompt is created.
\item If the string ``RIGHTMARGIN'' is received,
the next string received should be an integer giving
the new value for the flag \indexflag{RIGHTMARGIN}.
This allows the interface to change this flag
without having to interrupt another command that
may be running.
\item When popups are enabled, some other string starting with ``PROMPT''
is received.  In this case, the next string is
put on the \indexother{RESPONSE} property
of this ``PROMPT'' symbol.  This should be
a response to a particular (popup) prompt.
\end{enumerate}

So, the code in \indexfile{external-interface.lisp} handles
receiving input from the interface.  The other
problem is that of sending output to the interface.
This is handled by setting \indexother{*standard-output*}
to the socket stream and changing the \indexflag{STYLE}
to \indexother{ISTYLE} (``interface style'' defined in
\indexfile{interface-style.lisp}).  This style
is similar to the \indexother{XTERM} style, except with
more control information.  Control information is sent
by first sending a null character (ASCII 0) followed
by a byte giving information.  The current possible byte
values following a null character and their meanings
are listed below.  There are lisp functions in
\indexfile{interface-style.lisp} which send these bytes,
but anyone coding a new interface will need to know these values. 
\begin{enumerate}
\item [0] Switch to normal font mode.  In normal font mode,
  each character is communicated by a single byte.
\item [1] Switch to symbol font mode.  In symbol font mode,
  each symbol character is communicated by two bytes (allowing
  for many more symbol characters than normal font characters).
\item [2] Start a prompt message string.
\item [3] Start a prompt name string.
\item [4] Start a prompt argtyp string.  (This allows the interface to
recognize some special finite argtyp's such as boolean.)
\item [5] Start a list of prompt-options
\item [6] Start a list giving the prompt's default value.
\item [7] Start a prompt help string.
\item [8] End a prompt.
\item [9] A note that a command has finished executing.
\item [10] Start and end a string specifying the current top level.
\item [11] Open a window (eg, for proof windows or vpforms)
and start sending a string giving the
port value for a socket to connect to.
\item [12] End a string giving a prompt for a window and start sending
a string given a title for the window.
\item [13] End the title of a window and start sending a string
giving the width of the window.
\item [14] End sending the width of the window and start sending
the height of the window.
\item [15] End sending window information for a window with small fonts.
\item [16] End sending window information for a window with big fonts.
\item [17] Clear the contents of a window.
\item [18] Close a window.
\item [19] Change the color.  This should be followed by another
byte to indicate the color.  For now, this third byte can be
0 (black), 1 (red), 2 (blue), or 3 (green).
\end{enumerate}

\section{Generating the Java Menus}

There are two categories in the \TPS lisp code for menus:
\indexother{menu} and \indexother{menuitem}.  Everytime
a programmer adds a command (mexpr), flag, or top level
command, a corresponding menuitem should be defined.
This menuitem should have a parent menu to indicate where
the item lives.

The Java menu code is in the file \indexfile{TpsWin.java}
between the comment lines:
\begin{verbatim}
    // Menu Code BEGIN
\end{verbatim}
and
\begin{verbatim}
    // Menu Code END
\end{verbatim}
When you have added or changed menus or menuitems in
the lisp code and want the Java interface to reflect these
changes, perform the following steps:
\begin{enumerate}
\item Within \TPS, call the command \indexcommand{generate-java-menus}.
This will prompt for an output file, e.g., ``menus.java''.
This command will create an output file with Java code which should
be inserted into \indexfile{TpsWin.java}.
\item Delete the code between
\begin{verbatim}
    // Menu Code BEGIN
\end{verbatim}
and
\begin{verbatim}
    // Menu Code END
\end{verbatim}
in \indexfile{TpsWin.java}.
\item Insert the contents of the output file of \indexcommand{generate-java-menus}
(e.g., ``menus.java'') into \indexfile{TpsWin.java} between the comment lines
\begin{verbatim}
    // Menu Code BEGIN
\end{verbatim}
and
\begin{verbatim}
    // Menu Code END
\end{verbatim}
and save \indexfile{TpsWin.java}.
\item On each machine, find the Java directories which contains links
to the main java files (e.g., /home/theorem/tps/java/ and /home/theorem/tps/java/tps/).
cd to this
directory and call ``javac TpsWin.java'' to compile the new version.
\end{enumerate}

\section{Adding a New Symbol}

The Java information for the fonts is contained in the files \indexfile{TpsSmallFonts.java}
and \indexfile{TpsBigFonts.java}.  The lisp information containing the ``code'' for
the symbol is in \indexfile{interface-style.lisp}.
To add a new symbol for the Java interface,
one should add a new ``code'' for the symbol to the variable \indexother{istyle-characters}
in \indexfile{interface-style.lisp}.
For example, the epsilon character was added by including
\begin{verbatim}
    (epsilon 2 1)
\end{verbatim}
to \indexother{istyle-characters}.  This means that epsilon will be communicated
to the interface by sending the bytes 2 and 1 in symbol font mode.

Then one needs to add information about how to draw the new symbol to
the variables \verb+blank+, \verb+xstartData+, \verb+ystartData+, \verb+widthData+, 
and \verb+heightData+
in \indexfile{TpsSmallFonts.java} and \indexfile{TpsBigFonts.java}.
Each of these variables is set to a multi-dimensional array.
The 0'th element of each array corresponds to the normal fonts.
The rest are for symbol fonts.
For example, the information for epsilon should be put 
in the (2,1) position of each array.  This information (for epsilon
in \indexfile{TpsSmallFonts.java})
is as follows:
\begin{description}
  \item[blank] false (the epsilon character is not blank).
  \item[xstartData] {3,2,1,1,1,2,3} (This character is drawn using 7 rectangles starting
    from ``x'' coordinates 3, 2, 1, 1, 1, 2, 3, resp.)
  \item[ystartData] {10,9,8,7,6,5,4} (These are the ``y'' coordinates of the 7 rectangles.)
  \item[widthData] {4,1,1,4,1,1,4} (These are the widths of the 7 rectangles.)
  \item[heightData] {1,1,1,1,1,1,1} (These are the heights of the 7 rectangles.
    Since all heights are 1, the epsilon character is drawn by drawing 7 horizontal lines.)
\end{description}






