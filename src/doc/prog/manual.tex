% The outline for this is copied from Matt Bishop's Thesis.
\documentclass[11pt]{amsbook}
\usepackage{amsmath}
\usepackage{amscd}
\usepackage{amssymb}
\usepackage[T1]{fontenc} % encoding for european computer modern, loaded below
\usepackage[all]{xy} % commutative diagrams
\usepackage{bbold} % blackboard bold font
\usepackage{named} % the named bibliography style
\newfont{\bsc}{ecxc1200} % european computer modern, used for bold small caps in appendix a.
\makeindex % generate index data
\input tps.tex %requires TPSINPUTS to have the right path; these are in $tps/doc/lib
\input ndmacros.tex
\input vpd.tex
\newtheorem{definition}{Definition}[section] % a definition is labelled Definition and is numbered by section
\newtheorem{lemma}[definition]{Lemma} % lemmas are labelled Lemma and use the same numbering as definitions.
\newtheorem{example}[definition]{Example} % etc...
\newtheorem{theorem}[definition]{Theorem}
\newtheorem{algorithm}[definition]{Algorithm}
\newtheorem{proposition}[definition]{Proposition}
\newtheorem{conjecture}[definition]{Conjecture}
\numberwithin{figure}{chapter} % figures are 3.1, 3.2, 5.1,... rather than 1, 2, 3,...
\newenvironment{notation}{{\sc Notation}\it\ }{\rm}
\newenvironment{remark}{{\sc Remark}\ }{}
\newcommand{\missing}{\par{\bf *****INSERT PROOF HERE*****}\par} % a little reminder that something's missing...
\newcommand{\reml}{[\![}        % Left valuation-bracket, meaning ``remove this''
\newcommand{\remr}{]\!]}        % Right ditto
\newcommand{\uglylabel}[1]{{\hspace{-0.35in}\mbox{\small\bf #1}\hspace*{0.3in}}}
\newcommand{\fixme}{\marginpar[{\bf fix me! $\longrightarrow$}]{{\bf $\longleftarrow$ fix me!}}} % arrows in the margins...
\newcommand{\nchoose}[2]{$($\raisebox{-2pt}{$\stackrel{#1}{\scriptstyle #2}$}$)$} % in-line printing for `n choose k'.
\newcommand{\cal}{\mathfrak} % calligraphic font is really fraktur.
\newcommand{\foobar}[1]{\makebox[1.5em]{\rule[-.5ex]{0cm}{2ex}#1}} % used to label lattices in the results section
\newcommand{\comment}[1]{} % mimic the Scribe `comment' command.
\newcommand{\alg}[1]{{\em #1}\index{#1 algorithm}} % how to print the name of an algorithm and index it as well
\newcommand{\mmm}[2]{\alg{Merge}$(#1,#2)$} % \mmm{a}{b} prints Merge(a,b).
\newcommand{\flag}[1]{{\sc #1}\index{#1}} % flags are in small caps, and indexed
\newcommand{\bflag}[1]{{\bsc #1}\index{#1}} % bold flags, for appendix a.
\newcommand{\fval}[1]{{\sc #1}} % flag values in small caps...
\newcommand{\bfval}[1]{{\bsc #1}} % ...or bold ditto, in {description} contexts
\newcommand{\ftyp}[1]{{\sc #1}} % flag types, also in small caps.
\newcommand{\mexpr}[1]{{\tt #1}\index{#1}} % how to index a command (\command is already taken)
\newcommand{\simj}{\sim_{\!_J}} % twiddle-J.
\newcommand{\ssim}{\sim\!\!} % not, but with less space after it.
\newcommand{\nconv}{\stackrel{\sim}{\longrightarrow}}
\newcommand{\bconv}{\stackrel{\scriptscriptstyle\beta}{\displaystyle\longrightarrow}}
\newcommand{\nquiv}{\stackrel{\scriptscriptstyle\beta\sim}{\displaystyle =}}
\newcommand{\bquiv}{\stackrel{\beta}{\displaystyle =}}
\newcommand{\dquiv}{\stackrel{\scriptscriptstyle {\rm def}}{\displaystyle =}}
\newcommand{\rquiv}[1]{\stackrel{\scriptscriptstyle #1}{\displaystyle \longrightarrow}}
\newcommand{\cons}{\!^\frown} % list cons, like the ^ in a^(b)
\newcommand{\mystrut}{\rule[-0.75ex]{0ex}{2.5ex}}
\def\TPS{{\sc Tps3 }}
\def\ETPS{{\sc Etps }}
\def\OMEGA{\mbox{$\Omega${\sc mega}}}
\def\tps{{\sc Tps3}}
\def\LOUI{{\sc L$\Omega$UI}}
\def\SPASS{{\sc Spass}}
\def\OTTER{{\sc Otter}}
\def\PROTEIN{{\sc Protein}}
\def\CLAM{{\rm CL\raise.5ex\hbox{\sc a}M}}

% This is for Scribe compatibility, so that for some
% Scribe commands we can simply change the @ to a \
\def\wt{\sf}
\def\w{\sf}
\def\lisp{{\sc lisp}}
\newcommand{\indexother}[1]{#1\index{#1}}
\newcommand{\indexflag}[1]{#1\index{#1, Flag}}
\newcommand{\indexcommand}[1]{#1\index{#1, Command}}
\newcommand{\indexedop}[1]{#1\index{#1, EdOp}}
\newcommand{\indexmexpr}[1]{#1\index{#1, MExpr}}
\newcommand{\indexparameter}[1]{#1\index{#1, Parameter}}
\newcommand{\indexdata}[1]{#1\index{#1, Data}}
\newcommand{\indexfile}[1]{#1{\it \index{#1, File}}}
\newcommand{\indexfunction}[1]{#1\index{#1, Function}}
\newcommand{\indexstyle}[1]{#1\index{#1, Style}}
\newcommand{\indextypes}[1]{#1\index{#1, Type}}
\newcommand{\indexargtypes}[1]{#1\index{#1, Argument Type}}
\newcommand{\indexSyntax}[1]{#1\index{#1, Syntax}}
\newcommand{\indexProperty}[1]{#1\index{#1, Property}}
\newcommand{\indexData}[1]{#1\index{#1, Data}}
\newenvironment{Example}{ \\}{\\}

\begin{document}
\bibliographystyle{alpha}

%%%%%%%%%%%%%%%%%%%%%%%%%%%%%%%%%%%%%%%%%%%%%%%%%%%%%%%%%%%%%%%%%%%%%%%%
%List of NSF grants below copied from tpsdoc.lib
\frontmatter
\title{\TPS Programmer's Guide\thanks{
copyright \copyright 2000. Carnegie Mellon University.  All rights reserved.)
This manual is based upon work supported by
NSF grants MCS81-02870, DCR-8402532, CCR-8702699, 
CCR-9002546, CCR-9201893, CCR-9502878, CCR-9624683, CCR-9732312, CCR-0097179,
and a grant from the Center for Design of Educational Computing,
Carnegie Mellon University. Any opinions, findings, and conclusions or
recommendations are those of the author(s) and do not necessarily reflect
the views of the National Science Foundation.}
}
\author{Peter B. Andrews \\
Dan Nesmith \\
Frank Pfenning \\
Sunil Issar \\
Hongwei Xi \\
Matthew Bishop \\
Chad E. Brown \\
R\'{e}my Chr\'{e}tien
}
\date{Working Edition \\
\today}

\maketitle

\setcounter{tocdepth}{5}
\tableofcontents
\newpage

%%%%%%%%%%%%%%%%%%%%%%%%%%%%%%%%%%%%%%%%%%%%%%%%%%%%%%%%%%%%%%%%%%%%%%%%
%\mainmatter
%\input{results}
%\end{document}
%%%%%%%%%%%%%%%%%%%%%%%%%%%%%%%%%%%%%%%%%%%%%%%%%%%%%%%%%%%%%%%%%%%%%%%%
\chapter*{Preface}
\pagestyle{plain}
The following is a \TeX (actually, \LaTeX) version of the \TPS Programmer's Guide.
The original version is in Scribe format.
\mainmatter\pagestyle{headings}
\input{intro} 
\chapter{TPS Structures}

Notice that \TPS has a command \indexmexpr{TLIST} which outputs the
same information as the Lisp command {\it plist}, but formatted more
readably. So, for example, {\tt TLIST X2108} will show all of the slots 
in the structure X2108 (which is a proof).
 
\section{TPS Modules}

See the introductory chapter for a discussion of what \TPS modules are.

\subsection{The Tps3 Module Structure}

All modules are defined in one central file, called {\wt DEFPCK}.
You may want to look at this file to see examples of module definitions
and also a current list of all module known to \tps.

There is a partial order of
modules in \tps.  One whole set of modules called {\tt BARE} is
distinguished from the others.  All files in the module {\tt BARE} and all
of its submodules must always be present in a {\tt TPS3} core image.

When \TPS is built from {\bf Lisp}, some of the files in the {\tt BARE}
module can not be loaded with a module-loading command, since
it has not been defined.  Thus, even though every file for \TPS belongs
to a proper module, not all modules are loaded the same way because of
the ``bootstrapping'' problem.

Another quirk should be mentioned here.  A module called {\tt WFFS}
defines the basic operations of wffs.  The modules {\tt WFF-PRINT}
and {\tt WFF-PARSE} depend on {\tt WFFS.}  The module {\tt WFFS,} however,
cannot exist alone: the modules {\tt WFF-PRINT} and {\tt WFF-PARSE} must
be present also, even though this fact can not be deduced from the
module structure.


\subsection{Defining a New Module}

To define a new module for \tps, use the {\tt DEFMODULE} macro.
Its format is

\begin{verbatim}
(defmodule {\it name}
  (needed-modules {\it module} {\it module} ...)
  (macro-files {\it mfile} {\it mfile} ...)
  (files {\it file} {\it file} ...)
  (mhelp "{\it help-string}"))
\end{verbatim}

\begin{description}
\item [{\tt needed-modules}] These are all modules that must be loaded for the
module {\it name} to work.  Because of the transitive structure of modules
only the direct predecessors of the new module need to be listed.

\item [{\tt macro-files}] These are the files the compiler needs, before it can
compile any of the files in the module.  It is generally a good idea
to make a file with all the macro definitions (e.g. argument types,
flavors of labels, etc.) and separate it from the functions, commands,
etc. in the module.  This means clearer program structure, but also
minimal overhead for the compiler.

\item [{\tt files}] These are the rest of the files in the module.  When the module
is loaded, first the {\tt macro-files} are loaded, then the {\tt files.}
\end{description}

The new module should also be added into {\it defpck.lisp} at an appropriate point, and 
should be added into whichever of {\it tps-build.lisp}, {\it tps-compile.lisp},
{\it etps-build.lisp} and {\it etps-compile.lisp} are appropriate (these files are in the
same directory as the {\it Makefile}, not the main TPS directory).

\section{Categories}
\label{categories}

\TPS categories are in a sense data types. A category is a way to characterize a set of similar
objects which have properties of the same types, use the same auxiliary functions, are acted on 
by the same functions, etc.

Categories are orthogonal to the package/module structure, i.e. a category may have members which are 
defined in many different packages and modules. Categories group objects by functionality 
(how they behave) whereas packages and modules group objects by purpose (why they exist).

Categories are defined using the \indexother{defcategory} macro. For example, the 
definition of the category of \TPS top levels is:
%\begin{lispcode}
\begin{verbatim}
(defcategory toplevel
  (define deftoplevel)
  (properties
   (top-prompt-fn singlefn)
   (command-interpreter singlefn)
   (print-* singlefn)
   (top-level-category singlefn)
   (top-level-ctree singlefn)
   (top-cmd-interpret multiplefns)
   (top-cmd-decode singlefn)
   (mhelp single))
  (global-list global-toplevellist)
  (mhelp-line "top level")
  (mhelp-fn princ-mhelp))
\end{verbatim}
%\end{lispcode}

This shows a category whose individual members are defined with the {\tt deftoplevel} command, and
whose properties include the prompting function, a command interpreter, and so on. There is a 
global list called {\tt global-toplevellist} which will contain a list of all of the top levels 
defined, and an mhelp line "top level" (so that when you type {\tt HELP MATE}, \TPS knows to respond
"MATE is a top level".) The mhelp-fn is the function that will be used to print the help messages
for all the objects in this category. (See chapter ~\ref{help} for more information.)

The chapters of the facilities guide correspond to categories.
Within each chapter, the sections correspond to contexts.
In \tps, \indexother{global-categorylist} contains a list of all the
currently defined categories.

\section{Contexts}

Contexts are used to provide better help messages for the user. Each context is used to partition
the objects in a category into groups with similar tasks. For example, the objects in the 
category {\tt MEXPR} are grouped into contexts such as {\tt PRINTING} and {\tt EQUALITY RULES}.
(Contexts are themselves a category, of course: the definition is in {\it boot0.lisp}.)

New contexts are defined with the \indexother{defcontext} command, and are invoked with the 
single line {\tt (context {\it whatever}}) in the code (all this does is to set a variable 
{\tt current-context} to {\it whatever}). 

Here is a sample use of {\tt defcontext}:
%\begin{lispcode}
\begin{verbatim}
(defcontext tactics
  (short-id "Tactics")
  (order 61.92)
  (mhelp "Tactics and related functions."))
\end{verbatim}
%\end{lispcode}

The only property which is not immediately self-explanatory is {\tt order}; this is used to sort
the contexts into order before displaying them on the screen (or in manuals).

Contexts are used in the facilities guide (for example) to divide
chapters into sections. For example, the line
{\tt (context unification)}
occurs prior to the definition 
{\tt (defflag max-utree-depth ...)}
of the flag MAX-UTREE-DEPTH in the file {\it node.lisp},
and so this flag occurs in the section on unification in the
chapter on flags in the facilities guide.

To see the contexts into which the commands for a given top-level
are divided, just use the ? command at that top-level.
Look at \indexother{global-contextlist} in \TPS to see all the contexts.

\section{Flavors}
\label{flavors}

Some TPS structures (in particular, all expansion tree nodes, expansion variables, skolem terms and jforms) 
are defined as \indexother{flavors}; see the file {\it flavoring.lisp} for 
the details. These structures have many attached properties which allow 
wffops to be used on them as though they were gwffs; for example, the flavor
{\tt exp-var} in {\it etrees-exp-vars.lisp} has the properties
%\begin{tpsexample}
\begin{verbatim}
  (type (lambda (gwff) (type (exp-var-var gwff))))
  (gwff-p (lambda (gwff) (declare (ignore gwff)) T))
\end{verbatim}
%\end{tpsexample}
which state that the type of an {\tt exp-var} structure is the type of its variable, and all {\tt exp-var}s 
are gwffs. Errors of the form "Wff operation <wffop> cannot be applied to labels of flavor <label>" are almost always
caused by attempting to use a wffop on a flavor for which the corresponding property is undefined; for example,
if we deleted the lines above and recompiled TPS, any attempt to find the type of an expansion variable
would result in the error "Wff operation TYPE cannot be applied to labels of flavor EXP-VAR".

Flavors that are defined within TPS will also have the slot {\tt \indexother{bogus-slot}}; this slot is 
tested for by TPS to confirm that the flavor was defined by TPS, but the contents of this slot are never 
examined. This means that there is always one empty slot in each node of an expansion tree or jform which the
programmer can use to store information while a program is being tested (whereas if you define a new slot, you have to 
recompile all instances of a structure, which can be a nuisance). Obviously, once the new code is working,
you should define a new slot, change all references to bogus-slot and recompile TPS!

For examples of flavors of gwffs, see page \pageref{labels}.
\chapter{Top-Levels}\label{toplev}

\section{Defining a Top Level}

Top levels are a \TPS category, whose definition is given in section ~\ref{categories}.
For an example, let's look at the editor top level:

%\begin{lispcode}
\begin{verbatim}
(deftoplevel ed-top
  (top-prompt-fn ed-top-prompt)
  (command-interpreter ed-command-interpreter)
  (print-* ed-print-*)
  (top-level-category edop)
  (top-level-ctree ed-command-ctree)
  (top-cmd-decode opdecode)
  (mhelp "The top level of the formula editor."))
\end{verbatim}
%\end{lispcode}

This says that the top level {\tt ed-top} identifies itself by the function {\tt ed-top-prompt},
which is one of the more complicated prompt functions in \TPS; its only purpose is to 
print the {\tt <ed34>} messages at the start of each line in the editor, but the complications
are necessary because the editor can be entered recursively.

The next line of the toplevel definition gives the name of the command interpreter function.
The {\tt print-*} function is a function that gets called after every line; in this case, it's the
{\tt ed-print-*} function, which prints out the current wff if it has changed due to the last command.
The top level category is {\tt edop}, which is defined as follows:

%\begin{lispcode}
\begin{verbatim}
(defcategory edop
  (define defedop)
  (properties
   (alias single)
   (result-> singlefn)
   (edwff-argname single)
   (defaultfns multiplefns)
   (move-fn singlefn)
   (mhelp single))
  (global-list global-edoplist)
  (shadow t)
  (mhelp-line "editor command")
  (scribe-one-fn
    (lambda (item)
      (maint::scribe-doc-command 
       (format nil "@IndexEdop(~A)" (symbol-name item))
       (remove (get item 'edwff-argname) 
	       (get (get item 'alias) 'argnames))
       (or (cdr (assoc 'edop (get item 'mhelp)))
	   (cdr (assoc 'wffop (get (get item 'alias) 'mhelp)))))))
  (mhelp-fn edop-mhelp)))
\end{verbatim}
%\end{lispcode}

This category defines the sort of command found in the editor top level (compare 
the above definition with that of {\tt mexpr}, for example). So all the commands 
that can only be seen from the editor top level are defined with the {\tt defedop} 
command, as follows:

%\begin{lispcode}
\begin{verbatim}
(defedop o
  (alias invert-printedtflag)
  (mhelp "Invert PRINTEDTFLAG, that is switch automatic recording of wffs
in a file either on or off.  When switching on, the current wff will be
written to the PRINTEDTFILE. Notice that the resulting file will be in 
Scribe format; if you want something you can reload into TPS, then use
the SAVE command."))
\end{verbatim}
%\end{lispcode}

The {\tt top-command-ctree} is used for command completion, and the {\tt mhelp} 
property is obvious. This leaves {\tt top-cmd-decode}, which is the name of the
function that is called by the command interpreter to, for example, fill in the 
default arguments for an edop.

\section{Command Interpreters}

Each top level has its own command interpreter. The actual command interpreters
in much of the code are older versions; the code has since been simplified
considerably. New command interpreters, which may in time replace the older versions,
and which should certainly be used as the models for the command interpreters
of any new top levels, are in the two files \indexfile{command-interpreters-core.lisp} 
and \indexfile{command-interpreters-auto.lisp}.

\chapter{MExpr's} \label{defmexprs}
\TPS provides its own top-level.  It allows for default arguments
and provides a way of giving arguments (e.g. wffs) in some external
representation which is converted before the "real" function is called.
All this is also available in an interactive mode, where the user is
prompted for arguments after he has been told what the defaults are
and which alternatives are open.  The way all this has been implemented
is through MExpr's, which constitute special functional objects analogous
to Expr's or FExpr's in LISP.  Every \TPS command should be an MExpr
so that the facilities of \TPS' top-level can be utilized. 

\section{Defining MExpr's} \label{mexprargs}
Mexprs are special functional objects that are
recognized by the top level of \tps.
They can be defined with the {\tt defmexpr} macro, which has
a number of optional arguments.
The general format is ({\tt {}} indicate optional arguments)
\begin{verbatim}
(defmexpr {\it name}
	 {(ArgTypes {\it type1} {\it type2} ...)}
	 {(ArgNames {\it name1} {\it name2} ...)}
	 {(ArgHelp {\it help1} {\it help2} ...)}
	 {(DefaultFns  {\it fnspec1} {\it fnspec2} ...)}
         {(EnterFns {\it fnspec1} {\it fnspec2} ...)}
	 {(MainFns {\it fnspec1} {\it fnspec2} ...)}
	 {(CloseFns {\it fnspec1} {\it fnspec2} ...)}
         {(Print-Command {\it boolean})}
         {(Dont-Restore {\it boolean})}
	 {(MHelp "{\it comment}")}
\end{verbatim}

There are actually two other possible entries, {\tt Wffop-Typelist} and {\tt WffArgTypes}; these
are only used in mexprs which are generated automatically by the Rules package.

In the following a {\it function specification} is either a symbol naming
a function, or an expression of the form {\tt (Lambda {\it arglist} . {\it body})}.
We also assume that the main function which is to perform the command has
{\it n} arguments.  Then the phrases in the above definition have the
following meaning.
\begin{description}
\item [{\it name}\index{{\it name}}]
 This is the name of the MExpr as called by the user.

\item [{\tt ArgTypes}] This is a list which must have as many elements as the
function arguments, i.e. {\it n}.  {\it type1}, {\it type2}, ..., {\it typen} have
to be valid types, which means that they
have to have a non-{\tt NIL} {\tt ArgType} property.  Each argument supplied by the
user on the command line will be processed first by the corresponding
{\tt GetFn}.  In case an FExpr is to be called, each element of the
argument list is presupposed to be of the same type.  This type is
specified in parentheses.  If {\tt ArgTypes} is omitted, the function has
no arguments.

\item [{\tt ArgHelp}] This has to be a list of length {\it n}.  Each element is
a string describing the argument, or {\tt NIL}.  These quick helps
for arguments can be accessed via the {\tt ?} when being prompted
for the argument value.  For an FExpr, there should be only one string.

\item [{\tt DefaultFns}]
The {\it fnspecs} declared in this place are being processed in a
left-to-right order, where the result of one {\it fnspec} is passed on
to next.  A {\it fnspec} can signal an error (a {\tt THROW} with a
{\tt FAIL} label) if the arguments seem to be contradictory (e.g.  if a
planned line and a term is supplied for a {\tt P}-rule, but the term does
not appear in the proof), but it can count on the arguments being of
the correct type and in internal representation. 

In detail, each default {\it fnspec} must be either a symbol denoting
a function of {\it n} arguments, where {\it n} is the number of mexpr
arguments, or else a lambda expression of {\it n} arguments.
Each {\it fnspec} must return a list of
length {\it n}.  This list will then be handed on and processed by the next
{\it fnspec} as if it were the list of arguments supplied by the user.
Any entry which is not a {\tt \$} should be left unchanged.  The function
is not allowed to have side-effects.
As a general convention, the arguments which are not used by
a {\it fnspec} are not written out with their name, but replaced by
{\tt \%}{\it i}.  This makes it easier to see at one glance which defaults
are filled in by a certain {\it defaultspec}. 

\item [{\tt EnterFns}] {\it fnspec1}, {\it fnspec2}, ... is an arbitrary list of function
specifications.  They are called in succession with the value list
returned by the last default {\it fnspec}, before the {\tt MainFns} are called.

\item [{\tt MainFns}] {\it fnspec1}, {\it fnspec2}, ... is an arbitrary list of function
specifications.  They are called in succession with the value list
returned by the last default {\it fnspec}.  If none are specified, it is assumed
that there is a function named {\it name}, which can be called.
Notice that at this stage, no defaulted arguments may be left.
{\tt ComDeCode} (the command processing function) will refuse to call
any function, unless all the defaults are determined.  This clearly
divides the responsibilities between {\tt GetFn}'s, {\tt DefaultFn}'s and
{\tt MainFn}'s. Any {\it fnspec} may abort with an error by doing a
{\tt THROW} with a {\tt FAIL} label.  A {\tt THROW} with a {\tt TryNext} label
will be handled like a normal return.  A {\tt THROW} with a {\tt CutShort} label
means that none of the remaining {\tt MainFn}'s will be executed and the 
value of the {\tt THROW} will be handed on to the {\tt CloseFn}'s.

\item [{\tt CloseFns}] {\it fnspec1}, {\it fnspec2}, ... is a list of function
specifications.  They are called in succession with the value returned
by the last {\tt MainFn}.  Even if the {\tt MainFn}'s were FExpr's, each
{\it fnspec} has to describe an Expr.

\item [{\tt Dont-Restore}] {\it boolean} determines whether or not this command will be
restored, if it is saved using SAVE-WORK. For example, commands like HELP and ? should not
be restored.

\item [{\tt Print-Command}] {\it boolean} is used by RESTORE-WORK and EXECUTE-FILE, which 
both ask "Execute Print-Commands?"; this is how they know which commands are print 
commands.

\item [{\tt MHelp}] This has to be a string and will be available through
{\tt UserHelp} and the {\tt ??} if no {\tt QuickHelp} is available.

\end{description}

\section{Argument Types}
At the top-level of \TPS
explicitly declared argument types are available.  Many of the more important ones are
all declared in the file {\tt argtyp.lisp}.  They can be recognized by their
{\tt ArgType} property value, which is {\tt T}.  Each of argument type
has at least three properties, {\tt GetFn}, {\tt TestFn}, and {\tt PrintFn}.
{\tt GetFn} is responsible for translating the user's value
into internal representation, {\tt TestFn} tests if some object is of the given
type, and {\tt PrintFn} makes the internal
representation intelligible to the user.  

The defining command for the category {\tt argtype} is actually {\tt deftype\%\%}, but 
all definitions of argtypes should be made through the secondary macro {\tt DefType\%}.
Its format is as follows
({\tt {}} enclose optional arguments):
\begin{verbatim}
(DefType% {\it name}
	 (GetFn {\it fnspec})
	 (TestFn {\it fnspec})
	 (PrintFn {\it fnspec})
	{(Short-Prompt {\it boolean})}
	{(MHelp "{\it comment}")}
	{({\it property1} {\it value1}) ({\it property2} {\it value2}) ...})
\end{verbatim}
In the above a {\it fnspec} is either the name of a one-argument function,
or a list of forms which are to be evaluated as an implicit progn.
In the latter case, {\it name} stands for the argument supplied.
\begin{description}
\item [{\it name}\index{{\it name}}] 
The name of the argument type.  It will get a property value of {\tt T}
for the property {\tt ArgType} when the {\tt DefType\%} has been executed.


\item [{\tt GetFn}] Here {\it fnspec} defines the function used to process the argument
as supplied by the user on the command line.  The value returned by it
is then handed on to the main function executing the command.  No {\tt GetFn}
will ever receive a {\tt \$}.  It is simply not called, if the corresponding
argument in the command line is defaulted.  A {\tt GetFn} should signal an
error if the argument is not
of the correct type.  This will be implemented (as it is right now)
as a {\tt THROW} with the label {\tt FAIL}.  A special case of {\it fnspec}
for a {\tt GetFn} is {\tt TestFn}.  This means the {\tt GetFn} will
test if the supplied argument is of the correct type.  If yes, the argument
will simply be returned, otherwise an error will be signaled.
A {\tt GetFn} may have side-effects, but this has to be declared
under {\tt Side-Effects}.

\item [{\tt PrintFn}] Here {\it fnspec} should print the external representation
of its only argument.  It can expect this argument to be of the correct
type.  The value returned is ignored.  A {\tt PrintFn} may signal an error
if printing is not possible (e.g. if the current style does not
have a representation of the given data type).

\item [{\tt TestFn}] Here {\it fnspec} should return {\tt NIL} if its argument is not of
type {\it name}, and return something not {\tt NIL} otherwise.

\item [{\tt Short-Prompt}] {\it boolean} is only used in {\it otl-typ.lisp}, but I can't 
work out what for.

% \begin{comment}
% This was removed...
% {\tt Side-Effects}\\ {\it value1}, {\it value2}, ... constitute a list of identifiers
% which cause the {\tt GetFn} to have side-effects.  {\w {\tt (Side-Effects T)}}
% means that any argument to {\tt GetFn} will lead to side-effects.  This is
% necessary for instance in the {\tt RWff} argument type, since {\tt RDC}, {\tt RD} ,or
% {\tt PAD} have side-effects and thus should not be called more than once.
% It is useful however, to call a {\tt GetFn} on the same argument more than
% once to figure out defaults as well as possible.
% \end{comment}
\item [{\tt MHelp}] This is an optional documentation and is accessed during {\tt MHelp}
or after a {\tt ?} while the user supplies command arguments interactively.

\item [({\it property} {\it value})] Pairs like this allow for more information about the type.
\footnote{More properties may become useful, so
the {\tt Deftype\%} macro allows arbitrary property names.  Possibilities here
include {\tt EdFn} (for editing this argument type) or {\tt OutputFn} (to be able to
read back a data object of the specified type.)}
\end{description}
For example
\begin{verbatim}
(deftype% anything
  (getfn (lambda (anything) anything))
  (testfn (lambda (anything) (declare (ignore anything)) t))
  (printfn princ)
  (mhelp "Any legal LISP object."))

(deftype% integer+
  (getfn testfn)
  (testfn (and (integerp integer+) (> integer+ -1)))
  (printfn princ)
  (mhelp "A nonnegative integer."))

(deftype% boolean
  (getfn (cond (boolean t) (t nil)))
  (testfn (or (eq boolean t) (eq boolean nil)))
  (printfn (if boolean (princ t) (princ nil)))
  (mhelp "A Boolean value (NIL for false, T for true)."))
\end{verbatim}

No {\tt TestFn} or {\tt PrintFn} is allowed to have any
side-effects, since they may be called arbitrarily often.  No {\tt GetFn}
needs to expect
{\tt \$} as an argument, since defaults are now figured out elsewhere.
This avoids conflicts between different defaults for the same argument
type in different functions.  Hence {\tt GetFn} never computes the
default. 

\subsection{List Types}

The macro {\tt deflisttype} defines a list from an existing type:

%\begin{lispcode}
\begin{verbatim}
(deflisttype filespeclist filespec)
\end{verbatim}
%\end{lispcode}

This takes an existing type, {\tt filespec}, and produces a type of lists of filespecs.
It is also possible to specify other properties (the same properties as for {\tt deftype\%}),
in which case these properties override those of the original type. This is typically 
used to give the list type a different help message from the original type.

\subsection{Consed Types}

The macro {\tt defconstype} defines a type as a cons of two existing types:

%\begin{lispcode}
\begin{verbatim}
(defconstype subst-pair gvar gwff
  (mhelp "Means substitute gwff for gvar."))
\end{verbatim}
%\end{lispcode}

This takes two existing types, {\tt gvar} and {\tt gwff}, and produces a type {\tt subst-pair} of 
consed pairs {\tt (gvar . gwff)}.
It is also possible to specify other properties (the same properties as for {\tt deftype\%}),
in which case these properties override those of the original type. This is typically 
used to give the cons type a different help message from the original type.






% \begin<comment>
% Which other properties of argument types are allowed is determined
% by the global variable {\tt ArgTyProps}.
% {\tt ArgTyProps} is very similar to {\tt GrinProps} and serves the same
% purpose.  It contains a list of pairs
% whose first element is the name of a property associated with the
% argument type and whose second element is either {\tt LIST} or {\tt CONS}.
% A value of {\tt LIST} indicates that the property to be stored is just
% a single element, and hence has to be written out using
% @w[{\tt (LIST {\it property} (GET {\it type} {\it property}))}].  {\tt CONS} on the other
% hand means that the property will be a list and thus has to be written
% using
% @w[{\tt (CONS {\it property} (GET {\it type} {\it property}))}].  
% As an example consider the following definition made in the file
% {\tt BASICS}.
% \begin{verbatim}
% (DV ARGTYPROPS
%     ((GETFN . LIST)
%      (TESTFN . LIST)
%      (PRINTFN . LIST)
%      (SIDE-EFFECTS . CONS)
%      (MHELP . LIST)))
% \end{verbatim}
% \end<comment>

% \begin<comment>
% \section{The Tactic mechanism}
% {\it Carl's comments}
% Tactics and tacticals, though inspired by proof methods, are implemented
% in a general way for \tps. Basically, anything which can described as
% a goal-satisfying process using some fixed kind of object or state, can
% form a base for tactics. We define four entities to implement tactics.
% 
% \begin{description}
% \item [Tacticals] These are functions which take formal arguments, either parameters,
% tactic functions, tactic expressions or lists of these, and with additional
% arguments
% indicating the goals, object and bindings for parameters, return the same
% values as tactics. The application of a tactical to its formal arguments
% is a tactic expression. For example, {\tt (or deduct same)} is a
% tactic expression, with {\tt or} as the tactical.
% 
% \item [Base] A base determines how a tactic expression consisting of a non-tactic
% atom is to be interpreted. Besides the recognition function ({\tt isfn})
% and the function to "run" the atom, producing the same output as a tactic,
% there are properties to ensure the correct operation of certain tacticals.
% For example, {\tt fork} needs a process function in order for the forking
% of processes to make sense in a particular base (and to hack the setting
% of global variables). Much of the work of developing tactics for, say,
% rules consists of defining the base correctly. This makes
% a great deal more sense than to change the general procedure to adapt to
% changes in the implementation of rules.
% 
% \item [Tactic functions] These are functions made into objects for tactics.
% This allows them to be documented, as would be necessary for functions
% accessing an expression tree, and limited, as teachers allowing rule
% tactics might wish.
% 
% \item [Tactics] A Tactic is an object with a tactic expression in its {\tt defn}
% property. If uncompiled, the tactic interpreter, {\tt run-tactic-exp},
% will interpret that tactic expression. If compiled, its {\tt compile}
% property should contain a function to achieve the same effect as
% when it is uncompiled. [Compilation has not been implemented.]
% A tactic has arguments, listed in it {\tt mouths} property, which the 
% {\tt feed} tactical may fill using a feed function for the base.
% This provides a more general kind of argument-handling.
% 
% \end{description}
% \end<comment>

\chapter{Representing Well-formed formulae}

\section{Types}

\begin{description}
\item [\indextypes{\it typeconstant} ] ::= Type Constant

An identifier with a non-{\tt NIL} {\tt TypeConst} property.
For example, {\tt O} and {\tt I}:
%\begin{tpsexample}
\begin{verbatim}
(def-typeconst o
  (mhelp "The type of truth values."))
\end{verbatim}
%\end{tpsexample}

\item [\indextypes{\it typevariable} ] ::= Type Variable

 An identifier with a non-{\tt NIL} {\tt TypeVar} property.
\end{description}
It is the parsers responsibility to give the {\tt TypeVar} property to types 
not previously encountered.

\begin{description}
\item [\indextypes{\it typesymbol} ] ::= {\it typeconstant} | {\it typevariable} | 
{\tt ({\it typesymbol} . {\it typesymbol})}
\end{description}

\section{Terminal Objects of the Syntax}\label{terminalobjects}

Before going into detail about the terminal objects of the syntax,
some general remarks about type polymorphism in \TPS are needed.

\TPS supports polymorphic objects, like $\subseteq$ (subset), which is
a relation that may hold between sets of any type.  It must be understood,
however, that the parser completely eliminates this ambiguity of types,
when actually reading a wff.  In a given wff every proper subwff has
a type!  Therefore, there is a class of objects with polymorphic type,
which never appear in a wff, but nevertheless may be typed by the user.
The instances of those polymorphic abbreviations or polymorphic proper
symbols inside the formula will refer, however, to those polymorphic primitive
symbols or polymorphic abbreviating symbols.

For reasons of efficiency, binders are handled slightly differently.
Binders are also polymorphic in the sense that a certain binder, say
$\forall$, may bind variables of any type.  The case of binder, however,
is slightly different from that of polymorphic abbreviations, since
a binder is not a proper subwff.   Binders, therefore, are left without
having a proper type.  We must, however, be able to figure out the type
of any given bound wff.  Thus each binder carries the information
about the type of the scope, the bound variable and the resulting bound
wff with it.  See Section ~\ref{Binders} for more details.

The list below introduces syntactic categories of objects known to the
parser only, which are not legal in wffs themselves.

\begin{description}
\item [\indexSyntax{pmprsym} ] ::= Polymorphic Primitive Symbol

 {\it pmprsyms} are the {\tt STANDS-FOR} property of 
{\it pmpropsyms}, but cannot appear in {\it gwffs} themselves.  
Examples would be {\tt PI} or {\tt IOTA}.

\item [\indexSyntax{pmabbsym} ] ::= Polymorphic Abbreviating Symbol

 \indexSyntax{pmabbsym} are the {\tt STANDS-FOR} property of 
{\it pmabbrevs}, but cannot appear in {\it gwffs} themselves. 
Examples are {\tt SUBSET}, {\tt UNION}, or {\tt IMAGE}.

\end{description}

The following categories are the ``terminal'' objects of proper wffs.
The parser may not produce a formula with any other atomic (in the 
Lisp sense) object then from the list below.

\begin{description}
\item [\indexSyntax{logconst} ] ::= Logical Constants

 For example: {\tt AND,} {\tt OR,} {\tt IMPLIES,} {\tt NOT,} {\tt FALSEHOOD,} {\tt TRUTH}:
%\begin{tpsexample}
\begin{verbatim}
(def-logconst and
   (type "OOO")
   (printnotype t)
   (infix 5)
   (prt-associative t)
   (fo-single-symbol and)
   (mhelp "Denotes conjunction."))
\end{verbatim}
%\end{tpsexample}

\item [\indexSyntax{propsym} ] ::= Proper Symbols

 For example: {\tt P<OA>}, {\tt x<A>}, {\tt y<A>}, {\tt Q<OB>}, {\tt x<B>} are
proper symbols after parsing
$ \forall \,x \forall \,y . \,P_{\greeko\greeka\greeka} \,x \,y \land \,Q_{\greeko\greeka} \,x$.
This example
demonstrates part of the parser.  Since in a given wff, a proper symbol
may appear with more than one type, the type of each proper must somehow
be encoded in its name.  \TPS does this by appending the type, {\tt (} and
{\tt )} replaced by {\tt <} and {\tt >}, respectively, to the print name of the
symbol.

\item [\indexSyntax{pmpropsym} ] ::= Polymorphic Proper Symbols

 These are just like \indexSyntax{propsym}, except that they also have 
a {\tt STANDS-FOR} property, which is the polymorphic primitive symbol
(\indexSyntax{pmprsym}) this polymorphic proper symbol was constructed
from.  Note that this particular instance of the polymorphic primitive
symbol always has a specific given type.  For example: {\tt IOTA<I<OI>>}
is a pmpropsym after parsing $y_{\greeki} = \iota [QQ y]$:
%\begin{tpsexample}
\begin{verbatim}
(def-pmpropsym iota
  (type "A(OA)")
  (typelist ("A"))
  (printnotype t)
  (fo-single-symbol iota)
  (mhelp "Description operator"))
\end{verbatim}
%\end{tpsexample}

\item [\indexSyntax{abbrev} ] ::= Abbreviations

 For example: @EQUIV.  This is separate category from polymorphic
abbreviations only for reasons of efficiency.  An abbreviation could
be thought of as a polymorphic abbreviation with an empty list of
type variables. For example:
%\begin{tpsexample}
\begin{verbatim}
(def-abbrev equiv
  (type "OOO")
  (printnotype t)
  (fo-single-symbol equiv)
  (infix 2)
  (defn "[=(OOO)]"))
\end{verbatim}
%\end{tpsexample}

\item [\indexSyntax{pmabbrev} ] ::= Polymorphic Abbreviations

 For example: {\tt SUBSET<O<OA><OA>>}, {\tt SUBSET<O<OB><OB>>} are polymorphic
abbreviations after parsing {\wt A@f12(oa) @SUBSET B @or [R@f12(obb) a] @~
@SUBSET [R b]}. For example:
%\begin{tpsexample}
\begin{verbatim}
(def-abbrev subset
   (type "O(OA)(OA)")
   (typelist ("A"))
   (printnotype t)
   (infix 8)
   (fo-single-symbol subset)
   (defn "lambda P(OA) lambda R(OA). forall x . P x implies R x"))
\end{verbatim}
%\end{tpsexample}

\item [\indexSyntax{binder} ] ::=  Variable Binders

For example: $\forall$, $\exists$, $\lambda$, $\exists_1$. See the section below.

\item [\indexSyntax{label} ] ::= A Label referring to one or more other wffs.

 For example: {\tt AXIOM1}, {\tt ATM15}, {\tt LABEL6}.  See Section ~\ref{labels}.
\end{description}

In principle, the implementation is completely free to choose the
representation of the different terminal objects of the syntax.
The functions with test whether a given terminal object is of a given
kind is the only user visible functions.  Once defined, the particular
implementation of the object should not be needed or relied upon
by other functions.

It is explained more precisely what is meant by ``quick'' and ``slow''
predicates to decide whether a given object is in a certain syntactic
category in section ~\ref{quickslow}.  Here is a table of the different
syntactic categories with the ``slow'' test function for it and
the properties that are required or must be absent.
Keep in mind that the list reflects the current implementation, and
may not be reliable.

%\begin{Format, Group}
%@TabSet(1inch,2.25inch,4inch)
\begin{tabular}{lll}
Category & Predicate & Required Properties \\
& & Absent Properties \\
\\
{\it pmprsym} & {\tt PMPRSYM-P} & {\tt TYPE}, {\tt TYPELIST} \\ & & {\tt DEFN} \\
{\it pmabbsym} & {\tt PMABBSYM-P} & {\tt TYPE}, {\tt TYPELIST}, {\tt DEFN} \\
 \\
{\it logconst} & {\tt LOGCONST-P} & {\tt TYPE}, {\tt LOGCONST} \\
{\it propsym} & {\tt PROPSYM-P} & {\tt TYPE} \\ & & {\tt LOGCONST}, {\tt STANDS-FOR} \\
{\it pmpropsym} & {\tt PMPROPSYM-P} & {\tt TYPE}, {\tt POLYTYPELIST}, {\tt STANDS-FOR}
(a {\it pmprsym}) \\ 
\\
{\it pmabb} & {\tt PMPROPSYM-P} & {\tt TYPE}, {\tt POLYTYPELIST}, {\tt STANDS-FOR}
(a {\it pmabbsym}) \\
\\
{\it abbrev} & {\tt ABBREV-P} & {\tt TYPE}, {\tt DEFN} \\ & & {\tt TYPELIST} \\
{\it label} & {\tt LABEL-P} & {\tt FLAVOR} \\
 \\
{\it binder} & {\tt BINDER-P} & {\tt VAR-TYPE}, {\tt SCOPE-TYPE}, {\tt WFF-TYPE} \\ \\
\end{tabular}
%\end{Format}

\section{Explanation of Properties}
The various properties mentioned above are as follows:
\begin{description}
\item [{\tt TYPE} ] The type of the object.  Common are {\tt "OOO"} for
binary connectives and {\tt "I"} for individual constants.

\item [{\tt PRINTNOTYPE} ] In first-order mode, this is insignificant, but
if specified and {\tt T}, \TPS will never print types following the object.
It is almost always appropriate to specify this.

\item [{\tt INFIX} ] The binding priority of an infix operator.  This will declare
the connective to be infix.  The absolute value of {\tt INFIX} is irrelevant,
only the relative precedence of the infix and prefix operators matters.
If two binders have identical precedence, association will be to the left.
For example, if R1 and R2 are operators with {\tt INFIX} equal to 1 and 2,
respectively, {\tt "p R1 q R2 r R2 s"} will parse as 
{\tt "[p R1 [[q R2 r] R2 s]]"}.

\item [{\tt PREFIX} ] The binding priority of a prefix operator.  Binders are considered
prefix operators (see about binders below) and thus have a binding
priority.  The main purpose of a prefix binding priority is to allow
formulas like {\tt "~a=b"} to be parsed correctly as {\tt "~[a = b]"} by
giving {\tt =} precedence over {\tt ~}.

\item [{\tt PRT-ASSOCIATIVE} ] indicates whether to assume that the operator is
left associative during printing.  You may want to switch this off (specify
{\tt NIL}) for an infix operator like equivalence, say {\tt <=>}, since
{\tt "p <=> q <=> r"} is often considered to mean {\tt "p <=> q \& q <=> r"}.

\item [{\tt FO-SINGLE-SYMBOL} ] this is meaningful only in first-order mode and
declares the object to be a ``keyword'' in the sense that
it may be typed in all upper or lower case.  Moreover, the printer will
surround it by blanks if necessary to set off surrounding text.  Also
the parser will expect that the symbol is delimited by spaces, dots,
brackets, unless the symbol just consists of one letter, in which case
it doesn't matter.  You {\bf MUST} use this attribute in first-order
mode for an identifier with more than one character.

\item [{\tt MHELP} ] An optional help string.
\end{description}

Properties specific to binders are described in the section below about binders.
Here are some more examples. These examples do not actually exist under these names in \tps.
%\begin{tpsexample}
\begin{verbatim}
(def-logconst &
   (type "OOO")
   (printnotype t)
   (infix 5)
   (prt-associative t)
   (fo-single-symbol &)
   (mhelp "Conjunction."))
\end{verbatim}

Note that the {\tt (fo-single-symbol \&)} will make sure that spaces are printed
around {\tt \&} in formulas.

In the next example the character {\tt /} is used to make sure that the
disjunction is printed in lowercase, that is as {\tt v} instead of {\tt V}.

\begin{verbatim}
(def-logconst /v
   (type "OOO")
   (printnotype t)
   (infix 4)
   (prt-associative t)
   (fo-single-symbol /v)
   (mhelp "Disjunction."))

(def-logconst =>
   (type "OOO")
   (printnotype t)
   (infix 3)
   (fo-single-symbol =>)
   (mhelp "Implication."))
\end{verbatim}

We do not like spaces after negation.  So we do not declare it to be
a {\tt fo-single-symbol}.  That works only because {\tt -} consists of only
one character.

\begin{verbatim}
(def-logconst -
   (type "OO")
   (printnotype t)
   (prefix 6)
   (mhelp "Negation."))

\end{verbatim}
%\end{tpsexample}

\section{Non-terminal Objects of the Syntax}

%@TabSet(1inch)
%\begin{Format, Group}
\begin{tabular}{ll}
\indexSyntax[lsymbol] ::= &  {\it logconst} \\
&  | {\it propsym} \\
&  | {\it pmpropsym} \\
&  | {\it abbrev} \\
&  | {\it pmabbrev} \\
%\end{Format}
\end{tabular}

{\it lsymbol} roughly corresponds to what was called
\indexSyntax{hatom} {for Huet-atom from Huet's unification algorithm}
in the old representation.

{\bf Generalized WFFs}

%\begin{Format, Group}
\begin{tabular}{ll}
{\it gwff} ::=&   {\it lsymbol} \\
&  | {\tt (({\it propsym} . {\it binder}) . {\it gwff})}  ; Generalized binder \\
&  | {\tt ({\it gwff1} . {\it gwff2})}  where {\tt (cdr (type {\it gwff1}))} = {\tt (type {\it gwff2})} \\
&  | {\it label} \\
%\end{Format}
\end{tabular}

\section{Binders in TPS}\label{Binders} 

In the discussion about the internal representation of wffs the issue
of binders has been neglected so far.  Currently, TPS allows three
binders, $\lambda$, $\forall$, $\exists$ (plus some ``buggy'' fragments of
support for the $\exists_{1}$ binder).

Since most binders are inherently polymorphic, there is only one kind of
binder.  Notice that the definition is formulated such that a binder may
have a definition, but need not.

In order to determine the type of a bound wff, the type of the scope of
the binder must be matched against the type stored in the {\tt SCOPE-TYPE}
property.  Also, the type of the bound variable must match the
type in the {\tt VAR-TYPE} property.  These matches are performed, keeping
in mind that all types in the {\tt TYPELIST} property are considered to
be variables.  Then the bindings established during the match are used
to construct the type of the whole bound wff, using the {\tt WFF-TYPE}
property of the binder.

An example may illustrate this process.  The binder {\tt LAMBDA} has the
following properties

% @TabDivide(4)
\begin{tabular}{ll}
TYPELIST & (A B) \\
VAR-TYPE & B \\
SCOPE-TYPE & A \\
WFF-TYPE & (A . B)
\end{tabular}

When trying to determine the type of 
$\lambda x_\greeki . R_{\greeko\greeki\greeki} x$,
\TPS determines that {\tt A} must be $\greeki$, and that {\tt B} must be $\greeko\greeki$.
The type of the original formula is {\wt (A . B)} which then turns out to
be $\greeko\greeki\greeki$.

Note that {\tt TYPELIST} may be absent, i.e.  could be {\wt ()}, which
amounts to stating that the binder has no variable types.  Currently, we
are not using such binders.  An example would be {\it Foralln} , which can
bind only variables of type $\sigma$.

In addition to the properties mentioned above, a binder (except $\lambda$)
would have a definition.  One can then instantiate a binder just as a 
definition can be instantiated.  The definition is to be written with
two designated variables, one for the bound variable and one for the scope.
For example 
{\it THAT} {\it has definition}
$$\iota_{\greeka(\greeko\greeka)} . \lambda b_\greeka S_\greeko$$
Here the {\tt TypeList} would be ($\greeka$), designation for the bound
variable would be $b_\greeka$, designation for the scope would be
$S_\greeko$.

The internal representation for a binder inside a wff is always the
same and simply {\tt ((bdvar .  binder) .  scope)}, but all of the above
information must be present to determine the type of a wff, or to check
whether formulas are well-formed.

Fancy ``special effects'' such as 
$\forall x\in S . A$
must be handled
via special flavors of labels and are not treated as proper binders themselves.

Here are some examples of binders:
%\begin{tpsexample}
\begin{verbatim}

(def-binder lambda
   (typelist ("A" "B"))
   (var-type "A")
   (scope-type "B")
   (wff-type "BA")
   (prefix 100)
   (fo-single-symbol lambda)
   (mhelp "Church's lambda binder."))

(def-binder forall
   (typelist ())
   (var-type "I")
   (scope-type "O")
   (wff-type "O")
   (prefix 100)
   (fo-single-symbol forall)
   (mhelp "Universal quantifier."))
{\it The above definition is for math-logic-1, where forall can only bind individual 
variables. In math-logic-2, the definition is as follows:}

(def-binder forall
   (typelist ("A"))
   (var-type "A")
   (scope-type "O")
   (wff-type "O")
   (prefix 100)
   (fo-single-symbol forall)
   (mhelp "Universal quantifier."))
\end{verbatim}
%\end{tpsexample}

\subsection{An example: How to See the Wff Representations}

You can see examples of how wffs are represented by comparing the output 
of the editor commands \indexedop{P} and \indexedop{edwff}:

%\begin{tpsexample}
\begin{verbatim}
<44>ed x2106
<Ed45>p

FORALL x(I) [R(OI) x IMPLIES P(OI) x] AND FORALL x [~Q(OI) x IMPLIES R x]
 IMPLIES FORALL x.P x OR Q x

<Ed46>edwff
((IMPLIES (AND (|x<I>| . FORALL) (IMPLIES R<OI> . |x<I>|) P<OI> . |x<I>|) 
 (|x<I>| . FORALL) (IMPLIES NOT Q<OI> . |x<I>|) R<OI> . |x<I>|) 
 (|x<I>| . FORALL) (OR P<OI> . |x<I>|) Q<OI> . |x<I>|)
\end{verbatim}
%\end{tpsexample}

Another way to do this is as follows:

%\begin{tpsexample}
\begin{verbatim}
<3>ed x2106
<Ed4>cw
LABEL (SYMBOL):  [No Default]>x2106a
<Ed5>(plist 'x2106a)
(REPRESENTS ((IMPLIES (AND (|x<I>| . FORALL) (IMPLIES R<OI> . |x<I>|) 
 P<OI> . |x<I>|) (|x<I>| . FORALL) (IMPLIES NOT Q<OI> . |x<I>|) R<OI> . |x<I>|) 
 (|x<I>| . FORALL) (OR P<OI> . |x<I>|) Q<OI> . |x<I>|) FLAVOR WEAK)
\end{verbatim}
%\end{tpsexample}

And another way:

%\begin{tpsexample}
\begin{verbatim}
<2>(getwff-subtype 'gwff-p 'x2106) 

((IMPLIES (AND (|x<I>| . FORALL) 
                 (IMPLIES R<OI> . |x<I>|) P<OI> . |x<I>|) (|x<I>| . FORALL)
               (IMPLIES NOT Q<OI> . |x<I>|) R<OI> . |x<I>|) 
     (|x<I>| . FORALL) (OR P<OI> . |x<I>|) Q<OI> . |x<I>|)
\end{verbatim}
%\end{tpsexample}

And finally a way that only works at type O (the 0 below is a zero, not a capital O):

%\begin{tpsexample}
\begin{verbatim}
<3>(get-gwff0 'x2106)

((IMPLIES (AND (|x<I>| . FORALL) 
                 (IMPLIES R<OI> . |x<I>|) P<OI> . |x<I>|)
                  (|x<I>| . FORALL)
                   (IMPLIES NOT Q<OI> . |x<I>|) R<OI> . |x<I>|) 
                     (|x<I>| . FORALL)
                      (OR P<OI> . |x<I>|) Q<OI> . |x<I>|)
\end{verbatim}
%\end{tpsexample}

\section{Flavors and Labels of Gwffs}
\label{defflavors}
\label{labels}

It is sometimes desirable to be able to endow a gwff with additional
properties.  For example, one may wish to be able to refer to a
gwff by a short tag, or to specify that a particular gwff is actually
a node in an expansion tree.  For this purpose, \TPS provides the
facility of {\it labels} and {\it flavors} (see page \pageref{flavors}).  
A {\it label} is an object which, as far as \TPS is concerned, is merely
a special case of gwff. Labels thus stand for gwffs, but may have
additional properties and distinct representations.

{\it Flavors} are the classes into which labels are divided.  The definition
of a flavor specifies some common properties of a class
of labels, usually the behavior of wffops and predicates.  Also, a 
flavor's definition should specify what attributes each label of that 
flavor should have, and how it should be printed.

\subsection{Representation}

Each flavor is represented in \TPS by a Lisp structure of type {\tt flavor}, 
which has the following slots: {\tt wffop-hash-table}, which stores the
properties common to each instance of the flavor, in particular, how
wffops are to behave; {\tt constructor-fun}, which
is the name of the function to be called when a new label of the flavor
is to be created; {\tt name}, the flavor's name; and {\tt mhelp}, a description
of the flavor.  The values of these slots are automatically computed when
\TPS reads a {\tt defflavor} declaration.
The flavor structures are stored in a central hash table, called
{\tt *flavor-hash-table*}, keyed on the flavor names.  This also is 
updated automatically whenever a flavor is defined (or redefined).

There are two ways to represent labels (instances of flavors), and the
choice is made during the definition of the flavor.  The first, and more
traditional, way is to have each label be a Lisp symbol, with the attributes
of the label being kept on the symbol's property list.  
The second way is to make each label a Lisp structure.  The type of the
structure is the name of the flavor; thus an object's type can be used to
determine that it is a label of a certain flavor.

If one wishes to have labels be symbols, nothing must be done; this is the
default.  A flavor's labels will be structures only if one of two things
is declared in the {\tt defflavor}.  The first is that the property
{\tt structured} appears.  The second is if another flavor whose instances are
structures is specified to be {\it included} in the new flavor.

When a flavor's labels are to be structures, one will usually wish to
specify the {\tt printfn} property so that the labels will be printed
in a nice way.  This function must be one which is acceptable for use
in a {\tt defstruct}.  It is also required that one specify the slots, or
attributes, the
structures are to have, by including a list of the form
\begin{verbatim}
(instance-attributes (slot1 default1) ... (slotN defaultN))
\end{verbatim}
in the flavor definition.

\subsection{Using Labels}

The function {\tt define-label} is a generic way to create new labels of a
specified flavor.  The function call {\tt (define-label sym flavor-name)}
will do one of two things.  If {\tt flavor-name} is a flavor  whose
labels are symbols, then the property list of {\tt sym} will be updated
with property {\tt FLAVOR} having value {\tt flavor-name}.  If on the 
other hand, {\tt flavor-name} is a flavor having structures for labels,
then {\tt sym} will be setq'd to the value of the result of calling
the constructor function for {\tt flavor-name}, which will create
a structure of type {\tt flavor-name}.

To access the attributes of a label which is a symbol,  use {\tt get},
since all attributes will be on the symbol's property list.  
The attributes of a label which is a structure of type {\tt flavor-name}
can be accessed by using the standard Lisp accessing functions for
structures.  Thus, if one of the label's attributes is {\tt represents},
the attribute can be accessed by calling the function 
{\tt flavor-name-represents}. 

Flavors can be redefined or modified at any time.  This may be done if,
for example, one wished to extend a flavor's definition into a Lisp
package which was not always loaded.  Merely put another {\tt defflavor}
statement into the code.  You need only put the new or changed 
properties in the redefinition.  If, however, you wish to change the
attributes of a flavor which is a structure, you should put in all
of the attributes you desire, not just the new ones, and be sure
to declare any included flavor as well.  Note: it is possible to
change a flavor which uses symbols as labels into one which uses
structures, but if you fail to redefine code which depends on 
property lists, the program will be smashed to flinders.


\subsection{Inheritance and Subflavors}

Some flavors may be similar in many ways; in fact, some flavors may be
more specialized versions of other flavors.  One may wish a new flavor's
labels to be operated upon by most wffops in the same way as an existing
flavor's labels; this we will call inheritance of properties.  In addition,
one may wish a new flavor to actually be a subtype (in Lisp terms)
of an existing flavor, and have the attributes of the existing flavor's 
labels be included in the attributes of the new flavor's labels; this we
will call inclusion of attributes.   The {\tt defflavor} form allows either
or both types of sharing to be used.

Inheritance of properties is signalled in the {\tt defflavor} by a form such
as {\tt (inherit-properties {\it existing-flavor1} ... {\it existing-flavorN})}.
This will cause the properties in the {\tt wffop-hash-table} of the 
existing flavors to be placed into the {\tt wffop-hash-table} of the new
flavor. If any conflict of properties occurs, e.g., if {\it existing-flavorI}
and  {\it existing-flavorJ}, I < J, have a property with the same name, then
the value which {\it existing-flavorJ} has for that property will be 
the one inherited by the new flavor. A new flavor may inherit properties 
from any number of existing flavors.

In contrast, attributes may be included from only one other flavor.  This
can be done by using the form {\tt (include {\it existing-flavor})}. The 
existing flavor must be a flavor whose instances are structures, and the
new flavor's instances will also be structures whose slots include the
attributes of the existing flavor.  Thus the same accessing functions for
those slots will work on labels of both flavors.  To define default
values for those slots, add them to the {\tt include} form as if it were
an {\tt :include} specifier to a {\tt defstruct}; e.g., 
{\tt (include {\it existing-flavor} ({\it slot1 default1}))}.

\subsection{Examples}

Here are some examples of flavor definitions.

%\begin{lispcode}
\begin{verbatim}
(defflavor etree
  (mhelp "Defines common properties of expansion tree nodes.")
  (structured t)
  (instance-attributes
   (name '|| :type symbol)
   components				; a node's children
   (positive nil )			; true if node is positive in the
					;formula
   (junctive nil :type symbol)		; whether node acts as neutral,
					;conjunction, or disjunction
   free-vars				; expansion variables in whose scope
					; node occurs,
					; used for skolemizing
   parent				; parent of the node
   ;;to keep track of nodes from which this node originated when copying a
   ;;subtree
   (predecessor nil)
   (status 1))
   (printfn print-etree)
   (printwff (lambda (wff bracket depth)
	       (if print-nodenames (pp-symbol-space (etree-name wff))
		 (printwff 
		  (if print-deep (get-deep wff)
		    (get-shallow wff))
		  bracket depth))))
   ...many more properties...)
\end{verbatim}
%\end{lispcode}

{\tt \indexData{Etree}} labels will be structures, with several attributes.
The function used to print them will be {\tt print-etree}.

%\begin{lispcode}
\begin{verbatim}
(defflavor leaf
  (mhelp "A leaf label stands for a leaf node of an etree.")
  (inherit-properties etree)
  (instance-attributes
   shallow)
  (include etree (name (intern-str (create-namestring leaf-name))))))
\end{verbatim}
%\end{lispcode}

{\tt \indexData{Leaf}} labels will also be structures, with attributes including
those of {\tt etree}, as well as a new one called {\tt shallow}.  Note that
the {\tt name} attribute is given a default in the {\tt include} form.
{\tt Leaf} inherits all of the properties of {\tt etree}, including, for
example, its print function, unless they are explicitly redefined in
the definition of {\tt leaf}.

\chapter{Printing and Reading Well-formed formulas}
\section{Parsing}

Frank has implemented a type inference mechanism based on an algorithm by
Milner as modified by Dan Leivant.
Type inference is very local: The same variable, say "x" will
get different type variables assigned, when used in different formulas.
Since multiple use of names with different types is rare, the default
could be changed, so that after the first occurrence of an "x" during
a session {core image}, the type inferred the first time is remembered.

There are only a total of 26 type variables, so you may run out during
a session.  The function INITTYPES reset the way type variables are
assigned and treats everything except O and I as type variables.
Normally, a type variable once mentioned or assigned automatically
becomes a type constant.

If \indexflag{TYPE-IOTA-MODE} is {\tt NIL}, then 
TPS will assign type variables starting with Z and going backwards, as 
more are needed. \indexflag{TYPE-IOTA-MODE} defaults to {\tt T}.

Polymorphic abbreviations like SUBSET now may be given a type, so as to fix
the type of other variables.  E.g. the following is legal:
   "FORALL x . P x IMPLIES [Q x] IMPLIES . P SUBSET(O(OC)(OC)) Q"
Note that "x" will be typed "C" (Gamma).  The same typing could have been
achieved by
   "FORALL x(C) . P x IMPLIES [Q x] IMPLIES . P SUBSET Q"
If all the types were omitted and \indexflag{TYPE-IOTA-MODE} were {\tt NIL}, 
"x" would have been typed with the next available typevariable.

Using the same name for two variables of distinct type is legal, but not
recommended.  Consider, for example,
   "FORALL x . P x(I) AND . Q . x(II) a"
Here the type of the very first occurrence of "x" will be assumed as "II".
Leaving out the type of the third occurrence of "x" would have led to an
error message:  Rather than assume that "x(II)" was really meant, TPS
assumes instead that the scoping must have been incorrect, which seems much 
more likely.

All remaining type variables (after a parse) are automatically assumed
to be of {\it base-type} unless the flag {\tt TYPE-IOTA-MODE} is set in which
case they are assumed to be of type $\greeki$. In first-order mode identifiers have
only single characters (Thus "not Pxy" is parsed as "NOT . P x y").

When a wff is read in and parsed, each input token (where the number
of characters in a token is dependent on whether you are reading in
first-order-mode or not) is made into a lisp symbol which incorporates
the token's printed representation and type.  For example, entering
"x(A)" will result in a symbol being created whose print-name is
"x<A>".  When you try to print a symbol like this, first the part
without the type information is printed, then the type (if necessary)
is printed.  E.g., first we print "x", then print "(A)".  But the
information necessary to print "x" is really on the property list of
the symbol whose print-name is "x".  So all wffs of the form "x<...>"
will be printed the same way (except for the type).

So, if you enter "x1(A)", you get the symbol "x1<A>", but no
information about a superscript is put on the symbol "x1".  Thus when
you print it, you get no superscript, just "x1".  Where do
superscripts come from, then?  Well, when TPS renames a variable in
order to get a new one (such as alpha-normalizing a wff), it puts the
superscript information on the new symbol's property list.  I.e., if
we rename "x1<A>", we may get the symbol "x2<A>", and on the property
list of "x2", we get the superscript information.  Thus, the next time
the user types in "x2(A)" or even "x2(I)", the symbols created will
have the superscript information.

This can be a little confusing, because the "x1(A)" that you
originally entered still isn't superscripted, but the renamed
variables "x2", "x3", etc., will be.

\section{Printing of formulas}

\subsection{The Basics}

In this section we will talk about how a formula in internal representation
is printed on different output devices.  There are two main points
to take into consideration: how will the parts of the formula appear,
and where will they appear.  For the latter refer to section 
~\ref{Pretty-Printing}, the former we will discuss now.

\subsection{Prefix and Infix}

Since we deal with formulas of type theory, we can regard every formula
as built by application and $\lambda$-abstraction from a few primitives.
In order to make formulas more legible and closer to the form usually
used to represent formulas from first order logic, we furthermore have
quantification and definitions
internally, and quantification, definitions, and infix operators
for the purpose of input and output.

The application of a function to an argument is printed by simply
juxtaposing the function and its argument.  As customary in type
theory, we do not have an explicit notation for functions of more
than one argument.  Predicates are represented as functions with truth
values as their codomain.

Infix operators have to be declared as such.
Only conjunction, disjunction, and implication are automatically
declared to be infix operators.  In general,
infix operators
will be associated to the left, if  explicit brackets are missing.
For example
\begin{Example}
$A \land B \land C$    will be  $[[A \land B] \land C]$
\end{Example}
Internally every infix operator has a property \indexProperty{Infix}
which is a number. This number is the relative binding strength of
this infix operator. You will have to specify it, if you define a
new connective to be infix. The higher the priority, the stronger
the binding. As usual, `$\land$' binds stronger than `$\lor$' which has precedence
over `$\limplies$' (implication).

(As an aside, if you don't want conjunctions bound more tightly than disjunctions,
but want brackets to appear, make the {\tt INFIX} property of {\tt OR} the same
as {\tt AND}. Thus, do: {\tt (GET 'AND 'INFIX)}, to find it is 5, and then 
{\tt (PUTPROP 'OR 5 'INFIX)})

Unfortunately prefix operators like negation, do not currently  have
a binding strength associated with them and will always be associated
to the left.
This has to be kept in mind, when formulas are typed in.

Definitions can be infix or prefix and the same rules hold for them.
There are flags which control whether a definition or its instantiation
will be printed. Similarly, logical atoms can appear as names or as
values (or both).  In general the appearance of a formula and in particular
of a definition very much depends on which output device is used.  See
section ~\ref{Styles and Fonts} for more detail, but remember that this
only affects the way the primitive or defined symbols appear, but not
how the formula is assembled from its parts.

\subsection{Parameters and Flags}\label{Printing Flags}
The flags listed below are global parameters which can be set by the user
to control the way formulas are printed. These settings can be overridden
if specific commands are given.

\begin{description}

\item [\indexflag{PrintTypes}] = {\tt T} causes all types to be printed. 
	If a typed symbol occurs more than once, only the first occurrence
	will have a type symbol, unless the same symbol name appears in the
	same formula with a different type.

 = {\tt NIL} suppresses type symbols.

\item [\indexparameter{PrintDepth}] This is a parameter which determines how deep the recursion
	which prints the formula will go.  Subformulas located at a lower
	level will simply be replaced by an \&.  A {\tt PrintDepth} of 0 means
	that everything will be printed, regardless of its depth.  {\tt PrintDepth}
	has to be an integer. It is initialized to 0. The most useful
	application of this parameter is in the formula-editor, where one
	usually does not like to see the whole formula.

\item [\indexflag{AtomValFlag}] This flag should usually not be touched by the user. If it
	is true, under each atom its value will appear.

\item [\indexflag{AllScopeFlag}] This flag should be {\tt NIL} most of the time. If it is
	{\tt T} brackets and dots will always be inserted, i.e. no
	convention of associativity to the left is followed. The
	precedence values of infix operators are also ignored.
	It can be forced to {\tt T} by calling the function
	{\w \indexmexpr{PWScope GWff}}.
\end{description}

\subsection{Functions available}\label{Printing Functions}

There are of course a variety of occasions to print wffs, For example in plans,
as lines, after the {\tt P} or {\tt PP} -command in the editor etc.
Associated with these are different printing commands given 
by the user. Some of these commands override globally set parameters or
flags. Internally, however, there is only one function which prints wffs.
This function \indexfunction{PrtWff} is called whenever formulas have to be 
printed.  The various flags controlling the way printed formulas will appear,
will either be defaulted to the global value, or be passed to this
function as arguments. The general form of a call of {\tt PrtWff} is as follows

{\tt 
(PrtWff Wff {(Parameter$_1$ Value$_1$)} ... {(Parameter$_n$ Value$_n$)} )
}

Before the actual printing is done Parameter$_1$... Parameter$_n$ will be
set to Value$_1$ ... Value$_n$, resp. If a parameter of the following list
is not included in the call of the function, its global value will be
assumed. Possible parameters with their range and the section they are
explained in are

% @Tabdivide(3)
\begin{tabular}{lll}
\indexflag{PrintTypes} & T,NIL & ~\ref{Printing Flags} \\
\indexparameter{PrintDepth} & 0,1, ... & ~\ref{Printing Flags} \\
\indexflag{AllScopeFlag} & T,NIL & ~\ref{Printing Flags} \\
\indexflag{AtomValFlag} & T,NIL & ~\ref{Printing Flags} \\
\indexflag{PPWfflag} & T,NIL & ~\ref{Pretty-Printing Flags} \\
\indexflag{LocalLeftFlag} & T,NIL & ~\ref{Pretty-Printing Flags} \\
\indexflag{FilLineFlag} & T,NIL & ~\ref{Pretty-Printing Flags} \\
\indexflag{FlushLeftFlag} & T,NIL & ~\ref{Pretty-Printing Flags} \\
\indexparameter{Leftmargin} & 1 ... Rightmargin & ~\ref{More Printing Functions} \\
\indexparameter{Rightmargin} & 1, 2 ... & ~\ref{More Printing Functions} \\
\indexparameter{Style} & XTERM, SCRIBE, CONCEPT, \\
 & GENERIC, SAIL, TEX ... & ~\ref{Styles and Fonts} \\
\end{tabular}

\subsection{Styles and Fonts}\label{Styles and Fonts}

\TPS can work with a variety of different output devices, producing special 
characters like $\forall$ or $\land$ where possible, and spelling them out 
(as {\tt FORALL} and {\tt AND}) where not. Details of how to produce output 
files for various purposes are in the \ETPS and User's Manuals.

At no point does the user actually make a commitment whether to work with special
characters or not, since she can easily switch back and forth. The 
internal representation is completely independent of these switches
in the external representation.

A few commands, such as \indexfunction{VPForm} and \indexfunction{VPDiag} 
have an argument \indexparameter{Style} which specifies
the style in which a file is produced. Furthermore there is a flag, \indexflag{STYLE},
which \TPS will use in the absence of any other indication as to the 
appropriate form of output.

Along with the style the user can usually specify an appropriate linelength
by using the \indexflag{LEFTMARGIN} and \indexflag{RIGHTMARGIN} flags.
Some commands (most notably \indexcommand{SETUP-SLIDE-STYLE}) will change 
both the style and the default line length.

\begin{description}
\item [\indexstyle{CONCEPT}, \indexstyle{CONCEPT-S} ] this is the style used for a Concept terminal,
which might also
	occasionally also be useful to produce a file which can be
	displayed on the Concept terminal with {\tt CAT} or {\tt MORE}.
The difference between {\tt CONCEPT} and {\tt CONCEPT-S} is that the latter assumes 
that your Concept is equipped with special characters and the former does not.
If special characters are available,
	you will then get types as greek subscripts,
	the universal quantifier as $\forall$, etc.  The default linelength
	is 80.

\item [\indexstyle{GENERIC} ] this style assumes no special features and defaults the 
	linelength to 80.  For example the existential quantifier shows up as EXISTS
	and types are enclosed in parentheses.

\item [\indexstyle{GENERIC-STRING} ]  is much like {\tt GENERIC}, but prints in a format that 
can be re-read by \tps.

\item [\indexstyle{SCRIBE} ] corresponds to the style used by the Scribe text processor.
	A file produced in this style has to be processed by {\tt SCRIBE}
	before it can be printed. All special characters,
	superscripts and subscripts, etc. are available. The main drawback
	of a {\tt SCRIBE}-file is that precise formatting as necessary
	for vertical path diagrams is impossible. The font used is 10-point, except 
        when doing \indexcommand{SLIDEPROOF}, when an 18-point font is used.

\item [\indexstyle{TEX} ] is the output style used by the \TeX text processor.
A file produced in this style has to be processed by \TeX before it can be printed.
All special characters, superscripts, etc. are available, and vertical path diagrams
are correctly formatted (although often too wide to print). 

\item [\indexstyle{XTERM} ] produces the special characters used by X-windows. You should
set the value of \indexflag{RIGHTMARGIN} to reflect the width of the window containing 
\tps.

\item [\indexstyle{SAIL} ] {\tt SAIL} is a style (now all but obsolete) used for printing
on a Dover printer. The font used is 10-point, with 120 characters 
per line in landscape format (used for vertical path diagrams), and 86 in portrait format
(used for all other applications).
When you dover the file , you have to remember size and orientation and
specify it in the switches of your call of {\tt DOVER}.
A {\tt SAIL} file does not have subscripts, but has as variety
of other special characters.

\end{description}

From the information about the style, the low-level printing functions
determine which sequence of characters, including control characters, to
send to the selected output device.  If a symbol expands to a list of known 
symbols with different names (e.g. \indexData{EQUIVS} expands to an {\tt EQUIV} 
symbol with a superscript {\tt S}), then it has a property \indexProperty{FACE}
which contains this information. Various other properties give the way that the character 
is to be printed in different styles.
The \indexProperty{CFONT} property
is a pair {\tt (KSet . AsciiValue) . \indexData{Kset}} can be 0,1,2, or 3,
although currently only the character sets 0, 1, and 3 are used; this gives the 
appropriate character for a Concept terminal.
Similarly, the \indexProperty{DFONT} property is a string {\tt "whatever"} which will be printed
into Scribe files as {\tt @whatever}. The \indexProperty{TEXNAME} property does the
same for the \TeX output style.
There are some special fonts that are declared in the file
\indexfile{tps.mss}.  A list of the available special characters for
the Concept and for the Dover (in a {\tt SCRIBE}-file) are explicitly stored
in the files \indexfile{cfont.lisp} and \indexfile{dfont.lisp} and loaded into
\TPS at the time the system is being built.

Consider the following example:
\begin{Example}
SIGMA1 is a binder.
It has a property FACE of value (CAPSIGMA SUP1).

CAPSIGMA is a tex special character, a scribe special character, 
and a concept special character.
It has a property CFONT of value (3 .  83).
It has a property DFONT of value "g{S}".
It has a property TEXNAME of value "Sigma".

SUP1 is a tex special character, a scribe special character, 
and a concept special character.
It has a property CFONT of value (1 .  49).
It has a property DFONT of value "+{1}".
It has a property TEXNAME of value "sup1".
\end{Example}

In a scribe or tex file, or on a Concept with special characters, {\tt SIGMA1} will
appear as $\Sigma^1$; elsewhere it will be written as {\tt SIGMA1}. The actual Scribe
output produced will be \begin{verbatim}@g{S}@\;@^{1}@\;\end{verbatim}; the actual \TeX
output will be \begin{verbatim} \Sigma^{1} \end{verbatim}.

\subsection{More about Functions}\label{More Printing Functions}

In this section some more details of the functions which are used to
do the printing are given. 

As mentioned earlier, the main connection with the rest of \TPS is
the MACRO \indexfunction{PrtWff}.  It expands into a {\tt PROG} in which
all the parameters given as arguments are {\tt PROG}-variables.  In the body
of the {\tt PROG}, all parameters are set to the value specified in the call,
then the function \indexfunction{PWff} is called, just with {\tt Wff} as its
argument.  All the other parameters and flags are now global, or, in {\tt LISP}
terminology, special variables.

The function {\tt PWff} performs two main tasks. First a few special variables
are set to the correct value.   After this is done, {\tt PWff} checks whether
pretty-printing is desired, i.e. whether {\tt PPWfflag} is {\tt T}.
For an explanation of what happens during pretty-printing see section 
~\ref{Pretty-Printing} and in particular ~\ref{Pretty-Printing Functions}.
Otherwise the recursive function \indexfunction{PrintWffPlain} is called
with the appropriate arguments.

At this point the current style is available to the functions in the
flag \indexflag{STYLE}. The calling function has to make sure
that \indexflag{LEFTMARGIN} and
\indexflag{RIGHTMARGIN} will be bound.  They are important
for the printing functions in order to determine where to break lines,
and where to start formulas on the line.  This holds, whether pretty-
printing is switched on or off.

Below {\tt PWff} two functions appear.
{\tt PrintWffPlain} prints a formula without any delimiting symbols
around it.  For example (with {\tt STYLE SCRIBE})  
%\begin{Example,Spacing=1.5}
\begin{Example}
((x<I> . FORALL) . ((OR . (P<OI> . x<I>)) . q<O>))  appears as \\
$\forall x_\greeki . [P_{\greeko\greeki} x] \lor q_\greeko$ if BRACKETS = T and as \\
$\forall x_\greeki [[P_{\greeko\greeki} x] \lor q_\greeko]$ if BRACKETS = NIL .
\end{Example}
\indexfunction{PrintWffScope} delimits a composite formula with a preceding
dot, if the argument \indexparameter{BRACKETS} is {\tt T} , and with brackets
around it , if {\tt BRACKETS} is {\tt NIL}.  Other than that the functions are
identical.  In the above example we would get
%\begin{Example,Spacing=1.5}
\begin{Example}
$\forall x_\greeki . [P_{\greeko\greeki} x] \lor q_\greeko$ if Brackets = T and \\
$[\forall x_\greeki . [P_{\greeko\greeki} x] \lor q_\greeko]$ if Brackets = NIL
\end{Example}
Both {\tt PrintWffPlain} and {\tt PrintWffScope} call \indexfunction{PrintWff},
where the real work of distinguishing the different kinds of formulas
and symbols is being done.  The distinction between {\tt PrintWffPlain} and 
{\tt PrintWff} is only made for the sake of pretty-printing (see 
~\ref{Pretty-Printing Functions}).

At an even lower level is the function (actually a macro) \indexfunction{PCALL},
which determines the appropriate way to print a particular symbol in the 
current style, and prints an error if the relevant function is undefined.
{\tt PCALL} actually applies to printing functions, rather than characters, so
each function will have a different definition for different styles. For example,
in style scribe the \indexfunction{print-symbol} function is called \indexfunction{PP-SYMBOL-SCRIBE}, 
whereas in style xterm it's called \indexfunction{PP-SYMBOL-XTERM}. (Examine
the plists of {\tt SCRIBE} and {\tt XTERM} to verify this, if you like.)

\section{Pretty-Printing of Formulas}\label{Pretty-Printing}

The most commonly used way of printing formulas, such as lines or plans,
is to pretty-print them. This is a feature quite similar to the way LISP
pretty-prints functions. Formulas which are too long to fit on one line
of the current output device, are broken at the main connective and printed
in several lines. The main difference to the LISP pretty-printing is that
we have to consider infix operators.

The general structure of the functions doing the pretty-printing allows 
future changes to the way printing in general is done without making changes
to the pretty-printer.  Whenever a formula is to be pretty-printed the
usual printing functions as described above are called, but instead of
printing the characters, they will be appended to a list.  Later this list
is used to actually output the characters after the decision where to break
the formula has been made.  From this structure it is clear that all the
parameters and flags controlling the appearance of a formula on the several
printing devices still work in the way described before.  There are however,
a few additional flags which determine how subformulas will be arranged
within a line.

\subsection{Parameters and Flags} \label{Pretty-Printing Flags}

As new flags particularly for pretty-printing we have
\begin{description}

\item [\indexflag{PPWfflag} ] = {\tt T}  means that formulas will usually be pretty printed. 
This is the default value.

 = {\tt NIL} 
means that formulas never will be pretty printed unless the
command is given explicitly.

\item [\indexflag{LocalLeftFlag} ] ={\tt T} 
will cause the left hand side of an infix expression
to be aligned with the operator and not with the right hand side.

 = {\tt NIL} 
is the default and prints left and right hand side of an
infix expression with the same indentation.

\item [\indexflag{FilLineFlag} ] ={\tt T} 
will try to fill a line as much as possible before
starting a new one. This only makes a difference for associative
infix operators.

 = {\tt NIL} starts a new line for each of the arguments of an infix operator
even if only one of several arguments would be too long to fit on the 
remainder of the line.

\item [\indexflag{FlushleftFlag} ] = {\tt T} switches off indentation completely, i. e. every line will
be aligned with the left margin.

 = {\tt NIL}  indents the arguments of infix operators.
\end{description}

\subsection{Creating the PPlist} \index{PPlist} \label{PPlist}

The pretty-printing is achieved in two steps. During the first phase
printing will be done without any formatting and the characters are
not actually printed, but appended to a list, called {\tt PPlist}. In the second
phase, this list will then be printed. The decisions, when to start
a new line, how to indent etc. are only made in this second stage.

The {\tt PPlist} is of the following syntactical structure.

\begin{description}

\item [\indexData{pplist} ::=]  ((aplicnlist . (pdepth . pgroup)) . plength) 
           \\ | ((gencharlist . (pdepth . pgroup)) . plength)

\item [\indexData{aplicnlist} ::=]  (aplicn . aplicnlist) | {\tt NIL} | (aplicn . {\tt MARKATOM})

\item [\indexData{aplicn} ::=]  (pplist . pplist)

\item [\indexData{plength} ::=]  {\tt 0} | {\tt 1} | {\tt 2} | ...

\item [\indexData{pgroup} ::=]  {\tt BRACKETS} | {\tt DOT} | {\tt NIL}

\item [\indexData{pdepth} ::=]  {\tt 0} | {\tt 1} | {\tt 2} | ...

\item [\indexData{gencharlist} ::=]  (genchar . gencharlist) | {\tt NIL}

\item [\indexData{genchar} ::=]  char | (ascnumber) | (gencharlist)

\item [\indexData{char} ::=]  <any non-control character>

\item [\indexData{ascnumber} ::=]  {\tt 0} | {\tt 1} | ... | {\tt 127}

\end{description}

The {\tt PPlist} contains a list of all the top-level applications , along with
the grouping (pgroup),
its print-depth (pdepth) and its print-length (plength).
If the grouping is {\tt BRACKETS} brackets will be printed around the formula.
A grouping {\tt DOT} means that
a dot will precede the formula, otherwise the formula will just be printed
without any delimiting symbols. The plength is the total
length of the formula if printed in one line, including spaces, brackets,
a.s.o., but not control characters which are used to denote character
sets, or {\tt SCRIBE} -commands.

The pdepth is recursively defined as the maximum pdepth of the left-hand sides
plus the maximum pdepth of the right-hand sides of the
applications, if the {\tt PPlist} contains applications, and the plength
of the generalized-character list (gencharlist) otherwise.
The plength of a gencharlist is 
its length after all members of the form
`(gencharlist)' have been deleted.  This means that characters that have to
be sent to the selected output device but do not occupy space (in the final
document) will simply be enclosed in parentheses.  By this convention
the function which then formats and actually prints the formula from the 
{\tt PPlist} can keep track of the vertical position within a line.  The 
pdepth associated with each subformula is used to decide the amount of
indentation, as described below.

The list of applications, aplicnlist, typically contains 
contains only one pair with the left-hand side
a function, and the right-hand side the argument the function is
applied to.  In case we have infix operators or multiple conjunctions
or disjunctions, like $A \equiv B$, 
$A \land B \land C \land D$, or $E \lor F$, aplicnlist will contain a 
different pair for each argument.  The left-hand side contains the
infix operator, if one has to be printed in front of the argument, the
right-hand side contains the argument itself.  Quantifiers are regarded as
single applications, where the left-hand side is the quantifier plus the
quantified variable, while the right-hand side is its scope.
Consider the following examples.
\begin{Example}
$A \equiv B$ \\
will be translated to\\
aplicnlist  =  ( (<> . <A>) (<EQUIV> . <B>) )\\
\\
$A \land B \land C \land D$\\
will be translated to \\
aplicnlist  =  ( (<> . <A>) (<AND> . <B>) (<AND> . <C>)
                 (<AND> . <D>) )\\
\\
$E \lor F$\\
will be translated to\\
aplicnlist  =  ( (<> . <E>) (<OR> . <F>) )\\
\\
$\forall x_\greeki G$ \\
will be translated to \\
aplicnlist  =  ( (<FORALL X<I>> . <G>) ) \\
\\
where <x> denotes the PPlist corresponding to the subformula $x$,\\
and <> stands for the empty PPlist ((NIL . (0 . NIL)) . 0)
\end{Example}

A generalized character, genchar, is defined to be an arbitrary
non-control ASCII character, the number of an ASCII character in parentheses,
or another generalized character list in double parentheses.  When an ASCII
character is
printed it is assumed that the cursor advances one position, while
everything in the sub-gencharlist is assumed not to appear
on the screen or in the document after being processed by SCRIBE.

An aplicnlist with the structure (aplicn .  \indexData{{\tt MARKATOM}})
signals that the aplicn is the internal representation of a logical
atom (For example {\tt ATM15}).  In case AtomValFlag is {\tt T}, the program notes the
cursor position, whenever it encounters such an aplicnlist during
printing and prints the name of the atom in the next line at this
position.

\subsection{Printing the PPlist} \index{PPlist}
After the {\tt PPlist} is created by the function \indexfunction{PWff}, the actual
output is done by the function \indexfunction{PrintPPlist}.  This function
takes a {\tt PPlist} and {\tt INDENT} as arguments and has the following
basic structure.

\begin{description}
\item [(1) ] Does the formula fit on the remainder of the line
(from {\tt INDENT} to {\tt RightMargin}) ?
If yes, just print it from the {\tt PPlist}.
If not, go to (2).

\item [(2) ] Is the formula composed of subformulas ?
If not, go to the next line and print it at the very right.
If yes, go to (3).

\item [(3) ] Is the formula a single application ?
If yes, call {\tt PrintPPlist} recursively, first with the function
then with the argument such that the function will appear at {\tt INDENT} and
the argument right after the function.
If not, go to (4).

\item [(4) ] Print each application in the application list in a new line,
the operators at the vertical position {\tt INDENT} and the
arguments at the position {\tt INDENT} + maximal length of the operators.
\end{description}

This algorithm will be slightly different if the flags described above
do not have their default values. See section ~\ref{Pretty-Printing Flags} for a description.

Some heuristics are employed to avoid the pathological case where the
formula appears mostly in the rightmost 10\% of each line.  Used in these
heuristics is the print-depth (pdepth), which is equal to the furthest
extension of the formula to the right if printed with the above algorithm.
Whenever the pdepth is greater than the remainder of the line, the indentation
will be minimized to two spaces. This is most useful if special
characters are not available, for example if `$\forall$' is printed as `FORALL'.

\subsection{Pretty-Printing Functions}\label{Pretty-Printing Functions}

Most of the functions used for the first phase of pretty-printing, i.e. for
building the {\tt PPlist} are already described in section ~\ref{More Printing Functions}.
The internal flag {\tt PPVirtFlag} controls whether functions like {\tt PrintFnTTY}
will actually produce output or create a {\tt PPlist}.  Here it is now of
importance, what the different printing functions return, something that was 
completely irrelevant for direct printing.

The general schema can be described as follows.  \indexfunction{PrintWffPlain} and
\indexfunction{PrintWffScope} return a {\tt PPlist}.  If called from \indexfunction{PrintWff}, these
{\tt PPlists} are assembled to an aplicnlist and returned.  In this case {\tt PrintWff}
returns an aplicnlist.  The lower level functions, \indexfunction{PrintFnDover}
and \indexfunction{PrintFnTTY} return the gencharlist which contains the
characters that would be printed in direct mode.  Note that therefore {\tt PrintWff}
will sometimes return a gencharlist instead of an aplicnlist.  These
two are interchangeable as far as the definition of the {\tt PPlist} is
concerned, and can hence be treated identically by {\tt PrintWffPlain}
which constructs a {\tt PPlist} from them.  

The special parameters \indexparameter{PPWfflist} and \indexparameter{PPWfflength}
keep track of the characters "virtually printed" and the length of
the formula "virtually printed", respectively.

On the very lowest level \indexfunction{PPrinc} and \indexfunction{PPTyo} perform a {\tt PRINC}
or {\tt TYO} virtually by appending the appropriate characters to the
{\tt PPWfflist}.  Characters that do not appear in the final document
or on the screen, are virtually printed by \indexfunction{PPrinc0} and \indexfunction{PPTyo0}.
They prevent the counter {\tt PPWfflength} from being incremented.  Similar
functions are \indexfunction{PP-Enter-Kset} and \indexfunction{PPTyos} which correspond to
{\tt Enter-Kset} and {\tt TYOS}.

In the second phase of pretty-printing as described in the previous section
{\tt PrintPPlist} is the main function.  If the remainder of a {\tt PPlist} fits
on the rest of the current line, \indexfunction{SPrintPPlist} is called which just 
prints the {\tt PPlist} without any counting or formatting.

\subsection{JForms and Descr-JForms}\index{JForm}\index{Descr-JForm}

A JForm is an alternative way of representing well-formed formulas and
is used by the matingsearch package and for printing vertical path
diagrams. In JForms multiple conjunction are represented as lists and
not as trees. Consider the following example.
\begin{Example}
$A \land B \land C \land [D \lor E \lor F]$
\end{Example}
As a wff in internal representation this will be
\begin{Example}
( (AND . ((AND .((AND . A) . B)) . C))
  . ((OR . ((OR . D) . E)) . F))
\end{Example}
Obviously this is not a very suitable form for vertical path
diagrams.  As a JForm, however, the above wff would read as 
\begin{Example}
(AND A B C (OR D E F))
\end{Example}
which is already close to what we would like to see.

The function \indexfunction{Describe-VPForm} takes a JForm like the one
above as an argument and returns a Descr-JForm, where we have the
information about the height and width of the subformulas, which we
need in order to format the output, explicitly attached to the parts of
the JForm.

Quantifiers are handled similarly. Multiple identical quantifiers
are combined in a list whose first element is the quantifier and the
rest is the list of variables which are quantified.
\begin{Example}
$\forall x \forall y \exists z \exists u A$
\end{Example}
is in internal representation
\begin{Example}
((x . FORALL) . ((y . FORALL)
       . ((z . EXISTS) . ((u . EXISTS) . A)))),
\end{Example}
and as a JForm it looks like
\begin{Example}
((FORALL x y) ((EXISTS z u) A)) .
\end{Example}
The following is a formal description of what a JForm and a Descr-JForm
are. Note that a descr-jform is entirely an internal concept, used by the
file \indexfile{vpforms.lisp} for working out how to format a vpform; a 
jform is a concept which is accessible to users (e.g. users have commands
to translate from gwffs to jforms and back)

\begin{description} % Description,Spacing=1.5}
\item [\indexData{\it JForm} ::=] {\it Literal} | {\it SignAtom} | ({\tt OR} [{\it JForm}]$^n_2$)
	|({\tt AND}  [JForm]$^n_2$ )
	\\ | (({\tt FORALL}  [Var]$_1^n$) JForm)
	| (({\tt EXISTS}  [Var]$_1^n$) JForm)

\item [\indexData{\it Literal} ::=] {\tt LIT1 | LIT2 | ...}

\item [\indexData{\it SignAtom} ::=] ({\tt NOT} {\it Atom}) | ({\it Atom})

\item [\indexData{\it Var} ::=] < any logical variable >

\item [\indexData{\it Atom} ::=] < any logical atom >
\end{description}
It should be noted here that some programs might expect the arguments
of a JForm starting with {\tt OR} not to start itself with an {\tt OR},
the argument of a JForm starting with {\tt FORALL} not to start with another {\tt FORALL}
etc., but this is by no means essential for vertical path diagrams.

%\begin{Description,Spacing=1.5}
\begin{description}
\item [\indexData{Desc-Jform} ::=] (\{$^{Literal}_{SignAtom}$\} Height Width (Width Width) (GenCharList PPlist)
	\\ (({\tt OR} [Desc-JForm]$^n_2$ )
	 Height Width ([Cols]$^n_2$ ))
	\\ (({\tt AND} [Desc-JForm]$^n_2$ )
  	 Height Width ([Rows]$^n_2$ ))
	\\ (((\{$^{\tt FORALL}_{\tt EXISTS}$\} [Var]$^n_1$)
	 Desc-JForm) Height Width Width GenCharList)

\item [\indexData{Height} ::=] {\tt 0} | {\tt 1} | {\tt 2} | ...

\item [\indexData{Width} ::=] {\tt 0} | {\tt 1} | {\tt 2} | ...

\item [\indexData{Cols} ::=] {\tt 0} | {\tt 1} | {\tt 2} | ...

\item [\indexData{Rows} ::=] {\tt 0} | {\tt 1} | {\tt 2} | ...
\end{description}
In a Descr-JForm the second and third element (Height and Width) contain
the height and width of the JForm that is described by the Descr-JForm.
In case the JForm was a literal or a signed atom the next two elements
are lists. The left element of each of these sublists
gives the width or print-representation
of the literal or atom, the right element gives the width or print-representation
of the literal's or atom's value.

If the JForm was a conjunction or disjunction, the last element of the 
corresponding Descr-JForm is a list of the rows or columns in which the
conjuncts or disjuncts begin.

If we deal with a top-level quantifier in our JForm, the last two
elements contain the width and the print-representation of the
quantifier together with the quantified variables.  For a description of
a GenCharList or PPlist see section ~\ref{PPlist}.

\subsection{Some Functions}

The function which is called by {\tt VPForm} and {\tt VPDiag} is
\indexfunction{\%VPForm}.  The handling of the comment and the different
files that have to be opened is done here.
The main function which translates a JForm into
a Descr-JForm is
\indexfunction{Describe-VPForm}.  SignAtoms and Literals are described by
\indexfunction{Describe-VPAtom} and \indexfunction{Describe-VPLit} , respectively.  The
virtual printing functions used for this process are \indexfunction{FlatSym}
and \indexfunction{FlatWff}.

{\tt FlatSym} takes an arbitrary {\tt LISP} identifier as an argument and
returns a pair (gencharlist . length) for this identifier.
{\tt FlatWff} takes a wff as argument and returns a {\tt PPlist} for
it.

The main function which then prints the Descr-JForm is \indexfunction{Print}-VPForm.
It takes the line of the Descr-JForm which should be printed as an
additional argument.  On lower levels \indexfunction{\%SPrintAplicn} and
\indexfunction{\%SPrintPPlist} print an aplicn or a {\tt PPlist} much in the same fashion
{\tt SPrintAplicn} and {\tt SPrintPPlist} do, except that \indexfunction{\%\%PRINC} takes
the role of {\tt PRINC} and {\tt TYO}.  This is necessary from the way
the actual output is handled.  If the vertical path diagram does
not fit on one page, several temporary files are opened and each
file contains the information for one of the pages.  This means 
that the characters have to be counted and a new file to be selected
as the current ouput file, whenever the character count exceeds
the global parameter \indexparameter{VPFPage}.  The counting
as well as the change of the current output file is done by the 
function \indexfunction{\%\%PRINC}.  The argument has to be either a {\tt LISP}-atom,
in which case it will be {\tt PRINC}'ed , or a single element list, 
in which case this element will be {\tt TYO}'ed.


\section{How to speed up pretty-printing (a bit)}

Pretty printing in TPS or ETPS is slow, for various reasons.  One of
them is the tremendous amount of temporary list space used, which takes
time and more time through garbage collection.  Another is the forgetfulness
of the printing routine which recomputes length and other information
over and over again.  Below we will try to explore ways to improve
the performance of the pretty printer without sacrificing any of the niceness
of the output.

Let us recount which factors make pretty-printing wffs more difficult than
pretty-printing Lisp S-Expressions.  For once, Lisp does not have infix
operators and can therefore get by with a significantly smaller amount of
lookahead.  Moreover, the lookahead can be done during the printing, where
the extra time delay is hardly noticeable, while TPS' lookahead must
all be done ahead of time, before the first character is printed.  Secondly,
Lisp does not deal with a variety of output devices, which makes counting
symbol lengths as well as printing symbols much faster and more transparent.

The result of a first attempt at pretty-printing is described earlier in this
chapter.  The solution is nicely recursive and a lot of information is made
available for deciding where to break and how to indent lines.  It is a sad
fact that the algorithm does not reuse any information whatsoever.  For example,
the printed representation of identifiers is recomputed over and over again.
Even worse, the characters comprising the printed representation of an
identifier are stored in a list, copies of which typically occur
in many places in the {\it pplist} of a single wff.

Let us now look at some of the problems and possible solutions of the
pretty-printing problem.

\subsection{Static and Dynamic Parameters}
Crucial to finding a good solution is to understand which factors affect
the appearance of wffs when printed.  These can be divided into two
classes.
\begin{description}
\item {\it Static Parameters}.  Static parameters are not changed during the printing
of a given wff.  In particular their values are identical for a wff
and their subformulas.  Of course, they may be changed from one printing
task to another, but not within printing a particular wff.  Examples
of such static parameters are {\tt AllScopeFlag}, {\tt Style}, {\tt KsetsAvailable},
{\tt PrintAtomnames}, etc.  One other characteristic of static parameters
is that one frequently would like to (and sometimes does) expand the
number of static parameters.

\item {\it Dynamic Parameters}.  Dynamic parameters are the ones which change from
a wff to a subwff.  They are highly context-dependent and are often
not explicitly available as flags, but implicitly computed.  Examples
of such parameters are ``{\it should I print a type for this identifier?}'',
{\tt PrintDepth}, ``{\it should I print brackets or a dot?}''.  An example
for the last question would be that we can sometimes write
$Q_{\greeko\greeki} . f_{\greeki\greeki} \; x_\greeki$ and sometimes
$Q_{\greeko\greeki} [f_{\greeki\greeki} \; x_\greeki]$ depending on
the brackets in wff containing this as a subformula.
\end{description}

One can easily see that static parameters can be handled fairly easily,
while dynamic parameters can become a headache if we are trying to save
information about the appearance of wffs and symbols.

\subsection{A grand solution, and why it fails}

A first stab at a solution could be briefly described as follows:

During the printing of a wff we permanently attach relevant printing
information like length, depth, or printing characters to each label and symbol
in the wff.  When the label or symbol appears again somewhere else, the
information does not have to be recomputed.

We would then have to somehow code the information about the current
static and dynamic parameters into the property of the label or symbol
which stores this information.

With the aid of a hashing function this is straightforward for the
static parameters, since we can compute the name of the relevant property
once and for all for the printing of a wff.  For dynamic parameters
this is still in theory possible, but in practice unfeasible.  We would
have to recompute (rehash) the values of the dynamic and static parameters
for each subformula.  To see that this is very difficult, if not impossible,
consider the following example.

The simple wff $P_{\greeko\greeka\greeka} \; x_\greeka \; y_\greeka$
may
appear as $Pxy$, $P_{\greeko\greeka\greeka}\; x y$,
$Px_\greeka\;y_\greeka$,
$Pxy_\greeka$, etc., with almost endless
possibilities for larger wffs.  All the information about which symbols should
have types etc. would have to be coded into the property name for, say, the
printing length of a label.

This clearly demonstrates that a grand solution is infeasible.

\subsection{A modest solution, and why it works}
Everything would work out fine if we could limit the number of dynamic
parameters.  This can be achieved very simply by restricting ourselves
to saving information about symbols only, and not about labels in general.

Of the various dynamic parameters, only one survives this cut. ``{\it Do I
put a type on this identifier}'' is the only question that can be solved
from the context only.  This simplification also reduces the number of
static parameters, For example {\it AllScopeFlag} is irrelevant to the printing
of symbols (wffs without proper subwffs).

However, care must be taken when the appearance of identifiers is changed. 
We will return to this problem later in the section about other issues.

\subsection{Implementation}
All printing requests go through the function {\tt PWFF}.  When {\tt PWFF} is
entered all static parameters have their final value.  Inside {\tt PWFF}
we will set two more special (global) variables: {\tt Hash-Notype} and
{\tt Hash-Type}.

{\tt Hash-Type} and {\tt Hash-Notype} will have as value of the name of the
property, which contains the symbol's {\it pplist}.  When constructing
the {\it pplist} for the given wff (the first pass during pretty-printing),
it is checked whether symbols have the appropriate property.  If yes,
the symbol itself stands for a {\it pplist}. (We are thus modifying the recursive
definition of {\it pplist}.)  If not, the {\it pplist} will be computed and
stored under the appropriate name on the property list of the symbol.
In this case, too, the symbol itself will appear in the {\it pplist}.

During the actual printing phase of the {\it pplist}, the necessary information
about symbols is retrieved from the property lists of the identifiers.

This presents one additional problem:  we have to preserve the
information about the dynamic parameters in the {\it pplist} itself, so
that the correct property can be accessed.  This could be done  in
a very general way (but for specific problems maybe wasteful way)
namely by including the name of the relevant property in the
{\it pplist}.  Alternatively we may use the special circumstance that
there are usually more identifiers without type.  We would then only
mark those identifiers with type, while all others are assumed to be
printed without types.

The solution above requires some auxiliary data structures.  There
should be a global variable, say {\tt static-printing-flags}, which
contains a list of all flags affecting the printing of symbols.
Then there must be a function {\tt hash-printing-flags} which takes
one argument (signifying whether types are to be printed) and returns
an identifier coding the value of the {\tt static-printing-flags} and
the argument.

\subsection{Other Issues}
In the solution proposed above it is left open, whether the actual
ASCII character representation of a symbol should be computed once and
for all (for each set of static and dynamic parameters) and saved in a
list which is part of the {\it pplist}, or simply recomputed every time
the identifier is printed.  The first solution would require significantly
more permanently occupied list space, the second solution would take
more time during each printing.

Notice, that the time required for the printing is not that long, since
the identifier will have to be printed only during the actual printing
phase, not during the virtual printing phase.  The length is already
known through the symbols property list.  It therefore seems to be much
better only to save the printing length of the identifier.

Another issue arises, when we allow that the printing appearance of
identifiers be changed.  Since all
the length information attached to the identifier will be wrong, it
is necessary to remove that information.  In order to be able to do
this, we need to recognize the properties which stem from the printing
algorithm sketched above. The simplest way to achieve this is to
declare a global variable {\tt hash-properties}, which is a list
of all the properties that have been used for printing so far.  This
must be updated, whenever {\tt PWFF} is called.  The hope is that due to
the limited number of static and dynamic parameters this list remains
manageable in size.  An alternative would be to write the hashing
function in such a way that all names produced by it start with
a unique pattern, say {\tt *@*}.  One can then systematically
look for properties whose name starts with {\tt *@*}.

\subsection{How to save more in special cases}
There is a straightforward generalization of this to case where we would
like to save information about the appearance of arbitrary labels.  The most
general solution fails, as demonstrated above, but if we restrict ourselves
to cases where the number of dynamic parameters is limited, we can get
somewhere.

We could make a case distinction of the kind:  save and use printing
info for labels only if {\tt PrintDepth} is {\tt 0}, {\tt PrintTypes} is {\tt NIL},
{\tt AllScopeFlag} is {\tt NIL}.  The only remaining dynamic parameter that
comes to mind is the bracketing information (which can take two different
values).  This is what makes this fragment of the grand solution
feasible.

Notice that this is not just of academic interest.  ETPS in first-order
mode satisfies all the criteria above.

\section{Entering and printing formulas}

\subsection{Parsing of Wffs}
Wffs can be specified in \TPS in a variety of ways, e.g. as strings
and with or without special characters.  Regardless how
a wff is specified there are general rules of syntax which always apply.
Sometimes one has to distinguish between first-order mode and higher-order
mode with slightly different syntactic rules.  If the global variable
\indexflag{First-Order-Mode} is {\tt T}, all parsing will be done
in first-order mode.  Similarly, the global variable
\indexflag{First-Order-Print-Mode} determines whether wffs are
printed as first-order or higher-order formulas.  It is important
to note that wffs printed in higher-order mode can only be parsed in 
higher-order mode, and formulas  printed in first-order mode can only be
parsed in first-order mode.

\begin{itemize} %, Spacing=1.5}
\item {\bf Operator precedence} - The parser for wffs is a standard operator
precedence parser.  The binding priority of an infix or prefix operator
is a simple integer and conforms with the usual conventions on how
to restore brackets in formulas.  ``{\tt [}'' and ``{\tt ]}'' serve as 
brackets and a period ``{\tt .}'' is to be  replaced by a left bracket
and a matching right bracket as far right as consistent with the
brackets already present, when brackets are restored from left to right.
For operations of equal binding priority, association to the left is assumed.
In order of ascending priority we have \\
%\begin{format}
$\equiv$ or {\tt EQUIV} (2) \\
$\limplies$ or {\tt IMPLIES} (3) \\
$\lor$ or {\tt OR} (4) \\
$\land$ or {\tt AND} (5) \\
$\lnot$ or {\tt NOT} or \verb+~+ (100) \\
applications (like $Pxy$ or {\tt $[\lambda \; x \; x]t$}) \\
binders ($\lambda$,$\forall$,$\exists$)
%\end{format}

\item {\bf Types} - Function types are built from single letter primitive types.  Grouping
is indicated by parentheses ``{\tt (}'' and ``{\tt )}''.  The basic types
are @subomicron or {\tt O} for truth values and @subiota or {\tt I} for
individuals.  Any letter (except {\tt T}, i.e.  $_\tau$) may serve as a
typevariable.  A pair $_{(\greeka\greekb)}$ or {\tt (AB)} is the type of
a function from elements of type {\tt B} to type {\tt A}.
E.g.  {\tt (O(OI))} or $_{\greeko(\greeko\greeki)}$ is the type of
a collection of sets of individuals.  Association to the left is
assumed, so {\tt (OAAA)} or $_{\greeko\greeka\greeka\greeka}$
is the type of a
three place predicate on variables of type {\tt (A)}.

\item {\bf Identifiers in higher-order mode} - In higher-order mode identifiers
may consists of any string of ASCII and special characters.  Greek
subscripts are reserved for type symbols and superscripts may only
appear at the end of the identifier.  The following symbols terminate
identifiers:  ``{\tt <Space> [ ] ( ) .  ~ <Return> <Tab>}''.  They may not
appear inside an identifier.  Reserved for special purposes package are
``{\tt :  ; ` < >}'' and should therefore not be used.  Also with special characters
 $\forall$, $\exists$, and $\lambda$ are also single character identifiers.
In strings, superscripted numbers are preceded by ``{\tt \verb+~+ }''.

\item {\bf Identifiers in first-order mode} - In first-order mode all identifiers
consist of a single letter.  Upper and lower case letters denote 
distinct identifiers.  In addition there is a set of keywords,
currently {\tt AND, OR, IMPLIES, NOT, FORALL, EXISTS, LAMBDA, EQUIV},
which are multi-letter identifiers and are always converted to all
uppercase.  They have to be delimited by one of the terminating
characters listed above, while all other identifiers may be typed without
spaces in between.

\item {\bf Type inference} - \TPS implements a version of Milner's algorithm
to infer the most general type of a wff with no or incomplete
type information.  Internally every identifier in a wff is typed.
Only the first occurrence
of an identifier will be typed in printing, unless the same identifier occurs
with different types in the same wff.
\end{itemize}

\section{Printing Vertical Paths}\index{Vertical Paths} % @tag(vpf)

There are a number of operations available in the editor and mate top levels
for printing vertical path diagrams. Also,
the following wff operation is available for printing vertical diagrams
of jforms:

\begin{itemize}
\item \indexother{VPFORM JFORM \{FILE\} \{STYLE\} \{PRINTTYPES\} \{BRIEF\} \{VPFPAGE\}}
\end{itemize}

The default values are:
\begin{itemize}
\item {\tt File} defaults to {\tt TTY:}, the terminal.

\item {\tt Style} defaults to the value of the flag \indexflag{STYLE}.

\item {\tt PrintTypes} defaults to the value of the flag \indexflag{PRINTTYPES}.

\item {\tt Brief} has three possible settings: {\tt T} means that only the names of logical atoms will be printed, 
	and not their values, {\tt NIL} means that under each atom its value will appear, and
\item {\tt L} means that just the atomnames will be printed in the diagram
	but a legend which contains every atom with its value will be 
	appended to the first page of output.

\item \indexparameter{VpfPage} is the number of characters which fit on one line.

\item \indexparameter{AndHeight} is an optional global variable
	which is equal to the number of
	blank lines to be left for a conjunction. It defaults
	to 1. 

\item \indexparameter{ForallIndent} is another optional global variable, containing the number
	of columns the quantifier is set off its scope. The default is 1.

\end{itemize}

{\tt BRIEF} can assume the values	{\tt T} for printing the diagram in brief 
format,	{\tt L} for a  brief diagram, but with a legend (atomnames with their
associated values) at the end of the first page, {\tt LT} for a legend with
type symbols forced to print and {\tt NIL} which gives the the full diagram.

Both of these functions will prompt you for a comment after a few statistics
about the diagram are given. The comment will
be spread across the top lines of the diagram with carriage returns placed
where you type them. 

\section{Global Parameters and Flags} \label{printflag}% @tag(printflag)

The following Lisp identifiers are either flags or values used by the
functions which read or write formulas.

\begin{description}
\item [\indexparameter{CFontTable}] 
This is a two dimensional array which is used to translate between
special characters on the Concept screen and their internal name.  For
example, {\tt (CFontTable 1 91)} is {\tt AND}.

\item [\indexflag{FIRST-ORDER-PRINT-MODE}] 
If {\tt T} wffs will be printed in first-order mode, otherwise in higher-order
mode.

\item [\indexflag{FIRST-ORDER-MODE-PARSE}] 
If {\tt T}, wffs will be parsed in first-order mode, otherwise higher-order
parsing mode is in effect.  See the section on parsing for a more
detailed explanation.

\item [\indexflag{LOWERCASERAISE}] 
If this identifier is set to {\tt T} then lower case letters will be converted
to their upper case equivalents.  This conversion is done when the formula
is first parsed.  The default value is {\tt NIL}.

\item [\indexparameter{PC}] 
A variable used by the formula printing functions.  It stores the previous
character printed.  It is used to help determine spacing within the
formula.  Set to {\tt NIL} in \indexfile{prt.lisp}.  Not important to the user.

\item [\indexflag{PRINTDEPTH}] 
When a formula is printed, subformulas at a depth of more that
{\tt PrintDepth} are not printed, but replaced by a "{\tt \&}".
In the formula editor, it is
set to \indexflag{EDPRINTDEPTH}.  A {\tt PRINTDEPTH} of {\tt 0} means that the formula
will be printed up to arbitrary depth.

\item [\indexflag{PRINTTYPES}] 
If this is set to {\tt T}, type symbols will be printed at least once
on all primitive symbols.  Otherwise, no types are printed.  This
defaults to {\tt T}, and can be toggled with the command \indexcommand{shownotypes}.

\item [\indexparameter{SailCharacters}] 
This is a list of pairs, {\tt (SYMBOL . NUM)}.  Here {\tt NUM} is the position
in the {\tt SAIL} character set for {\tt SYMBOL}.
\end{description}

The following flags are used to control the way formulas are printed.  Usually
the default setting of all these flags will be adequate.  For more information
see the section on pretty-printing in the \TPS user manual.

\begin{description}
\item [\indexflag{PPWFFLAG}] if {\tt T}, formulas will be pretty-printed.  This is
the default setting, except in the editor, where you can achieve pretty-printing
with the @Ited(PP) command.

\item [\indexflag{FLUSHLEFTFLAG}] 
If {\tt T}, no line of a pretty-printed formula will be indented.  The default
is {\tt NIL}

\item [\indexflag{FILLINEFLAG}] 
If {\tt NIL}, every argument of an associative infix operator will have a
separate line.  The default in {\tt NIL}.

\item [\indexflag{LOCALLEFTFLAG}] 
If {\tt T}, arguments of infix operators start in the same column as
the operator.  The default is {\tt NIL}.

\item [\indexflag{ATOMVALFLAG}] 
If {\tt T}, the name of every atom will be printed below its value.

\item [\indexflag{ALLSCOPEFLAG}] 
If {\tt T}, all punctuations (``{\tt []}'', ``{\tt .}'') will appear in the formulas
to be printed.  No association to the left or precedence of logical
connectives will be assumed.
\end{description}

\section{Simple MetaWffs in TPS3}

Even though in \TPS the principle metalanguage is of course Lisp, it is
often convenient to be able to use simple notations from the metalanguage
and include them directly in the input format for Wffs.  In \TPS this is
achieved by providing a notation for certain kinds of {\tt WFFOPS} inside
an external specification of a wff.  This method is not perfect, but has
other advantages as well, as we shall see.

\subsection{The Notation}

The motivation behind the notation is an analogy to Lisp: we use the
backquote to introduce some Lisp form which is to be evaluated and
inserted into the Wff.  One restriction is that the wffop must return
a {\it gwff} (or a subtype, like a {\it gvar}).  The other is that \TPS must
have certain pieces of knowledge about the {\it wffop} used, in order to be
able to determine the type of the result of applying the {\it wffop}.

Some examples of external format and what they are parsed to:
\begin{verbatim}
"forall x. `(lcontr [[lambda x. P x x x] [f x]])"
\end{verbatim}
to
$$\forall x.P\; [f\; x]\; [f\; x]\; [f\; x]$$

\begin{verbatim}
"forall x exists y.
  `(lexpd z [f x] `(lexpd z [f x] [Q [f x] [f x] y] `(1)) `t)"
\end{verbatim}
to
$$\forall x\exists y.[\lambda z\; [\lambda z^{1}\; Q\; z^{1}\; z\; y]\; z]\; [f\; x]$$

\begin{verbatim}
"`(substitute-types `((A . (O . I))) [P(OA) subset Q(OA)])"
\end{verbatim}
to
$$P_{\greeko(\greeko\greeki)} \subseteq Q_{\greeko(\greeko\greeki)}.$$
(The latter could have been more easily specified as
\begin{verbatim}
(substitute-types (("A" "OI")) "[P(OA) subset Q]")
\end{verbatim}
but that is no longer possible when the formula is to be embedded
in another.)


Here are the general rules:
\begin{itemize}
\item In an ordinary wff, a backquote may precede something of the form
{\wt ({\it wffop} {\it arg} ... {\it arg})}, where {\it wffop} has all the necessary
type information.  The typecase of {\it wffop} is irrelevant.

\item Among {\wt ({\it arg} ... {\it arg})}, each argument is either a gwff (and may
contain other backquoted expressions) or a Lisp expression, which
is considered a constant.  This is necessary to supply arguments which
are not gwffs to a {\it wffop}.  Notice, that it must be the internal
representation of the argument!\footnote{At some point one could work at
removing this restriction, if types are handled properly.}
\end{itemize}

\section{More about Jforms}

Much of the code for handling jforms is in {\it jforms-labels.lisp}; see
{\tt defflavor jform} in this file for the definition.

In the same file we see:

%\begin{tpsexample}
\begin{verbatim}
(eval-when (load compile eval)
  (defflavor disjunction
    (mhelp "A disjunction label stands for a disjunction of wffs.")
    (inherit-properties jform)
    (include jform (type 'disjunction))
\end{verbatim}
%\end{tpsexample}

This tell us that a jform can be a disjunction.





\section{Printing Proofs}

Proofs printed in Scribe or \TeX are preceded by preambles which are
defined by the variables \indexflag{SCRIBE-PREAMBLE} and 
\indexflag{VPFORM-TEX-PREAMBLE}. The values of these flags are set
in the {\it tps3.ini} file. Since these preambles source
files in the directory {\it .../doc/lib}, things must be done carefully to
make sure that \indexcommand{SCRIBEPROOF} and \indexcommand{TEXPROOF} 
will insert the appropriate pathname when tps is distributed to other 
locations. Note that the Makefile creates the file {\it tps3.sys}, which contains the variable
sys-dir which shows where the tps was built.
	
When the Scribe preamble was changed to add
\begin{verbatim}
@@LibraryFile(KSets)
@@LibraryFile(Mathematics10)
\end{verbatim}
some of the hacks in {\it tps.mss} may have become obsolete (but harmless).
Mathematics10 is a file from the standard Scribe library; KSets
is a file belonging to \tps.


\chapter{Well-formed formulae operators}

\section{Operations on Wffs}

By definition, operations on wffs differ from commands in that they
return a meaningful value, usually another wff or a truth value.  While
commands are usually given at the top-level, operations are usually
used inside the editor.  In other respects, operations on wffs are very
similar to commands in \tps.  The types of the arguments and the type of
the result must be specified in the declaration of a {\it wffop}.
Moreover, help for the arguments and help for the wffop itself is
available.  Arguments for wffops may be typed exactly the way arguments
for commands are:  one at a time after a short help message.

You may frequently have to refer to chapter ~\ref{toplev}, since it will
be assumed below that you have a general idea of how the \TPS top-level
interprets commands.

\subsection{Arguments to Wffops}
In principle, arguments to (or results of) wffops can have any type
defined inside \tps.  There are some argument types which are mainly
used for wffops and rarely or not at all for commands.  They are the following
\begin{description}
\item [\indexargtypes{GWFF}] A generalized wff.

\item [\indexargtypes{BOOLEAN}] {\tt NIL} for ``false'', anything else for ``true''.
Internally these are converted {\tt NIL} and {\tt T} first.  In particular,
if a wffop has been declared to return an object of type {\tt BOOLEAN},
this wffop may return anything, but {\tt NIL} is printed as {\tt NIL}, while
everything else is printed as {\tt T}.

\item [\indexargtypes{TYPESYM}] A type symbol (in string representation).  This is
extremely useful for error messages (inside \indexfunction{THROWFAIL}).
For example, the type inference program may contain a line
%\begin{tpsexample}
\begin{verbatim}
(throwfail "Type " (t1.typesym) " does not match " (t2.typesym))
\end{verbatim}
%\end{tpsexample}
For most settings of the \indexflag{STYLE} flag, this will print the types as true
greek subscripts.

\item [\indexargtypes{GVAR}] A general variable.  This is only one of
a whole class of possible subtypes of wffs ({\tt GWFF}).  The {\tt GETFN} for
these special kinds of wffs can easily be described using the function
\indexfunction{GETWFF-SUBTYPE}, which takes a predicate as the first argument,
an {\tt RWFF} as the second.
\end{description}

As an example for the definition of a subtype of {\tt GWFF} serves the definition
of {\tt GVAR}:
\begin{verbatim}
(deftype gvar
  (getfn (getwff-subtype 'gvar-p gvar))
  (testfn gvar-p)
  (printfn printwffhere)
  (side-effects t)
  (no-side-effects edwff)
  (mhelp "	A gwff which must be a logical variable"))
\end{verbatim}

\subsection{Defining Wffops}\label{defwffop}
The format for defining a wffop is very similar to that for defining a
MExpr.  The function that does the definition is called
\indexfunction{DEFWFFOP}.  The general format is ({\tt {}} enclose optional
arguments)
\begin{verbatim}
(DefWffop <name>
	{(ArgTypes <type1> <type2> ...)}
	(ResultType <type>)
	{(ArgNames <name1> <name2> ...)}
	{(ArgHelp <help1> <help2> ...)}
	{(Applicable-Q <fnspec>)}
	{(Applicable-P <fnspec>)}
        {(WffArgTypes <type> ... <type>)}
        {(Wffop-Type <type>)}
        {(Wffop-Typelist (<typesymbol> ... <typesymbol>))}
        {(DefaultFns <fnspec1> <fnspec2> ...)}
        {(MainFns <fnspec1> <fnspec2> ...)}
        {(Replaces <wffop>)}
        {(Print-Op <boolean>)}
        {(Multiple-Recursion <boolean>)}
	{(MHelp "<comment>")})
\end{verbatim}

The keywords {\tt ArgTypes}, {\tt ArgNames}, {\tt ArgHelp}, {\tt DefaultFns},
{\tt MainFns} and {\tt MHelp} have the
same meaning as for commands (MExprs).  See Section ~\ref{mexprargs}.
You have to mention {\tt ArgNames} before {\tt Applicable-P}, if you want to make
use of the argnames without explicitly using lambda.
The other keywords are as follows:

\begin{description}
\item [{\tt RESULTTYPE}] is the only non-optional part of the declaration and
is used for printing the result of the wffop.

\item [{\tt APPLICABLE-Q}] is a ``quick'' predicate (see Section ~\ref{quickslow})
to decide whether the wffop is applicable to a given set of arguments.
If omitted (or explicitly stated to be \indexfunction{TRUEFN}), it means that the
wffop can always be applied.

\item [{\tt APPLICABLE-P}] is a ``slow'' predicate
which is supposed to check thoroughly whether the wffop is applicable.
Again, if one wants to state explicitly that a wffop is always applicable,
use \indexfunction{TRUEFN}.

\item [{\tt WFFARGTYPES}] There must be exactly as many {\it type} entries, as there
are arguments to the {\it wffop}.  Each {\it type} entry may be either a
type (in string format) or {\tt NIL}, which is used for arguments which
are not {\tt gwffs}.

\item [{\tt WFFOP-TYPE}] specifies a {\it type} in string format, which is the
type of the result the {\it wffop}, or {\tt NIL}, if the result is not a
{\it gwff}.

{\tt WFFOP-TYPELIST} \\ A list of type symbols which are to be considered
type variables in the definition of the {\it wffop}.

\item [{\tt REPLACES}] The wffop being defined is to replace some previously defined
wffop. This is used extremely rarely.

\item [{\tt PRINT-OP}] This is set to {\tt T} for printing operations (which are 
usually defined using the macro \indexother{DEFPRTOP}, which sets this 
property automatically). By default, this property has value {\tt NIL}.

\item [{\tt MULTIPLE-RECURSION}]  seems to be set to {\tt T} for most tests of
equality and {\tt NIL} everywhere else. I'm not entirely sure what it's for.
\end{description}

Here are some example which may shed more light onto the subject.

%\begin{tpsexample}
\begin{verbatim}

(defwffop substitute-l-term-var
  (argtypes gwff gvar gwff)
  (resulttype gwff)
  (argnames term var inwff)
  (arghelp "term" "var" "inwff")
  (wffargtypes "A" "A" "B")		; TERM and VAR are of type A
  (wffop-type "B")			; INWFF and result of type B
  (wffop-typelist "A" "B")		; where A and B may be any types.
  (mhelp "..."))

(defwffop lexpd
  (argtypes gvar gwff gwff occ-list)
  (resulttype gwff)
  (argnames var term inwff occurs)
  (arghelp "lambda variable" "term to be extracted" "contracted form"
	   "occurrences to be extracted")
  (wffargtypes "A" "A" "B" NIL)		; TERM and VAR are of type A,
					; INWFF is of type B, OCCURS is not
  (wffop-type "B")			; a gwff, result is of type B,
  (wffop-typelist "A" "B")		; where A and B may be any types.
  (applicable-p (lambda (var term inwff occurs)
		  (declare (ignore inwff occurs))
		  (type-equal term var)))
  (mhelp "..."))

(defwffop substitute-types
  (argtypes typealist gwff)
  (resulttype gwff)
  (argnames alist gwff)
  (arghelp "alist of types" "gwff")
  (mhelp "Substitute for types from list ((old . new) ...) in gwff."))

\end{verbatim}
%\end{tpsexample}

\subsection{Defining Recursive Wffops}

The category \indexother{wffrec\%} is for recursive wff functions. 
Such operations are defined with the \indexfunction{defwffrec} function;
they have only three properties: {\tt ARGNAMES}, {\tt MHELP}
and {\tt MULTIPLE-RECURSION}.

The point of this is that we needed a way of saving the 
{\tt ARGNAME} information for functions which
use an \indexfunction{APPLY-LABEL}, but are not wffops themselves.
These are defined as wffrecs.

Some examples: 

%\begin{tpsexample}
\begin{verbatim}
(defwffrec gwff-q
  (argnames gwff))

(defun gwff-q (gwff)
  (cond ((label-p gwff) (apply-label gwff (gwff-q gwff)))
	((lsymbol-p gwff) t)
	((atom gwff) nil)
	((and (boundwff-p gwff) (gvar-p (caar gwff)) (gwff-q (cdr gwff))))
	((and (gwff-q (car gwff)) (gwff-q (cdr gwff))))))

(defwffrec wffeq-def1
  (argnames wff1 wff2 varstack switch)
  (multiple-recursion t))
 
; the function wffeq-def1 is pages long, so it's not quoted here. Look 
; in file wffequ2.lisp for details.

\end{verbatim}
%\end{tpsexample}

\subsection{Defining a Function Performing a Wffop}
There are some necessary restrictions on how to define proper wffops,
other conventions are simply a matter of style.
The following are general guidelines,
which do not address the definition of flavors (see Section ~\ref{defflavors}).

\begin{enumerate}
\item All arguments to a wffop may be assumed to be of the correct type, when
the function is invoked.  This does not mean, that the function never
should check for an error, but at least the function does not have to check
whether an argument is well-formed, or whether an argument is a logical
variable and not an application.

\item Most user-level wffops get by without using any ``slow'' predicates for
constituents of a gwff.  Use the ``quick'' predicate and assume that
the argument is a gwff.

\item Make the name of a wffop as descriptive as possible.  The user will rarely
have to type this long name, since he will normally invoke wffops in the
editor, where they can be given short aliases.  See section ~\ref{EDOPS}.

\item When using auxiliary functions, make sure their name can be easily related
to the name of the main function.

\item Check the wff operations in the TPS3 Facilities Guide for Programmers and
Users before defining new functions. In particular, you should often use
\indexfunction{GAR} and \indexfunction{GDR} instead of car and cdr to 
manipulate wffs, since the wffs may have labels.

\item Always make sure you are invoking the ``quick'' test in the correct order,
since later tests rely on the fact that earlier tests failed.
\end{enumerate}

\subsection{Quick Test versus Slow Test}\label{quickslow}
Most predicates which test for certain kinds of subformulas come in two
incarnations: as a ``quick'' test and a ``slow'' test.  As a general
convention that should never be violated, both functions have the same
name except for the last character, which is {\tt -Q} for the quick
test and {\tt -P} for the slow test.

As a rule of thumb, quick predicates may assume a very restricted kind
of argument (e.g. a literal atom), but may not work recursively down
into the formula.  Slow predicates, however, may assume nothing about
the argument (they should always work), and often have to do a recursion
to see whether the predicate is true of the argument.

Quick predicates are most useful when in recursive functions that implement
a wffop.  Slow predicates are chiefly called inside the editor to test
that certain transformations or applications will be legal, {\it before they
are performed}.  Speed is usually not important when working in the editor,
but wffops in general should be optimized for speed, since time does make
a difference in automatic mode.

A list of the most useful quick predicates in the order in which they must
be called is supplied here.  See the comments attached to the predicates
in the source file if this list is unclear or ambiguous.

{\bf It is absolutely essential to understand the role of quick
predicates and the order of their invocation to write bug-free code!}

\begin{description}
\item [\indexfunction{LABEL-Q} {\it gwff}] tests for a label.  The standard action in this
case is {\tt (APPLY-LABEL GWFF ({\it wffop} {\it arg1} ... {\it argn}))} where
{\it wffop} is the wffop we are defining and {\it arg1} through {\it argn} are
its arguments. Always call this first, since any given argument may be a label.

\item [\indexfunction{LSYMBOL-Q} {\it gwff}] tests for a logical symbol.  This could either
be a variable, constant, or abbreviation. This must come after the test for {\it label}, 
but does not assume anything else.  There are several subtypes of {\it lsymbol} which
assume that their argument is a {\it lsymbol} and must be called in the
following order:
\begin{description}
\item [\indexfunction{LOGCONST-Q} {\it gwff}] a logical constant, which must have been declared
with {\tt DEF-LOGCONST.}

\item [\indexfunction{PROPSYM-Q} {\it gwff}] a proper symbol, that is something that has not
been declared a constant or abbreviation.

\item [\indexfunction{PMPROPSYM-Q} {\it gwff}] a polymorphic proper symbol (higher-order mode only).

\item [\indexfunction{PMABBREV-Q} {\it gwff}] a polymorphic abbreviation (higher-order mode only).

\item [\indexfunction{ABBREV-Q} {\it gwff} ] an abbreviation.
\end{description}

\item [\indexfunction{BOUNDWFF-Q}] 
Test whether the wff starts with a binder (of any type) and
assumes that we already know that it is neither {\it label} nor
a {\it lsymbol} (in Lisp terms: it must be a {\tt CONS} cell). Access the
bound variable with {\tt CAAR,} the binder with {\tt CDAR,} the scope of the binder
with {\tt CDR.}  Construct a new bound formula with {\tt (CONS (CONS {\it bdvar} {\it binder})
{\it scope})}.

\item [{\tt T}] This is the ``otherwise'' case, i.e. we have an application.
Access the ``function'' part with {\tt CAR,} the ``argument'' part with
{\tt CDR.}  Construct a new application with {\tt (CONS {\it function} {\it argument})}.
Remember also that all functions and predicates are curried.

\end{description}

\begin{center}
{\bf Examples of Wffops}
\end{center}

The following examples are taken from actual code\footnote{As of July 7th, 1994}.
%\begin{verbatim, LineWidth 80, LeftMargin -4}
%\begin{Text, Indent 1inch}
\begin{verbatim}

The following are two different substitution functions
SUBSTITUTE-TERM-VAR (currently in wffsub1.lisp) 
substitutes a term for a variable, but gives
and error if the term is not free for the variable in the wff.
SUBSTITUTE-L-TERM-VAR (currently in wffsub2.lisp) 
also substitutes a term for a variable,
but renames bound variables if a name conflict occurs.
There may be a global variable, say SUBST-FN, whose value is
the function used for substitution by default, or there may be a function
SUBSTITUTE, which checks certain flags to determine which function
to call.

(defwffop substitute-term-var
  (argtypes gwff gvar gwff)
  (wffargtypes "A" "A" "B")
  (resulttype gwff)
  (wffop-type "B")
  (wffop-typelist "A" "B")
  (argnames term var inwff)
  (arghelp "term" "var" "inwff")
  (applicable-p (lambda (term var inwff) (free-for term var inwff)))
  (mhelp
   "Substitute a term for the free occurrences of variable in a gwff."))

(defun substitute-term-var (term var inwff)
  "This function should be used with extreme caution. There's an underlying
  assumption that TERM is free for VAR in INWFF (which is true if TERM is
  a new variable)."
  (or (subst-term-var-rec (intern-subst term var) var inwff)
      inwff))

(defun subst-term-var-rec (term var inwff)
  (cond ((label-q inwff)
	 (apply-label inwff (subst-term-var-rec term var inwff)))
	((lsymbol-q inwff) (if (eq var inwff) term nil))
	((boundwff-q inwff)
	 (if (eq (caar inwff) var) nil
	     (let ((new-wff (subst-term-var-rec term var (cdr inwff))))
	       (if new-wff (cons (car inwff) new-wff) nil))))
	(t (let ((left (or (subst-term-var-rec term var (car inwff))
			   (car inwff)))
		 (right (or (subst-term-var-rec term var (cdr inwff))
			    (cdr inwff))))
	     (unless (and (eq left (car inwff)) (eq right (cdr inwff)))
		     (cons left right))))))

(defwffop substitute-l-term-var
  (argtypes gwff gvar gwff)
  (wffargtypes "A" "A" "B")
  (resulttype gwff)
  (wffop-type "B")
  (wffop-typelist "A" "B")
  (argnames term var inwff)
  (arghelp "term" "var" "inwff")
  (mhelp
   "Substitute a term for the free occurrences of variable in a gwff.
Bound variables may be renamed, using the function in the global
variable REN-VAR-FN."))

(defun substitute-l-term-var (term var inwff)
  (or (subst-l-term-rec (intern-subst term var) var inwff) inwff))

LCONTR (currently in wfflmbd2.lisp)
does a Lambda-contraction.  Notice the use of
THROWFAIL and the use of general predicates like LAMBDA-BD-P
rather than testing directly whether a given wff is bound by
Lambda.  This way, the function works, even if the CAR fo
the application is a label!

(defwffop lcontr
  (argtypes gwff)
  (wffargtypes "A")
  (resulttype gwff)
  (wffop-type "A")
  (wffop-typelist "A")
  (argnames reduct)
  (arghelp "gwff (reduct)")
  (applicable-p reduct-p)
  (mhelp "Lambda-contract a top-level reduct.
Bound variables may be renamed using REN-VAR-FN"))

(defun lcontr (reduct)
  (cond ((label-q reduct) (apply-label reduct (lcontr reduct)))
	((lsymbol-q reduct)
	 (throwfail "Cannot Lambda-contract " (reduct . gwff)
		    ", a logical symbol."))
	((boundwff-q reduct)
	 (throwfail "Cannot Lambda-contract " (reduct . gwff)
		    ", a bound wff."))
	(t (if (lambda-bd-p (car reduct))
	       (substitute-l-term-var (cdr reduct) (gar (car reduct))
				      (gdr (car reduct)))
	       (throwfail "Top-level application " (reduct . gwff)
			  " is not of the form [LAMBDA x A]t.")))))

FREE-FOR is a simple example of a predicate on wffs.
Here, the type of the result is declared to be BOOLEAN.

(defwffop free-for
  (argtypes gwff gvar gwff)
  (resulttype boolean)
  (argnames term var inwff)
  (arghelp "term" "var" "inwff")
  (applicable-q (lambda (term var inwff) (declare (ignore inwff))
			(type-equal term var)))
  (applicable-p (lambda (term var inwff) (declare (ignore inwff))
			(type-equal term var)))
  (mhelp "Tests whether a term is free for a variable in a wff."))

(defun free-for (term var inwff)
  (cond ((label-q inwff)
	 (apply-label inwff (free-for term var inwff)))
  	((lsymbol-q inwff) t)
	((boundwff-q inwff)
	 (cond ((eq (caar inwff) var) t)
	       ((free-in (caar inwff) term)
		(not (free-in var (cdr inwff))))
	       (t (free-for term var (cdr inwff)))))
	(t (and (free-for term var (car inwff))
		(free-for term var (cdr inwff))))))

TYPE (currently in wffprim.lisp) 
returns the type of the argument.  The name is a very
troublesome one and we may eventually need to change it globally so as not
to conflict with Common Lisp.

(defwffop type
	(argtypes gwff)
	(resulttype typesym)
	(argnames gwff)
	(arghelp "gwff")
	(mhelp "Return the type of a gwff."))

(defun type (gwff)
  (cond ((label-q gwff) (apply-label gwff (type gwff)))
	((lsymbol-q gwff) (get gwff 'type))
	((boundwff-q gwff) (boundwfftype gwff))
	(t (type-car (type (car gwff))))))

The following are a sequence of functions which instantiate abbreviations.
One can either instantiate a certain abbreviation everywhere
(INSTANTIATE-DEFN), instantiate all abbreviations (not recursively)
(INSTANTIATE-ALL), or instantiate the first abbreviates, counting
from left to right (INSTANTIATE-1).
The functions are implemented by one master function, one of whose
arguments is a predicate to be applied to an abbreviation.  This
predicate should return something non-NIL, if this occurrence is to be
instantiated, NIL otherwise.
Notice the subcases inside LSYMBOL-Q and the order of the quick
predicates in the OR clause.


(defwffop instantiate-defn
  (argtypes symbol gwff)
  (resulttype gwff)
  (argnames gabbr inwff)
  (arghelp "abbrev" "inwff")
  (applicable-p (lambda (gabbr inwff) (declare (ignore inwff))
			(or (abbrev-p gabbr) (pmabbsym-p gabbr))))
  (mhelp "Instantiate all occurrences of an abbreviation.
The occurrences will be lambda-contracted, but not lambda-normalized."))

(defun instantiate-defn (gabbr inwff)
  (instantiate-definitions 
   inwff #'(lambda (abbsym chkarg) (eq abbsym chkarg)) gabbr))


(defwffop instantiate-all
  (argtypes gwff symbollist)
  (resulttype gwff)
  (argnames inwff exceptions)
  (arghelp "inwff" "exceptions")
  (defaultfns (lambda (&rest rest)
		(mapcar #'(lambda (argdefault arg) 
			    (if (eq arg '$) argdefault arg))
			'($ NIL) rest)))
  (mhelp "Instantiate all definitions, except the ones specified
in the second argument."))

(defun instantiate-all (inwff exceptions)
  (instantiate-definitions
   inwff #'(lambda (abbsym chkarg) (not (memq abbsym chkarg))) exceptions))

(defwffop instantiate-1
  (argtypes gwff)
  (resulttype gwff)
  (argnames inwff)
  (arghelp "inwff")
  (mhelp "Instantiate the first abbreviation, left-to-right."))

(defun instantiate-1 (inwff)
  (let ((oneflag nil))
    (declare (special oneflag))
    (instantiate-definitions
     inwff #'(lambda (abbsym chkarg)
	       (declare (ignore abbsym chkarg) (special oneflag))
	       (prog1 (not oneflag) (setq oneflag t)))
     nil)))

(defwffrec instantiate-definitions
  (argnames inwff chkfn chkarg))

(defun instantiate-definitions (inwff chkfn chkarg)
  (cond ((label-q inwff)
	 (apply-label inwff (instantiate-definitions inwff chkfn chkarg)))
	((lsymbol-q inwff)
	 (cond ((or (logconst-q inwff) (propsym-q inwff) (pmpropsym-q inwff))
		inwff)
	       ((pmabbrev-q inwff)
		(if (funcall chkfn (get inwff 'stands-for) chkarg)
		    (get-pmdefn inwff) inwff))
	       ((abbrev-q inwff)
		(if (funcall chkfn inwff chkarg) (get-defn inwff) inwff))))
	((boundwff-q inwff)
	 (if (and (anyabbrev-q (binding inwff))
		  (funcall chkfn (binding inwff) chkarg))
	     (get-def-binder (binding inwff) (bindvar inwff) (gdr inwff))
	     (cons (car inwff)
		   (instantiate-definitions (gdr inwff) chkfn chkarg))))
	(t (let ((newcar (instantiate-definitions (car inwff) chkfn chkarg)))
	     (if (and (lambda-bd-p newcar) (not (lambda-bd-p (car inwff))))
		 (lcontr (cons newcar
			       (instantiate-definitions (cdr inwff)
							chkfn chkarg)))
		 (cons newcar
		       (instantiate-definitions (cdr inwff) chkfn chkarg)))))))

\end{verbatim}

\section{The formula editor}\label{EDOPS}

The formula editor is in many ways very similar to the top-level of
\tps.  The main difference is that we have an entity called 
``current wff'' or \indexData{edwff}, which can be operated on.
All the regular top-level commands can still be executed, but we
can now also call any {\it wffop} directly.  If we want the {\it wffop} to
act on the {\it edwff}, we can specify {\tt EDWFF} which is a legal
{\it gwff} inside the editor.  

This process is made even easier through the introduction of {\it edops}.
An \indexData{edop} is very similar to a {\it wffop}, but it ties into the
structure of the editor in two very important ways:  One argument can be
singled out, so that it will always be the {\it edwff}, and secondly the
{\it edop} will specify what happens to the result of the operations, which
is often the new {\it edwff}.  This is particularly useful for operations
which take one argument and return one wff as a value, like
lambda-normalization. It helps to give edops and wffops different names;
the name of a wffop should be longer and more descriptive than the name of
the edop for which it is an alias.

\section{Example of Playing with a Jform in the Editor}

%\begin{tpsexample}
\begin{verbatim}
<Ed9>sub x2115
<Ed10>neg
<Ed13>cjform
(AND ((FORALL x<I>) (OR ((FORALL y<I>) LIT0) ((FORALL z<I>) LIT1))) 
 ((FORALL u<I>) ((EXISTS v<I>) (OR LIT2 (AND LIT3 LIT4)))) 
 ((FORALL w<I>) (OR LIT5 LIT6)) ((EXISTS u<I>) ((FORALL v<I>) (OR LIT7 LIT8))))

<Ed14>edwff
(AND ((FORALL x<I>) (OR ((FORALL y<I>) LIT0) ((FORALL z<I>) LIT1))) 
 ((FORALL u<I>) ((EXISTS v<I>) (OR LIT2 (AND LIT3 LIT4))))
 ((FORALL w<I>) (OR LIT5 LIT6)) ((EXISTS u<I>) ((FORALL v<I>) (OR LIT7 LIT8))))

<Ed15>(setq aa edwff)
(AND ((FORALL x<I>) (OR ((FORALL y<I>) LIT0) ((FORALL z<I>) LIT1))) 
 ((FORALL u<I>) ((EXISTS v<I>) (OR LIT2 (AND LIT3 LIT4))))
 ((FORALL w<I>) (OR LIT5 LIT6)) ((EXISTS u<I>) ((FORALL v<I>) (OR LIT7 LIT8))))

<Ed16>aa
(AND ((FORALL x<I>) (OR ((FORALL y<I>) LIT0) ((FORALL z<I>) LIT1)))
 ((FORALL u<I>) ((EXISTS v<I>) (OR LIT2 (AND LIT3 LIT4))))
 ((FORALL w<I>) (OR LIT5 LIT6)) ((EXISTS u<I>) ((FORALL v<I>) (OR LIT7 LIT8))))

<Ed17>(auto::jform-parent aa)
NIL
vp
<Ed19>(setq bb (auto::conjunction-components aa))
(((FORALL x<I>) (OR ((FORALL y<I>) LIT0) ((FORALL z<I>) LIT1)))
 ((FORALL u<I>) ((EXISTS v<I>) (OR LIT2 (AND LIT3 LIT4))))
 ((FORALL w<I>) (OR LIT5 LIT6)) ((EXISTS u<I>) ((FORALL v<I>) (OR LIT7 LIT8))))

<Ed20>(length bb)
4
<Ed21>(car bb)
((FORALL x<I>) (OR ((FORALL y<I>) LIT0) ((FORALL z<I>) LIT1)))

<Ed22>(cadr bb)
((FORALL u<I>) ((EXISTS v<I>) (OR LIT2 (AND LIT3 LIT4))))

<Ed23>(auto::jform-parent (car bb))
(AND ((FORALL x<I>) (OR ((FORALL y<I>) LIT0) ((FORALL z<I>) LIT1))) 
 ((FORALL u<I>) ((EXISTS v<I>) (OR LIT2 (AND LIT3 LIT4)))) 
 ((FORALL w<I>) (OR LIT5 LIT6)) ((EXISTS u<I>) ((FORALL v<I>) (OR LIT7 LIT8))))

<Ed24>(setq cc '((FORALL x<I>) (OR ((FORALL y<I>) LIT0) ((FORALL z<I>) LIT1))))
((FORALL X<I>) (OR ((FORALL Y<I>) LIT0) ((FORALL Z<I>) LIT1)))

<Ed26>bb
(((FORALL x<I>) (OR ((FORALL y<I>) LIT0) ((FORALL z<I>) LIT1)))
 ((FORALL u<I>) ((EXISTS v<I>) (OR LIT2 (AND LIT3 LIT4))))
 ((FORALL w<I>) (OR LIT5 LIT6)) ((EXISTS u<I>) ((FORALL v<I>) (OR LIT7 LIT8))))

<Ed27>(car bb)
((FORALL x<I>) (OR ((FORALL y<I>) LIT0) ((FORALL z<I>) LIT1)))

<Ed28>cc
((FORALL X<I>) (OR ((FORALL Y<I>) LIT0) ((FORALL Z<I>) LIT1)))

(These look the same, but they are quite different.)
\end{verbatim}
%\end{tpsexample}

\section{Defining an EDOP}

An {\it edop} does not define an operation on wffs, it simply
{\bf refers} to one.  Thus typically we have a {\it wffop} associated
with every {\it edop}, and the {\it edop} inherits almost all of its
properties from the associated {\it wffop}, in particular the
help, the argument types, the {\it applicable} predicates etc.

A definition of an {\it edop} itself then looks as follows
({\tt {}} enclose optional arguments)
\begin{verbatim}
(DefEdop <name>
	{(Alias <wffop>)}
	(Result-> <destination>)
	{(Edwff-Argname <name>)}
        {(DefaultFns <fnspec1> <fnspec2> ...)}
        {(Move-Fn <fnspec>)}
	{(MHelp "<comment>")})
\end{verbatim}

In the above definition, the properties have the following meanings:

\begin{description}
\item [{\tt ALIAS}] This is the name of the {\it wffop} this {\it edop} refers to.  It must
be properly declared using the {\tt DEFWFFOP} declaration.

\item [{\tt RESULT->}] This provides part of the added power of {\it edops}.  {\it destination}
indicates what to do with the result of applying the {\it wffop} in {\tt ALIAS}
to the arguments.  {\it destination} can be any of the following:
\begin{description}
\item [{\it omitted}] If omitted, the appropriate print function for the type of
result returned by the {\tt ALIAS} {\it wffop} will be applied to the result.

\item [{\tt EDWFF}] This means that the result of the operation is made the new
current wff ({\it edwff}) in the editor.

\item [{\tt EXECUTE}] This means that the result of the operation is a list of
editor commands which are to be executed.  This may seem strange, but
is actually very useful for commands like {\tt FI} (find the first infix operator),
or {\tt ED?} (move to edit the first ill-formed subpart).  The argument
type \indexargtypes{ED-COMMAND} was introduced for this purpose only.

\item [{\it fnspec}] If the value is none of the above, but is specified, it is
assumed to be an arbitrary function of one argument, which is applied
to the result returned by the {\it edop}.
\end{description}

\item [{\tt EDWFF-ARGNAME}] This is the name of the argument that will be filled with the
{\it edwff}; see the {\tt ARGNAME} property of MExprs, in section ~\ref{mexprargs}, 
for more information.

\item [{\tt DEFAULTFNS}] See the arguments for MExprs, in section ~\ref{mexprargs}.

\item [{\tt MOVE-FN}] This means that the result of the operation will be the new current
wff and moreover that the operation qualifies as a ``move'', namely
that we should store what we currently have before executing the command,
and then use {\it replace-fn} on the value returned after then next {\tt 0} or {\tt $\hat{}$}. 
For example, the editor command {\tt A} moves to the ``function part'' of an 
application.  Moreover, when we return via {\tt 0} or {\tt $\hat{}$}, we need to replace 
this ``function part''.

\end{description}

\section{Useful functions}
A useful function in defining {\it edops} is \indexfunction{EDSEARCH}.
{\wt EDSEARCH {\it gwff} {\it predicate}} will go through {\it gwff} from left
to right and test at every subformula, whether {\it predicate} is
true of that subformula.  If such a subformula is found, {\it EDSEARCH}
will return a list of editor moving commands which will move down
to this subformula.  If the predicate is true of the {\it gwff} itself,
{\tt EDSEARCH} will return {\wt (P)}, the command to print the current wff.
If no subformula satisfying
{\it predicate} is found, {\tt EDSEARCH} will return {\tt NIL}.  For example

\begin{verbatim}

(defedop fb
  (alias find-binder)
  (result-> execute)
  (mhelp "Find the first binder (left to right)")
  (edwff-argname gwff))

(defwffop find-binder
  (argtypes gwff)
  (resulttype edcommand)
  (argnames gwff)
  (arghelp "gwff")
  (mhelp "Find the first binder (left to right)"))

(defun find-binder (gwff) (edsearch gwff (function boundwff-p)))

\end{verbatim}

\section{Examples}

Consider the following examples.\footnote{As taken from the code, 7th July 1994.}

\begin{verbatim}

(defedop ib
  (alias instantiate-binder)
  (result-> edwff)
  (edwff-argname bdwff))

(defwffop instantiate-binder
  (argtypes gwff gwff)
  (resulttype gwff)
  (argnames term bdwff)
  (arghelp "term" "bound wff")
  (applicable-p (lambda (term bdwff)
		  (and (ae-bd-wff-p bdwff) (type-equal (gar bdwff) term))))
  (mhelp
   "Instantiate a top-level universal or existential binder with a term."))

(defun instantiate-binder (term bdwff)
  (cond ((label-q bdwff)
	 (apply-label bdwff (instantiate-binder term bdwff)))
	((lsymbol-q bdwff)
	 (throwfail "Cannot instantiate " (bdwff . gwff)
		    ", a logical symbol."))
	((boundwff-q bdwff)
	 (cond ((ae-bd-wff-p bdwff)
		(substitute-l-term-var term (caar bdwff) (cdr bdwff)))
	       (t
		(throwfail "Instantiate only existential or universal quantifiers," t
			   "not " ((cdar bdwff) . fsym) "."))))
	(t (throwfail "Cannot instantiate an application."))))

(defedop subst
  (alias substitute-l-term-var)
  (result-> edwff)
  (edwff-argname inwff))

(defedop db
  (alias delete-leftmost-binder)
  (result-> execute)
  (edwff-argname gwff))

(defwffop delete-leftmost-binder
  (argtypes gwff)
  (resulttype ed-command)
  (argnames gwff)
  (arghelp "gwff")
  (mhelp "Delete the leftmost binder in a wff."))

(defun delete-leftmost-binder (gwff)
  (let ((bdwff-cmds (find-binder gwff)))
    (append (ldiff bdwff-cmds (member 'p bdwff-cmds))
	    `(sub (delete-binder edwff)))))


(defwffop delete-binder
  (argtypes gwff)
  (resulttype gwff)
  (argnames bdwff)
  (arghelp "bound wff")
  (applicable-q ae-bd-wff-p)
  (applicable-q ae-bd-wff-p)
  (mhelp "Delete a top-level universal or existential binder."))

(defun delete-binder (bdwff)
  (cond ((label-q bdwff)
	 (apply-label bdwff (delete-binder bdwff)))
	((lsymbol-q bdwff)
	 (throwfail "Cannot delete binder from " (bdwff . gwff)
		    ", a logical symbol."))
	((boundwff-q bdwff)
	 (cdr bdwff))
	(t (throwfail "Cannot delete binder from an application."))))

\end{verbatim}

\subsection{Global Parameters and Flags}
The following are the flags and parameters controlling the output 
of the editing session. Note that there are also editor windows,
which have separate flags; type {\tt SEARCH "EDWIN" T} to see a list of these.

\begin{description}
\item [\indexflag{PRINTEDTFILE}]  The name of the file in which wffs are recorded.

\item [\indexflag{PRINTEDTFLAG}] 
If {\tt T}, a copy of the current editing in {\tt ED} will be printed into
the file given by {\tt PRINTEDTFILE}. The prompt will also be changed to {\tt -ED}
or {\tt }ED+.

\item [\indexflag{PRINTEDTFLAG-SLIDES}]  As {\tt PRINTEDTFLAG}, but the output is
in Scribe 18-point style.

\item [\indexflag{PRINTEDTOPS}]  contains the name of a function which tests whether
or not to print a particular wff to the {\tt PRINTEDTFILE}.

\item [\indexflag{VPD-FILENAME}]  is the equivalent of {\tt PRINTEDTFILE} for vertical
path diagrams.

\item [\indexflag{PRINTVPDFLAG}]  is the equivalent of {\tt PRINTEDTFLAG} for
vertical path diagrams.
\end{description}
The flags and parameters listed below
are the counterparts of flags described in full detail on page
~\ref{printflag}.  They have the identical meaning, except that they
are effective in the editor, while their counterparts are effective on
the top-level of \tps.
\begin{description}
\item [\indexflag{EDPPWFFLAG}] 
If {\tt T}, wffs in the editor will generally be pretty-printed.  Default is {\tt NIL}.

\item [\indexflag{EDPRINTDEPTH}] 
The value used as \indexflag{PRINTDEPTH} within the formula editor.  It is 
initialized to {\tt 0}.

\end{description}

\section{The formula parser}

\subsection{Data Structures}
\begin{description}

\item [\indexData{ByteStream}]
This list stores essentially the printing characters which are in its
input.  {\tt CR}, {\tt LF}, and {\tt TAB} characters are replaced with a space.  The
ending {\tt ESC} does not appear in this list.  All elements are {\tt INTERN}
identifiers. See the function \indexfunction{bytestream-tty} in {\it wffing.lisp}.

\item [\indexData{RdCList}]
This data structure appears to be all but obsolete; the last remnants 
of it are in the file {\it wffing.lisp}. {\tt RdC} refers to the Concept terminal.
This list contains either integers between 0 and 127, lists containing
precisely one of 0, 1, or 3, or the identifier \indexData{CRLF}.  The lists
represent character set switches, the integers represent characters,
and {\tt CRLF} represents a carriage return/line feed combination.

\item [\indexData{LexList}]
This is a list of lexical objects, i.e. it contains name for logical
objects which will appear in the fully parsed formula.  It also
contains the brackets "[","]", and ".".  It also contains the type
symbols from the initial input.  These are distinguishable from the
other items in the list since they are stored as lists.  Hence,
LexList is a "flat" list of these three things.

\item [\indexData{TypeAssoc}]
This is an association list which associates to those identifiers in
the LexList which got a type, that type.  This is necessary so that an
identifier which is typed explicitly at one place in the formula can
have that type attributed to it at other non-typed occurrences.

\item [\indexData{GroupList}]
This is essentially the same as {\tt LexList}, except that the bracket
identifiers are removed, and nested s-expressions are used to denote
groupings. Type symbols are also "attached" to the identifier
preceding it.  Hence a GroupList contains only logical identifiers -
some with types and some without - grouped in a hierarchical fashion.

\item [\indexData{PreWff}]
This data structure is like that of the wff structure, except that not
all items are correctly typed yet.  The full prefix organization is
present in this formula.  The types for polymorphic definitions,
however, are not yet computed.
\end{description}

\subsection{Processing}

Input is first processed into {\tt ByteStream}s and then into {\tt LexList}s
by the function \indexfunction{LexScan}.

\indexfunction{GroupScan} now operates on {\tt LexList} in order to construct the
{\tt GroupList}.  This function has no arguments and uses a special
variable, called \indexparameter{LexList}, to communicate with
recursive calls to itself.  {\tt GroupScan} is also responsible for building
the {\tt TypeAssoc} list.

\indexfunction{InfixScan} converts a {\tt GroupList} into a {\tt PreWff}.
This requires using
the standard infix parser.  \indexfunction{MakeTerm} is used to build the prefix
subformulas of the input. 

Now that all logical items appear in their final positions, the actual
types of polymorphic abbreviations can be determined.  This is the job
of \indexfunction{FinalScan}.  This function takes a {\tt PreWff} and 
returns with a {\tt WFF}.

This is not a very efficient algorithm. A few of the passes could be joined
together, and a few might be made more efficient by using destructive
changes.  The parser, however, is rather easy to upgrade.



\chapter{Help and Documentation}
\label{help}

\section{Providing Help}

When the user types the command \indexcommand{HELP object}, \TPS will first
try to determine which category {\tt object} is in (it may be in several, in which case it will 
produce a list of categories and then print the help for each separately).

Recall from the entry about categories (section ~\ref{categories}) that each category has 
{\tt mhelp-line} and {\tt mhelp-fn} properties. The {\tt mhelp-line} is a short phrase that describes 
each object in the category (for example the category {\tt PMPROPSYM} has the mhelp line 
"polymorphic proper symbol"). The mhelp-fn is a function that will print out the help for a specific
object. For many simple categories (e.g. {\tt context}), 
the function {\tt \indexother{princ-mhelp}} is sufficient; this simply 
prints the {\tt mhelp} string attached to the object in question. Other categories need more complex
help (for example {\tt mexpr}), and so have their own specially-defined mhelp functions.

An example of a {\tt mexpr} which has some automatically constructed help information
is the function \indexcommand{EXTRACT-TEST-INFO} in \indexfile{maint.lisp}.

When writing help messages or mhelp functions, keep in mind that the information given 
should contain all the information that a {\it user} would want to know. 
More detailed help for maintainers and programmers should 
be written down and incorporated into this manual, not added into the online documentation.

\subsection{Mhelp and Scribe}
The online documentation can be used to generate a facilities guide, so
it is important that you be aware that the mhelp properties and mhelp functions
you define for new objects or categories will be used to generate Scribe
files.  Take a look at the files {\it mhelp.lisp} and {\it scrdoc.lisp} and see
how this works.  You may need to set things up properly so that the entries you
are introducing are put into the index of such guides.  Look at the
file {\it tpsdoc.mss} in the doc/lib area to see how the indexing is done.

\subsection{Mhelp and \LaTeX }
The online documentation can also be used to generate a facilities guide using \LaTeX. 
Take a look at the files {\it mhelp.lisp} and {\it latexdoc.lisp} and see
how this works. Because of the restrictions of \LaTeX concerning special command characters, such as
``\$ '' or ``\textbackslash '', the help processing may need special care: an alternative to the usual
{\tt \indexother{princ-mhelp}} function is provided to handle such characters in {\it latexdoc.lisp} as
{\tt \indexother{princ-mhelp-latex}}. Many specific \LaTeX commands are defined in the {\it tps.tex} and
{\it tpsdoc.tex} files in the doc/lib area. The User's guide describes how the processing of such manuals
is done.
You may need to set things up properly so that the entries you
are introducing are put into the index of such guides.  Look at the
file {\it latexdoc.lisp} in the lisp area to see how the indexing is done.

\section{The Info Category}

There is a category of objects called \indexother{INFO} which is used solely for providing
help on symbols that would otherwise not have help messages (for example, the various settings
of some of the flags, such as PR97 or APPLY-MATCH-ALL-FRDPAIRS). You can attach a help message
to any symbol {\it foo} with:

%\begin{tpsexample}
\begin{verbatim}
(definfo foo (mhelp "Help text."))
\end{verbatim}
%\end{tpsexample}

\section{Printed Documentation}

The directories with root {\it /home/theorem} mentioned below are
on gtps.

{\it /home/theorem/project/doc/files.dir} contains information about
\TPS documentation files.

{\it /home/theorem/project/doc/etps/tps-cs.mss} describes how to
access \TPS in the cmu cs cell.

{\it /home/theorem/project/doc/etps/etps-andrew.mss} describes how
to access \ETPS in the andrew cell.

{\it /home/theorem/project/doc/<topic>/manual.mss} is the main file
for the manual on <topic> (one of: {\tt char}, {\tt etps}, {\tt facilities}, 
{\tt grader}, {\tt prog}, {\tt teacher} and {\tt user}).

See the \TPS User Manual for additional information.

When new facilities are added to \ETPS, copy the information
about them from the automatically produced {\it facilities.mss} and {\it facilities.tex} into the
appropriate \ETPS mss/tex file.

\section{Indexing in the Manuals}

The basic mechanisms are in {\it /home/theorem/project/doc/lib/index.lib}
and {\it /home/theorem/project/doc/lib/indexcat.mss}. Note the comment on
the use of {\tt @IndexCategory} in the former file.  In the \TeX
version of the Programmer's Guide, there are indexing commands defined
which mimic the role of the corresponding Scribe commands.

{\tt @indexother{DIY-TAC}} in the text on page <pagenumber> puts
{\tt DIY-TAC <pagenumber>} into the index.

{\tt @index*X*(WORD}) in the text on page <pagenumber> puts
{\tt WORD, *Y* <pagenumber>} into the index.

Example:
{\tt @indexcommand{DO-GRADES}} in the text on page <pagenumber> puts
{\tt DO-GRADES, System Command <pagenumber>}
into the index. Here is a partial list of 
of possible values for {\tt *X*} and {\tt *Y*}, where the complete
list is in {\it /home/theorem/project/doc/lib/indexcat.mss}.

\begin{itemize}
\item {\tt *X*} = command gives {\tt *Y*} = System Command
 
\item {\tt *X*} = edop gives {\tt *Y*} = Editor Command
 
\item {\tt *X*} = flag gives {\tt *Y*} = flag
 
\item {\tt *X*} = function gives {\tt *Y*} = function
 
\item {\tt *X*} = style gives {\tt *Y*} = style

\item {\tt *X*} = mexpr gives {\tt *Y*} = mexpr
\end{itemize}

See "quitting" in the index of the ETPS manual to see the 
effect of the following lines in the file
{\it /home/theorem/project/doc/etps/system.mss}:

%\begin{tpsexample}
\begin{verbatim}
@@seealso[Primary="Quitting",Other="@{\tt EXIT}"]
@@seealso[Primary="Quitting",Other="@{\tt END-PRFW}"]
@@seealso[Primary="Quitting",Other="@{\tt OK}"]
@@indexentry[key="Quitting",entry="Quitting"]
\end{verbatim}
%\end{tpsexample}

\section{Other commands in the manuals}

Any other Scribe commands may be used in the manuals; for example
we use the {\tt typewriter} font given by @t for command 
names, and the {\it italic} font given by @i for file names.

In the \TeX versions of the manuals, one uses the corresponding
\TeX commands.

We also have @TPS in Scribe (and \verb+\TPS+ in \TeX)
to print the string "\TPS", and @HTPS in Scribe to do the
same in headers.

\section{Converting Scribe to \LaTeX ~documentation}

The aime of this section is to provide helpful information on how to
program a new documentation device.

\subsection{The {\it latexdoc.lisp} file}

This file was written as an equivalent to {\it scrdoc.lisp}. Functions and macros
are essentially equivalent. Nevertheless, while the Scribe documentation
system contains several calls to special function, such as {\tt scribe-one-fn}, which
are internal properties of special objects (e.g. tactics), {\it latex-doc.lisp} contains
every \LaTeX -specific help-formatting function.

Some of these functions uses the tex style in the \TPS system, which is described in
{\it deftex.lisp}. As the style properties were thought to be used with Scribe, some
of them cannot be easily translated in \LaTeX . This obstacle occurs for instance when
using the Scribe \verb+@Tabset+ function, which has no strict equivalent in \LaTeX , where it
is usually replaced by the \verb+Description+ environment.

\subsection{Special Characters}

The mhelp properties of many \TPS objects present special characters: when Scribe prevents the use
of \@ , a lot of characters have to be escaped, using a prefixed ``\textbackslash '', when generating a \LaTeX ~document. The most common are: \~ , \@ , \# , \$ , \% , and \& . Note that the backslash character in \LaTeX ~is
\verb+\textbackslash+ and the character ``\textasciicircum '' is \verb+\textasciicircum+.

In order to prevent the programmer from editing every mhelp property, a function {\tt princ-mhelp-latex} is used
instead of the regular {\tt princ-mhelp}. This function simply replaces every occurence of a protected character
by the correspondant \LaTeX ~sequence.

\subsection{\LaTeX ~Macros}

To facilitate the conversion from Scribe to \LaTeX ~commands, a great number of new commands and macros for \LaTeX
~are defined in {\it tps.tex} and {\it tpsdoc.tex}. These commands enable us to use commands such as \verb+\greeka+ instead of the regular \verb+\alpha+.
\chapter{Setting and Varying Flags}
\label{flags}

%Some facilities which can be helpful in finding modes
%appropriate for proving particular theorems automatically are
%discussed in section \ref{modes}.  





The settings of flags are very important during automatic search.  For
example, flags like \indexflag{REWRITE-DEFNS} and
\indexflag{REWRITE-EQUALITIES} must be set to T (true) if you want
definitions and equalities, respectively, to be rewritten when the
expansion tree is created from the formula to be proved.  To see all
of the flags which will affect mating-search, check the facilities
guide.

Since there are many flags in {\TPS}, the flags which are most crucial
to mating search have been collected together under the subject
\indexother{IMPORTANT}, so that typing LIST IMPORTANT will produce a
list of the settings of the dozen or so most important flags for
mating search.

{\TPS} has a facility called \indexcommand{UNIFORM-SEARCH},
which attempts to find the correct flag settings
on its own; it is capable of finding correct settings for many of the theorems
which {\TPS} has proven to date. See section \ref{testtop} for more details.


\section{Review, flags, and modes}


The Review top-level is provided to make it easier to update flag-settings.
Enter the top-level with the command \indexcommand{REVIEW}.  The commands provided
in this top-level (for a listing, enter \indexcommand{?}) give information about the
various flags available.  Each flag is listed under one or more subjects.
You may use the command \indexcommand{SETFLAG} to change the value of a flag, or
you may simply type the name of the flag itself. When entering the argument for
a flag, you may type \indexcommand{??} for more information about the argument
type. In some cases, {\TPS} will warn the user that the flag is irrelevant, given
the settings of other flags.
File names and extensions should be entered as strings (i.e. surrounded
by double quotes); although in some circumstances one can omit the quotes, it
is never an error to include them.

All the flags for a subject can be listed by using the \indexcommand{LIST} command;
{\tt help list} will tell you more. You may search the help messages of flags by
using the \indexcommand{KEY} command; for example {\tt key `unification' all} will
find the flags whose help messages mention unification, in all subjects. A similar
command \indexcommand{SEARCH} allows you to search the help messages of all {\TPS}
objects.

To update many flags at once, use \indexcommand{UPDATE} or 
\indexcommand{UPDATE-RELEVANT}.
The \indexcommand{UPDATE} commmand 
allows the user to set all the flags in given subjects.
Since certain flag values render some other flags
irrelevant, there is a command called \indexcommand{UPDATE-RELEVANT} which
allows users to conveniently take into account the relationships
between flags and the decisions previously made while setting flags
interactively.  Since proof procedures are constantly being developed
or refined, when \indexcommand{UPDATE-RELEVANT} is called, the user is given the
option of using the current flag relevancy information in memory,
loading flag relevancy information which has been saved to a file, or
rebuilding flag relevancy information automatically by processing the Lisp
source files.  

Modes are collections of flags with specified settings.  By selecting a
mode, you simultaneously set all the included flags.  You can also
define your own modes and save them.  See the \indexcommand{INSERT} command,
in the library top level, for more information.

See section \ref{searchanalysis} for information on how to use
a known proof to suggest {\TPS} flag settings, and Chapter
\ref{ms-guide} for information about certain flags which affect the
search processes.

Currently defined flags and modes are listed in the facilities guide
\cite{AndrewsTPS88e}.

\section{Test: Multiple Searches on the Same Problem}\label{testtop}

The \indexcommand{TEST} top level is designed to allow a succession of searches with
varying flag settings to be performed without any interaction from the user.

One of the major uses of this top level is to find settings of the various
flags which will produce a proof of a theorem; this is done by the
\indexcommand{UNIFORM-SEARCH} command, which is explained later in
section \ref{uniform}.

The top level is entered with the \indexcommand{TEST} command; as with
the mate top level, this will prompt for a gwff and ask whether you
want to open a vpwindow.

There are three major groups of TEST commands; those which are for
specifying a list of flags to be varied, those which deal with the
associated library functions and those which actually perform the search.

There is one major new data structure: a searchlist is a list of search items;
each search item contains the name of a flag, a default (initial) value for the flag,
and a range of values over which it should be varied. The user can have
several searchlists in memory at once, and can switch between them
with the \indexcommand{NEW-SEARCHLIST} command. See below for an example of how to
construct a searchlist. Searchlists can be stored in the library with the
\indexcommand{INSERT} command (from the test top level), and retrieved with the
\indexcommand{FETCH} command (also from the test top level). There is an additional
(optional) function attached to each searchlist; this will be called on each iteration of
the search, and may do such things as setting flags which are not in the searchlist (so, for
example, \indexflag{MAX-SEARCH-DEPTH} and \indexflag{MAX-UTREE-DEPTH} can be kept equal to
each other by having one in the searchlist and the other set by the optional function).
The function UNIFORM-SEARCH-FUNCTION is used by the UNIFORM-SEARCH procedures.

The searchlist also contains (invisible to the user) internal flags which determine the current
position in the search; this means that it is possible to interrupt a search,
save the current searchlist in the library, and later on to reload that searchlist
(even into a different core image) and continue the search from the point where it was
interrupted.

Once a searchlist is constructed, the \indexcommand{GO} command will start a search
and the \indexcommand{CONTINUE} command will continue a search after an interruption.
The flags in the subject \indexsubject{test-top} control many of the parameters of the
search; type {\tt list test-top} for more information. The user has the option of opening
a test-window, analogous to the vpwindow, which will show a summary of the search so far.
If the search finds a proof, it will terminate; otherwise, it will increase the time limit
in the flag \indexflag{TEST-INITIAL-TIME-LIMIT} and try again. Once the search terminates,
it will define a new mode which can be saved in the library using \indexcommand{INSERT}
{\it from the test top level}. (The \indexcommand{INSERT} in the library top level is different;
when you try to save a mode, it asks you to specify all the flag settings. The \indexcommand{INSERT}
in the test top level merely asks for the name of the mode.)

\subsection{How to Build A Searchlist Without Any Effort}

The commands \indexcommand{VARY-MODE} and \indexcommand{QUICK-DEFINE} are
useful for defining searchlists quickly. The former takes an existing mode,
and steps through it flag by flag, building a searchlist by offering to
add each flag of the mode and an appropriate range to the new searchlist.
The latter uses a pre-defined list of flags, either those listed in the
\indexflag{TEST-FASTER-*} group or those listed in the \indexflag{TEST-EASIER-*}
group, to create a searchlist in which each of the flags has up to
\indexflag{TEST-MAX-SEARCH-VALUES} possible values.

Given a large searchlist, it is possible to trim it down using the \indexcommand{SCALE-UP}
and \indexcommand{SCALE-DOWN} commands; for each of the flags in the TEST-EASIER-* or TEST-FASTER-*
lists (respectively), these commands compare the range of the flags in the
searchlist with the current values of the flags, and remove all values that would not make the
search easier (in the case of SCALE-UP) or faster (in the case of SCALE-DOWN).

In some cases, it is not necessary to build a searchlist at all. The commands \indexcommand{PRESS-DOWN},
\indexcommand{PUSH-UP} and \indexcommand{FIND-BEST-MODE} all build their own searchlists
using the flags listed in the TEST-FASTER-* flags (for PRESS-DOWN and FIND-BEST-MODE)
and those listed in the TEST-EASIER-* flags (for PUSH-UP and FIND-BEST-MODE). Below are
examples of how to use these commands.

\subsection{Using TEST to Improve a Successful Mode}
\begin{tpsexample}
<18>test sample-theorem !
<test19>mode mode-sample-theorem
{\it mode-sample-theorem is a mode in which sample-theorem can be proven.}
<test20>list test-top

Subject: TEST-TOP
  TEST-FASTER-IF-HIGH:   (MIN-QUICK-DEPTH)
  TEST-FASTER-IF-LOW:    (MAX-SEARCH-DEPTH SEARCH-TIME-LIMIT MAX-UTREE-DEPTH
                          MAX-MATES MAX-SEARCH-LIMIT)
  TEST-FASTER-IF-NIL:    ()
  TEST-FASTER-IF-T:      (MIN-QUANTIFIER-SCOPE MS-SPLIT)
  TEST-FIX-UNIF-DEPTHS:  T
  TEST-INCREASE-TIME:    0
  TEST-INITIAL-TIME-LIMIT: 30
  TEST-MAX-SEARCH-VALUES: 10
  TEST-REDUCE-TIME:      T
{\it Some flags have been omitted. When we do PRESS-DOWN, it will
automatically create a searchlist which will vary each
of the flags listed in the TEST-FASTER-* flags above, and then
start searching with a maximum time of 30 seconds, decreasing the
time as it goes, and fixing the unification depths after the
first successful search.}
<test21>press-down
{\it The search is started. While it works, we can go away for a
cup of coffee (or a two-week vacation to Mexico, depending on how
difficult sample-theorem is).}
\end{tpsexample}

\subsection{Using TEST to Discover a Successful Mode}
\begin{tpsexample}
<18>test sample-theorem !
<test19>mode bad-mode
{\it bad-mode is a mode in which sample-theorem cannot be proven.}
<test20>list test-top

Subject: TEST-TOP
  TEST-EASIER-IF-HIGH:   (MAX-SEARCH-DEPTH SEARCH-TIME-LIMIT NUM-OF-DUPS
                          MAX-UTREE-DEPTH MAX-MATES MAX-SEARCH-LIMIT)
  TEST-EASIER-IF-LOW:    (MIN-QUICK-DEPTH)
  TEST-EASIER-IF-NIL:    ()
  TEST-EASIER-IF-T:      (ETA-RULE MIN-QUANTIFIER-SCOPE MS-SPLIT)
  TEST-INCREASE-TIME:    10
  TEST-INITIAL-TIME-LIMIT: 30
  TEST-MAX-SEARCH-VALUES: 10
{\it Some flags have been omitted. When we do PUSH-UP, it will
automatically create a searchlist which will vary each
of the flags listed in the TEST-EASIER-* flags above, and then
start searching with a maximum time of 30 seconds, increasing the
time by ten percent on each attempt.}
<test21>push-up
{\it The search is started. It will stop as soon as it finds a mode which works.
We can combine both PUSH-UP and PRESS-DOWN (first push-up to find a successful
mode, then press-down to find a better mode) by using the command FIND-BEST-MODE
instead of PUSH-UP.}
\end{tpsexample}

\subsection{Building A Searchlist with TEST}
\begin{tpsexample}
<12>list test-top
Subject: TEST-TOP
  TEST-INITIAL-TIME-LIMIT: 30
  TEST-NEXT-SEARCH-FN:   EXHAUSTIVE-SEARCH
  TEST-REDUCE-TIME:      T
  TEST-VERBOSE:          T
  TESTWIN-HEIGHT:        24
  TESTWIN-WIDTH:         80
{\it These are the flags available. Note the first three; the first says that
each search will be allowed 30 seconds, the second that the function
EXHAUSTIVE-SEARCH will be used to determine which flag settings are used next,
and the third that if a proof is found, the time allowed for the next proof
will be decreased.}
<13>test
GWFF (GWFF0-OR-EPROOF): Gwff or Eproof [No Default]>x5305
DEEPEN (YESNO): Deepen? [Yes]>
WINDOW (YESNO): Open Vpform Window? [No]>yes
File to send copy Vpform output to (`` '' to discard) [``vpwin.vpw'']>``test-top.vpw''

Use CLOSE-MATEVPW when you want to close the vpwindow.
<test14>?
Top Levels:    LEAVE
Mating search: CLOSE-TESTWIN CONTINUE GO OPEN-TESTWIN SEARCH-ORDER
Searchlists:   ADD-FLAG ADD-FLAG* ADD-SUBJECTS NEW-SEARCHLIST REM-FLAG
               REM-FLAG* SEARCHLISTS SHOW-SEARCHLIST
Library:       DELETE FETCH INSERT

<test15>new-searchlist
NAME (SYMBOL): Name of new searchlist [No Default]>x5305-search
{\it We begin to define a new searchlist, containing the flags to be varied.}
<test16>searchlists
X5305-SEARCH

{\it ...this is currently the only searchlist in memory...}
<test17>add-flag*
FLAG (TPSFLAG): Flag to be added [No Default]>max-search-depth
INIT (ANYTHING): Initial value of flag [No Default]>42
RANGE (ANYTHING-LIST): List of possible values to use [No Default]>4 8 10

Add another flag?  [Yes]>

FLAG (TPSFLAG): Flag to be added [No Default]>max-utree-depth
INIT (ANYTHING): Initial value of flag [No Default]>20
RANGE (ANYTHING-LIST): List of possible values to use [No Default]>4 8 10

Add another flag?  [Yes]>no

{\it We have added two flags to our searchlist, with four values for each.
Note also that there are commands to add a single flag (ADD-FLAG) or
every flag in a given list of subjects (ADD-SUBJECTS), as well as
commands to remove a flag or flags (REM-FLAG, REM-FLAG*).}
<test18>show-searchlist
NAME (SYMBOL): Name of searchlist [X5305-SEARCH]>
Searchlist X5305-SEARCH is as follows:
MAX-UTREE-DEPTH = 20, default is 20, range is [(20 4 8 10)]
MAX-SEARCH-DEPTH = 42, default is 42, range is [(42 4 8 10)]

{\it Next, we decide to check how many searches there will be, and in what
order they will be performed.}
<test19>search-order
NAME (SYMBOL): Name of searchlist [X5305-SEARCH]>
There are 16 possible settings of these flags.
Search : (MAX-UTREE-DEPTH = 20) (MAX-SEARCH-DEPTH = 42)
Search : (MAX-UTREE-DEPTH = 10) (MAX-SEARCH-DEPTH = 42)
Search : (MAX-UTREE-DEPTH = 8) (MAX-SEARCH-DEPTH = 42)
Search : (MAX-UTREE-DEPTH = 4) (MAX-SEARCH-DEPTH = 42)
Search : (MAX-UTREE-DEPTH = 20) (MAX-SEARCH-DEPTH = 10)
Search : (MAX-UTREE-DEPTH = 10) (MAX-SEARCH-DEPTH = 10)
Search : (MAX-UTREE-DEPTH = 8) (MAX-SEARCH-DEPTH = 10)
Search : (MAX-UTREE-DEPTH = 4) (MAX-SEARCH-DEPTH = 10)
Search : (MAX-UTREE-DEPTH = 20) (MAX-SEARCH-DEPTH = 8)
Search : (MAX-UTREE-DEPTH = 10) (MAX-SEARCH-DEPTH = 8)
Search : (MAX-UTREE-DEPTH = 8) (MAX-SEARCH-DEPTH = 8)
Search : (MAX-UTREE-DEPTH = 4) (MAX-SEARCH-DEPTH = 8)
Search : (MAX-UTREE-DEPTH = 20) (MAX-SEARCH-DEPTH = 4)
Search : (MAX-UTREE-DEPTH = 10) (MAX-SEARCH-DEPTH = 4)
Search : (MAX-UTREE-DEPTH = 8) (MAX-SEARCH-DEPTH = 4)
Search : (MAX-UTREE-DEPTH = 4) (MAX-SEARCH-DEPTH = 4)

{\it Because TEST-NEXT-SEARCH-FN is EXHAUSTIVE-SEARCH, the search will try
every possible combination of these flags, as shown above.}

<test20>go
MODENAME (SYMBOL): Name for optimal mode [TEST-BESTMODE]>x5305-bestmode
TESTWIN (YESNO): Open a window for test-top summary? [No]>yes
File to send test-top summary to (`` '' to discard) [``info.test'']>

Use CLOSE-TESTWIN when you want to close the testwindow.

Changing flag settings as follows:
(MAX-UTREE-DEPTH = 20) (MAX-SEARCH-DEPTH = 42)

{\it lots of output omitted...}

The time used in each process:
-----------------------------------------------------------------------------
Process Name         | Realtime | Internal-runtime |  GC-time   | I-GC-time
-----------------------------------------------------------------------------
                                (Interrupted)
Mating Search        |       77 |            52.13 |       0.00 |      52.13
                      (1.3 mins)
-----------------------------------------------------------------------------

Changed MAX-UTREE-DEPTH = 20, default is 20, range is [(20 4 8 10)]

Changed MAX-SEARCH-DEPTH = 42, default is 42, range is [(42 4 8 10)]
Search finished.Changing flag settings as follows:
(MAX-UTREE-DEPTH = 20) (MAX-SEARCH-DEPTH = 42)
Finished varying flags; succeeded in proof.
Best mode was ((MAX-SEARCH-DEPTH 42) (MAX-UTREE-DEPTH 8))
Best time was 28.667

Have defined a mode called X5305-BESTMODE; use INSERT to put this into the library.

{\it We ran the search; it terminated and defined a new mode for us.}
<test21>help X5305-BESTMODE
X5305-BESTMODE is a mode.
A mode resulting from a test-search.
Flags are set as follows:
   Flag                     Value in Mode         Current Value
   MAX-SEARCH-DEPTH         42                    42
   MAX-UTREE-DEPTH          8                     20

<test22>close-testwin
Closed test-window file : info.test
Shall I delete the output file info.test?  [No]>yes

\end{tpsexample}

\subsection{Uniform Search: Finding Successful Modes Automatically}\label{uniform}

There are two top-level commands in {\TPS} which search for a successful mode for
proving a theorem, without requiring the user to master the \indexcommand{TEST} top level first.
They are \indexcommand{UNIFORM-SEARCH} and \indexcommand{UNIFORM-SEARCH-L}.

\indexcommand{UNIFORM-SEARCH-L} is analogous to \indexcommand{DIY-L}; it attempts
to find a successful mode which will allow {\TPS} to fill in a subproof during the
interactive construction of a larger theorem. Apart from the fact that it generates
subproofs within a given range of lines, it works in exactly the same way as
\indexcommand{UNIFORM-SEARCH}, and so the rest of this section is devoted entirely to
\indexcommand{UNIFORM-SEARCH}.

\indexcommand{UNIFORM-SEARCH} takes three essential arguments: a gwff which is to be proven,
a mode and a searchlist. By default, the mode is called UNIFORM-SEARCH-MODE and the
searchlist is called UNIFORM-SEARCH-2; if these objects are not present in your library,
you should provide appropriate alternatives.

The idea is that {\TPS} will set all of the flags to the values given in UNIFORM-SEARCH-MODE,
and then vary the flags as prescribed by UNIFORM-SEARCH-2. So UNIFORM-SEARCH-MODE should
be a `neutral' mode which does not constrain the search very much, and which sets all of
the less important flags to reasonable values.

UNIFORM-SEARCH-2 might be defined as follows (for example):
\begin{tpsexample}
DEFAULT-MS = MS91-6, default is MS91-6, range is [(MS91-6 MS91-7)]
MAX-SUBSTS-VAR = 3, default is 3, range is [(3 5 7)]
MAX-MATES = 1, default is 1, range is [(1 2 3 4 6)]
NUM-OF-DUPS = 0, default is 0, range is [(0 1 2)]
REWRITE-EQUALITIES = ALL, default is ALL, range is [(ALL LAZY2 LEIBNIZ)]
REWRITE-DEFNS = (EAGER), default is (EAGER), range is [((EAGER) (LAZY2))]
SEARCH-TIME-LIMIT = 30, default is 30, range is [(30 120 240 920 1840 3600 7200)]
...plus the function UNIFORM-SEARCH-FUNCTION
\end{tpsexample}

As you can see, this varies the most important flags in {\TPS} over a reasonable range of values.
Once you have entered the TEST top level with UNIFORM-SEARCH, all that remains is to type GO,
and wait for a proof. Of course, proofs in UNIFORM-SEARCH usually take longer to be found than proofs in which
the correct mode is already known.

If a proof is found, a new mode will be defined, which can be stored in the library, and by merging the etree
and then calling NAT-ETREE this proof can be translated into a natural deduction proof, exactly
as in the MATE top level.

One final word about UNIFORM-SEARCH: it will also offer to modify the searchlist for you.
This will speed up the search, if possible, by removing those flags in the searchlist
which will have no effect on the proof of the given gwff. In particular, it will remove
unification depths for first-order gwffs, REWRITE-DEFNS for gwffs that contain no definitions,
REWRITE-EQUALITIES for those that contain no equalities, and so on. It will also change the
settings of DEFAULT-MS to MS88 and/or MS90-3 if it determines that there are no primitive
substitutions for the given gwff.


\section{Search Analysis: Facilities for Setting Flags and Tracing Automatic Search}\label{searchanalysis}

If one has a natural deduction proof of a theorem, one can use the
command \indexcommand{AUTO-SUGGEST} to obtain suggested settings for certain flags of
a mode with which that theorem can be proved automatically.
\indexcommand{AUTO-SUGGEST} will also show all of the instantiations of quantifiers
that are necessary for that proof.  The command \indexcommand{ETR-AUTO-SUGGEST} does
the same thing when given an expansion proof. Such an expansion proof
could be the result of translating a natural deduction proof into an
expansion proof using the command \indexcommand{NAT-ETREE}.
The commands \indexcommand{MS03-LIFT} and \indexcommand{MS04-LIFT} 
in the \indexother{EXT-MATE} top level
suggests flag settings for the corresponding extensional search
procedures when given an extensional expansion proof.

One can obtain an expansion proof for a gwff by several methods,
including constructing a mating by hand
in the \indexcommand{MATE} top level.
An easier way is to use \indexcommand{NAT-ETREE} (see section \ref{natetr})
to translate a natural deduction proof into an expansion proof.
Once we have an expansion proof, we can use this to suggest
flag values and to trace the \indexother{MS98-1} search procedure.

In this section we will consider two examples: THM12
and X2116.

\subsection{Example: Setting Flags for THM12}\label{THM12-NAT-ETREE}

Our first example concerns THM12:

$\forall R _{\greeko\greeki}   \forall S _{\greeko\greeki} .R = S \implies  \forall X _{\greeki} .S X \implies R X$

The following excerpts from a TPS session shows how we can
use \indexcommand{NAT-ETREE} to get an expansion proof
and then use this expansion proof to suggest flag settings.

\begin{tpsexample}
<401>prove THM12
. . . ; ***prove the theorem, perhaps interactively***
<431>pstatus
 No planned lines
<432>nat-etree
PREFIX (SYMBOL): Name of the Proof [THM12]>
Proof THM12-2 restored.

. . . ; ***nat-etree preprocesses the proof,
. . . ;    converts it to a sequent calculus derivation,
. . . ;    performs cut elimination
. . . ;    then translates to an expansion tree with a complete mating.***
|            L117             |
| \(\sim\)R^152 X^141 OR S^17 X^141  |
|                             |
|            L115             |
| \(\sim\)S^17 X^141 OR R^152 X^141  |
|                             |
|            L116             |
|\(\sim\)[\(\sim\)S^17 X^141 OR R^152 X^141]|
Number of vpaths: 1
((L115  . L116 ))    ; note this is a connection between nonatomic wffs

Adding new connection: (L115 . L116)
If you want to translate the expansion proof back to a natural deduction proof,
you must first merge the etree.  If you want to use the expansion proof to determine
flag settings for automatic search, you should not merge the etree.
Merge The Etree? [Yes]>n           ; ***we don't merge the tree***
The expansion proof Eproof:EPR122  can be used to trace MS98-1 search procedure.
The current natural deduction proof THM12-2 is a modified version
of the original natural deduction proof.

Use RECONSIDER THM12 to return to the original proof.

<433>mate
GWFF (GWFF0-OR-LABEL-OR-EPROOF): Gwff or Eproof [Eproof:EPR122 ]>
DEEPEN (YESNO): Deepen? [Yes]>n
REINIT (YESNO): Reinitialize Variable Names? [Yes]>n
WINDOW (YESNO): Open Vpform Window? [No]>
. . .
<Mate437>etr-info  ; ***etr-info lists the expansion terms***
Expansion Terms:
X^141(I)           ; ***only one (easy) expansion term in the proof***

<Mate438>etr-auto-suggest
. . .
MS98-INIT suggestion: 1
MAX-SUBSTS-VAR should be 1
NUM-OF-DUPS should be 0
MS98-NUM-OF-DUPS should be 1
MAX-MATES should be 1
Do you want to define a mode with these settings? [Yes]>
Name for mode?  [MODE-THM12-2-SUGGEST]>

<Mate439>leave
Merge the expansion tree? [Yes]>n   ; ***let's still not merge***


<440>mode MODE-THM12-2-SUGGEST
\end{tpsexample}

This suggested mode will prove the theorem.  In a later
section we will continue this example to see how we can
use the eproof to trace the \indexother{MS98-1} search procedure.

\subsection{Example: Setting Flags for X2116}\label{X2116-NAT-ETREE}

Suppose we have a natural deduction proof for X2116:

$\forall x _{\greeki}   \exists y _{\greeki}  [P _{\greeko\greeki}  x \implies R _{\greeko\greeki\greeki}  x [g _{\greeki\greeki}  .h _{\greeki\greeki}  y] \and P y] \and  \forall w _{\greeki}  [P w \implies P [g w] \and P .h w]
 \implies  \forall x .P x \implies  \exists y .R x y \and P y$

The following excerpts from a TPS session shows how we can
use \indexcommand{NAT-ETREE} to get an expansion proof
and then use this expansion proof to suggest flag settings.

\begin{tpsexample}
<405>nat-etree
PREFIX (SYMBOL): Name of the Proof [X2116]>
. . .
|              |        L90        |   |
|    L89       |R x^329 [g .h y^70]|   |
|  \(\sim\)P x^329 OR |                   |   |
|              |       L92         |   |
|              |      P y^70       |   |
|                                      |
|                   |    L102     |    |
|        L99        |P [g .h y^70]|    |
|    \(\sim\)P [h y^70] OR |             |    |
|                   |    L104     |    |
|                   |P [h .h y^70]|    |
|                                      |
|                  |   L100   |        |
|         L96      |P [g y^70]|        |
|       \(\sim\)P y^70 OR |          |        |
|                  |   L98    |        |
|                  |P [h y^70]|        |
|                                      |
|                 L88                  |
|               P x^329                |
|                                      |
|        L91                  L103     |
|\(\sim\)R x^329 [g .h y^70] OR \(\sim\)P [g .h y^70]|
Number of vpaths: 16
((L102  . L103 ) (L98  . L99 ) (L92  . L96 ) (L88  . L89 ) (L90  . L91 ) (L88  . L89 ))

. . .
If you want to translate the expansion proof back to a natural deduction proof,
you must first merge the etree.  If you want to use the expansion proof to determine
flag settings for automatic search, you should not merge the etree.
Merge The Etree? [Yes]>n
The expansion proof Eproof:EPR116  can be used to trace MS98-1 search procedure.
The current natural deduction proof X2116-1 is a modified version
of the original natural deduction proof.

Use RECONSIDER X2116 to return to the original proof.

<406>mate
GWFF (GWFF0-OR-LABEL-OR-EPROOF): Gwff or Eproof [Eproof:EPR116 ]>
DEEPEN (YESNO): Deepen? [Yes]>n
REINIT (YESNO): Reinitialize Variable Names? [Yes]>n
WINDOW (YESNO): Open Vpform Window? [No]>n
. . .
<Mate409>show-mating   ; ***note the mating has six connections***

Active mating:
(L88 . L89)  (L90 . L91)  (L88 . L89)
(L92 . L96)  (L98 . L96.1)  (L100.1 . L103)

<Mate410>etr-info  ; ***there are 4 expansion terms, with reasonable sizes***
Expansion Terms:
g(II).h(II) y^70(I)
h(II) y^70(I)
y^70(I)
x^329(I)

<Mate411>etr-auto-suggest ; ***to get suggested flag settings***
. . .
MS98-INIT suggestion: 1
MAX-SUBSTS-VAR should be 3
NUM-OF-DUPS should be 1
MS98-NUM-OF-DUPS should be 1
MAX-MATES should be 1
Do you want to define a mode with these settings? [Yes]>
Name for mode?  [MODE-X2116-1-SUGGEST]>


<Mate412>mode MODE-X2116-1-SUGGEST
\end{tpsexample}

These flag settings are sufficient for \indexother{MS98-1} to prove the
theorem.  Later we will continue this example to show how
to use the expansion proof to trace \indexother{MS98-1}.

\subsection{Tracing MS98-1}\label{ms98-trace}

Currently there are some facilities for tracing the automatic search
procedure \indexother{MS98-1} (see Matt Bishop's thesis for details
of this search procedure).  The flag \indexflag{MS98-VERBOSE} can
be set to {\tt T} to simply obtain more output.  If we have an expansion
proof already given, we can obtain a finer trace on MS98-1 to find
information about the search.  For example, we may be able to find
which connections failed to be added to the mating.

A background expansion proof may be the result of calling
\indexcommand{NAT-ETREE} as described above.
We may also construct a complete mating interactively
in the \indexcommand{MATE} top level, then use
\indexcommand{SET-BACKGROUND-EPROOF} to save this expansion
proof as the one to use for tracing.  Consider the following
example session showing how we might get a mating
for X2116 (instead of using \indexcommand{NAT-ETREE}
as in section \ref{X2116-NAT-ETREE}).

\begin{tpsexample}
<0>exercise x2116
Would you like to load a mode for this theorem? [No]>y
2 modes for this theorem are known:
1) MODE-X2116  1999-04-23  0 seconds  (read only)
2) MS98-FO-MODE  1999-04-23  1 seconds  (read only)
3) None of these.
Input a number between 1 and 3: [1]>
. . .

<1>mate 100 y n n
<Mate2>go
. . .
Eureka!  Proof complete..
. . .
<Mate3>show-mating

Active mating:
(L21.1 . L14.1)  (L20.1 . L9)  (L12.1 . L15)
(L7.1 . L10)  (L12 . L10)  (L7 . L17)
(L21 . L15)
is complete.

<Mate4>set-background-eproof
EPR (EPROOF): Eproof [Eproof:EPR0 ]>
\end{tpsexample}

Once there is a background expansion proof,
the information given by the mating in the background proof
can be transferred to a search expansion tree via `colors'.
For each connection, we create a `color' (really just
a generated symbol) which is associated with all nodes
in the search expansion tree which could correspond to the connected
nodes in the background tree.  The examples in subsections
\ref{THM12-NAT-ETREE} and \ref{X2116-NAT-ETREE}
should make the role of colors clearer.

The flag \indexflag{MS98-TRACE} can
be used to obtain information about how the search is performing
relative to the background eproof.  The value of this flag is a list of
values.  Ordinarily, when tracing is off, \indexflag{MS98-TRACE}
will have the value NIL.  Certain symbols have the following meanings
if they occur on the value of \indexflag{MS98-TRACE}:

\begin{description}
\item[] {\tt MATING} -- If the symbol MATING is on the list,
search as usual, printing when `good' connections and components
are found.
A `good' connection is one in which the two nodes share a common color.
A `good' component contains only good connections.
A search succeeds when it generates all the good connections, and combines
these into good components, merging these components until the complete mating
is generated.  Note that successfully generating connections and merging components
depends on unification, and in particular, on the value of \indexflag{MAX-SUBSTS-VAR}.

\item[] {\tt MATING-FILTER} -- This prints the same information as {\tt MATING}, but
only generates good connections and only builds good components.
This value is useful for finding if the unification bounds (\indexflag{MAX-SUBSTS-VAR})
are set in such a way that the search can possibly succeed.

\end{description}

After setting \indexflag{MS98-VERBOSE} and \indexflag{MS98-TRACE},
one can invoke the search procedure \indexother{MS98-1} using
the mate command \indexcommand{MS98-1} or the mate command
\indexcommand{GO} if \indexflag{DEFAULT-MS} is set to \indexother{MS98-1}.

The next two subsections show how the tracing works
in practice by continuing the examples in subsections
\ref{THM12-NAT-ETREE} and \ref{X2116-NAT-ETREE}.

\subsection{Example: Tracing THM12}\label{THM12-MS98-TRACE}

Suppose the background expansion proof is a proof for THM12,
and suppose we are in a mode such as the suggested mode
from \indexcommand{ETR-AUTO-SUGGEST} in section \ref{THM12-NAT-ETREE}.

First, examine the background expansion proof more closely.
The expansion proof and jform are shown in Figure \ref{thm12-vp}.

\begin{figure}
\begin{tpsexample}
<459>ptree*
                                        [SEL11]
                                      [ R^152(OI) ]
                                            |
                                            |
                                            |
                                        [SEL10]
                                      [ S^17(OI) ]
                                            |
                                            |
                                            |
                                        [IMP86]
            R^152(OI) = S^17(OI) IMPLIES FORALL X(I).S^17 X IMPLIES R^152 X
                                            |
                      /-------------------------------------------\
                      |                                           |
                  [REW10]                                      [SEL9]
                   [EXT=]                                   [ X^141(I) ]
                      |                                           |
                      |                                           |
                      |                                           |
                   [EXP7]                                      [L116]
                [ X^141(I) ]                  S^17(OI) X^141(I) IMPLIES R^152(OI) X^141
                      |                                           |
                      |
                      |
                   [REW9]
                   [EXT=]
                      |
                      |
                      |
                   [REW8]
              [EQUIV-IMPLICS]
                      |
                      |
                      |
                  [CONJ74]
                      *
                      |
           /---------------------\
           |                     |
        [L117]                [L115]
           * S^17(OI) X^141(I) IMPLIES R^152(OI) X^141

<460>vp ; ***JFORM***
|                       SEL11                        |
|EXISTS R EXISTS S [R = S AND EXISTS X .S X AND \(\sim\)R X]|
|                                                    |
|                        L117                        |
|             \(\sim\)R^152 X^141 OR S^17 X^141             |
|                                                    |
|                        L115                        |
|             \(\sim\)S^17 X^141 OR R^152 X^141             |
|                                                    |
|                       L116                         |
|           \(\sim\)[\(\sim\)S^17 X^141 OR R^152 X^141]            |
Number of vpaths: 1

<461>show-mating   ; ***Complete Mating***
(L115 . L116)
\end{tpsexample}
\caption{Expansion Tree and JForm for THM12}
\label{thm12-vp}
\end{figure}

There is only one connection, so this should correspond to a single
color.  This is an interesting example because the search expansion tree will
have a somewhat different form.  Since an EQUIV occurs in the formula,
the form of the tree will depend on the value of \indexflag{REWRITE-EQUIVS}.

First, suppose \indexflag{REWRITE-EQUIVS} is set to 1.  The search
expansion tree in this case is shown in Figure \ref{thm12-search-etree1}.

\begin{figure}
\begin{tpsexample}
<466>ptree*
                                        [SEL11]
                                      [ R^152(OI) ]
                                            |
                                            |
                                            |
                                        [SEL10]
                                      [ S^17(OI) ]
                                            |
                                            |
                                            |
                                        [IMP86]
                                            |
                                            |
                      /-------------------------------------------\
                      |                                           |
                  [REW10]                                      [SEL9]
                   [EXT=]                                   [ X^141(I) ]
                      |                                           |
                      |                                           |
                      |                                           |
                   [EXP7]                                      [L116]
                [ X^141(I) ]                                      |
                      |                                           |
                      |
                      |
                   [REW9]
                   [EXT=]
                      |
                      |
                      |
                   [REW8]
              [EQUIV-IMPLICS]
                      |
                      |
                      |
                  [CONJ74]
                      |
                      |
           /---------------------\
           |                     |
        [L117]                [L115]
           |                     |
\end{tpsexample}
\caption{Search Expansion Tree with REWRITE-EQUIVS 1}
\label{thm12-search-etree1}.
\end{figure}

To use tracing, set MS98-TRACE to an appropriate value,
such as (MATING) or (MATING MATING-FILTER).
Then execute the DIY command (or the MS98-1 command in
the mate top level), and {\TPS} will color the nodes
before performing mating search.

A new color corresponding to the (L115 . L116) connection is created.
{\TPS} needs to ensure all nodes corresponding to L115 and L116 are given this color.
When trying to find the nodes that correspond to L115 (in the background),
{\TPS} first runs into a problem at the correspondence between
REW8 (in the background) to REW2 (in the search expansion tree).  In REW8,
EQUIV is expanded as a conjunction of implications.  In the search
tree, EQUIV is expanded as a disjunction of conjunctions.
So, {\TPS} colors every node beneath REW2.  To find the nodes that correspond
to L116.  {\TPS} finds IMP1 corresponds structurally, so {\TPS} colors every node
beneath IMP1.  As a result, every leaf gets the single color in this case.
So, tracing is basically useless in this case.  Every two literals will
share the only color, and so will be considered `good'.

However, if REWRITE-EQUIVS is set to 4, the EQUIV in the search expansion tree
will be expanded as a conjunction of implications.  This will make
the search expansion tree correspond more closely to the background tree.
This is shown in Figure \ref{thm12-search-etree2}.

\begin{figure}
\begin{tpsexample}
<475>ptree*
                                         [SEL0]
                                      [ R^159(OI) ]
                                            |
                                            |
                                            |
                                         [SEL1]
                                      [ S^24(OI) ]
                                            |
                                            |
                                            |
                                         [IMP0]
                                            |
                                            |
                      /-------------------------------------------\
                      |                                           |
                   [REW0]                                      [SEL2]
                   [EXT=]                                   [ X^148(I) ]
                      |                                           |
                      |                                           |
                      |                                           |
                   [EXP0]                                      [IMP3]
                [ x^346(I) ]                                      |
                      |                                           |
                      |                                /---------------------\
                      |                                |                     |
                   [REW1]                            [L16]                 [L17]
                   [EXT=]                              |                     |
                      |                                |                     |
                      |
                      |
                   [REW2]
              [EQUIV-IMPLICS]
                      |
                      |
                      |
                  [CONJ0]
                      |
                      |
           /---------------------\
           |                     |
        [IMP1]                [IMP2]
           |                     |
           |                     |
      /---------\           /---------\
      |         |           |         |
    [L11]     [L12]       [L13]     [L14]
      |         |           |         |

\end{tpsexample}
\caption{Search Expansion Tree with REWRITE-EQUIVS 4}
\label{thm12-search-etree2}.
\end{figure}

In this case, we can see that L115 corresponds to the node IMP2.
So, {\TPS} colors IMP2, L13, and L14.  L116 corresponds to the node IMP3,
so {\TPS} colors IMP3, L16, and L17.
This time the leaves L11 and L12 are left uncolored,
so the trace should have an effect.

Suppose \indexflag{MS98-TRACE} is set to (MATING MATING-FILTER).

\begin{tpsexample}
<Mate500>ms98-1
Substitutions in this jform:
None.
Transfering mating information from background eproof
Eproof:EPR122
transfering conn (L115 . L116)
COLORS:
COLOR62 - L14 L13 L17 L16
. . .
Trying to Unify L17 with L14  ; ***both have COLOR62***
Component 1 is good:
. . .
Trying to Unify L16 with L13  ; ***both have COLOR62***
Component 2 is good:
. . .
Component 3 is good:
Success! The following is a complete mating:
((L17 . L14) (L16 . L13))
. . .
\end{tpsexample}

\subsection{Example: Tracing X2116}\label{X2116-MS98-TRACE}

Suppose the background expansion proof is a proof for X2116,
and suppose we are in a mode such as the suggested mode
from \indexcommand{ETR-AUTO-SUGGEST} in section \ref{THM12-NAT-ETREE}.

Since the mating in this example has six connections, there
will be six colors.

\begin{tpsexample}
<414>exercise x2116
. . .
<415>mate 100 y n n
. . .
<Mate417>ms98-trace
MS98-TRACE [(MATING MATING-FILTER)]>

<Mate418>ms98-1
Substitutions in this jform:
None.
Transfering mating information from background eproof
Eproof:EPR116
transfering conn (L88 . L89)
transfering conn (L90 . L91)
transfering conn (L88 . L89)
transfering conn (L92 . L96)
transfering conn (L98 . L99)
transfering conn (L102 . L103)
COLORS:
COLOR50 - L14 L14.1 L21 L21.1
COLOR49 - L15 L15.1 L12 L12.1
COLOR48 - L10 L10.1 L12 L12.1
COLOR47 - L17 L7 L7.1
COLOR46 - L9 L9.1 L20 L20.1
COLOR45 - L17 L7 L7.1
\end{tpsexample}

We can see in this example that the colors closely correspond
to the literals used in each connection.  Note also that
duplicate leaves get a common color, because any of the
children of an expansion node can correspond to any children
of the corresponding expansion node.  For example, L102
in the background tree could correspond to either
L14 or L14.1 in the search expansion tree.

In the session, {\TPS} continues searching, printing information about
`good' components, and filtering out those which are not `good'.


\chapter{The Monitor}
\label{monitor}

The \indexother{monitor} is designed to be called during automatic proof searches; its basic
operation is described in the User Manual. There are three basic steps required to 
write a new monitor function, which are described below, using the monitor function 
\indexother{monitor-check} as an example. More examples are in the file {\it monitor.lisp}.

\section{The Defmonitor Command}

The command \indexcommand{defmonitor} behaves just like {\tt defmexpr}, the only difference being
that the function it defines does not appear in the list when the user types {\tt ?}. This command
will be called by the user before the search is begun, and should be able to accept any required 
parameters (or to calculate them from globally accessible variables at the time the command is
called).

So, for example, the {\tt defmonitor} part of \indexother{monitor-check} looks like this:

%\begin{tpsexample}
\begin{verbatim}
(defmonitor monitor-check
  (argtypes string)
  (argnames prefix)
  (arghelp "Marker string")
  (mainfns monitor-chk)
  (mhelp "Prints out the given string every time the monitor is called, 
followed by the place from which it was called."))

(defun monitor-chk (string)
  (setq *current-monitorfn* 'monitor-check)
  (setq *current-monitorfn-params* string)
  (setq *monitorfn-params-print* 'msg))
\end{verbatim}
%\end{tpsexample}

Note that this accepts a marker string as input from the user (other monitor functions may 
look for a list of connections, or flags, or the name of an option set; it may be necessary 
to define a new data type to accommodate the desired input). It then calls a secondary 
function, which in this case needs to do very little further processing in order to 
establish the three parameters which are {\it required} for every such function: {\tt *current-monitorfn*}
contains a symbol corresponding to the name of the monitor function, {\tt *current-monitorfn-params*} 
contains the user-supplied parameters (in any form you like, since your function will be the only 
place where they are used) and {\tt *monitorfn-params-print*} contains the name of a function that can 
print out {\tt *current-monitorfn-params*} in a readable way, for use by the commands \indexcommand{monitor}
and \indexcommand{nomonitor}. The latter should be set to {\tt nil} if you can't be bothered to write such 
a function.

\section{The Breakpoints}

In the relevant parts of the mating search code, you should insert breakpoints of the form:

%\begin{tpsexample}
\begin{verbatim}
(if monitorflag 
    (funcall (symbol-function *current-monitorfn-params*) 
             <place> <alist>))
\end{verbatim}
%\end{tpsexample}

The value of {\it place} should reflect what part of the code the breakpoint is at. So, for example,
it might be {\tt 'new-mating}, {\tt 'added-conn} or {\tt 'duplicating}.

The value of {\it alist} should be an association list of local variables and things that your monitor
function will need. For example, {\it alist} might be {\tt (('mating . active-mating) ('pfd . nil))}; it might 
equally well be just {\tt nil}.

All breakpoints should have exactly this pattern. By typing {\it grep "(if monitorflag (funcall" *.lisp} in
the {\it tpslisp} directory, you can get a listing of all the currently defined breakpoints.

\section{The Actual Function}

This is the function which will actually be called during mating search. By convention, it has the
same name as the {\tt defmonitor} function. Normally, it will first check the value of {\it place}, to
see if it has been called from the correct place; it can then use the {\tt assoc} command to retrieve the
relevant entries from {\it alist}. Theoretically, it should be completely non-destructive so as to ensure 
that the mating search continues properly; of course, you may be as destructive as you like, provided 
you understand what you're doing...

The function for {\tt monitor-check} is as follows; notice that this does not check {\it place} since it 
is intended to act at every single breakpoint.

%\begin{tpsexample}
\begin{verbatim}
(defun monitor-check (place alist)
  (declare (ignore alist))
  (msg *current-monitorfn-params* place t)) 
\end{verbatim}
%\end{tpsexample}


\chapter{The Rules Module}
\label{rules}


% \begin{comment}
% \chapter{Inference Rules}\label{Rules}
% 
% {\bf Make revisions to \$progdoc/rules.mss}
% {\bf Someone  familiar with the rules module needs to look at this chapter,
% and update it appropriately.}
% 
% \section{Etc}
% 
% In TPS3, the Rules Module is part of the TPS3 core image, rather than
% a separate core-image.
% 
% Individual Rule files are assembled using ASSEMBLE-FILE. Modules of
% rules are assembled using ASSEMBLE-MOD.
% 
% \section{Implementing Tactics and Tacticals}
% 
% \subsection{Questions and Difficulties}
% 
% The method of tactics and tacticals allows one to expand a collection of
% rules so that many rules may be used or tested in a single step. Specifically,
% this method formalizes certain ways of using existing rules, singly or
% in combination, to create new rules or advisors. The primitive ways are called
% tacticals, and the rules/advisors formed from them are called tactics.
% 
% This brings us quickly to a major problem in implementing tactics and
% tacticals. Since the ways in which the original rules are used are not uniform,
% the notion of combining them becomes vague. If we repeat universal
% instantiation, do we mean to ask for the substituted term, as in the full
% interactive use of the rule, or to take the standard defaults, or to consult
% the expansion tree or some exogenous heuristic (one which is not a tactic
% or otherwise based on the `endogenous' formalism of {\TPS} : inference rules,
% proof outlines, etc.) for the appropriate term?
% When does the iteration end: when all universal quantifiers have been
% instantiated at least once, or when a particular wff is generated, or when
% the user says stop?
% 
% It would seem that we would want to make the operation of
% tacticals dependent on the mode of proof, but even more on the strategy
% adopted. For whatever formalism we choose to direct our proof, we would
% like to interrupt that direction occasionally, maybe seeing only which
% path it plans to take and then giving our own choice.
% Thus, the {\tt GO} and {\tt SUGGEST} facilities, as well as the interactive
% mode should be found in each strategy.
% 
% Besides aesthetic questions as to the printing and keeping of lines and
% the style of the justifications, this dependence affects the efficiency
% of tactics. It may be possible, for a given heuristic, to generate
% sequences of substitution terms more quickly than to determine each term
% when it is called for. In any event, we would like each tactic to be optimally
% efficient for the proof-mode and strategy in which it is called.
% 
% Finally, we want our treatment of tacticals to adequately handle those
% rules which are not merely forwards or backwards. This we would expect
% from the dependence on strategies. The problem of applying, testing
% or trying rules with many potential directions seems to be one of fitting
% those rules into a strategy of proof. Of course, some strategies will
% be too open-ended (say the default strategy for {\ETPS}) to determine
% the appropriate application of some rule. In such cases, {\tt GO} or
% {\tt SUGGEST} may be little help. If we allow strategies to be more open-ended
% for {\tt GO} than for {\tt SUGGEST}, we then have the desired behavior for {\ETPS}.
% 
% \section{Random}
% \end{comment}
Inference rules in {\TPS} are created by typing rule definitions into a .rules
file and then building (or assembling) the rules in that file.
One result of building a rule is the creation of a command which
calls it. The same command may be
used to apply a rule in both its forward or backward directions, that is,
from the top (hypotheses and their consequents, called `support lines') or
the bottom (the conjectures, called `plan lines') of the proof. In our own
rules, we have adopted the convention of naming them as if they were to be
applied only in the forward direction. Thus `ICONJ' (Introduce CONJunction)
takes two
support lines and derives their conjunction (forward) or a plan line asserting
a conjunction and creates two new plan lines, one for each conjunct (backward).

\section{Defining Inference Rules}

The following definition of the inference rule {\tt ABU} provides
a good example of how such rules are defined.

\begin{lispcode}

(defirule abu
  (lines (p1 (h) () `forall y(A). `(S y x(A) A)')
	 (p2 (h) () `forall x(A). A' (`AB' (`x(A)') (p1))))
  (restrictions (free-for `y(A)' `x(A)' `A(O)')
		(not-free-in `y(A)' `A(O)'))
  (support-transformation ((p2 'ss)) ((p1 'ss)))
  (itemshelp (p1 `Lower Universally Quantified Line')
	     (p2 `Higher Universally Quantified Line')
	     (`x(A)' `Universally Quantified Variable in Higher Line')
	     (`A(O)' `Scope of Quantifier in Higher Line')
	     (`y(A)' `Universally Quantified Variable in Lower Line')
	     (S `Scope of Quantifer in Lower Line'))
  (mhelp `Rule to change a top level occurrence of a universally quantified
 variable.'))

\end{lispcode}

The defining macro is DEFIRULE. Next follows the name of the rule
being defined, in this case ABU for alphabetic change of a universally
bound variable. Then comes a list of lists setting the values of
several properties; the property being set is the first item of its list.

The first property we set is the {\tt LINES} property. This establishes the kind
of lines the rule will act on. In our example, {\tt P1} is a line with an
arbitrary hypothesis set {\tt h}, no new hypotheses (the empty list following
the list containing {\tt h}) and a wff matching a quoted expression of some
complexity. The first part of the expression seems clear enough: {\tt P1}
will be a universally quantified wff. But what is {\tt `(S y x(A) A)} ?
Just our way of denoting wff-transformations within an expression
similar to a wff. The backquote means `evaluate this Lisp form', in our case
a call to the substitution function {\tt S}, replacing free occurrences
of {\tt y} with {\tt x} in a wff which we call {\tt A}. See the Facilities Guide for
a list of the functions which can be used like {\tt S} (these are called
{\tt WFFOP}s). {\tt x} is of a type
called {\tt A} which just happens to have the same name as the wff we are
substituting into, but this causes TPS no confusion; when a primitive symbol is
followed by an expression in parentheses, that expression is a type expression
and not a wff.

The second line, {\tt P2}, looks similar. It has the same set of hypotheses,
{\tt h}, and it also introduces no new hypotheses. The quoted expression,
though, is much simpler; it indicates that the wff asserted by {\tt P2}
is universally quantified by the variable we substituted into {\tt A}
in {\tt P1} and quantifies over that same {\tt A}. An actual use of the rule
may have {\tt P2} bound by {\tt y} and {\tt P1} by {\tt x}, and this is fine as
long as they correspond in the way that {\tt x} and {\tt y} do in the DEFIRULE.
That is, the variables in a DEFIRULE are not actual variables in the logical
system, but part of a pattern-matching device for the rule. The quoted
expression is followed by a list indicating the justification for the line.
The first item is the name of the justification, in our case `AB' for
alphabetic change of bound variable. The second item is a list of parameters
(excluding lines) which figure in the justification; here we indicate
that the bound variable has been changed to the variable matched by {\tt x}.
The quotes around {\tt x} are necessary.
The last item is a list of lines from which the line {\tt P2} is derived,
in this case {\tt P1}.

The next property, {\tt RESTRICTIONS}, is optional, depending on whether
or not the rule can only be applied if certain conditions are met.
In this example, the variable matching {\tt y} must be free for the variable
matching {\tt x} in the wff matching {\tt A} and similarly for the `not free in'
restriction. Note that each argument to the restrictions is typed.
In restrictions, you must give each wff variable a type (or make sure it can be
inferred). Otherwise, the default type will be used, giving an undefined
symbol as an argument to the restriction function.

The next property, {\tt SUPPORT-TRANSFORMATION}, tells TPS how this rule
will change the proof structure. In our example, the support lines for
{\tt P2}, indicated by {\tt 'ss}, will be assigned to {\tt P1}, if the rule
is applied backwards.  In other rules, the abbreviation {\tt pp} may be
seen as the first member of a support-transformation list.  {\tt pp} will
match any planned line with the specified lines as supports; e.g., if
{\tt pp P1} appeared as the left hand side of a support transformation,
the transformation would be applied to every planned line which had
{\tt P1} as a support.

The {\tt ITEMSHELP} property specifies the help the command for the rule
will give on each argument. Arguments include all lines defined in the
{\tt LINES} property, all matching variables (except types) and the
name of functions ({\tt WFFOP}s) called from within the quoted expressions
(in case not all of its arguments are specified in time).

The {\tt MHELP} property, as always, provides a short description of the rule
for the {\tt HELP} command and for documentation.


\section{Assembling the Rules}

Once you have typed your {\tt DEFIRULE}s into a .rules file, the next step
is to assemble the rules. Assembling creates Lisp-code files which can
be loaded and/or compiled. You may assemble individual rules files
with {\tt ASSEMBLE-FILE} or whole modules (collections of files) with
{\tt ASSEMBLE-MOD}. The latter is preferable, not only because
it combines many steps in one, but because the initialization for the package
will be called less frequently. {\tt ASSEMBLE-FILE}
finds the proper initialization, not very cleverly, by asking for the package
to which the file belongs.


Before assembling your rules, the correct mode should be loaded: Call {\tt REVIEW}, entering its toplevel.
Call {\tt MODE} with % \comment{either {\tt \indexother{MATH-LOGIC1-MODE}} or
% {\tt \indexother{MATH-LOGIC-2-MODE}}}
 {\tt \indexother{RULES}}
as an argument.

% \comment{Directories or pathnames should not be specified when you are asked
% for the .RULES file.}
After assembling, you need only compile (if desired) and load to make
your rules available in that session, e.g., via the command {\tt CLOAD}.
Before compiling and loading your rules, you should go into {\tt REVIEW} and
set the mode to {\tt RULES}, so that the wffops which appear in your
rules will be properly interpreted.
To make your rules more permanently
available, create a package (in the {\tt DEFPCK} file) containing
the name of your assembled rules file (the same as the .rules file
but with a lisp extension) and load that package when building {\TPS} or
{\ETPS}.

\subsection{An example}

We added a new rule definition to the file {\it ml2-logic7a.rules}, making it
necessary to reassemble and recompile {\it ml2-logic7a.lisp}. This was done
as follows:

\begin{alltt}
<123>mode rules
<124>assemble-file
RULE-FILE (FILESPEC): Rule source file to be assembled [No Default]>ml2-logic7a
PART-OF (SYMBOL): Module the file is part of [OTLSUGGEST]>math-logic-2-rules
<125>mode rules
<125>cload ml2-logic7a
\end{alltt}

\input{rule-example}

\begin{comment}

% % \comment{Old documentation}
\section{Specifications of Inference Rules}
This section describes how the {\TPS} user can define his own
inference rules.  Sets of these inference rules can then be made available
as packages
which can be added to the basic {\TPS} core image.  The functions discussed
below are part of the package {\tt RULES}\index{Rules}, which is necessary
for defining inference rules.

It may be helpful to think of the {\tt RULES} package as a compiler.  It can read
source files or rule descriptions in various forms and compiles them into
{\tt LISP} code.  This {\tt LISP} code is stored in files
which can be retrieved from within {\TPS}.

Inside the {\tt RULES} package it is always assumed that the user is working
on a particular set of inference rules which he is trying to expand,
modify, or define. Every such set of rules has a name. {\tt S4}, {\tt CLASSIC},
or {\tt INTUITION} are examples of such names.
Source file name(s) and the rule file name are derived from the name of the
set of rules as described below.  The `rule file' is the file containing
the {\tt LISP} code produced from the rule descriptions, which typically come
from the `source file'.  Both can be read and written directly by {\tt LISP}
which also keeps track of different versions of the same
file as well as changes made to a file during a editing session etc.
The {\tt LISP} functions {\tt DSKIN}, {\tt DSKOUT}, and {\tt CHANGES} are sufficient for
handling all the necessary file operations.  If you are using \VIDI,
a `\VIDI file' containing rule descriptions, possibly with special characters,
may also be present.

The {\tt RULES} package produces two functions for every inference rule
specification.  One is prefixed by {\tt D-} and allows the application
of the rule in the forward direction, i.e. it can be used to infer new
lines in the partial proof from lines that have been proven already.
The other function is prefixed by {\tt P-} and allows the application
of the inference rule in the backward direction.  This function inserts
new lines into the incomplete proof which would justify previously unproven
lines ({\it planned lines}).  These new lines are now planned lines.
Inference rules are allowed to contain primitive operators, like
\LAMBDA-contraction, \LAMBDA-expansion, etc.  Their definition and
their use is laid down in a file {\tt PRIMOP}, which can be extended
by the user.  Primitive operators typically take wffs as arguments
and return wffs.  Primitive operators which have no simple inverse
(like \LAMBDA-normalization) often make it impossible to construct
both {\tt D-} and {\tt P-} versions for a given inference rule. In this case
a warning will be given.


\subsection{Available commands}
The following is a complete list of commands which are available to the
user of the {\tt RULES} package.
{\bf This section should be deleted as this can be generated automatically.
However, the MHELP strings should first be updated to reflect the information
here.}
\begin{description}
\item[] \indexmexpr{RuleFile {\it FileName}}
defines {\it FileName} to be the current rule file.  The associated
current source file will have a {\tt Z} appended to {\it FileName}. ({\tt Z} replaces
the sixth character if {\it FileName} consists of six or more characters.)
If these files exist in the user's or any library directory, they will
be read for modification.  Otherwise, the appropriate data
structures will be set up, and the files will be ready for {\tt DSKOUT}.
{\it FileName} follows the {\tt LISP} conventions and may include an extension,
as well as a PPN and a device specification.  The global variables
\indexparameter{RuleFileFNS} and \indexparameter{SourceFileFNS} are set
to identifiers which have the table of contents for the rule file or
the source file, respectively, as their value.  \indexparameter{Rule-File}
is another global variable which will be bound to {\it FileName}.

\item[] \indexmexpr{Define-Rule \{{\it RuleName}\} \{{\it RuleDef}\}}
This parses {\it RuleDef} and creates two {\tt LISP} function definitions, namely
{\tt D-{\it RuleName}} and {\tt P-{\it Rulename}}.  The rule definition will be printed
into every selected {\TPS} channel after it is parsed.  {\it RuleDef} defaults
to the definition specified in the source file.  The current rule and source
files are updated appropriately, so that any changes can be detected with
{\tt CHANGES} and saved with {\tt DSKOUT}.  Notice that {\it RuleName} will be appended
to the current source file, if it is not already present.  In case {\it RuleDef}
cannot be parsed correctly, an error message will be printed and neither
the rule file nor the source file will be updated.  For a description of
the argument type {\it RuleDef}, see page \pageref{ruledef}.  If {\it RuleName}
is left unspecified, {\it RuleDef} must start with `{\tt RULE: }{\it RuleName}'.

\item[] \indexmexpr{Compile-File \{{\it RuleFile}\}}
compiles every inference rule specification in the source file associated
with {\it RuleFile}.  During compilation every rule will be stated.
{\it RuleFile} defaults to the current rule file as specified in the last
{\tt RuleFile} command.

\item[] \indexmexpr{State-Rule {\it RuleName}}
prints {\it RuleName} into every open channel.  If {\it RuleName} had not been
parsed before, either with {\tt Compile-File} or {\tt Define-Rule}, but is
defined in the source file, {\TPS} will try to parse it first and the print it
from its internal representation.  This is necessary in order to make
the way the rule is stated independent of the way it was specified to the
system, which could be a string, a {\tt RdClist}, etc.

\item[] \indexmexpr{List-Rules \{{\it RuleFile}\} \{{\it ListFile}\} \{{\it Style}\} \{{\it Channel}\}}
prints every rule in {\it RuleFile} into {\it ListFile} and all currently selected
{\TPS} channels.  {\it RuleFile} defaults to the current rule file.  If no
{\it ListFile} is specified, the rules will just be printed into every selected
{\TPS} channel.  {\it Style} and {\it Channel} are as in the {\tt OPEN} command and
apply to the {\it ListFile}.
\end{description}

Any form of a rule description is first converted into a \itt{LexList}, a
data structure used in an intermediate step in parsing formulas.  This
{\tt LexList} is what the preprocessing functions will actually give to
{\tt Define-Rule} as an argument.
The argument type {\it RuleDef} can be any of the following.
\begin{description}
\item[] \label{Ruledef}\index{{\it RuleDef}}
\itt{\$} [defaulted]
see {\tt Define-Rule}.

\item[] a string
the string representation of a rule definition as used in the source file.
The string may not contain any special characters. {\tt !} stands for \ASSERT.

\item[] \itt{RD}
The user will be prompted in the next line for a rule definition
without special characters.  The input must be terminated with
{\tt <esc>}\index{{\tt <esc>}} and can be aborted with \itt{\^G}.
{\bf obsolete -SI}

\item[] \itt{RDC}
The user will be prompted for a rule definition which may include
special characters.  The input must be terminated with
{\tt <esc><esc>}\index{{\tt <esc><esc>}} and can be aborted with \itt{\^G}.
See page \pageref{rdc} for a description of the input format for wffs.
{\bf obsolete -SI}

\item[] \itt{PAD}
enters the {\tt PAD} and puts the terminal (Concept) into local mode.
The restrictions on the rule definition are the same as for {\tt RD}.  You can
transmit the contents of the {\tt PAD} using the SEND key.  {\bf obsolete -SI}

\item[] \itt{VIDI}
refers to the rule definition as created with the most recent \itt{\$r} inside
\VIDI. {\bf obsolete -SI}

\item[] an identifier
can mean a string, if the identifier is bound to a string, a \itt{RdCList}, if
the identifier is bound to a {\tt RdCList}, or a \itt{LexList}, if the identifier has
a \itt{RDEF} property which is a
{\tt LexList}.  A {\tt RdCList} is created during a {\tt RdC}, a {\tt LexList} is
the result of an {\tt \$r} command in \VIDI.
\end{description}

\subsection{Some sample specifications of inference rules}
This section contains some sample inference rules which demonstrate some
features of the syntax and the rule compiler.  A more precise and complete
grammar is given in section \ref{rulegrammar}.
\begin{alltt}
\tabdivide{2}
{\tt 
D 1 }{\it H}{\tt  }\assert {\tt  A}\(\sb{\greeko}${\tt  \and  B}$\sb{\greeko}${\tt ;} &  {\rm The declaration of \ScriptH as a {\tt WFFSET} has}
{\tt P 2 }{\it H}{\tt  }\assert {\tt  A; AndE:1;} &  {\rm been omitted, since \ScriptH and \ScriptG are}
{\tt P 3 }{\it H}{\tt  }\assert {\tt  B; AndE:1;} &  {\rm predeclared to be {\tt WFFSET}s.}

{\tt D 1 }{\it H}{\tt  }\assert {\tt  A}\(\sb{\greeko}${\tt ;} &  {\rm Notice that here \ScriptH appears twice.}
{\tt D 2 }{\it H}{\tt  }\assert {\tt  B}\(\sb{\greeko}${\tt ;} &  {\rm It will be the union of the hypotheses of lines 1 }
{\tt P 3 }{\it H}{\tt  }\assert {\tt  A \and  B; AndI:1,2;} &  {\rm and 2.}

{\tt D 1 }{\it H}{\tt ,A}\(\sb{\greeko}${\tt  }\assert {\tt  B}$\sb{\greeko}${\tt ;} &  {\rm Here one of the arguments of the {\tt D}-rule will}
{\tt P 2 }{\it H}{\tt  }\assert {\tt  A }\implies{\tt  B; ImpI:1;} &  {\rm be {\tt HYP-A<O>}, the number of a line asserting}
 &  {\rm {\tt A\Subomicron}.  One of the optional arguments}
 &  {\rm of the {\tt P}-rule will be {\tt HYP-A<O>}, the number}
 &  {\rm of the line which will be {\tt A\subomicron \Assert A\subomicron}.}

{\tt D 1 }{\it H}{\tt  }\assert {\tt  A}\(\sb{\greeko}${\tt  }\implies{\tt  B}$\sb{\greeko}${\tt ;
P 2 }{\it H}{\tt ,A }\assert {\tt  B; ImpE:1;

D 1 }{\it H}{\tt  }\assert {\tt  A}\(\sb{\greeko}${\tt ;
D 2 }{\it H}{\tt  }\assert {\tt  A }\implies{\tt  B}$\sb{\greeko}${\tt ;
P 3 }{\it H}{\tt  }\assert {\tt  B; MP:1,2;

D 1 }{\it H}{\tt  }\assert {\tt  \(\forall$ x}$\sb{\greeka}${\tt  . A}$\sb{\greeko}${\tt ;} &  {\rm In this rule, {\tt D} and {\tt P} lines have the same number.}
{\tt P 1 }{\it H}{\tt  }\assert {\tt  `LCONTR` [}$\lambda${\tt x . A] B}$\sb{\greeka}${\tt ; UnivE: 1;}
{\tt B free for x in A;} &  {\rm This is legal since no conflict can arise.}
 &  {\rm The keywords {\tt free}, {\tt for} and {\tt in} must be}
 &  {\rm  lower case.}
{\tt 
D 1 }{\it H}{\tt  }\assert {\tt  `LCONTR` [}$\lambda${\tt x}$\sb{\greeka}${\tt  . A}$\sb{\greeko}${\tt ]B}$\sb{\greeka}${\tt ;} &  {\rm In the {\tt D}-rule compiled from this definition, }
{\tt P 1 }{\it H}{\tt  }\assert {\tt  $\forall$ x . A; UnivI: 1;} &  {\rm the user will have to supply {\tt x\subalpha}, {\tt B\subalpha}, and}
{\tt B is variable;} &  {\rm a list of occurrences of {\tt B\subalpha} in the assertion}
{\tt B not free in }{\it H}{\tt ;} &  {\rm of line {\tt D1}, so that {\TPS} can reconstruct {\tt A\subomicron}.}

{\tt D 1 }{\it H}{\tt  }\assert {\tt  `LCONTR` [}$\lambda${\tt x}$\sb{\greeka}${\tt  . A}$\sb{\greeko}${\tt ]B}$\sb{\greeka}${\tt ;
P 1 }{\it H}{\tt  }\assert {\tt  $\exists$ x . A; ExistI: 1;

D 1 }{\it H}{\tt  }\assert {\tt  $\exists$ x}$\sb{\greeka}${\tt  . A}$\sb{\greeko}${\tt ;
P 1 }{\it H}{\tt  }\assert {\tt  `LCONTR` [}$\lambda${\tt x.A]B}$\sb{\greeka}${\tt ; ExistE: 1;
B is variable;
B free for x in A;
B not free in }{\it H}{\tt ;

D 1 }{\it H}{\tt  }\assert {\tt  $\exists$ x}$\sb{\greeka}${\tt .A}$\sb{\greeko}${\tt ;} &  {\rm Here in {\tt RuleC} we see the use of abbreviations.}
{\tt D 2 *D2* }\assert {\tt  `LCONTR` [}$\lambda${\tt x.A]B}$\sb{\greeka}${\tt ;} &  {\rm {\tt *D2*} stands for the assertion of line {\tt D2}, which}
{\tt D 3 }{\it G}{\tt ,*D2* }\assert {\tt  C}$\sb{\greeko}${\tt ;} &  {\rm otherwise could not appear as a hypothesis.}
{\tt P 1 }{\it G}{\tt  }\assert {\tt  C}$\sb{\greeko}${\tt ; RuleC: 1,2,3;
B is variable;
B free for x in A;
B not free in }{\it H}{\tt ;
B not free in C;
B not free in $\exists${}x.A;

A 1 `LCONTR` [$\lambda${\tt x.A]B}$\sb{\greeka}${\tt ;} &  {\rm This is an equivalent way of stating the first part}
{\tt D 1 }{\it H}{\tt  }\assert {\tt  $\exists$ x}$\sb{\greeka}${\tt .A}$\sb{\greeko}${\tt ;} &  {\rm of the previous rule, now using an {\it abbwff}}
{\tt D 2 *A1* }\assert {\tt *A1*;} &  {\rm construct. {\tt *A1*} stands for the wff}
{\tt D 3 }{\it G}{\tt ,*A1* }\assert {\tt  C}$\sb{\greeko}${\tt ;} &  {\rm following {\tt A 1}.}
{\tt P 1 }{\it G}{\tt  }\assert {\tt  C}$\sb{\greeko}${\tt ; RuleC: 1,2,3;}
\end{alltt}

\subsection{The order of arguments}\label{argorder}
The following convention has been adopted to decide the order of
arguments in the {\tt D}-rules created by the rule compiler.  In
general the name of the argument will be connected to
its meaning.  This is particularly helpful if the new way of
supplying arguments to {\tt MExpr}'s interactively is utilized.

\begin{enumerate}
\item {\it dlines} in the order they appear in the rule definition.  The
argument type is {\tt LineNumber}.  The name of the argument for
the line started with `{\tt D }{\it n}' will be {\tt D}{\it n}.

\item {\it wffs} in the order new wffs appear in the rule definition.
The argument type will be {\tt RWff}, the name of the argument
for the wff `{\tt A\subomicron}' will be {\tt A<O>}.
The {\it wffs} include special arguments that stem from inverting
primitive operators.

\item {\it hypotheses} in the order unknown hypotheses appear in the rule
definition.  The argument type is {\tt LineNumber}.  The name of the argument
for the hypothesis `{\tt B\subalpha}' will be {\tt HYP-B<A>}.

\item {\it prargs}, i.e. arguments of primitive operators as they appear in
the rule definition.  The argument type depends on the types of the
arguments of the primitive operator as specified in the file {\tt PRIMOP}.

\item {\it new hypotheses}, i.e. line numbers at which to insert new lines
of the form {\obeyspaces `{\tt A\Subomicron \Assert A\Subomicron}'}.  These can
be defaulted.  The argument type is {\tt NUM}, their naming
convention is identical to those for hypotheses.

\item {\it plines} in the order they appear in the rule definition.  Their
argument type is {\tt NUM} and they can be defaulted.  The name of the
argument for the line starting with `{\tt P }{\it n}' is {\tt P}{\it n}.
\end{enumerate}
The order of arguments for the {\tt P}-rules is strictly symmetric to
the argument order of {\tt D}-rules.
\begin{enumerate}
\item {\it plines} in the order they appear in the rule definition.  The
argument type is {\tt LineNumber}.  The name of the argument for
the line started with `{\tt P }{\it n}' will be {\tt P}{\it n}.

\item {\it wffs} in the order new wffs appear in the rule definition.
The argument type will be {\tt RWff}, the name of the argument
for the wff `{\tt A\subomicron}' will be {\tt A<O>}.
The {\it wffs} include special arguments that stem from inverting
primitive operators.

\item {\it hypotheses} in the order unknown hypotheses appear in the rule
definition.  The argument type is {\tt LineNumber}.  The name of the argument
for the hypothesis `{\tt B\subalpha}' will be {\tt HYP-B<A>}.

\item {\it prargs}, i.e. arguments of primitive operators as they appear in
the rule definition.  The argument type depends on the types of the
arguments of the primitive operator as specified in the file {\tt PRIMOP}.

\item {\it new hypotheses}, i.e. line numbers at which to insert new lines
of the form {\obeyspaces `{\tt A\Subomicron \Assert A\Subomicron}'}.  These can
be defaulted.  The argument type is {\tt NUM}, their naming
convention is identical to those for hypotheses.

\item {\it dlines} in the order they appear in the rule definition.  Their
argument type is {\tt NUM} and they can be defaulted.  The name of the
argument for the line starting with `{\tt D }{\it n}' is {\tt D}{\it n}.
\end{enumerate}

\section{A grammar for specifying inference rules}\label{rulegrammar}
In the following grammar in BNF style terminal symbols are underlined.
[{\it token}]\0inf means that {\it token} can be repeated 0 or more times.
<{\it name}> means that {\it name} can be any {\tt LISP} object which is a {\it name}.
\{{\it field}\} indicates that {\it field} is optional.  Note that the case of
characters matters, i.e. capital letters have to be capital, and lower
case letters have to be lower case.  Spaces are critical only where they
are needed to separate identifiers, just as in formulas.  The symbols
`{\tt [ ] ( ) . ; , : <return> <tab>}' separate identifiers and thus need
not to be surrounded by spaces.
\begin{description}
\item[rule ::=]	 \{\uxt{RULE:} <identifier>
\uxt{;}\}\{\uxt{COMMENT:} comment\uxt{;}\}[declaration]\0inf{adiline}\0inf{pline}\1inf{restriction}\0inf

\item[adiline ::=]	 abbwff | dline | iline

\item[abbwff ::=]	  \uxt{A} <number> wff\uxt{;}

\item[iline ::=]	  \uxt{I} <number> \{hypotheses\} \uxt{\assert} assertion\uxt{;}

\item[dline ::=]	  \uxt{D} <number> \{hypotheses\} \uxt{\assert} assertion\uxt{;}

\item[pline ::=]	  \uxt{P} <number> \{hypotheses\} \uxt{\assert} assertion\uxt{;} justification\uxt{;}

\item[declaration::=]	 \uxt{CONSTANT:} wff [\uxt{,}wff]\0inf\uxt{;}
| \uxt{WFFSET:} wffset [\uxt{,}wffset]\0inf\uxt{;}

\item[hypotheses ::=]	 hyp [\uxt{,}hyp]\0inf

\item[hyp ::=]	 wff\subomicron | wffset

\item[assertion ::=]	 wff\subomicron | \uxt{`}primop\uxt{`} prarg [\uxt{,}prarg]\0inf

\item[primop ::=]	 \uxt{LCONTR} | \uxt{LEXPD} | {\it others as defined by the user}

\item[prarg ::=]	 wff | occlist | {\it others as defined by the user}

\item[occlist ::=]	 \uxt{OCC}<number> | \uxt{[}<number>[\uxt{,}<number>]\0inf\uxt{]}

\item[wffset ::=]	 <identifier>

\item[justification ::=]	 <identifier> \{\uxt{:} wff [\uxt{,}wff]\0inf\}
\{\uxt{:} lineno [\uxt{,}lineno]\0inf\}

\item[restriction ::=]	   wff \{\uxt{not}\} \uxt{free} \uxt{in} wff\uxt{;}
| wff \{\uxt{not}\} \uxt{free} \uxt{in} wffset\uxt{;}
% \comment{\{@!$^{\hbox{wff}}$@/$_{\hbox{wffset}}$\}\uxt{;}}
| wff \{\uxt{not}\} \uxt{free} \uxt{for} wff \uxt{in} wff\uxt{;}
| wff \{\uxt{not}\} \uxt{free} \uxt{for} wff \uxt{in} wffset\uxt{;}
% \comment{\{@!$^{\hbox{wff}}$@/$_{\hbox{wffset}}$\}\uxt{;}}
| wff \uxt{is} \uxt{variable}\uxt{;}
| {\it others as defined by the user}

\item[comment ::=]	 {\it any sequence of characters not containing ` ( ) or ;}

\item[wff ::=]	  {\it any wff legal in {\TPS}}
\end{description}
Note that wffs may contain the following identifiers which stand for
assertions of lines or abbreviations which have been previously defined:
\begin{description}
\item[{\tt *A}{\it n}{\tt *}]	 stands for the wff of the {\tt abbwff} with number {\it n}.

\item[{\tt *D}{\it n}{\tt *}]	 stands for the assertion of the {\tt dline} with number {\it n}.

\item[{\tt *I}{\it n}{\tt *}]	 stands for the assertion of the {\tt iline} with number {\it n}.

\item[{\tt *P}{\it n}{\tt *}]	 stands for the assertion of the {\tt pline} with number {\it n}.
\end{description}
While the first can have any type, the last three have to be of type {\tt O}.
These special identifiers do not have to, but may be typed.

% \comment{\string{KsetSize=10}
% %\input{lib:ksets.mss}
% %\input{lib:symb10.mss}}
\commandstring{IN=`\member1{)

\section{Multiple Rules of Inference}

The plain {\tt RULES} package lacks facilities to specify relatively simple
inference rules, such as a rule which would infer every conjunct in
a large conjunction.  Instead of trying to improve the syntax of rule
descriptions to include constructions such as {\tt A$_{\hbox{1}}$ \AND ... \AND A$_{\hbox{n}}$},
the solution proposed here considers simple inference rules as
indivisible steps in a ``programming language''.  The ``programming language''
is what we are concerned with here.  We would like to have simple
and easy to use facilities to build {\it multiple inference rules}
from other ones.

One can certainly think of many such {\it rule construction languages}.
Lisp itself is certainly one choice that comes to mind.  We could write
commands which would invoke other inference rules as subroutines
appropriately and so achieve the desired outline transformation.
We do not want to exclude that possibility, since there will certainly
be rules to complicated to build up any other way.  Two drawbacks of
this method have to be noted, however:  Firstly, we have to decide
in advance which inference rules we would like to have available
as multiple rules, and, secondly, one will certainly want to write many
different multiple inference rules for different logical systems, which
requires a lot of special purpose programming.

Here we propose an alternative, which will allow us to treat many cases
of multiple inference rules at the non-programmer level.

\subsection{Regular Expressions}

Regular expressions R are frequently used in computer science and have
many nice properties.  Let us define regular expressions abstractly
first.  The definition is by induction.

\begin{enumerate}
\item u\in\CapSigma is a regular expression.  \CapSigma is the {\it underlying set},
often also called the underlying {\it alphabet}.

\item If u,v\in R then u+v\in R.  This can be interpreted as union, alternation,
or disjunction.

\item If u,v\in R then uv\in R.  This can be interpreted as concatenation.

\item If u\in R then u$^{\hbox{*}}$\in R.  This is the {\it Kleene star} and represents
potentially infinite repetition.
\end{enumerate}

\subsection{Regular Expressions as Rule Constructions}
We can now exploit the simple constructive nature of regular expressions
to build our rule construction language.  Let \CapSigma be the set of
primitive rules, presumable defined by the {\tt RULES} package.  We then define
the extended set of rules R by the same kind of induction.

\begin{enumerate}
If r\in\CapSigma then r\in R

If r,s\in R then r+s\in R.  r+s stands for the alternation of the two rules:
apply either r or s, whichever is possible (i.e. matches the given input line).
There must be a restriction on the uniformity of the rules, e.g. they must
take the same number of {\it dlines} into the same number of {\it plines}.  We may
loosen the analogy with regular expressions by postulating that the
elements in a sum are tested for a match from left to right.  The rules
do not have to be exclusive.

If r,s\in R then r\&s\in R.  r\&s stands for the successive application of the
two rules.  First apply r, than apply s to the results of r.  Again, some
restriction on the number of arguments of type {\it line} will have to be imposed.

If r\in R then r$^{\hbox{*}}$\in R.  r$^{\hbox{*}}$ stands for the repeated application of r.
First r is applied to the arguments, then to the results of the first operation etc
etc until we have no possible match left.
\end{enumerate}

\subsection{Some Examples}
Here are some examples.  Overlook some problems with the actual syntax
of definitions of multiple rules; this will have to be decided later.

\begin{alltt}\tabdivide{2}
CONJ* := <CONJ>*

Rule: CONJ &  {\it then {\tt CONJ}* looks like}
D 1 H \Assert A \AND B; &  D 1 H \Assert A \AND ... \AND Z;
P 2 H \Assert A; RuleP:1; &  P 2 H \Assert A; RuleP:1;
P 3 H \Assert B; Rulep:1; &  ...
 &  P n H \Assert Z; RuleP:1;

PUSH := <PUSHU>+<PUSHE>
PUSH* := <PUSH>*

Rule: PUSHU
D 1 H \Assert $\forall$x.A \AND B;
P 1 H \Assert $\forall$x A \AND $\forall$x B; RuleQ:1;

Rule: PUSHE
D 1 H \Assert $\exists$x.A \lor B;
P 1 H \Assert $\exists$x A \lor $\exists$x B; RuleQ:1;

{\it Then {\tt PUSH} can be applied to a line of either variety and {\tt PUSH*}
will distribute a quantifier over a multiple disjunction or conjunction}

DISTU := <PUSH*>\&<CONJ*>

{\it {\tt DISTU} will push in universal quantifiers over conjunctions, then
assert the conjunctions in separate lines.}

UI* := <UI>*

{\it {\tt UI*} allows to instantiate a whole series of quantifiers}
\end{alltt}

\comment{End old documentation}

\end{comment}

\chapter{ETPS}\label{etps}

\section{The Outline Modules}

The {\tt OUTLINE} modules in TPS have two main subparts; 
the bookkeeping functions, and the \indexcommand{GO} command which
gives sophisticated help or constructs a proof automatically.
They are collected in the modules of the form {\tt OTL*}.

In ETPS only the bookkeeping functions are present.

The discussion below is aimed at understanding the {\tt OUTLINE} modules
independently of the system, but we generally assume we are working in
\ETPS.  If \TPS differs, this is noted.

We often talk about {\it proofs}, even though they are properly only incomplete
proofs or {\it proof outlines}.  It is assumed that the reader knows what
planned lines ({\it plines}) and deduced lines ({\it dlines}) are.  This and 
general familiarity with \ETPS are necessary to understand this discussion.

\subsection{Proofs as a Data Structure}

Proofs in \ETPS are represented by a single atom with a variety of properties.
The global variable {\tt DPROOF} has as value the name of the current proof.
In case you are working, say on exercise {\tt X6200}, {\tt DPROOF} will have the 
value {\tt X6200}.  The current proof name then has a variety of properties.

\begin{description}
\item [{\tt LINES}] {\tt ({\it line} {\it line} ...)}.  This is simply an ordered list of all
lines in the current proof without repetition.  The order is such that
lines with a lower number appear first in the list.
\\ {\bf WARNING}:  This property is frequently changed destructively.
As a consequence it may never be empty and should be used for other
purposes only in a copy.

\item [{\tt LINEALIASES}] {\tt (({\it line} . {\it no})({\it line} . {\it no}) ...)}.  This is an 
unordered association list correlating lines with their numbers.  No line 
should ever appear in more than one pair, and neither should a number.  Try to 
think of arguments for and against this representation, compared to one where
the number of a line is stored on the line's property list.
\\ {\bf WARNING}:  This property is frequently changed destructively.
As a consequence it may never be empty and should be used for other
purposes only in a copy.

\item [{\tt PLANS}] {\tt {(({\it pline} . {\it supportlist})({\it pline} . {\it supportlist}) ...)}}.
This stores the important {\it plan}-{\it support} {\it structure} of the current
proof.  {\it pline} is a still unjustified line in the current proof,
{\it supportlist} is a list of deduced lines supporting a {\it pline}.
A {\it pline} may never have a justification other than {\tt PLAN}{\it i}, 
a {\it sline} (support line) must always be completely justified, i.e.
may not ultimately depend on a planned line.

The association list is ordered such that the most recently affected
{\it pline} is closer to the front of the list.  The order can be changed
explicitly with the {\tt SUBPROOF} command.

{\bf WARNING}:  This property is frequently changed destructively.
As a consequence it may never be empty and should be used for other
purposes only in a copy.

\item [{\tt GAPS}] ({\it gap} {\it gap} ...). This is a list of the gaps between lines in the
proof. Each gap has the properties ({\tt MIN-LABEL} {\it line}) and ({\tt MAX-LABEL} {\it line}).

\item [{\tt NEXTPLAN-NO}] {\it integer}. This is just the next number that will be used for 
a planned line.

\item [{\tt ASSERTION}] {\it gwff}. This is the assertion being proven.
\end{description}

When a proof is saved using the \indexcommand{SAVEPROOF} command, a checksum may be generated.
This is used by \ETPS to verify that the saved proof has not been manually edited by a student
(otherwise it would be possible to edit out the planned lines and convince \ETPS to issue the 
\indexcommand{DONE} command). Since it takes time to generate the checksum, it is only 
generated if the flag \indexflag{EXPERTFLAG} is NIL. This means that proofs written by \TPS
with \indexflag{EXPERTFLAG} T cannot be read into \ETPS with \indexflag{EXPERTFLAG} NIL.


\subsection{Proof Lines as a Data Structure}

Proof lines in \ETPS have a variety of properties:

\begin{description}
\item [{\tt REPRESENTS}] This is the wff asserted by the line.  In the original 
TPS this had to be an
atomized wff of very particular structure, which lead to numerous problems
in higher-order logic.  In \ETPS this has been maintained for the present. Our
goal, of course, is to allow arbitrary {\it gwffs} as {\tt REPRESENTS} of lines.

\item [{\tt HYPOTHESES}] This is a list of lines assumed as hypotheses for the line.
The list of hypotheses is ordered (lowest numbered line first), but to
my knowledge no function assumes this.  It simply looks better in the output.
No line should appear twice as an hypothesis (this fact may actually be used
here and there).

\item [{\tt JUSTIFICATION}]  {\tt {{\it RULE} {\it gwfflist} {\it linelist}}} The line can be
inferred by an inference rule {\it RULE}from {\it linelist}.  
{\it gwfflist} has somehow been used to infer the line.

\item [{\tt LINENUMBER}] The line number associated with the line.
\end{description}


\section{Defaults for Line Numbers - a Specification}

There will never be an absolutely correct way
of assigning default for line numbers; we can merely make sure that
the result will always be logically correct - the rest is often
a matter of style and the kind of heuristics used.

Below we give a description of the tasks to be done by a function
\indexfunction{LINE-NO-DEFAULTS} which is called during the application of every
inference rule {in interactive mode}.  We will set the stage by giving
some, not necessarily exhaustive, examples of what meaning to assign
to the data structures and what output to expect from the function.

\subsection{The {\it support} data structure}

At each stage in a proof, we have associated with it a {\it \indexother{support}} structure,
which, for any given planned line ({\it \indexother{pline}}), tells us which deduced lines
({\it \indexother{dline}s}), we expect to use in the proof of the {\it pline}.

Thus the support structure is of the form
$$((p_1\; d_{11} \ldots d_{1x_1})\ldots (p_p\; d_{p1}\ldots d_{px_p}))$$

One may assume the following:
\begin{enumerate}
\item The $p_i$ are pairwise distinct.

\item The $d_{ik}$ are pairwise distinct for every fixed $i$ and $1\leq k \leq x_i$.

\item For each $i$, $d_{ik} < p_i$ for all $1\leq k \leq x_i$.

\item The planned lines $p_i$ are ordered such that the ones the user is
expected to work on first appear closer to the front.  In particular,
$p_1$ is the planned line worked on most recently.

\item Similarly, for a given $i$, the $d_{ik}$ are ordered such that the
one the user is expected to use first appear earlier.
\end{enumerate}

With each rule definition, there will be a description of how the {\it support}
structure changes.  This is given as two {\it support} structure templates,
using the name given to the lines in the rule specification.

\subsection{Examples}

The examples below are not complete, in the sense that not the full
description of the rule (for the rules module) is given, we have merely
extracted what is important in our context.  {\it p} and {\it d} are
placeholders for a {\it pline} or any number of {\it dlines}, respectively,
which are found in the support structure of the current proof, but are
merely copied in the application of the particular rule described.

\begin{verbatim}
Rule of Cases

*(D1)  H      !A(O) OR B(O)                                          
 (H2)  H,H2   !A(O)                                Case 1: D1
 (P3)  H,H2   !C(O)                                          
 (H4)  H,H4   !B(O)                                Case 2: D1
 (P5)  H,H4   !C(O)                                          
*(P6)  H      !C(O)                           Cases: D1 P3 P5

Support Transformation: (P6 D1 ss) ==> (P3 H2 ss) (P5 H4 ss) 
\end{verbatim}

Note that the specified support transformation tells \TPS what lines
it expects to be there, when the rule is applied, and which lines
should be new.  In this case, {\tt P} and {\tt Dab} are expected to be new,
the others are to be constructed.  Of course, these are only defaults,
and the user can apply the rule with any combination of lines present or
absent.


\begin{verbatim}
Induction Rule

 (D1) H     ! P 0
 (H2) H,H2  ! P m                           Inductive Assumption on m
 (D3) H,H2  ! P . Succ m
*(P4) H     ! FORALL n . NAT n IMPLIES P n           Induction: D1 D3

Support Transformation: (P4 ss) ==> (D1 ss) (D3 H2 ss)
\end{verbatim}

\begin{verbatim}
  Forward Conjunction Rule

 (P1)  H      !A(O)                                          
 (P2)  H      !B(O)                                          
*(P3)  H      !A(O) AND B(O)                               Conj: P1 P2

Support Transformation: (P3 ss) ==> (P1 ss) (P2 ss) 
\end{verbatim}


\begin{verbatim}
Backward Conjunction Rule

*(D1)  H      !A(O) AND B(O)                                          
 (D2)  H      !A(O)                                  Conj: D1
 (D3)  H      !B(O)                                  Conj: D1

Support Transformation: (pp D1 ss) ==> (pp D2 D3 ss) 
\end{verbatim}


\subsection{The LINE-NO-DEFAULTS functions}

There are two functions whose job it is to determine defaults for line
numbers.  The reason we need two functions is, that some of the lines
which appear on the left-hand side of the {\it support-transformation},
may reappear on the right.  The way we handle these connections, is
that we first determine the defaults for lines which are supposed
to exist (the left-hand side of the {\it support-transformation}), then
substitute those values into the right-hand side and call the second
default function.

The function \indexfunction{LINE-NO-DEFAULTS-FROM} is called with one argument
{\wt {line-no-defaults-from {\it default-exist}}} and \indexfunction{LINE-NO-DEFAULTS-TO}
is called with two arguments
{\wt {line-no-defaults-to {\it default-exist} {\it default-new}}}
where

\begin{description}
\item [{\it default-exist} ]  is the left hand side of the support transformation
specified for the rule, with lines that we need the default replaced
by a {\tt \$}, while the other lines are numbers (which means they either have
been figured out by an earlier default function or specified by the user).
Something which is neither {\tt \$} nor a number is one of the ``variables''
{\it d} or {\it p} standing for other {\it dlines} or {\it plines} in the current
{\it support} structure.  They must simply be returned (in the proper place,
of course).

\item [{\it default-new} ]  is the right hand side of the support transformation
specified for the rule, with the same interpretation as for {\it default-exist}.
\end{description}

The output of {\tt LINE-NO-DEFAULTS-FROM} should be a
{\tt {\it default-exists-figured}}, the output of {\tt LINE-NO-DEFAULTS-TO} is
a list {\wt ({\it default-exists-figured} {\it default-new-figured})} in
which all {\tt \$} of the arguments have been filled in.  These functions may
also do a {\tt THROWFAIL}, if one of the requirements R for logical
correctness cannot be satisfied in the given proof structure.

Also note that all lines in {\it default-exists} have already been determined,
when {\it default-new} is called.

The specification which must be meet by the {\tt LINE-NO-DEFAULTS-{\it x}}
functions can be grouped into three classes: requirements which ensure
the logical correctness of the rule application (R), requirements which
make the defaults sensible for the ``usual'' application of the rule (D)
and should never be deviated from, and desired properties, which need
not be satisfied, but approximate what the user would like to see most
of the time.

Note that the scope in this function is restricted by the fact that
it does not examine the logical structure assertions or hypotheses
of the lines in the proof.  This is accomplished by a completely different
mechanism and is not the responsibility of the function.  For instance,
it is perfectly sensible for {\tt LINE-NO-DEFAULTS} to suggest the first
{\it pline} in the current support structure for the backwards conjunction
rule, even though it may not be a conjunction at all! [This may cause mayhem
in rule tactics. The latter assumes that if there is a correct default,
the default function will choose it. Since a new line, properly located,
is always a correct, and possibly a useful default, tactics may miss
an opportunity to apply a rule.]

Subsequently, we will assume that
\begin{description}
\item [{\it default-exist} is ]  $((p_1\; d_{11} \ldots d_{1x_1})\ldots (p_p\; d_{p1}\ldots d_{px_p}))$

\item [{\it default-new} is ]  $((q_1\; e_{11} \ldots e_{1y_1})\ldots (q_q\; e_{q1}\ldots e_{qy_q}))$
\end{description}

\begin{center}
{\bf Requirements for Logical Correctness}
\end{center}

\begin{description}
\item [$R_{1}$ ]  $q_j < p_i$ for all $1\leq i \leq p$, $1\leq j \leq q$.

\item [$R_{2}$ ]  $e_{jk} < q_j$ for all $1\leq j \leq q$, $1\leq k \leq y_j$

\item [$R_{3}$ ]  $d_{ik} < p_i$ for all $1 \leq i \leq p$, $1\leq k \leq x_i$
\end{description}

\begin{center}
{\bf Sensible Defaults Requirements}
\end{center}

The requirements below only make sense, if the lines specified by the user
do not already violate them.  In that case, they must be relaxed to
apply only to the remaining unspecified lines.
\begin{description}
\item [$D_{1}$ ]  A {\it plan-support} pair suggested for an element of {\it default-exist}
must always match a {\it plan-support} pair in the current {\it support} structure
of the (incomplete) proof.

\item [$D_{2}$ ]  A {\it plan-support} pair suggested for an element of {\it default-new}
must consist of entirely new lines, and no two lines among all the suggested
defaults may have the same number.
\end{description}

\begin{center}
{\bf Wishful Thinking}
\end{center}

The following are constraints we would like to met, but is of course
not always possible.

\begin{description}
\item [$W_{1}$ ]  $q_j < q_{j+1}$ for all $1 \leq j < q$

\item [$W_{2}$ ]  $q_j < e_{j+1,k}$ for all $1 \leq j < q$ and 
$1 \leq j \leq q_{j+1}$

\item [$W_{3}$ ]  $d_{ik} < e_{jl}$ for all $1\leq i \leq p$,
$1\leq k \leq x_i$, $1\leq j \leq q$, $1\leq l \leq y_j$.

\item [$W_{4}$ ]  $\lnot\exists {\tt L}. \max{e_{jk}} < {\tt L} < q_j$ for
all $1 \leq j \leq q$.

\item [$W_{5}$ ]  Let $gap_{j} = q_j - \max{e_{jk}}$ for $1 \leq j \leq q$, if
$\{e_{jk} | 1 \leq k \leq y_j\}$ is non-empty, otherwise let $gap_{j} =
q_j - \max{\{{\tt L} | {\tt L} < q_j \lor \exists n\neq j . {\tt L} = q_n\}}$.
Then maximize $gap_{j}$, giving equal ``weight'' to all $1\leq j \leq q$.

\item [$W_{6}$ ]  Minimize $b = \min{e_{jl}} - \max{d_{ik}}$, with an alternative
similar to $W_{5}$ in case any of the sets is empty.

\item [$W_{7}$ ]  Minimize $t = \min{p_i} - \max{q_j}$, with an alternative similar
to $W_{5}$ in case any of the sets is empty.
\end{description}


\section{Updating the {\it support} structure}

Part of the execution of a rule application, is updating the plan
structure; this is one of the reasons why with every rule there 
comes a description of how the plan structure should be updated.
Below we will give a description of what the
function \indexfunction{UPDATE-PLAN} is supposed to accomplish, even in cases
when the rule is used in a way different from the defaults.  Again,
we assume the {\tt UPDATE-PLAN} is called as in 

{\tt (update-plan {\it default-exist} {\it default-new})} \\
where

\begin{description}
\item [{\it default-exist} is ]  $((p_1\; d_{11} \ldots d_{1x_1})\ldots (p_p\; d_{p1}\ldots d_{px_p}))$

\item [{\it default-new} is ]  $((q_1\; e_{11} \ldots e_{1y_1})\ldots (q_q\; e_{q1}\ldots e_{qy_q}))$
\end{description}

Recall that there may be variables appearing in place of a line number.
The following restrictions should be noted:

\begin{itemize}
\item For each {\it plan-support} pair
$(p_i\; d_{i1} \ldots  d_{ix_i})$ in {\it default-exist}, there is at most one
occurrence of a variable among $d_{i1},\ldots,d_{ix_i}$.

\item Any occurrence of a variable in {\it default-exist} is unique.
\end{itemize}

When {\tt UPDATE-PLAN} is called, all arguments are filled in, that is each
place is occupied either by a variable or a line number.

\subsection{{\it support} Structure Transformation in the Default Case}

If the rule is used completely in the default direction, i.e. all
{\it plan-support} pairs in {\it default-exist} exist in the current
{\it support} structure and all pairs in {\it default-new} consist of new lines,
then the effect of the rule application on the {\it support} structure is
straightforward:

\begin{itemize}
\item Delete all pairs matching $(p_i\; d_{i1}\ldots d_{ix_i})$ from the
{\it support} structure and attach to the front the pairs 
$(q_j\; e_{j1} \ldots e_{jy})$.

\item A variable in place of a $p_i$ matches {\bf any} {\it plan-support} pair in the
current proof, as long as the $d_{ik}$ match the corresponding support
lines.

\item A variable in place of a $d_{ik}$ matches the lines which are not matched by
any of the line numbers.  If $p_i$ is a variable, every match for $p_i$
produces a corresponding match of $d_{ik}$.

\item A variable in place of $q_j$ must occur as some $p_i$ and as many copies
of $(q_j\; e_{j1}\ldots e_{jy_j})$
are produced as there are matches of $p_i$.

\item A variable in place of $e_{jl}$ must occur as some $d_{ik}$ and the matched
list of lines in filled in.
\end{itemize}

\subsection{What if ...?}

We will go through all cases which differ from the default application
of the rule and specify what should happen to the {\it support} structure.
Of course, \TPS can not always correctly predict what the user had in
mind, when applying a rule, so the following must partly be considered
heuristics, but they will not always implement the user's devious
intentions.

\begin{enumerate}

\item \label{backtrack} % @tag(backtrack)
{\bf What if . . . } a $p_i$ exists, but is not a {\it pline?}  This case is
delicate and perhaps frequently occurs, if the user does not bother deleting
some lines before backtracking after some previous mistake.  Here execute
a {\tt PLAN-AGAIN} (which may become smart about support lines)\footnote{This is
a small project in itself!}.  This will make $p_i$ into a planned line and
we can handle it the usual way.

\item {\bf What if . . . } a $p_i$ does not exist?  Then, very likely, a rule meant to be used
backwards, was applied in a forward way.  We can't do much here: just ignore
the relevant part of {\it default-exist} completely.

\item {\bf What if . . . } a $p_i$ is a variable, but $d_{ik}$ don't match anything in the current
{\it support} structure.  This is already a special case of something discussed
in the previous section.

\item {\bf What if . . . } a $d_{ik}$ does not exist?  Then we must enter it as a planned line, collecting
$\{d_{il} | l \lneq k\}$ as its support lines.

\item {\bf What if . . . } a $d_{ik}$ does exist, but does not support $p_i$ ($p_i$ not a variable)?  Then
somehow $d_{ik}$ was improperly erased form the supports of $p_i$.
Just treat $d_{ik}$, as if it were supporting $p_i$.

\item {\bf What if . . . } a $q_j$ is a variable (thus exists as a {\it pline})
and matched a line number identical to a $e_{jk}$?  Then we are closing a
gap with a forward rule:  Do not enter the $j^{th}$ {\it plan-support} pair
into the {\it support} structure.

\item {\bf What if . . . } a $q_j$ already exists as a {\it pline}?  In this (probably very rare case)
we are reducing the proof of one planned line to the proof of another planned
line.  Add the $e_{jk}$ as additional support lines (also, of course, pulling it
to the front of the {\it support} structure).

\item {\bf What if . . . } a $q_j$ exists as a {\it dline}?  Here we already proved what we need,
so leave this {\it plan-support} pair out when constructing the new {\it support}
structure.

\item \label{dlineexist} % @tag(dlineexist)
{\bf What if . . . } a $e_{jk}$ exists as a {\it dline}?  Here we may be in a situation similar to
~\ref{backtrack}.  The justification of $e_{jk}$ will be changed according to
the current rule applied.  As far as the {\it support} structure is concerned,
we don't treat it specially.

\item {\bf What if . . . } a $e_{jk}$ exists as a {\it pline}?  Here we are justifying a planned line.
Delete the {\it plan-support} pair for $e_{jk}$ from the current {\it support}
structure.  The justification of $e_{jk}$ will be changed appropriately.

\item {\bf What if . . . } a $e_{jk}$ exists as a {\it hline}?  If {\tt TREAT-HLINES-AS-DLINES} is {\tt T}, do
what you would do to a {\it dlines} (see ~\ref{dlineexist}).  Otherwise, nothing
special is done.

\end{enumerate}

\subsection{Entering Lines into the Proof Outline}

The descriptions in the previous section can, when read carefully, also
serve as a guide to what should happen when entering a line into the
proof outline.  Of course, what should be done is clear, if we are in the
all-default case.  Otherwise we may have to change some justifications
as indicated in the previous section, but otherwise existing lines are left
alone.

Entering lines into the proof could be taken over by the same function, if
we handed it linelabels instead of line numbers in {\it default-exist} and
{\it default-new}.

\section{Defaults for Sets of Hypothesis}

In \tps, the user will rarely ever have to deal explicitly with sets of
hypothesis.  However the detail can be controlled by a flag
called \indexflag{AUTO-GENERATE-HYPS}.  If this flag is {\tt T}, \TPS will not only
generate smart defaults for sets of hypothesis, but make them strong
defaults, which means that the user will never be asked to specify
hypotheses for a line.

There some restrictions on what the user of the {\tt RULES} module may
specify as hypothesis in a rule.  Ignoring for the moment the problem of
fixed hypotheses, like sets representing axioms of extensionality of an
axiom of infinity, the hypotheses for each line $l$ may have the form
$H, s_{1}, \ldots , s_{n}$, where $H$ is a
meta-notation for a set of lines and the $s_i$ are labels for lines
present elsewhere in the rule specification.  Let us use $H_l$ for
this set of specified hypotheses for line $l$.

Note the restriction that there may be only one variable standing for
``arbitrary'' sets of lines in any single rule description.

Defaults {strong or not} for the hypotheses of lines are only calculated
after all line numbers have been specified.  This includes existent and
non-existent lines equally.  The algorithm below will always generate
legal applications of the rule, at the same time generating the ``correct''
set of hypotheses for each line.  The algorithm will almost always be
adequate, in the sense that the user will almost never need to explicitly
add hypotheses to a deduced line or drop hypotheses from a planned line.
There are cases, however, where this may still be necessary (see discussion
below).

\subsection{The Algorithm}

Here, unlike in other parts of the {\tt OUTLINE} modules, we do not need to
refer to the {\it support} structure.  Instead let us view the rule as if
we were to infer the {\it plines} from all the {\it dlines} specified in the rule,
and let us disregard hypothesis lines ({\it hlines}) for the moment.

For a given line {\it l} (in the rule specification) we now let $S_l$ stand
for the set of lines in the hypotheses which were explicitly specified
in the rule description (corresponds to $s_1,\ldots,s_n$
above) and let $L_l$ the actual list of hypotheses for the
line, which must either be matched or constructed (depending on whether
the line existed or not).  Furthermore let $H$ stand for the unique
name for an ``arbitrary'' set of lines which appears in zero or more of
the lines in the rule description.

Let us first consider the case that the hypotheses specified in the rule
description do not contain $H$.  For {\it dlines} {\it d} we must check

$L_d\subseteq S_d$

and for {\it plines} {\it p} we need to check

$S_p \subseteq L_p$

For {\it dlines} $d$ which contain $H$ among their hypotheses,
we must satisfy

$L_d \subseteq H \cup S_d$

and, if we are filling in hypotheses for a new line, we would like to choose
$L_d$ as large as possible, so it satisfies this equation.  From another point
of view, namely when we match existent lines, we find out some constraint on
$H$:

$L_d \setminus S_d \subseteq H$

On the other hand, for any given {\it pline} $p$, we obtain

$H \cup S_p \subseteq L_p$ 

or equivalently 

$H \subseteq L_p$ and $S_p \subseteq L_p$

Here we would like to make $L_p$ as small as possible (the fewer hypotheses
we used, the stronger the statement result).  Alternatively, the second line
can again be viewed as a constraint on $H$ when matching an existent
{\it pline}.

This leads to the following algorithm for determining set of hypothesis:

\begin{enumerate}
\item Let $D_{exist}$ be the set of {\it dlines} which exist in the current proof.
Then set 
$$H_{lower} = \bigcup \{L_d - S_d | d \in D_{exist} \land
H \in H_d\}.$$
  Also let $L_d$ be the strong default for the
hypotheses of line $d$ for each $d\in D_{exists}$.

\item \label{latermodified} % @tag(latermodified)
Let $P_{exist}$ be the set of {\it plines} which exist in the current proof.
Then set 
$$H_{upper} = \bigcap \{L_p |  p \in P_{exist} \land
H\in H_p\}.$$
Also let $L_p$ be the strong default for the
hypotheses of line $p$ for each $p\in P_{exists}$.

\item If {\it not}  $H_{lower} \subseteq H_{upper}$   the application of the inference
rule is illegal. (Do a {\tt THROWFAIL} with proper message.)

\item If both, $H_{lower}$ and $H_{upper}$ are undefined (empty intersection or union,
respectively), do not fill in any further defaults.

\item If exactly one of $H_{lower}$ and $H_{upper}$ is undefined, let $H_{lower} := H_{upper}$
or vice versa.

\item For non-existent {\it dlines} {\it d}, we let $L_d = H_{upper} \cup S_d$.
If {\tt AUTO-GENERATE-HYPS} is {\tt T}, make $L_d$ the strong default for that
argument, otherwise just a regular default.

\item For non-existent {\it plines} {\it p}, we let $L_p = H_{lower} \cup S_p$.
If {\tt AUTO-GENERATE-HYPS} is {\tt T}, make $L_d$ the strong default for that
argument, otherwise just a regular default.
\end{enumerate}

This algorithm is coded in a separate function for each rule.
For the rule {\it rule}, the function is called {\wt {\it rule}-HYP-DEFAULTS}
and is called (when appropriate) form within {\wt {\it rule}-DEFAULTS}.

\subsection{When the Algorithm is not Sufficient}

We must of course consider the case, when a restriction like
``x not free in $H$'' is imposed upon applications of the inference
rule.  Since we fill in $H_{upper}$ for the hypotheses of the {\it dlines} which
do not exist, we must check whether ``x not free in $H_{upper}$''.  It may
be the case, however, that all {\it dlines} actually already existed.  In this
case, it would be sufficient for the validity of the rule application, to check
whether ``x not free in $H_{lower}$''.  To see this may think of the rule as
first a legal application of the inference rule, leaving out the extra
hypotheses, then enlarging the set of hypotheses of the inferred line,
possibly with lines which contain ``x'' free.

This situation can also come up, when  not all the {\it dlines} are specified.
Then we may have been able to make the inference rule application legal,
by leaving out the lines {\tt H} from $H_{upper}$, which violate the condition
``x not free in (the assertion of) {\tt H}''.

This leads to a simple modification of the algorithm above, which would need
much more information about the rule (namely the restrictions), where
we modify the definition of $H_{upper}$ in step ~\ref{latermodified} by 

~\ref{latermodified}$^{*}$.
$H_{upper} = \bigcap \{L_p | p \in P_{exist} \land H\in H_p
\land L_p \;{\rm satisfies\; any\; restriction\; on\; } H\}$.

It seems more reasonable, however, not to place that restriction, but rather
give an error message.  Otherwise the user may only find out much later, that
some of the hypotheses he expected to be able to use, have not been included
in the {\it dlines}, since they violated a restriction.  This makes it necessary,
however, to give the user explicit rules which allow adding hypotheses to
a deduced line or dropping hypotheses from a planned line.

\subsection{Hypothesis Lines}

There are two principal ways hypothesis lines ({\it hlines}) can be treated
in \TPS and since there is very little extra work required, both are
provided for.  The flag \indexflag{TREAT-HLINES-AS-DLINES} controls how hypotheses
lines are handled.

If {\tt TREAT-HLINES-AS-DLINES} is {\tt T}, an {\it hline} may have more
hypotheses than simply {\it hline}.  Also, {\it hlines} may have descriptive
justifications like ``Case {a}'' or ``Ind. Hyp. for n''.  The price you
pay is that hypotheses lines become unique to a subproof and should not
be used elsewhere.  In this case, {\it hlines} are truly treated as
{\it dlines}, and in the above algorithm for determining default for
lines, we mean {\it dline} or {\it hline} whenever we say {\it dline}.

If {\tt TREAT-HLINES-AS-DLINES} is {\tt NIL}, every {\it hline} has exactly
one hypothesis: itself.  Also the justification for any {\it hline} will
be the same, namely the value of the flag \indexflag{HLINE-JUSTIFICATION}
(by default {\tt Hyp}).  What you gain in this case is, that the same
hypothesis line may used many different places in the given proof.  The
default for the hypotheses of an {\it hline} will always be strong and
equal to {\tt ({\it hline})}, anything else will result in an error, even if
perhaps logically correct. Also, in this case, 
if \indexflag{CLEANUP-SAME} is {\tt T}, then \indexcommand{CLEANUP} will
eliminate unnecessary hypotheses.



\input{mating}
\chapter{Merging}

Once a complete mating is found, we enter a merging process.  
The merging process performs the following steps, some of which are described
in more detail in separate sections.

Note: merging still contains bugs, although not very many. If a correct mating
is merged and produces a translation error, or a message of the form "The formula is not provable
as there's no connection on the following path: <path>", then it's likely that a bug in merging is
the culprit. Within merging, the routines for REMOVE-LEIBNIZ, CLEANUP-ETREE, and PRETTIFY are the most likely 
causes of problems. The first only applies for formulae with equality, and can be checked by
trying again with the flag REMOVE-LEIBNIZ set to NIL. For the other two, you need to use \indexother{merge-debug}; 
type {\tt setq auto::merge-debug t} before calling merging, and you can step 
through the process, inspecting the etree at each step and omitting the optional steps. 
This can be a great help in discovering which part of the merging process is causing the bug.

{\bf Note:} In November, 2000, merging was changed to handle the case when a mating
contains nonleaf nodes.  The changes were to REMOVE-LEIBNIZ and RAISE-LAMBDA-NODES.
Also, the final phase was separated into a cleanup phase and a prettify phase.
The prettify code was for the most part rewritten.
In the process of making these changes, this section of the Programmer's Guide
was extended to reflect the current state of merging.  

\begin{itemize}
%\item The function \indexfunction{set-all-parents} ensures that the parent
%slot of each etree node is really the parent node.
\item The expansion tree is processed by the function
\indexfunction{etr-merge} (see section~\ref{etr-merge}) which applies the substitutions for
expansion variables in the expansion tree and merges duplicate expansion nodes.
It returns both the new etree and an alist of nodes corresponding to the
mating.  This is the part that actually corresponds to
the ``merging'' algorithm (Algorithm 84) in Frank Pfenning's thesis\cite{Pfenning86}.

\item Duplicate connections and
connections between nodes that do not occur in the tree
are deleted from the mating
(actually, connection list, in the
local variable \verb+new-conn-list+).

\item If dual instantiation\cite{Bishop98} is used, \indexfunction{modify-dual-rewrites}
is called (see section~\ref{modify-dual-rewrites}).
Then, connections between nodes no longer in the tree are
removed from the mating.

\item \indexfunction{prune-unmated-branches} is called.
If \indexflag{MERGE-MINIMIZE-MATING} is set to T,
this function removes children of expansion nodes
which are not needed to have a complete mating.
The function also calls \indexfunction{replace-non-leaf-leaves} on the etree,
which replaces empty expansion nodes with leaves.  (Note: \indexfunction{leaf-p*}
returns T on any node that has no kids, except true and false nodes.)
See section~\ref{prune-unmated-branches}.

\item If the skolem method (determined by \indexflag{SKOLEM-DEFAULT}) used
is not NIL, then
\indexfunction{subst-skol-terms} is called. 
This replaces terms such as $X M N$ with skolem-terms $SK$
whenever there is a skolem-term $SK$ with \verb+TERM+
slot $X M N$.
See section~\ref{subst-skol-terms}.

\item If the top of the tree is a conjunction whose
first child is an $ADD-TRUTH$-rewrite, then delete the conjunction
leaving only the second child as the expansion tree.
(See \indexflag{TRUTHVALUES-HACK} and \indexflag{ADD-TRUTH}.)

\item If \indexflag{REMOVE-LEIBNIZ} is set to T,
then $Leibniz=$-rewrite nodes are removed.  This is a somewhat
complicated process based on an algorithm described in Frank Pfenning's
thesis\cite{Pfenning86}.  See section~\ref{remove-leibniz}.

\item \indexfunction{subst-vars-for-params} is called.
This replaces skolem-terms with the variable bound
by the corresponding quantifier, if this is possible.
If this is not possible, then we replace the skolem-term
with the value of its \verb+PARAMETER+ slot.

\item $\lambda$-rewrite nodes are raised over any propositional connectives
and skolem/selection nodes.  This lifting stops at expansion nodes
and rewrite nodes other than equiv-implics, equiv-disjs and lambda.
This also moves connections to lambda nodes, with the result that
no connection in the mating involves a $\lambda$-rewrite node after
this step is performed.  See section~\ref{raise-lambda-nodes}.

\item The etree is converted to a propositional jform (including any
nonleaf nodes in the mating) and the current set of connections
is used to set \indexother{active-mating}.

\item The etree is cleaned up by calling \indexfunction{cleanup-etree}.  This $\lambda$-normalizes
expansion terms, may remove some $\lambda$-rewrite nodes, and may modify $Subst=$-rewrite nodes.
See section~\ref{cleanup-etree}

\item The etree is prettified by calling \indexfunction{prettify-etree}.  See section~\ref{prettify}.
This renames bound variables and free variables in the etree that do not occur in the original wff.
We must be careful to avoid variable capture when doing this renaming.  (There were bugs with the old code
because of variable capturing.)

\end{itemize}

\section{Applying Substitutions and Merging Duplicate Expansions}\label{etr-merge}

The functions \indexfunction{etr-merge} and \indexfunction{merge-all}
are in the file {\it\indexfile{mating-merge.lisp}}.  These functions are used
to preprocess the expansion tree in order to make the rest of merging
more efficient.  For a discussion of why this preprocessing is done
first, see section~\ref{unneeded-node-p}.

The function \indexfunction{etr-merge}
calls \indexfunction{make-mating-lists} to create the alist of
mated nodes and the substitution for expansion variables corresponding to the mating.
Then the function \indexfunction{prune-status-0} deletes children of expansion nodes
which have status zero.  The function \indexfunction{substitute-in-etree}
is used to apply the substitution to the etree (this puts the appropriate terms
into the SUBST slot of the expansion variables).
Then the functions \indexfunction{strip-exp-vars-for-etree}
and \indexfunction{strip-exp-vars} are used to replace all expansion variables by
their SUBST slot.
Finally, \indexfunction{merge-all} is called.

The function \indexfunction{merge-all} takes an expansion tree and a mating, and descends into the tree.
At each expansion node, if two expansion terms are identical, their
corresponding trees are merged.  The resulting tree replaces the two
original ones, and the substitution returned is applied to the terms
and trees. The resulting tree and mating are returned.

The actual merging of two children of expansion nodes
is carried out by \indexfunction{treemerge}.
The algorithm is described as Algorithm 84 in Frank Pfenning's thesis\cite{Pfenning86}.
The algorithm also must build a substitution replacing some selected variables
with other selected variables and apply this to the tree.
The function returns three values: the new etree,
the substitution \verb+merge-theta+ for selected variables,
and the new mating.

\section{Detecting Unneeded Nodes}\label{unneeded-node-p}

The function \indexfunction{unneeded-node-p} is used
both by \indexfunction{modify-dual-rewrites}
and \indexfunction{prune-unmated-branches}.  It is defined
in the file {\it\indexfile{mating-merge.lisp}}.
The purpose of \indexfunction{unneeded-node-p} is to
determine if a node is needed to have a complete mating.
If the node has zero status, then
it is not needed.  If the flag \indexflag{MERGE-MINIMIZE-MATING}
is set to NIL, then we insist that it is needed.
Otherwise, we temporarily set the status to zero (essentially
removing the node from the tree),
and use \indexfunction{SPANS} to check if the mating still spans all paths.

% The rest of this section is essentially taken from the old Merging documentation
% in the Programmer's Guide
The function \indexfunction{SPANS} calls
\indexfunction{SPANNING-CLIST-PATH}, which calls \indexfunction{FIND-CHEAPEST-CLIST-SPANNING-PATH}, which
\indexfunction{FIND-ALT-CHEAPEST-CLIST-SPANNING-PATH}.   Note that even in x5207, which is
relatively small, 12 calls to \indexfunction{UNNEEDED-NODE-P} result in 295 calls to
\indexfunction{FIND-CHEAPEST-CLIST-SPANNING-PATH}; these functions are the main reason why 
merging can be so slow, especially in proofs created 
by MS90-3 or MS90-9.

You could possibly (as was done at one time) not test this spanning 
condition, and just check to see if every expansion actually has a connection 
below it.  The problem here is that in ms90-3, by the time we get to the merging
process, we have mated every possible pair in the tree, whether the connection
is necessary or not.  That is why unneeded-node-p was modified to be more 
rigorous, because otherwise it was almost useless.  Additionally, there may be
embedded falsehood nodes below it, which are required to close some paths,
even if there are no mated nodes below it. 

A better spanning function should be used, though actually the one used
is already propositional, but of an earlier generation than Sunil's
propositional search function.  One should
realize, however, that the procedure should use the mating provided (and not 
the eager "mate-everything", because our mating might {\it not} be that big).  
In fact, Dan wrote such a \indexfunction{SPANS} that uses a variant of \indexfunction{PROP-MSEARCH},
and the time used in X5207 by \indexfunction{SPANS} went from 1 second to about .3 sec. 
Unfortunately, \indexfunction{PROP-MSEARCH} (or rather, \indexfunction{PROP-FIND-CHEAPEST-PATH}) appears to 
have the "empty disjunction causes confusion" bug.  (Try MS90-3 on
the formula "falsehood implies A"). 

More drastic changes were tougher to implement. There were a few suggestions:
\begin{itemize}
\item What this is doing is a lot of duplicated effort, so perhaps it would be possible to cache some results.
This would be pretty space-intensive; e.g. THM131 has an astronomical number of vpaths
when it begins merging. It turned out that attempts to make SPANS better by caching the results were pretty silly, 
because the way it is invoked, you can't tell the difference between
sets of arguments.  The differences are made by changing the status of
various lower-level expansion nodes.  So that attempt was abandoned.

\item Perhaps it would be possible to check the paths which the suspect node was on.
It's not clear how to do this.

\item Of course, it might be possible to avoid some of the calls to spans
in the first place (though possibly not with MS90-3), but even eliminating 
half would only save 3 days in the wolf-goat-cabbage problem, without changing what it does. 
\end{itemize}

In the end, the solution used was as follows: when path-focused duplication has been
used, the expansion proof will often have a great deal of redundancy
in the sense that the same expansion term will be used for a given variable
many times. More precisely, if one defines an expansion branch by
looking at sequences of nested expansion nodes, attaching one expansion
term to each expansion node in the sequence, there will be many identical
expansion branches. So one can start by merging the tree in the
sense of eliminating this redundancy (see section~\ref{etr-merge}), and then apply to this much simpler
tree the procedure for deleting unnecessary expansion terms which
we think is using so much time. It turned out to be easiest to do this
by throwing away the mating, and reconstructing it by propositional search
after the tree has been cut down to size. Of course, one could also
preserve the original mating by "merging" it appropriately as one collapsed
the tree.

The precise way in which this was done, in the file {\it\indexfile{mating-merge.lisp}}, was:
\begin{enumerate}
\item Don't do pruning of unnecessary nodes at the beginning of the merge,
when the tree is its greatest size. 

\item Instead, {\it do} prune all branches that couldn't possibly have been used 
They are those that have a zero status. This is probably not necessary,
but certainly makes debugging easier and doesn't cost much.  (See section~\ref{etr-merge}.)

\item After merging of identical expansions has been done, call the original
pruning function, \indexfunction{prune-unmated-branches} (see Section~\ref{prune-unmated-branches}).
\end{enumerate}

Note that the merge process does (or should, anyway) merge the mating 
appropriately as the tree collapses.
On THM131 this takes the time spent on merging from 7 days down to 12 minutes.
This is not so surprising, because it begins with 113 million paths, and after
the merging of duplicate expansions, it's down to around 442 thousand. 

The matingstree top level has its own approach to merging, which is essentially
step (2) above, in which all unused 
expansions are simply thrown away, followed by a regular merge as detailed above. 
Putting step (2) first here 
is necessary because the master expansion tree has many nodes which are irrelevant to any particular proof.

\section{Modify-Dual-Rewrites}\label{modify-dual-rewrites}

The functions for MODIFY-DUAL-REWRITES are in
the file {\it\indexfile{mating-merge.lisp}}.  This is
only called when dual instantiation is used.
The main function, \indexfunction{modify-dual-rewrites},
is described in this section.
First, this function uses the global variable
\indexother{*hacked-rewrites-list*}.  The value
of the global is a list of elements of the form
\begin{verbatim}
(<rewrite node> . 
    (<instantiated wff-or-symbol> . 
                  <leaf with uninstantiated form>))
\end{verbatim}
This list is sorted so that the names of the rewrites are increasing.
The main body is a dolist considering each element of \verb+*hacked-rewrites-list*+.
For each subtree of the form
$$\erew{1}{Equivwffs}{A^-}{\erewstar{}{A\,\lor\, A_1^-}{\edisj{1}{C \, \land \, D^-}{\erewstar{}{C^-}{\eleaf{1}{C_1^-}}}{node^-}}}$$
or 
$$\erew{1}{Equivwffs}{A^+}{\erewstar{}{A\,\land\, A_1^+}{\econj{1}{C \, \land \, D^+}{\erewstar{}{C^+}{\eleaf{1}{C_1^+}}}{node^+}}}$$
is converted to either only use one branch or a slightly modified
tree with an explicit dual rewrite (separate from expanding the definition).

The local variables in the loop are
\begin{itemize}
\item \verb+junct+  $DISJ1$ or $CONJ1$
\item \verb+gsym+  A symbol standing in for the uninstantiated formula $A$.
\end{itemize}

First, we check to make sure the rewrite node $REW1$ is a rewrite
still in the etree.
Assuming it is in the tree, we consider several cases
\begin{enumerate}
\item  If the left child of \verb+junct+ is not needed in
the mating, then the uninstantiated definition is not needed.
We replace the \verb+junct+ node with its second child and
call \indexfunction{fix-shallow-chain} to change the shallow formulas
of the rewrites between $REW1$ and \verb+junct+ to be the second conjunct.
\item  If the right child of \verb+junct+ (\verb+realrew+) is not needed in
the mating, then the instantiated definition is not needed.
We replace the \verb+junct+ node with its first child and
call \indexfunction{fix-shallow-chain} to change the shallow formulas
of the rewrites between $REW1$ and \verb+junct+ to be the first conjunct,
replacing \verb+gsym+ by $A$ when necessary.
\item  Otherwise, both are needed.  In this case, we change the tree to have
a form like
$$\erew{0}{Dual}{A^-}{\erew{1}{Equivwffs}{A\,\lor\, A^-}{\erewstar{}{A\,\lor\, A_1^-}{\edisj{1}{C \, \land \, D^-}{\erewstar{}{C^-}{\eleaf{1}{C_1^-}}}{node^-}}}}$$
This makes the dual rewrites easier to recognize and handle in the cleanup code.
\end{enumerate}

% This is the way Matt handled dual rewrites in merging.
% I've replaced it with modify-dual-rewrites described above. - cebrown 9/8/01
% \section{Prune-Unmated-Rewrites}\label{prune-unmated-rewrites}
% 
% The functions for PRUNE-UNMATED-REWRITES are in
% the file {\it\indexfile{mating-merge.lisp}}.  This is
% only called when dual instantiation is used.
% The main function, \indexfunction{prune-unmated-rewrites},
% is described in this section.
% First, this function uses the global variable
% \indexother{*hacked-rewrites-list*}.  The value
% of the global is a list of elements of the form
% \begin{verbatim}
% (<rewrite node> . 
%     (<instantiated wff-or-symbol> . 
%                   <leaf with uninstantiated form>))
% \end{verbatim}
% This list is sorted so that the names of the rewrites are increasing.
% The main body is a do loop considering each element of \verb+*hacked-rewrites-list*+.
% The algorithm appears to be designed for a general case where dual instantiation subtrees have the form
% $$\erew{1}{Equivwffs}{A^-}{\erewstar{}{A\,\lor\, A_1^-}{\edisj{1}{C \, \land \, D^-}{\erewstar{}{C^-}{\eleaf{1}{C_1^-}}}{node^-}}}$$
% or 
% $$\erew{1}{Equivwffs}{A^+}{\erewstar{}{A\,\land\, A_1^+}{\econj{1}{C \, \land \, D^+}{\erewstar{}{C^+}{\eleaf{1}{C_1^+}}}{node^+}}}$$
% However, in most cases (every case?) we will have a tree of the form
% $$\erew{1}{Equivwffs}{A^-}{\edisj{1}{A\,\lor\, A_1^-}{\eleaf{1}{A^-}}{node^-}}$$
% or
% $$\erew{1}{Equivwffs}{A^+}{\econj{1}{A\,\land\, A_1^+}{\eleaf{1}{A^+}}{node^+}}$$
% (in particular, the local variable \verb+chain+ will have value NIL).
% In some cases we can replace such a node with a tree of the form
% $$\erew{1}{Equivwffs}{A}{\erewstar{}{A_1}{node}}$$
% or simply the leaf
% $$\eleaf{1}{A}$$
% (There are notes below about possible bugs if we encounter the general case.)
% 
% The do loop includes local variables described below.
% \begin{itemize}
% \item \verb+junct+  $DISJ1$ or $CONJ1$
% \item \verb+mated-nodes+  A list of names of nodes that occur in the mating.
% \item \verb+chain+  A boolean flag which is set to T
% when there is a nonempty chain of rewrites between \verb+junct+
% and $REW1$ (usually there is no chain).
% \end{itemize}
% First, we check to make sure the rewrite node $REW1$ is still in the etree
% (as it may have been deleted by an earlier pass of the loop -- sorting
% should guarantee that we process ancestors of rewrite nodes before descendants).
% Assuming it is in the tree, we consider several cases
% \begin{enumerate}
% \item  If the right child of \verb+junct+ (\verb+realrew+) contains no
% mated nodes, then the instantiated definition (``real rewrite'') is not used.  So,
% we remove this entry from \verb+*hacked-rewrites-list*+ and
% we replace $REW1$ with $LEAF1$.  {\bf Possible Bug:} This assumes
% $LEAF1$ and $REW1$ have the same shallow formula, which may always
% be true.  However, the code is designed to handle chains of rewrites,
% so Matt must have conceived of a case when there are such rewrites.
% With such rewrites, it is not clear $LEAF1$ and $REW1$ have the same
% shallow formula.
% \item  If there are connections involving nodes below the right child
% of \verb+junct+, consider the mating with such connections deleted
% (\verb+newclist+).  If this is a complete mating, then the instantiated
% definition (``real rewrite'') is not needed.  So,
% we remove this entry from \verb+*hacked-rewrites-list*+ and
% we replace $REW1$ with $LEAF1$.  {\bf Possible Bug Again:} This case also assumes
% $REW1$ and $LEAF1$ have the same shallow formula.
% \item  Assuming the instantiated definition is needed,
% check if $LEAF1$ occurs in the mating or if it is not needed
% (as judged by the function \indexfunction{unneeded-node-p}, see section~\ref{unneeded-node-p}).
% \begin{enumerate}
% \item If there are no abbreviations in the shallow formula of $REW1$,
% then this must be an equality rewrite.  We remove this entry from
% \verb+*hacked-rewrites-list*+ and replace $REW1$
% with the right child of \verb+junct+.  ({\bf Probable Bug:}  There doesn't seem 
% to be any reason why $REW1$ would have the same shallow formula as the right child
% of \verb+junct+.  Probably the intention was to replace \verb+junct+
% with its right child as in the next case.)
% \item Otherwise, we assume it is a definition instantiation (other than equality).
% ({\bf Possible Bug:}  One could imagine a case in which equality is expanded, but
% there are still abbreviations.)  We remove this entry from 
% \verb+*hacked-rewrites-list*+ and replace \verb+junct+ with its right child.
% (If \verb+chain+ is T, then \indexfunction{fix-shallow-chain} replaces the
% shallow formula of each such rewrite with the right side of the conjunction or disjunction, i.e.,
% the half of the shallow formulas corresponding only to the instantiated part.)
% \end{enumerate}
% \item  Otherwise, both sides are needed, so we do nothing.
% \end{enumerate}
% 
% Just before returning, \indexfunction{prune-unmated-rewrites} calls
% \indexfunction{replace-non-leaf-leaves} to replace TRUE nodes and
% FALSE nodes with leaves.

\section{Prune-Unmated-Branches}\label{prune-unmated-branches}

The purpose of the function \indexfunction{prune-unmated-branches}
in the file {\it\indexfile{mating-merge.lisp}} is to delete some children of expansion
nodes by checking
if it is really needed in the proof.  Each node is checked using \indexfunction{unneeded-node-p}
(see section~\ref{unneeded-node-p}).

Before returning, this function calls \indexfunction{replace-non-leaf-leaves} on the etree,
which replaces all empty expansion nodes with leaves.  (Note: \indexfunction{leaf-p*}
returns T on any node that has no children, except true and false nodes.)

\section{Subst-Skol-Terms}\label{subst-skol-terms}

The function \indexfunction{subst-skol-terms}
in the file {\it\indexfile{mating-merge.lisp}}
is used to replace terms in the etree which
correspond to skolem terms by the skolem-term itself.
For example, the etree might contain a skolem-term $SK$
with slots
\begin{itemize}
\item \verb+PARAMETER+ : for example, $X_{\greeko\greekb}$
\item \verb+TERM+ : for example, $X^2_{\greeko\greekb\greeka\greeka} M N$
\end{itemize}
The shallow formulas and expansion terms in the etree
might contain subterms of the form
$X^2_{\greeko\greekb\greeka\greeka} M N$
(up to $\lambda$-conversion).  These are replaced
by the skolem-term $SK$.

The function \indexfunction{subst-skol-terms-main}
actually does the work.
It takes an argument \verb+skol-terms+, which is
an list of pairs \verb+(<term> . <skolem-term>)+
where each \verb+<skolem-term>+ is the skolem term for
a skolem node, and the \verb+<term>+ is the gwff
in the \verb+TERM+ slot of the \verb+<skolem-term>+.
The function traverses each term doing the (destructive)
replacement
in the shallow formula and/or expansion terms.

\section{Remove-Leibniz}\label{remove-leibniz}

The functions for REMOVE-LEIBNIZ are in the {\it
\indexfile{mating-merge-eq.lisp}}.  The functions
described here are
\begin{itemize}
\item \indexfunction{remove-leibniz-nodes}
\item \indexfunction{pre-process-nonleaf-leibniz-connections}
\item \indexfunction{remove-leibniz}
\item \indexfunction{cleanup-leibniz-expansions}
\item \indexfunction{remove-spurious-connections}
\item \indexfunction{check-shallow-formulas}
\item \indexfunction{apply-thm-146}
\end{itemize}

\begin{enumerate}
\item {\bf\indexfunction{remove-leibniz-nodes}}  This is the main function.
It collects negative and positive $Leibniz=$-rewrite nodes.
The connection list is pre-processed (by \indexfunction{pre-process-nonleaf-leibniz-connections})
so that any mates to nonleaf nodes
strictly below a negative $Leibniz=$-rewrite is replaced by mating the leaves
below the node.  Next, the function \indexfunction{remove-leibniz} is called
on each negative $Leibniz=$-rewrite node, possibly changing the connection list.
This has the result of changing negative subtrees of the form
$$\erew{1}{Leibniz=}{A = B^-}{\sel{1}{\forall q \, .\, q\, A\, \limplies\, q\, B^-}
{q_0}{\eimp{1}{q_0\, A\, \limplies \, q_0\, B^-}
{\erewstar{\lambda/Equivwffs}{q_0\, A^+}{\eleaf{1}{q_0\, A_0^+}}}
{\erewstar{\lambda/Equivwffs}{q_0\, B^-}{\eleaf{2}{q_0\, B_0^-}}}}}$$
by trees of the form
$$\erewstar{\lambda}{A = B^-}{\erewstar{Equivwffs}{A_1 = B_1^-}{\erew{1}{Refl=}{C = C^-}{\truenode{1^-}}}}$$
or simply a leaf
$$\eleaf{2}{A = B^-}$$

{\bf Remark:}  The notation $\erewstar{}{C}{\cdots}$ indicates a chain of
rewrites starting with shallow formula $C$.  This chain may be empty,
a single node, or several nodes.

In such cases, connections to $LEAF1^+$ are deleted from the connection list.
The function \indexfunction{deepen-negated-leaves} is called, but this
(apparently) has no effect unless \indexfunction{make-left-side-refl}
returns NIL (and it currently always returns T).

Finally, \indexfunction{cleanup-leibniz-expansions} is called in order to
change positive $Leibniz=$-rewrites to $Subst=$-rewrites, destructively changing
a positive subtree of the form
$$\erew{1}{Leibniz=}{A = B^+}{\erewstar{}{\forall q\, . \, q\, A \, \limplies\, q\, B^+}
{\gexpnode{1}{\forall q\, . \, q\, A_1 \, \limplies\, q\, B_1^+}
{Q_1}{\erewstar{\lambda}{Q_1\, A_1 \, \limplies\, Q_1\, B_1^+}
{\eimp{1}{C_1\, \limplies \, D_1^+}{\cdots}{\cdots}}}
{Q_n}{\cdots}}}$$
to a subtree of the form
$$\erewstar{}{A = B^+}{\erew{1}{Subst=}{A_1 = B_1^+}
{\gexpnode{1}{\forall q\, . \, q\, A_1 \, \limplies\, q\, B_1^+}
{Q_{i_1}}{\eimp{1}{Q_{i_1}\, A_1 \, \limplies\, Q_{i_1}\, B_1^+}{\erewstarnf{\lambda}{\cdots}}{\erewstarnf{\lambda}{\cdots}}}
{Q_{i_m}}{\cdots}}}$$
where $1 \leq i_1 \leq \cdots \leq i_m \leq n$.

\item {\bf\indexfunction{pre-process-nonleaf-leibniz-connections}}   Since the
\indexfunction{remove-leibniz} function may replace negative subtrees of the form
$$\erew{1}{Leibniz=}{A = B^-}{\sel{1}{\forall q \, .\, q\, A\, \limplies\, q\, B^-}
{q_0}{\eimp{1}{q_0\, A\, \limplies \, q_0\, B^-}
{\erewstar{\lambda/Equivwffs}{q_0\, A^+}{\eleaf{1}{q_0\, A_0^+}}}
{\erewstar{\lambda/Equivwffs}{q_0\, B^-}{\eleaf{2}{q_0\, B_0^-}}}}}$$
with a leaf of the form
$$\eleaf{2}{A = B^-}$$
we must have some way of dealing with connections to nonleaf nodes such as $SEL1$
and $IMP1$.
This pre-processing function replaces a connection such as $(IMP1\, .\, IMP3)$
with connections to the leaves $(LEAF1\, .\, LEAF3)\, (LEAF2\, .\, LEAF4)$
where $IMP3$ is a positive subtree of the form
$${\eimp{3}{q\, A\, \limplies \, q\, B^+}
{\erewstar{\lambda/Equivwffs}{q_0\, A^-}{\eleaf{3}{q_0\, A_0^-}}}
{\erewstar{\lambda/Equivwffs}{q_0\, B^+}{\eleaf{4}{q_0\, B_0^+}}}}$$

Such connections between leaves are dealt with in the function \indexfunction{remove-leibniz}.

Note that we do allow connections to the node $REW1$ to remain in the mating.
This connection may also be modified in the function \indexfunction{remove-leibniz}.

\item {\bf\indexfunction{remove-leibniz}}  This function
basically corresponds to the proof of Theorem 138
in Frank Pfenning's thesis\cite{Pfenning86}.  
Also, Remark 140 in Frank Pfenning's discusses the choice
between using the formula 
(\verb+subwff+)
$[\lambda x\, . \, A_0 = x]$ or $[\lambda x\, . \, \lnot x = B_0]$
in the algorithm below.  One may lead to a more elegant proof.
In \TPS the choice is made by a call to the function
\indexfunction{make-left-side-refl}, which currently
always returns T.

Suppose we are given a negative subtree of the form
$$\erew{1}{Leibniz=}{A = B^-}{\sel{1}{\forall q \, .\, q\, A\, \limplies\, q\, B^-}
{q_0}{\eimp{1}{q_0\, A\, \limplies \, q_0\, B^-}
{\erewstar{\lambda/Equivwffs}{q_0\, A^+}{\eleaf{1}{q_0\, A_0^+}}}
{\erewstar{\lambda/Equivwffs}{q_0\, B^-}{\eleaf{2}{q_0\, B_0^-}}}}}$$
(Actually, the selection node $SEL1$ might be a Skolem node, but
this is treated the same way.)
The local variables in the function are given the following values:
\begin{itemize}
\item  \verb+param-node+ $SEL1$
\item  \verb+param+ $q_0$
\item  \verb+imp-node+ $IMP1$
\item  \verb+new-refl-node+ $LEAF1$ (or, $LEAF2$ if \indexfunction{make-left-side-refl} were to return NIL)
\item  \verb+subwff+ $[\lambda x\, . \, A_0 = x]$ (or, $[\lambda x\, . \, \lnot x = B_0]$ if \indexfunction{make-left-side-refl} were to return NIL)
\item \verb+new-non-refl-node+ $LEAF2$ (or, $LEAF1$ if \indexfunction{make-left-side-refl} were to return NIL)
\item \verb+non-refl-branch+ the second son of $IMP1$, which is a rewrite node or $LEAF2$
(or, the first son of $IMP1$ if \indexfunction{make-left-side-refl} were to return NIL)
\item \verb+mated-to-refl-node+  list of nodes mated to $LEAF1$
\end{itemize}
We consider two cases
\begin{enumerate}
\item  If $LEAF1$ is connected to $LEAF2$, then $A_0$ and $B_0$
must be identical.  Let \verb+lhs+ be $A$ and \verb+rhs+ be $B$.
If these are identical wffs, then simply change the etree to
be 
$$\erew{1}{Refl=}{A = B^-}{\truenode{1^-}}$$
Otherwise, let $A_1$ (\verb+lhs*+) be the $\lambda$-normal form of $A$
and $B_1$ (\verb+rhs*+) be the $\lambda$-normal form of $B$.
If these are contain no abbreviations, they should be identical, so
we change the etree to be
$$\erew{2}{\lambda}{A = B^-}{\erew{3}{Refl=}{A_1 = B_1}{\truenode{1^-}}}$$
Otherwise, let $A_2$ and $B_2$ be the result of instantiating all definitions (except equiv)
in $A_1$ and $B_1$, resp.  These should be identical, so we can change the subtree
to be
$$\erew{2}{\lambda}{A = B^-}{\erew{3}{Equivwffs}{A_1 = B_1^-}
{\erew{4}{Refl=}{A_2 = B_2^-}{\truenode{1^-}}}}$$

Finally, we remove all connections involving $LEAF1$, $LEAF2$, or
$REW1$.  
We still have a complete mating without these connections.
First note that any path which would have passed through any of these nodes
would have passed through all of them.  Now, the corresponding path
in the jform for the new tree must pass through $TRUE1^-$.

{\bf Possible Bug:}  The point of Theorem 138
in Frank Pfenning's thesis\cite{Pfenning86} is to remove all Leibniz selected variables $q_0$.
However, in this case we are not substituting for the $q_0$, so there may
still be references to it in the tree.  It's unclear, however, if this
causes a problem in this special case.  If it does turn out to be a bug,
probably the fix is to substitute 
the value of \verb+subwff+ for $q_0$.


\item  If $LEAF1$ and $LEAF2$ are not mated, substitute the value of \verb+subwff+
for $q_0$ in the etree.  Assume \verb+subwff+ has value $[\lambda x\, . \, A_0 = x]$ (or, $[\lambda x\, . \, \lnot x = B_0]$.
(Note that this substitution automatically $\lambda$-normalizes and puts $\lambda$-rewrites above the leaves
if they are needed.)  So, the subtree starting at $IMP1$ now has the form
$$\eimp{1}{q_0\, A\, \limplies \, q_0\, B^-}
{\erewstar{\lambda/Equivwffs}{q_0\, A^+}{\eleaf{1}{A_0 = A_0^+}}}
{\erewstar{\lambda/Equivwffs}{q_0\, B^-}{\eleaf{2}{A_0 = B_0^-}}}$$
If $A = B$ is the same as $A_0 = B_0$ up to $\alpha$-conversion,
we replace $REW1$ with $LEAF2$.
Otherwise, replace $REW1$ with a subtree of the form
$$\erew{1}{\lambda}{A = B^-}{\eleaf{2}{A_0 = B_0^-}}$$
or
$$\erew{1}{Equivwffs}{A = B^-}{\eleaf{2}{A_0 = B_0^-}}$$
Since $LEAF1$ is no longer in the tree, we must
delete any connection $(LEAF1 \, . \, LEAF3)$.
We replace each such $LEAF3$ with
$$\erew{3}{Refl=}{A_0 = A_0^-}{\truenode{1^-}}$$
and delete any connection with $LEAF3$.
\end{enumerate}

Consider the following examples, which arise in a proof
of THM15B using mode MODE-THM15B-C.

$$\erew{5}{Leibniz=}{F\, X = X^-}{\skol{5}
{\forall q \, . \, q\, [F\, X] \, \limplies \, q\, X^-}
{Q}
{\eimp{7}{Q\, [F\, X]\,\limplies\, Q\, X^-}
{\eleaf{37}{Q \, [F \, X]^+}}
{\eleaf{38}{Q \, X^-}}}}$$
with connections
$$(LEAF28\, .\, LEAF109)\, (LEAF38\, .\, LEAF33)\, (LEAF37\, .\, LEAF32)$$
$$(LEAF29\, .\, LEAF22)\, (LEAF21\, .\, LEAF108)\, (LEAF105\, .\, LEAF101)$$
$$(LEAF104\, .\, LEAF100)\, (LEAF94\, .\, LEAF93)$$
becomes
$${\eleaf{38}{F\, X = X^-}}$$
removing the connection
$$(LEAF37\, . \, LEAF32)$$

$$\erew{15}{Leibniz=}{[\lambda w \, . \, F \, . \, w \, X] \, F = [\lambda u \, . \, u\, . \, F \, X] \, F^-}
{\skol{5}
{\forall q \, . \, q\, [[\lambda w \, . \, F \, . \, w \, X] \, F] \limplies q\,  [[\lambda u \, . \, u\, . \, F \, X] \, F]^-}
{Q^{18}}{\eimp{13}{Q^{18}\, [[\lambda w \, . \, F \, . \, w \, X] \, F] \limplies Q^{18}\,  [[\lambda u \, . \, u\, . \, F \, X] \, F]^-}
{\erew{19}{\lambda}{Q^{18}\, [[\lambda w \, . \, F \, . \, w \, X] \, F]^+}{\eleaf{93}{Q^{18} \, . \, F\, . \, F\, X^+}}}
{\erew{20}{\lambda}{Q^{18}\,  [[\lambda u \, . \, u\, . \, F \, X] \, F]^-}{\eleaf{94}{Q^{18} \, . \, F\, . \, F\, X^-}}}}}$$
with connections
$$(LEAF94\, .\, LEAF93)\, (LEAF38\, .\, LEAF33)\, (LEAF29\, .\, LEAF22)$$
$$(LEAF21\, .\, LEAF108)\, (LEAF105\, .\, LEAF101)$$
becomes
$$\erew{38}{\lambda}{[\lambda w \, . \, F \, . \, w \, X] \, F = [\lambda u \, . \, u\, . \, F \, X] \, F^-}
{\erew{39}{Refl=}{F\, .\, F\, X = F\, . \, F\, X^-}
{\truenode{3^-}}}$$
removing the connection
$$(LEAF94\, .\, LEAF93)$$

\item {\bf\indexfunction{cleanup-leibniz-expansions}}
Start with a positive subtree (\verb+eq-rew-node+) of the form
$$\erew{1}{Leibniz=}{A = B^+}{\erewstar{}{\forall q\, . \, q\, A \, \limplies\, q\, B^+}
{\gexpnode{1}{\forall q\, . \, q\, A_1 \, \limplies\, q\, B_1^+}
{Q_1}{\erewstar{\lambda}{Q_1\, A_1 \, \limplies\, Q_1\, B_1^+}
{\eimp{1}{C_1\, \limplies \, D_1^+}{\cdots}{\cdots}}}
{Q_n}{\cdots}}}$$
A do loop early in the function pushes the justifications for the initial
rewrites up one step, and changes the shallow formulas from
$$\forall q \, . \, q \, A^\ast \, \limplies\, q\, B^\ast$$
to be the uninstantiated equation
$$A^\ast = B^\ast.$$
Also, the last rewrite in the chain is changed to have justification $Subst=$.
So, the intermediate tree has the form
$$\erewstar{}{A = B^+}
{\erew{n}{Subst=}{A_1 = B_1^+}
{\gexpnode{1}{\forall q\, . \, q\, A_1 \, \limplies\, q\, B_1^+}
{Q_1}{\erewstar{\lambda}{Q_1\, A_1 \, \limplies\, Q_1\, B_1^+}
{\eimp{1}{C_1\, \limplies \, D_1^+}{\cdots}{\cdots}}}
{Q_n}{\cdots}}}$$
After the do loop, the local variable
\verb+exp-node+ has value $EXP1$.

Next, for each child of the expansion node of the form
$$\erewstar{\lambda}{Q_i\, A_1 \, \limplies\, Q_i\, B_1^+}
{\eimp{i}{C_i\, \limplies \, D_i^+}{\cdots}{\cdots}}$$
we remove the initial $\lambda$-rewrites, possibly
adding a $\lambda$-rewrite beneath $IMPi$.
(The function \indexfunction{check-shallow-formulas}
does part of this work.)
This replaces the corresponding son of $EXP1$ with
a subtree of the form
$$\eimp{i}{Q_i\, A_1 \, \limplies\, Q_i\, B_1^+}
{\erewstar{\lambda}{Q_i\, A_1^-}{\cdots}}
{\erewstar{\lambda}{Q_i\, B_1^+}{\cdots}}$$

Next, we call the function \indexfunction{apply-thm-146}.
This may replace some $Subst=$-rewrite nodes with leaves.
There is a description of this function later in this
section.  The function corresponds to Theorem 146
in Frank Pfenning's thesis\cite{Pfenning86}.

As a final step (which occurs in the code in the return
portion of the outermost dolist loop), the
function \indexfunction{remove-spurious-connections}
is called.  This function cleans the mating and expansion
tree, and returns the new mating.

\item {\bf\indexfunction{remove-spurious-connections}}
This function finds connections between nodes with shallow
formula $A=A$ (\verb+bad-conn+).  One of these nodes must be negative
so we can deepen it (if necessary)
to replace it with a new tree of the form
$$\erew{1}{Refl=}{A=A^-}{\truenode{1^-}}$$
Then, we remove this connection from the mating.
After simplifying the mating in this way,
we delete any children of expansion nodes immediately beneath
$Subst=$-rewrite nodes which are
not used in the mating.
Then we call \indexfunction{apply-thm-146}
because we may have simplified some $Subst=$-rewrite
node to be of the appropriate form.
Finally, we return the connection list.

\item {\bf\indexfunction{check-shallow-formulas}}
This function takes a positive equational rewrite node
$REW1$ with shallow formula
$A = B$, an expansion node 
$EXP1$ which is a child of $REW1$, and an implication node $IMPi$ which is a child
of $EXP1$.  Suppose $IMPi$ has shallow formula
$C\, \limplies\, D$.  
This function checks if the $D$ can be obtained from $C$
by replacing occurrences of $A$ by $B$.
If not, the relevant expansion term $Q_i$ must
have the form $[\lambda z\, . \, P]$.
So, we add
$\lambda$-rewrites are added beneath the implication
to make the implication node have the form
$$\eimp{i}{[A/z]P\, \limplies \, [B/z]P}
{\erew{2}{\lambda}{[A/z]P}{node_1}}
{\erew{3}{\lambda}{[B/z]P}{node_2}}$$
Note that we can clearly obtain $[B/z]P$
formula from $[A/z]P$ by replacing some occurrences
of $A$ by $B$.  So, the tree has the appropriate form.

\item {\bf\indexfunction{apply-thm-146}}
This function corresponds to Theorem 146 in Frank Pfenning's
thesis\cite{Pfenning86}.  
If the subtree passed to the function has
the form
$$\erew{n}{Subst=}{A = B^+}
{\uexpnode{1}{\forall q\, . \, q\, A \, \limplies\, q\, B^+}
{Q}{\eimp{1}{C\, \limplies \, D^+}
{\etrgap{\eleaf{1}{A=B^+}}}
{\cdots}}}$$
we can replace this subtree with $LEAF1$.
If the subtree passed to the function has
the form
$$\erew{n}{Subst=}{A = B^+}
{\uexpnode{1}{\forall q\, . \, q\, A \, \limplies\, q\, B^+}
{Q}{\eimp{1}{C\, \limplies \, D^+}
{\cdots}
{\etrgap{\eleaf{2}{A=B^+}}}}}$$
we can replace this subtree with $LEAF2$.
\end{enumerate}

\section{Raise-Lambda-Nodes}\label{raise-lambda-nodes}

The functions of RAISE-LAMBDA-NODES are in the
file {\it\indexfile{mating-merge2.lisp}}.  The
main function is \indexfunction{raise-lambda-nodes}
which calls the following auxiliary functions:
\begin{itemize}
\item \indexfunction{raise-lambda-nodes-skol} (commutes a $\lambda$-rewrite with
a selection or skolem node)
\item \indexfunction{raise-lambda-nodes-aux1} (commutes one or two $\lambda$-rewrites with a conjunction,
disjunction, or implication node)
\item \indexfunction{raise-lambda-nodes-neg} (commutes a $\lambda$-rewrite with a negation node)
\item \indexfunction{raise-lambda-nodes-ab}  (commutes a $\lambda$-rewrite with an $AB$-rewrite
by destructively changing the justifications of the two rewrites and the shallow formula
of the lower rewrite)
\item \indexfunction{raise-lambda-nodes-equiv}  (commutes a $\lambda$-rewrite with an \\
$EQUIV-IMPLICS$-rewrite
or $EQUIV-DISJS$-rewrite by destructively changing the two rewrites)
\end{itemize}

Since $\lambda$-rewrite nodes may be destroyed during this process,
we may need to change the mating.  In fact, we maintain the
following invariant.

{\bf Invariant:}  Once a tree has been processed, there are no
connections to any $\lambda$-rewrite nodes in that tree.
(Note that it is legal to mate two nodes even if the shallow formulas
are only the same up to $\lambda$-normal form.)

The function \indexfunction{raise-lambda-nodes} processes each
child of a subtree (unless the node is a $Subst=$-rewrite, which is handled differently),
then simply returns the resulting tree and
connection list, except in certain cases.
These cases correspond
to the trees on the left of the following diagrams.  We return
the tree on the right.  (Here, $A_\lambda$ is the $\lambda$-normal form of $A$.)

$$\sel{1}{\forall x \, . \, A}{a}{\erew{1}{\lambda}{[a/x]A}{node}} \Rightarrow 
\erew{2}{\lambda}{\forall x \, . \, A}{\sel{1}{\forall x \, . \, A_\lambda}{a}{node}}$$

$$\skol{1}{\forall x \, . \, A}{a}{\erew{1}{\lambda}{[a/x]A}{node}} \Rightarrow 
\erew{2}{\lambda}{\forall x \, . \, A}{\skol{1}{\forall x \, . \, A_\lambda}{a}{node}}$$

$$\econj{1}{A\, \land\, B}{\erew{1}{\lambda}{A}{node_1}}{\erew{2}{\lambda}{B}{node_2}}
\Rightarrow 
\erew{3}{\lambda}{A\, \land\, B}{\econj{1}{A_\lambda \, \land\, B_\lambda}{node_1}{node_2}}$$

$$\econj{1}{A\,\land\, B}{\erew{1}{\lambda}{A}{node_1}}{node_2}
\Rightarrow \erew{2}{\lambda}{A\,\land\, B}{\econj{1}{A_\lambda \,\land\, B}{node_1}{node_2}}$$

$$\econj{1}{A\,\land\, B}{node_1}{\erew{1}{\lambda}{B}{node_2}}
\Rightarrow \erew{2}{\lambda}{A\,\land\, B}{\econj{1}{A \,\land\, B_\lambda}{node_1}{node_2}}$$

$$\edisj{1}{A\, \lor\, B}{\erew{1}{\lambda}{A}{node_1}}{\erew{2}{\lambda}{B}{node_2}}
\Rightarrow 
\erew{3}{\lambda}{A\, \lor\, B}{\edisj{1}{A_\lambda \, \lor\, B_\lambda}{node_1}{node_2}}$$

$$\edisj{1}{A\,\lor\, B}{\erew{1}{\lambda}{A}{node_1}}{node_2}
\Rightarrow \erew{2}{\lambda}{A\,\lor\, B}{\edisj{1}{A_\lambda \,\lor\, B}{node_1}{node_2}}$$

$$\edisj{1}{A\,\lor\, B}{node_1}{\erew{1}{\lambda}{B}{node_2}}
\Rightarrow \erew{2}{\lambda}{A\,\lor\, B}{\edisj{1}{A \,\lor\, B_\lambda}{node_1}{node_2}}$$

$$\eimp{1}{A\, \limplies\, B}{\erew{1}{\lambda}{A}{node_1}}{\erew{2}{\lambda}{B}{node_2}}
\Rightarrow 
\erew{3}{\lambda}{A\, \limplies\, B}{\eimp{1}{A_\lambda \, \limplies\, B_\lambda}{node_1}{node_2}}$$

$$\eimp{1}{A\,\limplies\, B}{\erew{1}{\lambda}{A}{node_1}}{node_2}
\Rightarrow \erew{2}{\lambda}{A\,\limplies\, B}{\eimp{1}{A_\lambda \,\limplies\, B}{node_1}{node_2}}$$

$$\eimp{1}{A\,\limplies\, B}{node_1}{\erew{1}{\lambda}{B}{node_2}}
\Rightarrow \erew{2}{\lambda}{A\,\limplies\, B}{\eimp{1}{A \,\limplies\, B_\lambda}{node_1}{node_2}}$$

$$\eneg{1}{\lnot A}{\erew{1}{\lambda}{A}{node}}
\Rightarrow \erew{2}{\lambda}{\lnot A}{\eneg{1}{\lnot A_\lambda}{node}}$$

The rest of the transformation rules are for two rewrite nodes.
Note that these are destructive operations.

$$\erew{1}{\lambda}{A}{\erew{2}{\lambda}{A_\lambda}{node}} \Rightarrow
\erew{1}{\lambda}{A}{node}$$

$$\erew{1}{AB}{A}{\erew{2}{\lambda}{B}{node}} \Rightarrow
\erew{1}{\lambda}{A}{\erew{2}{AB}{A_\lambda}{node}}$$

$$\erew{1}{Equiv-Implics}{A}{\erew{2}{\lambda}{B}{node}} \Rightarrow
\erew{1}{\lambda}{A}{\erew{2}{Equiv-Implics}{A_\lambda}{node}}$$

$$\erew{1}{Equiv-Disjs}{A}{\erew{2}{\lambda}{B}{node}} \Rightarrow
\erew{1}{\lambda}{A}{\erew{2}{Equiv-Implics}{A_\lambda}{node}}$$

After applying the transformation, if the result is a $\lambda$-rewrite
node $REW$, then we move any connection from $REW$ to its child.
(Actually, in the code we only do this if the original tree is a rewrite,
since in all other cases the top node of the resulting etree could
only be a {\it new} $\lambda$-rewrite.  This is true because only the rewrite
transformation rules are destructive.)

We need to make sure the invariant holds.
If we are given a $Subst=$-rewrite node $REW1$ to process, we start by
pushing connections to $\lambda$-rewrite nodes below $REW1$ to the child of
the $\lambda$-rewrite.  This forces the invariant to hold (no translation applies
to a $Subst=$-rewrite).

In all other cases, the invariant holds for the children because
of the recursive call.

So we have a situation in which there are no connections to $\lambda$-rewrite nodes
which are subtrees of the node $N$ of interest.  Consider the following cases:
\begin{enumerate}
\item Suppose $N$ is not a rewrite.  In this case, there are no connections to $\lambda$-rewrite nodes
in $N$.  Since all the transformations for non-rewrites can only create a new $\lambda$-rewrite node,
there will be no connections to $\lambda$-rewrites in the result (connections can only involve
nodes in the tree before the transformation).
\item Suppose $N$ is a rewrite node.  Again, there are no connections to $\lambda$-rewrite nodes
in $N$.  However, there may be connections to $N$ itself.
Since the $AB$, $Equiv-Implics$, and $Equiv-Disjs$ transformations are destructive,
the node $N$ may become a $\lambda$-rewrite (if it was not already).  In these cases, we have pushed the connections from $N$
to the child of $N$.  We can see the child of $N$ is not $\lambda$-rewrite nodes by examining the
transformation rules.
\end{enumerate}

{\bf Remark about Subst=:}  $Subst=$-rewrites are processed further during the
CLEANUP-ETREE stage (in the function \indexfunction{cleanup-rewrite-node}) described in section~\ref{cleanup-etree}.

\section{Cleanup-Etree}\label{cleanup-etree}

The code for CLEANUP-ETREE is in the file {\it\indexfile{mating-merge-eq.lisp}}.
This (terribly complicated) procedure comes after merging, because we assume that all
exp-vars and skolem-terms have been removed, leaving just ordinary wffs.

First, a general description of the procedure:
\begin{enumerate}
\item  At each expansion term, normalize it and reduce superscripts on
      the bound variables, and make a new expansion 
      which is a "copy", 
\begin{enumerate}
\item  remove unnecessary lambda-norm steps.
\item  make the leaves the same name, so mating still holds
\end{enumerate}
\item  Remove original expansion.
\end{enumerate}
In reality, we just create a whole new expansion tree, not sharing
with original tree at all.

The main functions described below are
\begin{enumerate}
\item \indexfunction{cleanup-etree}
\item \indexfunction{cleanup-all-expansions}  
\item \indexfunction{cleanup-expansion}  
\item \indexfunction{cleanup-rewrite-node}
\end{enumerate}

\begin{enumerate}
\item {\bf\indexfunction{cleanup-etree}}  This is the function called by \indexfunction{merge-tree-real}.
It calls \indexfunction{cleanup-all-expansions} to rebuild the expansion tree cleaning up along the way.

\item {\bf\indexfunction{cleanup-all-expansions}}  Despite the name, this actually
builds a completely new copy of the etree, with special attention paid to
expansion and rewrite nodes.
The arguments are
\begin{itemize}
\item \verb+etree+ the old etree node
\item \verb+shallow+ the shallow formula for the new node, which may be only $\lambda$-equal to the old shallow
\item \verb+parent+ the parent for the new node being created
\item \verb+lambda-normal-p+ a boolean indicating if \verb+shallow+ is $\lambda$-normal
\end{itemize}
For each expansion term and corresponding kid,
call \indexfunction{cleanup-expansion} to obtain the new
($\lambda$-normal) term and new kid.
For rewrite nodes, call \indexfunction{cleanup-rewrite-node}.

\item {\bf \indexfunction{cleanup-expansion}}  We $\lambda$-normalize
the expansion term $t$ to obtain $t'$.  If the new shallow formula
is $\forall u\, . \, A$ or $\exists u\, . \, A$, then the new shallow (\verb+newshallow+)
for the kid is $[t'/u] A$.  If \verb+newshallow+ is
not $\lambda$-normal, then rewrite nodes may need to be included between
the expansion node and this kid.  Then we recursively call \indexfunction{cleanup-all-expansions}
on the kid with the new shallow formula $B$.

\item {\bf \indexfunction{cleanup-rewrite-node}}
There are cases for the different
kinds of rewrite nodes.
\begin{itemize}
\item $\lambda$, $\beta$, $\eta$:  We can skip this rewrite if the new shallow
formula is already $\lambda$-normal.  
Otherwise, we copy the node, normalize the new shallow formula, and
recursively call \indexfunction{cleanup-all-expansions} on the kid.
\item $Subst=$, $Leibniz=$, $Ext=$ (and $Both=$, which is not currently fully supported):
We copy the node, expand the equality in the new shallow formula
and recursively call \indexfunction{cleanup-all-expansions} on the kid.

If the node is a positive $Subst=$-rewrite, we process the new tree further.
We start with a tree of the form
$$\erew{1}{Subst=}{A = B^+}
{\gexpnode{1}{\forall q\, . \, q\, A \, \limplies\, q\, B^+}
{Q_1}{\eimp{1}{Q_1\, A\, \limplies \, Q_1\, B^+}{node_1}{node_2}}
{Q_n}{\cdots}}$$

If $node_1$ or $node_2$ is of the form
$$\erew{2}{\lambda/\beta/\eta}{Q_1 C}{node: D}$$
where $D$ is the $\lambda$-normal form (or, $\beta$- or $\eta$-normal form)
of $Q_1 C$, then
\indexfunction{reduce-rewrites}
modifies the node to be of the form
$$\erew{2}{\lambda/\beta/\eta}{E}{node: D}$$
(and they both, in fact, should be)
where $E$ is $[C/z]F$ and $F$ is the $\lambda$-normal form (or, $\beta$- or
$\eta$-normal form) of $Q_1 z$.
If in fact, $E$ and $D$ are $\alpha$-equal,
we simply replace the node with $node: D$, removing the rewrite altogether.
That is, we change the children of the implications so that
the shallow formulas are normalized, until all redexes are
in $A$ or $B$, or are of the form $[A\, M]$ or $[B\, M]$.

Next, if the implication node is of the form
$$\eimp{1}{\lnot\lnot C\, \limplies \, \lnot\lnot D}{\eneg{1}{\lnot\lnot C}{\eneg{2}{\lnot C}{node_1}}}
{\eneg{3}{\lnot\lnot D}{\eneg{4}{\lnot D}{node_2}}}$$
then \indexfunction{remove-double-negations-merge}
deletes the four negation nodes iteratively until
the tree does not have this form.

There is also code to make sure $IMP1$ is the son of $EXP1$,
but this should already be true because of the function
\indexfunction{cleanup-leibniz-expansions}, described in section~\ref{remove-leibniz}.

Finally \indexfunction{check-shallow-formula} (described in section~\ref{remove-leibniz})
is called in case we need to adjust the children of $IMP1$ to have the
appropriate form.

\item $Equiv-Implics$, $Equiv-Disjs$:  In these cases, we simply copy the node,
rewrite the equivalence in the new shallow formula in the same way as the old,
and recursively call \indexfunction{cleanup-all-expansions} on the kid.
\item $Refl=$:  If the new shallow formula is of the form $A = A$,
we simply copy the node, and recursively
call \indexfunction{cleanup-all-expansions} with new shallow $TRUTH$ on the kid.
If the new shallow formula does not have this form, we build a chain of $\lambda$,
$AB$, $EQUIVWFFS$ rewrites until it does have the form $A = A$, and end the
chain with a true node.  (An example where the new shallow formula does
not have the form $A = A$ is THM144A with mode MODE-THM144A.)
\item $Dual$: The shallow changes from $A$ to $A\lor A$ or
$A\land A$ depending on the polarity of the node, and make
the recursive call.
\item $Equivwffs$:  In this case, essentially just copy the
node, replace the shallow formula with \verb+shallow+ as usual, and recursively call
\indexfunction{cleanup-all-expansions} on the kid.  The tricky part
is computing the new shallow formula for the kid.  We used to mimic
deepening, in particular, using the value of the flag \indexflag{REWRITE-DEFNS}.
Now, we find a chain of explicit steps (expanding defns, lambda normalization, etc)
from the new shallow to the old shallow of the child of the rewrite.

\item $Ruleq$, $Ruleq-Univ$:  We again copy the
node, replace the shallow formula with \verb+shallow+ as usual, and recursively call
\indexfunction{cleanup-all-expansions} on the kid.  And again, the tricky part
is calculating what the new shallow formula of the kid should be.
What it does is let $A$ (\verb+newshallow+) be the 
gwff in the \verb+RULEQ-SHALLOW+ slot of the node.
If this slot is empty, it lets $A$ be the \indexfunction{min-quant-scope}
of \verb+shallow+ if the justification is $Ruleq$, or just the
same \verb+shallow+ if the justification is $Ruleq-Univ$.
The new shallow of the kid is $Qx_1\cdots Qx_n \, . \, A$
where $Q$ is $\forall$ if the node is positive and $\exists$ if the node is
negative, and the $x_i$'s are the variables introduced in this node.
{\bf Possible Bug:}  It is not clear why this case is handled
the way it is.

\item $Truthp$:  We copy the node replacing the shallow formula as usual.
Then recursively call \indexfunction{cleanup-all-expansions} on the kid
and a shallow
formula $A^* \, \lor\, \lnot TRUTH$
obtained from the old shallow formula $A$ by replacing occurrences
of $FALSEHOOD$ with $\lnot TRUTH$ to obtain $A^*$.
{\bf Possible Bug:}  One would expect the new shallow formula of the kid
to be something like $B^* \, \lor \, \lnot TRUTH$ where $B$ is the
new shallow formula of the rewrite.  It's not clear that this causes a problem
though.
\end{itemize}
\end{enumerate}

\section{Prettify}\label{prettify}

The code for PRETTIFY is in the file {\it\indexfile{mating-merge-eq.lisp}}.
Merging used to prettify the variables in the etree during the CLEANUP-ETREE
phase, but there were some examples where prettify would lead to illegal
variable captures.  To fix this, the two phases have been separated,
and all PRETTIFY does is rename all the free and bound variables in the etree.
We go to great lengths to ensure that the renaming does not lead to variable capture.

First, we should work out the theory of such renamings.
We would like to find two substitutions $\theta$ and $\alpha$
taking variables to variables.  The intention is to use $\theta$
for free variables and $\alpha$ for bound variables.  Let $R^\theta_\alpha$
be the renaming function on etrees and wffs.  
On an etree $Q$, $R^\theta_\alpha(Q)$ is simply the result of applying
$R^\theta_\alpha$ to all shallow formulas, expansion terms, and selected variables.
Next we define $R^\theta_\alpha$ inductively on wffs:
\begin{itemize}
\item $R^\theta_\alpha(x) = \alpha(x)$ if $x$ is bound.
\item $R^\theta_\alpha(y) = \theta(y)$ if $y$ is free.
\item $R^\theta_\alpha(c) = c$ if $c$ is a constant.
\item $R^\theta_\alpha(B \, x \, . \, M) = B \, \alpha(x) \, R^\theta_\alpha(M)$
where $B$ is a binder.
\item $R^\theta_\alpha([M\, N]) = [R^\theta_\alpha(M) \, R^\theta_\alpha(N)]$.
\end{itemize}
We would like the renamed wff to be the same as the result of doing some $\alpha$-conversions and
substituting for the free variables.  So, with respect to all the wffs $M$ in the etree $Q$,
we need to know there is an $M'$ with $M =_\alpha M'$ such that $\theta$ is a legal substitution for $M'$
(avoiding any variable capture), and so that $R^\theta_\alpha(M) \equiv \theta(M')$ (identical wffs).

We can guarantee this if we have $\theta$ and $\alpha$ satisfy two conditions with respect to the etree $Q$.
\begin{itemize}
\item [(*)]  For distinct variables $x$ and $z$, and
any subwff $[B_1 \, x \, . \cdots [B_2\, z \, . \, M] \cdots]$ of any wff in $Q$
where $x$ is free in $M$, we must have $\alpha(x)\neq\alpha(z)$.
\item [(**)] For any subwff $[B_2\, z \, . \, M]$ of any wff $N$ in $Q$
where a free occurrence of $y$ in $N$ is in $M$, we must have $\alpha(z)\neq\theta(y)$.
\end{itemize}
In order to prove the result we want, for any collection of variables $\Gamma$,
define an ordinary substitution $\phi^\Gamma$ by
\begin{itemize}
\item $\phi^\Gamma(x) = \alpha(x)$ if $x\in \Gamma$
\item $\phi^\Gamma(y) = \theta(y)$ if $y\not\in \Gamma$
\end{itemize}
Clearly, $\phi$ depends on $\Gamma$, $\theta$, and $\alpha$.
We will omit the $\Gamma$ superscript when possible.
The idea, of course, is that $\Gamma$ is the context of bound variables.
Note that if $\Gamma$ is empty, then $\phi^{\{.\}} \equiv \theta$.

{\bf Proposition.}  Suppose $Q$ is an etree and $\theta$ and $\alpha$ are variable renamings
satisfying conditions (*) and (**) with respect to $Q$.
Let $M$ be any occurrences of a subwff of a wff in $Q$.  Let $\Gamma$ be the collection of variables 
bound in the context $M$ occurs.  Let $\phi^\Gamma$ be defined as above.
Then, there is an $M'$ satifying
\begin{itemize}
\item $M =_\alpha M'$,
\item for each $x\in\Gamma$, $\alpha(x)$ is free for $x$ in $M'$,
\item for each $y\not\in\Gamma$, $\theta(y)$ is free for $y$ in $M'$
(these conditions together give that $\phi$ is a legal substitution for $M'$),
\item $R^\theta_\alpha(M) \equiv \phi(M')$.
\end{itemize}
In particular, if $M$ is a shallow formula, expansion term, or selected variable
(so that $\Gamma$ is empty), we have $R^\theta_\alpha(M) \equiv \theta(M')$ as desired.

{\bf Proof.}
We can prove this by induction on $M$.
\begin{itemize}
\item Suppose $M$ is a variable or constant.  Let $M'$ be $M$.
\item Suppose $M$ is $[N\, P]$.  By induction we have $N'$ and $P'$
with $N =_\alpha N'$, $P =_\alpha P'$, and satisfying the other conditions.
It is easy to see that letting $M'$ be $[N'\, P']$ works.
\item Suppose $M$ is $[B\, z \, .\, N]$ for some binder $B$.
By induction on $N$ and $\Gamma\cup \{z\}$, we have an $N'$
satisfying
\begin{itemize}
\item $N =_\alpha N'$
\item for each $x\in\Gamma$, $\alpha(x)$ is free for $x$ in $N'$
\item $\alpha(z)$ is free for $z$ in $N'$
\item for each $y\not\in\Gamma\cup\{z\}$, $\theta(y)$ is free for $y$ in $N'$
\item $R^\theta_\alpha(N) \equiv \phi^{\Gamma\cup\{z\}}(N')$
\end{itemize}
Let $M'$ be $[B\, \alpha(z) \, . \, [\alpha(z)/z]N']$.
Note that $M =_\alpha M'$ since $\alpha(z)$ is free for $z$ in $N'$.

Suppose we have $x\in\Gamma$.  To check if $\alpha(x)$ is free for $x$ in $M'$,
we need to know if $x$ occurs free in $N'$ in the scope of a binder for $\alpha(x)$.
By the induction hypothesis, we know such a binder cannot be in $N'$ itself,
so the binder would have to be the outermost binder of $M'$, the one for $\alpha(z)$.
That is, we must have $\alpha(x) = \alpha(z)$.  In such a case, we need
to ensure that $x$ does not occur free at all in $M'$.
Now, condition (*) ensures us that $M$ is not of the form $[B\, z\, . N]$
where $x$ (bound in context) occurs free in $N$.  So, $x$ cannot occur free in
the $\alpha$-equivalent $M'$.
So, $\alpha(x)$ is free for $x$ in $M'$.

Suppose $z\not\in\Gamma$.  We need to show $\theta(z)$ is free for $z$ in $M'$.
But this is immediate since $z$ does not occur free in
$[B\, \alpha(z) \, . \, [\alpha(z)/z]N']$. 

Suppose we have $y\not\in\Gamma$, where $y$ is not $z$.
We need to show $\theta(y)$ is free for $y$ in $M'$.
So, we need to show no free occurence of $y$ in $M'$ occurs
withen the scope of a binder for $\theta(y)$.  By the induction
hypothesis, we know no such binder occurs in $N'$.
So, the binder would have to be the outermost binder of $M'$, 
the one for $\alpha(z)$.  That is, we must have $\alpha(z)=\theta(y)$.
In this case, we need to ensure $y$ does not occur free at all in $M'$.
Equivalently, $y$ should not be free in $M$.  But this is precisely what
condition (**) ensures.

Finally, we have
$$R^\theta_\alpha(M) \equiv [B \, \alpha(z) \, . \, R^\theta_\alpha(N)]
\equiv [B \, \alpha(z) \, . \, \phi^{\Gamma\cup\{z\}}(N')]
\equiv \phi(M').$$
\end{itemize}
$\Box$

Now, the algorithm should build $\theta$ and $\alpha$ for $Q$
so that they satisfy (*) and (**).  We also want $\theta$ to
be an injective renaming, so that no two selected variables will
be identified.  Given partial renamings $\theta$ and $\alpha$
satisfying (*) and (**), we need to know if an extension will
continue to satisfy the conditions.  These tests
are implemented by \verb+prettify-free-legal-p+ and \verb+prettify-bound-legal-p+
and are described below.

The actual PRETTIFY functions are
\begin{itemize}
\item {\bf \indexfunction{prettify-etree}}  This is the main function called by {\bf \indexfunction{merge-tree-real}}.
This calls {\bf \indexfunction{prettify-process-vars-in-etree}} to collect the names of free variables and bound variables
in the etree.  We distinguish the ones which occur in the topmost shallow formula since these were supplied by the
user and should not be renamed.  (We also do not rename frees
which are introduced by a rewrite.  This is a bit unusual, but can happen.  An example is
a rewrite instantiating the abbreviation \verb+PLUS+ which introduces the free \verb+S+.
Here we are really thinking of \verb+S+ as being part of the signature, but there's nothing to explicitly
indicate that \verb+S+ is not a variable.)
We start off by sending each such free to itself and each such bound to itself.
(It is easy to see that conditions (*) and (**) are satisfied by any partial identities for $\theta$ and $\alpha$.)
Then we extend $\theta$ and $\alpha$ in stages.  First, whenever $y$ is a selected variable corresponding to a bound
variable $x$ in the original wff, we let $\theta(y) = x$ if this is legal.  Next, whenever $y$ is a selected variable
associated with a bound variable $x$, we try to send both of these to the same ``pretty'' (no superscript) variable,
if this is possible.  (The function {\bf \indexfunction{prettify-identify-free-bound}} is used to try to
send a free and a bound to the same pretty variable.)  Third, we check for bound variables $z_i$
which occur in a subwff of the form
$$[[\lambda z_1 \cdots \lambda z_n \, . \, M] \, A_1 \, \cdots\, A_n]$$
where $A_i$ is a free or bound variable
(such $A_i$ are stored in the properties \verb+bound-try-to-equate-free+
and \verb+bound-try-to-equate-bound+)
.  In this case, we try to send $A_i$ and $z_i$ to the same pretty variable,
if possible (see the functions {\bf \indexfunction{prettify-identify-free-bound}} and
{\bf \indexfunction{prettify-identify-bound-bound}}).  Finally, we choose the rest of $\theta$
and the rest of $\alpha$ using the functions {\bf \indexfunction{get-best-alt-free-name}}
and {\bf \indexfunction{get-best-alt-bound-name}}.  This completes the computation of $\theta$ and
$\alpha$, so we call {\bf \indexfunction{rename-all-vars-in-etree}} to actually do the renaming.  Finally, {\bf \indexfunction{remove-unnecessary-ab-rews}} eliminates $\alpha$-rewrite nodes where the shallow does not change (this may happen since we renamed variables).
\item {\bf \indexfunction{prettify-process-vars-in-etree}}  This collects the frees and bounds in the etree
into the variables \verb+fixed-frees+, \verb+fixed-bounds+,  \\
\verb+frees-to-rename+, and \verb+bounds-to-rename+.
Each free has the property {\tt \indexother{free-must-avoid}} which is eventually set to all bound vars $z$
such that $y$ occurs free in the scope of a binder for $z$.  Each bound variable has
a similar property {\tt \indexother{bound-must-avoid}}.  (Note that we need to use two different names for the properties
since a variable may occur both free and bound in the etree.)
If a free variable $y$ is a selected variable in the etree,
it has the property {\tt \indexother{sel-var-bound}} which is set to the bound variable corresponding to the outermost
binder at the selection node for $y$.
Bound variables $z$ also have properties {\tt \indexother{bound-try-to-equate-bound}}
and {\tt \indexother{bound-try-to-equate-to-free}}.  A variable $x$ will be on one of these lists if
there is a subwff of the form
$$[[\lambda z_1 \cdots \lambda z_n \, . \, M] \, A_1 \, \cdots\, A_n]$$
where $z$ is $z_i$ and $x$ is $A_i$ for some $i$.  In such cases, we will try to
send these variables to the same (pretty) renamed variable.
\item {\bf \indexfunction{prettify-free-rename}}  This extends $\theta$ to include $\theta(y) = y'$.
We also must propagate information about this commitment by including $y'$ in the list \verb+used-frees+,
representing the codomain of $\theta$ and by including $y'$ in the property \verb+not-alpha-image+
for any $b$ in the property \verb+free-must-avoid+ for $y$.  We will use this to ensure no such $b$
will later have $\alpha(b) = y'$.
\item {\bf \indexfunction{prettify-bound-rename}} This extends $\alpha$ to include $\alpha(z) = z'$.
Again, we propagate information by including $z'$ in the property \\
\verb+not-alpha-image+
for any $b$ in the property \verb+bound-must-avoid+ for $z$.  We will use this to ensure no such $b$
will later have $\alpha(b) = z'$.
\item {\bf \indexfunction{prettify-free-legal-p}}  This checks if it is legal to extend $\theta$ to include $\theta(y) = y'$.
First, we check to make sure $\theta(y)$ is not already defined and that $y'$ is not in the codomain of $\theta$ (\verb+used-frees+),
since we want $\theta$ to be injective.  Next, to ensure condition (**) will hold, we make sure there is
no bound $z$ with $\alpha(z) = y'$ and $z$ in the list \verb+free-must-avoid+ for $y$.
\item {\bf \indexfunction{prettify-bound-legal-p}}  This checks if it is legal to extend $\alpha$ to include $\alpha(z) = z'$.
First, we check to make sure $\alpha(z)$ is not already defined.  Of course, we do not mind if many bound variables
are mapped to the same variable, because this is often how we make the proof pretty, so we do not need to check the codomain
of $\alpha$.  We must make sure that $z'$ is not on the list in the property {\tt \indexother{not-alpha-image}} for $z$.
If it is, then there is some free or bound $x$ which is sent to $z'$ and occurs free in some subwff where it would be captured
by a binder $B \, z$ if this binder were renamed to $B \, z'$.  Also, we must make sure there is no bound $b$ on the
list in the property {\tt \indexother{bound-must-avoid}} for $z$ such that $\alpha(b) = z'$.  In such a case,
there would be an occurrence of z which would be captured by a binder for $b$ upon renaming.
These checks ensure condition (*) will hold.
\item {\bf \indexfunction{prettify-free-bound-legal-p}}  Checks if we can send both $y$ and $z$ to a variable $v$.
This involves a bit more checking than just checking that both commitments are independently legal.  Comments
in the code explain the check.
\item {\bf \indexfunction{prettify-bound-bound-legal-p}}  Checks if we can send both $x$ and $z$ to a variable $v$.
This involves a bit more checking than just checking that both commitments are independently legal.  Comments
in the code explain the check.
\item {\bf \indexfunction{prettify-identify-free-bound}}  Given a free $y$ and bound $z$, if both are already committed,
do nothing.  If one is committed to a pretty variable and the other is not committed, send the other to the same pretty
variable, if this is legal.  If neither are committed, compute alternative names for each.  If either have a pretty alternative $v$
which is legal for the other, send both to this $v$.
\item {\bf \indexfunction{prettify-identify-bound-bound}}  Similar to {\bf \indexfunction{prettify-identify-free-bound}}
except with a bound $x$ and another bound $z$.
\item {\bf \indexfunction{pretty-var-p}}  Returns T if the var does not have a superscript.
\item {\bf \indexfunction{get-best-alt-name}}  Given a variable and a legality test, finds
a new legal variable to replace it.
If the old variable is
$w$, $w^n$, $h$, or $h^n$, then the new variable will be given a name
based on whether the type is of a proposition, predicate, relation, function, or individual.
(This depends on the values of the globals 
{\tt \indexother{proposition-var}}
{\tt \indexother{predicate-var}}
{\tt \indexother{relation-var}}
{\tt \indexother{function-var}}
{\tt \indexother{individual-var}}.
The values of some of these globals were being
{\it randomly changed} by a call to \indexfunction{randomvars}
at the beginning of {\bf\indexfunction{merge-tree-real}}.
This may make prettify bugs difficult to reproduce.
We have decided to comment out this call to \indexfunction{randomvars}.)
If the old variable is anything else, say $x$ or $x^n$, then the new
variable will be of the form $x$ or $x^m$.
\item {\bf \indexfunction{get-best-alt-free-name}}  Returns the nicest legal alternative for a free $y$,
using {\bf \indexfunction{get-best-alt-name}} and {\bf \indexfunction{prettify-free-legal-p}}.
\item {\bf \indexfunction{get-best-alt-bound-name}} Returns the nicest legal alternative for a bound $z$,
using {\bf \indexfunction{get-best-alt-name}} and {\bf \indexfunction{prettify-bound-legal-p}}.
\item {\bf \indexfunction{rename-all-vars-in-etree}}  This corresponds to $R^\theta_\alpha$ on etrees.
\item {\bf \indexfunction{scope-problem-p}}  Checks to make sure the new var and old
var are either both free in context, or were both bound by the same binder.  If not, return T.
This will cause \verb+rename-all-vars-in-wff+ to throw a failure, indicating a bug in PRETTIFY.
\item {\bf \indexfunction{rename-all-vars-in-wff}}  This corresponds to $R^\theta_\alpha$
on wffs described above.  We do check to make sure we are avoiding variable capture. 
If a variable capture does occur, there is a bug in PRETTIFY and a failure is thrown.
\end{itemize}

% The rest is old documentation related to cleanup-etree when it also prettified. much of this was written and then commented out 11/00 - cebrown

% As we go through the tree, we are going to be doing some prettifying of
% terms, such as when we instantiate a definition, or do a
% lambda-normalization.  This means that certain bound variables
% will be changed from what they were in the tree which we were
% given.  The danger is that perhaps we have (when
% \indexfunction{subst-vars-for-params} was called earlier
% in the merging process) used the bound variables as the actual
% instantiations.  This could occur at skolem or selection nodes,
% e.g., where a skolem node's shallow formula was $\forall w^3 \, . \, P \, w^3$, and its instantiation parameter was $w^3$;
% the shallow formula
% may have been rewritten to $\forall w\, . \,  P \, w$, so we want to make the
% new instantiation parameter $w$ instead.  Otherwise rules such as
% substitution of equality will fail during the translation process.
% Of course, if we change the instantiation, we must change it
% throughout the tree.  So we keep track of such renamings in
% \verb+list-of-renamings+, and apply the substitution after we have
% finished with the whole process. 
% 
% Two special variables are
% \begin{enumerate}
% \item {\tt \indexother{*renamed-variables*}} is an alist of bound variables (e.g., $x, y, \cdots$)
% which have been renamed to fresh variables (e.g., $xb, ya,\cdots$) by an internal $\alpha$-renaming.
% It appears that these are renamings of variables bound in definitions instantiated in the etree.
% The alist is of the form $(xb\, . \, x)\, (ya\, . \, y)$
% (sending the new vars to old).  (See the description below of \indexfunction{ren-var-xa-internal}.)
% \item {\tt \indexother{list-of-renamings}} is an alist of variables taking
% some selected variables in the etree to renamed selected variables.  This alist
% is of the form $(x^1\, . \, xa)\, (y^2\, . \, yc)$.
% (See the description below of \indexfunction{cleanup-all-expansions}.)
% \end{enumerate}
% 
% \item \indexfunction{prettify-term}
% \item \indexfunction{prettify-term-1}
% \item \indexfunction{prettify-term-aux}
% \item \indexfunction{ren-var-xa-internal}
% \item \indexfunction{sort-out-renamings}
% \end{enumerate}
% 
% and create
% two alists of variables stored in \verb+list-of-renamings+ and \\
% \verb+*renamed-variables*+.
% \verb+list-of-renamings+ has elements of the form $(x^1\, . \, xa)$ where $xa$ is a selected
% variable in the new etree.\\
% \verb+*renamed-variables*+ has elements of the form $(xa\, . \, x)$ where $xa$ is a variable
% used in the new etree.  \verb+*renamed-variables*+ is not an injective substitution as it
% may have two elements such as $(xa\, . \, x)$ and $(xb\, . \, x)$.
% The function \indexfunction{sort-out-renamings} changes \\
% \verb+*renamed-variables*+
% changes some of these conflicts, using\\
%  \verb+list-of-renamings+, so that
% variables that actually occur as selection variables in the new tree will
% not be identified by the substitution.
% The \verb+list-of-renamings+ is applied to the new etree, presumably because the old selected variables
% might still occur in expansion terms and shallow formulas.
% 
% Finally, code at the end of this function takes each pair $(xa \, . \, x)$ in
% \verb+*renamed-variables*+ and uses \indexfunction{carelessly-rename} to
% change occurrences of $xa$ with the ``pretty'' version of $x$ (which is often just $x$ itself).
% 
% \item {\bf \indexfunction{prettify-term}} Just calls \indexfunction{prettify-term-1} which
% actually does the work.
% \item {\bf \indexfunction{prettify-term-1}}  Changes the names of some bound variables, using
% \indexfunction{prettify-term-aux} to determine the new name of a bound variable.
% \item {\bf \indexfunction{prettify-term-aux}}  Given a variable and a term, this finds
% a new variable which can legally replace the old one.  Also, the new variable must
% not already be in the image of \verb+list-of-renamings+.
% If the old variable is
% $w$, $w^n$, $h$, or $h^n$, then the new variable will be given a name
% based on whether the type is of a predicate, function, or individual.
% (This depends on the values of the globals 
% \indexother{\verb+predicate-var+}
% \indexother{\verb+function-var+}
% \indexother{\verb+individual-var+}.
% The values of these globals are lists whose order may be
% {\it randomly changed} by a call to \indexfunction{randomvars}
% at the beginning of \indexfunction{merge-tree-real}.
% This may make prettify bugs difficult to reproduce.
% A quick way around this, when debugging, is to note the
% values of these globals when the bug occurs, explicitly
% set the globals to these values before the next run, and temporarily 
% comment out the call to \indexfunction{randomvars}.)
% If the old variable is anything else, say $x$ or $x^n$, then the new
% variable will be of the form $x$ or $x^m$.
% 
% \item {\bf \indexfunction{ren-var-xa-internal}}  A casual examination of the code
% does not reveal how this function is called.  The main function of PRETTIFY,
% \indexfunction{cleanup-etree}, dynamically sets the flag \indexflag{REN-VAR-FN}
% to be this function.  So, during the execution of \indexfunction{cleanup-etree},
% \indexfunction{ren-var-xa-internal} may be called by \indexfunction{ab-normalize-main},
% called by \indexfunction{ab-normalize}, called by \indexfunction{get-pmdefn},
% called by \indexfunction{instantiate-definitions}.
% These are some (all?) of the possible ways \indexfunction{instantiate-definitions} could be called:
% \begin{enumerate}
% \item \indexfunction{prettify-term-aux} calls \indexfunction{instantiate-all} which calls \indexfunction{instantiate-definitions}.
% \item \indexfunction{cleanup-rewrite-node} calls \indexfunction{instantiate-1-from-list} which calls \indexfunction{instantiate-definitions}.
% \item \indexfunction{cleanup-rewrite-node} calls \indexfunction{instantiate-some} which calls \indexfunction{instantiate-defn}
% which calls \indexfunction{instantiate-definitions}.
% \end{enumerate}
% So, when certain definitions are instantiated, the function may be used to
% change the names of bound variables.  It may also be called in other ways,
% but I don't know of any.
% 
% Now, what does the renaming function do?  It takes a variable such as $x^1$
% and changes it to some new variable such as $xa$ (it tries appending letters to the
% root until it gets a fresh variable name).  Then the pair $(\, xa\, . \, x^1\, )$
% is pushed on to the global \verb+*renamed-variables*+ alist.
% \item {\bf \indexfunction{sort-out-renamings}}  This function
% takes two alists of variables, corresponding to the values
% of \verb+list-of-renamings+ and \\
% \verb+*renamed-variables*+.
% The return value is a modification of the second alist
% meant to ensure that new selected variables go to distinct variables,
% by using the (inverse of the) first alist to find an alternative value.
% 
% For example, suppose the value of \verb+list-of-renamings+ (\verb+old-to-xa+) is
% \begin{verbatim}
% ((|p^128<OAA>| . |pza<OAA>|) (|p^123<OAA>| . |py<OAA>|))
% \end{verbatim}
% and the value of \verb+*renamed-variables*+ (\verb+xa-to-new+) is
% \begin{verbatim}
% ((|pzb<OAA>| . |p<OAA>|) (|pza<OAA>| . |p<OAA>|)
%  (|pz<OAA>| . |p<OAA>|) (|py<OAA>| . |p<OAA>|))
% \end{verbatim}
% In this case, $|pza<OAA>|$ and $|py<OAA>|$ are selected
% variables in the new etree, and so should be sent to
% distinct variables.
% The function returns 
% \begin{verbatim}
% ((|py<OAA>| . |p<OAA>|) (|pza<OAA>| . |p^128<OAA>|)
%  (|pzb<OAA>| . |p<OAA>|) (|pz<OAA>| . |p<OAA>|))
% \end{verbatim}
% having decided to send $|pza<OAA>|$ back to the old
% selection variable $|p^128<OAA>|$.
% 
% The return value is set to \verb+*renamed-variables*+ in \indexfunction{cleanup-etree}.
% \end{enumerate}

\section{Merging Extensional Expansion Proofs}

The code for
merging extensional expansion proofs 
is completely different than the corresponding
code for merging expansion proofs.  Essentially
we translate from an open dag (ext-exp-open-dag)
to a ground dag (ext-exp-dag) via the function \indexfunction{eeod-to-eed-node}
(see \indexfile{ext-exp-open-dags.lisp}).
This translation process deletes any unnecessary parts of the extensional expansion proof.

Prettify for extensional expansion proofs is performed by \indexfunction{ext-exp-dag-prettify}
in \indexfile{ext-exp-dags.lisp}.  The code is similar to the prettify code in \indexfile{mating-merge-eq.lisp}.


\chapter{Unification}

The relevant files are: {\it ms90-3-node.lisp}, {\it ms90-3-unif*.lisp}, 
{\it node.lisp}, {\it unif*.lisp}

\TPS has four unification algorithms, two for first-order logic and
two for the full type theory. Here we are mainly concerned with the two
type theory ones, which differ as follows:

\begin{itemize}
\item UN88 is called by those procedures which do not use
path-focused duplication, and by the \TPS UNIFY top level.
Each variable is a symbol. We use lazy reduction. Head normal form.
General implementation. Can use different strategies for searching the
unification tree. Default breadth-first. Requires storing almost the
entire tree.
When called from mating-search, we search for a success node or generate the
tree to a pre-determined maximum depth.

\item UN90 is called by those procedures which do use
path-focused duplication. No interactive interface exists now.
Each variable has the form (symbol . number)
Terms are reduced to $\lambda$-normal form as in Huet's paper.
Depth-first search. Stores only non-failure leaf node.
Does not store the entire unification tree.
When called from mating-search, we search for the first non-failure node
within the pre-determined maximum depth.  Search for a success node 
only when the mating is complete.
Major drawback: Needs modification to implement subsumption. 
\end{itemize}


\section{Data Structures}
\section{Computing Head Normal Form}
\section{Control Structure}
\section{First-Order Unification}
\section{Subsumption Checking}
There is a subsumption checker for UN88 which uses the slot {\tt subsumed} in 
each node of the unification tree; this is implemented in the file {\it unif-subs.lisp}.
The subsumption-checker is passed the new node and a list of other nodes which 
might subsume it. If \indexflag{SUBSUMPTION-CHECK} is NIL, it returns immediately.
Otherwise, it first checks the flags \indexflag{SUBSUMPTION-NODES} and \indexflag{SUBSUMPTION-DEPTH}
and eliminates all nodes from the list that do not fit the criteria established by these two flags 
(so it might, for example, pick out just those nodes at a depth of less than ten which lie 
either on the path to the new node or at the leaves of the current unification tree). Since
it is possible to add new disagreement pairs to the leaves of the tree under some conditions,
it also rejects any nodes that do not represent the same original set of disagreement pairs
as the new node.

Then it computes a hash function, somewhat similar to Goedel-numbering, by considering each wff
in the set of disagreement pairs at a node. The hash function has to ignore variables, because
we want to catch nodes that are the same up to a change in the h-variables that have been introduced.
These hash numbers are calculated once and then stored
in the {\tt subsumed} slot in the following format: for a dpairset 
%\begin{tpsexample}
\begin{verbatim}
((A1 . B1) (A2 . B2) ...)
\end{verbatim}
%\end{tpsexample}
we first calculate the hash numbers for each wff, and generate the following list:
%\begin{tpsexample}
\begin{verbatim}
(((#A1 . #B1) (A1 . B1)) ((#A2 . #B2) (A2 . B2)) ...)
\end{verbatim}
%\end{tpsexample}
Then, for each disagreement pair, if \#Bi < \#Ai we replace it with ((\#Bi . \#Ai) (Bi . Ai)).
Finally, we sort the list lexicographically by the pairs of hash numbers and store it in the
{\tt subsumed} slot. In future, if we return to this node, we can just read off the hash
function without recalculating it.

Now \TPS compares the dotted pairs of numbers from the hash functions of the new and old node.
If those for the new node are equal to, or a superset of, those for the old node, then we need
to do some more detailed checking. This is the point at which \TPS prints a "?", if \indexflag{UNIFY-VERBOSE}
is not SILENT. Otherwise we know there is no subsumption and proceed to the next node.

If there is still a possibility of subsumption, the next thing to do is to enumerate all the 
ways in which the old node might be considered a subset of the new one. If we are lucky, 
each dotted pair of numbers in a given node will be different from each other and from all other dotted pairs
at that node, and there will only be one way in which this could happen. If we aren't so lucky (if 
there are several disagreement pairs that get the same pair of hash numbers, or if there is a 
disagreement pair where the hash numbers for both wffs are the same), there may
be multiple ways to think about. For each possible way, we output two disagreement pair lists, which 
will be the entire old node and that subset of the new node to which it might correspond, ordered 
so that the nth element of one is supposed to compare to the nth element of the other, for all n.

Next, for each one of these possible ways, we take the two disagreement pair sets given, and begin
to rename the h-variables in them. We start at the left of both sets, and build up a substitution as we move rightwards,
comparing each term to the other symbol-by-symbol.
(Note that we are only replacing variables with other variables.) 
If we reach the end of the term without contradicting ourselves, we output a
"!" and the new node is subsumed. If we fail (because the substitution is inconsistent, or because we reach 
two different variables neither of which is an h-variable), we fail immediately
and go on to the next arrangement, if there is one.

Subsumption-checking can be very slow; set the flag \indexflag{SUBSUMPTION-DEPTH} with care.
Because of this, it was necessary to add time-checking to unification (it was previously only 
done between considering connections). The functions {\tt unify}, {\tt unify-ho-rec} and
{\tt subsumption-check} now check the time if they are called from within a procedure that 
uses time limits (and in order to implement this, many other unification functions have 
been given optional "start-time" and "time-limit" arguments that they do nothing 
with except passing them on to the next function).

\section{Notes}

The code that \TPS uses to handle double-negations is part of the unification 
code. See \indexfunction{imitation-eta} in the file \indexfile{unif-match.lisp}



\chapter{Set Variables}

Inductively, a set type is either the type of propositions
$\greeko$ or a function type $\alpha\beta$ where $\alpha$
is a set type and $\beta$ is any type.  A term of a set type
represents either a proposition, a set, or a relation.

When we refer to set variables, we generally mean
an expansion variable of a set type.  To find
an expansion proof with expansion variables we
may need to instantiate these variables.
Propositional variables are relatively easy to instantiate,
since there are only two possible truth values, $\top$ and $\bot$.
It is far more difficult to instantiate set variables which
are not of propositional type.  Semantically, these set types
may correspond to infinite domains.  Syntactically, we have
logical constants and quantifiers which can be used to create
terms of set types.
In some cases, but certainly not in all cases, the instantiations
can be find using higher-order unification.  For instantiations which require
logical constants or quantifiers, the original method used by
\TPS is that of PRIMSUBS (see section~\ref{setvars:primsubs}).

For references on PRIMSUBS, see ~\cite{Andrews89}, ~\cite{Andrews95b}, and ~\cite{Andrews99}.
Some work regarding instantiating set variables in other
contexts include   % put Felty's paper in logictex.bib
~\cite{Bledsoe77}, ~\cite{Bledsoe79}, ~\cite{Bledsoe83}, ~\cite{Bledsoe93}, 
and ~\cite{Bailin93}.
The SCAN algorithm (see ) % put scan papers in logictex.bib
reduces some second-order formulas
to equivalent first-order formulas, avoiding the need
to instantiate the set variables.

\section{Primitive Substitutions}\label{setvars:primsubs}

Set variables can be instantiated in a pre-processing stage
using Primitive Substitutions.  This depends on the value
of several flags.  There is a subject PRIMSUBS.
The command \verb+LIST PRIMSUBS+ in \TPS will list
the relevant flags.  Some of the main flags that determine
if and how primitive substitutions are generated are
\indexflag{DEFAULT-MS}, \indexflag{DEFAULT-EXPAND},
and \indexflag{PRIMSUB-METHOD}.

Some examples, in principle, might require applying primsubs
beneath primsubs.  An example discussed in ~\cite{Andrews00a}
is the injective version of Cantor's Theorem, {\bf X5309}.

\section{Using Unification to Compute Setsubs}\label{setvars:pr00}

If \indexflag{DEFAULT-MS} is set to \indexother{MS98-1},
\indexflag{MS98-INIT} is set to 2 or 3, and
\indexflag{PRIMSUB-METHOD} is set to \indexother{PR00},
then unification is used to compute instantiations
during pre-processing.
% Explain the difference between setsubs and primsubs + dissolution + example

\section{Set Constraints}\label{setvars:constraints}

An alternative to instantiating set variables in a pre-processing
step is to intertwine instantiating set variables with the mating search.

% Say more here - do mating search on the rigid part, then
% find set variable constraints, then solve some of these, iterate.

Let $v$ be a set variable occurring at the head of some literal.

We write constraints in sequent form.  A constraint is, in practice,
a list of positive and negative literals (or expansion tree nodes).
Usually, positive literals are written on the left of the sequent and negative
literals are written on the right.  In some cases, we write
a positive literal on the right or a negative literal on the left 
and interpret it as the negation of the literal.
A sequent corresponds to a subset of a vertical path on a jform.

Minimal constraints for $v$ are a collection of sequents (these correspond
to subsets of vertical paths in the jform) of the form
$$\Psi | \Gamma(v)\rightarrow [v\, \overline{t}]$$
where $\Gamma(v)$ is a collection of literals 
are not negative literals with $v$ at the head,
and $\Psi$ is a list of selection variables which occur
in the sequent and are banned from occurring in the instantiation
for $v$ (see section~\ref{acyclicity}).
In general, $v$ can occur inside the body of the
literals in $\Gamma$, and this case will be discussed below.
We do not allow $v$ to occur in the argument terms $\overline{t}$.

Maximal constraints can be defined and handled in a dual way. 
We concentrate on minimal constraints for the present.  % say more

It is very easy to see that any collection of minimal constraints
can be simultaneously solved, since $v = \lambda \overline{z} \top$
is a solution.  In interesting cases, this instantiation will fail
other conditions we might need $v$ to satisfy.  What we would prefer
to have is an ``optimal'' solution to the constraints.  In the case of minimal
constraints, ``optimal'' means a minimal solution.

First, consider the case of a single minimal constraint of the form
$$\Psi | \Gamma \rightarrow [v\, \overline{z}].$$
where the arguments $\overline{z}$ are distinct variables
that occur in $\Psi$.  Also, assume that there are no
other variables in $\Psi$.
In this case, we can directly define
the minimal solution as an intersection:
$$\lambda \overline{z} \bigwedge{\Gamma}.$$
The notation $\bigwedge(\Gamma)$ means
$A_1 \wedge\cdots\wedge A_n$
where each $A_j$ is either a positive literal
in $\Gamma$ or $A_j$ is $\lnot B_j$ where
$B_j$ is a negative literal in $\Gamma$.
The case where $\Psi = \overline{z},\overline{w}$
for some extra selected variables $\overline{w}$
is only slightly more complicated.  We can directly
define the minimal solution in this case as
$$\lambda \overline{z} \exists \overline{w} \bigwedge{\Gamma}.$$
This is a legal instantiation for $v$ since all the variables
in $\Psi$ are bound.

Next, consider the more general case in which the minimal constraint is of the form
$$\Psi | \Gamma \rightarrow [v\, \overline{t}]$$
where the arguments $\overline{t}$ need not be distinct 
variables from $\Psi$.  Let $n$ be the length of $\overline{t}$.
In this case, the easiest way to directly define the minimal solution
is
$$\lambda x^1\cdots \lambda x^n \exists \overline{w} . 
x^1 = t^1 \wedge \cdots \wedge x^n = t^n \wedge \bigwedge(\Gamma)$$
for new variables $x^j$ of the same type as $t^j$,
and $\overline{w} = \Psi$.
However, in practice it is easier if we distinguish between
arguments $t^j$ which are actually variables in $\Psi$ and those which are not.
So, let us write $\Psi = \overline{z},\overline{w}$
where each $z\in\overline{t}$.  
For each $j=1,\ldots, n$, 
if $t^j\in\overline{z}$ and
$t^k\neq t^j$ for $k<j$, then let $x^j = t^j$.  Otherwise,
let $x^j$ be a new variable of the same type as $t^j$.
Let $Eqns = \{x^j = t^j | t^j \mbox{ is not the variable } x^j\}$.
In this case, we can directly define the minimal solution as
$$\lambda x^1\cdots \lambda x^n \exists \overline{w} .
   \bigwedge(Eqns) \wedge \bigwedge(\Gamma).$$
Clearly, if $\overline{t}$ actually is the list $\overline{z}$, then this
solution is
$$\lambda \overline{z} \exists \overline{w} . \bigwedge(\Gamma)$$
as before.

Now, consider the case in which there are several minimal constraints of the form
$$\Psi_i | \Gamma_i \rightarrow [v\, \overline{t^i}]$$
for $i=1,\ldots,n$.  Again, we assume $v$ does not occur in $\Gamma_i$,
so we can directly define the minimal solution.
First, let $M_i$ be the minimal solution for each of the
constraints individually defined as above:
$$\lambda \overline{x_i} \exists \overline{w_i} .
   \bigwedge(Eqns_i) \wedge \bigwedge(\Gamma_i).$$
Then we take the union of these
$$\lambda \overline{x} \bigvee_i [M_i \overline{x}]$$
to get the minimal solution for the combined constraints.

Rather than diving into the general case in which $v$ may occur in $\Gamma$,
first consider a familiar example in which we have the two constraints
$$\rightarrow [v\, 0]$$
and
$$w | [v\, w] \rightarrow [v\, [S\, w]].$$
The minimal solution to this is the least set containing
$0$ and closed under the function $S$.  Since we have the
full power of higher-order logic, we can define such a solution
by
$$\lambda x \forall p \, .\, [[p\, 0]\,\land\,[\forall z\, . [v\, z] \limplies [v\, [S\, z]]]]
\,\limplies\, p x.$$

It should be clear at this point that the terms defining these
solutions can become quite large.  Making such instantiations
can be prohibitive in theorem proving, because we may need
to perform unification with the solution in some other part
of the problem.  This suggests it may be simpler to use
the instantiation to prove there is a set satisfying the
conditions we want.  Then using this lemma, the set that
exists is represented by a selected variable (a {\it very}
small term).  Another motivation for using such lemmas
is that we gain more control over what properties of the
sets are included in the lemma.
See section~\ref{exp-pf-lemmas} for more on the implementation
of expansion proofs using lemmas.

Suppose we have a general set of minimal constraints
$$\Psi_i | \Gamma_i(v)\rightarrow [v\, \overline{t_i}].$$
Consider what properties of $v$ an existence lemma should include.
The most obvious is the condition $C_i(v)$:
$$\forall \overline{w_i} . \Gamma_i(v)\rightarrow [v\, \overline{t_i}]$$
where $\Psi_i = \overline{w_i}$.  Of course, as mentioned above,
$C_i(\lambda \overline{x} \top)$, so we certainly need to include more
conditions.  Two other conditions are {\it inversion principles} and
{\it inductive principles}.  An inversion principle for the set of
constraints would be of the form
$$\forall \overline{x} . v \overline{x} \limplies D(v,\overline{x})$$
for some formula $D$.  In words, this principle says that any
element $\overline{x}$ of $v$ must be of a form satisfying $D(v,\overline{x})$.
An induction principle is a statement that the solution really is minimal.
This would be a higher-order statement of the form
$$\forall \overline{p} . \bigwedge_i(C_i'(p))\,\limplies\, v\subseteq p$$
where $C_i'$ is a condition similar to $C_i$ defined above.
The general case below will explain the difference between $C_i'$
and $C_i$.

Consider the minimal constraints
$$y | [A\, y] \rightarrow [v\, y]$$
and
$$z | [B\, z] \rightarrow [v\, z].$$
Here $C_1(u)$ is 
$$\forall y \, . \, [A\, y]\, \limplies [v\, y]$$
and $C_2(u)$ is
$$\forall z \, . \, [B\, z]\, \limplies [v\, z].$$
The inversion principle here would be
$$\forall x\, . [v\, x]\, \limplies\, .\,[A\, x]\, \lor\, [B\, x].$$
The induction principle here would be
$$\forall p\, . C_1(p)\,\land\, C_2(p)\,\limplies\, v\subseteq p.$$
In such cases where none of the minimal constraints are of the
form
$$\Psi | \Gamma(v) \rightarrow [v\, \overline{t}]$$
where $v$ {\it does} occur in $\Gamma$,
the set is fully determined by the conjunction of the
$C_i$ constraints and the inversion principle, since these
amount to an extensional definition of $v$.
In the general case, the induction principle determines the set.
Though the inversion principle follows from the induction principle,
it may be helpful to have the inversion principle when we are
trying to use the lemma to prove the theorem.
The flag \indexflag{INCLUDE-INDUCTION-PRINCIPLE} controls
whether the induction principle is included in set existence lemmas.% at least for now!

Let us return to the familiar example of the constraints
$$\rightarrow [v\, 0]$$
and
$$w | [v\, w] \rightarrow [v\, [S\, w]].$$
In this case, $C_1(v)$ is $[v\, 0]$, and $C_2(v)$ is
$\forall w\,.\, [v\, w]\,\limplies\, [v\, [S\, w]]$.
The inversion principle is the familiar statement
$$\forall x\,.\, [v\, x]\,\limplies\, [[x = 0]\, \lor\,
\exists w\,.\, [x = [S\, w]]\,\land\, [v\, w]].$$
The induction principle is the equally familiar statement
$$\forall p\,.\, [[p\, 0] \land [\forall w\,.\, [p\, w]\,\limplies\, [p\, [S\, w]]]]\,
\limplies \forall w \, .\, [v\, w]\,\limplies\, [p\, w].$$
The inversion principle follows from the induction principle
by instantiating $p$ with
$$\lambda x\,.\, [[x = 0]\, \lor\,
\exists w\,.\, [x = [S\, w]]\,\land\, [v\, w]],$$
but it is clearly easier, if we need the inversion principle,
to have it in the lemma rather than needing to prove it
by instantiating the new set variable $p$.

In the general case in which some constraints
contain a $\Gamma(v)$ with $v$ free in $\Gamma(v)$,
we need machinery to show that there is a solution $v$
satisfying the constraints, the inversion principle,
and the induction principle.  This machinery is provided
by the Knaster-Tarski Fixed Point Theorem.

\subsection{Knaster-Tarski Fixed Point Theorem}\label{knaster-tarski}

The Knaster-Tarski Fixed Point Theorem states that monotone
set functions have fixed points.  There are also versions
showing there are least and greatest fixed points (in fact,
there are a lattice of such fixed points).  % ref Dana Scott & other domain theory sources

{\bf Definitions:}  Suppose $K:\wp(A)\rightarrow \wp(A)$
for a power set $\wp(A)$.  A {\it pre-fixed point} of $K$ is a set $v$ such
that $K(v)\subseteq v$.  A {\it post-fixed point} of $K$ is a set $v$ such
that $v\subseteq K(v)$.  A {\it fixed point} of $K$ is a set $v$ satisfying $K(v) = v$.

{\bf Knaster-Tarski Fixed Point Theorem:}  Suppose $K:\wp(A)\rightarrow \wp(A)$
for a power set $\wp(A)$.  Further suppose $K$ is monotone function in
the sense that for every $v\subseteq w\subseteq A$, $K(v) \subseteq K(w)$.
Then there is a fixed point $u$ of $K$.

{\bf Proof:}  Let $u = \bigcap\{v\in\wp(A) | K(v) \subseteq v\}$.
That is, we define $u$ to be the intersection of all the pre-fixed points of $K$.
We need to show $K(u)\subseteq u$ and $u\subseteq K(u)$.

First, we show $K(u)\subseteq u$.  Suppose $z\in K(u)$.  To show $z\in u$,
we need to show $z\in v$ for every pre-fixed point $v$.  Let $v$ be a
pre-fixed point.  By the definition of $u$, we have $u\subseteq v$.
Since $K$ is monotone, $K(u)\subseteq K(v)$, so $z\in K(v)$.  But
$v$ is a pre-fixed point, so $z\in K(v)\subseteq v$.  Thus, $u$ is
itself a pre-fixed point.  (In fact, it is clearly the least pre-fixed point.)

Since $u$ is a pre-fixed point and $K$ is monotone,
we have $K(K(u)) \subseteq K(u)$.  So, $K(u)$ is a pre-fixed point.
Since $u$ is the least pre-fixed point, we have $u\subseteq K(u)$.
$\Box$

This proof actually shows the following form of the theorem:

{\bf Knaster-Tarski Fixed Point Theorem (Least):}  Suppose $K:\wp(A)\rightarrow \wp(A)$
for a power set $\wp(A)$.  Further suppose $K$ is monotone function in
the sense that for every $v\subseteq w\subseteq A$, $K(v) \subseteq K(w)$.
There is a least pre-fixed point $u$ of $K$ which is also a fixed point of $K$.

A dual proof shows

{\bf Knaster-Tarski Fixed Point Theorem (Greatest):}  Suppose $K:\wp(A)\rightarrow \wp(A)$
for a power set $\wp(A)$.  Further suppose $K$ is monotone function in
the sense that for every $v\subseteq w\subseteq A$, $K(v) \subseteq K(w)$.
There is a greatest post-fixed point $u$ of $K$ which is also a fixed point of $K$.

These statements and proofs have a straightforward representation in
type theory.  The same proof idea works regardless of the arity of the
set type of $u$.  The functions \indexfunction{make-knaster-tarski-lemma},
\indexfunction{make-knaster-tarski-negf}, 
\indexfunction{make-knaster-tarski-leastfp-lemma},
\indexfunction{make-knaster-tarski-leastfp-negf}, 
\indexfunction{make-knaster-tarski-gfp-lemma},
\indexfunction{make-knaster-tarski-gfp-negf},
and others with similar names,
construct (an ftree representation of) an expansion proof of
these different versions of the Knaster-Tarski Theorem
for a given set type.
A proof of the Knaster-Tarski Theorem for the set type of $u$ can be used as a lemma
to show a set existence lemma for $u$ which includes the constraint properties,
the inversion principle, and the induction principle.

{\bf THM2} in \TPS is a version of the Knaster-Tarksi Theorem.

When we apply the Knaster-Tarski theorem, we need to instantiate
$K$ with a set function we know is monotone.  This can be ensured
syntactically.

{\bf Definition:}  We can define a set of terms {\it positive} and
{\it negative} with respect to $u$ by induction.
\begin{itemize}
\item $[u\,\overline{t}]$ is positive with respect to $u$, so long as
$u$ does not occur free in $\overline{t}$.
\item $P_1 \lor P_2$ is positive (negative) with respect to $u$ if both $P_1$ and $P_2$ are.
\item $P_1 \land P_2$ is positive (negative) with respect to $u$ if both $P_1$ and $P_2$ are.
\item $P_1 \limplies P_2$ is positive (negative) with respect to $u$ if $P_1$ 
is negative (positive) with respect to $u$ and $P_2$
is positive (negative) with respect to $u$.
\item $\lnot P$ is positive (negative) with respect to $u$ if
$P$ is negative (positive) with respect to $u$.
\item $\forall x P$ is positive (negative) with respect to $u$ if
$P$ is, or if $x$ is the same variable as $u$.
\item $\exists x P$ is positive (negative) with respect to $u$ if
$P$ is, or if $x$ is the same variable as $u$.
\end{itemize}

{\bf Proposition:} Suppose $P$ is positive with respect to $u$.
Then $\lambda u\lambda \overline{z} P(u,\overline{z})$
represents a monotone function of $u$.

{\bf Proof:}  We can prove this by induction on $P$.
% fill this in
$\Box$

The function \indexfunction{mon-fn-negf} generates an expansion
proof that a particular $\lambda u\lambda \overline{z} P(u,\overline{z})$
is monotone where $P$ is positive with respect to $u$.
This function returns an ftree and pushes new connections
on the special variable \indexother{clist}.

We will solve some constraints by building a monotone function $K$
from the constraints.  Some constraints do not directly give a monotone
function of the set variable.  In these cases, we may want to find
a best approximating.

{\bf Definition:}  Given a function $F:\wp(A)\rightarrow\wp(A)$, let
$K_F:\wp(A)\rightarrow\wp(A)$ and $K^F:\wp(A)\rightarrow\wp(A)$ be
defined by 
$$K_F(u) = \{z | \forall w\, [u\subseteq w\, \limplies \,z\in F(w)]\}$$
and
$$K^F(u) = \{z | \exists w\, [w\subseteq u \,\land \,z\in F(w)]\}.$$

{\bf Proposition:}  $K_F$ and $K^F$ are monotone set functions.
For all $u$, $K_F(u)\subseteq F(u)\subseteq K^F(u)$.  Futhermore,
if $L:\wp(A)\rightarrow\wp(A)$ is a montone set function and
forall $u$, $L(u)\subseteq F(u)$, then forall $u$, $L(u)\subseteq K_F(u)$.
Similarly, if $R:\wp(A)\rightarrow\wp(A)$ is a montone set function and
forall $u$, $F(u)\subseteq R(u)$, then forall $u$, $K^F(u)\subseteq R(u)$.
So, $K_F$ and $K^F$ are the best monotone upper and lower aproximations of $F$.

{\bf Proof:}  Suppose $u\subseteq v$ and $z\in K_F(u)$.  We need to show
that $z\in F(w)$ for every $w\supseteq v$.  Given such a $w$,
apply the definition
of $K_F$ to the set $w$ to obtain $[u\subseteq w]\,\limplies\, z\in F(w)$.
Since $u\subseteq v\subseteq w$, we have $z\in F(v)$.  So, $K_F$ is monotone.  Similarly,
suppose $u\subseteq v$ and $z\in K^F(u)$.  Apply the definition of $K^F(u)$
to obtain a set $w\subseteq u$ with $z\in w$.  Now, since
$w\subseteq u\subseteq v$, this $w$ can be used
to witness that $z\in K^F(v)$.  So, $K^F$ is monotone.

Let $u\in\wp(A)$ be given.  We need to show $K_F(u)\subseteq F(u)$
and $F(u)\subseteq K^F(u)$.  Suppose $z\in K_F(u)$.  Apply the definition
of $K_F$ to $u$.  Since $u\subseteq u$, we have $z\in F(u)$.
Next, suppose $z\in F(u)$.  Then $u$ can witness that $z\in K^F(u)$.

Now suppose $L$ is a monotone function such that for every $u\in\wp(A)$,
$L(u)\subseteq F(u)$.  We need to show for every $u\in\wp(A)$, $L(u)\subseteq K_F(u)$.
Let $u\in\wp(A)$ be given and suppose $z\in L(u)$.  Since $L$ is monotone,
for every $w\supseteq u$, $z\in L(w)$.  So, $z\in F(w)$.  This shows $z\in K_F(u)$
and we are done.

Now suppose $R$ is a monotone function such that for every $u\in\wp(A)$,
$F(u)\subseteq R(u)$.  We need to show for every $u\in\wp(A)$, $K^F(u)\subseteq R(u)$.
Let $u\in\wp(A)$ be given and suppose $z\in K^F(u)$.  So, there is a $w\subseteq u$
with $z\in F(w)$.  So, $z\in R(w)$.  Since $R$ is monotone and $w\subseteq u$,
we have $z\in R(u)$ and we are done.
$\Box$

\subsection{Tracing Through An Example}

Consider again the two minimal constraints
$$\rightarrow [v\, 0]$$
and
$$w | [v\, w] \rightarrow [v\, [S\, w]]$$
where $v$ has type $\greeko\greeki$.
Let us name the literals
\begin{itemize}
\item [L1] $v\, 0$
\item [L2] $v\, w$
\item [L3] $v\, [S\, w]$
\end{itemize}
In the function \indexfunction{ftree-solve-constraint-set},
consider some of the local variables:
\begin{itemize}
\item \verb+f+ is an ftree containing the nodes L1, L2, and L3,
and the expansion variable $v$.
\item \verb+constrs+ has the value ((L1) (L3 L2))
\item \verb+v+ has the value $v$.
\item \verb+vsel+ is a new variable $v^1$ of the same type as $v$.
\item \verb+f3+ is \verb+f+ with $v^1$ (not an expansion variable)
substituted for the expansion variable $v$.
\item \verb+banned-occurs+ ends up having the value (($w$) NIL),
representing $\Psi_1$ and $\Psi_2$.
\item \verb+misc-occurs+ is NIL, assuming $S$ and $0$ are constants.
\item \verb+paths+: ((LF3 LF2) (LF1)) where each LFi
is the ftree node in \verb+f3+ corresponding to Li.
\end{itemize}

Then \indexfunction{make-ftree-setvar-soln} is called with
\begin{itemize}
\item \verb+vsel+: $v^1$
\item \verb+kind+: MIN
\item \verb+paths+: ((LF3 LF2) (LF1))
\item \verb+banned-occurs+: (($w$) NIL)
\item \verb+misc-occurs+: NIL
\item \verb+rec-flag+: T
\end{itemize}
A special dynamic variable \verb+clist+ is used to collect connections
created in the process of building the lemma.
The function \indexfunction{make-min-inv-princ} constructs the inversion principle
$$\forall \,x . \,v^1 \,x \supset \exists \,w [ \,x \,= \,S \,w \land \,v^1 \,w] \lor \,x \,= \,0.$$
Next, \indexfunction{make-min-setvar-lemma-posf} constructs the full lemma \
\begin{center}
\begin{tabular}{r}
$ \exists \,v^2 . [\forall \,w^1 [ \,v^2 \,w^1 \supset \,v^2 . \,S \,w^1 ] \land \,v^2 \,0]$ \\
$\land\,\forall \,x^1 [ \,v^2 \,x^1 \supset \exists \,w [ \,x^1 \,= \,S \,w \land \,v^2 \,w ] \lor \,x^1 \,= \,0 ]$ \\
$\land \forall p\, .\, [\forall \,w^1 [ \,p \,w^1 \supset \,p . \,S \,w^1 ] \land \,p \,0]
\limplies\, \forall x^2\, . \, [v^1\, x^2]\,\limplies\, [p\, x^2].$  \\
\end{tabular}
\end{center}
and the positive ftree that will correspond to how the lemma can be used.
In particular, connections to the nodes LF1, LF2, and LF3 are created
which solve the constraints (i.e., block every verticle path through LF1,
and every vertical path through LF2 and LF3).

In simpler examples which do not require a recursive definition
using the Knaster-Tarski Fixed Point Theorem,
the function \indexfunction{make-min-setvar-lemma-negf}
constructs an ftree proof of the lemma.
In this case, the lemma does require a recursive definition,
so the function \indexfunction{make-clos-setvar-lemma-negf}
is called.  If \indexflag{INCLUDE-INDUCTION-PRINCIPLE} is T,
we need a strong form of the Knaster-Tarski Theorem, in which
we know the set $u$ is the least pre-fixed point.
The function \indexflag{make-knaster-tarski-leastfp-lemma}
is called to construct an ftree proof for this lemma.
Otherwise, \indexflag{make-knaster-tarski-lemma}
gives an ftree proof of the simpler version.
The function \indexflag{make-clos-setvar-lemma-negf-0}
uses the Knaster-Tarski lemma to prove the set existence lemma.

Now, let us examine each of these steps in more detail.

\begin{enumerate}
\item {\bf \indexfunction{make-min-inv-princ}}  Given $v^1$, the paths
(LF1 (LF3 LF2)), and the banned variable information
(($w$) NIL), use the type of $v^1$ to create a formula
$$\forall \,x . \, [v^1\, x] \, \limplies\, InvP_2$$
where $InvP_2$ is formed by \indexfunction{make-min-inv-princ-2}.
$InvP_2$ needs to be positive with respect to $v^1$ so that
we can apply the Knaster-Tarski Theorem later.

\begin{itemize}
\item {\bf \indexfunction{make-min-inv-princ-2}}  Given the bound list
(\verb+zl+) ($x$), the atomic formula (\verb+vzl+) $v x$, as well as
the information above, form a disjunction of inversion principles
corresponding to each path:
$$InvP_4^{(LF3\, LF2)} \, \lor\, InvP_4^{(LF1)}$$
where each $InvP_4$ formula (positive with respect to $v^1$) is constructed by a call to 
\indexfunction{make-min-inv-princ-3}.

\item {\bf \indexfunction{make-min-inv-princ-3}}  Given one of the constraints,
construct equations and a substitution sending some banned variables (those occurring
as an argument of the main literal of the constraint) to the variables
constructed by \indexfunction{make-min-inv-princ} (in this case, $x$).
In this case we have two constraints.  

First consider, (LF1), i.e., $v^1\, 0$.  In this case we only have one literal,
the main literal of the constraint.  Since $0$ is a constant, and not a banned
selected variable, we make an equation $x = 0$ between the argument of $v^1\, x$
and the argument of the main constraint literal $v^1\, 0$.  In general, there
may be several arguments, giving several equations.  Given these equations,
\indexfunction{make-min-inv-princ-4} constructs the formula $InvP_4^{(LF1)}$.

Second consider, (LF3 LF2), i.e., $w|v^1\, w \rightarrow v^1 \, [S\, w]$.
In this case, the main literal is $v^1\, [S\, w]$.  If the argument $[S\, w]$
were $w$, we could replace $w$ by $x$ in the constraint for the purpose
of computing the inversion principle.
Instead, we make an equation $x = [S\, w]$ and call
\indexfunction{make-min-inv-princ-4} to construct the formula $InvP_4^{(LF3\, LF2)}$.

\item {\bf \indexfunction{make-min-inv-princ-4}}  Constructs a formula
existentially binding the remaining banned variables in the constraint.
Since (LF1) has no banned variables, $InvP_4^{(LF1)}$ is $InvP_5^{(LF1)}$
constructed by \indexfunction{make-min-inv-princ-5}.  Since (LF3 LF2)
has the banned variable $w$, $InvP_4^{(LF3\, LF2)}$ is of the form
$$\exists w\, . \, InvP_5^{(LF3\, LF2)}$$
where $InvP_5^{(LF3\, LF2)}$ is constructed by \indexfunction{make-min-inv-princ-5}.

\item {\bf \indexfunction{make-min-inv-princ-5}}  Constructs a conjunct of
the equations generated in \indexfunction{make-min-inv-princ-3} corresponding
to the constraint.  

For the constraint (LF1), $InvP_5^{(LF1)}$ is $x=0$.
Since in this constraint, the constraint has no literals in this
constraint other than the main literal LF1, \indexfunction{make-min-inv-princ-6}
makes no contributation.

For the constraint (LF3 LF2), $InvP_5^{(LF3\, LF2)}$ is
$x=[S\, w]\,\land\, InvP_6^{(LF2)}$
where $InvP_6^{(LF2)}$ is constructed by \indexfunction{make-min-inv-princ-6}
using the extra literals of the constraint, in this case (LF2).

\item {\bf \indexfunction{make-min-inv-princ-6}}  Constructs a conjunct of
formulas for each extra literal of the constraint.  In this case, there is
the one literal LF2, giving $InvP_6^{(LF2)}$ as $v^1\, w$.

In the general case, when a literal LF does not contain the set selected
variable (\verb+vsel+) $v^1$, $InvP_6^{(LF\, .\, <LIST>)}$ will be of
the form $A\,\lor\, InvP_6^{<LIST>}$ where $A$ is the shallow formula
of LF or its negation (if LF is negative).  This will also be
the form if LF is a positive literal only containing $v^1$ at the head.

The more complicated case is when $v^1$ occurs inside the body of the
literal.  Suppose $A(v^1)$ is the shallow formula of such a literal LF
(or its negation if the literal is negative).
Since we will want a formula which is positive with respect to $v^1$,
we let $InvP_6^{(LF\, . \, LEAF)}$ be of a form such as
$$\exists w^i_{\greeko\greeki}\, [[\forall x\, .\, [w^i\, x]\,\limplies\, [v^1_{\greeko\greeki}\, x]]\,\land\, A(w^i)].$$
Of course, in general $\greeko\greeki$ may be any set type and $x$ may be
a list of variables.
See section~\ref{knaster-tarski} for a semantic description of this
as a least monotone upper approximation.
{\bf THM2} is an example where such an approximation is necessary.
\end{itemize}

In the end, we have constructed the inversion principle as follows:
$$\forall \,x . \, [v^1\, x] \, \limplies\, InvP_2$$
$$\forall \,x . \, [v^1\, x] \, \limplies\, .\,InvP_4^{(LF3\, LF2)} \, \lor\, InvP_4^{(LF1)}$$
$$\forall \,x . \, [v^1\, x] \, \limplies\, .\,\exists w\, InvP_5^{(LF3\, LF2)} \, \lor\, InvP_5^{(LF1)}$$
$$\forall \,x . \, [v^1\, x] \, \limplies\, .\,\exists w\, [x=[S\, w]\,\land\, InvP_6^{(LF2)}] \, \lor\, [x=0]$$
$$\forall \,x . \, [v^1\, x] \, \limplies\, .\,\exists w\, [x=[S\, w]\,\land\, [v^1\, w]] \, \lor\, [x=0].$$

\item {\bf \indexfunction{make-min-setvar-lemma-posf}}  Constructs the set
existence lemma \
\begin{center}
\begin{tabular}{r}
$ \exists \,v^2 . \forall \,w^1 [ \,v^2 \,w^1 \supset \,v^2 . \,S \,w^1 ] \land \,v^2 \,0$ \\
$\land \, \forall \,x^1 [ \,v^2 \,x^1 \supset \exists \,w [ \,x^1 \,= \,S \,w \land \,v^2 \,w ] \lor \,x^1 \,= \,0 ]$ \\
$\land \forall p\, .\, [\forall \,w^1 [ \,p \,w^1 \supset \,p . \,S \,w^1 ] \land \,p \,0]
\limplies\, \forall x^2\, . \, [v^2\, x^2]\,\limplies\, [p\, x^2].$ \\
\end{tabular}
\end{center}
and a positive ftree used to solve the constraints.
In general, this would start by universally quantifying
any extra expansion variables and selected variables occurring in the constraints
(\verb+misc-occurs+).  In this case, there are none.  We create a fresh variable
(\verb+v2+) $v^2$ to play the role of the set variable in the formula.
\indexfunction{make-min-setvar-lemma-posf-1} is called to construct
a positive ftree $POSF_1$.  The shallow formula of $POSF_1$ is $LEM_1(v^1)$ the body of
the set existence lemma, using the selected variable $v^1$.
So, we return a selected node $POSF$ with shallow $\exists v^2\,.\, LEM_1(v^2)$
and child $POSF_1$.

\begin{itemize}
\item {\bf \indexfunction{make-min-setvar-lemma-posf-1}}
The main part of the lemma we need to construct is the part that
solves the constraints.  The function \indexfunction{make-min-setvar-lemma-posf-3}
returns a postive ftree $POSF_2$ with shallow formula $Main(v^1)$.  It also
adds connections between literals in $POSF_2$ and the literals LF1, LF2, and LF3
of the constraints.

We have already constructed the inversion principle $InvP$.
If \indexflag{INCLUDE-INDUCTION-PRINCIPLE} is NIL, we construct a positive
ftree for $Main(v^1)\, \land \, InvP$.
If \indexflag{INCLUDE-INDUCTION-PRINCIPLE} is T, we construct a positive
ftree for $Main(v^1)\, \land \, [InvP\,\land\, IndP]$ where $IndP$ is an
induction principle.  In this case, the induction principle is
$$\forall p\, .\, Main(p)\,\limplies\, \forall x^2\, . \, [v^1\, x^2]\,\limplies\, [p\, x^2].$$
The positive ftrees for the inversion principle and the induction principle
are simply constructed by expanding the formula as an ftree, duplicating
the outermost quantifier \indexflag{NUM-OF-DUPS} times.  These are combined
using conjunction nodes with $POSF_2$.

\item {\bf \indexfunction{make-min-setvar-lemma-posf-2}}  For each constraint $P$,
we make a conjunction of $POSF_3^P$ obtained by
calling \indexfunction{make-min-setvar-lemma-posf-3}.  In this case, we have a conjunction of
$POSF_3^{(LF3\, LF2)}$ and $POSF_3^{(LF1)}$.

\item {\bf \indexfunction{make-min-setvar-lemma-posf-3}}  If there are
banned variables in the constraint, we will make an expansion node.
In this case, we will have one child that corresponds expanding using
the banned variables (so we can mate to the literals in the constraints).
In case we will later want to use this part of the lemma elsewhere in the
proof, we also duplicate \indexflag{NUM-OF-DUPS} times.

The constraint (LF1) does not contain any banned variables and has
no extra literals.  So, we simply let $POSF_3^{(LF1)}$ be a positive
leaf with shallow $[v^1\, 0]$ (the same shallow as LF1).
This leaf is connected to the negative node LF1, solving this constraint.

The constraint (LF3 LF2) contains the banned variable $w$.
So, $POSF_3^{(LF3\, LF2)}$ is an expansion node.  
(We create a fresh variable $w^1$ to use as the bound variable in the shallow formula.)
One child of this expansion node
is $POSF_4^{(LF3\, LF2)}$ with expansion term $w$.  This child is constructed
and is used to solve the constraint by calling
\indexfunction{make-min-setvar-lemma-posf-4}.  If \indexflag{NUM-OF-DUPS} is
greater than $0$, we also have \indexflag{NUM-OF-DUPS} many other children
of $POSF_3^{(LF3\, LF2)}$ expanded using expansion variables.  These children
could be used in the proof of the theorem.

\item {\bf \indexfunction{make-min-setvar-lemma-posf-4}}  This function
creates more expansion nodes corresponding to the rest of the banned variables
of the constraint.  Since (LF3 LF2) only has one banned variable, we
skip directly to constructing an implication node $POSF_4^{(LF3\, LF2)}$
where the first child is a negative ftree $POSF_5^{(LF2)}$ and the second
child is a positive leaf with shallow formula $[v^1\, [S\, w]]$ (the
same as LF3).  This is connected to the node LF3.
The negative ftree $POSF_5^{(LF2)}$ is constructed by
\indexfunction{make-min-setvar-lemma-posf-5}
and used to block LF2.

\item {\bf \indexfunction{make-min-setvar-lemma-posf-5}}  This function constructs
a conjunction corresponding to the extra literals in a given constraint.
In this case, we only have the one extra literal LF2.  So, $POSF_5^{(LF2)}$
is $POSF_6^{LF2}$ constructed by \indexfunction{make-min-setvar-lemma-posf-6}.

\item {\bf \indexfunction{make-min-setvar-lemma-posf-6}}  Given a literal in
a constraint, this creates a corresponding leaf and mates it to the literal
in the constraint.  Since LF2 is positive, $POSF_6^{LF2}$ is a negative leaf
with shallow formula $[v^1\, w]$ and this is mated to LF2.
\end{itemize}

So, to sum up, we created a positive ftree for the lemma along with
connections to the nodes in the constraints.  We can follow the construction
by the noting that the construction of the shallow formulas proceeded
as
$$\exists v^2\,.\, LEM_1(v^2)$$
$$Main(v^1)\,\land\, InvP\,\land\, IndP$$
where $Main(v^1)$ was constructed as
$$Sh(POSF_3^{(LF3\, LF2)})\, \land \, Sh(POSF_3^{(LF1)})$$
$$\forall w^1 Sh(POSF_4^{(LF3\, LF2)})\, \land \, [v^1\, 0]$$
$$\forall w^1 [Sh(POSF_5^{(LF2)})\,\limplies\, [v^1\, [S\, w^1]]]\, \land \, [v^1\, 0]$$
$$\forall w^1 [[v^1\, w^1]\,\limplies\, [v^1\, [S\, w^1]]]\, \land \, [v^1\, 0]$$
(where $Sh(N)$ means the shallow formula of the node $N$).

\item {\bf \indexfunction{make-clos-setvar-lemma-negf}}  This produces
a negative ftree with connections giving the proof of the set existence lemma.
Let us assume \indexflag{INCLUDE-INDUCTION-PRINCIPLE} is set to T.
The function \indexfunction{make-knaster-tarski-leastfp-lemma} constructs
an ftree proof $Knaster-Tarski^-$ of the least fixed point version of the
Knaster-Tarski Theorem for the type of the set variable $v$.
In this case, $v$ has type $\greeko\greeki$, so the Knaster Tarski Theorem
generated is \
\begin{center}
\begin{tabular}{r}
$\forall K_{\greeko\greeki(\greeko\greeki)} \, . \,
\forall u_{\greeko\greeki}\forall v_{\greeko\greeki} 
[\forall z \, . \, u\, z \,\limplies\, v\, z]$ \\
$\limplies\, \exists u_{\greeko\greeki} \, . \, \forall z\, [K\, u\, z\, \limplies \, u\, z]
\,\land \,\forall z\, [u\, z\, \limplies \, K \, u\, z]$ \\
$\forall v_{\greeko\greeki}\, . \, \forall z\, [K\, v\, z\, \limplies \, v\, z]
\,\limplies\, \forall z\, .\, u\, z\, \limplies \, v\, z$ \\
\end{tabular}
\end{center}
The function \indexfunction{make-clos-setvar-lemma-negf-0}
does the work of constructing the negative ftree node $NEGF$ giving
the proof of the set existence lemma.  The special variable
\verb+expf+ is set to a positive ftree $Knaster-Tarski^+$ for the Knaster-Tarski
Lemma which is used to prove the set existence lemma.
Then we return the ftree
$$\econj{}{}{Knaster-Tarski^-}{\eimp{}{}{Knaster-Tarski^+}{NEGF_0^-}}$$
The new connections are added to the dynamic variable \verb+clist+.

\begin{itemize}
\item {\bf \indexfunction{make-clos-setvar-lemma-negf-0}}
Our goal is to construct a negative ftree $NEGF_0^-$ for the set existence lemma
\begin{center}
\begin{tabular}{r}
$ \exists \,v^2 . \forall \,w^1 [ \,v^2 \,w^1 \supset \,v^2 . \,S \,w^1 ] \land \,v^2 \,0$ \\
$\land \, \forall \,x^1 [ \,v^2 \,x^1 \supset \exists \,w [ \,x^1 \,= \,S \,w \land \,v^2 \,w ] \lor \,x^1 \,= \,0 ]$ \\
$\land \forall p\, .\, [\forall \,w^1 [ \,p \,w^1 \supset \,p . \,S \,w^1 ] \land \,p \,0]
\limplies\, \forall x^2\, . \, [v^2\, x^2]\,\limplies\, [p\, x^2].$ \\
\end{tabular}
\end{center}
In general, the set existence lemma universally binds the variables
in \verb+misc-occurs+.  In this case there are no such variables.
We also need to construct the positive $Knaster-Tarski^+$ node (\verb+expf+)
and add connections to \verb+clist+ between nodes in these two ftrees.

We proceed to the most important step, instantiating the $K$ in
the Knaster-Tarski Theorem.  The monotone function we want can
be extracted from the inversion principle.  The inversion principle (\verb+inv-princ+)
is
$$\forall x . \,v^1 \,x \supset \exists \,w [ \,x \,= \,S \,w \land \,v^1 \,w] \lor \,x \,= \,0.$$
and we construct the monotone function (\verb+monfn+):
$$\lambda v^1 \lambda x\, . \exists \,w [ \,x \,= \,S \,w \land \,v^1 \,w] \lor \,x \,= \,0.$$
Let us write $\lambda v^1\lambda x\, . \, P(v^1,x)$ for this term.
Note that $P(v^1,x)$ is positive with respect to $v^1$, so the function will be monotone.
We substitute this for $K$ so that $Knaster-Tarski^+$ is an expansion node
$$\uexpnode{}{\mbox{Knaster-Tarksi}}{\lambda v^1\lambda x\, . \, P(v^1,x)}{\erew{}{\lambda}{}{\eimp{}{}{MONNEGF^-}{EUF^+}}}$$
with one child using the monotone function as the expansion term.
The child of this node is a $\lambda$-rewrite passing to the normal form.
Below this is an implication node with two children $MONNEGF^-$ providing
a proof that $\lambda v^1\lambda x\, . \, P(v^1,x)$ is monotone and $EUF^+$ a positive node with shallow formula (\verb+eu+)
\begin{center}
\begin{tabular}{r}
$\exists u_{\greeko\greeki} \, . \, \forall z\, [P(u,z)\, \limplies \, u\, z]
\,\land \,\forall z\, [u\, z\, \limplies \, P(u,z)]$ \\
$\forall v_{\greeko\greeki}\, . \, \forall z\, [P(v,z)\, \limplies \, v\, z]
\,\limplies\, \forall z\, .\, u\, z\, \limplies \, v\, z.$ \\
\end{tabular}
\end{center}
The node $MONNEGF^-$ is constructed by the function \indexfunction{mon-fn-negf}.
This function implements the proof that set functions defined by positive formulas
are monotone.

Let $u^1$ be a new selected variable to use in the selection node $EUF^+$:
$$\sel{}{EU}{u^1}{\econj{}{}{Kuuf^+}{\econj{}{}{uKuf^+}{lff^+}}}$$
where $EU$ is the right side of the top implication of the Knaster-Tarski Theorem.
The node $Kuuf^+$ is has shallow formula $Kuu$:
$$\forall z\, .\, [\exists \,w [ \,z \,= \,S \,w \land \,u^1 \,w] \lor \,z \,= \,0]
\,\limplies\, u^1\, z$$
and is constructed during the process of constructing $NEGF_0^-$ below.
We start with $Kuuf^+$ as a leaf.
The node $uKuf^+$ is a leaf that corresponds directly to
the inversion principle in the set existence lemma
$$\forall z\, [u^1\, z\, \limplies \, \exists \,w [ \,z \,= \,S \,w \land \,u^1 \,w] \lor \,z \,= \,0.$$
Finally, we will construct a node $lff^+$ has shallow formula \
\begin{center}
\begin{tabular}{r}
$\forall p\, \forall z\, [[\exists \,w [ \,z \,= \,S \,w \land \,u^1 \,w] \lor \,z \,= \,0]
\,\limplies\, u^1\, z]$ \\
$\limplies\, \forall z\, .\, u^1\, z\,\limplies\, p\, z$ \\
\end{tabular}
\end{center}
obtained from the fact that $u^1$ is the least pre-fixed point.

The selected variable $u^1$ is used as the expansion term $NEGF_0^-$
we want to construct for the set existence lemma.  So, $NEGF_0^-$
will have the form
$$\uexpnode{}{\mbox{Set Existence}}{u^1}{\econj{}{}{NEGF_1^-}{\econj{}{}{InvF^-}{IndF^-}}}$$
where $NEGF_1^-$ (\verb+f+)
is constructed by \indexfunction{make-clos-setvar-lemma-negf-1},
$InvF^-$ (\verb+l+) is a leaf corresponding to the inversion principle,
and $IndF^-$ (\verb+indf+) constructed by \indexfunction{make-clos-setvar-ind-negf}
corresponds to the induction principle.
The leaves $InvF^-$ and $uKuf^+$ are mated on \verb+clist+.

\item {\bf \indexfunction{make-clos-setvar-lemma-negf-1}}  This function
constructs $NEGF_1^-$ with shallow formula
$$\forall \,w [ \,u^1 \,w \supset \,u^1 . \,S \,w ] \land \,u^2 \,0.$$
Along with this, we will construct $Kuuf^+$ and put connections onto \verb+clist+.
So, we build $NEGF_1^-$ as a conjunction 
$NEGF_2^{(LF3\, LF2)}\land NEGF_2^{(LF1)}$
where $NEGF_2^{(LF3\, LF2)}$ and $NEGF_2^{(LF1)}$
corresponding to the two constraints is constructed by
\indexfunction{make-clos-setvar-lemma-negf-2}.
We also use an integer \verb+n+ to keep up with which constraint we
are considering.

\item {\bf \indexfunction{make-clos-setvar-lemma-negf-2}}
Given a constraint $C$ (\verb+paths+), this function builds
a negative ftree $NEGF_2^C$ and changes $Kuuf^+$ and \verb+clist+.

First, consider the constraint (LF1).  The shallow formula of
$NEGF_2^{(LF1)}$ should be $u^1\, 0$.  Here, we let
\indexfunction{make-clos-setvar-lemma-negf-3} do the work
of constructing an ftree which will be merged with $Kuuf^+$.
Then \indexfunction{make-clos-setvar-lemma-negf-4} creates
the negative leaf $NEGF_2^{(LF1)}$ mated with part of
the ftree generated in \indexfunction{make-clos-setvar-lemma-negf-4}.

The second constraint we need $NEGF_2^{(LF3\, LF2)}$ to have shallow
formula
$$\forall w \, . \, u^1\, w\,\limplies\, u^1\, . \, S\, w.$$
First, we create a new selected variable $w^2$ and let $NEGF_2^{(LF3\, LF2)}$
be 
$$\sel{}{\forall w \, . \, u^1\, w\,\limplies\, u^1\, . \, S\, w}{w^2}{\eimp{}{}{NEGF_3^{(LF2)}}{LEAF3}}$$
where $NEGF_3^{(LF2)}$ is constructed by \indexfunction{make-clos-setvar-lemma-negf-3}
(along with an ftree to be merged with $Kuuf^+$)
and $LEAF3$ is a negative leaf with shallow $u^1\, . \, S\, w^2$
(mated to part of the new part of $Kuuf^+$).

\item {\bf \indexfunction{make-clos-setvar-lemma-negf-3}}  Suppose we are
given a constraint $C$ ($\Gamma\rightarrow A$) and the arguments (\verb+args+) of the main literal of the constraint.
This function constructs three positive ftrees.  The first is an ftree $Kuuf_3^C$
which will be merged with $Kuuf^+$.  The second is $NEGF_3^\Gamma$ where $L$
is the list of extra literals $L$ of the constraint, or NIL if there are no extra
literals.  If $NEGF_3^\Gamma$ is not nil, it is
a positive ftree corresponding to a conjunct of the extra literals $L$ of the constraint.
The third is a leaf inside the first ftree which will be used in the mating.

Recall that $Kuu$ is
$$\forall z\, .\, [\exists \,w [ \,z \,= \,S \,w \land \,u^1 \,w] \lor \,z \,= \,0]
\,\limplies\, u^1\, z,$$
and this will be the shallow formula of $Kuuf_3^C$.
The third ftree returned will be the leaf corresponding to $u^1\, z$
beneath $Kuuf_3^C$.

In the first constraint (LF1), the args are $(0)$.  We use the list of
arguments as expansion terms to create $Kuuf_3^{(LF1)}$.  Here, $Kuuf_3^{(LF1)}$ will be
$$\uexpnode{}{Kuu}{0}{\eimp{}{}{Kuuf_5^{(LF1)}}{LEAF:u^1\, 0^+}}$$
where $LEAF:u^1\, 0^+$ will be the third ftree returned.
$Kuuf_5^{(LF1)}$ is a negative ftree constructed by
\indexfunction{make-clos-setvar-lemma-negf-5}.
In this case there are no extra literals, so the second return value is NIL.

In the second constraint (LF3 LF2), the one argument is $S\, w^2$.
In this case, $Kuuf_3^{(LF3\, LF2)}$ is 
$$\uexpnode{}{Kuu}{S\, w^2}{\eimp{}{}{Kuuf_5^{(LF3\, LF2)}}{u^1\, . \,S\, w^2}}$$
where $Kuuf_5^{(LF3\, LF2)}$ is constructed
along with $NEGF_3^{(LF2)}$
by \indexfunction{make-clos-setvar-lemma-negf-5}.

\item {\bf \indexfunction{make-clos-setvar-lemma-negf-4}}  This simply makes
a negative leaf and mates it to a given positive leaf.

\item {\bf \indexfunction{make-clos-setvar-lemma-negf-5}}  Given a constraint
$C$ ($\Gamma\rightarrow A$),
this function returns two values.  The first is a negative ftree 
$Kuuf_5^C$, and the second is a positive ftree $NEGF_8^\Gamma$ (or NIL if $\Gamma$ is empty)
constructed later.

For the constraint (LF1) $Kuu_5^{(LF1)}$ should have shallow formula
$$[\exists \,w [ \,z \,= \,S \,w \land \,u^1 \,w] \lor \,0 \,= \,0].$$
So, is $Kuu_5^{(LF1)}$ of the form
$$\edisj{}{}{UNUSEDLEAF}{Kuu_6^{(LF1)}}$$
where $Kuu_6^{(LF1)}$ is constructed by \indexfunction{make-clos-setvar-lemma-negf-6}.

For the constraint (LF3 LF2) $Kuu_5^{(LF3\, LF2)}$ should have shallow formula
$$[\exists \,w [ S\, w^2 \,= \,S \,w \land \,u^1 \,w] \lor \,[S\,w^2] \,= \,0].$$
In this case, $Kuu_5^{(LF3\, LF2)}$ is of the form
$$\edisj{}{}{Kuu_6^{(LF3\, LF2)}}{UNUSEDLEAF}$$
where $Kuu_6^{(LF3\, LF2)}$ is constructed by \indexfunction{make-clos-setvar-lemma-negf-6}
(along with $NEGF_8^{(LF2)}$).

\item {\bf \indexfunction{make-clos-setvar-lemma-negf-6}}  This function
constructs $Kuu_6^C$ using the selected variables created in
\indexfunction{make-clos-setvar-lemma-negf-2} as expansion terms,
then calls \indexfunction{make-clos-setvar-lemma-negf-7} to construct
the rest of $Kuu_6^C$ and $NEGF_8^C$.  At this step we distinguish
between arguments (\verb+args2+) that correspond to these selected variables,
and remove one of the arguments corresponding to each such selected variable.
The arguments sent to \indexfunction{make-clos-setvar-lemma-negf-7} correspond
to equations in the formula \verb+wff+ (originating with the inversion principle).

Since the constraint (LF1) has not banned variables, $Kuu_6^{(LF1)}$ 
is constructed as $Kuu_7^{(LF1)}$ 
by \indexfunction{make-clos-setvar-lemma-negf-7}.

The constraint (LF3 LF2) has the single banned variable $w$.
The corresponding selected variable $w^2$ is used to create $Kuu_6^{(LF3\, LF2)}$ as
$$\uexpnode{}{[\exists \,w [ S\, w^2 \,= \,S \,w \land \,u^1 \,w] \lor \,[S\,w^2] \,= \,0]}{w^2}{Kuu_7^{(LF3\, LF2)}}$$
where $Kuu_7^{(LF3\, LF2)}$ by \indexfunction{make-clos-setvar-lemma-negf-7}
(along with $NEGF_8^{(LF2)}$).

\item {\bf \indexfunction{make-clos-setvar-lemma-negf-7}}
The arguments that are sent to this function are those from the main
literal which were not the first argument corresponding to the selected
variables generated in \indexfunction{make-clos-setvar-lemma-negf-2}.
These arguments correspond to a conjunction of reflexive equations.  
If there is more than one literal in the constraint,more
conjuncts are formed when \indexfunction{make-clos-setvar-lemma-negf-8}
is called.

The first constraint (LF1) has no extra literals, so when this function is
called $Kuu_7^{(LF1)}$ with shallow (\verb+wff+) $0=0$
is simply constructed as
$$\erew{}{REFL=}{0 = 0}{\truenode{}}$$

The second constraint (LF3 LF2) does have the extra literal LF2.
When this is called $Kuu_7^{(LF3\, LF2)}$ which shallow (\verb+wff+)
$$[[S\, w^2]\, = \, [S\, w^2]]\,\land\, [u^1\, w^2]$$
is constructed as
$$\econj{}{}{\erew{}{REFL=}{}{\truenode{}}}{Kuu_8^{(LF2)}}$$
where $Kuu_8^{(LF2)}$ is constructed by
\indexfunction{make-clos-setvar-lemma-negf-8}.

\item {\bf \indexfunction{make-clos-setvar-lemma-negf-8}}
Given the left side $\Gamma$ of a constraint, a negative
ftree $Kuu_8^\Gamma$ and positive ftree $NEGF_8^\Gamma$ 
are constructed along with connections between them.
These are constructed as conjuncts for each literal in $\Gamma$
using \indexfunction{make-clos-setvar-lemma-negf-9} to
construct the children of the conjuncts.

In our example, $Kuu_8^{(LF2)}$ is simply $Kuu_9^{LF2}$ since there
is only one literal.  Similarly, $NEGF_8^{(LF2)}$ is $NEGF_9^{LF2}$.

\item {\bf \indexfunction{make-clos-setvar-lemma-negf-9}}
Usually this function is given two wffs which are $\alpha$-equal.
A positive leaf and a negative leaf are created, mated, and returned.
Another case is when set variable occurs embedded in the literal of the constraint.
In this case, the first wff \verb+wff1+ is of the form
$$\exists w^i_{\greeko\greeki}\, [[\forall \overline{x}\, .\, [w^i\, \overline{x}]\,\limplies\, [u^1_{\greeko\greeki}\, \overline{x}]]\,\land\, A(w^i)].$$
where the second wff \verb+wff2+ is $A(u^1)$.  In this case, we create
a positive ftree of the form
$$\uexpnode{}{\exists w^i_{\greeko\greeki}\, [[\forall \overline{x}\, .\, [w^i\, \overline{x}]\,\limplies\, [u^1_{\greeko\greeki}\, \overline{x}]]\,\land\, A(w^i)]}{u^1}{\econj{}{}
{\sel{}{\forall \overline{x}\, .\, [w^i\, \overline{x}]\,\limplies\, [u^1_{\greeko\greeki}\, \overline{x}]}{\overline{a}}{\eimp{}{}{LEAF1:[u^1\, \overline{a}]}{LEAF2:[u^1\, \overline{a}]}}}
{LEAF3:A(u^1)}}$$
and a negative leaf with shallow formula $A(u^1)$ which we mate to $LEAF3$.
Also, $LEAF1$ and $LEAF2$ are mated.
These two ftrees are returned.

In our example, the literal is LF2.  The formulas are both $u^1\, w^2$,
so we create $Kuu_9^{LF2}$ as a positive leaf and $NEGF_9^{LF2}$
as a negative leaf.  The two are mated and returned.

\item {\bf \indexfunction{make-clos-setvar-ind-negf}}  This function
is called with the least pre-fixed point property formula (\verb+lfwff+),
the induction property formula (\verb+indwff+), and the list of constraints (\verb+paths+).
The goal is to prove the induction property from the pre-fixed point
property by constructing a negative ftree $IndF^-$
for the induction property, a positive ftree $lff^+$
for the least pre-fixed point property,
and a complete set of connections between nodes in them.
The dynamic variable \verb+lff+ is set to $lff^+$.
$IndF^-$ is returned.

In our example, the least pre-fixed property is
$$\forall v_{\greeko\greeki}\, . \, \forall z\, [[\exists \,w [ \,x \,= \,S \,w \land \,v \,w] \lor \,x \,= \,0]
\, \limplies \, v\, z]
\,\limplies\, \forall z\, .\, u^1\, z\, \limplies \, v\, z$$
and the induction property is
$$\forall p\, .\, [\forall \,w^1 [ \,p \,w^1 \supset \,p . \,S \,w^1 ] \land \,p \,0]
\limplies\, \forall x^2\, . \, [u^1\, x^2]\,\limplies\, [p\, x^2].$$
The first thing we do is choose a new selected variable $p^1_{\greeko\greeki}$
(\verb+p+).
Let $P_1$ (\verb+lfpre+) be
$$\forall z\, [[\exists \,w [ \,z \,= \,S \,w \land \,p^1 \,w] \lor \,z \,= \,0]
\, \limplies \, p^1\, z],$$
$P_2$ be
$$\forall z\, .\, u^1\, z\, \limplies \, p^1\, z,$$
$I_1$ (\verb+indhyp+) be
$$\forall \,w^1 [ \,p^1 \,w^1 \supset \,p^1 . \,S \,w^1 ] \land \,p^1 \,0,$$
and $I_2$ be
$$\forall x^2\, . \, [u^1\, x^2]\,\limplies\, [p^1\, x^2].$$
We construct the positive ftree $lff^+$ as
$$\uexpnode{}{\mbox{Least Pre-fixed Point}}{p^1}
{\eimp{}{}{lff_1:P_1^-}{LEAF1:P_2^+}}$$
and the negative ftree $IndF^-$ as
$$\sel{}{\mbox{Induction Principle}}{p^1}
{\eimp{}{}{IndF_1:I_1^+}{LEAF2:I_2^-}}$$
where $lff_1$ and $IndF_1$ are constructed by \indexfunction{make-clos-setvar-ind-negf}.
Since $P_2$ and $I_2$ are $\alpha$-equal, we can mate $LEAF1$ and $LEAF2$.

\item {\bf \indexfunction{make-clos-setvar-ind-negf-1}}
We start with two wffs: \verb+wff1+ of the form
$C_1\,\land \cdots \land C_n$
\verb+wff2+ of the form
$$\forall \overline{z} \, . \, [D_1(\overline{z})\,\lor\,\cdots\,\lor\, D_n(\overline{z})]\,\limplies\, [p\, \overline{z}].$$
where $n$ is the number of constraints.
Let $\overline{a}$ be new selected variables and $LEAF1$ be a negative literal
with shallow formula $[p\, \overline{a}]$.  We return two values, 
a negative ftree of the form
$$\sel{}{\forall \overline{z} \, . \, [D_1(\overline{z})\,\lor\,\cdots\,\lor\, D_n(\overline{z})]\,\limplies\, [p\, \overline{z}]}
{\overline{a}}{NF_2}$$
and a positive ftree $PF_2$ with shallow formula \verb+wff1+
where $PF_2$ and $NF_2$ are constructed by
\indexfunction{make-clos-setvar-ind-negf-2}.

In our example, \verb+wff1+ is
$$\forall \,w^1 [ \,p^1 \,w^1 \supset \,p^1 . \,S \,w^1 ] \land \,p^1 \,0,$$
and \verb+wff2+ is
$$\forall z\, [[\exists \,w [ \,z \,= \,S \,w \land \,p^1 \,w] \lor \,z \,= \,0]
\, \limplies \, p^1\, z].$$
We choose a new selected variable $z^1_\greeki$ and create a negative leaf
$LEAF1:p^1\, z^1$.
We call \indexfunction{make-clos-setvar-ind-negf-2} with
$$\forall \,w^1 [ \,p^1 \,w^1 \supset \,p^1 . \,S \,w^1 ] \land \,p^1 \,0,$$
$$\exists \,w [ \,z^1 \,= \,S \,w \land \,p^1 \,w] \lor \,z^1 \,= \,0,$$
and $LEAF1$ to get two positive ftrees, $PF_3^1$ and $PF_3^2$.
We return the positive ftree
$PF_3^1$ and the negative ftree
$$\eimp{}{}{PF_3^2}{LEAF1}$$

\item {\bf \indexfunction{make-clos-setvar-ind-negf-2}}
We start with $n$ constraints (given by \verb+paths+ and \verb+banned-occurs+),
and two wffs, one (\verb+wff1+) of the form
$C_1\,\land \cdots \land C_n$,
and the other (\verb+wff2+) of the form
$$D_1(\overline{z})\,\lor\,\cdots\,\lor\, D_n(\overline{z}).$$
We consider each constraint
$$\Psi_1|\Gamma_i\rightarrow A_i,$$
conjunct $C_i$ and disjunct $D_i$.
We distinguish between variables $z\in\Psi_i$ which
occur as arguments in $A_i$ and those $w\in\Psi_i$ which do not.
When $z\in\Psi_1$ occurs as an argument in $A_i$,
the corresponding argument in the shallow formula of $LEAF1$
is a variable $a$ created in the previous step.
Let $\theta_1$ (\verb+psi-z-assoc+) be the substitution sending $\theta(z)=a$
for each such $z$ and $a$.
Those $w\in\Psi_i$ which do not occur as
arguments are stored on $\Psi^0_i$.  This information is passed
to \indexfunction{make-clos-setvar-ind-negf-3}
which returns two positive ftrees $PF_{D_i}$ and
$PF_{C_i}$ and returns
$$\econj{}{}{PF_{C_1}}{\econj{}{}{PF_{C_2}}{\etra{\cdots}{PF_{C_n}}}}$$
and
$$\edisj{}{}{PF_{D_1}}{\edisj{}{}{PF_{D_2}}{\etra{\cdots}{PF_{D_n}}}}$$

In our example, the two constraints are of the form
$$v^1\, 0$$
and
$$w|v^1\, w\rightarrow v^1\, [S\, w].$$
Neither $0$ nor $[S\, w]$ is a variable in the corresponding $\Psi$'s,
so $\theta_1$ is empty.  $\Psi^0$ is NIL in one case
and ($w$) in the other.

\item {\bf \indexfunction{make-clos-setvar-ind-negf-3}}
Given two wff's $C$ (\verb+wff1+) and $D$ (\verb+wff2+) and $\Psi^0 = \overline{w}$
where \verb+wff2+ is of the form
$\exists \overline{w^1} D'(\overline{w^1})$, we create new selected variables
$\overline{w^2}$.  Let $\theta_2$ (\verb+psi-w-assoc+) send each $w$ to $w^2$.
We call \indexfunction{make-clos-setvar-ind-negf-4}
with $C$ and $D'(\overline{w^2})$ to obtain
two positive ftrees $PF_4^1$ and $PF_4^2$ with shallow formulas
$C$ and $D'(\overline{w^2})$.  We then return $PF_4^1$ and 
$$\sel{}{\exists \overline{w^1} D'(\overline{w^1})}{\overline{w^2}}{PF_4^2}$$

The constraint $\emptyset |\cdot\rightarrow v^1\, 0$ has an empty $\Psi$,
so we proceed directly to \indexfunction{make-clos-setvar-ind-negf-4}
with
$\,p^1 \,0$
and
$\,z^1 \,= \,0$ to get the two positive ftrees.

The constraint $w |v^1\, w\rightarrow v^1\, . \, S\, w$ has the variable $w$
in $\Psi$, and $D$ is
$$\exists w [ \,z^1 \,= \,S \,w \land \,p^1 \,w].$$
We create a new selected variable $w^2$ and call 
\indexfunction{make-clos-setvar-ind-negf-4}
with
$$\forall \,w^1 [ \,p^1 \,w^1 \supset \,p^1 . \,S \,w^1 ]$$
and 
$$z^1 \,= \,S \,w^2 \land \,p^1 \,w^2.$$
The second positive ftree $PF_4^2$ constructed has
shallow $z^1 \,= \,S \,w^2 \land \,p^1 \,w^2$.  We return the
first positive ftree along with
$$\sel{}{\exists w [ \,z^1 \,= \,S \,w \land \,p^1 \,w]}{w^2}{PF_4^2}$$

\item {\bf \indexfunction{make-clos-setvar-ind-negf-4}}
We are given a constraint $\Psi|\Gamma\rightarrow A$ and two wff's $C$ and $D$
where $C$ is of the form
$\forall \overline{y} C' \limplies A'$
(or $\forall \overline{y} A'$ if $\Gamma$ is empty)
where $\Psi=\overline{y'}$.  Define a substitution $\theta$ with $dom(\theta) = \overline{y}$
as follows.
Either $y'\in\overline{y'}$ is a $z\in\Psi$ which occurs as an argument in $A$,
and so is in the domain of $\theta_1$ (\verb+psi-z-assoc+), or
$y$ is a $w\in\Psi$ in the domain of $\theta_2$ (\verb+psi-w-assoc+).
In the first case, let $\theta(y) = \theta_1(y')$.  In the second case,
let $\theta(y) = \theta_2(y')$.
We use these association lists to determine
the expansion term for $y$.  
Then \indexfunction{make-clos-setvar-ind-negf-5}
is called with $\theta(C')$ (or NIL) (\verb+hyp+) and $\theta(A')$ (\verb+conc+).
This constructs two positive ftrees, 
$PF_5^1$ with shallow $\theta(C'\limplies A')$
and a positive ftree $PF_5^2$ with shallow $D$.
We return
$$\uexpnode{}{C}{\theta(\overline(y))}{PF_5^1}$$
and $PF_5^2$.

Again, the constraint $\emptyset |\cdot\rightarrow v^1\, 0$ has an empty $\Psi$,
so we proceed directly to \indexfunction{make-clos-setvar-ind-negf-5}
with
$p^1 \,0$
and
$z^1 \,= \,0$ to get the two positive ftrees.

For the constraint $w |v^1\, w\rightarrow v^1\, . \, S\, w$,
$C$ is
$$\forall \,w^1 [ \,p^1 \,w^1 \supset \,p^1 . \,S \,w^1 ].$$
$\theta(w^1) = \theta(w) = w^2$ by definition.
\indexfunction{make-clos-setvar-ind-negf-5} then constructs
two positive ftrees, $PF_5^1$ with shallow formula
$[ \,p^1 \,w^2 \supset \,p^1 . \,S \,w^2 ]$
and $PF_5^2$ with shallow formula 
$z^1 \,= \,S \,w^2 \land \,p^1 \,w^2$.
We return
$$\uexpnode{}{\forall \,w^1 [ \,p^1 \,w^1 \supset \,p^1 . \,S \,w^1 ]}{w^2}{PF_5^1}$$
and $PF_5^2$.

\item {\bf \indexfunction{make-clos-setvar-ind-negf-5}}
This function starts with the negative leaf $LEAF1$ (\verb+leaf+)
constructed in \indexfunction{make-clos-setvar-ind-negf-1}.
This function compares the arguments in the shallow formula
$p(b_1,\ldots,b_n)$ of $LEAF1$ \verb+leaf+
and the given formula $p(a_1,\ldots,a_n)$ (\verb+conc+).  

We recursively compare the arguments.  Suppose we have a negative
leaf \verb+leaf+ with shallow formula
$p(a_1,\ldots,a_{i-1},b_i,\ldots,b_n)$
If $b_i$ and $a_i$ are syntactically equal,
then these correspond to some $z\in\Psi$.  Otherwise,
there should be an equation at the beginning of \verb+wff2+.
That is, either \verb+wff2+ is $[b_i\, =\,a_i]$ or $[b_i\,=\,a_i]\,\land \,D'.$
The function \indexfunction{make-ftree-subst} creates a positive ftree
with shallow $[b_i\, = \, a_i]$, a negative leaf $LEAF2$ with
shallow formula $p(a_1,\ldots,a_i,b_{i+1},\ldots,b_n)$
(and some new connections to add to the connection list).
Let $EQ_{i_j}$ be these positive ftrees for these equations.

So, after $n$ steps, we have a negative leaf (\verb+leaf+) with shallow formula
$p(a_1,\ldots,a_n)$.  We create a positive leaf $LEAF2$ (\verb+leaf2+) with this shallow
and mate the two leaves.  If $\Gamma$ is empty, we return 
this positive ftree along with the positive ftree
$$\econj{}{}{EQ_{i_1}}{\etra{\cdots}{\econj{}{}{EQ_{i_{m-1}}}{EQ_{i_m}}}}$$
If $\Gamma$ is nonempty, then the rest of $D$ (after removing the equations)
should be $\alpha$-equal to $C$ (\verb+hyp+).  We
create a negative leaf $LEAF3$ and a positive leaf $LEAF4$ with these shallows
and return the positive ftrees
$$\eimp{}{}{LEAF4}{LEAF2}$$
and the positive ftree
$$\econj{}{}{EQ_{i_1}}{\etra{\cdots}{\econj{}{}{EQ_{i_m}}{LEAF3}}}$$

In the constraint (LF1), $\Gamma$ is empty and $D$ is
$z^1 \,= \,0$.  The shallow formula of $LEAF1$ is $p^1 z^1$.
We construct $EQ_1$ as
$$\erew{}{z^1\,=\, 0}{Leibniz=}
{\erew{}{[\lambda x\,\forall y\, .\forall q\,.\,q\,x\,\limplies\,q\,y]\, z^1\, 0}{\lambda}
{\uexpnode{}{\forall q\,.\,q\,z^1\,\limplies\,q\,0}{\lambda x\,.\,\lnot\,[p^1\,x]}
{\erew{}{[\lambda x\,.\,\lnot\,[p^1\,x]]\,z^1\,\limplies\,[\lambda x\,.\,\lnot\,[p^1\,x]]\,0}
{\lambda}
{\eimp{}{}{\eneg{}{}{LEAF3^{+}}}
{\eneg{}{}{{LEAF1'}^{-}}}
}}}}$$
where $LEAF3^+$ mates to $LEAF1^-$.  
We also create a positive leaf $LEAF2$ with shallow formula $p^1\, 0$
and mate this to ${LEAF1'}^-$.  We return $LEAF2$ and $EQ_1$.

In the constraint (LF3 LF2), $D$ is 
$z^1 \,= \,S \,w^2 \land \,p^1 \,w^2$.
As above, a positive ftree $EQ_2$ for the equation $z^1\, = \, S\, w^2$
giving a mate for $LEAF1^-$ and a negative leaf $LEAF3^-$ with shallow
formula $p^1\, .\, S\, w^2$.  We also create a positive leaf $LEAF2^+$
to mate with $LEAF3$.  The ``leftover'' part of $D$, $p^1\, w^2$ corresponds
to $\Gamma$.  We create positive and negative leaves $LEAF4^+$ and $LEAF5^-$
with this shallow formula and mate the two.  We return the two positive ftrees
$$\eimp{}{}{LEAF5^-}{LEAF2^+}$$
and
$$\econj{}{}{EQ_2}{LEAF4^+}.$$

\end{itemize}

\item {\bf \indexfunction{make-knaster-tarski-leastfp-lemma}}

\end{enumerate}
\chapter{Tactics and Tacticals}

% I have no idea who wanted to modify this file, or why. - cebrown 2/3/00
{\bf Modify tactics.tex}  

\section{Overview}

Ordinarily in \tps, the user proceeds by performing a series of
atomic actions, each one specified directly.  For example, in constructing
a proof, she may first apply the deduct rule, then the rule of cases, then
the deduct rule again, etc..  These actions are related temporally, but
not necessarily in any other way; the goal which is attacked by 
one action may result in several new goals, yet there is no distinction
between goals produced by one action and those produced by another.  
In addition, this use of small steps prohibits the user from outlining
a general procedure
to be followed.  A complex strategy cannot be expressed in these terms, 
and thus
the user must resign herself to proceeding by plodding along, using simple
(often trivial and tedious) applications of rules.

Tactics offer a way to encode strategies into new commands, using a 
goal-oriented approach.  With the use of tacticals, more complex tactics
(and hence strategies) may be built.  Tactics and tacticals are, in essence,
a programming language in which one may specify techniques for solving
goals.

Tactics are called  partial subgoaling methods by
\cite{Gordon79}.  What this means is that a tactic is a
function which, given a goal to be accomplished,
will return a list of new goals, along with a procedure by which the
original goal can be achieved given that the new goals are first
achieved.  Tactics also may fail, that is, they may not be applicable
to the goal with which they are invoked.

Tacticals operate upon tactics in much the same way that functionals
operate upon functions.  By the use of tacticals, one may create
a tactic that repeatedly carries out a single tactic, or composes
two or more tactics.  This allows one to combine many small tactics
into a large tactic which represents a general strategy for solving goals.

As implemented in \tps, a tactic is a function which takes a goal
as an argument and returns four values: a list of new goals, a message
which tells what the tactic did (or didn't do), a token indicating what
the tactic did, and a validation, which is a lambda expression which takes
as many arguments as the number of new goals, and which, given solutions
for the new goals, combines the solutions into a solution for the original
goal. It is possible that the validation is used nowhere in the code, and
that it should be phased out.

Consider this example.  Suppose we are trying to define tactics which
will convert an arithmetic expression in infix form to one in
prefix form and evaluate it.  One tactic might,
if given a goal of the form "A / B", where A and B are themselves 
arithmetic expressions in infix form, return the list ("A" "B"),
some message, the token "succeed", and the 
validation {\tt (lambda (x y) (/ x y))}.  If now we solve the new goals "A"
and "B" (i.e., find their prefix forms and evaluate them),
and apply the validation as a function to their solutions, we get
a solution to the original goal "A / B".

When we use a tactic, we must know for what purpose the tactic is being
invoked.  We call this purpose the {\it use} of the tactic.  Some examples
of uses are {\tt nat-ded} for carrying out natural deduction proofs, {\tt nat-etree}
for translating natural deduction proofs to expansion proofs (not yet implemented), and {\tt etree-nat}
for translating expansion proofs to natural deductions.  A single tactic may
have definitions for each of these uses.  In contrast to tactics, tacticals
are defined independent of any specific tactic use; some of the auxiliary
functions they use, however, such as copying the current goal, may depend
upon the current tactic use.  For this purpose, the current tactic use
is determined by the flag \indexflag{tacuse}.  Resetting this flag resets
the default tactic use. Though a tactic can be called
with only a single use, that tactic can call other tactics with different
uses.  See the examples in the section "Using Tactics".

Another important parameter used by a tactic is the {\it mode}.  There are
two tactic modes, {\tt auto} and {\tt interactive}.  The definition of
a tactic may make a distinction between these two modes; the current
mode is determined by the flag  \index{tacmode}, and resetting this
flag resets the default tactic mode.  Ideally, a tactic
operating in {\tt auto} mode should require no input from the user, while
a tactic in {\tt interactive} mode may request that the user make some decisions,
e.g., that the tactic actually be carried out.  It may be desirable, however,
that some tactics ignore the mode, compound tactics (those tactics created
by the use of tacticals and other tactics) among them.

One may wish to have tactics print informative messages as they operate;
the flag \indexflag{tactic-verbose} can be set to T to allow this to
occur, and tactics can be defined so that messages are printed when
{\tt tactic-verbose} is so set. Each tactic should call the function {\it tactic-output} 
with two arguments. The first argument should be a string containing the information to be
printed, and the second argument T if the tactic succeeds, and NIL
otherwise.  {\it tactic-output} will, depending on the second argument and
the current value of {\tt tactic-verbose}, either print or not print the
first argument.

\section{Syntax for Tactics and Tacticals}

The \TPS category for tactics is called {\tt tactic}.  The defining 
function for tactics is {\tt deftactic}. The variable {\tt auto::*global-tacticlist*}
contains a list of all tactics. Each tactic definition has
the following form:

%\begin{programexample}
\begin{verbatim}
(deftactic tactic
  {(<tactic-use> <tactic-defn> [<help-string>])}+
)
\end{verbatim}
%\end{programexample}

with components defined below:

% \begin{Format}
% @tabclear
% @tabset(1.5 inches)
\begin{tabular}{ll}
\indexSyntax{tactic-use} ::= & {\tt nat-ded | nat-etree | etree-nat} \\
\indexSyntax{tactic-mode} ::= & {\tt auto | interactive} \\
\indexSyntax{tactic-defn} ::= & {\it primitive-tactic} | {\it compound-tactic} \\
\indexSyntax{primitive-tactic} ::= & {\tt (lambda (goal) {{\it form}}*)} \\
 & This lambda expression should return four values of the form:  \\
 & {\it goal-list msg token validation}. \\
\indexSyntax{compound-tactic}::= & ({\it tactical} {{\it tactic-exp}}*) \\
\indexSyntax{tactic-exp}::= & {\tt tactic}  a tactic which is already defined \\
 & | ({\it tactic} {\tt [:use {\it tactic-use}] [:mode {\it tactic-mode}] [:goal {\it goal}]}) \\
 & | {\it compound-tactic} \\
 & | ({\tt call} {\it command}) ; where {\it command} is a command which could be \\
 & given at the \TPS top level \\
\indexSyntax{goal} ::= & a goal, which depends on the tactic's use,  \\
 & e.g., a planned line when the tactic-use is {\tt nat-ded}. \\
\indexSyntax{goal-list} ::= & ({{\it goal}}*) \\
\indexSyntax{msg} ::= & {\it string} \\
\indexSyntax{token} ::= & {\tt complete}  meaning that all goals have been exhausted \\
 & {\tt | succeed}  meaning that the tactic has succeeded \\
 & {\tt | nil}  meaning that the tactic was called only for side effect \\
 & {\tt | fail}  meaning that the tactic was not applicable  \\
 & {\tt | abort}  meaning that something has gone wrong, such as an undefined  \\
 & tactic \\
\end{tabular}
% \end{format}

Tacticals are kept in the \TPS category {\tt tactical}, with defining
function {\tt deftactical}.  Their definition has the following form:

%\begin{programexample}
\begin{verbatim}
(deftactical tactical
  (defn <tacl-defn>)
  (mhelp <string>))
\end{verbatim}
%\end{programexample}

with 

% \begin{format}@tabclear
% @tabset(1.5 inches)
\begin{tabular}{ll}
\indexSyntax{tacl-defn} ::= & {\it primitive-tacl-defn | compound-tacl-defn} \\
\indexSyntax{primitive-tacl-defn} ::= & {\tt (lambda (goal tac-list) {{\it form}}*)} \\
 & This lambda-expression, where {\tt tac-list} stands for a possibly \\
 & empty list of tactic-exp's, should be independent of the tactic's  \\
 & use and current mode.  It should return values like those returned  \\
 & by a {\it primitive-tac-defn}. \\
\indexSyntax{compound-tacl-defn} ::= & {\tt (tac-lambda ({{\it symbol}}*) {\it tactic-exp})}  \\
 & Here the tactic-exp should use the symbols in the \\
 & tac-lambda-list as dummy variables. \\
\end{tabular}
%\end{format}

\pagebreak
Here is an example of a definition of a primitive tactic.
%\begin{programexample}
\begin{verbatim}
(deftactic finished-p
 (nat-ded 
  (lambda (goal)
    (if (proof-plans dproof)
	(progn
	 (when tactic-verbose (msgf "Proof not complete." t))
	 (values nil "Proof not complete." 'fail))
	(progn
	 (when tactic-verbose (msgf "Proof complete." t))
	 (values nil "Proof complete." 'succeed))))
  "Returns success if all goals have been met, otherwise
returns failure."))
\end{verbatim}
%\end{programexample}

This tactic is defined for just one use, namely {\tt nat-ded}, or natural
deduction.  It merely checks to see whether there are any planned lines
in the current proof, returning failure if any remain, otherwise
returning success.  This tactic is used only as a predicate, so the
goal-list it returns is nil, as is the validation.

As an example of a compound tactic, we have
%\begin{programexample}
\begin{verbatim}
(deftactic make-nice
  (nat-ded
   (sequence (call cleanup) (call squeeze) (call pall))
   "Calls commands to clean up the proof, squeeze the line 
numbers, and then print the result."))
\end{verbatim}
%\end{programexample}

Again, this tactic is defined only for the use {\tt nat-ded}.  {\tt sequence} is
a tactical which calls the tactic expressions given it as arguments
in succession.

Here is an example of a primitive tactical.
%\begin{programexample}
\begin{verbatim}
(deftactical idtac
  (defn
    (lambda (goal tac-list)
      (values (if goal (list goal)) "IDTAC" 'succeed 
	      '(lambda (x) x))))
  (mhelp "Tactical which always succeeds, returns its goal 
unchanged."))
\end{verbatim}
%\end{programexample}

The following is an example of a compound tactical.  {\tt then} and {\tt orelse} are tacticals.

%\begin{programexample}
\begin{verbatim}
(deftactical then*
  (defn 
    (tac-lambda (tac1 tac2)
      (then tac1 (then (orelse tac2 (idtac)) (idtac)))))
  (mhelp "(THEN* tactic1 tactic2) will first apply tactic1; if it
fails then failure is returned, otherwise tactic2 is applied to 
each resulting goal.  If tactic2 fails on any of these goals, 
then the new goals obtained as a result of applying tactic1 are 
returned, otherwise the new goals obtained as the result of 
applying both tactic1 and tactic2 are returned."))
\end{verbatim}
%\end{programexample}

\section{Tacticals}
There are several tacticals available.  Many of them are taken directly from
\cite{Gordon79}.  After the name of each tactical is given
an example of how it is used, followed by a description of the behavior
of the tactical
when called with {\tt goal} as its goal.  The newgoals and validation returned
are described only when the tactical succeeds.


\begin{enumerate}
\item {\tt idtac: (idtac)} \\
Returns {\tt (goal), (lambda (x) x)}.

\item {\tt failtac: (failtac)}\\
Returns failure

\item {\tt call: (call command)}\\
Executes command as if it were entered at top level of \tps.  This is used
only for side-effects.  Returns {\tt (goal), (lambda (x) x)}.

\item {\tt orelse: (orelse tactic1 tactic2 ... tacticN)}\\
If N=0 return failure, else apply {\tt tactic1} to {\tt goal}. If this fails, call {\tt (orelse tactic2 tactic3 ... tacticN)} on {\tt goal}, else return the result of applying {\tt tactic1} to {\tt goal}.\\

\item {\tt then: (then tactic1 tactic2)}\\
Apply {\tt tactic1} to {\tt goal}. If this fails, return failure, else apply {\tt tactic2} to each of the subgoals generated by {\tt tactic1}.\\
If this fails on any subgoal, return failure, else return the list of new subgoals returned from the calls to {\tt tactic2}, and the lambda-expression representing the combination of applying {\tt tactic1} followed by {\tt tactic2}.\\
Note that if {\tt tactic1} returns no subgoals, {\tt tactic2} will not be called.

\item {\tt repeat: (repeat tactic)}\\
Behaves like {\tt (orelse (then tactic (repeat tactic)) (idtac))}.

\item {\tt then*: (then* tactic1 tactic2)}\\
Defined by:\\
{\tt (then tactic1 (then (orelse tactic2 (idtac)) (idtac)))}.  This tactical is taken from \cite{Felty86}.

\item {\tt then**: (then** tactic1 tactic2)}\\
Acts like {\tt then}, except that no copying of the goal or related structures will be done. 

\item {\tt ifthen: (ifthen test tactic1)} or \\
        {\tt (ifthen test tactic1 tactic2)}\\
First evaluates {\tt test}, which may be either a tactic or (if user is an expert) an arbitrary LISP expression.  If test is a tactic and does not fail, or is an arbitrary LISP expression that does not evaluate to nil, then {\tt tactic1} will be called on {\tt goal} and its results returned. Otherwise, if {\tt tactic2} is present, the results of calling {\tt tactic2} on {\tt goal} will be returned, else failure is returned.  {\tt test} should be some kind of predicate; any new subgoals it returns will be ignored by {\tt ifthen}.

\item {\tt sequence: (sequence tactic1 tactic2 ... tacticN)}\\
Applies {\tt tactic1, ... , tacticN} in succession regardless of their success or failure.  Their results are composed.

\item {\tt compose: (compose tactic1 ... tacticN)}\\
Applies {\tt tactic1, ..., tacticN} in succession, composing their results until one of them fails.  Defined by:\\
{\tt (idtac)} if {\tt N}=0\\
{\tt (then* tactic1 (compose tactic2 ... tacticN))} if {\tt N} > 0.

\item {\tt try: (try tactic)}\\
Defined by: {\tt (then tactic (failtac))}.  Succeeds only if tactic returns no new subgoals, in which case it returns the results from applying {\tt tactic}. 

\item {\tt no-goal: (no-goal)}\\
Succeeds iff goal is nil.
\end{enumerate}


\section{Using Tactics}

To use a tactic from the top level, the command \indexmexpr{use-tactic} has
been defined.  Use-tactic takes three arguments: a {\it tactic-exp}, a 
{\it tactic-use},
and a {\it tactic-mode}.  The last two arguments default to the values of
\indexflag{tacuse} and \indexflag{tacmode}, respectively.
Remember that a {\it tactic-exp} can be either the name of
a tactic or a compound tactic.  Here are some examples:

%\begin{group}
\begin{verbatim}
<1> use-tactic propositional nat-ded auto

<2> use-tactic (repeat (orelse same-tac deduct-tac)) 
               $ interactive

<3> use-tactic (sequence (call pall) (call cleanup) (call pall)) !

<4> use-tactic (sequence (foo :use nat-etree :mode auto) 
                         (bar :use nat-ded :mode interactive)) !
\end{verbatim}
%\end{group}
Note that in the fourth example, the default use and mode are overridden
by the keyword specifications in the tactic-exp itself.  Thus during the
execution of this compound tactic, {\tt foo} will be called for one use and
in one mode, then {\tt bar} will be called with a different use and mode.

Remember, setting the value of the flag \indexflag{tactic-verbose} to T will
cause the tactics to send informative messages as they execute.




\subsection{Implementation of tactics and tacticals}

\begin{enumerate}
\item Main files (in order of importance): tactics-macros, tacticals,
tacticals-macros, tactics-aux.  These files contain tactic-related functions 
of a general nature.  Most tactics are actually contained in other Lisp packages.

\item When a tactic is executed, two global variables affect its
execution: tacuse (the tactic's use), and tacmode (the current mode).
Tacuse determines for what reason the tactic is being called.  Current
uses are etree-nat (translation of \indexother{eproof} to natural deduction) and
nat-ded (construction of a natural deduction proof without any mating
information).  A single tactic may be defined for more than one use.
Tacmode can have the value of either auto or interactive.  Each tactic
should take this value into account during operation.  In general,
this means that when the value is interactive, the user should be
advised that the tactic is about to be applied and should be allowed
to abort it.  When the value is auto, the tactic should just be
carried out if applicable.  

\item For each use, a number of auxiliary functions needed by the
tacticals must be defined.

\begin{enumerate}
\item get-tac-goal: if a goal has not been specified, get the next one,
e.g., the active planned line.

\item copy-tac-goal: copy the current goal into a new goal, so that subsequent
actions can be performed without destroying the current goal.  Allows
later backtracking if necessary.

\item save-tac-goal: put the current goal into a form suitable for
saving.  This is not actually used by current tactics.

\item restore-tac-goal: backtrack to the previous goal. 
This is not actually used by current tactics.

\item update-tac-goal: given the old (saved) goal, and the new goal on which
some progress has been made, update the old goal to reflect the
progress made.
\end{enumerate}

Tacticals must be independent of the value of tacuse.  
They cannot make any assumptions
about the structure of the goals, etc.

The main function used in applying tactics is apply-tactic.  This
is a function that takes a tactic as argument, and allows keyword
arguments of :goal, :use and :mode.  If not specified, the use and
mode default to the global values of tacuse and tacmode.
If they are specified,
the values given then override the global values of tacuse and tacmode.
apply-tactic and (every tactic) returns four values.  
 The first is a
list of goals, the second a string with some kind of message, the
third a token which indicates the result of the tactic and the fourth
a validation, which, if non-nil, should be a function which specifies
how solutions to the returned goals can be combined to solve the original goal.

apply-tactic works as follows:

\begin{enumerate}
\item  Checks that tactic is a valid tactic.

\item If a goal has not been specified, calls get-tac-goal.

\item If the tactic is an atom:
  \begin{enumerate}
\item     gets the tactic's definition for the use.

\item     if the definition is primitive, and the goal is nil, return
(nil "Goals exhausted." 'complete nil).  If the goal is non-nil, apply the 
tactic to the goal.

\item     if the definition is compound, call apply-tactical on the
definition and the goal.
  \end{enumerate}

\item  If the tactic's definition begins with a tactic, call apply-tactic
recursively,  using those optional arguments.

\item  If the tactic begins with a tactical, call apply-tactical.
\end{enumerate}

 Whenever a tactic begins with a tactical, the function
apply-tactical is used.  It takes two arguments, a goal and a tactic.
It is assumed that the tactic begins with a tactical.  
apply-tactical works as follows:


\begin{enumerate}
\item Get the definition for the tactical.

\item If the definition is primitive (a lambda expression), funcall the
definition on the goal and the remainder (cdr) of the tactic. 

\item If the definition is compound (i.e., is defined in terms of other
tacticals and begins with tac-lambda), expand the definition,
substituting the arguments provided in the tactic's definition for the
dummy  arguments in the tactical's definition.  Then call
apply-tactical recursively.

\item Otherwise abort, returning abort as the token (third value
returned).
\end{enumerate}

\item  Validations:  Though the validation mechanism is in place, no use
is made of them in the current tactic uses, since any changes in the
constructed proofs are made immediately, not saved.  Validations must
be modified as tacticals are executed, since during their execution,
the order of goals may be changed.  For example, a tactical may
repeatedly apply a tactic to a goal, then to all the new goals
created, etc., until it fails on all of them.  When it succeeds on a
successor goal, the validation returned must be integrated into the validation
which was returned for the first application of the tactic on the
original goal.  The function make-validation is used for this purpose.
\end{enumerate}


\chapter{Proof Translations}
\section{Data Structures}
\section{EProofs to Nproofs}
Here is a summary of what happens after matingsearch has terminated with
a proof. 
The functions involved are located in the files {\it mating-merge.lisp},
{\it mating-merge2.lisp} and {\it mating-merge-eq.lisp}.

\begin{enumerate}
\item Apply Pfenning's Merge algorithm ({\tt etr-merge}), put resulting
      etree and mating in variable \indexother{current-eproof}
  \begin{enumerate}
\item  Get a list of the connections from mating

\item Get a list of substitutions required
      \begin{enumerate}
\item  Extract substitutions from unification tree
        
\item  Replace occurrences of PI and SIGMA by quantifiers
        
\item  Lambda-normalize each substitution
        
\item  Alpha-beta-normalize each substitution
      \end{enumerate}
    
\item  Prune any unused expansions from the tree
    
\item  Make substitutions for variables
    
\item  Carry out merging ({\tt merge-all})
  \end{enumerate}

\item  Replace skolem terms by parameters (if applicable) 
({\tt subst-skol-terms})

\item  If remove-leibniz is T, apply Pfenning's algorithm for removing
      the Leibniz equality nodes for substitution of equality nodes
      ({\tt remove-leibniz-nodes})

\item  Try to replace selected parameters by the actual bound variables
      ({\tt subst-vars-for-params}), not always possible because of restriction
      that a parameter should appear at most once

\item  Raise lambda rewrite nodes, so that in natural deduction the lambda
      normalization occurs as soon as possible. ({\tt raise-lambda-nodes})

\item  Clean up the etree ({\tt cleanup-etree}).  For each expansion term
      in the tree,
   \begin{enumerate}
\item  Lambda-normalize it
      
\item  Minimize the superscripts on bound variables
      
\item Make a new expansion with the new term
      
\item  Deepen the new expansion like the original, but removing 
            unnecessary lambda-norm steps.
      
\item  Remove the original expansion
   \end{enumerate}

 Begin natural deduction proof, using \indexother{current-eproof}
   \begin{itemize}
\item  Set up planned line 
	\begin{enumerate}
\item  Use shallow wff of the current-eproof's etree
          
\item  Give it the tree as value for its NODE property
          
\item  Give it the current-eproof's mating (list of pairs of node names)
                as its MATING property
	\end{enumerate}
   \end{itemize}

 Call {\tt use-tactic} with tactic desired
  \begin{enumerate}
\item  Each line in the proof will correspond to a node in the etree; the 
natural deduction proof is stored in the variable \indexother{dproof}.
    
\item   Here is an important property which should remain invariant during
    the translation process:  It should always be the case that the 
    line-mating of the planned line is a p-acceptable mating for the
    etree that one could construct by making an implication whose antecedent
    is the conjunction of the line-nodes of the supports, and whose 
    consequent is the line-node of the planned line.  This will assure us
    that we have sufficient information to carry out the translation.
  \end{enumerate}
\end{enumerate}

It was observed that when path-focused duplication had been
used, the expansion proof would often have a great deal of redundancy
in the sense that the same expansion term would be used for a given variable
many times. More precisely, if one defines an expansion branch by
looking at sequences of nested expansion nodes, attaching one expansion
term to each expansion node in the sequence, there would be many identical
expansion branches. 

In response to this, {\it mating-merge.lisp} was modified in the following ways:
\begin{itemize}
\item Don't do pruning of unnecessary nodes at the beginning of the merge,
when the tree is its greatest size. 

\item Instead, prune all branches that couldn't possibly have been used; 
they are those that have a zero status. This is probably not necessary,
but certainly makes debugging easier and doesn't cost much.

\item After merging of identical expansions has been done, call the original
pruning function.
\end{itemize}

\section{NProofs to Eproofs}
There are three versions of \indexcommand{NAT-ETREE},
the command for translating natural deductions into
expansion tree proofs.  The user can choose between
the three by setting the flag \indexflag{NAT-ETREE-VERSION}
to one of the following values:
\begin{enumerate}
\item {\bf OLD}  (the original version)
\item {\bf HX} (Hongwei Xi's version, written in the early
to mid 1990's)
\item {\bf CEB} (Chad E. Brown's version, written in early 2000)
\end{enumerate}
Also, note that setting the flag \indexflag{NATREE-DEBUG}
to T is useful for debugging the {\bf HX} and {\bf CEB}
versions.  The subsections that follow describe each of these
versions in greater detail.

After using \indexcommand{NAT-ETREE} to translate to an
expansion proof, the user can use this expansion proof
to suggest flag settings (via the mate commands
\indexother{ETR-INFO} and \indexcommand{ETREE-AUTO-SUGGEST})
or to trace MS98-1 (using the flag \indexflag{MS98-TRACE}.  
See the User's Manual for a description
of these facilities.  The User's Manual also has examples.

\subsection{Chad's Nat-Etree}\label{ceb-nat-etr}

To use this version of \indexcommand{NAT-ETREE}, set
\indexflag{NAT-ETREE-VERSION} to {\bf CEB}.
The main functions for this version are in the
files \indexfile{ceb-nat-etr.lisp} and \indexfile{ceb-nat-seq.lisp}.  
The relevant functions are:
\begin{description}
\item [\indexfunction{ceb-nat-etree}]  
This is the main function.  It preprocesses the proof to 
\begin{itemize}
\item remove applications of {\it Subst=} and {\it Sym=},
\item expand applications of {\it RuleP} and other propositional rules
(e.g., {\it Assoc}) in terms of more
primitive inference rules,
\item attempt to expand any applications of {\it RuleQ},
\item replace instances of {\it Assert} by hypotheses
which are discharged at the end of the proof,
\item and replace applications of the {\it Cases} rule
using more than two disjuncts by multiple applications
of the {\it Cases} rule using two disjuncts.
\end{itemize}
The function then calls \indexfunction{ceb-proof-to-natree} to build
the natree version of the natural deduction proof, calls
\indexfunction{natree-to-ftree-main} to build the ftree representation
of the expansion tree and a complete mating.  Finally, this
is converted into an ordinary expansion proof which may optionally
be merged.  (Merging is appropriate if the user plans to translate
back to a natural deduction proof, but inappropriate if the user
is trying to gather information about a potential automatic proof.)
\item [\indexfunction{ceb-proof-to-natree}]  This is a modification
of Hongwei's 
\indexfunction{proof-to-natree} (see \indexfile{hx-natree-top.lisp}).
This function builds the natree, changing some justifications to $RuleP$
or $RuleQ$, and changing the variables in applications of $UGen$ and $RuleC$
so they are unique in the entire proof (i.e., the natree rules satisfy a global
eigenvariable condition, since the etree selection variables must be distinct).
\item [\indexfunction{natree-to-ftree-main}]  This function calls
\indexfunction{natree-to-ftree-seq-normal} to build a sequent calculus derivation
(see ~\ref{ftree-seq}) from the natree.  Then \indexfunction{ftree-seq-weaken-early}
modifies the derivation so that the weaken rule is applied eagerly.  This may
eliminate certain unnecessary cuts and simplify the derivation.
Then the cut elimination function \indexfunction{ftree-seq-cut-elim}
(see section~\ref{ftree-seq-mix-elim})
is used to make the derivation cut-free.
Finally, \indexfunction{cutfree-ftree-seq-to-ftrees} is used to obtain
an ftree and a complete mating from the cut-free derivation.
\item [\indexfunction{natree-to-ftree-seq-normal}, \indexfunction{natree-to-ftree-seq-extraction}]
These 
functions are mutually recursive and provide the main algorithm for constructing
the sequent calculus derivation.  A description of the algorithm is below.  The function \indexfunction{natree-to-ftree-seq-normal}
is called on natree nodes which are considered normal.  (These would be annotated with a $\Uparrow$.)
The function \indexfunction{natree-to-ftree-seq-extraction} is called on natree nodes which
are considered extractions.  (These would be annotated with a $\downarrow$.)
\end{description}

Frank Pfenning's ATP
class contained notes on annotating (intuitionistic first-order)
normal natural deduction proofs, and gave a constructive proof
(algorithm) that every natural deduction proof
translates into a sequent calculus proof.  Also, normal natural deduction proofs
translate to cut-free sequent calculus proofs.
The idea of using annotations carries over to classical higher-order
logic.

\subsubsection{Normal Deductions}

The idea of a normal deduction is that the proof works down
using elimination rules, and up using introduction rules,
meeting in the middle.  We can formalize this idea by saying that
a natural deduction proof is normal if its assertions can be annotated,
so that the assertions involved in the applications of rules of inference
are as described below.
Technically, we are defining normal natural deductions by mutually
defining normal deductions ($\Uparrow$) and extraction deductions ($\downarrow$).

\subsubsection{Annotations of the Assertions in a Proof}

First, the basic rules which allow one to infer normal deductions ($\Uparrow$).
$$ \ianc{A\Uparrow}{\forall x A\Uparrow}{UGen}$$
where $x$ is not free in any hypotheses.
$$ \ibnc{\exists y A\downarrow}{\hypo{\ian{}{[x/y]A\downarrow}{}}{C\Uparrow}}{C\Uparrow}{RuleC}$$
where $x$ is not free in any hypotheses.
$$ \ianc{[t/x]A\Uparrow}{\exists x A\Uparrow}{EGen}$$
$$ \ianc{A\Uparrow}{A\lor B\Uparrow}{IDisj-L} \;
 \ianc{B\Uparrow}{A\lor B\Uparrow}{IDisj-R}$$
$$ \ibnc{A\Uparrow}{B\Uparrow}{A \land B\Uparrow}{Conj} $$
$$ \ianc{\bot\downarrow}{A\Uparrow}{Absurd}$$
$$ \ianc{\hypo{\ian{}{\neg A\downarrow}{}}{\bot\Uparrow}}{A\Uparrow}{Indirect} \;
 \ianc{\hypo{\ian{}{A\downarrow}{}}{\bot\Uparrow}}{\neg A\Uparrow}{NegIntro} \;
 \ianc{\hypo{\ian{}{A\downarrow}{}}{B\Uparrow}}{A\supset B\Uparrow}{Deduct}$$
$$ \ibnc{\neg A\downarrow}{A\Uparrow}{C\Uparrow}{NegElim}$$
$$ \ibnc{A \vee B\downarrow}{\hypo{\ian{}{A\downarrow}{}}{C\Uparrow}\quad\qquad\hypo{\ian{}{B\downarrow}{}}{C\Uparrow}}{C\Uparrow}{Cases}$$

TPS also has rules {\it Cases3} and {\it Cases4} which may be used to
eliminate disjunctions with three or four disjuncts, resp.  
Such rule applications are replaced by iterations of the binary
{\it Cases} rule in a preprocessing step using
\indexfunction{expand-cases}.

Next, the basic rules which allow one to infer extraction deductions ($\downarrow$).
$$ \ianc{A \land B\downarrow}{A\downarrow}{Conj} \hspace{2em} \ianc{A \land B\downarrow}{B\downarrow}{Conj}$$
$$ \ianc{\forall x A\downarrow}{[t/x]A\downarrow}{UI}$$
$$ \ibnc{A \limplies B\downarrow}{A\Uparrow}{B\downarrow}{MP}$$

Notice that hypothesis lines are always considered extraction derivations.
Such lines may be justified by any of the following:
$Hyp$, $Choose$, $Assume negation$, $Case 1$, $Case 2$, $Case 3$, $Case 4$.

We need a coercion rule, as every extraction is a normal derivation:
$$ \ianc{A\downarrow}{A\Uparrow}{coercion}$$
In a TPS natural deduction style proof, this coercion step will not usually be
explicit.  Instead, a single line will be given the property of being
a coercion, in which case we know it has both annotations $\downarrow$
and $\Uparrow$, and that these annotations were assigned in a way consistent
with the coercion rule above.  Often, when interactively constructing
a natural deduction proof in TPS, one finds that a planned line is the
same as a support line, and finishes the subgoal using SAME.  This would
correspond to the coercion rule above.

The backward coercion rule 
$$ \ianc{A\Uparrow}{A\downarrow}{bcoercion}$$
is used to pass from normal deductions to extractions.
Backwards coercions correspond to instances of cut.
A separate, interesting project in TPS would be to
program a normalization procedure.
Such a procedure would
find instances of the backward coercion rule when
annotating a proof, identify to what kind of ``redex''
the backward coercion rule corresponds, and perform the
reduction.  For this to work we would need to define the
notion of redex so that every proof which needs the backward
coercion rule to be annotated (proofs that are not normal)
must have a redex.  Also, we could not prove that reduction
terminates -- a task equivalent to constructively proving 
cut-elimination in classical higher-order logic.
Instead, the current code translates backwards coercion
as an application of cut, and then uses a cut elimination
procedure (which may not terminate) to obtain a cut-free
proof.  (See section~\ref{ftree-seq-mix-elim})

\subsubsection{Some Nonstandard ND Rules}

There is code to replace ``fancy'' propositional rules
like $RuleP$ with subderivations using primitive rules.  
See the commands \indexcommand{ELIMINATE-ALL-RULEP-APPS}, \indexcommand{ELIMINATE-RULEP-LINE},
and \indexcommand{ELIMINATE-CONJ*-RULEP-APPS}.  The command \indexcommand{ELIMINATE-CONJ*-RULEP-APPS}
only expands those $RuleP$ applications which can be replaced by applications of $Conj$ rules.
The other two commands use Sunil's fast propositional search to find an expansion proof
and uses the tactic \indexother{BASIC-PROP-TAC} to translate back to natural deduction to fill
in the gap using only primitive propositional rules.

Of course, the $Same$ rule just propagates the annotation.
So, with respect to annotations, there are two versions of this rule:
$$\ianc{A\downarrow}{A\downarrow}{Same As}\hspace{2em}
\ianc{A\Uparrow}{A\Uparrow}{Same As}$$

When converting normal natural deduction to expansion tree
proofs, we only consider formulas up to $\alpha$-conversion,
so we can ignore the corresponding ND rule.  But effectively,
we allow this rule to be annotated in either of two ways,
as with the $Same$ rule:
$$\ianc{\forall x A\downarrow}{\forall y [y/x]A\downarrow}{AB}\hspace{2em}
\ianc{\forall x A\Uparrow}{\forall y [y/x]A\Uparrow}{AB}$$

$Neg$, $NNF$, and $NNF-Expand$ can be used to make small first-order
inferences (from $\neg \forall x . A$ to $\exists x . \neg A$, etc.).
Since we only care about formulas up to $\alpha\beta$ and negation-normal-form, we
can treat these rules the same way as the $Same$ and $AB$ rules.

Applications of $Assert$ (other than $Assert Refl=$) are replaced by
explicit hypotheses in a preprocessing step by the function 
\indexfunction{make-assert-a-hyp}.  So, when building the natree,
there should be no instances of $Assert$ other than $Refl=$.
$Assert Refl=$ is annotated as a normal deduction:
$$\ianc{}{A=A\Uparrow}{Assert Refl=}$$
The idea is that we work backwards to an instance of reflexivity.

Definitions can be eliminated or introduced, and the annotations
reflect this.  Also, elimination and introduction of definitions includes
some $\beta$-reduction.  Suppose the abbreviation $A$ is defined to
be $\lambda x_1\cdots\lambda x_n . \psi [x_1,\ldots, x_n]$ in the following
annotated rule schemas.
$$\ianc{\phi[A\; B_1\; \cdots \; B_n]\downarrow}{\phi[\psi[B_1,\ldots,B_n]]\downarrow}{Defn}$$
$$\ianc{\phi[\psi[B_1,\ldots,B_n]]\Uparrow}{\phi[A\; B_1\; \cdots \; B_n]\Uparrow}{Defn}$$

When annotating $\lambda$ rules, the arrows point in the
direction of normalization.
$$\ianc{B\downarrow}{A\downarrow}{Lambda}\hspace{2em}
\ianc{A\Uparrow}{B\Uparrow}{Lambda}$$
where $A$  is the $\beta\eta$-normal form of $B$.
There are also rules $Beta$ and $Eta$ which are treated similarly.

\subsubsection{Equality Rules}

We assume that the proof has been preprocessed to remove
applications of substitution of equals, and applications of
symmetry, so there is no need to annotate these rules (for now).

As noted above, reflexivity is treated as a normal deduction:
$$\ianc{}{A = A\Uparrow}{Refl=}$$

There are two ways to apply extensionality consistent with the
idea of annotations (both correspond to expanding an equation using
extensionality in the corresponding expansion tree).  Also, there
are two kinds of extensionality (functional and propositional).
$$\ianc{f_{\greekb\greeka} = g_{\greekb\greeka}\downarrow}{\forall x_\greeka . f x = g x\downarrow}{Ext=}
\hspace{2em}
\ianc{P_\greeko = Q_\greeko\downarrow}{P_\greeko \equiv Q_\greeko \downarrow}{Ext=}$$
$$\ianc{\forall x_\greeka . f x = g x\Uparrow}{f_{\greekb\greeka} = g_{\greekb\greeka}\Uparrow}{Ext=}
\hspace{2em}
\ianc{P_\greeko \equiv Q_\greeko\Uparrow}{P_\greeko = Q_\greeko\Uparrow}{Ext=}$$

Leibniz equality is handled just like definition expansion.
$$\ianc{A_\greeka = B_\greeka\downarrow}{\forall q_{\greeko\greeka} . q A \limplies q B\downarrow}{Equiv-eq}$$
$$\ianc{\forall q_{\greeko\greeka} . q A \limplies q B\Uparrow}{A_\greeka = B_\greeka\Uparrow}{Equiv-eq}$$

\subsubsection{A Sequent Calculus}\label{ftree-seq}

In the file \indexfile{ftree-seq.lisp}, a sequent calculus
is implemented.  The file contains code to convert
sequent calculus derivations into
expansion proofs, and a cut elimination
algorithm.  This is a two sided sequent calculus
with sequents $\Gamma \rightarrow \Delta$.
The code refers to formulas in $\Gamma$ as positive
(as opposed to ``left'') and formulas in $\Delta$
as negative (as opposed to ``right'') to correspond
to the parity of expansion tree nodes.
$\Gamma$ and $\Delta$ are lists, as opposed to multisets
or sets, so order and multiplicity are important.

There are many variations of sequent calculi for classical
logic.  For example, consider the two
variants of the negative rule for $\land$:
$$\ibnc{\Gamma\rightarrow A,\Delta}{\Gamma\rightarrow B,\Delta}{\Gamma\rightarrow A\land B,\Delta}{}$$
$$\ibnc{\Gamma_1\rightarrow A,\Delta_1}{\Gamma_2\rightarrow B,\Delta_2}{\Gamma_1,\Gamma_2\rightarrow A\land B,\Delta_1,\Delta_2}{}$$
Furthermore, there is the issue of the positions of
the main formulas (i.e., must $A$ and $B$ be the first formulas
on the list?)
Different kinds of rules determine what structural rules the sequent calculus
should have.  The sequent calculus implemented in \indexfile{ftree-seq}
has the following logical rules:

\begin{center}
$\ibnc{\Gamma_1\rightarrow A,\Delta_1}{\Gamma_2\rightarrow B,\Delta_2}{\Gamma_1,\Gamma_2\rightarrow A\land B,\Delta_1,\Delta_2}{\land-}$\hspace{2cm}
$\ianc{A,B,\Gamma\rightarrow \Delta}{A\land B,\Gamma\rightarrow \Delta}{\land+}$\\[.5cm]
$\ianc{\Gamma\rightarrow A,B,\Delta}{\Gamma\rightarrow A\lor B,\Delta}{\lor-}$\hspace{2cm}
$\ibnc{A,\Gamma_1\rightarrow \Delta_1}{B,\Gamma_2\rightarrow \Delta_2}{A\lor B,\Gamma_1,\Gamma_2\rightarrow \Delta_1,\Delta_2}{\lor+}$\\[.5cm]
$\ianc{A,\Gamma\rightarrow B,\Delta}{\Gamma\rightarrow A\limplies B,\Delta}{\limplies-}$\hspace{2cm}
$\ibnc{\Gamma_1\rightarrow A,\Delta_1}{B,\Gamma_2\rightarrow \Delta_2}{A\limplies B,\Gamma_1,\Gamma_2\rightarrow \Delta_1,\Delta_2}{\limplies+}$\\[.5cm]
$\ianc{A,\Gamma\rightarrow \Delta}{\Gamma\rightarrow \lnot A,\Delta}{\lnot-}$\hspace{2cm}
$\ianc{\Gamma\rightarrow A,\Delta}{\lnot A,\Gamma\rightarrow \Delta}{\lnot+}$\\[.5cm]
$\ianc{A(a),\Gamma\rightarrow \Delta}{\forall x A(x),\Gamma\rightarrow \Delta}{SEL+^a}$\hspace{2cm}
$\ianc{\Gamma\rightarrow A(a),\Delta}{\Gamma\rightarrow \forall x A(x),\Delta}{SEL-^a}$\\[.5cm]
$\ianc{A(t),\Gamma\rightarrow \Delta}{\forall x A(x),\Gamma\rightarrow \Delta}{EXP+^t}$\hspace{2cm}
$\ianc{\Gamma\rightarrow A(t),\Delta}{\Gamma\rightarrow \exists x A(x),\Delta}{EXP-^t}$\\[.5cm]
\end{center}

There are also rewrite rules:\\[.5cm]
\begin{center}
$\ianc{\Gamma\rightarrow [A\limplies B]\land [B\limplies A],\Delta}{\Gamma\rightarrow A\equiv B,\Delta}{REW(\equiv)-}$\\[.5cm]
$\ianc{[A\limplies B]\land [B\limplies A],\Gamma\rightarrow \Delta}{A\equiv B,\Gamma\rightarrow \Delta}{REW(\equiv)+}$\\[.5cm]
$\ianc{\Gamma\rightarrow A^{\lambda,\beta,\eta},\Delta}{\Gamma\rightarrow A,\Delta}{REW({\lambda,\beta,\eta})-}$\hspace{2cm}
$\ianc{A^{\lambda,\beta,\eta},\Gamma\rightarrow \Delta}{A,\Gamma\rightarrow \Delta}{REW({\lambda,\beta,\eta})+}$\\[.5cm]
\end{center}
where $A^{\lambda,\beta,\eta}$ is either the $\beta\eta$-normal form,
$\beta$-normal form, or $\eta$-normal form of $A$.\\[.5cm]
\begin{center}
$\ianc{\Gamma\rightarrow A,\Delta}{\Gamma\rightarrow B,\Delta}{REW(AB)-}$\hspace{2cm}
$\ianc{A,\Gamma\rightarrow \Delta}{B,\Gamma\rightarrow \Delta}{REW(AB)+}$\\[.5cm]
\end{center}
where $A$ and $B$ are $\alpha$-equivalent.\\[.5cm]
\begin{center}
$\ianc{\Gamma\rightarrow A,\Delta}{\Gamma\rightarrow B,\Delta}{REW(EQUIVWFFS)-}$\\[.5cm]
$\ianc{A,\Gamma\rightarrow \Delta}{B,\Gamma\rightarrow \Delta}{REW(EQUIVWFFS)+}$\\[.5cm]
\end{center}
where $A$ is the result of expanding some abbreviations in $B$.\\[.5cm]
\begin{center}
$\ianc{\Gamma\rightarrow A,\Delta}{\Gamma\rightarrow B,\Delta}{REW(Leibniz=)-}$\hspace{2cm}
$\ianc{A,\Gamma\rightarrow \Delta}{B,\Gamma\rightarrow \Delta}{REW(Leibniz=)+}$\\[.5cm]
\end{center}
where $A$ is the result of expanding some equalities in $B$ using the Leibniz
definition of equality.\\[.5cm]
\begin{center}
$\ianc{\Gamma\rightarrow A,\Delta}{\Gamma\rightarrow B,\Delta}{REW(Ext=)-}$\hspace{2cm}
$\ianc{A,\Gamma\rightarrow \Delta}{B,\Gamma\rightarrow \Delta}{REW(Ext=)+}$\\[.5cm]
\end{center}
where $A$ is the result of expanding some equalities in $B$ using extensionality.
(This does not provide a complete calculus for extensionality without a cut rule.
So, sometimes cut elimination will fail if these extensionality rules are used.)

The structural rules are:

\begin{center}
$\ianc{\Gamma\rightarrow \Delta}{\Gamma\rightarrow A,\Delta}{weaken-}$\hspace{2cm}
$\ianc{\Gamma\rightarrow \Delta}{A,\Gamma\rightarrow \Delta}{weaken+}$\\[.5cm]
$\ianc{\Gamma\rightarrow A,A,\Delta}{\Gamma\rightarrow A,\Delta}{merge-}$\hspace{2cm}
$\ianc{A,A,\Gamma\rightarrow \Delta}{A,\Gamma\rightarrow \Delta}{merge+}$\\[.5cm]
$\ianc{\Gamma\rightarrow\Delta_1,A,\Delta_2}{\Gamma\rightarrow A,\Delta_1,\Delta_2}{focus^n+}$
where $\Delta_1$ has length $n$.\\[.5cm]
$\ianc{\Gamma_1,A,\Gamma_2\rightarrow\Delta}{A,\Gamma_1,\Gamma_2\rightarrow\Delta}{focus^n+}$
where $\Gamma_1$ has length $n$.\\[.5cm]
\end{center}

Finally, we have an initial rule and a cut rule:

\begin{center}
$\ianc{}{A\rightarrow A}{init}$\hspace{2cm}
$\ibnc{\Gamma_1\rightarrow A,\Delta_1}{A,\Gamma_2\rightarrow \Delta_2}{\Gamma_1,\Gamma_2\rightarrow \Delta_1,\Delta_2}{cut}$\\[.5cm]
\end{center}

In all these rules, it is important that the formulas
appear in the positions as indicated in the diagrams above.
The focus rule gives us the only way to reorder
the formulas of the sequent.  This forces us to do some
tedious shuffling in some places, but makes it easier to
perform recursion on the sequent derivations, since we
have a very good idea of how the rule application looks.

The sequent calculus is 
similar to the ftree representation of expansion trees (see section~\ref{ftrees}),
and the file includes a function
\indexfunction{cutfree-ftree-seq-to-ftrees}
which translates a cut-free sequent calculus derivation of
$\Gamma\rightarrow\Delta$
to two lists of ftrees
$\Gamma^*$ and $\Delta^*$,
and a list of connections $M$.
For each wff $A\in\Gamma$, there is a
correponding positive ftree $F\in\Gamma^*$ with shallow 
formula $A$.
For each wff $A\in\Delta$, there is a
correponding negative ftree $F\in\Delta^*$ with shallow 
formula $A$.  The list of connections $M$ gives a complete
mating for the ftree
$\bigwedge(\Gamma^*) \limplies \bigvee(\Delta^*)$.
In particular, a cut-free sequent calculus derivation of
$\rightarrow A$ will be translated into a negative ftree
with shallow formula $A$ and a complete mating $M$.

Regarding this translation to ftrees,
the names of the logical and rewrite rules
correspond to the construction of the corresponding ftree.
The $weaken$ and $focus$ structural rules are relatively easy to handle.
Applications of $merge$ require the use of a $merge$ algorithm for ftrees
(in the file \indexfile{ftrees}).
The $init$ rule corresponds to two mated nodes.
And, of course, we cannot translate an application of $cut$.

\subsubsection{Translating from Natural Deduction to Sequent Calculus}

In the sequent calculus described above, the order and multiplicity
of formulas is important.  However, in describing the algorithm below,
we are more interested in sets of formulas.  So, let us use the
notation $Set(\Gamma)$ to denote the set of fomulas on the list $\Gamma$.

Normal natural deductions are converted into the sequent calculus 
via two mutually recursive algorithms:

\begin{enumerate}
\item  \indexfunction{natree-to-ftree-seq-normal}:
Suppose we are given a line $\Gamma \vdash C\Uparrow$.
Then we can compute a derivation of a sequent
$\Gamma_1\rightarrow C$ where $Set(\Gamma_1)\subseteq Set(\Gamma)$.
\item  \indexfunction{natree-to-ftree-seq-extraction}:
Given a line $\Gamma \vdash B\downarrow$
and a derivation of a sequent $\Gamma_1 \rightarrow C$
where $Set(\Gamma_1) \subseteq Set(\Gamma)\cup \{B\}$.
Then we can compute a derivation of
a sequent
$\Gamma_2\rightarrow C$
where $Set(\Gamma_2) \subseteq Set(\Gamma)$.
(That is, we have eliminated occurrences of $B$ on the
positive side.)
\end{enumerate}

We can show a few cases to demonstrate how the algorithms
work.

{\bf Case:}
Coercion.
$$ \ianc{C\downarrow}{C\Uparrow}{coercion}$$
with hypotheses $\Gamma$.
We need a derivation of some $\Gamma_2\rightarrow C$.
We can apply the second induction hypothesis
to the initial sequent $C\rightarrow C$ (with
$\Gamma_1$ empty)
to obtain a derivation of such a sequent $\Gamma_2\rightarrow C$.

{\bf Case:}  Hyp.  Suppose the line is
$$\Gamma \vdash B$$
where $B$ is in $\Gamma$,
and suppose we are given a derivation of a sequent
$\Gamma_1 \rightarrow C$ with
$Set(\Gamma_1)\subseteq Set(\Gamma)\cup \{B\}$.
Since $B\in\Gamma$, we have
$Set(\Gamma_1)\subseteq Set(\Gamma)$
and we are done.

{\bf Case:}  Deduct.  This case is easy, as are most of the ``introduction''
rules.  Suppose we have
$$\ianc{\above{\DD}{\Gamma, A\vdash B\Uparrow}}{\Gamma \vdash A\supset B\Uparrow}{Deduct}$$
By induction we have a derivation of a sequent
$\Gamma_1\rightarrow B$
where $Set(\Gamma_1)\subseteq Set(\Gamma)\cup \{A\}$.
Using the structural rules (see the function \indexfunction{ftree-seq-merge-focus-all-pos})
we obtain a derivation with $A$ at the front:
$A,\Gamma_2\rightarrow B$
where $Set(\Gamma_2) \subseteq Set(\Gamma)$.
Applying the $\limplies-$ rule, we have a derivation of
$\Gamma_2\rightarrow A\limplies B$ as desired.

{\bf Case:}  MP.  This case is interesting, because a naive   
algorithm would be forced to treat this case like a ``cut'' in
the sequent calculus.  Suppose we have
$$ \ibnc{\above{\DD}{\Gamma \vdash A \limplies B\downarrow}}{\above\EE{\Gamma \vdash A\Uparrow}}{\Gamma \vdash B\downarrow}{MP}$$
Since this is an extraction, we must be given a derivation of
$\DD_1$ a sequent $\Gamma_1\rightarrow C$
where $Set(\Gamma_1) \subseteq Set(\Gamma)\cup \{B\}$.
Applying structural rules to $\DD_1$, we have a derivation $\DD_2$
of a sequent $B,\Gamma_2\rightarrow C$
with $Set(\Gamma_2)\subseteq Set(\Gamma)$.

The first algorithm applied to $\EE$ gives a derivation of some
$\Gamma_3\rightarrow A$ where $Set(\Gamma_3)\subseteq Set(\Gamma)$.
If we apply $\limplies+$ as follows:
$$\ibnc{\Gamma_3\rightarrow A}{B,\Gamma_2\rightarrow C}{A\limplies B,\Gamma_3,Gamma_2\rightarrow C}{\limplies+}$$
then we can call the second algorithm on this derivation and $\DD$
to obtain a derivation of some $\Gamma_4\rightarrow C$
with $Set(\Gamma_4)\subseteq Set(\Gamma)$.

{\bf Case:}  Backwards Coercion.  
$$ \ianc{\above{DD}{B\Uparrow}}{B\downarrow}{bcoercion}$$
with hypotheses $\Gamma$.
Suppose we are given a derivation of some $\Gamma_1\rightarrow C$
where $Set(\Gamma_1)\subseteq Set(\Gamma)\cup \{B\}$.
Using structureal rules we obtain a derivation of a sequent
$B,\Gamma_2\rightarrow C$ where
$Set(\Gamma_2)\subseteq Set(\Gamma)$.
We want to remove $B$ from the positive side.
Applying the first algorithm to $\DD$, we obtain a
derivation of a sequent $\Gamma_3\rightarrow B$
with $Set(\Gamma_3)\subseteq Set(\Gamma)$.
An application of $cut$ gives us the sequent we desire:
$$\ibnc{\Gamma_3\rightarrow B}{B,\Gamma_2\rightarrow C}{\Gamma_3,\Gamma_2\rightarrow C}{cut}$$

{\bf Remark:}  We check equality of wff's up to $\alpha$-conversion and negation-normal-form.
Because we check up to negation-normal-form, applications of $Neg$ and $NNF$ rules can be
treated the same way as the $Same$ and $AB$ rules.

{\bf Note:}  If \indexflag{NATREE-DEBUG} is set to T, then at each step,
the code double checks that the derivation is well formed.

After we have a sequent calculus derivation, cut elimination can be
used to try to remove applications of cut (see section~\ref{ftree-seq-mix-elim}.  If we obtain a cut-free
derivation, this can be translated into an ftree with a complete mating.

% {\bf Example:}
% 
% Consider the following natural deduction proof of
% $$\forall \,x [ \,P \,x \supset \,P . \,f \,x] \supset . \,P \,a \supset \,P . \,f . \,f \,a$$
% 
% {\it Show the proof.}
% 
% This is a normal proof, as it can be annotated as follows (without using backward coercion):
% 
% {\it Fill in the rest of this example and work through the conversion to an expansion
% tree proof step by step.}

\subsubsection{Normalization of Proofs}

There is now a \TPS command \indexcommand{NORMALIZE-PROOF}
that converts a natural deduction proof (or a natural
deduction proof with asserted lemmas which have natural
deduction proofs in memory) into a sequent calculus proof
(with cuts), then uses the cut-elimination algorithm
to obtain a cut-free proof (assuming termination),
and finally translates back to natural deduction.
The resulting natural deduction proof is normal.

 
If we decided to normalize natural deduction proofs directly
(without passing through a sequent calculus), we would
need to identify
possible redexes (pairs of rule applications which must use
backward coercion to be annotated), and show how to reduce these.
There are many such redexes.  The following is a typical example:

$$ \ianc{\ibnc{\above\DD{A}}{\above\EE{B}}{A \land B\downarrow}{Conj}}{A\downarrow}{Conj} \rightarrow \above\DD{A}$$

In first order logic, one can show that some measure on the proof reduces
when a redex is reduced, so that the process will terminate with
a normal proof.  In higher order logic, showing termination is equivalent
to showing termination of cut-elimination.

Actually carrying this out is a possible future project.  Though
this is much less important since we now have a cut elimination algorithm
implemented.

\subsection{Hongwei's Nat-Etree}

This is a brief description of Hongwei's code
for \indexcommand{NAT-ETREE}.  To use this code,
set \indexflag{NAT-ETREE-VERSION} to {\bf HX}.

%{\bf Note:  Hongwei wanted this code to work for all natural deductions,
%whether normal or not.  So, given a natural deduction proof which
%is not normal, try this one, and you might get lucky.}

\indexfunction{ATTACH-DUP-INFO-TO-NATREE} is the main function, which
is called recursively on the subproofs of a given natural
deduction. The goal of \indexfunction{ATTACH-DUP-INFO-TO-NATREE} is to
construct an expansion tree, with no mating attached,
corresponding to a given natural deduction. The constructed
expansion tree contains all the correct duplications done on
quantifiers and all substitutions done on variables.
A propositional search will be called on the generated expansion
tree to recover the mating and generate an expansion proof.
Then \indexcommand{ETREE-NAT} can produce a natural deduction corresponding
to the constructed expansion proof.

The following is an oversimplified case.

Given natural deductions N1 and N2 with conclusions
A and B, respectively, and N derived from N1 and N2
by conjunction introduction. \indexfunction{ATTACH-DUP-INFO-TO-NATREE}
called on N generates two recursive calls on N1 and N2,
and get the expansion proofs corresponding to N1 and N2,
respectively, with which it constructs an expansion proof
corresponding to N.

An important feature of \indexfunction{ATTACH-DUP-INFO-TO-NATREE} is that
it can deal with all natural deductions, with or without
cuts in them. This is mainly achieved by substitution and
merge. This essentially corresponds to the idea in Frank
Pfenning's thesis, though his setting is sequent calculus.
On the other hand, the implementation differs significantly
since natural deductions grow in both ways when compared with
sequent calculus. This is reflected in the code of
\indexfunction{ATTACH-DUP-INFO-TO-NATREE} which travels through a natural
deduction twice, from bottom to top and from top to bottom,
to catch all the information needed to duplicate quantifiers
correctly.

Overview of the files:
\begin{itemize} 
\item \indexfile{hx-natree-top} contains the definition of the data structure,
some print functions and top commands.

\item \indexfile{hx-natree-duplication} contains the code of \indexfunction{ATTACH-DUP-INFO-TO-NATREE}
and some auxiliary functions such as \indexfunction{UPWARD-UPDATE-NATREE}. Also many
functions for constructing expansion trees are defined here.

\item \indexfile{hx-natree-rulep} contains the code for handling \indexfunction{RULEP}. This is done
by using hash tables to store positive and negative duplication
information. Then cuts are eliminated by substitution and merge.
The case in \indexfunction{ATTACH-DUP-INFO-TO-NATREE} which deals with implication
is a much simplified version of this strategy, and helps understand
the algorithm.

\item \indexfile{hx-natree-aux} contains the code of merge functions and the ones handling
rewrite nodes. Presumably there are some bugs in handling rewrites, and
this can be found in the comments mixed with the code. Also a new version
of \indexfunction{ETREE-TO-JFORM-REC} is defined here to cope with a modified date structure
\indexother{ETREE}.

\item \indexfile{hx-natree-cleanup} contains the functions which clean up the expansion
proofs before they can be used by \indexcommand{ETREE-NAT}. This is temporary crutch,
and should be replaced by some systematic methods. For instance, one
could construct brand new expansion proofs according to a constructed
one rather than modify it to fit the needs of \indexcommand{ETREE-NAT}. This yields
a better chance to avoid some problems caused by rewrite nodes. 

\item \indexfile{hx-natree-debug} contains some simple debugging facilities such as some
display function and some modified versions of the main functions in the
code. A suggested way is to modify the code using these debugging functions
and trace them. More facilities are needed to eliminate sophisticated bugs.
\end{itemize}

Selection nodes, not Skolem nodes, are used in the constructed expansion
trees. The prevents us from setting the \indexflag{MIN-QUANT-ETREE} flag to simplify a
proof. It is a little daunting task to modify the code for \indexflag{MIN-QUANT-ETREE},
but the benefits are also clear: both \indexcommand{NAT-ETREE} and non-pfd procedures can
take advantage of the modification.

\subsection{The Original Nat-Etree}

{\bf Note: What follows is a description of how NAT-ETREE used to work.}
To use this code set \indexflag{NAT-ETREE-VERSION} to {\bf OLD}.

Legend has it that the code was written by Dan Nesmith and
influenced by the ideas of Frank Pfenning.  Frank's thesis
contains ideas for translating from a cut-free sequent calculus
to expansion tree proofs.

\begin{enumerate}
\item Important files: nat-etr (defines functions which are independent
of the particular rules of inference used); ml-nat-etr1 and
ml-nat-etr2 (which define translations for the rules in the standard TPS).

\item There are three global variables which are used throughout the
translation process: DPROOF, which is the nproof to
be translated; LINE-NODE-LIST, which is an association list which
associates each line of the proof to the node which represents it in
the expansion tree which is being constructed; MATE-LIST, which is a
list of connections in the expansion proof which is being constructed.

\item At the beginning of the translation process, the current proof is
copied because modifications will be made to it.  (It is restored when
the translation is complete.)  The copy is stored in the variable
DPROOF.  Next the function SAME-IFY is called.  This attempts to undo
the effects of the CLEANUP function, and to make explicit the
"connections" in the proof.  This is done because, in an nproof, 
a single line can represent more than one node in an
expansion proof.  SAME-IFY tries to add lines to the proof in such a
way that each line corresponds to exactly one expansion tree node.  

\item After the proof has been massaged by SAME-IFY, the initial root
node of the expansion tree is constructed.  This node is merely a
leaf whose shallow formula is the assertion of the last line of the
nproof.  LINE-NODE-LIST is initialized to contain
just the association of this leaf node with the last line of the
proof, and MATE-LIST is set to nil.

\item Next the function NAT-XLATE is called on the last line of the
proof.  NAT-XLATE, depending on the line's justification, calls
auxiliary functions which carry out the translation, and which usually
call NAT-XLATE recursively to translate lines by which the current
line is justified.  When the justification "Same as" is found, this
indicates that the node associated with this line and the node which
is associated with the line it is the same as should be mated in the
expansion proof.

\item Example:  Suppose we have the following nproof:
\begin{verbatim}
(1) 1  !  A            Hyp
(2)    !  A implies A  Deduct: 1

SAME-IFY will construct the new proof:

(1) 1  !  A            Hyp
(2) 1  !  A            Same as: 1 
(3)    !  A implies A  Deduct: 2
\end{verbatim}

Then a leaf node LEAF0 is constructed with shallow formula 
"A implies A", and LINE-NODE-LIST is set to ((3 . LEAF0)). 
NAT-XLATE is called, and because line 3 is justified using the
deduction rule, LEAF0 is deepened to an implication node, say IMP0,
with children LEAF1 and LEAF2.  Then LINE-NODE-LIST is updated to be
((1 . LEAF1) (2 . LEAF2) (3 . IMP0)), and NAT-XLATE is called
recursively on lines 1 and 2.  Since line 1 is justified by "Hyp",
NAT-XLATE does nothing.  Since line 2 is justified by "Same as: 1",
NAT-XLATE updates the value of MATING-LIST to (("LEAF1" . "LEAF2")), a
connection consisting of the nodes which represent lines 1 and 2.

\item In an nproof that is not cut-free, there will exist lines which do
not arise from deepening the expansion tree which represents the last
line of the nproof.  Currently, NAT-XLATE will get very confused and
probably blow up.  The justification "RuleP" causes other
difficulties, because it generally requires that several connections
be made, involving lines whose nodes haven't been deepened to the
literal level yet.  The function XLATE-RULEP attempts to do this, but
does not always succeed.  This is true because RULEP can also be used
to justify a line whose node is actually a child of the justifying
line, e.g.: 
\begin{verbatim}
(45)  ! A and B 
(46)  ! A        RuleP: 45
\end{verbatim}
Though XLATE-RULEP can handle this situation, it cannot handle more
complex ones such as:
\begin{verbatim}
(16)  ! A
(17)  ! A implies B
(18)  ! B            RuleP: 16 17
\end{verbatim}
Ideally, SAME-IFY would identify these situations before the
translation process is begun, but it does not.
\end{enumerate}


\section{Cut Elimination}
A cut elimination algorithm is worked out in Frank's
Thesis.  First he defines a notion of expansion development
(a sort of sequent calculus for sets of expansion trees
with rules for quantifiers, merging, and cuts).
Then he gives reductions on expansion developments with
the hope that these reductions result in an expansion tree.
Frank's algorithm is not currently implemented as part of TPS.
Of course, there are many ways of representing cuts and performing
cut elimination.

\subsection{An Example of a Loop in a Cut Elimination Algorithm}

One approach we tried (Chad, Summer 2001)
was to include explicit
CUT and MERGE nodes in expansion trees, defining redexes,
and doing cut elimination by contracting redexes.
This section contains a brief outline of the approach
and an example that shows how a loop can occur.

A CUT node is of the form
$$\etrca{CUT}{}{C^*}{}{B^*}{}{B^{**}}$$
where $C^*$ has shallow formula $C$,
and $B^*$ and $B^{**}$ have the same shallow
formula, but opposite polarity.  The
polarity and shallow formula
of the CUT node is the same as the
polarity of the shallow formula of $C^*$.
The deep formula of the cut node is
$$deep(C^*) \land [deep(B^*) \lor deep(B^{**})]$$
A MERGE node is of the form
$$\etrba{MERGE}{}{A^*}{}{A^{**}}$$
where $A^*$ and $A^{**}$ have the same polarity
and same shallow formula $A$.  The shallow
formula of the MERGE node is also $A$.
The polarity is also inherited from the children.
The deep formula is given by
$$deep(A^*) \land deep(A^{**}).$$

With these nodes, the translation from a natural deduction proof
(see subsection \ref{ceb-nat-etr})
can now be extended to translate backward coercions.
Ignoring hypothesis for simplicity,
suppose we have a backward coercion
$$ \ianc{B\Uparrow}{B\downarrow}{bcoercion}$$
a negative expansion tree $C^*$ with shallow
formula $C$, and a positive expansion tree $B^*$
with shallow formula $B$.  By induction hypothesis,
we can obtain an appropriate positive expansion tree
$B^{**}$ with shallow formula $B$.  Then the algorithm
returns
$$\etrca{CUT}{}{C^*}{}{B^*}{}{B^{**}}$$

There is one more modification required in the algorithm.
When translating a hypothesis line, we used the merge algorithm
on expansion trees.  However, merge is not defined on expansion
trees containing CUT nodes.  So, instead we make an explicit
MERGE node with the two trees as children.  The actual merging
would be done during cut/merge elimination.  That is,  if
we are translating a hypothesis line
$$A_1,\ldots, A_n \vdash A_j,$$
then we must start with positive expansion trees $A_i^{**}$,
a positive expansion tree $A_i^{*}$,
and a negative expansion tree $C^*$.
Instead of associating the hypothesis line with the merge
of $A_i^{*}$ and $A_i^{**}$, we now associate the line
with the node
$$\etrba{MERGE}{}{A_i^{*}}{}{A_i^{**}}$$

The most interesting case is eliminating a CUT between
a selection and an expansion node.  There may be
multiple expansion nodes and the acyclicity of the dependency
relation may affect the order in which the expansion terms
are processed.  We were trying to reduce such a CUT
by doing a global merge of two modified expansion trees.
Suppose the expansion tree is $Q$ and contains a CUT node
of the form
$$\etrca{CUT}{}{C}{}{\sel{}{\forall x . A(x)}{a}{P(a):A(a)}}{}
{\gexpnode{}{\forall x . A(x)}{t_1}{P_1:A(t_1)}{t_n}{P_n:A(t_n)}}$$
Let us use the notation $Q[CUT]$ to indicate this CUT node inside the
larger tree $Q$.
Assuming we do not have a problem with acyclicity, the CUT node
should reduce as follows:
$$\etrba{MERGE}{}{\{t_1/a\} Q[CUT1]}{}{Q[CUT2]}$$
where $CUT1$ is
$$\etrca{CUT1}{}{C}{}{P(t_1):A(t_1)}{}{P_1:A(t_1)}$$
and $CUT2$ is
$$\etrca{CUT2}{}{C}{}{\sel{}{\forall x . A(x)}{a}{P:A(a)}}{}
{\gexpnode{}{\forall x . A(x)}{t_2}{P_2:A(t_2)}{t_n}{P_n:A(t_n)}}.$$

However, this leads to a loop as in the following simple example.
Consider the following proof of 
$$ \forall \,x_{\greeki} \forall \,y_{\greeki} \,A_{\greeko\greeki\greeki} \,x \,y \supset \,A \,a_{\greeki} \,a \land \,A \,b_{\greeki} \,b$$
using the lemma
$$ \forall \,z_{\greeki} \,A_{\greeko\greeki\greeki} \,z \,z.$$

\vbox{\indent
\linenumbox {(1)}\hypnumbox {1}\turnstile\partformula{$ \forall \,x_{\greeki} \forall \,y_{\greeki} \,A_{\greeko\greeki\greeki} \,x \,y$}\lastformula\judgelink{Hyp}
}\filbreak
\vbox{\indent
\linenumbox {(2)}\hypnumbox {1}\turnstile\partformula{$ \forall \,y_{\greeki} \,A_{\greeko\greeki\greeki} \,z_{\greeki} \,y$}\lastformula\judgelink{UI:\ $  \,z_{\greeki}$\  1}
}\filbreak
\vbox{\indent
\linenumbox {(3)}\hypnumbox {1}\turnstile\partformula{$ \,A_{\greeko\greeki\greeki} \,z_{\greeki} \,z$}\lastformula\judgelink{UI:\ $  \,z_{\greeki}$\  2}
}\filbreak
\vbox{\indent
\linenumbox {(4)}\hypnumbox {1}\turnstile\partformula{$ \forall \,z_{\greeki} \,A_{\greeko\greeki\greeki} \,z \,z$}\lastformula\judgelink{UGen:\ $  \,z_{\greeki}$\  3}
}\filbreak
\vbox{\indent
\linenumbox {(5)}\hypnumbox {1}\turnstile\partformula{$ \,A_{\greeko\greeki\greeki} \,a_{\greeki} \,a$}\lastformula\judgelink{UI:\ $  \,a_{\greeki}$\  4}
}\filbreak
\vbox{\indent
\linenumbox {(6)}\hypnumbox {1}\turnstile\partformula{$ \,A_{\greeko\greeki\greeki} \,b_{\greeki} \,b$}\lastformula\judgelink{UI:\ $  \,b_{\greeki}$\  4}
}\filbreak
\vbox{\indent
\linenumbox {(7)}\hypnumbox {1}\turnstile\partformula{$ \,A_{\greeko\greeki\greeki} \,a_{\greeki} \,a \land \,A \,b_{\greeki} \,b$}\lastformula\judgelink{Conj: 5 6}
}\filbreak
\vbox{\indent
\linenumbox {(8)}\hypnumbox {}\turnstile\partformula{$ \forall \,x_{\greeki} \forall \,y_{\greeki} \,A_{\greeko\greeki\greeki} \,x \,y \supset \,A \,a_{\greeki} \,a \land \,A \,b_{\greeki} \,b$}\lastformula\judgelink{Deduct: 7}
}\filbreak

When translating this proof we obtain a tree $Q$ with two cut nodes
$$\etrca{CUT1}{}{R_1}{}{\etraa{SEL1}{c}{P_1(c):A\, c\, c}}{}{\etraa{EXP1}{a}{P_2:A\, a\, a}}$$
and
$$\etrca{CUT2}{}{R_2}{}{\etraa{SEL2}{d}{P_3(d):A\, d\, d}}{}{\etraa{EXP2}{b}{P_4:A\, b\, b}}$$
corresponding to the two applications of the lemma in line 4.
Contracting $CUT1$ gives
$$\etrba{MERGE}{}{\{a/c\}Q[CUT3]}{}{Q[CUT4]}$$
where $CUT3$ and $CUT4$ are
$$\etrca{CUT3}{}{\{a/c\}R_1}{}{P_1(a):A\, a\, a}{}{P_2:A\, a\, a}$$
and
$$\etrca{CUT4}{}{R_1}{}{\etraa{SEL3}{c}{P_1(c):A\, c\, c}}{}{EXP3}.$$
Both $CUT3$ and $CUT4$ are easy to eliminate.
($CUT3$ can be replaced by $\{a/c\}R_1$ and
$CUT4$ can be replaced by $R_1$, with appropriate
changes to the complete mating.)
However, note that $CUT2$ appears in both sides of the merge in
$$\etrba{MERGE}{}{\{a/c\}Q[CUT3]}{}{Q[CUT4]}.$$
Let us call these two occurrences $CUT2.1$ and $CUT2.2$.
Eventually, we will want to reduce one of these cuts
in some reduced tree $Q'[CUT2.1][CUT2.2]$.  Suppose we
reduce $CUT2.1$.  Similar to the reduction of $CUT1$ above,
this will copy $CUT2.2$ so that there are $CUT2.2.1$ and
$CUT2.2.2$.  The loop is evident.

It is conceivable that we could get around this loop by
developing a notion of ``expansion DAG'' (directed acyclic graph)
so that the CUT would not actually be duplicated.  But it isn't
clear that this would eliminate all such loops.  

Also, there are
some technical problems with the reductions above.  For instance,
a selection node may get copied, which means the result will be
a tree with two selection nodes that use {\it the same} selected
variable.  This doesn't seem to be a serious problem because we
could probably allow such a situation whenever the least common
ancestor of two such selection nodes is a MERGE node.  But these
details would have to be worked out to make this approach work.

\subsection{Cut and Mix Elimination in this Sequent Calculus}\label{ftree-seq-mix-elim}

There is a cut elimination algorithm implemented for the
sequent calculus described in section~\ref{ftree-seq}.
Suppose we have an instance
of the cut rule:

$$\ibnc{\above{\DD_1}{\Gamma_1\rightarrow A,\Delta_1}}{\above{\DD_2}{A,\Gamma_2\rightarrow \Delta_2}}{\Gamma_1,\Gamma_2\rightarrow \Delta_1,\Delta_2}{cut}$$

A cut elimination algorithm should take cut-free derivations $\DD_1$
and $\DD_2$ and return a cut-free derivation $\EE$ of 
$\Gamma_1,\Gamma_2\rightarrow\Delta_1,\Delta_2$.
This is usually done by a recursive algorithm on the two
derivations where at each step either one of the derivations
gets smaller, or the cut formula gets smaller (while the
derivations may get much larger).  Since we have an explicit
$merge$ (contraction) rule, we have the following problematic case.
Suppose $\DD_1$ ends with an application of $merge$:
$$\ianc{\above{\DD_3}{\Gamma_1\rightarrow A,A,\Delta_1}}{\Gamma_1\rightarrow A,\Delta_1}{merge-}$$
where $A$ is the cut formula.  Also, suppose $A$ is the principal formula of $\DD_2$.
The most natural way to handle this case is to first perform cut elimination
on $\DD_3$ and $\DD_2$, giving a cut-free derivation
$$\above{\DD_5}{\Gamma_1,\Gamma_2\rightarrow A,\Delta_1,\Delta_2}$$
Then we could call cut elimination again with $\DD_5$ and $\DD_2$
to obtain a cut-free derivation
$$\above{\DD_6}{\Gamma_1,\Gamma_2,\Gamma_2\rightarrow \Delta_1,\Delta_2,\Delta_2}$$
With some applications of $focus$ and $merge$ to shuffle and merge the formulas,
we would have the cut-free derivation of
$$\above{\DD_7}{\Gamma_1,\Gamma_2\rightarrow \Delta_1,\Delta_2}$$
we desire.  {\it However,} the derivation $\DD_5$ will in general be
bigger than $\DD_3$, while the cut formula $A$ remains the same.
So, this recursive call to cut elimination may not terminate (even with
a first-order derivation). % closed-thm1.prf gives an example

We can get around this problem by performing mix-elimination 
(as did Gentzen ~\cite{Gentzen69}) instead of cut-elimination.
Because we care about the order of formulas in the sequent,
mix elimination is complicated.  The main functions are
\indexfunction{ftree-seq-mix-elim-1} and \indexfunction{ftree-seq-mix-elim-principal}.
To understand mix elimination, 
first consider a few generic examples:


Suppose we have two cut-free derivations of
$$B\rightarrow A,C$$
and
$$D,A\rightarrow E$$
where $A$ is the mix formula, in negative position $0$ in the
first sequent and positive position $1$ in the second sequent.
Mix elimination might return a cut-free derivation of
$$B,D\rightarrow C,E$$
along with two lists of indices
\begin{enumerate}
\item $((t\; . \; 0) \; (nil\; . \; 0))$ (indicating that $B$ corresponds
to the $0$'th positive formula of the first sequent, and
$D$ corresponds to the $0$'th positive formula of the second sequent)
\item $((t\; . \; 1) \; (nil\; . \; 0))$ (indicating that $C$ corresponds
to negative position $1$ in the first sequent and
$E$ corresponds to negative position $0$ in the second sequent)
\end{enumerate}
We say ``might'' because the return value depends, of course,
on the given derivations, not just the sequents.
Other possible outputs include a cut-free derivation of
$$B\rightarrow C,C$$
with two lists of indices
\begin{enumerate}
\item $((t\; . \; 0))$ (again, $B$ corresponds to positive position $0$ in
the first sequent), and
\item $((t\; . \; 1)\; (t\; . \; 1))$ (both occurrences of $C$ correspond
to negative position $1$ in the first sequent).
\end{enumerate}

The point is that we have eliminated the two mix formulas
(two occurrences of $A$) and retain only residues of the other
formulas in the two given sequents.  The other formulas may occur
several times, or not at all.  The lists of indices indicate
where the formulas originally occured.  An ``index'' is a pair
$(<bool>\; . \; <nat>)$ where
\begin{itemize}
\item If $<bool>$ is $t$, the formula is the residue of a formula
in the first sequent.
\item If $<bool>$ is $nil$, the formula is the residue of a formula
in the second sequent.
\item $<nat>$ is a natural number indicating the position of the
formula in the first or second sequent.
\end{itemize}

For another example, given two cut-free derivations of
$$B,C\rightarrow A,A,D$$
and
$$E,A,F\rightarrow G$$
with mix formulas
might return
\begin{itemize}
\item
\begin{itemize}
\item a cut-free derivation of $C,F,E\rightarrow G,D$,
\item positive indices $((t\; .\; 1)\; (nil\; .\; 2)\; (nil \; . \; 0))$,
and
\item negative indices $((nil\; . \;0)\; (t\; . \; 2))$
\end{itemize}
\item or,
\begin{itemize}
\item a cut-free derivation of $B,B\rightarrow G$,
\item positive indices $((t\; . \;0)\; (t\; . \; 0))$, and
\item negative indices $((nil \; . \; 0))$
\end{itemize}
\item or,
\begin{itemize}
\item a cut-free derivation of $F,E\rightarrow D,D$,
\item positive indices $((nil\; . \; 2)\; (nil\; . \; 0))$, and
\item negative indices $((t\; . \; 2)\; (t \; . \; 2))$.
\end{itemize}
\end{itemize}

In general, we start with two derivations
$$\above{\DD_1}{\Gamma_1\rightarrow \Delta_1}$$
and
$$\above{\DD_2}{\Gamma_2\rightarrow \Delta_2}$$
and two lists $(i_1\cdots i_k)$ and $(j_1\cdots j_l)$
of natural numbers.  The natural numbers give us
the positions of the mix formulas in $\Delta_1$ and $\Gamma_2$.
Let $\Delta_1$ be a list
$(A_1 \cdots A_n)$ and $\Gamma_2$ be a list
$(B_1 \cdots B_m)$.  The mix formulas are
$A_{i_r}$ and $B_{j_s}$.  These formulas should have
a common reduct with respect to $\lambda$-reduction,
expanding abbreviations, and expansion equalities
using either extensionality or Leibniz.
Mix elimination returns a cut-free derivation of
$$\Gamma\rightarrow\Delta$$
and two lists of $indl^+$ and $indl^-$
of indices indicating the preimage of
the formulas in $\Gamma$ and $\Delta$.
$\Gamma$ and $indl^+$ are lists of the same length.
For each $B$ in $\Gamma$ there is a corresponding
index $(b . j)$ where either $b$ is $t$
and $j$ is the position of $B$ in $\Gamma_1$
or $b$ is $nil$ and $j\not\in\{j_1,\ldots, j_l\}$ is the position of
$B$ in $\Gamma_2$.
Similarly, $\Delta$ and $indl^-$ have the same length.
For each $A$ in $\Delta$ there is a corresponding index
$(a . i)$ where either $a$ is $t$ and $i\not\in\{i_1,\ldots, i_k\}$ is the position
of $A$ in $\Delta_2$
or $a$ is $nil$ and $i$ is the position of $A$ in $\Delta_2$.

The main function which attempts to eliminate all cuts from
a derivation is \indexfunction{ftree-seq-cut-elim}.
This is a simple recursive algorithm which first eliminates
cuts from premisses, then either imitates the rule
or, if the rule was $cut$, uses mix elimination to obtain
a cut-free derivation from the cut-free derivations of the
premisses.  That is,
given two derivations
$$\above{\DD_1}{\Gamma_1\rightarrow A,\Delta_1}$$
and
$$\above{\DD_2}{A,\Gamma_2\rightarrow \Delta_2}$$
we call mix elimination with these derivations
and two lists of positions $(0)$ and $(0)$.
Mix elimination returns a cut-free derivation of
$\Gamma\rightarrow\Delta$
and two lists of indices.
Using the indices and the $focus$, $weaken$, and $merge$ rules, we can
shuffle the formulas in $\Gamma$ and $\Delta$
to obtain a derivation of
$\Gamma_1,\Gamma_2\rightarrow \Delta_1,\Delta_2$.
This final shuffling is performed by
\indexfunction{ftree-seq-mix-elim-finish}.

\subsection{The Mix Elimination Algorithm}

The mix elimination algorithm works by recursion on the two
given derivations.  Suppose we are given two cut-free derivations
$$\above{\DD_1}{\Gamma_1\rightarrow \Delta_1}$$
and
$$\above{\DD_2}{\Gamma_2\rightarrow \Delta_2}$$
and two lists $nl_1 = (i_1\cdots i_k)$ and $nl_2 = (j_1\cdots j_l)$
indicating the positions of the mix formulas in $\Delta_1$
and $\Gamma_2$.

There are several cases to consider:

\begin{itemize}
\item {\bf Mix Formula Does Not Occur}.  That is, $nl_1 = nil$ or $nl_2 = nil$.
If $nl_1$ is $nil$, then we can return
\begin{itemize}
\item $$\above{\DD_1}{\Gamma_1\rightarrow\Delta_1}$$
\item $((t\; .\; 0)\; \cdots \; (t\; . \; (n - 1)))$ where $n$ is the length of $\Gamma_1$
\item $((t\; .\; 0)\; \cdots \; (t\; . \; (m - 1)))$ where $m$ is the length of $\Delta_1$
\end{itemize}
The indices indicate that every formula is the residual of a formula from
the first sequent.
If $nl_2$ is $nil$, we can return $\DD_2$ and indices which indicate
all the formulas are residuals of the second sequent.
\item {\bf INIT}  Suppose $\DD_1$ is an initial sequent $A\rightarrow A$.
($A$ must be the mix formula, or $nl_1$ must be $nil$.)
Ideally, we would like to return $\DD_2$, and replace the indices
for the mix formula occurrences in $\Gamma_2$ by the index $(t\; .\; 0)$
indicating the positive $A$ from $\DD_1$.
The only problem is that each occurrence of the mix formula in
$\Gamma_2$ is a $B$ equal to $A$ only in the sense that the
two formulas have a common reduct with respect to
$\lambda$-reduction
and expansions of abbreviations and equalities.
The function \indexfunction{ftree-seq-replace-equivwffs} replaces
each such $B$ by $A$ in $\DD_2$.
We can handle $\DD_2$ initial similarly.
\item {\bf FOCUS, WEAKEN, MERGE}  If either $\DD_1$ or $\DD_2$ is
a $focus$, $weaken$, or $merge$ step, we can simply recursively
call with the premiss and the positions shuffled appropriately.
\item {\bf REWRITES}  If either $\DD_1$ or $\DD_2$
is a $\lambda$, $EQUIVWFFS$, $Leibniz=$, or $Ext=$ rewrite,
and the mix formula is the principal formula, we can usually simply
recursively call the algorithm with the premiss and the same positions.
This is possible because we are only maintaining that the mix formulas
are the same up to having a common reduct via such rewriting steps.
The actual rewriting is done in the INIT step.

{\bf Exception:  Nonanalytic Uses of Extensionality}  Suppose one
occurrence of the mix formula is $A[=]$
and another occurrence of the mix formula is 
$A[\forall q\; . \; q \; M_\greeko \;\limplies\; q\; N_\greeko].$
If $A[=]$ is the principal formula of an $Ext=$ rewrite, then 
$$A[\equiv]$$ and $$A[\forall q\; . \; q \; M_\greeko \;\limplies\; q\; N_\greeko]$$
{\it do not} have a common reduct.  The recursion fails at such a step.
In general this problem occurs when one instance of equality
is expanded as Leibniz and another corresponding instance is
expanded as (functional or propositional) extensionality.
The real problem is that the extensionality rules by themselves
without cut
do not give a complete calculus for extensional higher-order logic.
To get a cut-free extensional calculus, one needs to be able to
have initial sequents (i.e., ``mate'') {\em modulo equations}
and then use decomposition rules and extensionality rules to
solve the introduced equations.
Benzmuller's thesis gives rules for handling extensionality
in the context of resolution.
The case for sequent calculi should be described in upcoming technical reports by
Benzmuller, Brown and Kohlhase.
There should also be information in Chad E. Brown's thesis.
\item {\bf NONPRINCIPAL LOGICAL RULE}  If either $\DD_1$ or $\DD_2$
ends with a logical rule and the principal formula is not a mix formula,
then we can recursively call mix elimination on the premisses and
imitate the rule.  For example, if $\DD_1$ is
$$\ibnc{\above{\DD_{11}}{\Gamma_{11}\rightarrow A,\Delta_{11}}}{\above{\DD_{12}}{\Gamma_{12}\rightarrow B,\Delta_{12}}}{\Gamma_{11},\Gamma_{12}\rightarrow A\land B,\Delta_{11},\Delta_{12}}{\land-}$$
the function \indexfunction{ftree-seq-invert-position-list}
uses the positions of the mix formulas in
$$A\land B,\Delta_{11},\Delta_{12}$$
to find the positions of the mix formulas in
$\Delta_{11}$ and $\Delta_{12}$.
Using this we can call mix elimination twice to obtain
$$\above{\DD_3}{\Gamma_3\rightarrow\Delta_3}$$
and
$$\above{\DD_4}{\Gamma_4\rightarrow\Delta_4}$$
The function \indexfunction{ftree-seq-mix-elim-imitate-rule}
finishes this case.  First, \indexfunction{ftree-seq-bring-to-front} uses
structural rules to bring the residuals of $A$ and $B$ to the front of
$\Delta_3$ and $\Delta_4$, so we have
$$\above{\DD'_3}{\Gamma_3\rightarrow A,\Delta'_3}$$
and
$$\above{\DD'_4}{\Gamma_4\rightarrow B,\Delta'_4}$$
At this point, we apply the $\land-$ rule:
$$\ibnc{\above{\DD'_{3}}{\Gamma_3\rightarrow A,\Delta'_3}}{\above{\DD'_4}{\Gamma_4\rightarrow B,\Delta'_4}}{\Gamma_3,\Gamma_4\rightarrow A\land B,\Delta'_3,\Delta'_4}{\land-}$$
and compute the indices of the residuals.
\item {\bf PRINCIPAL} The final, and most important case, is when both mix formulas
are principal.  This is handled by the function
\indexfunction{ftree-seq-mix-elim-principal}.
\end{itemize}

In each of the cases above, one of the derivations gets smaller
in the recursive call.
The case when both mix formulas are principal is complicated
by the need to perform several recursive calls, requiring us
to adjust and compose the indices as we go along.  The cases
for each connective and quantifier are described next.

\begin{itemize}
\item $\top$,$\bot$:  This cannot happen, because the only
way $\top$ can be positive principal in $\DD_2$ is if the
last step is $focus$, $weaken$, or $merge$, which were handled
above.  Similarly, if $\bot$ is negative principal in $\DD_1$,
then it must end with a $focus$, $weaken$, or $merge$.
\item $REW(\equiv)$:  $\DD_1$ is
$$\ianc{\above{\DD_{11}}{\Gamma_1\rightarrow [A\limplies B]\land [B\limplies A],\Delta_1}}{\Gamma_1\rightarrow A\equiv B,\Delta_1}{REW(\equiv)-}$$
$\DD_2$ is
$$\ianc{\above{\DD_{21}}{[A\limplies B]\land [B\limplies A],\Gamma_2\rightarrow \Delta_2}}{A\equiv B,\Gamma_2\rightarrow \Delta_2}{REW(\equiv)+}$$
First we recursively call mix elimination for all the unexpanded formulas $A'\equiv B'$
with $\DD_1$ and $\DD_{21}$ (smaller than $\DD_2$)
giving $\DD_3$:
$$[A\limplies B]\land [B\limplies A],\Gamma_3\rightarrow \Delta_3$$
Next, recursively call mix elimination for all the unexpanded formulas $A'\equiv B'$ with $\DD_{11}$ (smaller than $\DD_1$) and $\DD_2$
giving $\DD_4$:
$$\Gamma_4\rightarrow [A\limplies B]\land [B\limplies A],\Delta_4$$
Finally, we can call mix elimination for the two occurrences of the
``smaller'' mix formula $[A\limplies B]\land [B\limplies A]$
with $\DD_4$ and $\DD_3$ 
giving
$\DD_5$:
$$\Gamma_5\rightarrow \Delta_5$$
Of course, with if we weigh $\equiv$ more than $\land$ and $\limplies$,
we can say $[A\limplies B]\land [B\limplies A]$ is ``smaller'' than $A\equiv B$.
So, this case does not cause a problem with termination.
The hard part is tracing the residuals to return the proper indices.
Each formula in $C$ in $\Gamma_5$ is either the residual of some
$C$ in $\Gamma_4$ or $C$ in $\Gamma_3$.  If the preimage is in $\Gamma_4$,
then $C$ is a residual of a $C$ in $\Gamma_1$ or $\Gamma_2$.  Once
we compute the preimage of the preimage, we have the proper index.

The following diagram is helpful when trying to compute preimages:
$$\ibnc{\ibnc{\DD_{11}}{\DD_2}{\DD_4}{elim}}{\ibnc{\DD_1}{\DD_{21}}{\DD_3}{elim}}{\DD_5}{elim}$$
\item $EXP-,SEL+$ or $SEL-,EXP+$:  Suppose $\DD_1$ is
$$\ianc{\above{\DD_{11}}{\Gamma_1\rightarrow A(t),\Delta_1}}{\Gamma_1\rightarrow \exists x A(x),\Delta_1}{EXP-^t}$$
and $\DD_2$ is
$$\ianc{\above{\DD_{21}}{A(a),\Gamma_2\rightarrow \Delta_2}}{\exists x A(x),\Gamma_2\rightarrow \Delta_2}{SEL+^a}$$
As described in the $REW(\equiv)$ case,
we must first eliminate all the nonprincipal occurrences
of the mix formulas by recursive calls using $\DD_1$ and $\DD_{21}$
to obtain $\DD_3(a)$:
$$A(a),\Gamma_3\rightarrow \Delta_3$$
and with $\DD_{11}$ and $\DD_2$ to obtain $\DD_4$:
$$\Gamma_4\rightarrow A(t),\Delta_4$$
Then we substitute $t$ for $a$ in $\DD_3(a)$ to obtain $\DD_3(t)$,
and recursively call to eliminate the two occurrences of
$A(t)$ using $\DD_4$ and $\DD_3(t)$.

{\bf Remark About Termination:}  Usually $A(t)$ will be ``smaller''
than $\exists x A(x)$.  But, of course, in higher order logic there
are cases where the formula is most certainly not smaller.
The most obvious example is when $A$ is $\exists \; x_\greeko \; x$
and $t$ is also $\exists \; x_\greeko\; x$, so that
$A(t)$ is actually the same as $\exists x A(x)$.
We don't actually know whether the algorithm always
terminates.

As in the $REW(\equiv)$ case,
computing the indices involves computing preimages of preimages
using the following diagram
$$\ibnc{\ibnc{\DD_{11}}{\DD_2}{\DD_4}{elim}}{\ibnc{\DD_1}{\DD_{21}}{\DD_3}{elim}}{\DD_5}{elim}$$
The $SEL-,EXP+$ case is similar.
\item $\lnot$:  This case is relatively simple, we first make two
recursive calls to eliminate the nonprincipal occurrences of the
mix formula $\lnot A$ giving $\DD_3$:
$$\Gamma_3\rightarrow A,\Delta_3$$
and $\DD_4$:
$$A,\Gamma_4\rightarrow \Delta_4$$
Finally, we eliminate the two occurrences of $A$ (smaller than $\lnot A$)
in $\DD_3$ and $\DD_4$ (the order is the opposite of the previous cases).
Again, the indices are preimages of preimages, though we must
be careful to account for the changes in lengths of the two sides
as $\lnot A$ in the final sequents and $A$ in the premisses are on opposite
sides.  The diagram that applies here is
$$\ibnc{\ibnc{\DD_1}{\DD_{21}}{\DD_3}{elim}}{\ibnc{\DD_1}{\DD_{21}}{\DD_4}{elim}}{\DD_5}{elim}$$
\item $\land$:  There are three relevant premisses here.
$\DD_1$ has the form
$$\ibnc{\above{\DD_{11}}{\Gamma_{11}\rightarrow A,\Delta_{11}}}{\above{\DD_{12}}{\Gamma_{12}\rightarrow B,\Delta_{12}}}{\Gamma_{11},\Gamma_{12}\rightarrow A\land B,\Delta_{11},\Delta_{12}}{\land-}$$
and $\DD_2$ has the form
$$\ianc{\above{\DD_{21}}{A,B,\Gamma\rightarrow \Delta}}{A\land B,\Gamma\rightarrow \Delta}{\land+}$$
Recursive calls eliminate the nonprincipal occurrences of the
mix formula.
\begin{itemize}
\item $\DD_3$: ($\DD_1$ mix $\DD_{21}$ to eliminate $A\land B$) 
$\Gamma_3\rightarrow\Delta_3$
where $\Gamma_3$ contains residuals of $A$ and $B$.
\item $\DD_4$: ($\DD_{11}$ mix $\DD_2$ to eliminate $A\land B$) $\Gamma_4\rightarrow \Delta_4$
where $\Delta_4$ contains residuals of $A$
\item $\DD_5$: ($\DD_{12}$ mix $\DD_2$ to eliminate $A\land B$) $\Gamma_5\rightarrow \Delta_5$
where $\Delta_5$ contains residuals of $B$
\item $\DD_6$: ($\DD_5$ mix $\DD_3$ to eliminate residuals of $B$)
$\Gamma_6\rightarrow\Delta_6$
where $\Gamma_6$ contains residuals of $A$
\item $\DD_7$: ($\DD_4$ mix $\DD_6$ to eliminate residuals of $A$)
$\Gamma_7\rightarrow\Delta_7$
where there are no residuals of $A$, $B$, or $A\land B$.
\end{itemize}
Since there were more recursive calls in this case, we must
compute preimages of preimages of preimages in some cases,
and preimages of preimages in other cases, as indicated by the following
diagram:
$$\ibnc{\ibnc{\DD_{11}}{\DD_2}{\DD_4}{}}
{\ibnc{\ibnc{\DD_{12}}{\DD_2}{\DD_5}{}}{\ibnc{\DD_1}{\DD_{21}}{\DD_3}{}}
{\DD_6}{}}{\DD_7}{}$$
Each wff in the sequent proven by $\DD_7$ can be traced back to
a non-mix formula in either $\DD_1$ or $\DD_2$ using this diagram.
\item $\lor$:  This case is similar to the $\land$ case, but $\DD_1$
has one premiss and $\DD_2$ has two premisses.   $\DD_1$ is
$$\ianc{\above{\DD_{11}}{\Gamma\rightarrow A,B,\Delta}}{\Gamma\rightarrow A\lor B,\Delta}{\lor-}$$
and $\DD_2$ is
$$\ibnc{\above{\DD_{21}}{A,\Gamma_1\rightarrow \Delta_1}}{\above{\DD_{22}}{B,\Gamma_2\rightarrow \Delta_2}}{A\lor B,\Gamma_1,\Gamma_2\rightarrow \Delta_1,\Delta_2}{\lor+}$$
The following diagram indicates the order of the recursive calls.
$$\ibnc{\ibnc{\ibnc{\DD_{11}}{\DD_2}{\DD_4}{}}{\ibnc{\DD_1}{\DD_{22}}{\DD_5}{}}
{\DD_6}{}}{\ibnc{\DD_1}{\DD_{21}}{\DD_3}{}}{\DD_7}{}$$
\item $\limplies$:
This is similar to the $\lor$ case.  The following diagram
indicates the order of the recursive calls.
$$\ibnc{\ibnc{\DD_1}{\DD_{21}}{\DD_3}{}}
{\ibnc{\ibnc{\DD_{11}}{\DD_2}{\DD_4}{}}{\ibnc{\DD_1}{\DD_{22}}{\DD_5}{}}{\DD_6}{}}
{\DD_7}{}$$
\end{itemize}


\section{Cut-free Extensional Sequent Derivations to Extensional Expansion Proofs}

There is an implementation of the extensional sequent calculus
in Chad E. Brown's thesis (cf.~\cite{Brown2004a}) in \TPS.
The structure {\indexother{ext-seq}} defined in \indexfile{ext-exp-dag-macros.lisp}
represents sequent derivations in this extensional sequent calculus.
Sequent derivations can be created, manipulated, saved and restored in 
the \indexother{EXT-SEQ} top level.  If a sequent derivation is cut-free,
then the command \indexcommand{CUTFREE-TO-EDAG} (implemented in \indexfile{ext-seq-top.lisp})
will translate the sequent derivation to an extensional expansion proof.
The proof that this translation works is in Chapter 7 Section 10 of Chad E. Brown's thesis (cf.~\cite{Brown2004a}).

\section{Extensional Expansion Proofs to NProofs}

\TPS can translate an extensional expansion proof to a natural deduction proof
using the code in the file \indexfile{ext-exp-dags-nd.lisp}.
This code is automatically called when either of the extensional search procedures
\indexother{MS04-2} or \indexother{MS03-7} successful find a proof.  The translation code can also be explicitly
called from the \indexother{EXT-MATE} top level using the command \indexcommand{ETREE-NAT}.

There is an algorithm described in Chapter 7 Section 9 of Chad E. Brown's thesis (cf.~\cite{Brown2004a})
which translates extensional expansion proofs to extensional sequent derivations.
This algorithm is not implemented as part of \TPS, but the same ideas are used for
the algorithm translating from extensional expansion proofs to natural deduction proofs.


\chapter{Library}

% @comment(the old library documentation, which is misleading, is in library.mss)

The library commands are documented in the user manual.

A library can currently only occupy one directory (i.e. subdirectories
may not be used), although users are given the ability to refer additionally to a
common directory of basic definitions by using the \indexflag{BACKUP-LIB-DIR} flag.

Many library commands are essentially written as two copies of the same function, 
the first of which checks the default library directory and the second of which
checks the backup directory. The second piece of code is surrounded by {\tt unwind-protect}
commands in order to make sure that the \indexflag{DEFAULT-LIB-DIR} and \indexflag{BACKUP-LIB-DIR}
flags always end up correct. Users may not write to the backup directory.

The index for each library directory is stored in the {\it libindex.rec} file in that directory;
this file that is read every time the directory is changed. Objects are removed from the library 
by deleting them from the appropriate {\it .lib} file and removing their entry from the {\it libindex.rec}
file. This may result in a {\it .lib} file of zero length, in which case the file is deleted.

Objects which are loaded by the user are re-defined as \TPS objects; library objects of type 
MODE or MODE1 become \TPS modes, whereas gwffs and abbreviations each become both theorems
(of type {\it library}) and abbreviations. Notice that this blurs the distinction between a gwff and an 
abbreviation. Users are allowed to redefine \TPS theorems of type {\it library}, and their corresponding
abbreviations; theorems of other types may not be redefined (this is to prevent users from accidentally
overwriting standard abbreviations with their own library definitions).

Library definitions are parsed every time they are written, and this involves re-loading all of the
needed objects. Since the needed objects are often abbreviations, this will frequently result in 
their redefinition, and so \lisp will generate warning messages. If a large file is being re-parsed,
this can take a long time and produce a huge number of warnings.

\section{Converting TPTP Problems to \TPS~library items}

Every needed function or flag is set in {\it library2.lisp}. The utility is made of
two library commands: \indexcommand{INSERT-TPTP}, to insert one TPTP problem into \TPS, and
\indexcommand{INSERT-TPTP*} to automatically call insert-tptp on an entire directory. These
two commands act on .tps files, which are generated using the TPTP2X utility. One flag, \indexflag{INSERT-LIB-DIR}, 
defines the output directory for the newly created items.

Please note that a modified format.tps file is used, in order to prevent
conflit with other objects inside of tps. The original file uses 'const', 'def',
'axiom' and 'thm' as functions: they have been replaced by 'const-insert',
'def-insert', 'axiom-insert' and 'thm-insert'.

The principal issue when converting TPTP problems into TPS items is to avoid
using already defined objects. For this reason, every inserted item is
suffixed, usually with the name of the destination library file (e.g. 'one'
becomes 'one-ALG2684').

More information about how to process this conversion are available in the User's guide.
\chapter{Teaching Records}\label{Teach}

\section{Events in TPS3}

The primary purpose of events in \TPS is to collect information about
the usage of the system.  That includes support of features such as
automatic grading of exercises and keeping statistics on the application of
inference rules.

Events, once defined and initialized, can be signalled from anywhere in
\tps.  Settings of flags, ordinarily collected into modes, control if,
when, and where signalled events are recorded. Siganlling of events can 
be suppressed by changing the values of the flags in subject \indexother{EVENTS}.
Notice that this, of course, only suppresses the signalling, not the events
themselves!

In \ETPS, a basic set of events is predefined, and the events are signalled
automatically whenever appropriate.  If these events are then recorded
depends on your \ETPS profile.

There are some restrictions on events that should be respected, if
you plan to use {\tt REPORT} to extract statistics from the files recording
events.  Most importantly: {\bf No two events should be written to the
same file.}  If you would like to record different things into the
same file, make one event with one template and allow several kinds of
occurrences of the event.  For an example, see the event {\tt PROOF-ACTION}
below.

\subsection{Defining an Event}

If you are using \ETPS, it is unlikely that you need to define an event
yourself.  However, a lot of general information about events is given
in the following description.

Events are defined using the {\tt DEFEVENT} macro.
Its format is

\begin{verbatim}
(defevent <name>
  (event-args <arg1> ... <argn>)
  (template <list>)
  (template-names <list>)
  (signal-hook <hook-function>)
  (write-when <write-when>)
  (write-file <file-parameter>)
  (write-hook <hook-function>)
  (mhelp <help-string>))
\end{verbatim}

\begin{description}

\item [{\tt event-args}]  list of arguments passed on by {\tt SIGNAL-EVENT} for any event
	of this kind.

\item [{\tt template}]  constructs the list to be written.  
        It is not assumed that every event is
	time-stamped or has the user-id.  The template
        must only contain globally evaluable forms and the arguments
	of the particular event signalled.  It could be the source of
        subtle bugs, if some variables are not declared special.

\item [{\tt template-names}]  names for the individual entries in the template.
These names are used by the {\tt REPORT} facility.  As general conventions,
when the template form is a variable, use the same name for the
template name (e.g. {\tt DPROOF}).  If the template form is {\tt (STATUS {\it statusfn})}
use {\it statusfn} as the template name (e.g. {\tt DATE} for {\tt (STATUS DATE)} or
{\tt USERID} for {\tt (STATUS USERID)}).

\item [{\tt signal-hook}]  an optional function to be called whenever the
	the event is signalled.  This should {\bf not} to the writing of
	the information, but may be used to do something else.  If the
        function does a {\tt THROWFAIL}, the calling {\tt SIGNAL-EVENT} will
        return {\tt NIL}, which means failure of the event.  The arguments
        of the function should be the same as {\tt EVENT-ARGS}.

\item [{\tt write-when}]  one of {\tt IMMEDIATE}, {\tt NEVER}, or an integer {\it n}, which means
     to write after an implementation depended period of {\it n}.
     At the moment this will write, whenever the number of inputs = {\it n}
     * {\tt EVENT-CYCLE}, where {\tt EVENT-CYCLE} is a global variable, say 5.

\item [{\tt write-file}]  the name of the global {\tt FLAG} with the filename of the
     file for the message to be appended to.

\item [{\tt write-hook}]  an optional function to be called whenever a number
	(>0) of events are written.  Its first argument is the file it will
        write to, if the write-hook returns.  Its second argument is the
        list of evaluated templates to be written.  If an event is to be
        written immediately, this will always be a list of length 1.

\item [{\tt mhelp}]  The mhelp string for the event.
\end{description}

Remember that an event is ignored, until {\tt (INIT-EVENTS)} or {\tt (INIT-EVENT
{\it event})} has been called.

\subsection{Signalling Events}

\TPS provides a function {\tt SIGNAL-EVENT}, which takes a variable number
of arguments.  The first argument is the kind of event to be signalled,
the rest of the arguments are the event-args for this particular event.
{\tt SIGNAL-EVENT} will return {\tt T} or {\tt NIL}, depending on whether the action
to be taken in case of the event was successful or not.  Note that when
an event is disabled (see below), signalling the event will always be
successful.  There are basically three cases in which an event will be
considered unsuccessful: if the {\tt SIGNAL-HOOK} is specified and does a
{\tt THROWFAIL}, if {\tt WRITE-WHEN} is {\tt IMMEDIATE} and either the {\tt WRITE-HOOK}
(if specify) does a {\tt THROWFAIL}, or if for some reason the writing to
the file fails (if the file does not exists, or is not accessible
because it has the wrong protection, for example).

It is the caller's responsibility to make use of the returned value of
{\tt SIGNAL-EVENT}.  For example, the signalling of {\tt DONE-EXERCISE} below.

If {\tt WRITE-WHEN} is a number, the evaluated templates will be collected
into a list {\it event{\tt -LIST}}.  This list is periodically written out and
cleared.  The interval is determined by {\tt EVENT-CYCLE}, a global flag
(see description of {\tt WRITE-WHEN} above).  The list is also written out
when the function {\tt EXIT} is called, but not if the user exits \TPS with
{\tt $\hat{}$C}.  Note that if events have been signalled, the writing is done
without considering whether the event is disabled or not.  This ensures
that events signalled are always recorded, except for the {\tt $\hat{}$C} safety valve.

Events may be disabled, which means that signalling them will always
be successful, but will not lead to a recordable entry.  This is done
by setting or binding the flag {\it event{\tt -ENABLED}} to {\tt NIL} (initially
set to {\tt T}).  For example, the line {\tt (setq error-enabled nil)} 
in your {\tt .INI} file will make sure that no MacLisp error will be recorded.
For a maintainer using expert mode, this is probably a good idea.

\subsection{Examples}

Here are some examples take from the file {\tt ETPS-EVENTS}.  Interspersed
is also the code from the places where the events are signalled.

%\begin{tpsexample}
\begin{verbatim}

(defflag error-file
  (flagtype filespec)
  (default "etps3.error")
  (subjects events)
  (mhelp "The file recording the events of errors."))

(defevent error
  (event-args error-args)
  (template ((status-userid) error-args))
  (template-names (userid error-args))
  (write-when immediate)
  (write-file error-file)    ; a global variable, eg
			     ; `((tpsrec: *) etps error)
  (signal-hook count-errors) ; count errors to avoid infinite loops
  (mhelp "The event of a Lisp Error."))
\end{verbatim}
{\tt DT} is used to freeze the daytime upon invocation of {\tt DONE-EXC} so that
the code is computed correctly.  The code is computed by {\tt CODE-LIST},
implementing some ``trap-door function''.

\begin{verbatim}
(defvar computed-code 0)

(defvar dt '(0 0 0)) 

(defvar score-file)
(defflag score-file
  (flagtype filespec)
  (default "etps3.scores")
  (subjects events)
  (mhelp "The file recording completed exercises."))

(defevent done-exc
  (event-args numberoflines)
  (template ((status-userid) dproof numberoflines computed-code
			     (status-date) dt))
  (template-names (userid dproof numberoflines computed-code date daytime))
  (signal-hook done-exc-hook)
  (write-when immediate)
  (write-file score-file)
  (mhelp "The event of completing an exercise."))

(defun done-exc-hook (numberoflines)
  ;; The done-exc-hook will compute the code written to the file.
  ;; Freeze the time of day right now.
  (declare (special numberoflines))
  ;; because of the (eval `(list ..)) below.
  (setq dt (status-daytime))
  (setq computed-code 0)
  (setq computed-code (code-list (eval `(list ,@(get 'done-exc 'template))))))

(defflag proof-file
  (flagtype filespec)
  (default "etps3.proof")
  (subjects events)
  (mhelp "The file recording started and completed proofs."))

(defevent proof-action
  (event-args kind)
  (template ((status-userid) kind dproof (status-date) (status-daytime)))
  (template-names (userid kind dproof date daytime))
  (write-when immediate)
  (write-file proof-file)
  (mhelp "The event of completing any proof."))

(defflag advice-file
  (flagtype filespec)
  (default "etps3.advice")
  (subjects events)
  (mhelp "The file recording advice."))

(defevent advice-asked
  (event-args hint-p)
  (template ((status-userid) dproof hint-p))
  (template-names (userid dproof hint-p))
  (write-when 1)
  (write-file advice-file)
  (mhelp "Event of user asking for advice."))

\end{verbatim}
%\end{tpsexample}

Here is how the {\tt DONE-EXC} and {\tt PROOF-ACTION} are used in the code of
the {\tt DONE} command.  We don't care if the {\tt PROOF-ACTION} was successful
(it will usually be), but it's very important that the user knows
when a {\tt DONE-EXC} was unsuccessful, since it is used for automatic
grading.

%\begin{tpsexample}
\begin{verbatim}
(defun done ()
  ...
  (if (funcall (get 'exercise 'testfn) dproof)
      (do ()
	  ((signal-event 'done-exc (length (get dproof 'lines)))
	   (msgf "Score file updated."))
	(msgf "Could not write score file.  Trying again ... (abort with ^G)")
	(sleep 1/2))
      (msgf "You have completed the proof.  Since this is not an assigned exercise,"
	    t "the score file will not be updated."))
  (signal-event 'proof-action 'done))
\end{verbatim}
%\end{tpsexample}

\section{The Report Package}

The \indexother{REPORT} package in \TPS allows the processing of data
from EVENTS. Each report draws on a single event, reading
its data from the record-file of that event. The execution
of a report begins with its BEGIN-FN being run. Then 
the DO-FN is called repetitively on the value of the EVENTARGS
in each record from the record-file of the event, until that
file is exhausted or the special variable DO-STOP is given a non-NIL
value. Finally, the END-FN is called. The arguments
for the report command are given to the BEGIN-FN and END-FN.
The DO-FN can only access these values if they are assigned to
certain PASSED-ARGS, in the BEGIN-FN. Also, all updated values
which need to be used by later iterations of the DO-FN or by
the END-FN should be PASSED-ARGS initialized (if the default NIL
is not acceptable in the BEGIN-FN.

NOTE: The names of PASSED-ARGS should be different from
other arguments (ARGNAMES and EVENTARGS). Also, they should
be different from other variables in those functions where
you use them and from the variables which DEFREPORT always 
introduces into the function for the report: FILE, INP and DO-STOP.

The definition of the category of REPORTCMD, follows:

%\begin{LispCode}
\begin{verbatim}
(defcategory reportcmd
  (define defreport1)
  (properties 
   (source-event single)
   (eventargs multiple)   ;; selected variables in the var-template of event
   (argnames multiple)
   (argtypes multiple)
   (arghelp multiple)
   (passed-args multiple) ;; values needed by DO-FN (init in BEGIN-FN)
   (defaultfns multiplefns)
   (begin-fn singlefn)    ;; args = argnames    
   (do-fn singlefn)       ;; args = eventargs ;; special = passed-args
   (end-fn singlefn)      ;; args = argnames
   (mhelp single))
  (global-list global-reportlist)
  (mhelp-line "report")
  (mhelp-fn princ-mhelp)
  (cat-help "A task to be done by REPORT."))

\end{verbatim}
%\end{LispCode}

	The creation of a new report consists of a DEFREPORT statement
(\indexother{DEFREPORT} is a macro that invokes \indexother{DEFREPORT1})
and the definition of the BEGIN-FN, DO-FN and END-FN. Any PASSED-ARGS
used in these functions should be declared special. It is suggested
that most of the computation be done by general functions which are more
readily usable by other reports. In keeping with this philosophy,
the report EXER-TABLE uses the general function MAKE-TABLE. The latter
takes three arguments as input:  a list of column-indices, a list of
indexed entries (row-index, column-index, entry) and the maximum printing size
of row-indices. With these, it produces a table of the entries.
EXER-TABLE merely calls this on data it extracts from the record file
for the DONE-EXC event. The definition for EXER-TABLE follows:

%\begin{LispCode}
\begin{verbatim}
(defreport exer-table
  (source-event done-exc)
  (eventargs userid dproof numberoflines date)
  (argtypes date)
  (argnames since)
  (defaultfns (lambda (since)
		(cond ((eq since '$) (setq since since-default)))
		(list since-default)))
  (passed-args since1 bin exerlis maxnam)
  (begin-fn exertable-beg)
  (do-fn exertable-do)
  (end-fn exertable-end)
  (mhelp "Constructs table of student performance."))

(defun exertable-beg (since)
  (declare (special since1 maxnam))	;the only non-Nil passed-args
  (setq since1 since)
  (setq maxnam 1))

(defun exertable-do (userid dproof numberoflines date)
  (declare (special since1 bin exerlis maxnam))
  (if (greatdate date since1)
      (progn
       (setq bin (cons (list userid dproof numberoflines) bin))
       (setq exerlis 
	     (if (member dproof exerlis) exerlis (cons dproof exerlis)))    
       (setq maxnam (max (flatc userid) maxnam)))))

(defun exertable-end (since)
  (declare (special bin exerlis maxnam))
  (if bin 
      (progn
       (make-table exerlis bin maxnam)
       (msg t "On exercises completed since ")
       (write-date since)
       (msg "." t))
      (progn
       (msg t "No exercises completed since ")
       (write-date since)
       (msg "." t))))

\end{verbatim}
%\end{LispCode}


\chapter{The Grader Program}

(Programmers should be aware that the GRADER program has its own manual.)

\section{The Startup Switch}

In theory, adding the switch {\tt -grader} to the command line which 
starts up \TPS should start up the Grader program directly. The code which 
implements this is in \indexfile{tps3-save.lisp}.

In practice, some modifications may be needed depending on the particular 
Lisp being used. For example:

\begin{itemize}
\item When starting up in CMUlisp on an IBM RT, the error {\tt "Switch does not exist"}
will be given. This is just Lisp complaining that it doesn't recognize the switch;
it passes the switch on to \TPS anyway, so this is no cause for concern.

\item When using Allegro Lisp version 4.1 or later, a {\tt --} symbol is used to separate
Lisp options from user options. So, on early versions of Allegro Lisp the line to
start up grader is:
{\tt xterm {\it <many xterm switches>} -e /usr/theorem/bin/run-tps -grader \&}
whereas for later versions it is:
{\tt xterm {\it <many xterm switches>} -e /usr/theorem/bin/run-tps -- -grader \&}
\end{itemize}

\chapter{Running TPS With An Interface}

There is an interface for \TPS written in Java.
Running \TPS through such an interface is similar to running
\TPS within an xterm window, except the Java interface
supports menus and popup prompts.  The \TPS lisp code
now includes general facilities for communicating with
such an interface (when running under Allegro).

To start \TPS with the java interface, one can use the command line
argument 
\indexother{-javainterface} along with other relevant information
as shown below:
\begin{verbatim}
lisp -I tps3.dxl -- -javainterface cd javafiles \;
     /usr/bin/java TpsStart
\end{verbatim}
The command line arguments following -javainterface
should form a shell command which run the interface.
In this case, the shell command would be ``cd javafiles; /usr/bin/java TpsStart''.
Other command line arguments which have meaning for the ``java TpsStart'' command
are listed below.
\begin{description}
\item[\indexother{-big}]  Use the bigger sized fonts.
\item[\indexother{-x2}]  Multiply the font size by 2.
\item[\indexother{-x4}]  Multiply the font size by 4.
\item[\indexother{-nopopups}]  Do not use ``popup'' style prompts.
  Instead, the Java window should behave more like the x-window interface.
\end{description}
The remaining command line arguments should be followed by a non-negative integer.
\begin{description}
\item[\indexother{-screenx}]  The initial horizontal size of the Java window.
\item[\indexother{-screeny}]  The initial vertical size of the Java window.
\item[\indexother{-rightOffset}]  The amount of extra room given to the right margin.
\item[\indexother{-bottomOffset}]  The amount of extra room given to the bottom margin.
\item[\indexother{-maxChars}]  The maximum number of characters to hold in the buffer.
  This should be large enough that you can scroll back and see previous \TPS output.
  The default value of 20000 should usually be enough.
\end{description}

These other arguments should be preceeded by a command line ``-other''.
This tells TPS that the remaining command line information should be passed
to the call to ``java TpsStart''.  For example,
\begin{verbatim}
lisp -I tps3.dxl -- -javainterface cd javafiles \;
   /usr/bin/java TpsStart -other -big -rightOffset 10 -nopopups
\end{verbatim}
which tells
To send these command line arguments to ``java TpsStart'', they should
For example, ``java TpsStart -big'' instructs Java to use the big fonts,
and ``-x2'' and ``-x4'' instruct Java to multiply the size of the fonts
by 2 or 4, respectively.  The extra argument ``-nopopups'' will
provide an alternative to popup prompts.

Another way to use the java interface is to start \TPS as usual,
then use the command \indexcommand{JAVAWIN}.  This requires the
flag \indexflag{JAVA-COMM} to be set appropriately.

When \TPS is started in -javaservice mode, it uses the rest of the
command line arguments to start the Java interface and creates
sockets connecting \TPS to the Java interface.  Two processes
are spawned, one to receive input from the Java interface
(either from a prompt or from a menu item selection), and
another to actually run \TPS commands.

The rest of the description does not particularly depend on
the Java interface, so I will simply refer to ``the interface''
and attempt to emphasize that such an interface could be
implemented in a variety of ways.

The code to receive input from the interface is written
in \indexcommand{external-interface.lisp}.  It listens to the
socket stream and collects characters into strings
separated by null characters (ASCII 0).  
There are a few possibilities.
\begin{enumerate}
\item If the string ''COMMAND'' is received,
the next string is a command \TPS should run
(or the response to a prompt if popups are disabled).
All the input does with this command string is 
attach it to the \indexother{COMMAND} symbol
as the property \indexother{RESPONSE}.
The main process will accept this string as input
from linereadp since, when running through the interface,
a function \indexfunction{read-line-sym} will wait
for this \indexother{RESPONSE} property to be set.
An exceptional case is when the string after
``COMMAND'' is ``INTERRUPT''.  In this case,
the main process is killed and a new process
with a top level prompt is created.
\item If the string ``RIGHTMARGIN'' is received,
the next string received should be an integer giving
the new value for the flag \indexflag{RIGHTMARGIN}.
This allows the interface to change this flag
without having to interrupt another command that
may be running.
\item When popups are enabled, some other string starting with ``PROMPT''
is received.  In this case, the next string is
put on the \indexother{RESPONSE} property
of this ``PROMPT'' symbol.  This should be
a response to a particular (popup) prompt.
\end{enumerate}

So, the code in \indexfile{external-interface.lisp} handles
receiving input from the interface.  The other
problem is that of sending output to the interface.
This is handled by setting \indexother{*standard-output*}
to the socket stream and changing the \indexflag{STYLE}
to \indexother{ISTYLE} (``interface style'' defined in
\indexfile{interface-style.lisp}).  This style
is similar to the \indexother{XTERM} style, except with
more control information.  Control information is sent
by first sending a null character (ASCII 0) followed
by a byte giving information.  The current possible byte
values following a null character and their meanings
are listed below.  There are lisp functions in
\indexfile{interface-style.lisp} which send these bytes,
but anyone coding a new interface will need to know these values. 
\begin{enumerate}
\item [0] Switch to normal font mode.  In normal font mode,
  each character is communicated by a single byte.
\item [1] Switch to symbol font mode.  In symbol font mode,
  each symbol character is communicated by two bytes (allowing
  for many more symbol characters than normal font characters).
\item [2] Start a prompt message string.
\item [3] Start a prompt name string.
\item [4] Start a prompt argtyp string.  (This allows the interface to
recognize some special finite argtyp's such as boolean.)
\item [5] Start a list of prompt-options
\item [6] Start a list giving the prompt's default value.
\item [7] Start a prompt help string.
\item [8] End a prompt.
\item [9] A note that a command has finished executing.
\item [10] Start and end a string specifying the current top level.
\item [11] Open a window (eg, for proof windows or vpforms)
and start sending a string giving the
port value for a socket to connect to.
\item [12] End a string giving a prompt for a window and start sending
a string given a title for the window.
\item [13] End the title of a window and start sending a string
giving the width of the window.
\item [14] End sending the width of the window and start sending
the height of the window.
\item [15] End sending window information for a window with small fonts.
\item [16] End sending window information for a window with big fonts.
\item [17] Clear the contents of a window.
\item [18] Close a window.
\item [19] Change the color.  This should be followed by another
byte to indicate the color.  For now, this third byte can be
0 (black), 1 (red), 2 (blue), or 3 (green).
\end{enumerate}

\section{Generating the Java Menus}

There are two categories in the \TPS lisp code for menus:
\indexother{menu} and \indexother{menuitem}.  Everytime
a programmer adds a command (mexpr), flag, or top level
command, a corresponding menuitem should be defined.
This menuitem should have a parent menu to indicate where
the item lives.

The Java menu code is in the file \indexfile{TpsWin.java}
between the comment lines:
\begin{verbatim}
    // Menu Code BEGIN
\end{verbatim}
and
\begin{verbatim}
    // Menu Code END
\end{verbatim}
When you have added or changed menus or menuitems in
the lisp code and want the Java interface to reflect these
changes, perform the following steps:
\begin{enumerate}
\item Within \TPS, call the command \indexcommand{generate-java-menus}.
This will prompt for an output file, e.g., ``menus.java''.
This command will create an output file with Java code which should
be inserted into \indexfile{TpsWin.java}.
\item Delete the code between
\begin{verbatim}
    // Menu Code BEGIN
\end{verbatim}
and
\begin{verbatim}
    // Menu Code END
\end{verbatim}
in \indexfile{TpsWin.java}.
\item Insert the contents of the output file of \indexcommand{generate-java-menus}
(e.g., ``menus.java'') into \indexfile{TpsWin.java} between the comment lines
\begin{verbatim}
    // Menu Code BEGIN
\end{verbatim}
and
\begin{verbatim}
    // Menu Code END
\end{verbatim}
and save \indexfile{TpsWin.java}.
\item On each machine, find the Java directories which contains links
to the main java files (e.g., /home/theorem/tps/java/ and /home/theorem/tps/java/tps/).
cd to this
directory and call ``javac TpsWin.java'' to compile the new version.
\end{enumerate}

\section{Adding a New Symbol}

The Java information for the fonts is contained in the files \indexfile{TpsSmallFonts.java}
and \indexfile{TpsBigFonts.java}.  The lisp information containing the ``code'' for
the symbol is in \indexfile{interface-style.lisp}.
To add a new symbol for the Java interface,
one should add a new ``code'' for the symbol to the variable \indexother{istyle-characters}
in \indexfile{interface-style.lisp}.
For example, the epsilon character was added by including
\begin{verbatim}
    (epsilon 2 1)
\end{verbatim}
to \indexother{istyle-characters}.  This means that epsilon will be communicated
to the interface by sending the bytes 2 and 1 in symbol font mode.

Then one needs to add information about how to draw the new symbol to
the variables \verb+blank+, \verb+xstartData+, \verb+ystartData+, \verb+widthData+, 
and \verb+heightData+
in \indexfile{TpsSmallFonts.java} and \indexfile{TpsBigFonts.java}.
Each of these variables is set to a multi-dimensional array.
The 0'th element of each array corresponds to the normal fonts.
The rest are for symbol fonts.
For example, the information for epsilon should be put 
in the (2,1) position of each array.  This information (for epsilon
in \indexfile{TpsSmallFonts.java})
is as follows:
\begin{description}
  \item[blank] false (the epsilon character is not blank).
  \item[xstartData] {3,2,1,1,1,2,3} (This character is drawn using 7 rectangles starting
    from ``x'' coordinates 3, 2, 1, 1, 1, 2, 3, resp.)
  \item[ystartData] {10,9,8,7,6,5,4} (These are the ``y'' coordinates of the 7 rectangles.)
  \item[widthData] {4,1,1,4,1,1,4} (These are the widths of the 7 rectangles.)
  \item[heightData] {1,1,1,1,1,1,1} (These are the heights of the 7 rectangles.
    Since all heights are 1, the epsilon character is drawn by drawing 7 horizontal lines.)
\end{description}








\backmatter

\bibliography{logictex} %requires BIBINPUTS to point to $tps/doc/lib
\printindex

\end{document}

