\chapter{Representing Well-formed formulae}

\section{Types}

\begin{description}
\item [\indextypes{\it typeconstant} ] ::= Type Constant

An identifier with a non-{\tt NIL} {\tt TypeConst} property.
For example, {\tt O} and {\tt I}:
%\begin{tpsexample}
\begin{verbatim}
(def-typeconst o
  (mhelp "The type of truth values."))
\end{verbatim}
%\end{tpsexample}

\item [\indextypes{\it typevariable} ] ::= Type Variable

 An identifier with a non-{\tt NIL} {\tt TypeVar} property.
\end{description}
It is the parsers responsibility to give the {\tt TypeVar} property to types 
not previously encountered.

\begin{description}
\item [\indextypes{\it typesymbol} ] ::= {\it typeconstant} | {\it typevariable} | 
{\tt ({\it typesymbol} . {\it typesymbol})}
\end{description}

\section{Terminal Objects of the Syntax}\label{terminalobjects}

Before going into detail about the terminal objects of the syntax,
some general remarks about type polymorphism in \TPS are needed.

\TPS supports polymorphic objects, like $\subseteq$ (subset), which is
a relation that may hold between sets of any type.  It must be understood,
however, that the parser completely eliminates this ambiguity of types,
when actually reading a wff.  In a given wff every proper subwff has
a type!  Therefore, there is a class of objects with polymorphic type,
which never appear in a wff, but nevertheless may be typed by the user.
The instances of those polymorphic abbreviations or polymorphic proper
symbols inside the formula will refer, however, to those polymorphic primitive
symbols or polymorphic abbreviating symbols.

For reasons of efficiency, binders are handled slightly differently.
Binders are also polymorphic in the sense that a certain binder, say
$\forall$, may bind variables of any type.  The case of binder, however,
is slightly different from that of polymorphic abbreviations, since
a binder is not a proper subwff.   Binders, therefore, are left without
having a proper type.  We must, however, be able to figure out the type
of any given bound wff.  Thus each binder carries the information
about the type of the scope, the bound variable and the resulting bound
wff with it.  See Section ~\ref{Binders} for more details.

The list below introduces syntactic categories of objects known to the
parser only, which are not legal in wffs themselves.

\begin{description}
\item [\indexSyntax{pmprsym} ] ::= Polymorphic Primitive Symbol

 {\it pmprsyms} are the {\tt STANDS-FOR} property of 
{\it pmpropsyms}, but cannot appear in {\it gwffs} themselves.  
Examples would be {\tt PI} or {\tt IOTA}.

\item [\indexSyntax{pmabbsym} ] ::= Polymorphic Abbreviating Symbol

 \indexSyntax{pmabbsym} are the {\tt STANDS-FOR} property of 
{\it pmabbrevs}, but cannot appear in {\it gwffs} themselves. 
Examples are {\tt SUBSET}, {\tt UNION}, or {\tt IMAGE}.

\end{description}

The following categories are the ``terminal'' objects of proper wffs.
The parser may not produce a formula with any other atomic (in the 
Lisp sense) object then from the list below.

\begin{description}
\item [\indexSyntax{logconst} ] ::= Logical Constants

 For example: {\tt AND,} {\tt OR,} {\tt IMPLIES,} {\tt NOT,} {\tt FALSEHOOD,} {\tt TRUTH}:
%\begin{tpsexample}
\begin{verbatim}
(def-logconst and
   (type "OOO")
   (printnotype t)
   (infix 5)
   (prt-associative t)
   (fo-single-symbol and)
   (mhelp "Denotes conjunction."))
\end{verbatim}
%\end{tpsexample}

\item [\indexSyntax{propsym} ] ::= Proper Symbols

 For example: {\tt P<OA>}, {\tt x<A>}, {\tt y<A>}, {\tt Q<OB>}, {\tt x<B>} are
proper symbols after parsing
$ \forall \,x \forall \,y . \,P_{\greeko\greeka\greeka} \,x \,y \land \,Q_{\greeko\greeka} \,x$.
This example
demonstrates part of the parser.  Since in a given wff, a proper symbol
may appear with more than one type, the type of each proper must somehow
be encoded in its name.  \TPS does this by appending the type, {\tt (} and
{\tt )} replaced by {\tt <} and {\tt >}, respectively, to the print name of the
symbol.

\item [\indexSyntax{pmpropsym} ] ::= Polymorphic Proper Symbols

 These are just like \indexSyntax{propsym}, except that they also have 
a {\tt STANDS-FOR} property, which is the polymorphic primitive symbol
(\indexSyntax{pmprsym}) this polymorphic proper symbol was constructed
from.  Note that this particular instance of the polymorphic primitive
symbol always has a specific given type.  For example: {\tt IOTA<I<OI>>}
is a pmpropsym after parsing $y_{\greeki} = \iota [QQ y]$:
%\begin{tpsexample}
\begin{verbatim}
(def-pmpropsym iota
  (type "A(OA)")
  (typelist ("A"))
  (printnotype t)
  (fo-single-symbol iota)
  (mhelp "Description operator"))
\end{verbatim}
%\end{tpsexample}

\item [\indexSyntax{abbrev} ] ::= Abbreviations

 For example: @EQUIV.  This is separate category from polymorphic
abbreviations only for reasons of efficiency.  An abbreviation could
be thought of as a polymorphic abbreviation with an empty list of
type variables. For example:
%\begin{tpsexample}
\begin{verbatim}
(def-abbrev equiv
  (type "OOO")
  (printnotype t)
  (fo-single-symbol equiv)
  (infix 2)
  (defn "[=(OOO)]"))
\end{verbatim}
%\end{tpsexample}

\item [\indexSyntax{pmabbrev} ] ::= Polymorphic Abbreviations

 For example: {\tt SUBSET<O<OA><OA>>}, {\tt SUBSET<O<OB><OB>>} are polymorphic
abbreviations after parsing {\wt A@f12(oa) @SUBSET B @or [R@f12(obb) a] @~
@SUBSET [R b]}. For example:
%\begin{tpsexample}
\begin{verbatim}
(def-abbrev subset
   (type "O(OA)(OA)")
   (typelist ("A"))
   (printnotype t)
   (infix 8)
   (fo-single-symbol subset)
   (defn "lambda P(OA) lambda R(OA). forall x . P x implies R x"))
\end{verbatim}
%\end{tpsexample}

\item [\indexSyntax{binder} ] ::=  Variable Binders

For example: $\forall$, $\exists$, $\lambda$, $\exists_1$. See the section below.

\item [\indexSyntax{label} ] ::= A Label referring to one or more other wffs.

 For example: {\tt AXIOM1}, {\tt ATM15}, {\tt LABEL6}.  See Section ~\ref{labels}.
\end{description}

In principle, the implementation is completely free to choose the
representation of the different terminal objects of the syntax.
The functions with test whether a given terminal object is of a given
kind is the only user visible functions.  Once defined, the particular
implementation of the object should not be needed or relied upon
by other functions.

It is explained more precisely what is meant by ``quick'' and ``slow''
predicates to decide whether a given object is in a certain syntactic
category in section ~\ref{quickslow}.  Here is a table of the different
syntactic categories with the ``slow'' test function for it and
the properties that are required or must be absent.
Keep in mind that the list reflects the current implementation, and
may not be reliable.

%\begin{Format, Group}
%@TabSet(1inch,2.25inch,4inch)
\begin{tabular}{lll}
Category & Predicate & Required Properties \\
& & Absent Properties \\
\\
{\it pmprsym} & {\tt PMPRSYM-P} & {\tt TYPE}, {\tt TYPELIST} \\ & & {\tt DEFN} \\
{\it pmabbsym} & {\tt PMABBSYM-P} & {\tt TYPE}, {\tt TYPELIST}, {\tt DEFN} \\
 \\
{\it logconst} & {\tt LOGCONST-P} & {\tt TYPE}, {\tt LOGCONST} \\
{\it propsym} & {\tt PROPSYM-P} & {\tt TYPE} \\ & & {\tt LOGCONST}, {\tt STANDS-FOR} \\
{\it pmpropsym} & {\tt PMPROPSYM-P} & {\tt TYPE}, {\tt POLYTYPELIST}, {\tt STANDS-FOR}
(a {\it pmprsym}) \\ 
\\
{\it pmabb} & {\tt PMPROPSYM-P} & {\tt TYPE}, {\tt POLYTYPELIST}, {\tt STANDS-FOR}
(a {\it pmabbsym}) \\
\\
{\it abbrev} & {\tt ABBREV-P} & {\tt TYPE}, {\tt DEFN} \\ & & {\tt TYPELIST} \\
{\it label} & {\tt LABEL-P} & {\tt FLAVOR} \\
 \\
{\it binder} & {\tt BINDER-P} & {\tt VAR-TYPE}, {\tt SCOPE-TYPE}, {\tt WFF-TYPE} \\ \\
\end{tabular}
%\end{Format}

\section{Explanation of Properties}
The various properties mentioned above are as follows:
\begin{description}
\item [{\tt TYPE} ] The type of the object.  Common are {\tt "OOO"} for
binary connectives and {\tt "I"} for individual constants.

\item [{\tt PRINTNOTYPE} ] In first-order mode, this is insignificant, but
if specified and {\tt T}, \TPS will never print types following the object.
It is almost always appropriate to specify this.

\item [{\tt INFIX} ] The binding priority of an infix operator.  This will declare
the connective to be infix.  The absolute value of {\tt INFIX} is irrelevant,
only the relative precedence of the infix and prefix operators matters.
If two binders have identical precedence, association will be to the left.
For example, if R1 and R2 are operators with {\tt INFIX} equal to 1 and 2,
respectively, {\tt "p R1 q R2 r R2 s"} will parse as 
{\tt "[p R1 [[q R2 r] R2 s]]"}.

\item [{\tt PREFIX} ] The binding priority of a prefix operator.  Binders are considered
prefix operators (see about binders below) and thus have a binding
priority.  The main purpose of a prefix binding priority is to allow
formulas like {\tt "~a=b"} to be parsed correctly as {\tt "~[a = b]"} by
giving {\tt =} precedence over {\tt ~}.

\item [{\tt PRT-ASSOCIATIVE} ] indicates whether to assume that the operator is
left associative during printing.  You may want to switch this off (specify
{\tt NIL}) for an infix operator like equivalence, say {\tt <=>}, since
{\tt "p <=> q <=> r"} is often considered to mean {\tt "p <=> q \& q <=> r"}.

\item [{\tt FO-SINGLE-SYMBOL} ] this is meaningful only in first-order mode and
declares the object to be a ``keyword'' in the sense that
it may be typed in all upper or lower case.  Moreover, the printer will
surround it by blanks if necessary to set off surrounding text.  Also
the parser will expect that the symbol is delimited by spaces, dots,
brackets, unless the symbol just consists of one letter, in which case
it doesn't matter.  You {\bf MUST} use this attribute in first-order
mode for an identifier with more than one character.

\item [{\tt MHELP} ] An optional help string.
\end{description}

Properties specific to binders are described in the section below about binders.
Here are some more examples. These examples do not actually exist under these names in \tps.
%\begin{tpsexample}
\begin{verbatim}
(def-logconst &
   (type "OOO")
   (printnotype t)
   (infix 5)
   (prt-associative t)
   (fo-single-symbol &)
   (mhelp "Conjunction."))
\end{verbatim}

Note that the {\tt (fo-single-symbol \&)} will make sure that spaces are printed
around {\tt \&} in formulas.

In the next example the character {\tt /} is used to make sure that the
disjunction is printed in lowercase, that is as {\tt v} instead of {\tt V}.

\begin{verbatim}
(def-logconst /v
   (type "OOO")
   (printnotype t)
   (infix 4)
   (prt-associative t)
   (fo-single-symbol /v)
   (mhelp "Disjunction."))

(def-logconst =>
   (type "OOO")
   (printnotype t)
   (infix 3)
   (fo-single-symbol =>)
   (mhelp "Implication."))
\end{verbatim}

We do not like spaces after negation.  So we do not declare it to be
a {\tt fo-single-symbol}.  That works only because {\tt -} consists of only
one character.

\begin{verbatim}
(def-logconst -
   (type "OO")
   (printnotype t)
   (prefix 6)
   (mhelp "Negation."))

\end{verbatim}
%\end{tpsexample}

\section{Non-terminal Objects of the Syntax}

%@TabSet(1inch)
%\begin{Format, Group}
\begin{tabular}{ll}
\indexSyntax[lsymbol] ::= &  {\it logconst} \\
&  | {\it propsym} \\
&  | {\it pmpropsym} \\
&  | {\it abbrev} \\
&  | {\it pmabbrev} \\
%\end{Format}
\end{tabular}

{\it lsymbol} roughly corresponds to what was called
\indexSyntax{hatom} {for Huet-atom from Huet's unification algorithm}
in the old representation.

{\bf Generalized WFFs}

%\begin{Format, Group}
\begin{tabular}{ll}
{\it gwff} ::=&   {\it lsymbol} \\
&  | {\tt (({\it propsym} . {\it binder}) . {\it gwff})}  ; Generalized binder \\
&  | {\tt ({\it gwff1} . {\it gwff2})}  where {\tt (cdr (type {\it gwff1}))} = {\tt (type {\it gwff2})} \\
&  | {\it label} \\
%\end{Format}
\end{tabular}

\section{Binders in TPS}\label{Binders} 

In the discussion about the internal representation of wffs the issue
of binders has been neglected so far.  Currently, TPS allows three
binders, $\lambda$, $\forall$, $\exists$ (plus some ``buggy'' fragments of
support for the $\exists_{1}$ binder).

Since most binders are inherently polymorphic, there is only one kind of
binder.  Notice that the definition is formulated such that a binder may
have a definition, but need not.

In order to determine the type of a bound wff, the type of the scope of
the binder must be matched against the type stored in the {\tt SCOPE-TYPE}
property.  Also, the type of the bound variable must match the
type in the {\tt VAR-TYPE} property.  These matches are performed, keeping
in mind that all types in the {\tt TYPELIST} property are considered to
be variables.  Then the bindings established during the match are used
to construct the type of the whole bound wff, using the {\tt WFF-TYPE}
property of the binder.

An example may illustrate this process.  The binder {\tt LAMBDA} has the
following properties

% @TabDivide(4)
\begin{tabular}{ll}
TYPELIST & (A B) \\
VAR-TYPE & B \\
SCOPE-TYPE & A \\
WFF-TYPE & (A . B)
\end{tabular}

When trying to determine the type of 
$\lambda x_\greeki . R_{\greeko\greeki\greeki} x$,
\TPS determines that {\tt A} must be $\greeki$, and that {\tt B} must be $\greeko\greeki$.
The type of the original formula is {\wt (A . B)} which then turns out to
be $\greeko\greeki\greeki$.

Note that {\tt TYPELIST} may be absent, i.e.  could be {\wt ()}, which
amounts to stating that the binder has no variable types.  Currently, we
are not using such binders.  An example would be {\it Foralln} , which can
bind only variables of type $\sigma$.

In addition to the properties mentioned above, a binder (except $\lambda$)
would have a definition.  One can then instantiate a binder just as a 
definition can be instantiated.  The definition is to be written with
two designated variables, one for the bound variable and one for the scope.
For example 
{\it THAT} {\it has definition}
$$\iota_{\greeka(\greeko\greeka)} . \lambda b_\greeka S_\greeko$$
Here the {\tt TypeList} would be ($\greeka$), designation for the bound
variable would be $b_\greeka$, designation for the scope would be
$S_\greeko$.

The internal representation for a binder inside a wff is always the
same and simply {\tt ((bdvar .  binder) .  scope)}, but all of the above
information must be present to determine the type of a wff, or to check
whether formulas are well-formed.

Fancy ``special effects'' such as 
$\forall x\in S . A$
must be handled
via special flavors of labels and are not treated as proper binders themselves.

Here are some examples of binders:
%\begin{tpsexample}
\begin{verbatim}

(def-binder lambda
   (typelist ("A" "B"))
   (var-type "A")
   (scope-type "B")
   (wff-type "BA")
   (prefix 100)
   (fo-single-symbol lambda)
   (mhelp "Church's lambda binder."))

(def-binder forall
   (typelist ())
   (var-type "I")
   (scope-type "O")
   (wff-type "O")
   (prefix 100)
   (fo-single-symbol forall)
   (mhelp "Universal quantifier."))
{\it The above definition is for math-logic-1, where forall can only bind individual 
variables. In math-logic-2, the definition is as follows:}

(def-binder forall
   (typelist ("A"))
   (var-type "A")
   (scope-type "O")
   (wff-type "O")
   (prefix 100)
   (fo-single-symbol forall)
   (mhelp "Universal quantifier."))
\end{verbatim}
%\end{tpsexample}

\subsection{An example: How to See the Wff Representations}

You can see examples of how wffs are represented by comparing the output 
of the editor commands \indexedop{P} and \indexedop{edwff}:

%\begin{tpsexample}
\begin{verbatim}
<44>ed x2106
<Ed45>p

FORALL x(I) [R(OI) x IMPLIES P(OI) x] AND FORALL x [~Q(OI) x IMPLIES R x]
 IMPLIES FORALL x.P x OR Q x

<Ed46>edwff
((IMPLIES (AND (|x<I>| . FORALL) (IMPLIES R<OI> . |x<I>|) P<OI> . |x<I>|) 
 (|x<I>| . FORALL) (IMPLIES NOT Q<OI> . |x<I>|) R<OI> . |x<I>|) 
 (|x<I>| . FORALL) (OR P<OI> . |x<I>|) Q<OI> . |x<I>|)
\end{verbatim}
%\end{tpsexample}

Another way to do this is as follows:

%\begin{tpsexample}
\begin{verbatim}
<3>ed x2106
<Ed4>cw
LABEL (SYMBOL):  [No Default]>x2106a
<Ed5>(plist 'x2106a)
(REPRESENTS ((IMPLIES (AND (|x<I>| . FORALL) (IMPLIES R<OI> . |x<I>|) 
 P<OI> . |x<I>|) (|x<I>| . FORALL) (IMPLIES NOT Q<OI> . |x<I>|) R<OI> . |x<I>|) 
 (|x<I>| . FORALL) (OR P<OI> . |x<I>|) Q<OI> . |x<I>|) FLAVOR WEAK)
\end{verbatim}
%\end{tpsexample}

And another way:

%\begin{tpsexample}
\begin{verbatim}
<2>(getwff-subtype 'gwff-p 'x2106) 

((IMPLIES (AND (|x<I>| . FORALL) 
                 (IMPLIES R<OI> . |x<I>|) P<OI> . |x<I>|) (|x<I>| . FORALL)
               (IMPLIES NOT Q<OI> . |x<I>|) R<OI> . |x<I>|) 
     (|x<I>| . FORALL) (OR P<OI> . |x<I>|) Q<OI> . |x<I>|)
\end{verbatim}
%\end{tpsexample}

And finally a way that only works at type O (the 0 below is a zero, not a capital O):

%\begin{tpsexample}
\begin{verbatim}
<3>(get-gwff0 'x2106)

((IMPLIES (AND (|x<I>| . FORALL) 
                 (IMPLIES R<OI> . |x<I>|) P<OI> . |x<I>|)
                  (|x<I>| . FORALL)
                   (IMPLIES NOT Q<OI> . |x<I>|) R<OI> . |x<I>|) 
                     (|x<I>| . FORALL)
                      (OR P<OI> . |x<I>|) Q<OI> . |x<I>|)
\end{verbatim}
%\end{tpsexample}
