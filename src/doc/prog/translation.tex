\chapter{Proof Translations}
\section{Data Structures}
\section{EProofs to Nproofs}
Here is a summary of what happens after matingsearch has terminated with
a proof. 
The functions involved are located in the files {\it mating-merge.lisp},
{\it mating-merge2.lisp} and {\it mating-merge-eq.lisp}.

\begin{enumerate}
\item Apply Pfenning's Merge algorithm ({\tt etr-merge}), put resulting
      etree and mating in variable \indexother{current-eproof}
  \begin{enumerate}
\item  Get a list of the connections from mating

\item Get a list of substitutions required
      \begin{enumerate}
\item  Extract substitutions from unification tree
        
\item  Replace occurrences of PI and SIGMA by quantifiers
        
\item  Lambda-normalize each substitution
        
\item  Alpha-beta-normalize each substitution
      \end{enumerate}
    
\item  Prune any unused expansions from the tree
    
\item  Make substitutions for variables
    
\item  Carry out merging ({\tt merge-all})
  \end{enumerate}

\item  Replace skolem terms by parameters (if applicable) 
({\tt subst-skol-terms})

\item  If remove-leibniz is T, apply Pfenning's algorithm for removing
      the Leibniz equality nodes for substitution of equality nodes
      ({\tt remove-leibniz-nodes})

\item  Try to replace selected parameters by the actual bound variables
      ({\tt subst-vars-for-params}), not always possible because of restriction
      that a parameter should appear at most once

\item  Raise lambda rewrite nodes, so that in natural deduction the lambda
      normalization occurs as soon as possible. ({\tt raise-lambda-nodes})

\item  Clean up the etree ({\tt cleanup-etree}).  For each expansion term
      in the tree,
   \begin{enumerate}
\item  Lambda-normalize it
      
\item  Minimize the superscripts on bound variables
      
\item Make a new expansion with the new term
      
\item  Deepen the new expansion like the original, but removing 
            unnecessary lambda-norm steps.
      
\item  Remove the original expansion
   \end{enumerate}

 Begin natural deduction proof, using \indexother{current-eproof}
   \begin{itemize}
\item  Set up planned line 
	\begin{enumerate}
\item  Use shallow wff of the current-eproof's etree
          
\item  Give it the tree as value for its NODE property
          
\item  Give it the current-eproof's mating (list of pairs of node names)
                as its MATING property
	\end{enumerate}
   \end{itemize}

 Call {\tt use-tactic} with tactic desired
  \begin{enumerate}
\item  Each line in the proof will correspond to a node in the etree; the 
natural deduction proof is stored in the variable \indexother{dproof}.
    
\item   Here is an important property which should remain invariant during
    the translation process:  It should always be the case that the 
    line-mating of the planned line is a p-acceptable mating for the
    etree that one could construct by making an implication whose antecedent
    is the conjunction of the line-nodes of the supports, and whose 
    consequent is the line-node of the planned line.  This will assure us
    that we have sufficient information to carry out the translation.
  \end{enumerate}
\end{enumerate}

It was observed that when path-focused duplication had been
used, the expansion proof would often have a great deal of redundancy
in the sense that the same expansion term would be used for a given variable
many times. More precisely, if one defines an expansion branch by
looking at sequences of nested expansion nodes, attaching one expansion
term to each expansion node in the sequence, there would be many identical
expansion branches. 

In response to this, {\it mating-merge.lisp} was modified in the following ways:
\begin{itemize}
\item Don't do pruning of unnecessary nodes at the beginning of the merge,
when the tree is its greatest size. 

\item Instead, prune all branches that couldn't possibly have been used; 
they are those that have a zero status. This is probably not necessary,
but certainly makes debugging easier and doesn't cost much.

\item After merging of identical expansions has been done, call the original
pruning function.
\end{itemize}

\section{NProofs to Eproofs}
There are three versions of \indexcommand{NAT-ETREE},
the command for translating natural deductions into
expansion tree proofs.  The user can choose between
the three by setting the flag \indexflag{NAT-ETREE-VERSION}
to one of the following values:
\begin{enumerate}
\item {\bf OLD}  (the original version)
\item {\bf HX} (Hongwei Xi's version, written in the early
to mid 1990's)
\item {\bf CEB} (Chad E. Brown's version, written in early 2000)
\end{enumerate}
Also, note that setting the flag \indexflag{NATREE-DEBUG}
to T is useful for debugging the {\bf HX} and {\bf CEB}
versions.  The subsections that follow describe each of these
versions in greater detail.

After using \indexcommand{NAT-ETREE} to translate to an
expansion proof, the user can use this expansion proof
to suggest flag settings (via the mate commands
\indexother{ETR-INFO} and \indexcommand{ETREE-AUTO-SUGGEST})
or to trace MS98-1 (using the flag \indexflag{MS98-TRACE}.  
See the User's Manual for a description
of these facilities.  The User's Manual also has examples.

\subsection{Chad's Nat-Etree}\label{ceb-nat-etr}

To use this version of \indexcommand{NAT-ETREE}, set
\indexflag{NAT-ETREE-VERSION} to {\bf CEB}.
The main functions for this version are in the
files \indexfile{ceb-nat-etr.lisp} and \indexfile{ceb-nat-seq.lisp}.  
The relevant functions are:
\begin{description}
\item [\indexfunction{ceb-nat-etree}]  
This is the main function.  It preprocesses the proof to 
\begin{itemize}
\item remove applications of {\it Subst=} and {\it Sym=},
\item expand applications of {\it RuleP} and other propositional rules
(e.g., {\it Assoc}) in terms of more
primitive inference rules,
\item attempt to expand any applications of {\it RuleQ},
\item replace instances of {\it Assert} by hypotheses
which are discharged at the end of the proof,
\item and replace applications of the {\it Cases} rule
using more than two disjuncts by multiple applications
of the {\it Cases} rule using two disjuncts.
\end{itemize}
The function then calls \indexfunction{ceb-proof-to-natree} to build
the natree version of the natural deduction proof, calls
\indexfunction{natree-to-ftree-main} to build the ftree representation
of the expansion tree and a complete mating.  Finally, this
is converted into an ordinary expansion proof which may optionally
be merged.  (Merging is appropriate if the user plans to translate
back to a natural deduction proof, but inappropriate if the user
is trying to gather information about a potential automatic proof.)
\item [\indexfunction{ceb-proof-to-natree}]  This is a modification
of Hongwei's 
\indexfunction{proof-to-natree} (see \indexfile{hx-natree-top.lisp}).
This function builds the natree, changing some justifications to $RuleP$
or $RuleQ$, and changing the variables in applications of $UGen$ and $RuleC$
so they are unique in the entire proof (i.e., the natree rules satisfy a global
eigenvariable condition, since the etree selection variables must be distinct).
\item [\indexfunction{natree-to-ftree-main}]  This function calls
\indexfunction{natree-to-ftree-seq-normal} to build a sequent calculus derivation
(see ~\ref{ftree-seq}) from the natree.  Then \indexfunction{ftree-seq-weaken-early}
modifies the derivation so that the weaken rule is applied eagerly.  This may
eliminate certain unnecessary cuts and simplify the derivation.
Then the cut elimination function \indexfunction{ftree-seq-cut-elim}
(see section~\ref{ftree-seq-mix-elim})
is used to make the derivation cut-free.
Finally, \indexfunction{cutfree-ftree-seq-to-ftrees} is used to obtain
an ftree and a complete mating from the cut-free derivation.
\item [\indexfunction{natree-to-ftree-seq-normal}, \indexfunction{natree-to-ftree-seq-extraction}]
These 
functions are mutually recursive and provide the main algorithm for constructing
the sequent calculus derivation.  A description of the algorithm is below.  The function \indexfunction{natree-to-ftree-seq-normal}
is called on natree nodes which are considered normal.  (These would be annotated with a $\Uparrow$.)
The function \indexfunction{natree-to-ftree-seq-extraction} is called on natree nodes which
are considered extractions.  (These would be annotated with a $\downarrow$.)
\end{description}

Frank Pfenning's ATP
class contained notes on annotating (intuitionistic first-order)
normal natural deduction proofs, and gave a constructive proof
(algorithm) that every natural deduction proof
translates into a sequent calculus proof.  Also, normal natural deduction proofs
translate to cut-free sequent calculus proofs.
The idea of using annotations carries over to classical higher-order
logic.

\subsubsection{Normal Deductions}

The idea of a normal deduction is that the proof works down
using elimination rules, and up using introduction rules,
meeting in the middle.  We can formalize this idea by saying that
a natural deduction proof is normal if its assertions can be annotated,
so that the assertions involved in the applications of rules of inference
are as described below.
Technically, we are defining normal natural deductions by mutually
defining normal deductions ($\Uparrow$) and extraction deductions ($\downarrow$).

\subsubsection{Annotations of the Assertions in a Proof}

First, the basic rules which allow one to infer normal deductions ($\Uparrow$).
$$ \ianc{A\Uparrow}{\forall x A\Uparrow}{UGen}$$
where $x$ is not free in any hypotheses.
$$ \ibnc{\exists y A\downarrow}{\hypo{\ian{}{[x/y]A\downarrow}{}}{C\Uparrow}}{C\Uparrow}{RuleC}$$
where $x$ is not free in any hypotheses.
$$ \ianc{[t/x]A\Uparrow}{\exists x A\Uparrow}{EGen}$$
$$ \ianc{A\Uparrow}{A\lor B\Uparrow}{IDisj-L} \;
 \ianc{B\Uparrow}{A\lor B\Uparrow}{IDisj-R}$$
$$ \ibnc{A\Uparrow}{B\Uparrow}{A \land B\Uparrow}{Conj} $$
$$ \ianc{\bot\downarrow}{A\Uparrow}{Absurd}$$
$$ \ianc{\hypo{\ian{}{\neg A\downarrow}{}}{\bot\Uparrow}}{A\Uparrow}{Indirect} \;
 \ianc{\hypo{\ian{}{A\downarrow}{}}{\bot\Uparrow}}{\neg A\Uparrow}{NegIntro} \;
 \ianc{\hypo{\ian{}{A\downarrow}{}}{B\Uparrow}}{A\supset B\Uparrow}{Deduct}$$
$$ \ibnc{\neg A\downarrow}{A\Uparrow}{C\Uparrow}{NegElim}$$
$$ \ibnc{A \vee B\downarrow}{\hypo{\ian{}{A\downarrow}{}}{C\Uparrow}\quad\qquad\hypo{\ian{}{B\downarrow}{}}{C\Uparrow}}{C\Uparrow}{Cases}$$

TPS also has rules {\it Cases3} and {\it Cases4} which may be used to
eliminate disjunctions with three or four disjuncts, resp.  
Such rule applications are replaced by iterations of the binary
{\it Cases} rule in a preprocessing step using
\indexfunction{expand-cases}.

Next, the basic rules which allow one to infer extraction deductions ($\downarrow$).
$$ \ianc{A \land B\downarrow}{A\downarrow}{Conj} \hspace{2em} \ianc{A \land B\downarrow}{B\downarrow}{Conj}$$
$$ \ianc{\forall x A\downarrow}{[t/x]A\downarrow}{UI}$$
$$ \ibnc{A \limplies B\downarrow}{A\Uparrow}{B\downarrow}{MP}$$

Notice that hypothesis lines are always considered extraction derivations.
Such lines may be justified by any of the following:
$Hyp$, $Choose$, $Assume negation$, $Case 1$, $Case 2$, $Case 3$, $Case 4$.

We need a coercion rule, as every extraction is a normal derivation:
$$ \ianc{A\downarrow}{A\Uparrow}{coercion}$$
In a TPS natural deduction style proof, this coercion step will not usually be
explicit.  Instead, a single line will be given the property of being
a coercion, in which case we know it has both annotations $\downarrow$
and $\Uparrow$, and that these annotations were assigned in a way consistent
with the coercion rule above.  Often, when interactively constructing
a natural deduction proof in TPS, one finds that a planned line is the
same as a support line, and finishes the subgoal using SAME.  This would
correspond to the coercion rule above.

The backward coercion rule 
$$ \ianc{A\Uparrow}{A\downarrow}{bcoercion}$$
is used to pass from normal deductions to extractions.
Backwards coercions correspond to instances of cut.
A separate, interesting project in TPS would be to
program a normalization procedure.
Such a procedure would
find instances of the backward coercion rule when
annotating a proof, identify to what kind of ``redex''
the backward coercion rule corresponds, and perform the
reduction.  For this to work we would need to define the
notion of redex so that every proof which needs the backward
coercion rule to be annotated (proofs that are not normal)
must have a redex.  Also, we could not prove that reduction
terminates -- a task equivalent to constructively proving 
cut-elimination in classical higher-order logic.
Instead, the current code translates backwards coercion
as an application of cut, and then uses a cut elimination
procedure (which may not terminate) to obtain a cut-free
proof.  (See section~\ref{ftree-seq-mix-elim})

\subsubsection{Some Nonstandard ND Rules}

There is code to replace ``fancy'' propositional rules
like $RuleP$ with subderivations using primitive rules.  
See the commands \indexcommand{ELIMINATE-ALL-RULEP-APPS}, \indexcommand{ELIMINATE-RULEP-LINE},
and \indexcommand{ELIMINATE-CONJ*-RULEP-APPS}.  The command \indexcommand{ELIMINATE-CONJ*-RULEP-APPS}
only expands those $RuleP$ applications which can be replaced by applications of $Conj$ rules.
The other two commands use Sunil's fast propositional search to find an expansion proof
and uses the tactic \indexother{BASIC-PROP-TAC} to translate back to natural deduction to fill
in the gap using only primitive propositional rules.

Of course, the $Same$ rule just propagates the annotation.
So, with respect to annotations, there are two versions of this rule:
$$\ianc{A\downarrow}{A\downarrow}{Same As}\hspace{2em}
\ianc{A\Uparrow}{A\Uparrow}{Same As}$$

When converting normal natural deduction to expansion tree
proofs, we only consider formulas up to $\alpha$-conversion,
so we can ignore the corresponding ND rule.  But effectively,
we allow this rule to be annotated in either of two ways,
as with the $Same$ rule:
$$\ianc{\forall x A\downarrow}{\forall y [y/x]A\downarrow}{AB}\hspace{2em}
\ianc{\forall x A\Uparrow}{\forall y [y/x]A\Uparrow}{AB}$$

$Neg$, $NNF$, and $NNF-Expand$ can be used to make small first-order
inferences (from $\neg \forall x . A$ to $\exists x . \neg A$, etc.).
Since we only care about formulas up to $\alpha\beta$ and negation-normal-form, we
can treat these rules the same way as the $Same$ and $AB$ rules.

Applications of $Assert$ (other than $Assert Refl=$) are replaced by
explicit hypotheses in a preprocessing step by the function 
\indexfunction{make-assert-a-hyp}.  So, when building the natree,
there should be no instances of $Assert$ other than $Refl=$.
$Assert Refl=$ is annotated as a normal deduction:
$$\ianc{}{A=A\Uparrow}{Assert Refl=}$$
The idea is that we work backwards to an instance of reflexivity.

Definitions can be eliminated or introduced, and the annotations
reflect this.  Also, elimination and introduction of definitions includes
some $\beta$-reduction.  Suppose the abbreviation $A$ is defined to
be $\lambda x_1\cdots\lambda x_n . \psi [x_1,\ldots, x_n]$ in the following
annotated rule schemas.
$$\ianc{\phi[A\; B_1\; \cdots \; B_n]\downarrow}{\phi[\psi[B_1,\ldots,B_n]]\downarrow}{Defn}$$
$$\ianc{\phi[\psi[B_1,\ldots,B_n]]\Uparrow}{\phi[A\; B_1\; \cdots \; B_n]\Uparrow}{Defn}$$

When annotating $\lambda$ rules, the arrows point in the
direction of normalization.
$$\ianc{B\downarrow}{A\downarrow}{Lambda}\hspace{2em}
\ianc{A\Uparrow}{B\Uparrow}{Lambda}$$
where $A$  is the $\beta\eta$-normal form of $B$.
There are also rules $Beta$ and $Eta$ which are treated similarly.

\subsubsection{Equality Rules}

We assume that the proof has been preprocessed to remove
applications of substitution of equals, and applications of
symmetry, so there is no need to annotate these rules (for now).

As noted above, reflexivity is treated as a normal deduction:
$$\ianc{}{A = A\Uparrow}{Refl=}$$

There are two ways to apply extensionality consistent with the
idea of annotations (both correspond to expanding an equation using
extensionality in the corresponding expansion tree).  Also, there
are two kinds of extensionality (functional and propositional).
$$\ianc{f_{\greekb\greeka} = g_{\greekb\greeka}\downarrow}{\forall x_\greeka . f x = g x\downarrow}{Ext=}
\hspace{2em}
\ianc{P_\greeko = Q_\greeko\downarrow}{P_\greeko \equiv Q_\greeko \downarrow}{Ext=}$$
$$\ianc{\forall x_\greeka . f x = g x\Uparrow}{f_{\greekb\greeka} = g_{\greekb\greeka}\Uparrow}{Ext=}
\hspace{2em}
\ianc{P_\greeko \equiv Q_\greeko\Uparrow}{P_\greeko = Q_\greeko\Uparrow}{Ext=}$$

Leibniz equality is handled just like definition expansion.
$$\ianc{A_\greeka = B_\greeka\downarrow}{\forall q_{\greeko\greeka} . q A \limplies q B\downarrow}{Equiv-eq}$$
$$\ianc{\forall q_{\greeko\greeka} . q A \limplies q B\Uparrow}{A_\greeka = B_\greeka\Uparrow}{Equiv-eq}$$

\subsubsection{A Sequent Calculus}\label{ftree-seq}

In the file \indexfile{ftree-seq.lisp}, a sequent calculus
is implemented.  The file contains code to convert
sequent calculus derivations into
expansion proofs, and a cut elimination
algorithm.  This is a two sided sequent calculus
with sequents $\Gamma \rightarrow \Delta$.
The code refers to formulas in $\Gamma$ as positive
(as opposed to ``left'') and formulas in $\Delta$
as negative (as opposed to ``right'') to correspond
to the parity of expansion tree nodes.
$\Gamma$ and $\Delta$ are lists, as opposed to multisets
or sets, so order and multiplicity are important.

There are many variations of sequent calculi for classical
logic.  For example, consider the two
variants of the negative rule for $\land$:
$$\ibnc{\Gamma\rightarrow A,\Delta}{\Gamma\rightarrow B,\Delta}{\Gamma\rightarrow A\land B,\Delta}{}$$
$$\ibnc{\Gamma_1\rightarrow A,\Delta_1}{\Gamma_2\rightarrow B,\Delta_2}{\Gamma_1,\Gamma_2\rightarrow A\land B,\Delta_1,\Delta_2}{}$$
Furthermore, there is the issue of the positions of
the main formulas (i.e., must $A$ and $B$ be the first formulas
on the list?)
Different kinds of rules determine what structural rules the sequent calculus
should have.  The sequent calculus implemented in \indexfile{ftree-seq}
has the following logical rules:

\begin{center}
$\ibnc{\Gamma_1\rightarrow A,\Delta_1}{\Gamma_2\rightarrow B,\Delta_2}{\Gamma_1,\Gamma_2\rightarrow A\land B,\Delta_1,\Delta_2}{\land-}$\hspace{2cm}
$\ianc{A,B,\Gamma\rightarrow \Delta}{A\land B,\Gamma\rightarrow \Delta}{\land+}$\\[.5cm]
$\ianc{\Gamma\rightarrow A,B,\Delta}{\Gamma\rightarrow A\lor B,\Delta}{\lor-}$\hspace{2cm}
$\ibnc{A,\Gamma_1\rightarrow \Delta_1}{B,\Gamma_2\rightarrow \Delta_2}{A\lor B,\Gamma_1,\Gamma_2\rightarrow \Delta_1,\Delta_2}{\lor+}$\\[.5cm]
$\ianc{A,\Gamma\rightarrow B,\Delta}{\Gamma\rightarrow A\limplies B,\Delta}{\limplies-}$\hspace{2cm}
$\ibnc{\Gamma_1\rightarrow A,\Delta_1}{B,\Gamma_2\rightarrow \Delta_2}{A\limplies B,\Gamma_1,\Gamma_2\rightarrow \Delta_1,\Delta_2}{\limplies+}$\\[.5cm]
$\ianc{A,\Gamma\rightarrow \Delta}{\Gamma\rightarrow \lnot A,\Delta}{\lnot-}$\hspace{2cm}
$\ianc{\Gamma\rightarrow A,\Delta}{\lnot A,\Gamma\rightarrow \Delta}{\lnot+}$\\[.5cm]
$\ianc{A(a),\Gamma\rightarrow \Delta}{\forall x A(x),\Gamma\rightarrow \Delta}{SEL+^a}$\hspace{2cm}
$\ianc{\Gamma\rightarrow A(a),\Delta}{\Gamma\rightarrow \forall x A(x),\Delta}{SEL-^a}$\\[.5cm]
$\ianc{A(t),\Gamma\rightarrow \Delta}{\forall x A(x),\Gamma\rightarrow \Delta}{EXP+^t}$\hspace{2cm}
$\ianc{\Gamma\rightarrow A(t),\Delta}{\Gamma\rightarrow \exists x A(x),\Delta}{EXP-^t}$\\[.5cm]
\end{center}

There are also rewrite rules:\\[.5cm]
\begin{center}
$\ianc{\Gamma\rightarrow [A\limplies B]\land [B\limplies A],\Delta}{\Gamma\rightarrow A\equiv B,\Delta}{REW(\equiv)-}$\\[.5cm]
$\ianc{[A\limplies B]\land [B\limplies A],\Gamma\rightarrow \Delta}{A\equiv B,\Gamma\rightarrow \Delta}{REW(\equiv)+}$\\[.5cm]
$\ianc{\Gamma\rightarrow A^{\lambda,\beta,\eta},\Delta}{\Gamma\rightarrow A,\Delta}{REW({\lambda,\beta,\eta})-}$\hspace{2cm}
$\ianc{A^{\lambda,\beta,\eta},\Gamma\rightarrow \Delta}{A,\Gamma\rightarrow \Delta}{REW({\lambda,\beta,\eta})+}$\\[.5cm]
\end{center}
where $A^{\lambda,\beta,\eta}$ is either the $\beta\eta$-normal form,
$\beta$-normal form, or $\eta$-normal form of $A$.\\[.5cm]
\begin{center}
$\ianc{\Gamma\rightarrow A,\Delta}{\Gamma\rightarrow B,\Delta}{REW(AB)-}$\hspace{2cm}
$\ianc{A,\Gamma\rightarrow \Delta}{B,\Gamma\rightarrow \Delta}{REW(AB)+}$\\[.5cm]
\end{center}
where $A$ and $B$ are $\alpha$-equivalent.\\[.5cm]
\begin{center}
$\ianc{\Gamma\rightarrow A,\Delta}{\Gamma\rightarrow B,\Delta}{REW(EQUIVWFFS)-}$\\[.5cm]
$\ianc{A,\Gamma\rightarrow \Delta}{B,\Gamma\rightarrow \Delta}{REW(EQUIVWFFS)+}$\\[.5cm]
\end{center}
where $A$ is the result of expanding some abbreviations in $B$.\\[.5cm]
\begin{center}
$\ianc{\Gamma\rightarrow A,\Delta}{\Gamma\rightarrow B,\Delta}{REW(Leibniz=)-}$\hspace{2cm}
$\ianc{A,\Gamma\rightarrow \Delta}{B,\Gamma\rightarrow \Delta}{REW(Leibniz=)+}$\\[.5cm]
\end{center}
where $A$ is the result of expanding some equalities in $B$ using the Leibniz
definition of equality.\\[.5cm]
\begin{center}
$\ianc{\Gamma\rightarrow A,\Delta}{\Gamma\rightarrow B,\Delta}{REW(Ext=)-}$\hspace{2cm}
$\ianc{A,\Gamma\rightarrow \Delta}{B,\Gamma\rightarrow \Delta}{REW(Ext=)+}$\\[.5cm]
\end{center}
where $A$ is the result of expanding some equalities in $B$ using extensionality.
(This does not provide a complete calculus for extensionality without a cut rule.
So, sometimes cut elimination will fail if these extensionality rules are used.)

The structural rules are:

\begin{center}
$\ianc{\Gamma\rightarrow \Delta}{\Gamma\rightarrow A,\Delta}{weaken-}$\hspace{2cm}
$\ianc{\Gamma\rightarrow \Delta}{A,\Gamma\rightarrow \Delta}{weaken+}$\\[.5cm]
$\ianc{\Gamma\rightarrow A,A,\Delta}{\Gamma\rightarrow A,\Delta}{merge-}$\hspace{2cm}
$\ianc{A,A,\Gamma\rightarrow \Delta}{A,\Gamma\rightarrow \Delta}{merge+}$\\[.5cm]
$\ianc{\Gamma\rightarrow\Delta_1,A,\Delta_2}{\Gamma\rightarrow A,\Delta_1,\Delta_2}{focus^n+}$
where $\Delta_1$ has length $n$.\\[.5cm]
$\ianc{\Gamma_1,A,\Gamma_2\rightarrow\Delta}{A,\Gamma_1,\Gamma_2\rightarrow\Delta}{focus^n+}$
where $\Gamma_1$ has length $n$.\\[.5cm]
\end{center}

Finally, we have an initial rule and a cut rule:

\begin{center}
$\ianc{}{A\rightarrow A}{init}$\hspace{2cm}
$\ibnc{\Gamma_1\rightarrow A,\Delta_1}{A,\Gamma_2\rightarrow \Delta_2}{\Gamma_1,\Gamma_2\rightarrow \Delta_1,\Delta_2}{cut}$\\[.5cm]
\end{center}

In all these rules, it is important that the formulas
appear in the positions as indicated in the diagrams above.
The focus rule gives us the only way to reorder
the formulas of the sequent.  This forces us to do some
tedious shuffling in some places, but makes it easier to
perform recursion on the sequent derivations, since we
have a very good idea of how the rule application looks.

The sequent calculus is 
similar to the ftree representation of expansion trees (see section~\ref{ftrees}),
and the file includes a function
\indexfunction{cutfree-ftree-seq-to-ftrees}
which translates a cut-free sequent calculus derivation of
$\Gamma\rightarrow\Delta$
to two lists of ftrees
$\Gamma^*$ and $\Delta^*$,
and a list of connections $M$.
For each wff $A\in\Gamma$, there is a
correponding positive ftree $F\in\Gamma^*$ with shallow 
formula $A$.
For each wff $A\in\Delta$, there is a
correponding negative ftree $F\in\Delta^*$ with shallow 
formula $A$.  The list of connections $M$ gives a complete
mating for the ftree
$\bigwedge(\Gamma^*) \limplies \bigvee(\Delta^*)$.
In particular, a cut-free sequent calculus derivation of
$\rightarrow A$ will be translated into a negative ftree
with shallow formula $A$ and a complete mating $M$.

Regarding this translation to ftrees,
the names of the logical and rewrite rules
correspond to the construction of the corresponding ftree.
The $weaken$ and $focus$ structural rules are relatively easy to handle.
Applications of $merge$ require the use of a $merge$ algorithm for ftrees
(in the file \indexfile{ftrees}).
The $init$ rule corresponds to two mated nodes.
And, of course, we cannot translate an application of $cut$.

\subsubsection{Translating from Natural Deduction to Sequent Calculus}

In the sequent calculus described above, the order and multiplicity
of formulas is important.  However, in describing the algorithm below,
we are more interested in sets of formulas.  So, let us use the
notation $Set(\Gamma)$ to denote the set of fomulas on the list $\Gamma$.

Normal natural deductions are converted into the sequent calculus 
via two mutually recursive algorithms:

\begin{enumerate}
\item  \indexfunction{natree-to-ftree-seq-normal}:
Suppose we are given a line $\Gamma \vdash C\Uparrow$.
Then we can compute a derivation of a sequent
$\Gamma_1\rightarrow C$ where $Set(\Gamma_1)\subseteq Set(\Gamma)$.
\item  \indexfunction{natree-to-ftree-seq-extraction}:
Given a line $\Gamma \vdash B\downarrow$
and a derivation of a sequent $\Gamma_1 \rightarrow C$
where $Set(\Gamma_1) \subseteq Set(\Gamma)\cup \{B\}$.
Then we can compute a derivation of
a sequent
$\Gamma_2\rightarrow C$
where $Set(\Gamma_2) \subseteq Set(\Gamma)$.
(That is, we have eliminated occurrences of $B$ on the
positive side.)
\end{enumerate}

We can show a few cases to demonstrate how the algorithms
work.

{\bf Case:}
Coercion.
$$ \ianc{C\downarrow}{C\Uparrow}{coercion}$$
with hypotheses $\Gamma$.
We need a derivation of some $\Gamma_2\rightarrow C$.
We can apply the second induction hypothesis
to the initial sequent $C\rightarrow C$ (with
$\Gamma_1$ empty)
to obtain a derivation of such a sequent $\Gamma_2\rightarrow C$.

{\bf Case:}  Hyp.  Suppose the line is
$$\Gamma \vdash B$$
where $B$ is in $\Gamma$,
and suppose we are given a derivation of a sequent
$\Gamma_1 \rightarrow C$ with
$Set(\Gamma_1)\subseteq Set(\Gamma)\cup \{B\}$.
Since $B\in\Gamma$, we have
$Set(\Gamma_1)\subseteq Set(\Gamma)$
and we are done.

{\bf Case:}  Deduct.  This case is easy, as are most of the ``introduction''
rules.  Suppose we have
$$\ianc{\above{\DD}{\Gamma, A\vdash B\Uparrow}}{\Gamma \vdash A\supset B\Uparrow}{Deduct}$$
By induction we have a derivation of a sequent
$\Gamma_1\rightarrow B$
where $Set(\Gamma_1)\subseteq Set(\Gamma)\cup \{A\}$.
Using the structural rules (see the function \indexfunction{ftree-seq-merge-focus-all-pos})
we obtain a derivation with $A$ at the front:
$A,\Gamma_2\rightarrow B$
where $Set(\Gamma_2) \subseteq Set(\Gamma)$.
Applying the $\limplies-$ rule, we have a derivation of
$\Gamma_2\rightarrow A\limplies B$ as desired.

{\bf Case:}  MP.  This case is interesting, because a naive   
algorithm would be forced to treat this case like a ``cut'' in
the sequent calculus.  Suppose we have
$$ \ibnc{\above{\DD}{\Gamma \vdash A \limplies B\downarrow}}{\above\EE{\Gamma \vdash A\Uparrow}}{\Gamma \vdash B\downarrow}{MP}$$
Since this is an extraction, we must be given a derivation of
$\DD_1$ a sequent $\Gamma_1\rightarrow C$
where $Set(\Gamma_1) \subseteq Set(\Gamma)\cup \{B\}$.
Applying structural rules to $\DD_1$, we have a derivation $\DD_2$
of a sequent $B,\Gamma_2\rightarrow C$
with $Set(\Gamma_2)\subseteq Set(\Gamma)$.

The first algorithm applied to $\EE$ gives a derivation of some
$\Gamma_3\rightarrow A$ where $Set(\Gamma_3)\subseteq Set(\Gamma)$.
If we apply $\limplies+$ as follows:
$$\ibnc{\Gamma_3\rightarrow A}{B,\Gamma_2\rightarrow C}{A\limplies B,\Gamma_3,Gamma_2\rightarrow C}{\limplies+}$$
then we can call the second algorithm on this derivation and $\DD$
to obtain a derivation of some $\Gamma_4\rightarrow C$
with $Set(\Gamma_4)\subseteq Set(\Gamma)$.

{\bf Case:}  Backwards Coercion.  
$$ \ianc{\above{DD}{B\Uparrow}}{B\downarrow}{bcoercion}$$
with hypotheses $\Gamma$.
Suppose we are given a derivation of some $\Gamma_1\rightarrow C$
where $Set(\Gamma_1)\subseteq Set(\Gamma)\cup \{B\}$.
Using structureal rules we obtain a derivation of a sequent
$B,\Gamma_2\rightarrow C$ where
$Set(\Gamma_2)\subseteq Set(\Gamma)$.
We want to remove $B$ from the positive side.
Applying the first algorithm to $\DD$, we obtain a
derivation of a sequent $\Gamma_3\rightarrow B$
with $Set(\Gamma_3)\subseteq Set(\Gamma)$.
An application of $cut$ gives us the sequent we desire:
$$\ibnc{\Gamma_3\rightarrow B}{B,\Gamma_2\rightarrow C}{\Gamma_3,\Gamma_2\rightarrow C}{cut}$$

{\bf Remark:}  We check equality of wff's up to $\alpha$-conversion and negation-normal-form.
Because we check up to negation-normal-form, applications of $Neg$ and $NNF$ rules can be
treated the same way as the $Same$ and $AB$ rules.

{\bf Note:}  If \indexflag{NATREE-DEBUG} is set to T, then at each step,
the code double checks that the derivation is well formed.

After we have a sequent calculus derivation, cut elimination can be
used to try to remove applications of cut (see section~\ref{ftree-seq-mix-elim}.  If we obtain a cut-free
derivation, this can be translated into an ftree with a complete mating.

% {\bf Example:}
% 
% Consider the following natural deduction proof of
% $$\forall \,x [ \,P \,x \supset \,P . \,f \,x] \supset . \,P \,a \supset \,P . \,f . \,f \,a$$
% 
% {\it Show the proof.}
% 
% This is a normal proof, as it can be annotated as follows (without using backward coercion):
% 
% {\it Fill in the rest of this example and work through the conversion to an expansion
% tree proof step by step.}

\subsubsection{Normalization of Proofs}

There is now a \TPS command \indexcommand{NORMALIZE-PROOF}
that converts a natural deduction proof (or a natural
deduction proof with asserted lemmas which have natural
deduction proofs in memory) into a sequent calculus proof
(with cuts), then uses the cut-elimination algorithm
to obtain a cut-free proof (assuming termination),
and finally translates back to natural deduction.
The resulting natural deduction proof is normal.

 
If we decided to normalize natural deduction proofs directly
(without passing through a sequent calculus), we would
need to identify
possible redexes (pairs of rule applications which must use
backward coercion to be annotated), and show how to reduce these.
There are many such redexes.  The following is a typical example:

$$ \ianc{\ibnc{\above\DD{A}}{\above\EE{B}}{A \land B\downarrow}{Conj}}{A\downarrow}{Conj} \rightarrow \above\DD{A}$$

In first order logic, one can show that some measure on the proof reduces
when a redex is reduced, so that the process will terminate with
a normal proof.  In higher order logic, showing termination is equivalent
to showing termination of cut-elimination.

Actually carrying this out is a possible future project.  Though
this is much less important since we now have a cut elimination algorithm
implemented.

\subsection{Hongwei's Nat-Etree}

This is a brief description of Hongwei's code
for \indexcommand{NAT-ETREE}.  To use this code,
set \indexflag{NAT-ETREE-VERSION} to {\bf HX}.

%{\bf Note:  Hongwei wanted this code to work for all natural deductions,
%whether normal or not.  So, given a natural deduction proof which
%is not normal, try this one, and you might get lucky.}

\indexfunction{ATTACH-DUP-INFO-TO-NATREE} is the main function, which
is called recursively on the subproofs of a given natural
deduction. The goal of \indexfunction{ATTACH-DUP-INFO-TO-NATREE} is to
construct an expansion tree, with no mating attached,
corresponding to a given natural deduction. The constructed
expansion tree contains all the correct duplications done on
quantifiers and all substitutions done on variables.
A propositional search will be called on the generated expansion
tree to recover the mating and generate an expansion proof.
Then \indexcommand{ETREE-NAT} can produce a natural deduction corresponding
to the constructed expansion proof.

The following is an oversimplified case.

Given natural deductions N1 and N2 with conclusions
A and B, respectively, and N derived from N1 and N2
by conjunction introduction. \indexfunction{ATTACH-DUP-INFO-TO-NATREE}
called on N generates two recursive calls on N1 and N2,
and get the expansion proofs corresponding to N1 and N2,
respectively, with which it constructs an expansion proof
corresponding to N.

An important feature of \indexfunction{ATTACH-DUP-INFO-TO-NATREE} is that
it can deal with all natural deductions, with or without
cuts in them. This is mainly achieved by substitution and
merge. This essentially corresponds to the idea in Frank
Pfenning's thesis, though his setting is sequent calculus.
On the other hand, the implementation differs significantly
since natural deductions grow in both ways when compared with
sequent calculus. This is reflected in the code of
\indexfunction{ATTACH-DUP-INFO-TO-NATREE} which travels through a natural
deduction twice, from bottom to top and from top to bottom,
to catch all the information needed to duplicate quantifiers
correctly.

Overview of the files:
\begin{itemize} 
\item \indexfile{hx-natree-top} contains the definition of the data structure,
some print functions and top commands.

\item \indexfile{hx-natree-duplication} contains the code of \indexfunction{ATTACH-DUP-INFO-TO-NATREE}
and some auxiliary functions such as \indexfunction{UPWARD-UPDATE-NATREE}. Also many
functions for constructing expansion trees are defined here.

\item \indexfile{hx-natree-rulep} contains the code for handling \indexfunction{RULEP}. This is done
by using hash tables to store positive and negative duplication
information. Then cuts are eliminated by substitution and merge.
The case in \indexfunction{ATTACH-DUP-INFO-TO-NATREE} which deals with implication
is a much simplified version of this strategy, and helps understand
the algorithm.

\item \indexfile{hx-natree-aux} contains the code of merge functions and the ones handling
rewrite nodes. Presumably there are some bugs in handling rewrites, and
this can be found in the comments mixed with the code. Also a new version
of \indexfunction{ETREE-TO-JFORM-REC} is defined here to cope with a modified date structure
\indexother{ETREE}.

\item \indexfile{hx-natree-cleanup} contains the functions which clean up the expansion
proofs before they can be used by \indexcommand{ETREE-NAT}. This is temporary crutch,
and should be replaced by some systematic methods. For instance, one
could construct brand new expansion proofs according to a constructed
one rather than modify it to fit the needs of \indexcommand{ETREE-NAT}. This yields
a better chance to avoid some problems caused by rewrite nodes. 

\item \indexfile{hx-natree-debug} contains some simple debugging facilities such as some
display function and some modified versions of the main functions in the
code. A suggested way is to modify the code using these debugging functions
and trace them. More facilities are needed to eliminate sophisticated bugs.
\end{itemize}

Selection nodes, not Skolem nodes, are used in the constructed expansion
trees. The prevents us from setting the \indexflag{MIN-QUANT-ETREE} flag to simplify a
proof. It is a little daunting task to modify the code for \indexflag{MIN-QUANT-ETREE},
but the benefits are also clear: both \indexcommand{NAT-ETREE} and non-pfd procedures can
take advantage of the modification.

\subsection{The Original Nat-Etree}

{\bf Note: What follows is a description of how NAT-ETREE used to work.}
To use this code set \indexflag{NAT-ETREE-VERSION} to {\bf OLD}.

Legend has it that the code was written by Dan Nesmith and
influenced by the ideas of Frank Pfenning.  Frank's thesis
contains ideas for translating from a cut-free sequent calculus
to expansion tree proofs.

\begin{enumerate}
\item Important files: nat-etr (defines functions which are independent
of the particular rules of inference used); ml-nat-etr1 and
ml-nat-etr2 (which define translations for the rules in the standard TPS).

\item There are three global variables which are used throughout the
translation process: DPROOF, which is the nproof to
be translated; LINE-NODE-LIST, which is an association list which
associates each line of the proof to the node which represents it in
the expansion tree which is being constructed; MATE-LIST, which is a
list of connections in the expansion proof which is being constructed.

\item At the beginning of the translation process, the current proof is
copied because modifications will be made to it.  (It is restored when
the translation is complete.)  The copy is stored in the variable
DPROOF.  Next the function SAME-IFY is called.  This attempts to undo
the effects of the CLEANUP function, and to make explicit the
"connections" in the proof.  This is done because, in an nproof, 
a single line can represent more than one node in an
expansion proof.  SAME-IFY tries to add lines to the proof in such a
way that each line corresponds to exactly one expansion tree node.  

\item After the proof has been massaged by SAME-IFY, the initial root
node of the expansion tree is constructed.  This node is merely a
leaf whose shallow formula is the assertion of the last line of the
nproof.  LINE-NODE-LIST is initialized to contain
just the association of this leaf node with the last line of the
proof, and MATE-LIST is set to nil.

\item Next the function NAT-XLATE is called on the last line of the
proof.  NAT-XLATE, depending on the line's justification, calls
auxiliary functions which carry out the translation, and which usually
call NAT-XLATE recursively to translate lines by which the current
line is justified.  When the justification "Same as" is found, this
indicates that the node associated with this line and the node which
is associated with the line it is the same as should be mated in the
expansion proof.

\item Example:  Suppose we have the following nproof:
\begin{verbatim}
(1) 1  !  A            Hyp
(2)    !  A implies A  Deduct: 1

SAME-IFY will construct the new proof:

(1) 1  !  A            Hyp
(2) 1  !  A            Same as: 1 
(3)    !  A implies A  Deduct: 2
\end{verbatim}

Then a leaf node LEAF0 is constructed with shallow formula 
"A implies A", and LINE-NODE-LIST is set to ((3 . LEAF0)). 
NAT-XLATE is called, and because line 3 is justified using the
deduction rule, LEAF0 is deepened to an implication node, say IMP0,
with children LEAF1 and LEAF2.  Then LINE-NODE-LIST is updated to be
((1 . LEAF1) (2 . LEAF2) (3 . IMP0)), and NAT-XLATE is called
recursively on lines 1 and 2.  Since line 1 is justified by "Hyp",
NAT-XLATE does nothing.  Since line 2 is justified by "Same as: 1",
NAT-XLATE updates the value of MATING-LIST to (("LEAF1" . "LEAF2")), a
connection consisting of the nodes which represent lines 1 and 2.

\item In an nproof that is not cut-free, there will exist lines which do
not arise from deepening the expansion tree which represents the last
line of the nproof.  Currently, NAT-XLATE will get very confused and
probably blow up.  The justification "RuleP" causes other
difficulties, because it generally requires that several connections
be made, involving lines whose nodes haven't been deepened to the
literal level yet.  The function XLATE-RULEP attempts to do this, but
does not always succeed.  This is true because RULEP can also be used
to justify a line whose node is actually a child of the justifying
line, e.g.: 
\begin{verbatim}
(45)  ! A and B 
(46)  ! A        RuleP: 45
\end{verbatim}
Though XLATE-RULEP can handle this situation, it cannot handle more
complex ones such as:
\begin{verbatim}
(16)  ! A
(17)  ! A implies B
(18)  ! B            RuleP: 16 17
\end{verbatim}
Ideally, SAME-IFY would identify these situations before the
translation process is begun, but it does not.
\end{enumerate}


\section{Cut Elimination}
A cut elimination algorithm is worked out in Frank's
Thesis.  First he defines a notion of expansion development
(a sort of sequent calculus for sets of expansion trees
with rules for quantifiers, merging, and cuts).
Then he gives reductions on expansion developments with
the hope that these reductions result in an expansion tree.
Frank's algorithm is not currently implemented as part of TPS.
Of course, there are many ways of representing cuts and performing
cut elimination.

\subsection{An Example of a Loop in a Cut Elimination Algorithm}

One approach we tried (Chad, Summer 2001)
was to include explicit
CUT and MERGE nodes in expansion trees, defining redexes,
and doing cut elimination by contracting redexes.
This section contains a brief outline of the approach
and an example that shows how a loop can occur.

A CUT node is of the form
$$\etrca{CUT}{}{C^*}{}{B^*}{}{B^{**}}$$
where $C^*$ has shallow formula $C$,
and $B^*$ and $B^{**}$ have the same shallow
formula, but opposite polarity.  The
polarity and shallow formula
of the CUT node is the same as the
polarity of the shallow formula of $C^*$.
The deep formula of the cut node is
$$deep(C^*) \land [deep(B^*) \lor deep(B^{**})]$$
A MERGE node is of the form
$$\etrba{MERGE}{}{A^*}{}{A^{**}}$$
where $A^*$ and $A^{**}$ have the same polarity
and same shallow formula $A$.  The shallow
formula of the MERGE node is also $A$.
The polarity is also inherited from the children.
The deep formula is given by
$$deep(A^*) \land deep(A^{**}).$$

With these nodes, the translation from a natural deduction proof
(see subsection \ref{ceb-nat-etr})
can now be extended to translate backward coercions.
Ignoring hypothesis for simplicity,
suppose we have a backward coercion
$$ \ianc{B\Uparrow}{B\downarrow}{bcoercion}$$
a negative expansion tree $C^*$ with shallow
formula $C$, and a positive expansion tree $B^*$
with shallow formula $B$.  By induction hypothesis,
we can obtain an appropriate positive expansion tree
$B^{**}$ with shallow formula $B$.  Then the algorithm
returns
$$\etrca{CUT}{}{C^*}{}{B^*}{}{B^{**}}$$

There is one more modification required in the algorithm.
When translating a hypothesis line, we used the merge algorithm
on expansion trees.  However, merge is not defined on expansion
trees containing CUT nodes.  So, instead we make an explicit
MERGE node with the two trees as children.  The actual merging
would be done during cut/merge elimination.  That is,  if
we are translating a hypothesis line
$$A_1,\ldots, A_n \vdash A_j,$$
then we must start with positive expansion trees $A_i^{**}$,
a positive expansion tree $A_i^{*}$,
and a negative expansion tree $C^*$.
Instead of associating the hypothesis line with the merge
of $A_i^{*}$ and $A_i^{**}$, we now associate the line
with the node
$$\etrba{MERGE}{}{A_i^{*}}{}{A_i^{**}}$$

The most interesting case is eliminating a CUT between
a selection and an expansion node.  There may be
multiple expansion nodes and the acyclicity of the dependency
relation may affect the order in which the expansion terms
are processed.  We were trying to reduce such a CUT
by doing a global merge of two modified expansion trees.
Suppose the expansion tree is $Q$ and contains a CUT node
of the form
$$\etrca{CUT}{}{C}{}{\sel{}{\forall x . A(x)}{a}{P(a):A(a)}}{}
{\gexpnode{}{\forall x . A(x)}{t_1}{P_1:A(t_1)}{t_n}{P_n:A(t_n)}}$$
Let us use the notation $Q[CUT]$ to indicate this CUT node inside the
larger tree $Q$.
Assuming we do not have a problem with acyclicity, the CUT node
should reduce as follows:
$$\etrba{MERGE}{}{\{t_1/a\} Q[CUT1]}{}{Q[CUT2]}$$
where $CUT1$ is
$$\etrca{CUT1}{}{C}{}{P(t_1):A(t_1)}{}{P_1:A(t_1)}$$
and $CUT2$ is
$$\etrca{CUT2}{}{C}{}{\sel{}{\forall x . A(x)}{a}{P:A(a)}}{}
{\gexpnode{}{\forall x . A(x)}{t_2}{P_2:A(t_2)}{t_n}{P_n:A(t_n)}}.$$

However, this leads to a loop as in the following simple example.
Consider the following proof of 
$$ \forall \,x_{\greeki} \forall \,y_{\greeki} \,A_{\greeko\greeki\greeki} \,x \,y \supset \,A \,a_{\greeki} \,a \land \,A \,b_{\greeki} \,b$$
using the lemma
$$ \forall \,z_{\greeki} \,A_{\greeko\greeki\greeki} \,z \,z.$$

\vbox{\indent
\linenumbox {(1)}\hypnumbox {1}\turnstile\partformula{$ \forall \,x_{\greeki} \forall \,y_{\greeki} \,A_{\greeko\greeki\greeki} \,x \,y$}\lastformula\judgelink{Hyp}
}\filbreak
\vbox{\indent
\linenumbox {(2)}\hypnumbox {1}\turnstile\partformula{$ \forall \,y_{\greeki} \,A_{\greeko\greeki\greeki} \,z_{\greeki} \,y$}\lastformula\judgelink{UI:\ $  \,z_{\greeki}$\  1}
}\filbreak
\vbox{\indent
\linenumbox {(3)}\hypnumbox {1}\turnstile\partformula{$ \,A_{\greeko\greeki\greeki} \,z_{\greeki} \,z$}\lastformula\judgelink{UI:\ $  \,z_{\greeki}$\  2}
}\filbreak
\vbox{\indent
\linenumbox {(4)}\hypnumbox {1}\turnstile\partformula{$ \forall \,z_{\greeki} \,A_{\greeko\greeki\greeki} \,z \,z$}\lastformula\judgelink{UGen:\ $  \,z_{\greeki}$\  3}
}\filbreak
\vbox{\indent
\linenumbox {(5)}\hypnumbox {1}\turnstile\partformula{$ \,A_{\greeko\greeki\greeki} \,a_{\greeki} \,a$}\lastformula\judgelink{UI:\ $  \,a_{\greeki}$\  4}
}\filbreak
\vbox{\indent
\linenumbox {(6)}\hypnumbox {1}\turnstile\partformula{$ \,A_{\greeko\greeki\greeki} \,b_{\greeki} \,b$}\lastformula\judgelink{UI:\ $  \,b_{\greeki}$\  4}
}\filbreak
\vbox{\indent
\linenumbox {(7)}\hypnumbox {1}\turnstile\partformula{$ \,A_{\greeko\greeki\greeki} \,a_{\greeki} \,a \land \,A \,b_{\greeki} \,b$}\lastformula\judgelink{Conj: 5 6}
}\filbreak
\vbox{\indent
\linenumbox {(8)}\hypnumbox {}\turnstile\partformula{$ \forall \,x_{\greeki} \forall \,y_{\greeki} \,A_{\greeko\greeki\greeki} \,x \,y \supset \,A \,a_{\greeki} \,a \land \,A \,b_{\greeki} \,b$}\lastformula\judgelink{Deduct: 7}
}\filbreak

When translating this proof we obtain a tree $Q$ with two cut nodes
$$\etrca{CUT1}{}{R_1}{}{\etraa{SEL1}{c}{P_1(c):A\, c\, c}}{}{\etraa{EXP1}{a}{P_2:A\, a\, a}}$$
and
$$\etrca{CUT2}{}{R_2}{}{\etraa{SEL2}{d}{P_3(d):A\, d\, d}}{}{\etraa{EXP2}{b}{P_4:A\, b\, b}}$$
corresponding to the two applications of the lemma in line 4.
Contracting $CUT1$ gives
$$\etrba{MERGE}{}{\{a/c\}Q[CUT3]}{}{Q[CUT4]}$$
where $CUT3$ and $CUT4$ are
$$\etrca{CUT3}{}{\{a/c\}R_1}{}{P_1(a):A\, a\, a}{}{P_2:A\, a\, a}$$
and
$$\etrca{CUT4}{}{R_1}{}{\etraa{SEL3}{c}{P_1(c):A\, c\, c}}{}{EXP3}.$$
Both $CUT3$ and $CUT4$ are easy to eliminate.
($CUT3$ can be replaced by $\{a/c\}R_1$ and
$CUT4$ can be replaced by $R_1$, with appropriate
changes to the complete mating.)
However, note that $CUT2$ appears in both sides of the merge in
$$\etrba{MERGE}{}{\{a/c\}Q[CUT3]}{}{Q[CUT4]}.$$
Let us call these two occurrences $CUT2.1$ and $CUT2.2$.
Eventually, we will want to reduce one of these cuts
in some reduced tree $Q'[CUT2.1][CUT2.2]$.  Suppose we
reduce $CUT2.1$.  Similar to the reduction of $CUT1$ above,
this will copy $CUT2.2$ so that there are $CUT2.2.1$ and
$CUT2.2.2$.  The loop is evident.

It is conceivable that we could get around this loop by
developing a notion of ``expansion DAG'' (directed acyclic graph)
so that the CUT would not actually be duplicated.  But it isn't
clear that this would eliminate all such loops.  

Also, there are
some technical problems with the reductions above.  For instance,
a selection node may get copied, which means the result will be
a tree with two selection nodes that use {\it the same} selected
variable.  This doesn't seem to be a serious problem because we
could probably allow such a situation whenever the least common
ancestor of two such selection nodes is a MERGE node.  But these
details would have to be worked out to make this approach work.

\subsection{Cut and Mix Elimination in this Sequent Calculus}\label{ftree-seq-mix-elim}

There is a cut elimination algorithm implemented for the
sequent calculus described in section~\ref{ftree-seq}.
Suppose we have an instance
of the cut rule:

$$\ibnc{\above{\DD_1}{\Gamma_1\rightarrow A,\Delta_1}}{\above{\DD_2}{A,\Gamma_2\rightarrow \Delta_2}}{\Gamma_1,\Gamma_2\rightarrow \Delta_1,\Delta_2}{cut}$$

A cut elimination algorithm should take cut-free derivations $\DD_1$
and $\DD_2$ and return a cut-free derivation $\EE$ of 
$\Gamma_1,\Gamma_2\rightarrow\Delta_1,\Delta_2$.
This is usually done by a recursive algorithm on the two
derivations where at each step either one of the derivations
gets smaller, or the cut formula gets smaller (while the
derivations may get much larger).  Since we have an explicit
$merge$ (contraction) rule, we have the following problematic case.
Suppose $\DD_1$ ends with an application of $merge$:
$$\ianc{\above{\DD_3}{\Gamma_1\rightarrow A,A,\Delta_1}}{\Gamma_1\rightarrow A,\Delta_1}{merge-}$$
where $A$ is the cut formula.  Also, suppose $A$ is the principal formula of $\DD_2$.
The most natural way to handle this case is to first perform cut elimination
on $\DD_3$ and $\DD_2$, giving a cut-free derivation
$$\above{\DD_5}{\Gamma_1,\Gamma_2\rightarrow A,\Delta_1,\Delta_2}$$
Then we could call cut elimination again with $\DD_5$ and $\DD_2$
to obtain a cut-free derivation
$$\above{\DD_6}{\Gamma_1,\Gamma_2,\Gamma_2\rightarrow \Delta_1,\Delta_2,\Delta_2}$$
With some applications of $focus$ and $merge$ to shuffle and merge the formulas,
we would have the cut-free derivation of
$$\above{\DD_7}{\Gamma_1,\Gamma_2\rightarrow \Delta_1,\Delta_2}$$
we desire.  {\it However,} the derivation $\DD_5$ will in general be
bigger than $\DD_3$, while the cut formula $A$ remains the same.
So, this recursive call to cut elimination may not terminate (even with
a first-order derivation). % closed-thm1.prf gives an example

We can get around this problem by performing mix-elimination 
(as did Gentzen ~\cite{Gentzen69}) instead of cut-elimination.
Because we care about the order of formulas in the sequent,
mix elimination is complicated.  The main functions are
\indexfunction{ftree-seq-mix-elim-1} and \indexfunction{ftree-seq-mix-elim-principal}.
To understand mix elimination, 
first consider a few generic examples:


Suppose we have two cut-free derivations of
$$B\rightarrow A,C$$
and
$$D,A\rightarrow E$$
where $A$ is the mix formula, in negative position $0$ in the
first sequent and positive position $1$ in the second sequent.
Mix elimination might return a cut-free derivation of
$$B,D\rightarrow C,E$$
along with two lists of indices
\begin{enumerate}
\item $((t\; . \; 0) \; (nil\; . \; 0))$ (indicating that $B$ corresponds
to the $0$'th positive formula of the first sequent, and
$D$ corresponds to the $0$'th positive formula of the second sequent)
\item $((t\; . \; 1) \; (nil\; . \; 0))$ (indicating that $C$ corresponds
to negative position $1$ in the first sequent and
$E$ corresponds to negative position $0$ in the second sequent)
\end{enumerate}
We say ``might'' because the return value depends, of course,
on the given derivations, not just the sequents.
Other possible outputs include a cut-free derivation of
$$B\rightarrow C,C$$
with two lists of indices
\begin{enumerate}
\item $((t\; . \; 0))$ (again, $B$ corresponds to positive position $0$ in
the first sequent), and
\item $((t\; . \; 1)\; (t\; . \; 1))$ (both occurrences of $C$ correspond
to negative position $1$ in the first sequent).
\end{enumerate}

The point is that we have eliminated the two mix formulas
(two occurrences of $A$) and retain only residues of the other
formulas in the two given sequents.  The other formulas may occur
several times, or not at all.  The lists of indices indicate
where the formulas originally occured.  An ``index'' is a pair
$(<bool>\; . \; <nat>)$ where
\begin{itemize}
\item If $<bool>$ is $t$, the formula is the residue of a formula
in the first sequent.
\item If $<bool>$ is $nil$, the formula is the residue of a formula
in the second sequent.
\item $<nat>$ is a natural number indicating the position of the
formula in the first or second sequent.
\end{itemize}

For another example, given two cut-free derivations of
$$B,C\rightarrow A,A,D$$
and
$$E,A,F\rightarrow G$$
with mix formulas
might return
\begin{itemize}
\item
\begin{itemize}
\item a cut-free derivation of $C,F,E\rightarrow G,D$,
\item positive indices $((t\; .\; 1)\; (nil\; .\; 2)\; (nil \; . \; 0))$,
and
\item negative indices $((nil\; . \;0)\; (t\; . \; 2))$
\end{itemize}
\item or,
\begin{itemize}
\item a cut-free derivation of $B,B\rightarrow G$,
\item positive indices $((t\; . \;0)\; (t\; . \; 0))$, and
\item negative indices $((nil \; . \; 0))$
\end{itemize}
\item or,
\begin{itemize}
\item a cut-free derivation of $F,E\rightarrow D,D$,
\item positive indices $((nil\; . \; 2)\; (nil\; . \; 0))$, and
\item negative indices $((t\; . \; 2)\; (t \; . \; 2))$.
\end{itemize}
\end{itemize}

In general, we start with two derivations
$$\above{\DD_1}{\Gamma_1\rightarrow \Delta_1}$$
and
$$\above{\DD_2}{\Gamma_2\rightarrow \Delta_2}$$
and two lists $(i_1\cdots i_k)$ and $(j_1\cdots j_l)$
of natural numbers.  The natural numbers give us
the positions of the mix formulas in $\Delta_1$ and $\Gamma_2$.
Let $\Delta_1$ be a list
$(A_1 \cdots A_n)$ and $\Gamma_2$ be a list
$(B_1 \cdots B_m)$.  The mix formulas are
$A_{i_r}$ and $B_{j_s}$.  These formulas should have
a common reduct with respect to $\lambda$-reduction,
expanding abbreviations, and expansion equalities
using either extensionality or Leibniz.
Mix elimination returns a cut-free derivation of
$$\Gamma\rightarrow\Delta$$
and two lists of $indl^+$ and $indl^-$
of indices indicating the preimage of
the formulas in $\Gamma$ and $\Delta$.
$\Gamma$ and $indl^+$ are lists of the same length.
For each $B$ in $\Gamma$ there is a corresponding
index $(b . j)$ where either $b$ is $t$
and $j$ is the position of $B$ in $\Gamma_1$
or $b$ is $nil$ and $j\not\in\{j_1,\ldots, j_l\}$ is the position of
$B$ in $\Gamma_2$.
Similarly, $\Delta$ and $indl^-$ have the same length.
For each $A$ in $\Delta$ there is a corresponding index
$(a . i)$ where either $a$ is $t$ and $i\not\in\{i_1,\ldots, i_k\}$ is the position
of $A$ in $\Delta_2$
or $a$ is $nil$ and $i$ is the position of $A$ in $\Delta_2$.

The main function which attempts to eliminate all cuts from
a derivation is \indexfunction{ftree-seq-cut-elim}.
This is a simple recursive algorithm which first eliminates
cuts from premisses, then either imitates the rule
or, if the rule was $cut$, uses mix elimination to obtain
a cut-free derivation from the cut-free derivations of the
premisses.  That is,
given two derivations
$$\above{\DD_1}{\Gamma_1\rightarrow A,\Delta_1}$$
and
$$\above{\DD_2}{A,\Gamma_2\rightarrow \Delta_2}$$
we call mix elimination with these derivations
and two lists of positions $(0)$ and $(0)$.
Mix elimination returns a cut-free derivation of
$\Gamma\rightarrow\Delta$
and two lists of indices.
Using the indices and the $focus$, $weaken$, and $merge$ rules, we can
shuffle the formulas in $\Gamma$ and $\Delta$
to obtain a derivation of
$\Gamma_1,\Gamma_2\rightarrow \Delta_1,\Delta_2$.
This final shuffling is performed by
\indexfunction{ftree-seq-mix-elim-finish}.

\subsection{The Mix Elimination Algorithm}

The mix elimination algorithm works by recursion on the two
given derivations.  Suppose we are given two cut-free derivations
$$\above{\DD_1}{\Gamma_1\rightarrow \Delta_1}$$
and
$$\above{\DD_2}{\Gamma_2\rightarrow \Delta_2}$$
and two lists $nl_1 = (i_1\cdots i_k)$ and $nl_2 = (j_1\cdots j_l)$
indicating the positions of the mix formulas in $\Delta_1$
and $\Gamma_2$.

There are several cases to consider:

\begin{itemize}
\item {\bf Mix Formula Does Not Occur}.  That is, $nl_1 = nil$ or $nl_2 = nil$.
If $nl_1$ is $nil$, then we can return
\begin{itemize}
\item $$\above{\DD_1}{\Gamma_1\rightarrow\Delta_1}$$
\item $((t\; .\; 0)\; \cdots \; (t\; . \; (n - 1)))$ where $n$ is the length of $\Gamma_1$
\item $((t\; .\; 0)\; \cdots \; (t\; . \; (m - 1)))$ where $m$ is the length of $\Delta_1$
\end{itemize}
The indices indicate that every formula is the residual of a formula from
the first sequent.
If $nl_2$ is $nil$, we can return $\DD_2$ and indices which indicate
all the formulas are residuals of the second sequent.
\item {\bf INIT}  Suppose $\DD_1$ is an initial sequent $A\rightarrow A$.
($A$ must be the mix formula, or $nl_1$ must be $nil$.)
Ideally, we would like to return $\DD_2$, and replace the indices
for the mix formula occurrences in $\Gamma_2$ by the index $(t\; .\; 0)$
indicating the positive $A$ from $\DD_1$.
The only problem is that each occurrence of the mix formula in
$\Gamma_2$ is a $B$ equal to $A$ only in the sense that the
two formulas have a common reduct with respect to
$\lambda$-reduction
and expansions of abbreviations and equalities.
The function \indexfunction{ftree-seq-replace-equivwffs} replaces
each such $B$ by $A$ in $\DD_2$.
We can handle $\DD_2$ initial similarly.
\item {\bf FOCUS, WEAKEN, MERGE}  If either $\DD_1$ or $\DD_2$ is
a $focus$, $weaken$, or $merge$ step, we can simply recursively
call with the premiss and the positions shuffled appropriately.
\item {\bf REWRITES}  If either $\DD_1$ or $\DD_2$
is a $\lambda$, $EQUIVWFFS$, $Leibniz=$, or $Ext=$ rewrite,
and the mix formula is the principal formula, we can usually simply
recursively call the algorithm with the premiss and the same positions.
This is possible because we are only maintaining that the mix formulas
are the same up to having a common reduct via such rewriting steps.
The actual rewriting is done in the INIT step.

{\bf Exception:  Nonanalytic Uses of Extensionality}  Suppose one
occurrence of the mix formula is $A[=]$
and another occurrence of the mix formula is 
$A[\forall q\; . \; q \; M_\greeko \;\limplies\; q\; N_\greeko].$
If $A[=]$ is the principal formula of an $Ext=$ rewrite, then 
$$A[\equiv]$$ and $$A[\forall q\; . \; q \; M_\greeko \;\limplies\; q\; N_\greeko]$$
{\it do not} have a common reduct.  The recursion fails at such a step.
In general this problem occurs when one instance of equality
is expanded as Leibniz and another corresponding instance is
expanded as (functional or propositional) extensionality.
The real problem is that the extensionality rules by themselves
without cut
do not give a complete calculus for extensional higher-order logic.
To get a cut-free extensional calculus, one needs to be able to
have initial sequents (i.e., ``mate'') {\em modulo equations}
and then use decomposition rules and extensionality rules to
solve the introduced equations.
Benzmuller's thesis gives rules for handling extensionality
in the context of resolution.
The case for sequent calculi should be described in upcoming technical reports by
Benzmuller, Brown and Kohlhase.
There should also be information in Chad E. Brown's thesis.
\item {\bf NONPRINCIPAL LOGICAL RULE}  If either $\DD_1$ or $\DD_2$
ends with a logical rule and the principal formula is not a mix formula,
then we can recursively call mix elimination on the premisses and
imitate the rule.  For example, if $\DD_1$ is
$$\ibnc{\above{\DD_{11}}{\Gamma_{11}\rightarrow A,\Delta_{11}}}{\above{\DD_{12}}{\Gamma_{12}\rightarrow B,\Delta_{12}}}{\Gamma_{11},\Gamma_{12}\rightarrow A\land B,\Delta_{11},\Delta_{12}}{\land-}$$
the function \indexfunction{ftree-seq-invert-position-list}
uses the positions of the mix formulas in
$$A\land B,\Delta_{11},\Delta_{12}$$
to find the positions of the mix formulas in
$\Delta_{11}$ and $\Delta_{12}$.
Using this we can call mix elimination twice to obtain
$$\above{\DD_3}{\Gamma_3\rightarrow\Delta_3}$$
and
$$\above{\DD_4}{\Gamma_4\rightarrow\Delta_4}$$
The function \indexfunction{ftree-seq-mix-elim-imitate-rule}
finishes this case.  First, \indexfunction{ftree-seq-bring-to-front} uses
structural rules to bring the residuals of $A$ and $B$ to the front of
$\Delta_3$ and $\Delta_4$, so we have
$$\above{\DD'_3}{\Gamma_3\rightarrow A,\Delta'_3}$$
and
$$\above{\DD'_4}{\Gamma_4\rightarrow B,\Delta'_4}$$
At this point, we apply the $\land-$ rule:
$$\ibnc{\above{\DD'_{3}}{\Gamma_3\rightarrow A,\Delta'_3}}{\above{\DD'_4}{\Gamma_4\rightarrow B,\Delta'_4}}{\Gamma_3,\Gamma_4\rightarrow A\land B,\Delta'_3,\Delta'_4}{\land-}$$
and compute the indices of the residuals.
\item {\bf PRINCIPAL} The final, and most important case, is when both mix formulas
are principal.  This is handled by the function
\indexfunction{ftree-seq-mix-elim-principal}.
\end{itemize}

In each of the cases above, one of the derivations gets smaller
in the recursive call.
The case when both mix formulas are principal is complicated
by the need to perform several recursive calls, requiring us
to adjust and compose the indices as we go along.  The cases
for each connective and quantifier are described next.

\begin{itemize}
\item $\top$,$\bot$:  This cannot happen, because the only
way $\top$ can be positive principal in $\DD_2$ is if the
last step is $focus$, $weaken$, or $merge$, which were handled
above.  Similarly, if $\bot$ is negative principal in $\DD_1$,
then it must end with a $focus$, $weaken$, or $merge$.
\item $REW(\equiv)$:  $\DD_1$ is
$$\ianc{\above{\DD_{11}}{\Gamma_1\rightarrow [A\limplies B]\land [B\limplies A],\Delta_1}}{\Gamma_1\rightarrow A\equiv B,\Delta_1}{REW(\equiv)-}$$
$\DD_2$ is
$$\ianc{\above{\DD_{21}}{[A\limplies B]\land [B\limplies A],\Gamma_2\rightarrow \Delta_2}}{A\equiv B,\Gamma_2\rightarrow \Delta_2}{REW(\equiv)+}$$
First we recursively call mix elimination for all the unexpanded formulas $A'\equiv B'$
with $\DD_1$ and $\DD_{21}$ (smaller than $\DD_2$)
giving $\DD_3$:
$$[A\limplies B]\land [B\limplies A],\Gamma_3\rightarrow \Delta_3$$
Next, recursively call mix elimination for all the unexpanded formulas $A'\equiv B'$ with $\DD_{11}$ (smaller than $\DD_1$) and $\DD_2$
giving $\DD_4$:
$$\Gamma_4\rightarrow [A\limplies B]\land [B\limplies A],\Delta_4$$
Finally, we can call mix elimination for the two occurrences of the
``smaller'' mix formula $[A\limplies B]\land [B\limplies A]$
with $\DD_4$ and $\DD_3$ 
giving
$\DD_5$:
$$\Gamma_5\rightarrow \Delta_5$$
Of course, with if we weigh $\equiv$ more than $\land$ and $\limplies$,
we can say $[A\limplies B]\land [B\limplies A]$ is ``smaller'' than $A\equiv B$.
So, this case does not cause a problem with termination.
The hard part is tracing the residuals to return the proper indices.
Each formula in $C$ in $\Gamma_5$ is either the residual of some
$C$ in $\Gamma_4$ or $C$ in $\Gamma_3$.  If the preimage is in $\Gamma_4$,
then $C$ is a residual of a $C$ in $\Gamma_1$ or $\Gamma_2$.  Once
we compute the preimage of the preimage, we have the proper index.

The following diagram is helpful when trying to compute preimages:
$$\ibnc{\ibnc{\DD_{11}}{\DD_2}{\DD_4}{elim}}{\ibnc{\DD_1}{\DD_{21}}{\DD_3}{elim}}{\DD_5}{elim}$$
\item $EXP-,SEL+$ or $SEL-,EXP+$:  Suppose $\DD_1$ is
$$\ianc{\above{\DD_{11}}{\Gamma_1\rightarrow A(t),\Delta_1}}{\Gamma_1\rightarrow \exists x A(x),\Delta_1}{EXP-^t}$$
and $\DD_2$ is
$$\ianc{\above{\DD_{21}}{A(a),\Gamma_2\rightarrow \Delta_2}}{\exists x A(x),\Gamma_2\rightarrow \Delta_2}{SEL+^a}$$
As described in the $REW(\equiv)$ case,
we must first eliminate all the nonprincipal occurrences
of the mix formulas by recursive calls using $\DD_1$ and $\DD_{21}$
to obtain $\DD_3(a)$:
$$A(a),\Gamma_3\rightarrow \Delta_3$$
and with $\DD_{11}$ and $\DD_2$ to obtain $\DD_4$:
$$\Gamma_4\rightarrow A(t),\Delta_4$$
Then we substitute $t$ for $a$ in $\DD_3(a)$ to obtain $\DD_3(t)$,
and recursively call to eliminate the two occurrences of
$A(t)$ using $\DD_4$ and $\DD_3(t)$.

{\bf Remark About Termination:}  Usually $A(t)$ will be ``smaller''
than $\exists x A(x)$.  But, of course, in higher order logic there
are cases where the formula is most certainly not smaller.
The most obvious example is when $A$ is $\exists \; x_\greeko \; x$
and $t$ is also $\exists \; x_\greeko\; x$, so that
$A(t)$ is actually the same as $\exists x A(x)$.
We don't actually know whether the algorithm always
terminates.

As in the $REW(\equiv)$ case,
computing the indices involves computing preimages of preimages
using the following diagram
$$\ibnc{\ibnc{\DD_{11}}{\DD_2}{\DD_4}{elim}}{\ibnc{\DD_1}{\DD_{21}}{\DD_3}{elim}}{\DD_5}{elim}$$
The $SEL-,EXP+$ case is similar.
\item $\lnot$:  This case is relatively simple, we first make two
recursive calls to eliminate the nonprincipal occurrences of the
mix formula $\lnot A$ giving $\DD_3$:
$$\Gamma_3\rightarrow A,\Delta_3$$
and $\DD_4$:
$$A,\Gamma_4\rightarrow \Delta_4$$
Finally, we eliminate the two occurrences of $A$ (smaller than $\lnot A$)
in $\DD_3$ and $\DD_4$ (the order is the opposite of the previous cases).
Again, the indices are preimages of preimages, though we must
be careful to account for the changes in lengths of the two sides
as $\lnot A$ in the final sequents and $A$ in the premisses are on opposite
sides.  The diagram that applies here is
$$\ibnc{\ibnc{\DD_1}{\DD_{21}}{\DD_3}{elim}}{\ibnc{\DD_1}{\DD_{21}}{\DD_4}{elim}}{\DD_5}{elim}$$
\item $\land$:  There are three relevant premisses here.
$\DD_1$ has the form
$$\ibnc{\above{\DD_{11}}{\Gamma_{11}\rightarrow A,\Delta_{11}}}{\above{\DD_{12}}{\Gamma_{12}\rightarrow B,\Delta_{12}}}{\Gamma_{11},\Gamma_{12}\rightarrow A\land B,\Delta_{11},\Delta_{12}}{\land-}$$
and $\DD_2$ has the form
$$\ianc{\above{\DD_{21}}{A,B,\Gamma\rightarrow \Delta}}{A\land B,\Gamma\rightarrow \Delta}{\land+}$$
Recursive calls eliminate the nonprincipal occurrences of the
mix formula.
\begin{itemize}
\item $\DD_3$: ($\DD_1$ mix $\DD_{21}$ to eliminate $A\land B$) 
$\Gamma_3\rightarrow\Delta_3$
where $\Gamma_3$ contains residuals of $A$ and $B$.
\item $\DD_4$: ($\DD_{11}$ mix $\DD_2$ to eliminate $A\land B$) $\Gamma_4\rightarrow \Delta_4$
where $\Delta_4$ contains residuals of $A$
\item $\DD_5$: ($\DD_{12}$ mix $\DD_2$ to eliminate $A\land B$) $\Gamma_5\rightarrow \Delta_5$
where $\Delta_5$ contains residuals of $B$
\item $\DD_6$: ($\DD_5$ mix $\DD_3$ to eliminate residuals of $B$)
$\Gamma_6\rightarrow\Delta_6$
where $\Gamma_6$ contains residuals of $A$
\item $\DD_7$: ($\DD_4$ mix $\DD_6$ to eliminate residuals of $A$)
$\Gamma_7\rightarrow\Delta_7$
where there are no residuals of $A$, $B$, or $A\land B$.
\end{itemize}
Since there were more recursive calls in this case, we must
compute preimages of preimages of preimages in some cases,
and preimages of preimages in other cases, as indicated by the following
diagram:
$$\ibnc{\ibnc{\DD_{11}}{\DD_2}{\DD_4}{}}
{\ibnc{\ibnc{\DD_{12}}{\DD_2}{\DD_5}{}}{\ibnc{\DD_1}{\DD_{21}}{\DD_3}{}}
{\DD_6}{}}{\DD_7}{}$$
Each wff in the sequent proven by $\DD_7$ can be traced back to
a non-mix formula in either $\DD_1$ or $\DD_2$ using this diagram.
\item $\lor$:  This case is similar to the $\land$ case, but $\DD_1$
has one premiss and $\DD_2$ has two premisses.   $\DD_1$ is
$$\ianc{\above{\DD_{11}}{\Gamma\rightarrow A,B,\Delta}}{\Gamma\rightarrow A\lor B,\Delta}{\lor-}$$
and $\DD_2$ is
$$\ibnc{\above{\DD_{21}}{A,\Gamma_1\rightarrow \Delta_1}}{\above{\DD_{22}}{B,\Gamma_2\rightarrow \Delta_2}}{A\lor B,\Gamma_1,\Gamma_2\rightarrow \Delta_1,\Delta_2}{\lor+}$$
The following diagram indicates the order of the recursive calls.
$$\ibnc{\ibnc{\ibnc{\DD_{11}}{\DD_2}{\DD_4}{}}{\ibnc{\DD_1}{\DD_{22}}{\DD_5}{}}
{\DD_6}{}}{\ibnc{\DD_1}{\DD_{21}}{\DD_3}{}}{\DD_7}{}$$
\item $\limplies$:
This is similar to the $\lor$ case.  The following diagram
indicates the order of the recursive calls.
$$\ibnc{\ibnc{\DD_1}{\DD_{21}}{\DD_3}{}}
{\ibnc{\ibnc{\DD_{11}}{\DD_2}{\DD_4}{}}{\ibnc{\DD_1}{\DD_{22}}{\DD_5}{}}{\DD_6}{}}
{\DD_7}{}$$
\end{itemize}


\section{Cut-free Extensional Sequent Derivations to Extensional Expansion Proofs}

There is an implementation of the extensional sequent calculus
in Chad E. Brown's thesis (cf.~\cite{Brown2004a}) in \TPS.
The structure {\indexother{ext-seq}} defined in \indexfile{ext-exp-dag-macros.lisp}
represents sequent derivations in this extensional sequent calculus.
Sequent derivations can be created, manipulated, saved and restored in 
the \indexother{EXT-SEQ} top level.  If a sequent derivation is cut-free,
then the command \indexcommand{CUTFREE-TO-EDAG} (implemented in \indexfile{ext-seq-top.lisp})
will translate the sequent derivation to an extensional expansion proof.
The proof that this translation works is in Chapter 7 Section 10 of Chad E. Brown's thesis (cf.~\cite{Brown2004a}).

\section{Extensional Expansion Proofs to NProofs}

\TPS can translate an extensional expansion proof to a natural deduction proof
using the code in the file \indexfile{ext-exp-dags-nd.lisp}.
This code is automatically called when either of the extensional search procedures
\indexother{MS04-2} or \indexother{MS03-7} successful find a proof.  The translation code can also be explicitly
called from the \indexother{EXT-MATE} top level using the command \indexcommand{ETREE-NAT}.

There is an algorithm described in Chapter 7 Section 9 of Chad E. Brown's thesis (cf.~\cite{Brown2004a})
which translates extensional expansion proofs to extensional sequent derivations.
This algorithm is not implemented as part of \TPS, but the same ideas are used for
the algorithm translating from extensional expansion proofs to natural deduction proofs.

