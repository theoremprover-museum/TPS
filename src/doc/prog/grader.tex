\chapter{The Grader Program}

(Programmers should be aware that the GRADER program has its own manual.)

\section{The Startup Switch}

In theory, adding the switch {\tt -grader} to the command line which 
starts up \TPS should start up the Grader program directly. The code which 
implements this is in \indexfile{tps3-save.lisp}.

In practice, some modifications may be needed depending on the particular 
Lisp being used. For example:

\begin{itemize}
\item When starting up in CMUlisp on an IBM RT, the error {\tt "Switch does not exist"}
will be given. This is just Lisp complaining that it doesn't recognize the switch;
it passes the switch on to \TPS anyway, so this is no cause for concern.

\item When using Allegro Lisp version 4.1 or later, a {\tt --} symbol is used to separate
Lisp options from user options. So, on early versions of Allegro Lisp the line to
start up grader is:
{\tt xterm {\it <many xterm switches>} -e /usr/theorem/bin/run-tps -grader \&}
whereas for later versions it is:
{\tt xterm {\it <many xterm switches>} -e /usr/theorem/bin/run-tps -- -grader \&}
\end{itemize}
