\documentclass{article}
%Instructions for Using ETPS on Andrew

\usepackage{amssymb}
\usepackage{latexsym}
\usepackage{makeidx}

%the following package permits us to use the comment environment
\usepackage{verbatim}

%the following package permits us to use the alltt environment
%which is like verbatim but has certain technical advantages
\usepackage{alltt}

\input /home/theorem/tps/doc/lib/tps.tex
\input /home/theorem/tps/doc/lib/tpsdoc.tex

\setlength{\textwidth}{6.7in}
\setlength{\oddsidemargin}{0.1in}
\setlength{\evensidemargin}{0.1in}
\setlength{\topmargin}{-0.5in}
%\setlength{\footskip}{-0.5in}
%\setlength{\textheight}{9.4in}
\setlength{\textheight}{8.0in}
\setcounter{secnumdepth}{5}

\makeindex % generate index data

\begin{document}

\title{Using {\ETPS} on Andrew}
%\author{Peter B. Andrews}
\date{}				%%if you don't want the date to appear
\maketitle


\section{Accessing {\ETPS}}

While {\ETPS} can be installed on many computers, you must use the
{\ETPS} which is installed on Andrew in order to get credit for your work
when you issue the DONE command. However, you can access it remotely,
as is discussed below.

The Unix X11-window system makes it possible for you to run {\ETPS} in a
window with fonts that have special symbols for the logical
connectives and quantifiers. Formulas will appear just as they do in
the textbook. In addition, special windows called proofwindows
will display the proof, or relevant parts of it, as you construct it.
It is highly recommended that you run {\ETPS} using these facilities.
To do this, use XETPS (or XETPS-BIG) as described below.

\subsection{Accessing {\ETPS} on Andrew Linux Workstations}

{\ETPS} can be run on Andrew Linux workstations.
You may  find that you can also run it on other Linux-based
workstations which are connected to the campus network.
If you are not logged in on an Andrew workstation, you should do\newline{}
klog {\tt\char`\<}andrew userid{\tt\char`\>} -c andrew.cmu.edu\newline{}
so that {\ETPS} will be able to write in the score file when you
execute the DONE command.

To make things easy, add the line\newline{}
{\tt source /afs/andrew.cmu.edu/mcs/math/etps/etps-no-xset.cshrc}\newline{}
to your $\sim$/.cshrc file on Andrew. (If you don't have such a file, create one
with this line in it.)
Then log out and log in again, so
that the system will read the new .cshrc file.
(The x-server reads the .cshrc file when it starts up an xterm.)
The system will now set up aliases for you, so that the command\newline{}
{\tt etps}\newline{}
will start up {\ETPS} without you having to remember the exact path name.

If you have changed your Andrew Linux shell to something other than csh,
execute the command  \newline{}
{\tt csh -l}\newline{}
whenever you login on Andrew Linux to use  {\ETPS}. (If you are already using
csh, executing this command will be redundant, but won't do any harm.)

Actually, instead of using {\ETPS} it is best to run an enhanced
version of {\ETPS} called  XETPS which displays formulas using
special fonts. To run  XETPS when working directly on an
Andrew Linux Workstation, at the shell prompt first issue the
command \newline{}
{\tt etfonts} \newline{}
(which is discussed further in the section on
Fonts below) and then issue the command \newline{}
{\tt xetps}.

An alternative to {\tt xetps} is {\tt xetps-big}.  This uses very
large X fonts, and the window will take up most of your screen.

\begin{comment}



%This subsubsection became obsolete when the fonts were made part of 
% the Andrew computing environment. The change from  POSIX to UTF-8 locale to display fonts
seemed to make setting the fontpath necessary again, but this
subsection is still not used.


\subsubsection{Setting the Fontpath}

If you sourced the file /afs/andrew.cmu.edu/mcs/math/etps/etps.cshrc in your
$\sim$/.cshrc file, it will tell your X server where the fonts are.  
If you did not source it, but you wish to use XETPS or XETPS-BIG,
you should place the line\newline
\verb=/afs/andrew.cmu.edu/mcs/math/etps/etps.Xclients=\newline
in your $\sim$/.Xclients
file 
% \comment{Next phrase has not been tested}
 or your $\sim$/.xinitrc file
(if you are using one), or issue that command at the shell prompt once
the window system has started up.
Alternatively, just issue the command:\newline{}
{\tt xset +fp /afs/andrew.cmu.edu/mcs/math/etps/fonts/decfonts/}\newline{}
or\newline{}
{\tt /usr/misc/bin/xset fp+ /afs/andrew.cmu.edu/mcs/math/etps/fonts/decfonts/}

\pagebreak

\end{comment}

\subsubsection{Commands to Start {\ETPS} Directly}

If you have not set up the aliases, you can start 
{\ETPS} on Andrew with the following command:

{\tt /afs/andrew.cmu.edu/mcs/math/etps/bin/etps}

Similarly, you can start up XETPS with the command:\newline
{\tt xterm -geometry 80x54 -rw -fn vtsingle -fb vtsymbold -sb -n ETPS -T ETPS -e \newline
/afs/andrew.cmu.edu/mcs/math/etps/bin/etps \&}\newline
(which is all one line).



\subsection{Accessing {\ETPS} Remotely}

If you wish to run {\ETPS} remotely, you can connect 
from your local machine to one of the Andrew `unix servers' 
with the command \newline
{\tt ssh unix.andrew.cmu.edu}\newline
and run {\ETPS} there. The Unix servers are set up to allow multiple
users and are named unix1, unix2, unix3, ... .

\subsubsection{Using Uptime}

Use the {\tt uptime} command after logging in
on one of the Unix servers to learn what its  current load is.

\begin{comment}
\subsubsection{An Alternative to etps.cshrc}

If you will be connecting to an Andrew workstation from another
machine, you may find
that the xset command in etps.cshrc causes confusion for the local
machine, resulting in error messages. In this case, you can source\newline
/afs/andrew.cmu.edu/mcs/math/etps/etps-no-xset.cshrc\newline
instead of\newline
/afs/andrew.cmu.edu/mcs/math/etps/etps.cshrc\newline
in your $\sim$/.cshrc file on Andrew.
In this case, your font path will not automatically contain the {\ETPS}
fonts when you do log on to an Andrew workstation, but
an alias {\tt get-etps-fonts} is defined in etps-no-xset.cshrc,
so you can simply issue the command {\tt get-etps-fonts} when you log
onto an Andrew workstation to run {\ETPS}.

\end{comment}

\subsubsection{Fonts}

If you run {\ETPS} remotely, you can always use the
GENERIC style to display wffs. However, if X-windows are available on
your local machine, it will probably be worth your while to set things up so
that you can use the style XTERM to display wffs, particularly as the
wffs you deal with become more complex. You will need to install the
fonts on the local machine, and probably adjust the font path, in
order to use XETPS successfully. (Of course, this need only be done once.)
Details are discussed below.

Many recent Linux systems are using a UTF-8 locale to display fonts,
but the ETPS fonts seem to work only in the traditional POSIX
locale. To get the standard POSIX behavior while running on a Linux
machine which is currently using the UTF-8 locale, one can execute the
Linux command {\tt setenv LC{\_}ALL C}. The  {\tt etfonts} command on 
Andrew Linux workstations is an alias which does this, and also 
sets the fontpath appropriately.



\subsubsection{Accessing {\ETPS} Remotely Using Unix}

You will need to copy the ETPS fonts from
/afs/andrew.cmu.edu/mcs/math/etps/fonts to your local machine and use an
xset command analogous to that below (with the path adjusted
appropriately) to tell your X server where the fonts are: \newline{}
{\tt xset +fp /afs/andrew.cmu.edu/mcs/math/etps/fonts/decfonts/}\newline{}
or\newline{}
{\tt /usr/misc/bin/xset fp+ /afs/andrew.cmu.edu/mcs/math/etps/fonts/decfonts/}

If your local machine is a Linux
machine which is currently using the UTF-8 locale, you should change this
locale on your local machine temporarily by  executing the
Linux command \newline{}
{\tt setenv LC{\_}ALL} C \newline{}
before connecting to one of the Unix servers.
(See the discussion  above in the section on Fonts.) Then make the connection 
with the command \newline{}
{\tt ssh unix.andrew.cmu.edu}.

%\paragraph{Xhost}

If you connect to unix17 (for example) by using telnet,
you may need to issue the command\newline{}
{\tt xhost unix17.andrew.cmu.edu}\newline{}
on your local machine  so that  it will permit unix17 to open a window on your screen
when you start up XETPS. If you are using ssh instead of
telnet, this will probably not be necessary.

Now start XETPS with the command \newline{}
{\tt xetps}.

In certain circumstances, you may find that you need to 
issue the command {\tt etfonts} before  issuing the command 
{\tt xetps} or {\tt etps}.
%\pagebreak

\subsubsection{Accessing {\ETPS} Remotely Using Windows}


If your local machine is a PC running some version of Windows, you may
be able to use XWin32 to bring up simulated X-windows on your local machine.
First copy the ETPS fonts to your Windows machine as discussed below.
Then connect to one of the unix servers using XWin32, and issue the
command \newline{}
{\tt xetps}.

(At one time it was necessary to issue the command {\tt etfonts} prior to
issuing the command {\tt xetps}, but this no longer seems to be necessary.)

The details of copying the ETPS fonts to your Windows machine may vary
somewhat for different versions of Windows, but here is some
potentially helpful information from students who have done this
previously:

\paragraph{One Student's Experience Acessing XETPS with a Windows Machine}

I got the fonts for {\ETPS} to work on my Windows machine! It was actually
really simple. All I did was download all the fonts from
/afs/andrew.cmu.edu/mcs/math/etps/fonts to the font directory for XWin32,
which is by default C:{\tt\char`\\}Program Files{\tt\char`\\}StarNet{\tt\char`\\}X-Win32 5.1{\tt\char`\\}Fonts{\tt\char`\\}.
Then, from the Configuration screen for X-Win32 (X-Config), under the Font
tab, I selected each of the font paths it listed and hit the `Make
FONTS.DIR' button.

After that, XETPS worked fine with the fonts and all.

I think I actually did more than was necessary to get it to work,
since I don't think you need ALL the fonts in the {\ETPS} font directory,
just whichever vtsymbol font {\ETPS} uses, but I'm not sure which since
there are around 5 different vtsymbol fonts in there and I didn't take
time to see exactly what was needed.

\paragraph{Another Student's Experience Acessing XETPS with a Windows Machine}

Ftp all the fonts and the font.dir (although I think font.dir's not
necessary)  from the decfonts directory to the \linebreak
XWin32{\tt\char`\\}lib{\tt\char`\\}fonts{\tt\char`\\}etps-fonts
directory, where you have created the etps-fonts directory.

Go through the described method of adding fonts to the directory
described in the README file for XWin32, which
is:  run X-config, add this newly-created font dir etps-fonts to the path
(the `Font' tab in the newest version 5.03), rebuild font.dir (click on
apply button), close and restart XWin32, and proceed as usual (setenv
DISPLAY {\tt\char`\<}hostname{\tt\char`\>}:0.0, xetps, etc.).

This at least worked on my home machine.  I am not sure if we have
enough of the permissions necessary to do this in the cluster.

Now, when you setflag style to xterm again, it should work.

\subsubsection{Accessing {\ETPS} Remotely Using a Macintosh}

You may also be able to access {\ETPS} remotely using a Macintosh.

\paragraph{One Student's Experience Acessing XETPS with a Macintosh}

I got ETPS working with the nice xterm fonts on my Macintosh last
night. It was actually quite easy. I put the fonts that were
distributed for the other operating systems in /Library/Fonts/, opened
up the OS X X11 application, and then proceeded as if I was in a UNIX
environment. Which is to say: I used ssh with X forwarding to connect
to unix.andrew.cmu.edu, and then ran xetps.


\pagebreak

\section{Using {\ETPS}}

Remember that your work on {\ETPS} exercises, like other exercises,
is to be done independently unless collaboration is specifically
authorized. You are obligated to work on these exercises
without help from other people, documents which contain
solutions, or programs which could be used to generate
solutions to the exercises. You need to think about these exercises
for yourself in order to learn from them.

\subsection{Documentation}

To use {\ETPS} efficiently, you should be familiar with the
{\ETPS} manual. The postscript and pdf files for this manual 
can be found at:\newline{}
http://gtps.math.cmu.edu/tps-mans.html

{\ETPS} online documentation is at:\newline{}
http://gtps.math.cmu.edu/htmldoc-etps

\subsection{Starting {\ETPS} or XETPS}

You should run this program from a private directory, where the files
you will be creating cannot be read by others. Start the program as
discussed above. When you start up XETPS, you can click with the mouse
to place the left corner of the window.

\subsection{Using XETPS}

\subsubsection{Style of Output for Wffs}

Once the X-window has appeared and {\ETPS} has started, you need to tell
{\ETPS} that you want to use the special fonts for output.  At the {\ETPS}
prompt, issue the command \newline{}
{\tt {\tt\char`\<}0> setflag style} \newline{}
then, at the subsequent prompt \newline{}
{\tt xterm}\newline{}
If you have followed all the steps
above, the special symbols should appear when any wffs are printed by
{\ETPS}.

\subsubsection{Proofwindows}

Once you have started XETPS, you will probably wish to use the
BEGIN-PRFW command to start up windows containing the current subproof
and the complete proof.  You will need to iconify the lower window or
move it up on your screen by the usual methods for manipulating
X-windows so that you will have room to issue commands in the main
{\ETPS} window.

\subsubsection{Adjusting RIGHTMARGIN}

If you resize your {\ETPS} X-window and make it wider or narrower, you will
want to change the setting of the flag {\tt rightmargin}.  This is normally
set at 79 for 80 column output.  For example, if you change the window to be
only 50 columns wide, issue the command \newline{}
{\tt {\tt\char`\<}1> setflag rightmargin}\newline{}
then at the prompt enter\newline{}
 {\tt 49}

\subsection{Aborting a Command}

Respond to a prompt for an argument with {\tt ABORT} to abort the current
command and return to the top level.  You can also use the control character
{\tt CTRL-C} or {\tt CTRL-G} (depending on what machine and version of LISP is
being used) to interrupt {\ETPS} and return to the top level.

\subsection{Stopping {\ETPS}}

To temporarily suspend {\ETPS}, use the control character {\tt CTRL-Z}.  Then
you can restart {\ETPS} with the  Unix command {\tt fg}.  Of course, if you
are running {\ETPS} in an X-window, you should not need to suspend it. You
can just make it into an icon.

To kill {\ETPS}, use the {\tt EXIT} command from within {\ETPS}, or use
the Unix {\tt kill} command from outside {\ETPS}.
Warning:  it is possible to start
multiple versions of {\ETPS} and leave them hanging around.
This is a bad idea, since those that
you are not using will be a drag on the system, slowing things down
considerably.



\subsection{Filenames}

In the Unix system, filenames are case sensitive.  When
you enter a filename in {\ETPS}, it will be translated to lower case
{\it unless you enclose the filename in double quotes}.  For example, if
you answer {\tt X2106.WoRK} when prompted for a filename, {\ETPS} will interpret
this as {\tt x2106.work}, but the entry {\tt ``X2106.WoRK''} will be interpreted
as {\tt X2106.WoRK}, just as you entered it.

% \comment{
% In the Microvax version of {\ETPS}, you cannot use the `$\sim$user'
% convention of referring to a user's home directory.
% }

%\pagebreak

\subsection{Tex, Scribe and Printing}

The command {\tt TEXPROOF} works just as described in the {\ETPS} manual.
Once you have created your `.tex' file on Andrew, however, you should just
start TeX by, for example,
\begin{verbatim}
% tex x2106
This is TeX, Version 3.1415 (C version 6.1)
(x2106.tex (/afs/andrew.cmu.edu/mcs/math/etps/sun4_55/doc/lib/tps.tex))
[1]
Output written on x2106.dvi (1 page, 1212 bytes).
Transcript written on x2106.log.
\end{verbatim}
This will produce the file x2106.dvi, and you can use the system command
dvips to create x2106.ps from this:
\begin{verbatim}
dvips x2106
\end{verbatim}
% \comment{dvips x2106 -o x2106.ps}
To print such a file, do the following, replacing {\it printer} with
the name of one of the Postscript printers such as birch, pine, etc.\newline
\verb=% lpr -P={\it printer} \verb=x2106.ps=

You can look at the file after running tex but before printing it with the command
\begin{verbatim}
xdvi x2106
\end{verbatim}
Exit xdvi with the command `q'.

To create output suitable for LaTeX
rather than TeX, set the flag {\tt LATEX-EMULATION} to T while you are running {\ETPS}.

Alternatively, to print a proof
you can use the {\tt SCRIBEPROOF} command in {\ETPS} to produce a file
such as x2106.mss, run it through Scribe with the command `scribe x2106' to produce
the Postscript file x2106.ps. Scribe is not available on some machine types.

\subsection{Possible Problems}

Various {\ETPS} commands involve writing files.  These commands include
DONE, SAVEPROOF, and TEXPROOF.  Unfortunately, it quite often
happens that the fileservers which handle the directory in which you
wish to write are down.  This is especially common at night, when
the backing-up of AFS volumes is done.  When this happens, you will get
an error message saying something like `Clisp error trapped'.  If this occurs,
try the following:

\begin{enumerate}
\item Try storing the file on your local workstation disk instead of AFS.
As an example, instead of \newline
\verb=texproof ``x2106.tex''= \newline
try \newline
\verb=texproof ``/usr/tmp/x2106.tex''=.

\item Later, after the
fileservers are back up, you can copy the file to your home directory on AFS.

\item If the DONE command has failed, use SAVEPROOF to save your proof.
Later you can restore the proof and try the DONE command again.
\end{enumerate}


\end{document}
