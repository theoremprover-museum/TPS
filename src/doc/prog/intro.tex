\chapter{Introduction}

\TPS has been developed over several decades by a number of people,
some of whom never actually met each other. Attempts have been made
to maintain documentation for the program, but research progress was 
generally a higher priority, and obsolete documentation was not always
corrected or deleted. Therefore, this manual should be used with discretion.
The guidance it provides may be very helpful at times, but there is no
claim that it is generally adequate, and some of it may
be misleading or incorrect.

\begin{quotation}
"The {\it Guide}... is an indispensable
companion to all those who are keen to make sense of life in
an infinitely complex and confusing Universe, for though it
cannot hope to be useful or informative in all matters, it
does at least make the reassuring claim, that where it is
inaccurate it is at least {\it definitively} inaccurate. In cases
of major discrepancy it's always reality that's got it wrong."

Douglas Adams, {\it The Restaurant at the End of the Universe}
\end{quotation}

\section{Guidelines}

In addition to the information in this guide, fragmentary documentation
of the \TPS code can be found in the {\it \indexfile{tpsjobs-done}} file.
This is included in the \TPS distribution.

This guide assumes that the reader is familiar with Common Lisp, and
does not attempt to explain or summarize information that is available
elsewhere about the workings of that language, in particular, in
Steele's {\it Common Lisp the Language, (2nd ed.)}.

There are three major rules which should be followed whether maintaining
\TPS code, or just fooling around with it:

\begin{enumerate}
\item Always keep a backup copy of the files you are changing, so that when
you realize how badly you goofed, you can put things back the way they were.

\item Don't get too tricky.  Clever hacks may be amusing, and may indeed
give some increase (usually modest) in efficiency, but within weeks
you will have no idea how they work, and others will be even more
mystified.  Those who follow you in your task will curse and despise
you; consequently, your cute programs will probably be completely rewritten
anyway. 

\item Don't panic.
\end{enumerate}

See section ~\ref{writing} for more minor guidelines.

\section{\TPS Conventions}

\subsection{Filenames}

The extension of a filename should indicate what it contains:
{\it .lisp} for Lisp source code; {\it .exp} for export statements
({\it vide infra}); {\it .rules} for deduction rule definitions
({\tt defirule} statements); {\it .mss} for Scribe formatted
documentation; {\it .tex} for \TeX formatted documentation;
{\it .vpw} for vpwindow output; {\it .work} for work files; {\it .prf} for proofs.

Filenames should be descriptive of their contents, without being too long.  For example,
{\it functions.lisp} would be a stupid name, because from the name no
one would know what its purpose was.  If you have several related
files, it is a good idea to give them a common prefix, so that it is
clear just from their names that they are related. 

\subsection{Lisp packages and export files}

\TPS creates and uses several different Lisp packages.  (If you don't
know what I mean by Lisp package, read the chapter on packages in {\it 
Common Lisp the Language, (2nd ed)}.)  These packages are
created when \TPS is compiled or built, by {\tt make-package} forms in
the files {\tt tps-build.lisp} and {\tt tps-compile.lisp}.  The package
structure is set up so that common functions are placed in the package
{\tt CORE}, which is used by each of the other packages.  
The package {\tt MAINT} contains functions useful to the maintainer. 
The package {\tt AUTO} contains the automatic portions of \TPS, and the
package {\tt ML} contains the inference rules used in the mathematical
logic system.  The {\tt TEACHER} package contains files relevant to GRADER.

Within each Lisp package are several \TPS modules, 
groups of related source files which are clumped together.
These are defined in the file {\it \indexfile{defpck.lisp}}.

The {\tt CORE}
package contains functions such as those dealing with wff parsing and
printing, proof manipulation, and operating system interfaces (such as
basic file operations). It also contains functions for dealing with 
Scribe, vertical paths, editing, windows, review, etc... 
The other packages include: {\tt TEACHER}, for functions relating to the Grader
subsystem; {\tt AUTO} for functions relating to automatic proof
procedures such as mating search; {\tt ML} for things specific to the
Math Logic courses, such as exercises and proof rules.

\ETPS contains part of the {\tt CORE} package, part of the {\tt OUTLINE}
package and part of the {\tt RULES} package.

The idea is that only those symbols that are needed by other packages are
exported from their home package. In order to specify which symbols
should be exported, the files {\it core.exp}, {\it auto.exp}, etc.  These
files, one for each Lisp package, are loaded at the beginning of the
compilation process, before any code is loaded.  This way, any package
conflicts are detected immediately.

There is a special export file, called {\it special.exp}.  This file
contains export statements for symbols which may already exist in
certain Lisp implementations.  For example, some implementations already
contain a symbol {\tt EXIT}, while others do not.  Why is this a
problem?  Because if the {\tt CORE} packages uses a package from an
implementation (e.g., Allegro's {\tt EXCL} package), and that package
already exports the symbol {\tt EXIT} (so that {\tt EXIT} is imported by
{\tt CORE}),  then an error will result if we
try to export {\tt EXIT} from the {\tt CORE} package, i.e., you can't
export a symbol from a package other than its {\it home} package.
{\it special.exp} uses the
standard {\tt \#+} and {\tt \#-} macros to specify in which implementations
such symbols should be exported from the {\tt CORE} package.  Generally,
these nuisance symbols are found by trial and error when first porting
\TPS to a Lisp implementation, and some symbols may have to be moved
from {\it core.exp} to {\it special.exp}. (Another symbol in \$t(CORE) that can
cause problems with some Lisps is {\tt date}.)

Note that when \TPS starts up, the {\tt USER} (soon to be
{\tt COMMON-LISP-USER}, as the changes in Common Lisp suggested by the
X3J13 committee are implemented) is the value of the variable
{\tt *package*}.  What this means is that any symbols typed in by the
\TPS user will be interned in the {\tt USER} package.  Thus, any symbols
that could be inputted by the user as, say, a flag value, should be
exported from the package in which they were defined, otherwise \TPS
will  not realize they are supposed to be the same.  As an example,
the flag {\tt RULEP-MAINFN} can be given the value {\tt RULEP-SIMPLE}.
Since {\tt RULEP-SIMPLE} is defined in the {\tt CORE} package, it must be
exported in {\it core.exp}, so that when it is inputted, the symbol
{\tt CORE::RULEP-SIMPLE} is interned, not {\tt USER::RULEP-SIMPLE}.  Of
course, this presumes that the {\tt USER} package uses the {\tt CORE}
package (which it always does in \TPS).

\subsection{Implementation-specific differences}\label{sect:impl-spec-diffs}
Not all Lisp implementations are alike.  This is
particularly true in the areas of Common Lisp which are intentionally
unspecified, including things like how the Lisp top level works, how
file pathnames are represented, how the user exits the Lisp or saves
a core image. 

For this reason, certain \TPS source files contain 
{\tt \#+} and {\tt \#-} directives.  We try to keep the number of these
files to a minimum, so that when porting to new implementations, work
is minimized.  When using {\tt \#+} and {\tt \#-}, you should try to use as
specific a feature of the implementation as possible (but avoid using
the machine type unless that is the reason you have to make a change).
For example, the feature {\tt :allegro-v3.1} is probably better than
{\tt :allegro}, as I have found out to my dismay when Allegro 4.0 came
out.  Look at the lisp variable \indexother{*features*} to find what features that
version of lisp recognizes.  A few examples of features are listed below:
\begin{itemize}
  \item {\tt :allegro}  (Allegro Common Lisp)
  \item {\tt :clisp}  (Gnu Common Lisp)
  \item {\tt :cmu}  (CMU Common Lisp)
  \item {\tt :mswindows} (Microsoft Windows)
\end{itemize}

There is one feature ({\tt :andrew}) that is added when we compile \TPS/\ETPS for use
on the Andrew workstations (machines in the domain andrew.cmu.edu).
This is because on those machines, lisp implementations have problems
interfacing with the operating system and getting the proper home
directory of a user.  Thus special measures are taken in this case.
This feature is added in the {\tt .sys} files for the Andrew editions.

There are another two features which are added in the relevant {\tt .sys}
files for \ETPS and \TPS; these are \indexother{:TPS} and \indexother{:ETPS}.
This allows programmers to specify slightly different behaviour for the 
two systems (for example, when using the editor, you may have a window that
shows the vpform in \tps, but not in \ETPS).

The files which use {\tt \#+} and {\tt \#-} are principally {\it special.exp},
{\it boot0.lisp}, {\it boot1.lisp}, {\it tops20.lisp}, {\it tps3-save.lisp},
and {\it tps3-error.lisp}.

\subsection{\TPS modules}

\TPS source files are organized into \TPS modules. 
Basically, a \TPS module is just a list of source files, in the
sequence in which they are to be compiled/loaded.  All
\TPS modules are defined in the file {\it \indexfile{defpck.lisp}}.  Each source
file should be in some \TPS module, and that module should be indicated
in the file.  
(Conceivably, one might define two
different \TPS modules which had files in common, but we have never
done that.)  

Some files are designated as macro-files in the definition of the module
in {\it \indexfile{defpck.lisp}}.  
When a module is
compiled [loaded], the macro-files are compiled [loaded] first.
Also, there is code for loading the macro-files for a module without loading the
other files.  
When adding a file {\it foo.lisp} to a module,
designate it as a macro-file
if many of the other
files in that module use structures, macros, or variables defined in {\it foo.lisp}.

In addition to the files it contains, the definition of a \TPS module
also specifies the other modules which must also be present when it
is used.  

The \TPS module structure breaks up the source files into chunks,
each which has some particular purpose or purposes.  Then to build a
version of \TPS which has certain capabilities, one need
only load the modules required.  This is how the files
{\it tps-build.lisp} and {\it tps-compile.lisp} specify \TPS is to be
built. Note that {\it etps-build.lisp} and {\it etps-compile.lisp} just
load fewer modules than the build/compile files for \tps.  Likewise,
there are {\it grader-compile.lisp} and {\it grader-build.lisp} files for
building a Grader core image.

By using the module mechanism, a module may be modified by
adding, deleting, or modifying its constituent files, and other users don't
have to know; all they need to know is what the module provides.

Functions such as {\tt LOAD-MODULE}
are provided to load modules, making sure that any modules they
require are also loaded.

\subsection{File format}

In general, programmers should use only lower-case.  Why? For two
reasons.  It is easier to read, and in case-sensitive operating
systems like Unix, it is easier to use utilities such as fgrep and
gnu-emacs tags ({\it vide infra}) to search for occurrences of symbols.

Each \TPS source file should contain certain common elements. First is
a copyright notice, whose purpose is self-explanatory (just copy it
from another source file).  Make the copyright date current for any
new code.

The first line of the file, however, should be something like:
\begin{verbatim}
;;; -*- Mode:LISP; Package:CORE -*-
\end{verbatim}
The gnu-emacs editor will use this line to put the buffer in Lisp
mode, and if you are using one of the gnu-emacs interfaces to the
lisp, it will use the package information appropriately.  See the
documentation for such gnu-emacs/lisp interfaces.

The first non-comment form of each source file should be an
{\tt in-package} statement, to tell the compiler what package the file
should be loaded in. Recent implementations of lisp will object if
there is an {\tt \indexother{in-package}} command anywhere else in the file.

Next, the \TPS module of the file should be indicated, by a
{\tt part-of} statement, like {\tt (part-of {\it module-name})}.
This should match the entry given in {\it defpck.lisp}.

Don't forget {\tt context} statements.  Basically they just
reset the variable {\tt current-context}, which is used by other
functions to organize the documentation and help messages.

\section{Maintenance}

\subsection{Porting \TPS to a new Lisp}
As discussed above, the lisp-implementation-dependent parts of \TPS are confined to
a few files.  See the discussion above (Section~\ref{sect:impl-spec-diffs})
and the portion of the user's manual on setting things up for more details.
The following is a list of steps for compiling \TPS under a new lisp.

\begin{enumerate}
\item Modify the line in the \indexfile{Makefile} of the form
\begin{verbatim}
lisp = <lisp-executable>
\end{verbatim}
so that {\tt <lisp-executable>} is the executable for the new lisp.
\item
  Try to perform {\tt make tps}.  If you are very lucky, this will work.
  However, probably the first problem you will encounter involves conflicts
  with respect to exporting symbols.  
  We explain how to resolve these conflicts in step (\ref{newlisp:export}).
  Once these conflicts are resolved, go on to step (\ref{newlisp:funs})
\item\label{newlisp:export}
  Suppose the implementation of lisp complains about a conflict with an exported symbol {\tt tps-symbol}.
  We try to keep such conflicts localized to the file
  {\indexfile{special.exp}}.
  If {\tt tps-symbol} is exported in the file {\indexfile{special.exp}},
  then add a compiler directive {\tt \#+} (or {\tt \#-}) 
  to the export for {\tt tps-symbol}.
  If {\tt tps-symbol} is exported in some other export file,
  then move this export to {\indexfile{special.exp}}
  and add an appropriate compiler directive.
  (See Section~\ref{sect:impl-spec-diffs} for more information about compiler directives
  and the \indexother{*features*} lisp variable.)
\item\label{newlisp:funs}
  The next thing you will likely need to do is add definitions for certain functions and macros.
  For example, the functions \indexfunction{tps3-save}, 
  \indexfunction{linelength}, 
  \indexfunction{exit-from-lisp}, \indexfunction{status-userid},
  \indexfunction{call-system}, \indexfunction{setup-xterm-window}, \indexfunction{setup-big-xterm-window},
  \indexfunction{make-passive-socket}, \indexfunction{make-passive-socket-port},
  \indexfunction{connect-socket},
  \indexfunction{accept-socket-conn} and
  \indexfunction{pass-socket-local-pass}
  are defined for each implementation in the file \indexfile{tops20.lisp}. 
  The definitions of these functions for other versions of lisp should
  indicate what the definition should be for the new implementation of lisp.
  If you really do not know how to define the function,
  you can always define the function as the following example indicates:
\begin{verbatim}
#+:newlisp
(defun call-system ()
  (throwfail ``call-system undefined in lisp <newlisp>''))
\end{verbatim}
  Such definitions will limit some of the capabilities of TPS.

  Among these functions, \indexfunction{tps3-save} is the most vital for getting
  started.  The function \indexfunction{tps3-save} should create a core image file
  which will be used when starting \TPS. 
\item
  Call {\tt make tps}.  If all goes well, a core image file will be created
  and you are ready to run \TPS.
  Check the documentation for the implementation for lisp to find out
  how to start lisp with a given image file.

  If you are using Allegro Common Lisp version 4.1 or later, {\verb+--+} is
  used to separate user options from lisp options, and hence the standard 
  way of starting up the Grader program in X-windows becomes:

{\tt xterm -geometry 80x54--14-2 '\#723+0'  -fn vtsingle -fb vtsymbold-sb -n 
CTPS-Grader -T CTPS-Grader -e /usr/theorem/bin/run-tps \verb+--+ -grader \&+}
\end{enumerate}

{\bf Use of \indexother{defconstant}:}
Since files may be loaded more than once  during compilation and building of \TPS,
two identical uses of {\verb defconstant }  may occur.
As the behaviour of the second occurrence of  {\verb defconstant } is not specified
in the ANSI standard, and as some Lisps, such as Steel Bank Common Lisp, 
do not consider this use as correct, it is better to use the 
macro {\verb defconstnt }, which has been implemented to fit {\verb defconstant } and still be used with SBCL.
(Note that '\indexother{defconstnt}' has no 'a' at the end.) 

%  % The following discussion about int-char and string-char is obsolete - TPS does not use these. - ceb 6/8/2004
% There are (at least) two places in which the code is currently
% not ANSI standard, in that the function {\tt int-char} and the type {\tt string-char}
% (which do not exist in ANSI standard lisp) are used. When porting to a new lisp,
% one should be aware that it may be necessary to provide substitute definitions 
% for these. See {\tt defun int-char} and {\tt deftype tps-string-char} in {\it boot0.lisp}
% for an example of how this might be done.

{\bf Obsolete In 2004:}
Also, if you are using Kyoto Common Lisp, you will find that the way it
represents directories is a little unusual: all paths are relative unless
specified not to be. So, for example, {\it tps3.sys} should be changed to read:

\begin{verbatim}
(setq news-dir '(:root "usr" "tps"))
(setq source-path '((:root "usr" "tps" "bin") 
                    (:root "usr" "tps" "lisp")))

(setq compiled-dir '(:root "usr" "tps" "bin"))
(setq patch-file-dir '(:root "usr" "tps"))
\end{verbatim}

(assuming the main \TPS directory is {\it /usr/tps/}).

\subsection{Building \TPS}
See the user's manual for a description of how to set up and build a
new version of \tps/\ETPS.

The global variable \indexother{core-name} currently contains "TPS3";
it is defined in {\it \indexfile{tps3.sys}}, which is generated by the Makefile.
All files (news, note, ini, sys, patch, exe) use core-name as their
`name'.

File names and extensions should be strings rather than
quoted symbols, to avoid any ambiguity with the package qualifiers.

Changes to the code are put in the patch file {\it \indexfile{tps3.patch}} until 
\TPS is rebuilt. \ETPS and Grader have separate patch files.
When you change the file {\it nat-etr.lisp} (for
example), put the line {\tt (qload "nat-etr")} into {\it tps3.patch}.
In general, don't put {\tt (qload "nat-etr.lisp")} into the patch file, or the
uncompiled version of the file will be loaded. However, the export files
{\it *.exp} do need their extension.

Entries such as {\tt (qload "auto.exp")} which load exp files should
come before those loading lisp files. 
{\tt (qload "core.exp")} should come before loading other export files.
Macro files should come before other files in the same module.

Putting the line
{\tt (setq core::*always-compile-source-if-newer* T)}
near the beginning of the {\it tps3.patch} file, and \\
{\tt (setq core::*always-compile-source-if-newer* NIL)}
at the end of the same file
will cause files to be compiled automatically
whenever appropriate as one is starting up \tps, but then
restores the default value of {\tt *always-compile-source-if-newer*}
so that you will be able to decide whether or not to compile other
files as you load them.

Example: when ms91-6 and ms91-7 were introduced, {\it tps3.patch}
contained:
\begin{verbatim}
(qload "core.exp")
(qload "auto.exp")
(qload "defpck")
(qload "contexts-auto")
(load-module 'ms91)
(qload "diy")
\end{verbatim}

\subsection{Memory Management}

\TPS uses a huge amount of \indexother{memory} in the course of a long search, and it may be necessary 
to rearrange either the internal memory available in your computer or the maximum space
occupied by your version of Lisp. Both of these things vary; the former by system (type {\tt sys} to
find out what system you are using) the latter by the variety of Lisp. You can tell roughly how much memory
is being used in most versions of Lisp by turning on garbage collection messages and watching the numbers
they report.

After a long search, \TPS may fail with an error message that mentions not having enough 
\indexother{heap space}, or \indexother{stack space}, or \indexother{swap space}. Allegro 
Lisp is very good about indicating the real 
cause of the problem. CMU lisp turns off errors while it garbage collects, and unfortunately that's
when most of these errors occur, so if your CMU-based \TPS seizes up in mid-garbage collect and refuses to
stop even for $\hat{}$C, then you've probably run out of memory somewhere. Lucid Lisp turns off garbage 
collection when it approaches the internal memory limits (there is a good reason for this; see 
the Lucid manual), so if you get a message about garbage collection being off then the real
problem is probably a lack of memory. 
(\TPS never switches \indexother{garbage collection} off itself.)

\subsubsection{Heap Size and Stack Size}
On a Unix system, type {\tt limit} into a C-shell (or whatever shell you're using) to see a list 
of the upper limits on various things stored in memory. The ones you're most interested in will
be \indexother{datasize} and \indexother{stacksize}. If you are superuser, you can remove these restrictions
temporarily by typing {\tt unlimit datasize stacksize}, or possibly {\tt unlimit -h datasize stacksize}. 

To increase these limits permanently, you need superuser privileges. You will need to reconfigure the
kernel and reboot your system. On anything except an HP, write to {\tt gripe@cs} and ask them to do it,
unless you're confident about being able to do such things. On an HP, you can use their \indexother{SAM}
program (when nobody else is logged in, since you're going to do a reboot), as follows:
\begin{enumerate}
\item Log in as superuser, and type {\tt sam}.

\item Double-click on "Kernel Configuration"

\item  Double click on "Configurable Parameters"

\item Highlight the parameter "maxdsiz" and select "Modify" from the "Actions" menu.
Increase the value as high as you want. On our machines, it was initially 
0x04000000 and we increased it to 0x7B000000. If you choose too high a number,
it will be rejected and you can try again.

\item Check that the "value pending" column shows your new value for maxdsiz. If not,
pick "Refresh Screen" from the "Options" menu and do the last step again.

\item Now do the same for "maxssiz"; we increased it from 0x00800000 to
0x04FB0000.

\item Choose "Exit" from the "File" menu. You will get a barrage of questions, say
yes to all of them. (They will be something like: create the kernel now? 
replace the old kernel? reboot the system?)
\end{enumerate}

When the reboot is done, type {\tt limit} to check that the values have increased.

\subsubsection{Swap Space}
Swap space is that part of the memory (usually on disk) where the operating system
stores parts of the programs that are supposed to be in memory. This is how you 
can get away with running more programs than your RAM has space for.
Clearly, the amount of \indexother{swap space} you need will depend not only on how big
your \TPS grows, but also on what else is running at the same time.

Again, on anything but an HP it's time to go whining to {\tt gripe@cs} and get them to do it.
On an HP, start SAM as in the last section, and double-click on "Disks and File Systems".
Now double-click on "Swap". There are two sorts of swap space, device ({\tt dev}) and file system
({\tt fs}). The former is faster and should be given priority over the latter.

Here is where I don't quite understand what's going on, so if this information ever becomes
crucial it would be a good idea to check it. I believe that device swap space is simply a 
partition of the internal disk drive, and that it might be possible to create more space simply by
rearranging the partition. I have no idea how to do this.

For the time being, then, we'll restrict ourselves to filesystem swap space. You can mount 
one filesystem swap space on each disk you've got, so take a look at the list that SAM has given you.
If there are no {\tt fs} swap space listed, or there is a disk that doesn't have one, then you can 
create one by selecting "Add Filesystem Swap" from the "Actions" menu. Give it a reasonable 
number (you can use {\tt du} and {\tt df} to find out how much space there is on the disk at the moment,
and then choose some large fraction of that), and allocate a priority that is lower (which is to say,
a larger number; 0 is highest-priority) than the priorities of the {\tt dev} swapspace (so that you 
will use the fast swap space before the slow one). New swap space takes effect right away.

If you already have {\tt fs} swapspace on all disks, you can highlight the one you want to change 
and then choose "Modify Swap Space" from the "Actions" menu. Increase the size as you want.
Modifications only take place after the next reboot, but it is not necessary to reboot right away
as it is for the heap and stack space.

\subsubsection{Internal Limits in Lisp}
As if all that wasn't enough, your version of Lisp may also have some constraints on how large it can grow.
\begin{enumerate}
\item CMU Lisp has no such limits, as far as I know.

\item Lucid Lisp has them, and they are user-modifiable; type {\tt (room t)} into a Lucid \TPS to see 
what the current settings are. Look for "Memory Growth Limit"; if it seems too small, type
(for example)
{\tt (change-memory-management :growth-limit 2048)} into the \TPS to allocate 128Mb (2048 64kb segments).
You can also make this permanent by adding {\tt \#lucid(change-memory-management :growth-limit 2048+)}
to your {\it tps3.ini} file. Other parameters besides the overall size limit can also be changed; see the
Lucid manual for details.

\item Allegro Lisp also has a limit, but in this case it is set at the initial building of Lisp. Here you'll
have to retrieve the build directory for allegro (which is {\it /afs/cs/misc/allegro/build/} followed by 
the name of your system). We have a copy of this on {\it tps-1}, called {\it allegro4.2hp\_huge}, but it
requires some hacking to make it build properly. Follow the instructions in the README to build yourself
a new Lisp core image with more than the standard 60Mb data limit. If you aren't up to the 
hacking, once again the solution is to whine at {\tt gripe@cs}, who will forward your mail to the 
Allegro maintainer.
\end{enumerate}

\subsection{\TPS distribution}
\subsubsection{Making a tar file for Distribution} 

You can execute
{\tt /afs/andrew/mcs/math/TPS/admin/tps-dist/make-tar.exe}
from {\tt /home/theorem/project/dist}
(The date is computed automatically.)
The tar file is placed in /home/ftp/pub.

\begin{verbatim}
			KEY STEPS:
     Login to gtps as root (so that you can write a file in /home/ftp/pub).
su
     (You may need to do a klog, e.g., 
klog pa01 -c andrew.cmu.edu
      or 
klog cebrown -c andrew.cmu.edu
         so you appropriate permissions.)
cd /home/theorem/project/dist
/afs/andrew/mcs/math/TPS/admin/tps-dist/make-tar.exe

    [This does
     tar cvhf /home/ftp/pub/tps3-date.tar .
     gzip /home/ftp/pub/tps3-date.tar
     ln -sf /home/ftp/pub/tps3-date.tar.gz /home/httpd/html/tps3.tar.gz
      ]
    (It's important that the tar file is not put into the same directory
      as is being tarred, or it will try to work on itself.)

cd /home/ftp/pub
move old tar file to the subdirectory old. Delete the older one.

\end{verbatim}



%% description of what make-tps.exe does.
% \subsubsection{Distribution of \TPS via ftp} \\
%
%To make a tar file for distribution via ftp:
%\begin{enumerate}
%
%\item Login to gtps.math.cmu.edu.
%Delete any old tar files in /home/ftp/pub, or move them to
%/home/ftp/pub/old.
%
%\item {\tt cd /home/theorem/project/dist}.
%
%\item Clean up /home/theorem/project/dist and its subdirectories by deleting
%each backup, postscript, or dvi file.
%% @comment[See /afs/cs/usr/andrews/clean.hlp if you want to automate this.]
%
%\item While in the directory /home/theorem/project/dist,
%execute the following command at the Unix prompt:
%{\tt tar cvhf /home/ftp/pub/tps3-date.tar .}
%(where {\tt date} is the current date).
%A list of file names should pass by on the screen.  Just watch until
%you get a new Unix prompt.
%
%\item {\tt cd /home/ftp/pub}.
%
%\item If you wish to check the tar file, enter
%{\tt tar tvf tps3-date.tar}.
%The same list of file names should pass by.  These are the names of
%the files which are now in the tar file.
%
%\item {\tt compress tps3-date.tar}. This will create the file
%tps3-date.tar.Z, which is ready for people to transfer using ftp.
%
%\end{enumerate}
%
%     To see what files are in the tar file do:
%      gunzip -c tps3-date.tar.gz | tar tvf -

\subsubsection{Distribution of \TPS via http}

There is a perl script {\tt /home/httpd/cgi-bin/tpsdist.pl}
which is used to distribute \TPS via the gtps web site.
This perl script displays the distribution agreement
and asks for information from the remote user (name, email, etc.).
Once this information is given, the perl script updates
the log file {\verb+/home/theorem/tps-dist-logs/tpsdist_log+}.
(For this to work, {\tt apache} should be the owner of this log file.)
Then, the perl script outputs instructions and a link to the tar file.


\subsubsection{Obsolete Information about Making tar tapes of \TPS}

To make a tar archive onto a big round mag tape (the kind that is seen
in science fiction movies of the sixties and seventies, always
spinning aimlessly, supposedly to suggest the immense computing power
of some behemoth machine):
\begin{enumerate}
\item Go to the CS operator's room on the third floor, at the end of the
Wean Hall 3600 hallway.  

\item Tell the operator that you wish to write a tar tape from the machine
K.GP.  Give her the tape and go back to the terminal room. 

\item Log in on the K.

\item At the Unix prompt, enter {\tt assign mt}.  This gives you control of
the mag tape units.  

\item Enter {\tt cd /home/theorem/project/dist}.  This puts you in the proper
directory.

\item Clean up /home/theorem/project/dist and its subdirectories by deleting
each backup, postscript, or dvi file.

\item Determine which device {\it devname} you wish to use.  This depends on the 
density which you wish to write.  For 6250 bpi, let {\it devname} be
{\tt /dev/rmt16}, for 1600 bpi, let {\it devname} be {\tt /dev/rmt8}, and
for 800 bpi, let {\it devname} be {\tt /dev/rmt0}.  Generally, you can go
for 6250 bpi unless the intended recipient has indicated otherwise.

\item Execute the following command at the Unix prompt:
{\tt tar rvhf {\it devname} .}

\item A list of file names should pass by on the screen.  Just watch until
you get a new Unix prompt.

\item To check the tape, enter
{\tt tar tvf {\it devname}}.
The same list of file names should pass by.  These are the names of
the files which are now on the tape.  If there were already files on
the tape, you will see all of them listed as well.

\item If all is well, call the operator and tell them that you are done with
the tape and that they can dismount it.  Then execute {\tt exit} at the
Unix prompt, to give up control of the tape drive, and log out as usual.

\item Go back to the operator's room and pick up the tape.

\end{enumerate}

To make a tar archive onto a Sun cartridge tape:
\begin{enumerate}
\item Take the tape to the CS operator, and ask her to put it on the machine
O.GP.  That is a Sun3 running Mach.  Go back to a terminal and log in on
any machine.  Again, you want to be in the directory /home/theorem/project/dist.

\item Now, you need to make sure that you can write to the O's tape drive.
You want to check the owner of the file /../o/dev/rst8, and make sure it's
{\tt rfsd}.  If not, call the CS operator and ask them to assign it to rfsd so
that you can make the tar tape.  

\item Now, execute the following command at the Unix prompt:
{\tt tar cvhf /../o/dev/rst8 .}

\item A list of file names should pass by on the screen.  It will be very slow.

\item After you get a new Unix prompt, wait a few seconds for the tape to rewind, 
then check it by entering
{\tt tar tvf /../o/dev/rst8}.
The same list of file names should pass by.  These are the names of
the files which are now on the tape.  

\item Go back to the operator and ask for the tape.

\end{enumerate}



\section{How to \indexother{locate} something?}\label{finding}
Sometimes you will be looking at code, and will come across a function
or variable whose purpose is not familiar to you.  If it is not a standard Common
Lisp function, for which the Lisp functions {\tt documentation} and
{\tt apropos} may be useful, as well as reference books and user manuals,
there are three ways to \indexother{find} where it is defined.

The first method uses the gnu-emacs tags mechanism. Periodically, we run the 
{\tt etags} program on the
{\it .lisp} files in the source directory. One does this by entering the \TPS
lisp directory and then running the \indexother{etags} program; usually, 
this is done by typing {\tt M-x shell-command etags *.lisp}.  
This generates a file called
{\it \indexfile{TAGS}}, with entries for each line of code which begins with
{\tt (def...}. Then you can use the gnu-emacs {\tt find-tag} function (ESC-. , 
unless you've rearranged the emacs keys) to look
for the first occurrence of the symbol, and the {\tt tags-loop-continue} function
(ESC-,) to find the each subsequent occurrence.  
This can be slow if there are many symbols which begin with the
prefix for which you are searching, or if the symbol is overloaded by 
defining it for different purposes (e.g., {\tt LEAVE} is a
matingsearch command, a review command, a unification toplevel command).  
See the gnu-emacs documentation.

Certain functions, such as eproof-statuses, are defined implicitly,
and you won't find their definitions using the tags mechanism. If you
look at the definition of the structure \indexother{eproof} in the file
etrees-flags.lisp, however you will find:
\begin{verbatim}
(defstruct (eproof (:print-function print-eproof))
...
  (statuses (make-hash-table :test #'eq))
\end{verbatim}
This defines the function eproof-statuses.

The second method is to use the \TPS export files. Try
examining the files with a {\it .exp} extension.  Generally, comments 
tell which file each symbol comes from.  This method will fail,
however, if the symbol is not exported, or if the symbol has been
moved from the file in which it was originally defined without the
{\it .exp} having been updated.

\TPS has many global lists; the master list is called 
\indexother{global-definelist}, and in general each sort of \TPS object 
will have an associated global list. 

The last method is to use operating system utilities like grep and
fgrep to find all occurrences of the symbol.


\section{Utilities}
Utilities are commonly-used Lisp functions/macros. The functions (or macros) themselves 
are defined in the normal way, and then a \indexcommand{defutil} command is added into 
the code beside the function definition. The point of adding the {\tt defutil} command is
that utilities have their own \TPS category, you can get online help on them, and their 
help messages are printed into the Facilities Guide; this will help other \TPS 
programmers to find them in the future.
 
Examples are such functions as \indexother{msg} and \indexother{prompt-read}; see the
facilities guide for a complete list.

There aren't really very many \indexother{utilities}
at the minute, although it would be useful if more were defined, since then we could
avoid duplicating code in different places. So, if you write a useful macro or 
function {\it foo}, or discover one already written, please add a utility 
definition next to it in the code. This should look like:

%\begin{tpsexample}
\begin{verbatim}
(defutil foo
   (Form-Type function)
   (Keywords jforms printing)
   (Mhelp "Some useful words of wisdom about the function foo."))
\end{verbatim}
%\end{tpsexample}

{\tt Form-Type} should be either {\tt function} or {\tt macro}. {\tt Keywords} can be anything
you want, since it is currently ignored by \tps. {\tt Mhelp} is, of course, a help message.
Note: if your useful function is actually an operation on wffs, it should be defined as
a wffop or wffrec (recursive wffop) rather than as a utility; utilities are really 
intended to be functions that are useful to \TPS programmers but which do not fall into 
any other \TPS category.

\section{Overloading Commands}

There are certain symbols in \TPS that been {\it overloaded}, that is
they have been defined to have more than one meaning: they may be
simultaneously a matingsearch command, review command, and unification
command. This is done so that same symbol can have similar effect in different
top-levels. For example, {\it LEAVE} should leave the current top-level, 
as opposed to having a different exiting command for each top-level,
which would make things more difficult for the user to remember.

This can cause problems in \TPS unless programmers are careful.  You
see, we currently use the symbol's property list extensively to store
things.  When a matingsearch command (such as {\tt LEAVE}) is defined,
the actions that are to be taken when the user inputs the command are
stored on {\tt LEAVE}'s property list.  It is important, therefore, that
each category use different property names, so that there is never a
clash.  For example, if we used the property {\tt ACTION} for both
review commands and matingsearch commands, then {\tt LEAVE}'s property
list could not hold both simultaneously, but merely one or the other.
Better property names would be {\tt REVIEW-ACTION} and {\tt MATE-ACTION}.

\section{Output}

Some general tips for keeping the output as neat as possible:

\begin{itemize}
\item Avoid using the lisp function \indexfunction{y-or-n-p}, and stick to the \TPS
function \indexfunction{prompt-read}, so that the responses will go into work files
correctly.

\item \indexfunction{msg} and \indexfunction{msgf} (which is like {\tt msg} but adds a linefeed if necessary)
are \TPS functions for producing output. These functions take a sequence
of arguments, and evaluate and print out each argument in an appropriate format;
an argument t means go to a new line. See {\tt defutil msg}.

\item {\tt (msg (gwff1 . gwff))} will print out the correct representation
of the gwff, whereas {\tt (princ gwff1)} will just print its internal
representation.

\item To insert a call to runcount in the code:
{\tt msgf (\indexfunction{runcount})}

\item \indexfunction{stringdt} gives the time and date. \indexfunction{stringdtl} also inserts linefeeds

\item {\tt princ} often puts messages into a buffer.
To get them to print out, add the command {\tt {finish-output}}.
You may also have to do this when you use other
output commands, including msg. 

\item Windows (proofwindows, edwindows, vpwindows) all work by issuing a Unix shell
command which runs an xterm which, in turn, runs the Unix "tail" command 
recursively on an output file that \TPS creates by temporarily redirecting 
{\tt *standard-output*}. (Compare such commands as \indexcommand{SCRIPT} and
\indexcommand{SAVE-WORK}, which {\it permanently} redirect {\tt \indexother{*standard-output*}}.)
See the files 
\indexfile{tops20.lisp}, \indexfile{prfw.lisp}, \indexfile{edtop.lisp} and
\indexfile{vpforms.lisp} for more information.
\end{itemize}

\section{Compiling as much as possible}
In defining new \TPS objects, we often define as a side-effect new functions.
For example, when defining a new argument type, we define a testfn and a
getfn for that type, based on the values for those properties that are
given in the {\tt deftype\%} form. 

Currently, all such functions are compiled, by cleverly defining the definition
macros so that {\tt defun} forms are created during the compilation of a file.
If you define new categories that will create such functions, you will want
to do something similar, so that you aren't always running interpreted code.
See the files {\it argtyp.lisp} and {\it flavoring.lisp} for examples of how
this can be done.

\section{Writing New Code Without Making A Nuisance of Yourself}
\label{writing}

\begin{itemize}
\item Programmers should avoid referencing internal symbols of different
LISP packages.  If you are doing this, think about why it is
necessary.  Perhaps it is better to export the symbols, or rethink the
package structure.

\item Symbols should be exported before files containing them are compiled.
Otherwise you stand the risk of having those symbol-occurrences
interned in the wrong package.

\item Lisp macros can be very useful, but it is easy to overuse them.  It
can be very difficult to debug code that uses many macros, and because
there is no guarantee that macros will not be expanded when code is
loaded (and they are always expanded when compiled), modifying a macro
means recompiling every file in which it appears, which is quite a nuisance.

\item There are a multitude of functions in \tps, so one must be careful not
to inadvertently redefine a function or macro.  With the Lisp function
{\tt APROPOS}, you can check to see
whether a function name is already being used.
Use the TAGS table. See section ~\ref{finding}, above.

\item Try not to re-invent the wheel; look in all the likely places to see if some
of the code you need has already been written. If your new construct is
similar to an existing one, use {\tt grep -i} in the directory 
{\it /afs/andrew/mcs/math/TPS/lisp/} to find and examine all uses of the existing 
construct.

\item Remember that rules of inference should be written as {\it .rules} files and compiled 
with the \indexcommand{ASSEMBLE-RULE} command; if you modify the {\it .lisp} files 
directly, you run the risk of having your modifications accidentally overwritten
by future users.

\item When modifying copies of existing files, prior to installing them, 
rename the file temporarily (for example, preface the filename with your 
initials) so that if you compile your new code it won't overwrite the 
existing compiled file.

\item Don't install code until you've tested it! After installation, keep backup 
copies of the old files
in the {\it /home/theorem/project/old-source/} directory on gtps, and change their
extensions from {\it .lisp} to {\it .lisp-to-1997-jan-3} (or whatever). Delete
all Emacs backup files from the main lisp directory. Compile new code using
the CMU Common Lisp version of \TPS since that compiler is fussier than most.

\item Try to make sure that online documentation is included in all user
functions, argument types, etc. that you define. Also, you should at the very least 
put comments in your code; better yet, write some documentation for the manuals.
Note that online help can be associated with any symbol using the
\indexother{definfo} command.

\item If a new subject has been created, and this subject contains
flags important for automatic search, the function \indexfunction{mode} 
\indexfile{flagging.lisp} should be updated.  The code starts as

\begin{verbatim}
(defun mode (mode)
  (let ((already-set nil))
    (unless (eq mode (gettype 'tps-mode 'maint::quiet))
      (dolist (subject '(IMPORTANT MATING-SEARCH MS88 . . .
			 UNIFICATION PRIMSUBS MTREE-TOP))
\end{verbatim}
The list in the dolist contains all the subjects important for automatic search.
The new subject should be added to this list.  The purpose of the list is
to make sure important flags have their default value if they are not explicitly set by the mode.

\item When a new part of \TPS is developed, an appropriate module should be defined in
{\it /afs/andrew/mcs/math/TPS/lisp/defpck.lisp}. If a new file is being added to 
an existing module, just add it to the list in defpck.lisp, make sure the correct
heading is on the file, and export the filename from 
{\it /afs/andrew/mcs/math/TPS/lisp/<package>.exp}. (Actually, the exporting should 
be done automatically by \tps, but it won't hurt to do it manually as well.)

\item If a new package or module has been added, it must go into all the build and
compile files for \ETPS and \tps. (See, for example,
{\it /afs/andrew/mcs/math/TPS/common/tps-compile.lisp}.)  In general, it should
go into the ends of the list of modules, so that definitions it
depends upon will be loaded first. If a new module is added, be sure to add it 
to the {\it \indexfile{facilities.lisp}} and {\it \indexfile{facilities-short.lisp}}
files, otherwise it won't show up in the facilities guide.

\item After installing new code, remember to change the patch files, the {\it \indexfile{tpsjobs}}
file and the {\it \indexfile{tpsjobs-done}} file, and to send a mail message to the other 
people working on the program.
\end{itemize}

\section{\indexother{Debugging} Hints}

\begin{itemize}
\item Insert print commands in a temporary version of a file to see either 
which parts of the code are being used or what the current values of some
variables are.

\item Compile the file in several common lisps, especially in
cmulisp (or tps3cmu), and see if the error messages are helpful.

\item Try to reproduce the bug in a simpler form.

\item See in how many different contexts (such as different
matingsearch procedures) it arises, so you can isolate its essential
features.

\item Use the debugging features of your version of lisp (e.g. step and trace).

\item Change the values of the flags \indexflag{QUERY-USER}, \indexflag{MATING-VERBOSE},
\indexflag{UNIFY-VERBOSE}, \indexflag{TACTIC-VERBOSE}, 
\indexflag{OPTIONS-VERBOSE}, etc... to get more output.

\item Use the monitor. (See chapter ~\ref{monitor}.)

\item Use the lisp function {\tt plist} to inspect the property list of an object.
Use {\tt inspect} to see the values of the slots in a structure.

\item Errors in translation, or errors during verification of a merged jform ("The formula is not provable 
as there is no connection on the following path") are usually caused by merging. See the chapter on
merging, and in particular the note about using merge-debug.

\item The code in the file \indexfile{etrees-debug} can be useful for tracking
down bugs involving etrees.  (See subsection ~\ref{etrees}.)

\item Errors of the form "Wff operation <wffop> cannot be applied to labels of flavor <label>" are almost always
caused by attempting to use a wffop on a flavor for which the corresponding property is undefined. See the section
on flavors for more details.

\item Errors in structure-slot-accessor are often of the form "Structure for accessor <foo-slot> is not a <foo>".
For every structure <foo>, there is a test <foo-p>; use it! Of course, you should also work out how 
something that wasn't a <foo> managed to turn up at that point in the program anyway; often, it's an exceptional
case that you forgot to handle.

\item In Allegro, the function \indexfunction{dumplisp} can be used to save a Lisp image.
For example,
\verb+(excl:dumplisp :name ``saved-image.dxl'')+
will create a (large) file named ``saved-image.dxl''.  Then one can
use \verb+lisp -I saved-image.dxl+
to start lisp specifying this as the image file.
This will start lisp in the same
state (e.g., the global variables will have the same values)
as when dumplisp was called.  This is especially useful if the
bug shows up after running for a long time.

\item If the bug is new (for example, if you know it wasn't there last month), don't forget 
that the {\it \indexfile{tpsjobs-done}} file lists all of the files which have been changed, along 
with the reasons for each change and the date of each change. The {\it tps/tps/old-source/} directory
should contain backup copies of the changed files. Failing that, snapshots of the entire 
lisp directory (in the form of gzipped tar files made after each rebuild) are stored 
in the {\it tps/tps/tarfiles/} directory. Use cload to restore the old copies of the most likely
culprit files into a core image, until the bug disappears; then use ediff to
compare the old and new files.
\end{itemize}

\section{Miscellaneous}

\subsection{Counting Flags}

One can count the number of flags in \TPS as follows:

%\begin{tpsexample}
\begin{verbatim}
[btps]/afs/andrew/mcs/math/TPS/lisp% grep -i defflag *.lisp > flagcount
[btps]/afs/andrew/mcs/math/TPS/lisp% ls -l flagcount
{\it Edit the file flagcount to eliminate lines which do not define flags}
[btps]/afs/andrew/mcs/math/TPS/lisp% wc flagcount
     210     421    8327 flagcount
{\it The number of lines (210} is the number of flags.)
[btps]/afs/andrew/mcs/math/TPS/lisp% rm flagcount
\end{verbatim}
%\end{tpsexample}

The above counts the number of flags defined in the source code. The
number currently present in a particular version of \TPS can be found as follows:

%\begin{tpsexample}
\begin{verbatim}
(defun discard (list)
	(if (null list) nil
	  (if (or (listp (car list)) (memq (car list) (cdr list)))
;; if it's a list, it's a subject name, and we don't want to count them.
;; if it appears later on, we don't want to count it twice.
;; (may need to use franz:memq rather than memq)
	    (discard (cdr list))
	      (cons (car list) (discard (cdr list))))))

(msg "TPS has " (length (discard global-flaglist)) " flags.")
\end{verbatim}
%\end{tpsexample}

\subsection{Dealing with X Fonts}

\begin{itemize}
\item To enable your computer to find the fonts, put into an appropriate 
(such as {\it .Xclients}) file: {\tt xset }fp /tps/fonts/+,
using the appropriate pathname in place of {\it /tps/fonts/}. The {\tt }fp+
adds the new directory at the start of the path, because we've had trouble 
in the past with old fonts with the same name being earlier on in the path.

\item {\tt xset q} shows the fonts.

\item {\tt xset fp-} takes them out of the fontpath.

\item {\tt xlsfonts} lists the fonts available
Because the font list is usually very long,
you may prefer to use
{\tt xlsfonts | grep <fontname>} to check whether the font <fontname> 
is available.

\item {\tt xfd -fn <fontname> \&} shows all the characters in <fontname>
\end{itemize}

Dan Nesmith built the symbol fonts by starting with
the vtsingle font, because xterm requires a font that is
exactly the same size as vtsingle.  However, every
character is now different from the original; they were created by hand-colouring 
the pixels. It is, however, easier to
edit an existing font than to create one from scratch.

The exact duplicates of the Concept ROM fonts are found in symfont1.bdf and
symfont2.bdf, for a total of 256 characters (including a normal-sized epsilon).
Unfortunately, because of the limits below on xterm and lisp, it was necessary to
leave out some of these characters when making a single font, now called
vtsymbold.   The galsymbold font was created by splitting the font vtsymbold
into 128 bitmaps, then using an X10 program that would automatically blow
each bitmap to the proper size, then manually adjusting a few of
the characters.

There is a real limitation of the xterm program in that you get only two
fonts, one for normal text and one for bold text.  
We use the bold text
font for symbols, and switch back and forth between the fonts by sending
the appropriate escape symbols.  

% Inserted by ceb Jan 18 2005
The code for the ``appropriate escape symbols'' are in \indexfile{xterm.lisp}.
In this file functions
\indexfunction{xterm-bold-font}, \indexfunction{xterm-normal-font} 
and \indexfunction{pptyox} 
use codes to switch between bold and normal font.
There is currently some confusion about
how one switches to bold font.  The code in \indexfile{xterm.lisp}
switches to bold by sending (in ASCII) \verb+<ESC> [ 5 m+.
However, it appears the official ANSI code for switching
to bold is \verb+<ESC> [ 1 m+ 
while \verb+<ESC> [ 5 m+ is for blinking text:

\begin{verbatim}
  (see http://members.tripod.com/~oldboard/assembly/ansi_codes.html)

  ESC[n;n;...nm

  Set Graphics Rendition is used to set attributes as
  well as foreground and background colors.  If multiple
  parameters are used, they are executed in sequence, and
  the effects are cumulative. 'n' is one of the following
  attributes or colors:

  0  All attributes off
  1  Bold
  2  Dim
  4  Underline
  5  Blink
  7  Reverse Video
  8  Invisible
\end{verbatim}

The working hypothesis (of Chad) at the moment is
that the implemenations of xterm we have used for
\TPS render ``blinking text'' as bold, so that switching
to blink is the same as switching to bold.  In 2005,
while using xterm version \verb+X.Org 6.7.0(192)+
for \TPS, the symbols were not displaying as symbols.
Instead, what should have been symbols were blinking
normal text.  This can be fixed by changing the
codes in \indexfile{xterm.lisp} to send \verb+<ESC> [ 1 m+ 
instead of \verb+<ESC> [ 5 m+.

It is possible there was good reason why the original programmer (Dan Nesmith?)
used blink instead of bold.  Instead of explicitly making the change,
there is a system flag \indexflag{XTERM-ANSI-BOLD} with default value
53 (the ASCII code for 5).  The value of this flag can be changed to
49 (the ASCII code for 1) when needed.  Perhaps in the future
the flag could be deleted and the change hard-coded once someone
is confident of what the number should always be.\footnote{In 2005 (and previously), 
a value of 53 worked for XTERM-ANSI-BOLD at
CMU while using xterm version XFree86 4.2.0(165), and a value of 49
worked at Saarbrucken while using xterm version X.Org 6.7.0(192).}

It would be nice if xterm could render more special symbols.
% End ceb Jan 18 2005 Insert
Though xterm would probably allow using
more than the "printable" characters, this appears very hard to get
lisp to do in general, that is, it's hard to figure out how to get lisp to
send these characters to the terminal.  This will probably never be
completely implementation-independent, because character sets are a very
unstable part of the lisp specification.

In \TPS you will only be able to get the symbols between 32 and 127, basically
because the lisp allows only those (without some kind of great hackery).

A few lines of attack for getting more characters suggest themselves:

\begin{enumerate}
\item Hack xterm to allow more than 2 fonts. (In fact, Dan has
done it to allow one more font, and thinks it would be possible to add up
to two more, for a total of one normal font and three symbol fonts.)
The disadvantage here would be having to distribute the new version of xterm,
and worrying about portability problems (which actually should be minimal,
but with lots of different machines and versions of X out there, not
predictable).  But this approach is fairly easy to get to from the current
state.

\item Give up on xterm.  There is a new version of gnu-emacs, version 19,
which allows the use of more than one font in a buffer.  You can then run
\TPS in a gnu-emacs buffer, and at the same time build in support for
command-completion, hypertext documentation, etc.  You can also get several
windows from a single emacs now, so you can still have the editor stuff
pop up a separate window.  It might also be easier to bind in support
for automatically running tex or scribe and displaying it.
This requires someone who can hack gnu-emacs lisp, but a lot of this
stuff has already been done by somebody, and it's just a matter of putting
it together.   In this case you could probably roll the fonts together into
a single large one.
\end{enumerate}
