\section{Search Analysis: Facilities for Setting Flags and Tracing Automatic Search}\label{searchanalysis}

If one has a natural deduction proof of a theorem, one can use the
command \indexcommand{AUTO-SUGGEST} to obtain suggested settings for certain flags of
a mode with which that theorem can be proved automatically.
\indexcommand{AUTO-SUGGEST} will also show all of the instantiations of quantifiers
that are necessary for that proof.  The command \indexcommand{ETR-AUTO-SUGGEST} does
the same thing when given an expansion proof. Such an expansion proof
could be the result of translating a natural deduction proof into an
expansion proof using the command \indexcommand{NAT-ETREE}.
The commands \indexcommand{MS03-LIFT} and \indexcommand{MS04-LIFT} 
in the \indexother{EXT-MATE} top level
suggests flag settings for the corresponding extensional search
procedures when given an extensional expansion proof.

One can obtain an expansion proof for a gwff by several methods,
including constructing a mating by hand
in the \indexcommand{MATE} top level.
An easier way is to use \indexcommand{NAT-ETREE} (see section \ref{natetr})
to translate a natural deduction proof into an expansion proof.
Once we have an expansion proof, we can use this to suggest
flag values and to trace the \indexother{MS98-1} search procedure.

In this section we will consider two examples: THM12
and X2116.

\subsection{Example: Setting Flags for THM12}\label{THM12-NAT-ETREE}

Our first example concerns THM12:

$\forall R _{\greeko\greeki}   \forall S _{\greeko\greeki} .R = S \implies  \forall X _{\greeki} .S X \implies R X$

The following excerpts from a TPS session shows how we can
use \indexcommand{NAT-ETREE} to get an expansion proof
and then use this expansion proof to suggest flag settings.

\begin{tpsexample}
<401>prove THM12
. . . ; ***prove the theorem, perhaps interactively***
<431>pstatus
 No planned lines
<432>nat-etree
PREFIX (SYMBOL): Name of the Proof [THM12]>
Proof THM12-2 restored.

. . . ; ***nat-etree preprocesses the proof,
. . . ;    converts it to a sequent calculus derivation,
. . . ;    performs cut elimination
. . . ;    then translates to an expansion tree with a complete mating.***
|            L117             |
| \(\sim\)R^152 X^141 OR S^17 X^141  |
|                             |
|            L115             |
| \(\sim\)S^17 X^141 OR R^152 X^141  |
|                             |
|            L116             |
|\(\sim\)[\(\sim\)S^17 X^141 OR R^152 X^141]|
Number of vpaths: 1
((L115  . L116 ))    ; note this is a connection between nonatomic wffs

Adding new connection: (L115 . L116)
If you want to translate the expansion proof back to a natural deduction proof,
you must first merge the etree.  If you want to use the expansion proof to determine
flag settings for automatic search, you should not merge the etree.
Merge The Etree? [Yes]>n           ; ***we don't merge the tree***
The expansion proof Eproof:EPR122  can be used to trace MS98-1 search procedure.
The current natural deduction proof THM12-2 is a modified version
of the original natural deduction proof.

Use RECONSIDER THM12 to return to the original proof.

<433>mate
GWFF (GWFF0-OR-LABEL-OR-EPROOF): Gwff or Eproof [Eproof:EPR122 ]>
DEEPEN (YESNO): Deepen? [Yes]>n
REINIT (YESNO): Reinitialize Variable Names? [Yes]>n
WINDOW (YESNO): Open Vpform Window? [No]>
. . .
<Mate437>etr-info  ; ***etr-info lists the expansion terms***
Expansion Terms:
X^141(I)           ; ***only one (easy) expansion term in the proof***

<Mate438>etr-auto-suggest
. . .
MS98-INIT suggestion: 1
MAX-SUBSTS-VAR should be 1
NUM-OF-DUPS should be 0
MS98-NUM-OF-DUPS should be 1
MAX-MATES should be 1
Do you want to define a mode with these settings? [Yes]>
Name for mode?  [MODE-THM12-2-SUGGEST]>

<Mate439>leave
Merge the expansion tree? [Yes]>n   ; ***let's still not merge***


<440>mode MODE-THM12-2-SUGGEST
\end{tpsexample}

This suggested mode will prove the theorem.  In a later
section we will continue this example to see how we can
use the eproof to trace the \indexother{MS98-1} search procedure.

\subsection{Example: Setting Flags for X2116}\label{X2116-NAT-ETREE}

Suppose we have a natural deduction proof for X2116:

$\forall x _{\greeki}   \exists y _{\greeki}  [P _{\greeko\greeki}  x \implies R _{\greeko\greeki\greeki}  x [g _{\greeki\greeki}  .h _{\greeki\greeki}  y] \and P y] \and  \forall w _{\greeki}  [P w \implies P [g w] \and P .h w]
 \implies  \forall x .P x \implies  \exists y .R x y \and P y$

The following excerpts from a TPS session shows how we can
use \indexcommand{NAT-ETREE} to get an expansion proof
and then use this expansion proof to suggest flag settings.

\begin{tpsexample}
<405>nat-etree
PREFIX (SYMBOL): Name of the Proof [X2116]>
. . .
|              |        L90        |   |
|    L89       |R x^329 [g .h y^70]|   |
|  \(\sim\)P x^329 OR |                   |   |
|              |       L92         |   |
|              |      P y^70       |   |
|                                      |
|                   |    L102     |    |
|        L99        |P [g .h y^70]|    |
|    \(\sim\)P [h y^70] OR |             |    |
|                   |    L104     |    |
|                   |P [h .h y^70]|    |
|                                      |
|                  |   L100   |        |
|         L96      |P [g y^70]|        |
|       \(\sim\)P y^70 OR |          |        |
|                  |   L98    |        |
|                  |P [h y^70]|        |
|                                      |
|                 L88                  |
|               P x^329                |
|                                      |
|        L91                  L103     |
|\(\sim\)R x^329 [g .h y^70] OR \(\sim\)P [g .h y^70]|
Number of vpaths: 16
((L102  . L103 ) (L98  . L99 ) (L92  . L96 ) (L88  . L89 ) (L90  . L91 ) (L88  . L89 ))

. . .
If you want to translate the expansion proof back to a natural deduction proof,
you must first merge the etree.  If you want to use the expansion proof to determine
flag settings for automatic search, you should not merge the etree.
Merge The Etree? [Yes]>n
The expansion proof Eproof:EPR116  can be used to trace MS98-1 search procedure.
The current natural deduction proof X2116-1 is a modified version
of the original natural deduction proof.

Use RECONSIDER X2116 to return to the original proof.

<406>mate
GWFF (GWFF0-OR-LABEL-OR-EPROOF): Gwff or Eproof [Eproof:EPR116 ]>
DEEPEN (YESNO): Deepen? [Yes]>n
REINIT (YESNO): Reinitialize Variable Names? [Yes]>n
WINDOW (YESNO): Open Vpform Window? [No]>n
. . .
<Mate409>show-mating   ; ***note the mating has six connections***

Active mating:
(L88 . L89)  (L90 . L91)  (L88 . L89)
(L92 . L96)  (L98 . L96.1)  (L100.1 . L103)

<Mate410>etr-info  ; ***there are 4 expansion terms, with reasonable sizes***
Expansion Terms:
g(II).h(II) y^70(I)
h(II) y^70(I)
y^70(I)
x^329(I)

<Mate411>etr-auto-suggest ; ***to get suggested flag settings***
. . .
MS98-INIT suggestion: 1
MAX-SUBSTS-VAR should be 3
NUM-OF-DUPS should be 1
MS98-NUM-OF-DUPS should be 1
MAX-MATES should be 1
Do you want to define a mode with these settings? [Yes]>
Name for mode?  [MODE-X2116-1-SUGGEST]>


<Mate412>mode MODE-X2116-1-SUGGEST
\end{tpsexample}

These flag settings are sufficient for \indexother{MS98-1} to prove the
theorem.  Later we will continue this example to show how
to use the expansion proof to trace \indexother{MS98-1}.

\subsection{Tracing MS98-1}\label{ms98-trace}

Currently there are some facilities for tracing the automatic search
procedure \indexother{MS98-1} (see Matt Bishop's thesis for details
of this search procedure).  The flag \indexflag{MS98-VERBOSE} can
be set to {\tt T} to simply obtain more output.  If we have an expansion
proof already given, we can obtain a finer trace on MS98-1 to find
information about the search.  For example, we may be able to find
which connections failed to be added to the mating.

A background expansion proof may be the result of calling
\indexcommand{NAT-ETREE} as described above.
We may also construct a complete mating interactively
in the \indexcommand{MATE} top level, then use
\indexcommand{SET-BACKGROUND-EPROOF} to save this expansion
proof as the one to use for tracing.  Consider the following
example session showing how we might get a mating
for X2116 (instead of using \indexcommand{NAT-ETREE}
as in section \ref{X2116-NAT-ETREE}).

\begin{tpsexample}
<0>exercise x2116
Would you like to load a mode for this theorem? [No]>y
2 modes for this theorem are known:
1) MODE-X2116  1999-04-23  0 seconds  (read only)
2) MS98-FO-MODE  1999-04-23  1 seconds  (read only)
3) None of these.
Input a number between 1 and 3: [1]>
. . .

<1>mate 100 y n n
<Mate2>go
. . .
Eureka!  Proof complete..
. . .
<Mate3>show-mating

Active mating:
(L21.1 . L14.1)  (L20.1 . L9)  (L12.1 . L15)
(L7.1 . L10)  (L12 . L10)  (L7 . L17)
(L21 . L15)
is complete.

<Mate4>set-background-eproof
EPR (EPROOF): Eproof [Eproof:EPR0 ]>
\end{tpsexample}

Once there is a background expansion proof,
the information given by the mating in the background proof
can be transferred to a search expansion tree via `colors'.
For each connection, we create a `color' (really just
a generated symbol) which is associated with all nodes
in the search expansion tree which could correspond to the connected
nodes in the background tree.  The examples in subsections
\ref{THM12-NAT-ETREE} and \ref{X2116-NAT-ETREE}
should make the role of colors clearer.

The flag \indexflag{MS98-TRACE} can
be used to obtain information about how the search is performing
relative to the background eproof.  The value of this flag is a list of
values.  Ordinarily, when tracing is off, \indexflag{MS98-TRACE}
will have the value NIL.  Certain symbols have the following meanings
if they occur on the value of \indexflag{MS98-TRACE}:

\begin{description}
\item[] {\tt MATING} -- If the symbol MATING is on the list,
search as usual, printing when `good' connections and components
are found.
A `good' connection is one in which the two nodes share a common color.
A `good' component contains only good connections.
A search succeeds when it generates all the good connections, and combines
these into good components, merging these components until the complete mating
is generated.  Note that successfully generating connections and merging components
depends on unification, and in particular, on the value of \indexflag{MAX-SUBSTS-VAR}.

\item[] {\tt MATING-FILTER} -- This prints the same information as {\tt MATING}, but
only generates good connections and only builds good components.
This value is useful for finding if the unification bounds (\indexflag{MAX-SUBSTS-VAR})
are set in such a way that the search can possibly succeed.

\end{description}

After setting \indexflag{MS98-VERBOSE} and \indexflag{MS98-TRACE},
one can invoke the search procedure \indexother{MS98-1} using
the mate command \indexcommand{MS98-1} or the mate command
\indexcommand{GO} if \indexflag{DEFAULT-MS} is set to \indexother{MS98-1}.

The next two subsections show how the tracing works
in practice by continuing the examples in subsections
\ref{THM12-NAT-ETREE} and \ref{X2116-NAT-ETREE}.

\subsection{Example: Tracing THM12}\label{THM12-MS98-TRACE}

Suppose the background expansion proof is a proof for THM12,
and suppose we are in a mode such as the suggested mode
from \indexcommand{ETR-AUTO-SUGGEST} in section \ref{THM12-NAT-ETREE}.

First, examine the background expansion proof more closely.
The expansion proof and jform are shown in Figure \ref{thm12-vp}.

\begin{figure}
\begin{tpsexample}
<459>ptree*
                                        [SEL11]
                                      [ R^152(OI) ]
                                            |
                                            |
                                            |
                                        [SEL10]
                                      [ S^17(OI) ]
                                            |
                                            |
                                            |
                                        [IMP86]
            R^152(OI) = S^17(OI) IMPLIES FORALL X(I).S^17 X IMPLIES R^152 X
                                            |
                      /-------------------------------------------\
                      |                                           |
                  [REW10]                                      [SEL9]
                   [EXT=]                                   [ X^141(I) ]
                      |                                           |
                      |                                           |
                      |                                           |
                   [EXP7]                                      [L116]
                [ X^141(I) ]                  S^17(OI) X^141(I) IMPLIES R^152(OI) X^141
                      |                                           |
                      |
                      |
                   [REW9]
                   [EXT=]
                      |
                      |
                      |
                   [REW8]
              [EQUIV-IMPLICS]
                      |
                      |
                      |
                  [CONJ74]
                      *
                      |
           /---------------------\
           |                     |
        [L117]                [L115]
           * S^17(OI) X^141(I) IMPLIES R^152(OI) X^141

<460>vp ; ***JFORM***
|                       SEL11                        |
|EXISTS R EXISTS S [R = S AND EXISTS X .S X AND \(\sim\)R X]|
|                                                    |
|                        L117                        |
|             \(\sim\)R^152 X^141 OR S^17 X^141             |
|                                                    |
|                        L115                        |
|             \(\sim\)S^17 X^141 OR R^152 X^141             |
|                                                    |
|                       L116                         |
|           \(\sim\)[\(\sim\)S^17 X^141 OR R^152 X^141]            |
Number of vpaths: 1

<461>show-mating   ; ***Complete Mating***
(L115 . L116)
\end{tpsexample}
\caption{Expansion Tree and JForm for THM12}
\label{thm12-vp}
\end{figure}

There is only one connection, so this should correspond to a single
color.  This is an interesting example because the search expansion tree will
have a somewhat different form.  Since an EQUIV occurs in the formula,
the form of the tree will depend on the value of \indexflag{REWRITE-EQUIVS}.

First, suppose \indexflag{REWRITE-EQUIVS} is set to 1.  The search
expansion tree in this case is shown in Figure \ref{thm12-search-etree1}.

\begin{figure}
\begin{tpsexample}
<466>ptree*
                                        [SEL11]
                                      [ R^152(OI) ]
                                            |
                                            |
                                            |
                                        [SEL10]
                                      [ S^17(OI) ]
                                            |
                                            |
                                            |
                                        [IMP86]
                                            |
                                            |
                      /-------------------------------------------\
                      |                                           |
                  [REW10]                                      [SEL9]
                   [EXT=]                                   [ X^141(I) ]
                      |                                           |
                      |                                           |
                      |                                           |
                   [EXP7]                                      [L116]
                [ X^141(I) ]                                      |
                      |                                           |
                      |
                      |
                   [REW9]
                   [EXT=]
                      |
                      |
                      |
                   [REW8]
              [EQUIV-IMPLICS]
                      |
                      |
                      |
                  [CONJ74]
                      |
                      |
           /---------------------\
           |                     |
        [L117]                [L115]
           |                     |
\end{tpsexample}
\caption{Search Expansion Tree with REWRITE-EQUIVS 1}
\label{thm12-search-etree1}.
\end{figure}

To use tracing, set MS98-TRACE to an appropriate value,
such as (MATING) or (MATING MATING-FILTER).
Then execute the DIY command (or the MS98-1 command in
the mate top level), and {\TPS} will color the nodes
before performing mating search.

A new color corresponding to the (L115 . L116) connection is created.
{\TPS} needs to ensure all nodes corresponding to L115 and L116 are given this color.
When trying to find the nodes that correspond to L115 (in the background),
{\TPS} first runs into a problem at the correspondence between
REW8 (in the background) to REW2 (in the search expansion tree).  In REW8,
EQUIV is expanded as a conjunction of implications.  In the search
tree, EQUIV is expanded as a disjunction of conjunctions.
So, {\TPS} colors every node beneath REW2.  To find the nodes that correspond
to L116.  {\TPS} finds IMP1 corresponds structurally, so {\TPS} colors every node
beneath IMP1.  As a result, every leaf gets the single color in this case.
So, tracing is basically useless in this case.  Every two literals will
share the only color, and so will be considered `good'.

However, if REWRITE-EQUIVS is set to 4, the EQUIV in the search expansion tree
will be expanded as a conjunction of implications.  This will make
the search expansion tree correspond more closely to the background tree.
This is shown in Figure \ref{thm12-search-etree2}.

\begin{figure}
\begin{tpsexample}
<475>ptree*
                                         [SEL0]
                                      [ R^159(OI) ]
                                            |
                                            |
                                            |
                                         [SEL1]
                                      [ S^24(OI) ]
                                            |
                                            |
                                            |
                                         [IMP0]
                                            |
                                            |
                      /-------------------------------------------\
                      |                                           |
                   [REW0]                                      [SEL2]
                   [EXT=]                                   [ X^148(I) ]
                      |                                           |
                      |                                           |
                      |                                           |
                   [EXP0]                                      [IMP3]
                [ x^346(I) ]                                      |
                      |                                           |
                      |                                /---------------------\
                      |                                |                     |
                   [REW1]                            [L16]                 [L17]
                   [EXT=]                              |                     |
                      |                                |                     |
                      |
                      |
                   [REW2]
              [EQUIV-IMPLICS]
                      |
                      |
                      |
                  [CONJ0]
                      |
                      |
           /---------------------\
           |                     |
        [IMP1]                [IMP2]
           |                     |
           |                     |
      /---------\           /---------\
      |         |           |         |
    [L11]     [L12]       [L13]     [L14]
      |         |           |         |

\end{tpsexample}
\caption{Search Expansion Tree with REWRITE-EQUIVS 4}
\label{thm12-search-etree2}.
\end{figure}

In this case, we can see that L115 corresponds to the node IMP2.
So, {\TPS} colors IMP2, L13, and L14.  L116 corresponds to the node IMP3,
so {\TPS} colors IMP3, L16, and L17.
This time the leaves L11 and L12 are left uncolored,
so the trace should have an effect.

Suppose \indexflag{MS98-TRACE} is set to (MATING MATING-FILTER).

\begin{tpsexample}
<Mate500>ms98-1
Substitutions in this jform:
None.
Transfering mating information from background eproof
Eproof:EPR122
transfering conn (L115 . L116)
COLORS:
COLOR62 - L14 L13 L17 L16
. . .
Trying to Unify L17 with L14  ; ***both have COLOR62***
Component 1 is good:
. . .
Trying to Unify L16 with L13  ; ***both have COLOR62***
Component 2 is good:
. . .
Component 3 is good:
Success! The following is a complete mating:
((L17 . L14) (L16 . L13))
. . .
\end{tpsexample}

\subsection{Example: Tracing X2116}\label{X2116-MS98-TRACE}

Suppose the background expansion proof is a proof for X2116,
and suppose we are in a mode such as the suggested mode
from \indexcommand{ETR-AUTO-SUGGEST} in section \ref{THM12-NAT-ETREE}.

Since the mating in this example has six connections, there
will be six colors.

\begin{tpsexample}
<414>exercise x2116
. . .
<415>mate 100 y n n
. . .
<Mate417>ms98-trace
MS98-TRACE [(MATING MATING-FILTER)]>

<Mate418>ms98-1
Substitutions in this jform:
None.
Transfering mating information from background eproof
Eproof:EPR116
transfering conn (L88 . L89)
transfering conn (L90 . L91)
transfering conn (L88 . L89)
transfering conn (L92 . L96)
transfering conn (L98 . L99)
transfering conn (L102 . L103)
COLORS:
COLOR50 - L14 L14.1 L21 L21.1
COLOR49 - L15 L15.1 L12 L12.1
COLOR48 - L10 L10.1 L12 L12.1
COLOR47 - L17 L7 L7.1
COLOR46 - L9 L9.1 L20 L20.1
COLOR45 - L17 L7 L7.1
\end{tpsexample}

We can see in this example that the colors closely correspond
to the literals used in each connection.  Note also that
duplicate leaves get a common color, because any of the
children of an expansion node can correspond to any children
of the corresponding expansion node.  For example, L102
in the background tree could correspond to either
L14 or L14.1 in the search expansion tree.

In the session, {\TPS} continues searching, printing information about
`good' components, and filtering out those which are not `good'.
