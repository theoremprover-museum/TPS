\chapter{Writing Inference Rules}

Information on how to write new rules using the rules module
is in the User Manual.

The chapter about \ETPS (chapter ~\ref{etps}) has some discussion
of how default line numbers and default hypotheses are generated in 
the various {\tt OUTLINE} modules. This should correspond fairly 
closely to the way in which the automatically-generated rules
generate their defaults. The same chapter also has a discussion of
support transformations which illustrates the way in which rules are
defined.

{\bf Do not edit the automatically-generated files.} This means any file 
{\it whatever.lisp} that has a companion file {\it whatever.rules}. If you edit 
the lisp files directly, your changes will be lost if the rules are ever recompiled.

There are some rules which are defined directly as commands (mexpr's).
Examples are \indexcommand{RULEP}, \indexcommand{TYPESUBST}, \indexcommand{ASSERT}, 
\indexcommand{ADD-HYPS} and \indexcommand{DELETE-HYPS}.
In order for these commands to show up when \indexcommand{LIST-RULES}
is called, the programmer should modify the definition of \indexfunction{list-rules}
in \indexfile{otl-help.lisp} to explicitly include them.
For example, \indexcommand{RULEP} is explicitly included by the following code:
\begin{verbatim}
    (when (get 'ML::rulep 'mainfns)
      (push 'RULEP gsrl)
      (push '(ML::RULES-2-PROP) gsrl))
\end{verbatim}
This includes \indexcommand{RULEP} in the context \indexother{ML::RULES-2-PROP}.


