\chapter{ETPS}\label{etps}

\section{The Outline Modules}

The {\tt OUTLINE} modules in TPS have two main subparts; 
the bookkeeping functions, and the \indexcommand{GO} command which
gives sophisticated help or constructs a proof automatically.
They are collected in the modules of the form {\tt OTL*}.

In ETPS only the bookkeeping functions are present.

The discussion below is aimed at understanding the {\tt OUTLINE} modules
independently of the system, but we generally assume we are working in
\ETPS.  If \TPS differs, this is noted.

We often talk about {\it proofs}, even though they are properly only incomplete
proofs or {\it proof outlines}.  It is assumed that the reader knows what
planned lines ({\it plines}) and deduced lines ({\it dlines}) are.  This and 
general familiarity with \ETPS are necessary to understand this discussion.

\subsection{Proofs as a Data Structure}

Proofs in \ETPS are represented by a single atom with a variety of properties.
The global variable {\tt DPROOF} has as value the name of the current proof.
In case you are working, say on exercise {\tt X6200}, {\tt DPROOF} will have the 
value {\tt X6200}.  The current proof name then has a variety of properties.

\begin{description}
\item [{\tt LINES}] {\tt ({\it line} {\it line} ...)}.  This is simply an ordered list of all
lines in the current proof without repetition.  The order is such that
lines with a lower number appear first in the list.
\\ {\bf WARNING}:  This property is frequently changed destructively.
As a consequence it may never be empty and should be used for other
purposes only in a copy.

\item [{\tt LINEALIASES}] {\tt (({\it line} . {\it no})({\it line} . {\it no}) ...)}.  This is an 
unordered association list correlating lines with their numbers.  No line 
should ever appear in more than one pair, and neither should a number.  Try to 
think of arguments for and against this representation, compared to one where
the number of a line is stored on the line's property list.
\\ {\bf WARNING}:  This property is frequently changed destructively.
As a consequence it may never be empty and should be used for other
purposes only in a copy.

\item [{\tt PLANS}] {\tt {(({\it pline} . {\it supportlist})({\it pline} . {\it supportlist}) ...)}}.
This stores the important {\it plan}-{\it support} {\it structure} of the current
proof.  {\it pline} is a still unjustified line in the current proof,
{\it supportlist} is a list of deduced lines supporting a {\it pline}.
A {\it pline} may never have a justification other than {\tt PLAN}{\it i}, 
a {\it sline} (support line) must always be completely justified, i.e.
may not ultimately depend on a planned line.

The association list is ordered such that the most recently affected
{\it pline} is closer to the front of the list.  The order can be changed
explicitly with the {\tt SUBPROOF} command.

{\bf WARNING}:  This property is frequently changed destructively.
As a consequence it may never be empty and should be used for other
purposes only in a copy.

\item [{\tt GAPS}] ({\it gap} {\it gap} ...). This is a list of the gaps between lines in the
proof. Each gap has the properties ({\tt MIN-LABEL} {\it line}) and ({\tt MAX-LABEL} {\it line}).

\item [{\tt NEXTPLAN-NO}] {\it integer}. This is just the next number that will be used for 
a planned line.

\item [{\tt ASSERTION}] {\it gwff}. This is the assertion being proven.
\end{description}

When a proof is saved using the \indexcommand{SAVEPROOF} command, a checksum may be generated.
This is used by \ETPS to verify that the saved proof has not been manually edited by a student
(otherwise it would be possible to edit out the planned lines and convince \ETPS to issue the 
\indexcommand{DONE} command). Since it takes time to generate the checksum, it is only 
generated if the flag \indexflag{EXPERTFLAG} is NIL. This means that proofs written by \TPS
with \indexflag{EXPERTFLAG} T cannot be read into \ETPS with \indexflag{EXPERTFLAG} NIL.


\subsection{Proof Lines as a Data Structure}

Proof lines in \ETPS have a variety of properties:

\begin{description}
\item [{\tt REPRESENTS}] This is the wff asserted by the line.  In the original 
TPS this had to be an
atomized wff of very particular structure, which lead to numerous problems
in higher-order logic.  In \ETPS this has been maintained for the present. Our
goal, of course, is to allow arbitrary {\it gwffs} as {\tt REPRESENTS} of lines.

\item [{\tt HYPOTHESES}] This is a list of lines assumed as hypotheses for the line.
The list of hypotheses is ordered (lowest numbered line first), but to
my knowledge no function assumes this.  It simply looks better in the output.
No line should appear twice as an hypothesis (this fact may actually be used
here and there).

\item [{\tt JUSTIFICATION}]  {\tt {{\it RULE} {\it gwfflist} {\it linelist}}} The line can be
inferred by an inference rule {\it RULE}from {\it linelist}.  
{\it gwfflist} has somehow been used to infer the line.

\item [{\tt LINENUMBER}] The line number associated with the line.
\end{description}


\section{Defaults for Line Numbers - a Specification}

There will never be an absolutely correct way
of assigning default for line numbers; we can merely make sure that
the result will always be logically correct - the rest is often
a matter of style and the kind of heuristics used.

Below we give a description of the tasks to be done by a function
\indexfunction{LINE-NO-DEFAULTS} which is called during the application of every
inference rule {in interactive mode}.  We will set the stage by giving
some, not necessarily exhaustive, examples of what meaning to assign
to the data structures and what output to expect from the function.

\subsection{The {\it support} data structure}

At each stage in a proof, we have associated with it a {\it \indexother{support}} structure,
which, for any given planned line ({\it \indexother{pline}}), tells us which deduced lines
({\it \indexother{dline}s}), we expect to use in the proof of the {\it pline}.

Thus the support structure is of the form
$$((p_1\; d_{11} \ldots d_{1x_1})\ldots (p_p\; d_{p1}\ldots d_{px_p}))$$

One may assume the following:
\begin{enumerate}
\item The $p_i$ are pairwise distinct.

\item The $d_{ik}$ are pairwise distinct for every fixed $i$ and $1\leq k \leq x_i$.

\item For each $i$, $d_{ik} < p_i$ for all $1\leq k \leq x_i$.

\item The planned lines $p_i$ are ordered such that the ones the user is
expected to work on first appear closer to the front.  In particular,
$p_1$ is the planned line worked on most recently.

\item Similarly, for a given $i$, the $d_{ik}$ are ordered such that the
one the user is expected to use first appear earlier.
\end{enumerate}

With each rule definition, there will be a description of how the {\it support}
structure changes.  This is given as two {\it support} structure templates,
using the name given to the lines in the rule specification.

\subsection{Examples}

The examples below are not complete, in the sense that not the full
description of the rule (for the rules module) is given, we have merely
extracted what is important in our context.  {\it p} and {\it d} are
placeholders for a {\it pline} or any number of {\it dlines}, respectively,
which are found in the support structure of the current proof, but are
merely copied in the application of the particular rule described.

\begin{verbatim}
Rule of Cases

*(D1)  H      !A(O) OR B(O)                                          
 (H2)  H,H2   !A(O)                                Case 1: D1
 (P3)  H,H2   !C(O)                                          
 (H4)  H,H4   !B(O)                                Case 2: D1
 (P5)  H,H4   !C(O)                                          
*(P6)  H      !C(O)                           Cases: D1 P3 P5

Support Transformation: (P6 D1 ss) ==> (P3 H2 ss) (P5 H4 ss) 
\end{verbatim}

Note that the specified support transformation tells \TPS what lines
it expects to be there, when the rule is applied, and which lines
should be new.  In this case, {\tt P} and {\tt Dab} are expected to be new,
the others are to be constructed.  Of course, these are only defaults,
and the user can apply the rule with any combination of lines present or
absent.


\begin{verbatim}
Induction Rule

 (D1) H     ! P 0
 (H2) H,H2  ! P m                           Inductive Assumption on m
 (D3) H,H2  ! P . Succ m
*(P4) H     ! FORALL n . NAT n IMPLIES P n           Induction: D1 D3

Support Transformation: (P4 ss) ==> (D1 ss) (D3 H2 ss)
\end{verbatim}

\begin{verbatim}
  Forward Conjunction Rule

 (P1)  H      !A(O)                                          
 (P2)  H      !B(O)                                          
*(P3)  H      !A(O) AND B(O)                               Conj: P1 P2

Support Transformation: (P3 ss) ==> (P1 ss) (P2 ss) 
\end{verbatim}


\begin{verbatim}
Backward Conjunction Rule

*(D1)  H      !A(O) AND B(O)                                          
 (D2)  H      !A(O)                                  Conj: D1
 (D3)  H      !B(O)                                  Conj: D1

Support Transformation: (pp D1 ss) ==> (pp D2 D3 ss) 
\end{verbatim}


\subsection{The LINE-NO-DEFAULTS functions}

There are two functions whose job it is to determine defaults for line
numbers.  The reason we need two functions is, that some of the lines
which appear on the left-hand side of the {\it support-transformation},
may reappear on the right.  The way we handle these connections, is
that we first determine the defaults for lines which are supposed
to exist (the left-hand side of the {\it support-transformation}), then
substitute those values into the right-hand side and call the second
default function.

The function \indexfunction{LINE-NO-DEFAULTS-FROM} is called with one argument
{\wt {line-no-defaults-from {\it default-exist}}} and \indexfunction{LINE-NO-DEFAULTS-TO}
is called with two arguments
{\wt {line-no-defaults-to {\it default-exist} {\it default-new}}}
where

\begin{description}
\item [{\it default-exist} ]  is the left hand side of the support transformation
specified for the rule, with lines that we need the default replaced
by a {\tt \$}, while the other lines are numbers (which means they either have
been figured out by an earlier default function or specified by the user).
Something which is neither {\tt \$} nor a number is one of the ``variables''
{\it d} or {\it p} standing for other {\it dlines} or {\it plines} in the current
{\it support} structure.  They must simply be returned (in the proper place,
of course).

\item [{\it default-new} ]  is the right hand side of the support transformation
specified for the rule, with the same interpretation as for {\it default-exist}.
\end{description}

The output of {\tt LINE-NO-DEFAULTS-FROM} should be a
{\tt {\it default-exists-figured}}, the output of {\tt LINE-NO-DEFAULTS-TO} is
a list {\wt ({\it default-exists-figured} {\it default-new-figured})} in
which all {\tt \$} of the arguments have been filled in.  These functions may
also do a {\tt THROWFAIL}, if one of the requirements R for logical
correctness cannot be satisfied in the given proof structure.

Also note that all lines in {\it default-exists} have already been determined,
when {\it default-new} is called.

The specification which must be meet by the {\tt LINE-NO-DEFAULTS-{\it x}}
functions can be grouped into three classes: requirements which ensure
the logical correctness of the rule application (R), requirements which
make the defaults sensible for the ``usual'' application of the rule (D)
and should never be deviated from, and desired properties, which need
not be satisfied, but approximate what the user would like to see most
of the time.

Note that the scope in this function is restricted by the fact that
it does not examine the logical structure assertions or hypotheses
of the lines in the proof.  This is accomplished by a completely different
mechanism and is not the responsibility of the function.  For instance,
it is perfectly sensible for {\tt LINE-NO-DEFAULTS} to suggest the first
{\it pline} in the current support structure for the backwards conjunction
rule, even though it may not be a conjunction at all! [This may cause mayhem
in rule tactics. The latter assumes that if there is a correct default,
the default function will choose it. Since a new line, properly located,
is always a correct, and possibly a useful default, tactics may miss
an opportunity to apply a rule.]

Subsequently, we will assume that
\begin{description}
\item [{\it default-exist} is ]  $((p_1\; d_{11} \ldots d_{1x_1})\ldots (p_p\; d_{p1}\ldots d_{px_p}))$

\item [{\it default-new} is ]  $((q_1\; e_{11} \ldots e_{1y_1})\ldots (q_q\; e_{q1}\ldots e_{qy_q}))$
\end{description}

\begin{center}
{\bf Requirements for Logical Correctness}
\end{center}

\begin{description}
\item [$R_{1}$ ]  $q_j < p_i$ for all $1\leq i \leq p$, $1\leq j \leq q$.

\item [$R_{2}$ ]  $e_{jk} < q_j$ for all $1\leq j \leq q$, $1\leq k \leq y_j$

\item [$R_{3}$ ]  $d_{ik} < p_i$ for all $1 \leq i \leq p$, $1\leq k \leq x_i$
\end{description}

\begin{center}
{\bf Sensible Defaults Requirements}
\end{center}

The requirements below only make sense, if the lines specified by the user
do not already violate them.  In that case, they must be relaxed to
apply only to the remaining unspecified lines.
\begin{description}
\item [$D_{1}$ ]  A {\it plan-support} pair suggested for an element of {\it default-exist}
must always match a {\it plan-support} pair in the current {\it support} structure
of the (incomplete) proof.

\item [$D_{2}$ ]  A {\it plan-support} pair suggested for an element of {\it default-new}
must consist of entirely new lines, and no two lines among all the suggested
defaults may have the same number.
\end{description}

\begin{center}
{\bf Wishful Thinking}
\end{center}

The following are constraints we would like to met, but is of course
not always possible.

\begin{description}
\item [$W_{1}$ ]  $q_j < q_{j+1}$ for all $1 \leq j < q$

\item [$W_{2}$ ]  $q_j < e_{j+1,k}$ for all $1 \leq j < q$ and 
$1 \leq j \leq q_{j+1}$

\item [$W_{3}$ ]  $d_{ik} < e_{jl}$ for all $1\leq i \leq p$,
$1\leq k \leq x_i$, $1\leq j \leq q$, $1\leq l \leq y_j$.

\item [$W_{4}$ ]  $\lnot\exists {\tt L}. \max{e_{jk}} < {\tt L} < q_j$ for
all $1 \leq j \leq q$.

\item [$W_{5}$ ]  Let $gap_{j} = q_j - \max{e_{jk}}$ for $1 \leq j \leq q$, if
$\{e_{jk} | 1 \leq k \leq y_j\}$ is non-empty, otherwise let $gap_{j} =
q_j - \max{\{{\tt L} | {\tt L} < q_j \lor \exists n\neq j . {\tt L} = q_n\}}$.
Then maximize $gap_{j}$, giving equal ``weight'' to all $1\leq j \leq q$.

\item [$W_{6}$ ]  Minimize $b = \min{e_{jl}} - \max{d_{ik}}$, with an alternative
similar to $W_{5}$ in case any of the sets is empty.

\item [$W_{7}$ ]  Minimize $t = \min{p_i} - \max{q_j}$, with an alternative similar
to $W_{5}$ in case any of the sets is empty.
\end{description}


\section{Updating the {\it support} structure}

Part of the execution of a rule application, is updating the plan
structure; this is one of the reasons why with every rule there 
comes a description of how the plan structure should be updated.
Below we will give a description of what the
function \indexfunction{UPDATE-PLAN} is supposed to accomplish, even in cases
when the rule is used in a way different from the defaults.  Again,
we assume the {\tt UPDATE-PLAN} is called as in 

{\tt (update-plan {\it default-exist} {\it default-new})} \\
where

\begin{description}
\item [{\it default-exist} is ]  $((p_1\; d_{11} \ldots d_{1x_1})\ldots (p_p\; d_{p1}\ldots d_{px_p}))$

\item [{\it default-new} is ]  $((q_1\; e_{11} \ldots e_{1y_1})\ldots (q_q\; e_{q1}\ldots e_{qy_q}))$
\end{description}

Recall that there may be variables appearing in place of a line number.
The following restrictions should be noted:

\begin{itemize}
\item For each {\it plan-support} pair
$(p_i\; d_{i1} \ldots  d_{ix_i})$ in {\it default-exist}, there is at most one
occurrence of a variable among $d_{i1},\ldots,d_{ix_i}$.

\item Any occurrence of a variable in {\it default-exist} is unique.
\end{itemize}

When {\tt UPDATE-PLAN} is called, all arguments are filled in, that is each
place is occupied either by a variable or a line number.

\subsection{{\it support} Structure Transformation in the Default Case}

If the rule is used completely in the default direction, i.e. all
{\it plan-support} pairs in {\it default-exist} exist in the current
{\it support} structure and all pairs in {\it default-new} consist of new lines,
then the effect of the rule application on the {\it support} structure is
straightforward:

\begin{itemize}
\item Delete all pairs matching $(p_i\; d_{i1}\ldots d_{ix_i})$ from the
{\it support} structure and attach to the front the pairs 
$(q_j\; e_{j1} \ldots e_{jy})$.

\item A variable in place of a $p_i$ matches {\bf any} {\it plan-support} pair in the
current proof, as long as the $d_{ik}$ match the corresponding support
lines.

\item A variable in place of a $d_{ik}$ matches the lines which are not matched by
any of the line numbers.  If $p_i$ is a variable, every match for $p_i$
produces a corresponding match of $d_{ik}$.

\item A variable in place of $q_j$ must occur as some $p_i$ and as many copies
of $(q_j\; e_{j1}\ldots e_{jy_j})$
are produced as there are matches of $p_i$.

\item A variable in place of $e_{jl}$ must occur as some $d_{ik}$ and the matched
list of lines in filled in.
\end{itemize}

\subsection{What if ...?}

We will go through all cases which differ from the default application
of the rule and specify what should happen to the {\it support} structure.
Of course, \TPS can not always correctly predict what the user had in
mind, when applying a rule, so the following must partly be considered
heuristics, but they will not always implement the user's devious
intentions.

\begin{enumerate}

\item \label{backtrack} % @tag(backtrack)
{\bf What if . . . } a $p_i$ exists, but is not a {\it pline?}  This case is
delicate and perhaps frequently occurs, if the user does not bother deleting
some lines before backtracking after some previous mistake.  Here execute
a {\tt PLAN-AGAIN} (which may become smart about support lines)\footnote{This is
a small project in itself!}.  This will make $p_i$ into a planned line and
we can handle it the usual way.

\item {\bf What if . . . } a $p_i$ does not exist?  Then, very likely, a rule meant to be used
backwards, was applied in a forward way.  We can't do much here: just ignore
the relevant part of {\it default-exist} completely.

\item {\bf What if . . . } a $p_i$ is a variable, but $d_{ik}$ don't match anything in the current
{\it support} structure.  This is already a special case of something discussed
in the previous section.

\item {\bf What if . . . } a $d_{ik}$ does not exist?  Then we must enter it as a planned line, collecting
$\{d_{il} | l \lneq k\}$ as its support lines.

\item {\bf What if . . . } a $d_{ik}$ does exist, but does not support $p_i$ ($p_i$ not a variable)?  Then
somehow $d_{ik}$ was improperly erased form the supports of $p_i$.
Just treat $d_{ik}$, as if it were supporting $p_i$.

\item {\bf What if . . . } a $q_j$ is a variable (thus exists as a {\it pline})
and matched a line number identical to a $e_{jk}$?  Then we are closing a
gap with a forward rule:  Do not enter the $j^{th}$ {\it plan-support} pair
into the {\it support} structure.

\item {\bf What if . . . } a $q_j$ already exists as a {\it pline}?  In this (probably very rare case)
we are reducing the proof of one planned line to the proof of another planned
line.  Add the $e_{jk}$ as additional support lines (also, of course, pulling it
to the front of the {\it support} structure).

\item {\bf What if . . . } a $q_j$ exists as a {\it dline}?  Here we already proved what we need,
so leave this {\it plan-support} pair out when constructing the new {\it support}
structure.

\item \label{dlineexist} % @tag(dlineexist)
{\bf What if . . . } a $e_{jk}$ exists as a {\it dline}?  Here we may be in a situation similar to
~\ref{backtrack}.  The justification of $e_{jk}$ will be changed according to
the current rule applied.  As far as the {\it support} structure is concerned,
we don't treat it specially.

\item {\bf What if . . . } a $e_{jk}$ exists as a {\it pline}?  Here we are justifying a planned line.
Delete the {\it plan-support} pair for $e_{jk}$ from the current {\it support}
structure.  The justification of $e_{jk}$ will be changed appropriately.

\item {\bf What if . . . } a $e_{jk}$ exists as a {\it hline}?  If {\tt TREAT-HLINES-AS-DLINES} is {\tt T}, do
what you would do to a {\it dlines} (see ~\ref{dlineexist}).  Otherwise, nothing
special is done.

\end{enumerate}

\subsection{Entering Lines into the Proof Outline}

The descriptions in the previous section can, when read carefully, also
serve as a guide to what should happen when entering a line into the
proof outline.  Of course, what should be done is clear, if we are in the
all-default case.  Otherwise we may have to change some justifications
as indicated in the previous section, but otherwise existing lines are left
alone.

Entering lines into the proof could be taken over by the same function, if
we handed it linelabels instead of line numbers in {\it default-exist} and
{\it default-new}.

\section{Defaults for Sets of Hypothesis}

In \tps, the user will rarely ever have to deal explicitly with sets of
hypothesis.  However the detail can be controlled by a flag
called \indexflag{AUTO-GENERATE-HYPS}.  If this flag is {\tt T}, \TPS will not only
generate smart defaults for sets of hypothesis, but make them strong
defaults, which means that the user will never be asked to specify
hypotheses for a line.

There some restrictions on what the user of the {\tt RULES} module may
specify as hypothesis in a rule.  Ignoring for the moment the problem of
fixed hypotheses, like sets representing axioms of extensionality of an
axiom of infinity, the hypotheses for each line $l$ may have the form
$H, s_{1}, \ldots , s_{n}$, where $H$ is a
meta-notation for a set of lines and the $s_i$ are labels for lines
present elsewhere in the rule specification.  Let us use $H_l$ for
this set of specified hypotheses for line $l$.

Note the restriction that there may be only one variable standing for
``arbitrary'' sets of lines in any single rule description.

Defaults {strong or not} for the hypotheses of lines are only calculated
after all line numbers have been specified.  This includes existent and
non-existent lines equally.  The algorithm below will always generate
legal applications of the rule, at the same time generating the ``correct''
set of hypotheses for each line.  The algorithm will almost always be
adequate, in the sense that the user will almost never need to explicitly
add hypotheses to a deduced line or drop hypotheses from a planned line.
There are cases, however, where this may still be necessary (see discussion
below).

\subsection{The Algorithm}

Here, unlike in other parts of the {\tt OUTLINE} modules, we do not need to
refer to the {\it support} structure.  Instead let us view the rule as if
we were to infer the {\it plines} from all the {\it dlines} specified in the rule,
and let us disregard hypothesis lines ({\it hlines}) for the moment.

For a given line {\it l} (in the rule specification) we now let $S_l$ stand
for the set of lines in the hypotheses which were explicitly specified
in the rule description (corresponds to $s_1,\ldots,s_n$
above) and let $L_l$ the actual list of hypotheses for the
line, which must either be matched or constructed (depending on whether
the line existed or not).  Furthermore let $H$ stand for the unique
name for an ``arbitrary'' set of lines which appears in zero or more of
the lines in the rule description.

Let us first consider the case that the hypotheses specified in the rule
description do not contain $H$.  For {\it dlines} {\it d} we must check

$L_d\subseteq S_d$

and for {\it plines} {\it p} we need to check

$S_p \subseteq L_p$

For {\it dlines} $d$ which contain $H$ among their hypotheses,
we must satisfy

$L_d \subseteq H \cup S_d$

and, if we are filling in hypotheses for a new line, we would like to choose
$L_d$ as large as possible, so it satisfies this equation.  From another point
of view, namely when we match existent lines, we find out some constraint on
$H$:

$L_d \setminus S_d \subseteq H$

On the other hand, for any given {\it pline} $p$, we obtain

$H \cup S_p \subseteq L_p$ 

or equivalently 

$H \subseteq L_p$ and $S_p \subseteq L_p$

Here we would like to make $L_p$ as small as possible (the fewer hypotheses
we used, the stronger the statement result).  Alternatively, the second line
can again be viewed as a constraint on $H$ when matching an existent
{\it pline}.

This leads to the following algorithm for determining set of hypothesis:

\begin{enumerate}
\item Let $D_{exist}$ be the set of {\it dlines} which exist in the current proof.
Then set 
$$H_{lower} = \bigcup \{L_d - S_d | d \in D_{exist} \land
H \in H_d\}.$$
  Also let $L_d$ be the strong default for the
hypotheses of line $d$ for each $d\in D_{exists}$.

\item \label{latermodified} % @tag(latermodified)
Let $P_{exist}$ be the set of {\it plines} which exist in the current proof.
Then set 
$$H_{upper} = \bigcap \{L_p |  p \in P_{exist} \land
H\in H_p\}.$$
Also let $L_p$ be the strong default for the
hypotheses of line $p$ for each $p\in P_{exists}$.

\item If {\it not}  $H_{lower} \subseteq H_{upper}$   the application of the inference
rule is illegal. (Do a {\tt THROWFAIL} with proper message.)

\item If both, $H_{lower}$ and $H_{upper}$ are undefined (empty intersection or union,
respectively), do not fill in any further defaults.

\item If exactly one of $H_{lower}$ and $H_{upper}$ is undefined, let $H_{lower} := H_{upper}$
or vice versa.

\item For non-existent {\it dlines} {\it d}, we let $L_d = H_{upper} \cup S_d$.
If {\tt AUTO-GENERATE-HYPS} is {\tt T}, make $L_d$ the strong default for that
argument, otherwise just a regular default.

\item For non-existent {\it plines} {\it p}, we let $L_p = H_{lower} \cup S_p$.
If {\tt AUTO-GENERATE-HYPS} is {\tt T}, make $L_d$ the strong default for that
argument, otherwise just a regular default.
\end{enumerate}

This algorithm is coded in a separate function for each rule.
For the rule {\it rule}, the function is called {\wt {\it rule}-HYP-DEFAULTS}
and is called (when appropriate) form within {\wt {\it rule}-DEFAULTS}.

\subsection{When the Algorithm is not Sufficient}

We must of course consider the case, when a restriction like
``x not free in $H$'' is imposed upon applications of the inference
rule.  Since we fill in $H_{upper}$ for the hypotheses of the {\it dlines} which
do not exist, we must check whether ``x not free in $H_{upper}$''.  It may
be the case, however, that all {\it dlines} actually already existed.  In this
case, it would be sufficient for the validity of the rule application, to check
whether ``x not free in $H_{lower}$''.  To see this may think of the rule as
first a legal application of the inference rule, leaving out the extra
hypotheses, then enlarging the set of hypotheses of the inferred line,
possibly with lines which contain ``x'' free.

This situation can also come up, when  not all the {\it dlines} are specified.
Then we may have been able to make the inference rule application legal,
by leaving out the lines {\tt H} from $H_{upper}$, which violate the condition
``x not free in (the assertion of) {\tt H}''.

This leads to a simple modification of the algorithm above, which would need
much more information about the rule (namely the restrictions), where
we modify the definition of $H_{upper}$ in step ~\ref{latermodified} by 

~\ref{latermodified}$^{*}$.
$H_{upper} = \bigcap \{L_p | p \in P_{exist} \land H\in H_p
\land L_p \;{\rm satisfies\; any\; restriction\; on\; } H\}$.

It seems more reasonable, however, not to place that restriction, but rather
give an error message.  Otherwise the user may only find out much later, that
some of the hypotheses he expected to be able to use, have not been included
in the {\it dlines}, since they violated a restriction.  This makes it necessary,
however, to give the user explicit rules which allow adding hypotheses to
a deduced line or dropping hypotheses from a planned line.

\subsection{Hypothesis Lines}

There are two principal ways hypothesis lines ({\it hlines}) can be treated
in \TPS and since there is very little extra work required, both are
provided for.  The flag \indexflag{TREAT-HLINES-AS-DLINES} controls how hypotheses
lines are handled.

If {\tt TREAT-HLINES-AS-DLINES} is {\tt T}, an {\it hline} may have more
hypotheses than simply {\it hline}.  Also, {\it hlines} may have descriptive
justifications like ``Case {a}'' or ``Ind. Hyp. for n''.  The price you
pay is that hypotheses lines become unique to a subproof and should not
be used elsewhere.  In this case, {\it hlines} are truly treated as
{\it dlines}, and in the above algorithm for determining default for
lines, we mean {\it dline} or {\it hline} whenever we say {\it dline}.

If {\tt TREAT-HLINES-AS-DLINES} is {\tt NIL}, every {\it hline} has exactly
one hypothesis: itself.  Also the justification for any {\it hline} will
be the same, namely the value of the flag \indexflag{HLINE-JUSTIFICATION}
(by default {\tt Hyp}).  What you gain in this case is, that the same
hypothesis line may used many different places in the given proof.  The
default for the hypotheses of an {\it hline} will always be strong and
equal to {\tt ({\it hline})}, anything else will result in an error, even if
perhaps logically correct. Also, in this case, 
if \indexflag{CLEANUP-SAME} is {\tt T}, then \indexcommand{CLEANUP} will
eliminate unnecessary hypotheses.


