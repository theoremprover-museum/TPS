\chapter{Tactics and Tacticals}

% I have no idea who wanted to modify this file, or why. - cebrown 2/3/00
{\bf Modify tactics.tex}  

\section{Overview}

Ordinarily in \tps, the user proceeds by performing a series of
atomic actions, each one specified directly.  For example, in constructing
a proof, she may first apply the deduct rule, then the rule of cases, then
the deduct rule again, etc..  These actions are related temporally, but
not necessarily in any other way; the goal which is attacked by 
one action may result in several new goals, yet there is no distinction
between goals produced by one action and those produced by another.  
In addition, this use of small steps prohibits the user from outlining
a general procedure
to be followed.  A complex strategy cannot be expressed in these terms, 
and thus
the user must resign herself to proceeding by plodding along, using simple
(often trivial and tedious) applications of rules.

Tactics offer a way to encode strategies into new commands, using a 
goal-oriented approach.  With the use of tacticals, more complex tactics
(and hence strategies) may be built.  Tactics and tacticals are, in essence,
a programming language in which one may specify techniques for solving
goals.

Tactics are called  partial subgoaling methods by
\cite{Gordon79}.  What this means is that a tactic is a
function which, given a goal to be accomplished,
will return a list of new goals, along with a procedure by which the
original goal can be achieved given that the new goals are first
achieved.  Tactics also may fail, that is, they may not be applicable
to the goal with which they are invoked.

Tacticals operate upon tactics in much the same way that functionals
operate upon functions.  By the use of tacticals, one may create
a tactic that repeatedly carries out a single tactic, or composes
two or more tactics.  This allows one to combine many small tactics
into a large tactic which represents a general strategy for solving goals.

As implemented in \tps, a tactic is a function which takes a goal
as an argument and returns four values: a list of new goals, a message
which tells what the tactic did (or didn't do), a token indicating what
the tactic did, and a validation, which is a lambda expression which takes
as many arguments as the number of new goals, and which, given solutions
for the new goals, combines the solutions into a solution for the original
goal. It is possible that the validation is used nowhere in the code, and
that it should be phased out.

Consider this example.  Suppose we are trying to define tactics which
will convert an arithmetic expression in infix form to one in
prefix form and evaluate it.  One tactic might,
if given a goal of the form "A / B", where A and B are themselves 
arithmetic expressions in infix form, return the list ("A" "B"),
some message, the token "succeed", and the 
validation {\tt (lambda (x y) (/ x y))}.  If now we solve the new goals "A"
and "B" (i.e., find their prefix forms and evaluate them),
and apply the validation as a function to their solutions, we get
a solution to the original goal "A / B".

When we use a tactic, we must know for what purpose the tactic is being
invoked.  We call this purpose the {\it use} of the tactic.  Some examples
of uses are {\tt nat-ded} for carrying out natural deduction proofs, {\tt nat-etree}
for translating natural deduction proofs to expansion proofs (not yet implemented), and {\tt etree-nat}
for translating expansion proofs to natural deductions.  A single tactic may
have definitions for each of these uses.  In contrast to tactics, tacticals
are defined independent of any specific tactic use; some of the auxiliary
functions they use, however, such as copying the current goal, may depend
upon the current tactic use.  For this purpose, the current tactic use
is determined by the flag \indexflag{tacuse}.  Resetting this flag resets
the default tactic use. Though a tactic can be called
with only a single use, that tactic can call other tactics with different
uses.  See the examples in the section "Using Tactics".

Another important parameter used by a tactic is the {\it mode}.  There are
two tactic modes, {\tt auto} and {\tt interactive}.  The definition of
a tactic may make a distinction between these two modes; the current
mode is determined by the flag  \index{tacmode}, and resetting this
flag resets the default tactic mode.  Ideally, a tactic
operating in {\tt auto} mode should require no input from the user, while
a tactic in {\tt interactive} mode may request that the user make some decisions,
e.g., that the tactic actually be carried out.  It may be desirable, however,
that some tactics ignore the mode, compound tactics (those tactics created
by the use of tacticals and other tactics) among them.

One may wish to have tactics print informative messages as they operate;
the flag \indexflag{tactic-verbose} can be set to T to allow this to
occur, and tactics can be defined so that messages are printed when
{\tt tactic-verbose} is so set. Each tactic should call the function {\it tactic-output} 
with two arguments. The first argument should be a string containing the information to be
printed, and the second argument T if the tactic succeeds, and NIL
otherwise.  {\it tactic-output} will, depending on the second argument and
the current value of {\tt tactic-verbose}, either print or not print the
first argument.

\section{Syntax for Tactics and Tacticals}

The \TPS category for tactics is called {\tt tactic}.  The defining 
function for tactics is {\tt deftactic}. The variable {\tt auto::*global-tacticlist*}
contains a list of all tactics. Each tactic definition has
the following form:

%\begin{programexample}
\begin{verbatim}
(deftactic tactic
  {(<tactic-use> <tactic-defn> [<help-string>])}+
)
\end{verbatim}
%\end{programexample}

with components defined below:

% \begin{Format}
% @tabclear
% @tabset(1.5 inches)
\begin{tabular}{ll}
\indexSyntax{tactic-use} ::= & {\tt nat-ded | nat-etree | etree-nat} \\
\indexSyntax{tactic-mode} ::= & {\tt auto | interactive} \\
\indexSyntax{tactic-defn} ::= & {\it primitive-tactic} | {\it compound-tactic} \\
\indexSyntax{primitive-tactic} ::= & {\tt (lambda (goal) {{\it form}}*)} \\
 & This lambda expression should return four values of the form:  \\
 & {\it goal-list msg token validation}. \\
\indexSyntax{compound-tactic}::= & ({\it tactical} {{\it tactic-exp}}*) \\
\indexSyntax{tactic-exp}::= & {\tt tactic}  a tactic which is already defined \\
 & | ({\it tactic} {\tt [:use {\it tactic-use}] [:mode {\it tactic-mode}] [:goal {\it goal}]}) \\
 & | {\it compound-tactic} \\
 & | ({\tt call} {\it command}) ; where {\it command} is a command which could be \\
 & given at the \TPS top level \\
\indexSyntax{goal} ::= & a goal, which depends on the tactic's use,  \\
 & e.g., a planned line when the tactic-use is {\tt nat-ded}. \\
\indexSyntax{goal-list} ::= & ({{\it goal}}*) \\
\indexSyntax{msg} ::= & {\it string} \\
\indexSyntax{token} ::= & {\tt complete}  meaning that all goals have been exhausted \\
 & {\tt | succeed}  meaning that the tactic has succeeded \\
 & {\tt | nil}  meaning that the tactic was called only for side effect \\
 & {\tt | fail}  meaning that the tactic was not applicable  \\
 & {\tt | abort}  meaning that something has gone wrong, such as an undefined  \\
 & tactic \\
\end{tabular}
% \end{format}

Tacticals are kept in the \TPS category {\tt tactical}, with defining
function {\tt deftactical}.  Their definition has the following form:

%\begin{programexample}
\begin{verbatim}
(deftactical tactical
  (defn <tacl-defn>)
  (mhelp <string>))
\end{verbatim}
%\end{programexample}

with 

% \begin{format}@tabclear
% @tabset(1.5 inches)
\begin{tabular}{ll}
\indexSyntax{tacl-defn} ::= & {\it primitive-tacl-defn | compound-tacl-defn} \\
\indexSyntax{primitive-tacl-defn} ::= & {\tt (lambda (goal tac-list) {{\it form}}*)} \\
 & This lambda-expression, where {\tt tac-list} stands for a possibly \\
 & empty list of tactic-exp's, should be independent of the tactic's  \\
 & use and current mode.  It should return values like those returned  \\
 & by a {\it primitive-tac-defn}. \\
\indexSyntax{compound-tacl-defn} ::= & {\tt (tac-lambda ({{\it symbol}}*) {\it tactic-exp})}  \\
 & Here the tactic-exp should use the symbols in the \\
 & tac-lambda-list as dummy variables. \\
\end{tabular}
%\end{format}

\pagebreak
Here is an example of a definition of a primitive tactic.
%\begin{programexample}
\begin{verbatim}
(deftactic finished-p
 (nat-ded 
  (lambda (goal)
    (if (proof-plans dproof)
	(progn
	 (when tactic-verbose (msgf "Proof not complete." t))
	 (values nil "Proof not complete." 'fail))
	(progn
	 (when tactic-verbose (msgf "Proof complete." t))
	 (values nil "Proof complete." 'succeed))))
  "Returns success if all goals have been met, otherwise
returns failure."))
\end{verbatim}
%\end{programexample}

This tactic is defined for just one use, namely {\tt nat-ded}, or natural
deduction.  It merely checks to see whether there are any planned lines
in the current proof, returning failure if any remain, otherwise
returning success.  This tactic is used only as a predicate, so the
goal-list it returns is nil, as is the validation.

As an example of a compound tactic, we have
%\begin{programexample}
\begin{verbatim}
(deftactic make-nice
  (nat-ded
   (sequence (call cleanup) (call squeeze) (call pall))
   "Calls commands to clean up the proof, squeeze the line 
numbers, and then print the result."))
\end{verbatim}
%\end{programexample}

Again, this tactic is defined only for the use {\tt nat-ded}.  {\tt sequence} is
a tactical which calls the tactic expressions given it as arguments
in succession.

Here is an example of a primitive tactical.
%\begin{programexample}
\begin{verbatim}
(deftactical idtac
  (defn
    (lambda (goal tac-list)
      (values (if goal (list goal)) "IDTAC" 'succeed 
	      '(lambda (x) x))))
  (mhelp "Tactical which always succeeds, returns its goal 
unchanged."))
\end{verbatim}
%\end{programexample}

The following is an example of a compound tactical.  {\tt then} and {\tt orelse} are tacticals.

%\begin{programexample}
\begin{verbatim}
(deftactical then*
  (defn 
    (tac-lambda (tac1 tac2)
      (then tac1 (then (orelse tac2 (idtac)) (idtac)))))
  (mhelp "(THEN* tactic1 tactic2) will first apply tactic1; if it
fails then failure is returned, otherwise tactic2 is applied to 
each resulting goal.  If tactic2 fails on any of these goals, 
then the new goals obtained as a result of applying tactic1 are 
returned, otherwise the new goals obtained as the result of 
applying both tactic1 and tactic2 are returned."))
\end{verbatim}
%\end{programexample}

\section{Tacticals}
There are several tacticals available.  Many of them are taken directly from
\cite{Gordon79}.  After the name of each tactical is given
an example of how it is used, followed by a description of the behavior
of the tactical
when called with {\tt goal} as its goal.  The newgoals and validation returned
are described only when the tactical succeeds.


\begin{enumerate}
\item {\tt idtac: (idtac)} \\
Returns {\tt (goal), (lambda (x) x)}.

\item {\tt failtac: (failtac)}\\
Returns failure

\item {\tt call: (call command)}\\
Executes command as if it were entered at top level of \tps.  This is used
only for side-effects.  Returns {\tt (goal), (lambda (x) x)}.

\item {\tt orelse: (orelse tactic1 tactic2 ... tacticN)}\\
If N=0 return failure, else apply {\tt tactic1} to {\tt goal}. If this fails, call {\tt (orelse tactic2 tactic3 ... tacticN)} on {\tt goal}, else return the result of applying {\tt tactic1} to {\tt goal}.\\

\item {\tt then: (then tactic1 tactic2)}\\
Apply {\tt tactic1} to {\tt goal}. If this fails, return failure, else apply {\tt tactic2} to each of the subgoals generated by {\tt tactic1}.\\
If this fails on any subgoal, return failure, else return the list of new subgoals returned from the calls to {\tt tactic2}, and the lambda-expression representing the combination of applying {\tt tactic1} followed by {\tt tactic2}.\\
Note that if {\tt tactic1} returns no subgoals, {\tt tactic2} will not be called.

\item {\tt repeat: (repeat tactic)}\\
Behaves like {\tt (orelse (then tactic (repeat tactic)) (idtac))}.

\item {\tt then*: (then* tactic1 tactic2)}\\
Defined by:\\
{\tt (then tactic1 (then (orelse tactic2 (idtac)) (idtac)))}.  This tactical is taken from \cite{Felty86}.

\item {\tt then**: (then** tactic1 tactic2)}\\
Acts like {\tt then}, except that no copying of the goal or related structures will be done. 

\item {\tt ifthen: (ifthen test tactic1)} or \\
        {\tt (ifthen test tactic1 tactic2)}\\
First evaluates {\tt test}, which may be either a tactic or (if user is an expert) an arbitrary LISP expression.  If test is a tactic and does not fail, or is an arbitrary LISP expression that does not evaluate to nil, then {\tt tactic1} will be called on {\tt goal} and its results returned. Otherwise, if {\tt tactic2} is present, the results of calling {\tt tactic2} on {\tt goal} will be returned, else failure is returned.  {\tt test} should be some kind of predicate; any new subgoals it returns will be ignored by {\tt ifthen}.

\item {\tt sequence: (sequence tactic1 tactic2 ... tacticN)}\\
Applies {\tt tactic1, ... , tacticN} in succession regardless of their success or failure.  Their results are composed.

\item {\tt compose: (compose tactic1 ... tacticN)}\\
Applies {\tt tactic1, ..., tacticN} in succession, composing their results until one of them fails.  Defined by:\\
{\tt (idtac)} if {\tt N}=0\\
{\tt (then* tactic1 (compose tactic2 ... tacticN))} if {\tt N} > 0.

\item {\tt try: (try tactic)}\\
Defined by: {\tt (then tactic (failtac))}.  Succeeds only if tactic returns no new subgoals, in which case it returns the results from applying {\tt tactic}. 

\item {\tt no-goal: (no-goal)}\\
Succeeds iff goal is nil.
\end{enumerate}


\section{Using Tactics}

To use a tactic from the top level, the command \indexmexpr{use-tactic} has
been defined.  Use-tactic takes three arguments: a {\it tactic-exp}, a 
{\it tactic-use},
and a {\it tactic-mode}.  The last two arguments default to the values of
\indexflag{tacuse} and \indexflag{tacmode}, respectively.
Remember that a {\it tactic-exp} can be either the name of
a tactic or a compound tactic.  Here are some examples:

%\begin{group}
\begin{verbatim}
<1> use-tactic propositional nat-ded auto

<2> use-tactic (repeat (orelse same-tac deduct-tac)) 
               $ interactive

<3> use-tactic (sequence (call pall) (call cleanup) (call pall)) !

<4> use-tactic (sequence (foo :use nat-etree :mode auto) 
                         (bar :use nat-ded :mode interactive)) !
\end{verbatim}
%\end{group}
Note that in the fourth example, the default use and mode are overridden
by the keyword specifications in the tactic-exp itself.  Thus during the
execution of this compound tactic, {\tt foo} will be called for one use and
in one mode, then {\tt bar} will be called with a different use and mode.

Remember, setting the value of the flag \indexflag{tactic-verbose} to T will
cause the tactics to send informative messages as they execute.




\subsection{Implementation of tactics and tacticals}

\begin{enumerate}
\item Main files (in order of importance): tactics-macros, tacticals,
tacticals-macros, tactics-aux.  These files contain tactic-related functions 
of a general nature.  Most tactics are actually contained in other Lisp packages.

\item When a tactic is executed, two global variables affect its
execution: tacuse (the tactic's use), and tacmode (the current mode).
Tacuse determines for what reason the tactic is being called.  Current
uses are etree-nat (translation of \indexother{eproof} to natural deduction) and
nat-ded (construction of a natural deduction proof without any mating
information).  A single tactic may be defined for more than one use.
Tacmode can have the value of either auto or interactive.  Each tactic
should take this value into account during operation.  In general,
this means that when the value is interactive, the user should be
advised that the tactic is about to be applied and should be allowed
to abort it.  When the value is auto, the tactic should just be
carried out if applicable.  

\item For each use, a number of auxiliary functions needed by the
tacticals must be defined.

\begin{enumerate}
\item get-tac-goal: if a goal has not been specified, get the next one,
e.g., the active planned line.

\item copy-tac-goal: copy the current goal into a new goal, so that subsequent
actions can be performed without destroying the current goal.  Allows
later backtracking if necessary.

\item save-tac-goal: put the current goal into a form suitable for
saving.  This is not actually used by current tactics.

\item restore-tac-goal: backtrack to the previous goal. 
This is not actually used by current tactics.

\item update-tac-goal: given the old (saved) goal, and the new goal on which
some progress has been made, update the old goal to reflect the
progress made.
\end{enumerate}

Tacticals must be independent of the value of tacuse.  
They cannot make any assumptions
about the structure of the goals, etc.

The main function used in applying tactics is apply-tactic.  This
is a function that takes a tactic as argument, and allows keyword
arguments of :goal, :use and :mode.  If not specified, the use and
mode default to the global values of tacuse and tacmode.
If they are specified,
the values given then override the global values of tacuse and tacmode.
apply-tactic and (every tactic) returns four values.  
 The first is a
list of goals, the second a string with some kind of message, the
third a token which indicates the result of the tactic and the fourth
a validation, which, if non-nil, should be a function which specifies
how solutions to the returned goals can be combined to solve the original goal.

apply-tactic works as follows:

\begin{enumerate}
\item  Checks that tactic is a valid tactic.

\item If a goal has not been specified, calls get-tac-goal.

\item If the tactic is an atom:
  \begin{enumerate}
\item     gets the tactic's definition for the use.

\item     if the definition is primitive, and the goal is nil, return
(nil "Goals exhausted." 'complete nil).  If the goal is non-nil, apply the 
tactic to the goal.

\item     if the definition is compound, call apply-tactical on the
definition and the goal.
  \end{enumerate}

\item  If the tactic's definition begins with a tactic, call apply-tactic
recursively,  using those optional arguments.

\item  If the tactic begins with a tactical, call apply-tactical.
\end{enumerate}

 Whenever a tactic begins with a tactical, the function
apply-tactical is used.  It takes two arguments, a goal and a tactic.
It is assumed that the tactic begins with a tactical.  
apply-tactical works as follows:


\begin{enumerate}
\item Get the definition for the tactical.

\item If the definition is primitive (a lambda expression), funcall the
definition on the goal and the remainder (cdr) of the tactic. 

\item If the definition is compound (i.e., is defined in terms of other
tacticals and begins with tac-lambda), expand the definition,
substituting the arguments provided in the tactic's definition for the
dummy  arguments in the tactical's definition.  Then call
apply-tactical recursively.

\item Otherwise abort, returning abort as the token (third value
returned).
\end{enumerate}

\item  Validations:  Though the validation mechanism is in place, no use
is made of them in the current tactic uses, since any changes in the
constructed proofs are made immediately, not saved.  Validations must
be modified as tacticals are executed, since during their execution,
the order of goals may be changed.  For example, a tactical may
repeatedly apply a tactic to a goal, then to all the new goals
created, etc., until it fails on all of them.  When it succeeds on a
successor goal, the validation returned must be integrated into the validation
which was returned for the first application of the tactic on the
original goal.  The function make-validation is used for this purpose.
\end{enumerate}

