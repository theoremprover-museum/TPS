\subsection{Refining the Output: the Monitor}

The monitor is intended to help users to obtain more useful output from the various mating-search procedures.
It acts in a way like the flag \indexflag{QUERY-USER}, which also allows the user to interact with
the search procedure (see the help flag for more information).

There are three monitor commands available, these are \indexmexpr{MONITOR}, \indexmexpr{NOMONITOR}
and \indexmexpr{MONITORLIST}. The first two turn the monitor on and off, respectively (this can also
be done by setting the flag \indexflag{MONITORFLAG}), and also
print out the details of the current monitor function. The latter lists all the available monitor functions.

As well as typing {\tt MONITOR} to turn the monitor on, the user must also type the name of a monitor
function. This may prompt for more input, and will then store the responses until they are needed.

If the monitor is switched on, then at intervals throughout the mating search  the current monitor
function will be called. This function might, for example, check to see if a particular mating has been
reached, and if it has then change the settings of some of the flags (so one could start a long search with
MATING-VERBOSE set to MIN and UNIFY-VERBOSE set to NIL, and switch these to MAX and T when the desired mating
appeared).

This is made much clearer by the following example:

\begin{tpsexample}
<51>mode mode-x5305
<52>mate x5305
<Mate53>ms88
{\it several pages of output, at the end of which we have a complete mating}
<Mate54>show-mating
Active mating: (LEAF58 . LEAF40)  (LEAF57 . LEAF55)  (LEAF51 . LEAF49)
(LEAF36 . LEAF38) is complete.
<Mate55>mating-verbose silent
<Mate56>unify-verbose nil
<Mate57>monitorlist
PUSH-MATING FOCUS-OTREE* FOCUS-OTREE FOCUS-OSET* FOCUS-OSET FOCUS-MATING
FOCUS-MATING* MONITOR-CHECK

To change the current monitor function to one of these functions,
just type the name of the required monitor function as a command from either
the main top level or the mate top level.

<Mate58>help focus-mating*
FOCUS-MATING* is a monitor function.
Reset some flags when a particular mating is reached. Differs
from FOCUS-MATING in that it returns the flags to their original
settings afterwards. The default mating is the mating that
is current at the time when this command is invoked (so the user
can often enter the mate top level, construct the mating manually and
then type FOCUS-MATING*). Otherwise, the mating should be typed in the form
((LEAFa . LEAFb) (LEAFc . LEAFd) ...)
The values used for the `original' flag settings will also
be those that are current at the time when this command is invoked.
The command format for FOCUS-MATING* is:

<n>FOCUS-MATING*       MATING         FLAGLIST     VALUELIST
                  `MATINGPAIRLIST' `TPSFLAGLIST' `SYMBOLLIST'

<Mate59>focus-mating*
MATING (MATINGPAIRLIST): Mating to watch for [((LEAF36 . LEAF38) (LEAF51 . LEAF49)
(LEAF57 . LEAF55) (LEAF58 . LEAF40))]>
FLAGLIST (TPSFLAGLIST): Flags to change [No Default]>mating-verbose unify-verbose
VALUELIST (SYMBOLLIST): New values for these flags [No Default]>max t
I can't find a leaf called LEAF36, but I'll continue anyway.
I can't find a leaf called LEAF38, but I'll continue anyway.
I can't find a leaf called LEAF51, but I'll continue anyway.
I can't find a leaf called LEAF49, but I'll continue anyway.
I can't find a leaf called LEAF57, but I'll continue anyway.
I can't find a leaf called LEAF55, but I'll continue anyway.
I can't find a leaf called LEAF58, but I'll continue anyway.
I can't find a leaf called LEAF40, but I'll continue anyway.
{\it A lot has happened here. Notice that the current active-mating appeared as the default list of mating pairs;
we could type in any other list instead. Note also that (because we have completed the mating) none of the
leaves can be found. This doesn't matter; we are allowed to specify non-existent leaves, so as to allow for
leaves that are generated during mating search. If the search will be with path-focused duplication, the
matingpairlist should not specify which copy of which literal is to be used (i.e. it should be in exactly the
same format as the above). Also note that if the connections were specified in a different order, or the
order of the literals in a pair were reversed, this would make no difference at all to the monitor function.}
<Mate60>leave
<64>mate x5305
{\it We leave and re-enter, and will try to cut down the output this time around}
<Mate65>monitor
The monitor is now ON.
The current monitor function is FOCUS-MATING*
When the current mating is : ((LEAF36 . LEAF38) (LEAF51 . LEAF49) (LEAF57 . LEAF55) (LEAF58 . LEAF40))
The following flags will be changed to the given new values :
(UNIFY-VERBOSE . T)(MATING-VERBOSE . MAX)
and afterwards, they will be changed back to the values :
(UNIFY-VERBOSE . NIL)(MATING-VERBOSE . SILENT)

<Mate66>ms88
{\it less than half a page of output, all relating to the above mating}
\end{tpsexample}

