\chapter{TPS Structures}

Notice that \TPS has a command \indexmexpr{TLIST} which outputs the
same information as the Lisp command {\it plist}, but formatted more
readably. So, for example, {\tt TLIST X2108} will show all of the slots 
in the structure X2108 (which is a proof).
 
\section{TPS Modules}

See the introductory chapter for a discussion of what \TPS modules are.

\subsection{The Tps3 Module Structure}

All modules are defined in one central file, called {\wt DEFPCK}.
You may want to look at this file to see examples of module definitions
and also a current list of all module known to \tps.

There is a partial order of
modules in \tps.  One whole set of modules called {\tt BARE} is
distinguished from the others.  All files in the module {\tt BARE} and all
of its submodules must always be present in a {\tt TPS3} core image.

When \TPS is built from {\bf Lisp}, some of the files in the {\tt BARE}
module can not be loaded with a module-loading command, since
it has not been defined.  Thus, even though every file for \TPS belongs
to a proper module, not all modules are loaded the same way because of
the ``bootstrapping'' problem.

Another quirk should be mentioned here.  A module called {\tt WFFS}
defines the basic operations of wffs.  The modules {\tt WFF-PRINT}
and {\tt WFF-PARSE} depend on {\tt WFFS.}  The module {\tt WFFS,} however,
cannot exist alone: the modules {\tt WFF-PRINT} and {\tt WFF-PARSE} must
be present also, even though this fact can not be deduced from the
module structure.


\subsection{Defining a New Module}

To define a new module for \tps, use the {\tt DEFMODULE} macro.
Its format is

\begin{verbatim}
(defmodule {\it name}
  (needed-modules {\it module} {\it module} ...)
  (macro-files {\it mfile} {\it mfile} ...)
  (files {\it file} {\it file} ...)
  (mhelp "{\it help-string}"))
\end{verbatim}

\begin{description}
\item [{\tt needed-modules}] These are all modules that must be loaded for the
module {\it name} to work.  Because of the transitive structure of modules
only the direct predecessors of the new module need to be listed.

\item [{\tt macro-files}] These are the files the compiler needs, before it can
compile any of the files in the module.  It is generally a good idea
to make a file with all the macro definitions (e.g. argument types,
flavors of labels, etc.) and separate it from the functions, commands,
etc. in the module.  This means clearer program structure, but also
minimal overhead for the compiler.

\item [{\tt files}] These are the rest of the files in the module.  When the module
is loaded, first the {\tt macro-files} are loaded, then the {\tt files.}
\end{description}

The new module should also be added into {\it defpck.lisp} at an appropriate point, and 
should be added into whichever of {\it tps-build.lisp}, {\it tps-compile.lisp},
{\it etps-build.lisp} and {\it etps-compile.lisp} are appropriate (these files are in the
same directory as the {\it Makefile}, not the main TPS directory).

\section{Categories}
\label{categories}

\TPS categories are in a sense data types. A category is a way to characterize a set of similar
objects which have properties of the same types, use the same auxiliary functions, are acted on 
by the same functions, etc.

Categories are orthogonal to the package/module structure, i.e. a category may have members which are 
defined in many different packages and modules. Categories group objects by functionality 
(how they behave) whereas packages and modules group objects by purpose (why they exist).

Categories are defined using the \indexother{defcategory} macro. For example, the 
definition of the category of \TPS top levels is:
%\begin{lispcode}
\begin{verbatim}
(defcategory toplevel
  (define deftoplevel)
  (properties
   (top-prompt-fn singlefn)
   (command-interpreter singlefn)
   (print-* singlefn)
   (top-level-category singlefn)
   (top-level-ctree singlefn)
   (top-cmd-interpret multiplefns)
   (top-cmd-decode singlefn)
   (mhelp single))
  (global-list global-toplevellist)
  (mhelp-line "top level")
  (mhelp-fn princ-mhelp))
\end{verbatim}
%\end{lispcode}

This shows a category whose individual members are defined with the {\tt deftoplevel} command, and
whose properties include the prompting function, a command interpreter, and so on. There is a 
global list called {\tt global-toplevellist} which will contain a list of all of the top levels 
defined, and an mhelp line "top level" (so that when you type {\tt HELP MATE}, \TPS knows to respond
"MATE is a top level".) The mhelp-fn is the function that will be used to print the help messages
for all the objects in this category. (See chapter ~\ref{help} for more information.)

The chapters of the facilities guide correspond to categories.
Within each chapter, the sections correspond to contexts.
In \tps, \indexother{global-categorylist} contains a list of all the
currently defined categories.

\section{Contexts}

Contexts are used to provide better help messages for the user. Each context is used to partition
the objects in a category into groups with similar tasks. For example, the objects in the 
category {\tt MEXPR} are grouped into contexts such as {\tt PRINTING} and {\tt EQUALITY RULES}.
(Contexts are themselves a category, of course: the definition is in {\it boot0.lisp}.)

New contexts are defined with the \indexother{defcontext} command, and are invoked with the 
single line {\tt (context {\it whatever}}) in the code (all this does is to set a variable 
{\tt current-context} to {\it whatever}). 

Here is a sample use of {\tt defcontext}:
%\begin{lispcode}
\begin{verbatim}
(defcontext tactics
  (short-id "Tactics")
  (order 61.92)
  (mhelp "Tactics and related functions."))
\end{verbatim}
%\end{lispcode}

The only property which is not immediately self-explanatory is {\tt order}; this is used to sort
the contexts into order before displaying them on the screen (or in manuals).

Contexts are used in the facilities guide (for example) to divide
chapters into sections. For example, the line
{\tt (context unification)}
occurs prior to the definition 
{\tt (defflag max-utree-depth ...)}
of the flag MAX-UTREE-DEPTH in the file {\it node.lisp},
and so this flag occurs in the section on unification in the
chapter on flags in the facilities guide.

To see the contexts into which the commands for a given top-level
are divided, just use the ? command at that top-level.
Look at \indexother{global-contextlist} in \TPS to see all the contexts.

\section{Flavors}
\label{flavors}

Some TPS structures (in particular, all expansion tree nodes, expansion variables, skolem terms and jforms) 
are defined as \indexother{flavors}; see the file {\it flavoring.lisp} for 
the details. These structures have many attached properties which allow 
wffops to be used on them as though they were gwffs; for example, the flavor
{\tt exp-var} in {\it etrees-exp-vars.lisp} has the properties
%\begin{tpsexample}
\begin{verbatim}
  (type (lambda (gwff) (type (exp-var-var gwff))))
  (gwff-p (lambda (gwff) (declare (ignore gwff)) T))
\end{verbatim}
%\end{tpsexample}
which state that the type of an {\tt exp-var} structure is the type of its variable, and all {\tt exp-var}s 
are gwffs. Errors of the form "Wff operation <wffop> cannot be applied to labels of flavor <label>" are almost always
caused by attempting to use a wffop on a flavor for which the corresponding property is undefined; for example,
if we deleted the lines above and recompiled TPS, any attempt to find the type of an expansion variable
would result in the error "Wff operation TYPE cannot be applied to labels of flavor EXP-VAR".

Flavors that are defined within TPS will also have the slot {\tt \indexother{bogus-slot}}; this slot is 
tested for by TPS to confirm that the flavor was defined by TPS, but the contents of this slot are never 
examined. This means that there is always one empty slot in each node of an expansion tree or jform which the
programmer can use to store information while a program is being tested (whereas if you define a new slot, you have to 
recompile all instances of a structure, which can be a nuisance). Obviously, once the new code is working,
you should define a new slot, change all references to bogus-slot and recompile TPS!

For examples of flavors of gwffs, see page \pageref{labels}.