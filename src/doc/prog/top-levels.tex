\chapter{Top-Levels}\label{toplev}

\section{Defining a Top Level}

Top levels are a \TPS category, whose definition is given in section ~\ref{categories}.
For an example, let's look at the editor top level:

%\begin{lispcode}
\begin{verbatim}
(deftoplevel ed-top
  (top-prompt-fn ed-top-prompt)
  (command-interpreter ed-command-interpreter)
  (print-* ed-print-*)
  (top-level-category edop)
  (top-level-ctree ed-command-ctree)
  (top-cmd-decode opdecode)
  (mhelp "The top level of the formula editor."))
\end{verbatim}
%\end{lispcode}

This says that the top level {\tt ed-top} identifies itself by the function {\tt ed-top-prompt},
which is one of the more complicated prompt functions in \TPS; its only purpose is to 
print the {\tt <ed34>} messages at the start of each line in the editor, but the complications
are necessary because the editor can be entered recursively.

The next line of the toplevel definition gives the name of the command interpreter function.
The {\tt print-*} function is a function that gets called after every line; in this case, it's the
{\tt ed-print-*} function, which prints out the current wff if it has changed due to the last command.
The top level category is {\tt edop}, which is defined as follows:

%\begin{lispcode}
\begin{verbatim}
(defcategory edop
  (define defedop)
  (properties
   (alias single)
   (result-> singlefn)
   (edwff-argname single)
   (defaultfns multiplefns)
   (move-fn singlefn)
   (mhelp single))
  (global-list global-edoplist)
  (shadow t)
  (mhelp-line "editor command")
  (scribe-one-fn
    (lambda (item)
      (maint::scribe-doc-command 
       (format nil "@IndexEdop(~A)" (symbol-name item))
       (remove (get item 'edwff-argname) 
	       (get (get item 'alias) 'argnames))
       (or (cdr (assoc 'edop (get item 'mhelp)))
	   (cdr (assoc 'wffop (get (get item 'alias) 'mhelp)))))))
  (mhelp-fn edop-mhelp)))
\end{verbatim}
%\end{lispcode}

This category defines the sort of command found in the editor top level (compare 
the above definition with that of {\tt mexpr}, for example). So all the commands 
that can only be seen from the editor top level are defined with the {\tt defedop} 
command, as follows:

%\begin{lispcode}
\begin{verbatim}
(defedop o
  (alias invert-printedtflag)
  (mhelp "Invert PRINTEDTFLAG, that is switch automatic recording of wffs
in a file either on or off.  When switching on, the current wff will be
written to the PRINTEDTFILE. Notice that the resulting file will be in 
Scribe format; if you want something you can reload into TPS, then use
the SAVE command."))
\end{verbatim}
%\end{lispcode}

The {\tt top-command-ctree} is used for command completion, and the {\tt mhelp} 
property is obvious. This leaves {\tt top-cmd-decode}, which is the name of the
function that is called by the command interpreter to, for example, fill in the 
default arguments for an edop.

\section{Command Interpreters}

Each top level has its own command interpreter. The actual command interpreters
in much of the code are older versions; the code has since been simplified
considerably. New command interpreters, which may in time replace the older versions,
and which should certainly be used as the models for the command interpreters
of any new top levels, are in the two files \indexfile{command-interpreters-core.lisp} 
and \indexfile{command-interpreters-auto.lisp}.
