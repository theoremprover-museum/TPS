\chapter{Flags}

Here is an example of how to add the flag {\it neg-prim-sub} to \TPS:

Insert into the file prim.lisp the code:
\begin{verbatim}
(defflag neg-prim-sub
  (flagtype boolean)
  (default nil)
  (subjects primsubs)
  (mhelp "When T, one of the primitive substitutions will introduce negation."))

\end{verbatim}

Actually, that code almost worked, but changing the flag did not have the
desired effect. The primsubs were stored in a hashtable which, once computed,
was never changed again, so the code had to be  
replaced by:

\begin{verbatim}
(defflag neg-prim-sub
  (flagtype boolean)
  (default nil)
  (change-fn (lambda (a b c)
	       (declare (ignore a b c))
	       (ini-prim-hashtable)))
  (subjects primsubs)
  (mhelp "When T, one of the primitive substitutions will introduce negation."))

\end{verbatim}

Also put into the file auto.exp the line
\begin{verbatim}
(export '(neg-prim-sub))
\end{verbatim}

There are two ways to update flags. One is to do it manually. This is
supported by function update-flag. The other way is to set flags
automatically. For example, you may have to do this in your .ini
file. If XXX is a flag and you want to set it to YYY, then you can add
a line (set-flag 'XXX YYY) in your .ini file. Sometimes, you may use
(setq XXX YYY), but this is highly discouraged because XXX may have a
"change-fn" associated with it, which should be called whenever you
set XXX. Flag \indexflag{HISTORY-SIZE} is such an example.
(Note that if the variable being set is just a variable, and not a \TPS
flag, then the setq form is correct.)

\section{Symbols as Flag Arguments}

If your new flag accepts symbols as arguments, and only certain symbols 
are acceptable (as in, for example, \indexflag{PRIMSUB-METHOD} or \indexflag{APPLY-MATCH}),
the symbols which can be used should have help messages attached somehow. This can either
be done by defining a new category for the arguments, such as ORDERCOM or DEV-STYLE, or it can
be done using the \indexother{definfo} command: {\tt (definfo foo (mhelp "Help text."))} attaches
the given text to the symbol {\tt foo}.

\section{Synonyms}

It is possible to define two flags with different names which are 
synonymous to each other, using the \indexcommand{defsynonym} macro.
The advantage of this is that it allows the name of a flag to be changed
(from the user's point of view) without requiring either a change in the
code or extensive editing of all the modes saved in the library.

For example:

%\begin{tpsexample}
\begin{verbatim}
(defsynonym SUBNAME
            (synonym TRUENAME)
            (replace-old T)
            (mhelp "SUBNAME is a synonym for the flag TRUENAME."))
\end{verbatim}
%\end{tpsexample}

defines a new synonym for {\tt TRUENAME}. The {\it replace-old} property 
determines whether or not the new synonym is to be regarded as the
"new name" of the flag, if replace-old is {\tt T} 
(and so to be recorded in the library, etc.) 
or merely as an alias, if replace-old is {\tt NIL}.

\section{Relevancy Relationships Between Flags}

When defining a new flag, one can specify relevancy relationships
between the flag and the values of other flags.  For example,
if the flag {\bf DEFAULT-MS} is set to {\bf MS90-3}, then
the flag {\bf MS98-NUM-OF-DUPS} is irrelevant.  On the other
hand, if {\bf DEFAULT-MS} is set to {\bf MS98-1}, then the
flag {\bf MS98-NUM-OF-DUPS} is relevant.  Since we expect the
relevancy information to be incomplete at any point in time,
it makes sense to explicitly record relevancy and irrelevancy
information separately.

The slots used to record these relationships are
\begin{itemize}
\item irrelevancy-preconditions
\item relevancy-preconditions
\item irrelevant-kids
\item relevant-kids
\end{itemize}

If we want to record the irrelevancy relationship in
the {\bf DEFAULT-MS}/{\bf MS98-NUM-OF-DUPS} 
example above, there are two ways.  The first is to
record this in the definition of {\bf DEFAULT-MS} using the
{\it irrelevant-kids} slot as shown below.
\begin{verbatim}
(defflag default-ms
  (flagtype searchtype)
  (default ms90-3)
  (subjects mating-search . . .)
  (change-fn (lambda (flag value pvalue)
	       (when (neq value pvalue) (update-otherdefs value))))
  (irrelevant-kids ((neq default-ms 'ms98-1) '(ms98-num-of-dups)))
  (mhelp . . . ))
\end{verbatim}
The format for this slot is a list of elements of the form
\verb+(<pred> <sexpr>)+
where \verb+<pred>+ is a condition on the value of the flag
being defined
and \verb+<sexpr>+ is an s-expression
which should evaluate to a list of flags.

Alternatively, we can specify the relationship when
we define {\bf MS98-NUM-OF-DUPS} using the slot {\it irrelevancy-preconditions}
\begin{verbatim}
(defflag ms98-num-of-dups
  (default nil)
  (flagtype null-or-posinteger)
  (subjects ms98-1)
  (irrelevancy-preconditions (default-ms (neq default-ms 'ms98-1)))
  (mhelp . . .))
If some positive integer n, we reject any component using more than n 
of the duplications."))
\end{verbatim}
The format for this slot is a list of pairs \verb+(<flag> <pred>)+
where \verb+<flag>+ is a flag and \verb+<pred>+
is a condition on the value of the flag \verb+<flag>+.

Relevancy relationships can be specified in an analogous way.
\begin{verbatim}
(defflag default-ms
  (flagtype searchtype)
  (default ms90-3)
  (subjects mating-search . . .)
  (change-fn (lambda (flag value pvalue)
	       (when (neq value pvalue) (update-otherdefs value))))
  (irrelevant-kids ((neq default-ms 'ms98-1) '(ms98-num-of-dups)))
  (relevant-kids ((eq default-ms 'ms98-1) '(ms98-num-of-dups)))
  (mhelp . . . ))
\end{verbatim}
or
\begin{verbatim}
(defflag ms98-num-of-dups
  (default nil)
  (flagtype null-or-posinteger)
  (subjects ms98-1)
  (irrelevancy-preconditions (default-ms (neq default-ms 'ms98-1)))
  (relevancy-preconditions (default-ms (eq default-ms 'ms98-1)))
  (mhelp . . .))
\end{verbatim}

The conditions given are compiled into a relevancy graph and an irrelevancy graph.
The graphs are labelled directed graphs, where the nodes are flags and the
arcs are labelled by the conditions.
These graphs are currently used in the following two ways.
\begin{enumerate}
\item The relevancy graph is used by the {\bf \indexcommand{UPDATE-RELEVANT}} command.
The user specifies a flag to update.  Based on the value given for that flag,
the user is then asked to specify values for the target flags for which the condition
on the arc is true.  
For example, consider the following session:
\begin{verbatim}
<0>update-relevant default-ms
DEFAULT-MS [MS98-1]>ms98-1
. . . 
MS98-NUM-OF-DUPS [NIL]>2
. . .
<1>update-relevant default-ms
DEFAULT-MS [MS98-1]>ms90-3
. . .
<2>
\end{verbatim}
\item The irrelevancy graph is used to warn the user when an irrelevant flag
is being set.  A flag $F_0$ is {\it never irrelevant}
if there are no arcs with $F_0$ as the target.
A flag $F_1$ is {\it irrelevant} when there is a path
from a flag $F_0$ to $F_1$, where $F_0$ is never irrelevant and
at least one of the conditions on the path evaluates to true.  
Consider the following session.
\begin{verbatim}
<2>default-ms
DEFAULT-MS [MS90-3]>

<3>ms98-num-of-dups
MS98-NUM-OF-DUPS [NIL]>3
WARNING: The setting of the flag DEFAULT-MS makes
     the value of the flag MS98-NUM-OF-DUPS irrelevant.

<4>default-ms
DEFAULT-MS [MS90-3]>ms98-1

<5>ms98-num-of-dups
MS98-NUM-OF-DUPS [3]>2

<6>
\end{verbatim}
\end{enumerate}

\subsection{Automatically Generating Flag Relevancy}

In addition to the relevancy information directly specified
in the \verb+defflag+ declarations in the code, there
is now code (in \indexfile{flag-deps.lisp}) to read
and analyze the code in the lisp files to determine
flag relevancy.  This code first reads the lisp files
and records all \verb+defun+ and \verb+defmacro+ definitions.
Then, it computes easy flag conditions in which flags
and calls to other functions occur.  At present an {\it easy
flag condition} is built from atoms
of the form {\it easy-flag-term} {\it easy-operator} {\it easy-flag-term}
using boolean operations and \verb+IF+.
An {\it easy-operator} is one of =, <, >, <=, >=, eq, eql, equal, or neq.
An {\it easy-flag-term} is a flag, NIL, T, a number, or any quoted term.
The important property these conditions should satisfy is that their
values are static, i.e., they do not depend on the dynamic environment.
(Note that some flags, such as \indexflag{FIRST-ORDER-MODE-MS} are
often dynamically set by the code.  This flag, for example, is removed
from the list of flags, along with any flag, such as \indexflag{RIGHTMARGIN},
that is never relevant for automatic search.  The code does attempt to recognize
when flags are dynamically bound in a certain context and take this into
consideration.)

A user is given the option of computing this flag relevancy information
when calling \indexcommand{UPDATE-RELEVANT} or \indexcommand{SHOW-RELEVANCE-PATHS}.
Another option is to load the relevance information from a file.
Such a file could have been created in a previous \TPS session
using \indexcommand{SAVE-FLAG-RELEVANCY-INFO}.








