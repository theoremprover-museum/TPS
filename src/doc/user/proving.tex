\chapter{Proving theorems}
\label{proving}

\section{Introduction}


{\TPS} can be used to prove theorems automatically, interactively, or by
using various combinations of automatic and interactive facilities.
Even if one is primarily interested in using it in automatic mode, one
should consult the {\bf ETPS User's Manual} \cite{AndrewsTPS88b} to
learn the basics of interacting with both {\ETPS} and {\TPS}.  Even if one
is proving theorems purely interactively, one should probably use {\TPS}
rather than {\ETPS} so that one can use the library facilities.

To start a proof in {\TPS}, use the \indexcommand{PROVE} command. One can
then develop the proof in natural deduction style
interactively, semi-automatically, or
automatically.

To develop a proof in semi-automatic mode, 
one can alternate between applying rules of inference interactively,
using a command called \indexcommand{GO} to apply rules suggested by
{\TPS}, using a command called \indexcommand{GO2} to call a number of
tactics which quickly apply mundane rules of inference, and using the
automatic facilities of {\TPS} to prove lemmas and to derive certain
lines of the proof from other specified lines.  One can develop parts
of the proof in whatever chronological order is most convenient. For
example, one could start by inserting into the proof the lines which
represent the basic outline of a plan for the proof, and then work on
filling in various parts of this outline.

One can invoke the facilities for finding or completing the proof
automatically with the \indexcommand{DIY} (``do it yourself'')
or \indexcommand{DIY-L} command. (The latter is used to fill in part of
a proof, such as a proof of a lemma.)
Before doing this one should set various flags which control the
search procedures. These flags play very important roles.  See chapter
\ref{flags} for some information on how to set flags.
The command \indexcommand{PIY} (``prove it yourself'') combines
the commands \indexcommand{PROVE} and \indexcommand{DIY}.

A set of flags and values for these flags is called a {\it
mode}. 
If one does not know an appropriate
mode when one wishes to invoke {\TPS}'s automatic procedures, one can
use commands which systematically try a variety of modes.  A {\it
bestmode} for a theorem is a mode which can be used to prove that
theorem automatically (and which will, in general, produce a proof
more quickly than other modes).  The {\TPS} library contains certain sets of
modes called {\it goodmodes} such that each of the theorems which
{\TPS} can currently prove automatically can be proven using at least
one of the goodmodes in the set.  For example, GOODMODES1 is a list of
68 modes, and each of the 639 theorems for which
bestmodes are currently saved can be proven automatically by at least
one of the modes in GOODMODES1.  Using the command \indexcommand{PIY2}
(``Prove It Yourself, version 2''), \indexcommand{DIY2}, or 
\indexcommand{DIY2-L}, one can direct {\TPS} to 
apply its proof procedures with each of these 68 modes in turn for a specified
time, then increment the time limit and repeat the process, and
continue in this way until a proof is found or space or patience is
exhausted.  Since {\TPS} can prove many theorems of moderate
difficulty within a few seconds (see 
\cite{Bishop98,Bishop99,Bishop99a,Andrews00a,Brown2002,Brown2004a} for some
examples), this makes {\TPS} extremely
convenient to use for filling in gaps of moderate difficulty while one
is constructing a major proof semi-interactively.


\section{Automatic mode}
%::::::::::::::
%auto-mode.tex
%::::::::::::::

Automatic proof in {\TPS} is based on the `mating-search' paradigm described
in \cite{Andrews2002a}, \cite{Andrews81},  and \cite{Andrews89}, or
enhancements of it described in \cite{Bishop99,Bishop99a,Brown2002,
Brown2004a}.
{\TPS} starts by searching for an expansion proof
\cite{Miller84,Miller87,Andrews89}
or an extensional expansion proof \cite{Brown2004a}, and
then translates this into a natural deduction proof.

There are several ways in which one can use the automatic facilities of \TPS.


The command \indexcommand{DIY} calls the mating-search facility
(see Chapter \ref{ms-guide}), and if that succeeds in finding a proof, applies a
tactic (such as {\tt COMPLETE-TRANSFORM-TAC} or {\tt PFENNING-TAC}) to
translate the expansion tree proof  into a natural
deduction proof. Using  \indexcommand{DIY} is the simplest way to
prove a theorem automatically.

The nature of the search initiated by \indexcommand{DIY} or by
\indexcommand{GO} will be heavily influenced by which setting you have
for the flag \indexflag{DEFAULT-MS}. If you do {\it setflag default-ms} and
then respond with ? when asked for an argument, you'll see the
available options.  Help messages will tell you a little about them, and
there is some more information in Chapter \ref{ms-guide}.


If you want to see more of what is going on in the search for a proof,
you may want to start proving a theorem by entering
the  MATE top-level with a particular wff and
experimenting with the mating-search commands.  To do this, use the
\indexcommand{MATE} command. See section \ref{ms-guide} for information
about this top-level.

After a mating has been found in the mating-search top-level and the
expansion proof constructed, you can apply the proper substitutions to
the expansion tree with the \indexcommand{MERGE-TREE} command, which
also eliminates redundant branches from the tree.  Or just use the
\indexcommand{LEAVE} command, which will apply the merge procedure and
then return you to the main top-level.  To build a natural deduction
proof using the information in the expansion proof you just
constructed, use the command \indexcommand{ETREE-NAT}.  You may also
enter mating-search with the most recent expansion proof created.
This is stored in the variable {\tt \indexother{CURRENT-EPROOF}},
and is the default
value for the {\tt MATE} command.  When you leave mating-search, this
variable is set to the current expansion proof, so you may reenter
with the same proof if desired.  The value of this variable is also
the expansion proof used when translating into a natural deduction
proof with {\tt ETREE-NAT}.  Another variable, {\tt LAST-EPROOF}, stores
the value of the expansion proof created before {\tt CURRENT-EPROOF}.
Either of these symbols may be entered when prompted by the {\tt MATE}
command for a wff or eproof.

Details of how to save output from the mating search are in Chapter
\ref{output}.

Let us discuss here exactly what the effects of the various mating
search and translation commands are.
\begin{itemize}
\item The command \indexcommand{MATE}, when it returns to the top level,
alters the value of
the variable {\tt CURRENT-EPROOF} to be the expansion proof constructed.

\item The command \indexcommand{ETREE-NAT} takes the expansion proof stored
in {\tt CURRENT-EPROOF}, places the information it contains on the specified
line of the natural deduction proof, then calls the specified tactic on the
natural deduction proof.

\item The command \indexcommand{DIY} will construct an expansion proof for
the specified planned line of the current natural deduction proof,
then, like {\tt ETREE-NAT}, will place the information it contains on
the specified line(s) of the natural deduction proof and call the
specified tactic.  The value of {\tt CURRENT-EPROOF} is not changed, however.
\end{itemize}

Note that once the information from the expansion proof has been
placed into the natural deduction proof, as by {\tt DIY} and
{\tt ETREE-NAT}, the value of {\tt CURRENT-EPROOF} is unnecessary to
the translation process, and one can use the command {\tt USE-TACTIC} to
apply translation tactics.
But since the {\tt MATE} command does not transfer the information fromn
{\tt CURRENT-EPROOF} into the natural deduction proof,
the {\tt ETREE-NAT} command must be applied after {\tt MATE} in
order to start the translation process.

It is possible to translate natural deduction proofs to expansion proofs.
This is accomplished by the command {\tt NAT-ETREE}.  The expansion proof
created will be stored in the variable {\tt CURRENT-EPROOF}.  At present,
this procedure
is not complete and will fail on some natural deduction proofs, including:
\begin{itemize}
\item Those which are not cut-free, i.e., do not satisfy the subformula property

\item Those which use substitution of equality rules

\item Those which use the assertion of axioms, like reflexivity of equality
\end{itemize}







\subsection{An Example Using DIY}

\begin{tpsexample}
{\it Comments are in italics.}
<1>save-work workfile
{\it We create a file called workfile.work which will record the commands used.
This is optional.}
<2>prove theorem-name
{\it We could also use the `exercise' command if the theorem to be proved is
an exercise in ETPS.}
<3>diy
\end{tpsexample}

\subsection{An Example Using MATE and ETREE-NAT}


\begin{tpsexample}
{\it Alternatively, we may proceed as follows:}
<1>save-work workfile
<2>prove theorem-name
<3>mate
<4>go
<5>merge-tree
{\it You will be prompted for this when leaving the mate top level.}
<6>etree-nat
{\it To convert the proof to natural deduction style.}
<7>daterec
{\it To store the timing information in the library.}
{\it You may also want to go into the library to store a mode recording the
current flags.}
<8>texproof
{\it To produce printable output.}
\end{tpsexample}


\subsection{A  Longer Example Using MATE and ETREE-NAT}
\label{MATE-then-ETREE-NAT}

To make the formulas easier to read, we have left off the type information.
Note that `\% f x' denotes the image of the set `x' under the function `f'.

\begin{tpsexample}
<8>exercise x5203
(100)        !  \% f [x INTERSECT y] SUBSET \% f x INTERSECT \% f y           PLAN1

{\it We call mating-search directly.}

<9>mate
GWFF (GWFF0): gwff [No Default]>x5203
POSITIVE (YESNO): Positive mode [No]>!

{\it We call the automatic proof search.}
<Mate1>go
...
Displaying VP diagram ...

|                    LEAF7                     |
|                    x T0                      |
|                                              |
|                    LEAF8                     |
|                    y T0                      |
|                                              |
|                    LEAF6                     |
|                  X0 = f T0                   |
|                                              |
|LEAF12      LEAF13       LEAF15      LEAF16   |
|\(\sim\)x t21 OR \(\sim\)X0 = f t21 OR \(\sim\)y t22 OR \(\sim\)X0 = f t22| 
..*.+1.*.+2.*.+3.*.+4..
Trying to unify mating:(4 3 2 1)
Substitution Stack:

t21   ->   T0
t22   ->   T0..
{\it Return to the main top-level.}
<Mate2>leave
Merging the expansion tree.  Please stand by.
****
T
{\it Begin the translation process using the expansion proof just constructed
in the mating-search top-level.}
<9>etree-nat
PREFIX (SYMBOL): Name of the Proof [X5203]>
NUM (LINE): Line Number for Theorem [100]>
TAC (TACTIC-EXP): Tactic to be Used [COMPLETE-TRANSFORM-TAC]>
MODE (TACTIC-MODE): Tactic Mode [AUTO]>

{\it We elide the output from the translation.}

<0>pall

(1)   1      !  EXISTS t18 .x t18 AND y t18 AND x4 = f t18                   Hyp
(2)   1,2    !  x t18 AND y t18 AND x4 = f t18                       Choose: t18
(3)   1,2    !  x t18                                                   RuleP: 2
(4)   1,2    !  y t18                                                   RuleP: 2
(5)   1,2    !  x4 = f t18                                              RuleP: 2
(88)  1,2    !  x t18 AND x4 = f t18                                  RuleP: 3 5
(89)  1,2    !  EXISTS t19 .x t19 AND x4 = f t19                    EGen: t18 88
(93)  1,2    !  y t18 AND x4 = f t18                                  RuleP: 4 5
(94)  1,2    !  EXISTS t20 .y t20 AND x4 = f t20                    EGen: t18 93
(95)  1,2    !      EXISTS t19 [x t19 AND x4 = f t19]
                 AND EXISTS t20 .y t20 AND x4 = f t20               RuleP: 89 94
(96)  1      !      EXISTS t19 [x t19 AND x4 = f t19]
                 AND EXISTS t20 .y t20 AND x4 = f t20                RuleC: 1 95
(97)         !          EXISTS t18 [x t18 AND y t18 AND x4 = f t18]
                 IMPLIES     EXISTS t19 [x t19 AND x4 = f t19]
                         AND EXISTS t20 .y t20 AND x4 = f t20         Deduct: 96
(98)         !  FORALL x4 .        EXISTS t18 [x t18 AND y t18 AND x4 = f t18]
                            IMPLIES     EXISTS t19 [x t19 AND x4 = f t19]
                                    AND EXISTS t20 .y t20 AND x4 = f t20
                                                                     UGen: x4 97
(99)         !  FORALL x4 .        EXISTS t18 [x t18 AND y t18 AND x4 = f t18]
                            IMPLIES     EXISTS t19 [x t19 AND x4 = f t19]
                                    AND EXISTS t20 .y t20 AND x4 = f t20
                                                                    Equality: 98
(100)        !  \% f [x INTERSECT y] SUBSET \% f x INTERSECT \% f y   EquivWffs: 99
\end{tpsexample}


\subsection{Automatically Produced Proofs with Lemmas}

Some search procedures (e.g., using \indexcommand{DIY} when \indexflag{DEFAULT-MS} is \indexother{MS98-1} and \indexflag{DELAY-SETVARS} is T)
will generate proofs which include lemmas which are asserted in the proofs.  The proofs of these lemmas are also generated
and can be examined by using \indexcommand{RECONSIDER}.  The name of the lemma is included with the
Assert justification.  In general, \indexcommand{PROOFLIST} lists the names of all the natural deduction proofs
in memory.  An example is the proof of THM2 produced using the mode EASY-SV-MODE.


\subsection{Interrupts: \^C, <CR>, \^G<CR>, M<CR> and T<CR>}\label{interrupt}

During mating search, if the value of the flag \indexflag{INTERRUPT-ENABLE} is
T, you can interrupt the search by typing {\tt <Return>}.  You will get
a new mating-search (or matingstree, as appropriate) top-level, where you can inspect
and/or alter the current expansion tree and mating.  Type \indexcommand{LEAVE} to return to mating-search.

Similarly, you can type {\tt \^G<Return>} (i.e. Control-G, then Return) to abandon the
search for good and return to the current top level. It is not possible to restart a
search after doing this.

Lastly, you can type {\tt M<Return>} to see the current mating, or {\tt T<Return>} to see a
printout of the time taken so far by the search. The search will not be interrupted at all if you do this.

Of course, there is one, more drastic, way to interrupt a search: press {\tt \^C}.
This will work regardless of the setting of \indexflag{INTERRUPT-ENABLE}, and will throw you into
the Lisp debugger. Leaving the debugger with a restart should return you to either the {\TPS} top level or the
current sub-toplevel; leaving it with a continue should carry on the search.

For Allegro Common Lisp (debuggers are not standardized, so this will be different in other lisps),
this works as follows:
\begin{alltt}
\^C                             (control C)
Error: Received signal number 2 (Keyboard interrupt)
Restart actions (select using :continue):
 0: continue computation
[1c] <cl>		
	{\it Here one can use debugger commands,
        or get into TPS on another level as follows:}
[1c] <cl> (secondary-top-main)
	{\it Now do what you want in TPS to examine things.
        To get back to the original TPS level:}
<73>\^C
Error: Received signal number 2 (Keyboard interrupt)
Restart actions (select using :continue):
 0: continue computation
 1: continue computation
[2c] <cl> :cont 1
{\it Now the original TPS process continues}
{\it Typing :res would have returned us to the TPS top level}
\end{alltt}


\section{Interactive Mode}

\subsection{Natural Deduction}

There are several examples showing how to construct natural deduction
proofs completely interactively in Chapter 4 of the {\bf ETPS User's
Manual} \cite{AndrewsTPS88b}.  Everything that can be done in {\ETPS}
can also be done in {\TPS}.

To prepare a demonstration of how to construct a proof
interactively, use \indexcommand{SAVE-WORK}.  To give the demonstration, use
\indexcommand{BEGIN-PRFW} with the flags \indexflag{PROOFW-ACTIVE},
\indexflag{PROOFW-ACTIVE+NOS} or
\indexflag{PROOFW-ALL} set to T, then
use \indexcommand{EXECUTE-FILE} to give the demonstration;
respond `yes' to the `STEPPING?' prompt.
The audience will see the proof being constructed step-by-step in the
proofwindows. (Note: Stepping will only stop between each command, so it will not stop
(for example) between each step of GO2. Also, if you change top level in a work file,
stepping will be turned off until you return to the original top level. It is possible to force
a stop in these situations by inserting a PAUSE command into the workfile; see the help message
for \indexcommand{PAUSE} for more details.)

Alternatively, you may wish to go into \indexcommand{MATE} to prove the theorem
automatically, then call \indexcommand{ETREE-NAT} in interactive mode to
demonstrate (with the aid of the proofwindows) how the natural deduction proof
can be constructed step-by-step. Make sure that the setting of \indexflag{ETREE-NAT-VERBOSE}
is appropriate before doing this!
(The user should note that if there is a command in the
work file that changes the top level - for example, \indexcommand{BEGIN-PRFW}
or \indexcommand{MATE} - then this command will also turn off stepping.
One can get around this limitation by splitting the work file into two at
the point where the change of top level occurs.)

While translating expansion proofs to natural deduction proofs, you
may wish to set the flag \indexflag{TACMODE} to {\tt INTERACTIVE}.  See Section
\ref{tactics} for information on the effect of this flag. If you are using
\indexcommand{ETREE-NAT} it is best to call \indexcommand{MERGE-TREE} first;
in fact, the mate top level will automatically prompt you for this if you
attempt to leave with a completed mating.

Equalities in expansion proofs can be translated into equational
proofs as laid out in \cite{Pfenning86}.  This does not include the
work on extensionality.  In order to have the proper expansion
proof transformation done during merging, one should have the flag
\indexflag{REMOVE-LEIBNIZ} set to {\tt T}, which is the default.  It is also best to
have the flags \indexflag{REWRITE-EQUAL-EXT}, \indexflag{REWRITE-EQUAL-EAGER}, and
\indexflag{REWRITE-ONLY-EXT} set to {\tt NIL}, and \indexflag{REWRITE-EQUALITIES} set
to {\tt T}.

From a {\TPS} prompt (this could be either a top level prompt or the prompt produced by a
command which requires you to input some arguments) you can type \indexother{PUSH} to suspend
what you're doing and start a new top level. The command \indexother{POP} will return from this
top level to the point where you typed \indexother{PUSH}. For example, you could suspend
an interactive session with {\tt ETREE-NAT} in order to print out the proof at various stages of
development.

It is also possible to interrupt an automatic search, change some flags and then continue with the
search; see section \ref{interrupt} for details.

\subsection{Extensional Sequent Calculus}

There is a top level \indexcommand{EXT-SEQ} which provides
an environment for constructing derivations of (one-sided) sequents
in the extensional sequent calculus described in \cite{Brown2004a}.
The command \indexcommand{?} will list all the available rules.
Some rules are basic while others are derived rules which can be
expanded later.  The cut rule is included, but certain commands
(namely, \indexcommand{CUTFREE-TO-EDAG}) will only work when given
cut-free derivations.

The command \indexcommand{GO2} in the top level EXT-SEQ will
automatically suggest rules which are applicable.  This can
make interactive construction of a sequent calculus derivation
much easier.

Sequent derivations can be saved and restored using \indexcommand{SAVEPROOF}
and \indexcommand{RESTOREPROOF}.

\subsection{Manipulating Proofs}\label{manip-pfs}

{\TPS} has many facilities for manipulating proofs.  There can be many
proofs in memory at the same time, and the command \indexcommand{PROOFLIST} lists the
names of all the natural deduction proofs or extensional sequent
derivations currently in memory, along with the theorems they
prove. One proof is designated as the current proof, and one can
change this with the \indexcommand{RECONSIDER} command.  Proofs can be saved as files
by \indexcommand{SAVEPROOF}, and restored to memory by \indexcommand{RESTOREPROOF}.

\indexcommand{CREATE-SUBPROOF} creates a new proof consisting of specified lines of
the current proof, plus all the lines on which they depend.
\indexcommand{MERGE-PROOFS} merges all of the lines of a subproof into the current
proof.  \indexcommand{TRANSFER-LINES} copies specified lines of a subproof, and all
lines on which they depend, into the current proof.

Various commands (\indexcommand{MOVE}, \indexcommand{DELETE}, 
\indexcommand{INTRODUCE-GAP}, \indexcommand{MODIFY-GAPS},
\indexcommand{RENUMBERALL}, \indexcommand{SQUEEZE}) are 
available for deleting or moving portions of
a proof, changing the gaps in the numbers between lines, and
renumbering the lines.  The \indexcommand{CLEANUP} command will delete all lines of a
completed proof which are not actually needed to prove the final line.

Sometimes one wishes to look at the main steps in a natural deduction
proof without looking at all the intermediate steps.  The command
\indexcommand{BUILD-PROOF-HIERARCHY} builds dependency information into a proof so
that the proof can be viewed as a hierarchy of subproofs.  The command
\indexcommand{PBRIEF} displays the proof lines included in the top levels of this
hierarchy to a depth specified by the user.  When one asks {\TPS} to
\indexcommand{EXPLAIN} a specified line in a proof, it displays in the same way the
lines of the proof which are used to prove the specified line.
\indexcommand{PRINT-PROOF-STRUCTURE} displays the hierarchy itself in terms of the
numbers of the proof lines.


\section{Combining Interactive and Automatic Searches}

The command \indexcommand{GO} will apply inference rules based
upon the structure of the formulas in the current proof structure --
breaking up conjunctions, applying the deduction rule, instantiating
definitions, etc.  This facility is rather shallow, and requires the user to
provide any terms for universal instantiation or existential generalization.
Thus while it may be useful in getting a proof started, it will
eventually fail.  The {\tt GO} facility is fairly static as well; to change
the priority of the rules and/or keep some rules from being applied requires
some programming (see the file {\tt ml2-prior.lisp}).

Tactics can also be used to do the same job.  In this case, the user can
build a tactic (see section \ref{tactics}) which will apply inference rules
in whatever order is desired.  Tactics allow the user to experiment with
different proof strategies and express his or her own creative spirit.
Tactics for applying most of the current inference rules are already defined.
See section \ref{usetac} for more information on commands which invoke
the most commonly used tactics
such as \indexmexpr{MONSTRO} and \indexmexpr{GO2}.

If a proof is being constructed interactively by natural deduction
commands, it is possible to use the command \indexcommand{DIY-L} to automatically
complete some of the subproofs. This calls DIY to prove a
lemma, adding lines to the proof within a specified range. This is useful
for quickly filling in the trivial parts of more difficult theorems which
you are proving interactively. See the help message for \indexcommand{DIY-L}
for details. To keep the proof short and readable, automatically-produced subproofs
need not be translated completely; see the help message of the flag \indexflag{USE-DIY}
for more information about this.

The tactic {\tt DIY-TAC} basically just
calls the  \indexcommand{DIY} command, and thus can be used in tactics which first do
some manipulation of the proof based upon the structure of the
formulas, then call mating-search when instantiations of quantifiers
must be found. (Note: if the flag \indexflag{USE-DIY} is set,
the translation may simply consist of justifying the goal line as `Automatic'
from the support lines. This is useful for keeping the proof short.)

We now give several examples of how to start a proof interactively
and continue automatically. Note that if you modify the expansion tree
interactively, you should use the command \indexcommand{CJFORM} before
attempting to construct a mating interactively.

\subsection{An Example Using Go to Start the Proof }

\begin{tpsexample}
<3>exercise x5203
(100)        !  \% f [x INTERSECT y] SUBSET \% f x INTERSECT \% f y           PLAN1
{\it We use the GO command to start the proof.}
<4>go
Considering planned line 100.
  IDEF 100
Command [(IDEF 100)]>
{\it Instantiate the definition of SUBSET.}
(99)         !  FORALL x .        \% f [x INTERSECT y] x
                           IMPLIES [\% f x INTERSECT \% f y] x               PLAN2
Considering planned line 99.
{\it The first two occurrences of x have different types, but they are not shown.}
  UGEN 99
Command [(UGEN 99)]>
(98)         !  \% f [x INTERSECT y] x IMPLIES [\% f x INTERSECT \% f y] x    PLAN3
Considering planned line 98.
  DEDUCT 98
Command [(DEDUCT 98)]>
(1)   1      !  \% f [x INTERSECT y] x                                        Hyp
(97)  1      !  [\% f x INTERSECT \% f y] x                                  PLAN4
Considering planned line 97.
  EDEF 1
  IDEF 97
Command [(EDEF 1)]>
(2)   1      !  EXISTS t .[x INTERSECT y] t AND x = f t                  Defn: 1
Considering planned line 97.
  RULEC 97 2
Command [(RULEC 97 2)]>
Some defaults could not be determined.
y (GWFF): Chosen Variable Name [No Default]>''t''
(3)   1,3    !  [x INTERSECT y] t AND x = f t                          Choose: t
(96)  1,3    !  [\% f x INTERSECT \% f y] x                                  PLAN7
Considering planned line 96.
  ECONJ 3
Command [(ECONJ 3)]>
(4)   1,3    !  [x INTERSECT y] t                                        Conj: 3
(5)   1,3    !  x = f t                                                  Conj: 3
Considering planned line 96.
  SUBST=L 5
  SUBST=R 5
  EDEF 4
  IDEF 96
Command [(SUBST=L 5)]>\^G
`[Command aborted.]'
{\it We abort the command because the advice doesn't seem very helpful.
Here are the current active lines of the proof.}
<5>\^p
(4)   1,3    !  [x INTERSECT y] t                                        Conj: 3
(5)   1,3    !  x = f t                                                  Conj: 3
               ...
(96)  1,3    !  [\% f x INTERSECT \% f y] x                                  PLAN7
T
{\it Call mating-search on the partial proof.}
<6>diy
GOAL (PLINE): Planned Line [96]>
SUPPORT (EXISTING-LINELIST): Support Lines [(4 5)]>
...
Displaying VP diagram ...

|                   LEAF88                   |
|                    x t                     |
|                                            |
|                   LEAF89                   |
|                    y t                     |
|                                            |
|                  LEAF90                    |
|                  x = f t                   |
|                                            |
|LEAF96      LEAF97      LEAF99     LEAF100  |
|\(\sim\)x t16 OR \(\sim\)x = f t16 OR \(\sim\)y t17 OR \(\sim\)x = f t17|
..*.+1.*.+2.*.+3.*.+4..
Trying to unify mating:(4 3 2 1)
Substitution Stack:

t16   ->   t
t17   ->   t.
Eureka!  Proof complete..
|                 LEAF88                 |
|                  x t                   |
|                                        |
|                 LEAF89                 |
|                  y t                   |
|                                        |
|                LEAF90                  |
|                x = f t                 |
|                                        |
|LEAF96     LEAF97     LEAF99    LEAF100 |
| \(\sim\)x t  OR \(\sim\)x = f t OR  \(\sim\)y t  OR \(\sim\)x = f t|
****
{\it A proof was found and now will be translated back to natural deduction.}
What tactic should be used for translation? [COMPLETE-TRANSFORM-TAC]>
Tactic mode? [AUTO]>

{\it For brevity we have elided the output from the translation process.}

\end{tpsexample}

{\it Here's the complete proof. Note that it is slightly different from
that shown in section \ref{MATE-then-ETREE-NAT}, where all the
definitions were instantiated at one time.}

\begin{tpsexample}

<7>pall
(1)   1      !  \% f [x INTERSECT y] x                                        Hyp
(2)   1      !  EXISTS t .[x INTERSECT y] t AND x = f t                  Defn: 1
(3)   1,3    !  [x INTERSECT y] t AND x = f t                          Choose: t
(4)   1,3    !  [x INTERSECT y] t                                        Conj: 3
(5)   1,3    !  x = f t                                                  Conj: 3
(6)   1,3    !  x t AND y t                                         EquivWffs: 4
(7)   1,3    !  x t                                                     RuleP: 6
(8)   1,3    !  y t                                                     RuleP: 6
(88)  1,3    !  x t AND x = f t                                       RuleP: 5 7
(89)  1,3    !  EXISTS t14 .x t14 AND x = f t14                       EGen: t 88
(93)  1,3    !  y t AND x = f t                                       RuleP: 5 8
(94)  1,3    !  EXISTS t15 .y t15 AND x = f t15                       EGen: t 93
(95)  1,3    !      EXISTS t14 [x t14 AND x = f t14]
                 AND EXISTS t15 .y t15 AND x = f t15                RuleP: 89 94
(96)  1,3    !  [\% f x INTERSECT \% f y] x                          EquivWffs: 95
(97)  1      !  [\% f x INTERSECT \% f y] x                            RuleC: 2 96
(98)         !  \% f [x INTERSECT y] x IMPLIES [\% f x INTERSECT \% f y] x
                                                                      Deduct: 97
(99)         !  FORALL x .        \% f [x INTERSECT y] x
                           IMPLIES [\% f x INTERSECT \% f y] x          UGen: x 98
(100)        !  \% f [x INTERSECT y] SUBSET \% f x INTERSECT \% f y        Defn: 99

\end{tpsexample}


\subsection{Duplicating Interactively and then Running Matingsearch}

\begin{tpsexample}
<11>mate
GWFF (GWFF0-OR-EPROOF): Gwff or Eproof [No Default]>thm-name
DEEPEN (YESNO): Deepen? [Yes]>
WINDOW (YESNO): Open Vpform Window? [No]>

<Mate12>vp
<Mate13>goto leaf58
{\it This should be the name of the literal you wish to duplicate.}
<Mate14>up
{\it Go to the appropriate expansion node.}
EXP8
<Mate18>vp
<Mate19>dup-var
<Mate21>dp*
<Mate23>\(\wedge\)
<Mate24>vp
{\it To check we got it right...}
<Mate27>go
{\it To start matingsearch.}
\end{tpsexample}

Users who are planning to both duplicate and Skolemize interactively should
note that duplication and Skolemization are not interchangeable. If
you duplicate first and then Skolemize you will get different Skolem
functions, whereas if you Skolemize and then duplicate you will get
the same Skolem function twice.

\subsection{Applying Primsubs Interactively and then Running Matingsearch}

\begin{tpsexample}
<23>mate
GWFF (GWFF0-OR-EPROOF): Gwff or Eproof [No Default]>thm-name
DEEPEN (YESNO): Deepen? [Yes]>
WINDOW (YESNO): Open Vpform Window? [No]>

<Mate24>name-prim
{\it We see that PRIM1 is the term we want to use.}
<Mate25>vp
{\it This will show the variable name we want. Note that pdeep and psh may
not show the right variable; things get renamed!}
<Mate26>prim-single
TERM (GWFF):  [No Default]>prim1
VAR (GWFF):  [No Default]>`R^3(OII)'
<Mate27>vp	
{\it Just to check...}
<Mate28>go
\end{tpsexample}

\subsection{Mating Interactively and then Unifying}

\begin{tpsexample}
<Mate24>cjform
<Mate25>vp
{\it We do this because the leaf names may have changed.}
<Mate26>add-conn*
{\it Now we type in a whole list of connections; we could also have used add-conn.}
{\it Suppose we now think we have a complete mating...}
<Mate27>complete-p
Mating is complete.
<Mate28>unify
{\it We enter the unification top level; this only works for higher-order problems.}
<Unif29>go

In case TPS halts with the message
MORE
you can proceed as follows:
<Unif30>MAX-UTREE-DEPTH
MAX-UTREE-DEPTH [20]>30
<Unif31>go
\end{tpsexample}

\subsection{Duplicating and Mating Interactively and then Converting to ND}

\begin{tpsexample}
<66>mate
<Mate67>vp
<Mate68>goto leaf58
<Mate69>up
<Mate73>vp
<Mate74>dup-var
<Mate76>dp*
<Mate78>^
<Mate79>vp
<Mate80>add-conn leaf29 leaf58
{\it ...as often as needed to get the mating...}
<Mate88>complete-p
<Mate89>show-mating
<Mate90>show-substs
<Mate91>vp
<Mate92>leave
Merge the expansion tree? [Yes]>
<93>etree-nat
\end{tpsexample}

\subsection{Using Prim-Single and Dup-Var Interactively}

\begin{tpsexample}
<8>mate
{\it We should check some flags before proceeding:}
<Mate10> PRIM-BDTYPES
<Mate11>MIN-PRIM-DEPTH
<Mate12>MAX-PRIM-DEPTH
<Mate13> PRIM-QUANTIFIER
<Mate16>vp
<Mate17>NAME-PRIM
<Mate24>PRIM-SINGLE
SUBST (GWFF):  [No Default]>prim14
VAR (GWFF):  [No Default]>`M^0(O(II))'
<Mate25>vp
<Mate30>etd
<Mate31>goto
NODE (SYMBOL):  [No Default]>exp2
<Mate33>dup-var
<Mate35>dp*
<Mate37>^
<Mate38>vp
<Mate39>add-conn*
<Mate40>complete-p
{\it We should check some flags before proceeding:}
<Mate42>MAX-SEARCH-DEPTH
<Mate43>MAX-UTREE-DEPTH
<Mate44>unify
<Unif45>go
<Unif47>leave
<Mate48>leave
Merge the expansion tree? [Yes]>
<50>etree-nat
<51>pall
{\it ...to see the final proof.}
\end{tpsexample}

