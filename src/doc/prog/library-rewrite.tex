\chapter{Library}

% @comment(the old library documentation, which is misleading, is in library.mss)

The library commands are documented in the user manual.

A library can currently only occupy one directory (i.e. subdirectories
may not be used), although users are given the ability to refer additionally to a
common directory of basic definitions by using the \indexflag{BACKUP-LIB-DIR} flag.

Many library commands are essentially written as two copies of the same function, 
the first of which checks the default library directory and the second of which
checks the backup directory. The second piece of code is surrounded by {\tt unwind-protect}
commands in order to make sure that the \indexflag{DEFAULT-LIB-DIR} and \indexflag{BACKUP-LIB-DIR}
flags always end up correct. Users may not write to the backup directory.

The index for each library directory is stored in the {\it libindex.rec} file in that directory;
this file that is read every time the directory is changed. Objects are removed from the library 
by deleting them from the appropriate {\it .lib} file and removing their entry from the {\it libindex.rec}
file. This may result in a {\it .lib} file of zero length, in which case the file is deleted.

Objects which are loaded by the user are re-defined as \TPS objects; library objects of type 
MODE or MODE1 become \TPS modes, whereas gwffs and abbreviations each become both theorems
(of type {\it library}) and abbreviations. Notice that this blurs the distinction between a gwff and an 
abbreviation. Users are allowed to redefine \TPS theorems of type {\it library}, and their corresponding
abbreviations; theorems of other types may not be redefined (this is to prevent users from accidentally
overwriting standard abbreviations with their own library definitions).

Library definitions are parsed every time they are written, and this involves re-loading all of the
needed objects. Since the needed objects are often abbreviations, this will frequently result in 
their redefinition, and so \lisp will generate warning messages. If a large file is being re-parsed,
this can take a long time and produce a huge number of warnings.

\section{Converting TPTP Problems to \TPS~library items}

Every needed function or flag is set in {\it library2.lisp}. The utility is made of
two library commands: \indexcommand{INSERT-TPTP}, to insert one TPTP problem into \TPS, and
\indexcommand{INSERT-TPTP*} to automatically call insert-tptp on an entire directory. These
two commands act on .tps files, which are generated using the TPTP2X utility. One flag, \indexflag{INSERT-LIB-DIR}, 
defines the output directory for the newly created items.

Please note that a modified format.tps file is used, in order to prevent
conflit with other objects inside of tps. The original file uses 'const', 'def',
'axiom' and 'thm' as functions: they have been replaced by 'const-insert',
'def-insert', 'axiom-insert' and 'thm-insert'.

The principal issue when converting TPTP problems into TPS items is to avoid
using already defined objects. For this reason, every inserted item is
suffixed, usually with the name of the destination library file (e.g. 'one'
becomes 'one-ALG2684').

More information about how to process this conversion are available in the User's guide.