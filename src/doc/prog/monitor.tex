\chapter{The Monitor}
\label{monitor}

The \indexother{monitor} is designed to be called during automatic proof searches; its basic
operation is described in the User Manual. There are three basic steps required to 
write a new monitor function, which are described below, using the monitor function 
\indexother{monitor-check} as an example. More examples are in the file {\it monitor.lisp}.

\section{The Defmonitor Command}

The command \indexcommand{defmonitor} behaves just like {\tt defmexpr}, the only difference being
that the function it defines does not appear in the list when the user types {\tt ?}. This command
will be called by the user before the search is begun, and should be able to accept any required 
parameters (or to calculate them from globally accessible variables at the time the command is
called).

So, for example, the {\tt defmonitor} part of \indexother{monitor-check} looks like this:

%\begin{tpsexample}
\begin{verbatim}
(defmonitor monitor-check
  (argtypes string)
  (argnames prefix)
  (arghelp "Marker string")
  (mainfns monitor-chk)
  (mhelp "Prints out the given string every time the monitor is called, 
followed by the place from which it was called."))

(defun monitor-chk (string)
  (setq *current-monitorfn* 'monitor-check)
  (setq *current-monitorfn-params* string)
  (setq *monitorfn-params-print* 'msg))
\end{verbatim}
%\end{tpsexample}

Note that this accepts a marker string as input from the user (other monitor functions may 
look for a list of connections, or flags, or the name of an option set; it may be necessary 
to define a new data type to accommodate the desired input). It then calls a secondary 
function, which in this case needs to do very little further processing in order to 
establish the three parameters which are {\it required} for every such function: {\tt *current-monitorfn*}
contains a symbol corresponding to the name of the monitor function, {\tt *current-monitorfn-params*} 
contains the user-supplied parameters (in any form you like, since your function will be the only 
place where they are used) and {\tt *monitorfn-params-print*} contains the name of a function that can 
print out {\tt *current-monitorfn-params*} in a readable way, for use by the commands \indexcommand{monitor}
and \indexcommand{nomonitor}. The latter should be set to {\tt nil} if you can't be bothered to write such 
a function.

\section{The Breakpoints}

In the relevant parts of the mating search code, you should insert breakpoints of the form:

%\begin{tpsexample}
\begin{verbatim}
(if monitorflag 
    (funcall (symbol-function *current-monitorfn-params*) 
             <place> <alist>))
\end{verbatim}
%\end{tpsexample}

The value of {\it place} should reflect what part of the code the breakpoint is at. So, for example,
it might be {\tt 'new-mating}, {\tt 'added-conn} or {\tt 'duplicating}.

The value of {\it alist} should be an association list of local variables and things that your monitor
function will need. For example, {\it alist} might be {\tt (('mating . active-mating) ('pfd . nil))}; it might 
equally well be just {\tt nil}.

All breakpoints should have exactly this pattern. By typing {\it grep "(if monitorflag (funcall" *.lisp} in
the {\it tpslisp} directory, you can get a listing of all the currently defined breakpoints.

\section{The Actual Function}

This is the function which will actually be called during mating search. By convention, it has the
same name as the {\tt defmonitor} function. Normally, it will first check the value of {\it place}, to
see if it has been called from the correct place; it can then use the {\tt assoc} command to retrieve the
relevant entries from {\it alist}. Theoretically, it should be completely non-destructive so as to ensure 
that the mating search continues properly; of course, you may be as destructive as you like, provided 
you understand what you're doing...

The function for {\tt monitor-check} is as follows; notice that this does not check {\it place} since it 
is intended to act at every single breakpoint.

%\begin{tpsexample}
\begin{verbatim}
(defun monitor-check (place alist)
  (declare (ignore alist))
  (msg *current-monitorfn-params* place t)) 
\end{verbatim}
%\end{tpsexample}

