A cut elimination algorithm is worked out in Frank's
Thesis.  First he defines a notion of expansion development
(a sort of sequent calculus for sets of expansion trees
with rules for quantifiers, merging, and cuts).
Then he gives reductions on expansion developments with
the hope that these reductions result in an expansion tree.
Frank's algorithm is not currently implemented as part of TPS.
Of course, there are many ways of representing cuts and performing
cut elimination.

\subsection{An Example of a Loop in a Cut Elimination Algorithm}

One approach we tried (Chad, Summer 2001)
was to include explicit
CUT and MERGE nodes in expansion trees, defining redexes,
and doing cut elimination by contracting redexes.
This section contains a brief outline of the approach
and an example that shows how a loop can occur.

A CUT node is of the form
$$\etrca{CUT}{}{C^*}{}{B^*}{}{B^{**}}$$
where $C^*$ has shallow formula $C$,
and $B^*$ and $B^{**}$ have the same shallow
formula, but opposite polarity.  The
polarity and shallow formula
of the CUT node is the same as the
polarity of the shallow formula of $C^*$.
The deep formula of the cut node is
$$deep(C^*) \land [deep(B^*) \lor deep(B^{**})]$$
A MERGE node is of the form
$$\etrba{MERGE}{}{A^*}{}{A^{**}}$$
where $A^*$ and $A^{**}$ have the same polarity
and same shallow formula $A$.  The shallow
formula of the MERGE node is also $A$.
The polarity is also inherited from the children.
The deep formula is given by
$$deep(A^*) \land deep(A^{**}).$$

With these nodes, the translation from a natural deduction proof
(see subsection \ref{ceb-nat-etr})
can now be extended to translate backward coercions.
Ignoring hypothesis for simplicity,
suppose we have a backward coercion
$$ \ianc{B\Uparrow}{B\downarrow}{bcoercion}$$
a negative expansion tree $C^*$ with shallow
formula $C$, and a positive expansion tree $B^*$
with shallow formula $B$.  By induction hypothesis,
we can obtain an appropriate positive expansion tree
$B^{**}$ with shallow formula $B$.  Then the algorithm
returns
$$\etrca{CUT}{}{C^*}{}{B^*}{}{B^{**}}$$

There is one more modification required in the algorithm.
When translating a hypothesis line, we used the merge algorithm
on expansion trees.  However, merge is not defined on expansion
trees containing CUT nodes.  So, instead we make an explicit
MERGE node with the two trees as children.  The actual merging
would be done during cut/merge elimination.  That is,  if
we are translating a hypothesis line
$$A_1,\ldots, A_n \vdash A_j,$$
then we must start with positive expansion trees $A_i^{**}$,
a positive expansion tree $A_i^{*}$,
and a negative expansion tree $C^*$.
Instead of associating the hypothesis line with the merge
of $A_i^{*}$ and $A_i^{**}$, we now associate the line
with the node
$$\etrba{MERGE}{}{A_i^{*}}{}{A_i^{**}}$$

The most interesting case is eliminating a CUT between
a selection and an expansion node.  There may be
multiple expansion nodes and the acyclicity of the dependency
relation may affect the order in which the expansion terms
are processed.  We were trying to reduce such a CUT
by doing a global merge of two modified expansion trees.
Suppose the expansion tree is $Q$ and contains a CUT node
of the form
$$\etrca{CUT}{}{C}{}{\sel{}{\forall x . A(x)}{a}{P(a):A(a)}}{}
{\gexpnode{}{\forall x . A(x)}{t_1}{P_1:A(t_1)}{t_n}{P_n:A(t_n)}}$$
Let us use the notation $Q[CUT]$ to indicate this CUT node inside the
larger tree $Q$.
Assuming we do not have a problem with acyclicity, the CUT node
should reduce as follows:
$$\etrba{MERGE}{}{\{t_1/a\} Q[CUT1]}{}{Q[CUT2]}$$
where $CUT1$ is
$$\etrca{CUT1}{}{C}{}{P(t_1):A(t_1)}{}{P_1:A(t_1)}$$
and $CUT2$ is
$$\etrca{CUT2}{}{C}{}{\sel{}{\forall x . A(x)}{a}{P:A(a)}}{}
{\gexpnode{}{\forall x . A(x)}{t_2}{P_2:A(t_2)}{t_n}{P_n:A(t_n)}}.$$

However, this leads to a loop as in the following simple example.
Consider the following proof of 
$$ \forall \,x_{\greeki} \forall \,y_{\greeki} \,A_{\greeko\greeki\greeki} \,x \,y \supset \,A \,a_{\greeki} \,a \land \,A \,b_{\greeki} \,b$$
using the lemma
$$ \forall \,z_{\greeki} \,A_{\greeko\greeki\greeki} \,z \,z.$$

\vbox{\indent
\linenumbox {(1)}\hypnumbox {1}\turnstile\partformula{$ \forall \,x_{\greeki} \forall \,y_{\greeki} \,A_{\greeko\greeki\greeki} \,x \,y$}\lastformula\judgelink{Hyp}
}\filbreak
\vbox{\indent
\linenumbox {(2)}\hypnumbox {1}\turnstile\partformula{$ \forall \,y_{\greeki} \,A_{\greeko\greeki\greeki} \,z_{\greeki} \,y$}\lastformula\judgelink{UI:\ $  \,z_{\greeki}$\  1}
}\filbreak
\vbox{\indent
\linenumbox {(3)}\hypnumbox {1}\turnstile\partformula{$ \,A_{\greeko\greeki\greeki} \,z_{\greeki} \,z$}\lastformula\judgelink{UI:\ $  \,z_{\greeki}$\  2}
}\filbreak
\vbox{\indent
\linenumbox {(4)}\hypnumbox {1}\turnstile\partformula{$ \forall \,z_{\greeki} \,A_{\greeko\greeki\greeki} \,z \,z$}\lastformula\judgelink{UGen:\ $  \,z_{\greeki}$\  3}
}\filbreak
\vbox{\indent
\linenumbox {(5)}\hypnumbox {1}\turnstile\partformula{$ \,A_{\greeko\greeki\greeki} \,a_{\greeki} \,a$}\lastformula\judgelink{UI:\ $  \,a_{\greeki}$\  4}
}\filbreak
\vbox{\indent
\linenumbox {(6)}\hypnumbox {1}\turnstile\partformula{$ \,A_{\greeko\greeki\greeki} \,b_{\greeki} \,b$}\lastformula\judgelink{UI:\ $  \,b_{\greeki}$\  4}
}\filbreak
\vbox{\indent
\linenumbox {(7)}\hypnumbox {1}\turnstile\partformula{$ \,A_{\greeko\greeki\greeki} \,a_{\greeki} \,a \land \,A \,b_{\greeki} \,b$}\lastformula\judgelink{Conj: 5 6}
}\filbreak
\vbox{\indent
\linenumbox {(8)}\hypnumbox {}\turnstile\partformula{$ \forall \,x_{\greeki} \forall \,y_{\greeki} \,A_{\greeko\greeki\greeki} \,x \,y \supset \,A \,a_{\greeki} \,a \land \,A \,b_{\greeki} \,b$}\lastformula\judgelink{Deduct: 7}
}\filbreak

When translating this proof we obtain a tree $Q$ with two cut nodes
$$\etrca{CUT1}{}{R_1}{}{\etraa{SEL1}{c}{P_1(c):A\, c\, c}}{}{\etraa{EXP1}{a}{P_2:A\, a\, a}}$$
and
$$\etrca{CUT2}{}{R_2}{}{\etraa{SEL2}{d}{P_3(d):A\, d\, d}}{}{\etraa{EXP2}{b}{P_4:A\, b\, b}}$$
corresponding to the two applications of the lemma in line 4.
Contracting $CUT1$ gives
$$\etrba{MERGE}{}{\{a/c\}Q[CUT3]}{}{Q[CUT4]}$$
where $CUT3$ and $CUT4$ are
$$\etrca{CUT3}{}{\{a/c\}R_1}{}{P_1(a):A\, a\, a}{}{P_2:A\, a\, a}$$
and
$$\etrca{CUT4}{}{R_1}{}{\etraa{SEL3}{c}{P_1(c):A\, c\, c}}{}{EXP3}.$$
Both $CUT3$ and $CUT4$ are easy to eliminate.
($CUT3$ can be replaced by $\{a/c\}R_1$ and
$CUT4$ can be replaced by $R_1$, with appropriate
changes to the complete mating.)
However, note that $CUT2$ appears in both sides of the merge in
$$\etrba{MERGE}{}{\{a/c\}Q[CUT3]}{}{Q[CUT4]}.$$
Let us call these two occurrences $CUT2.1$ and $CUT2.2$.
Eventually, we will want to reduce one of these cuts
in some reduced tree $Q'[CUT2.1][CUT2.2]$.  Suppose we
reduce $CUT2.1$.  Similar to the reduction of $CUT1$ above,
this will copy $CUT2.2$ so that there are $CUT2.2.1$ and
$CUT2.2.2$.  The loop is evident.

It is conceivable that we could get around this loop by
developing a notion of ``expansion DAG'' (directed acyclic graph)
so that the CUT would not actually be duplicated.  But it isn't
clear that this would eliminate all such loops.  

Also, there are
some technical problems with the reductions above.  For instance,
a selection node may get copied, which means the result will be
a tree with two selection nodes that use {\it the same} selected
variable.  This doesn't seem to be a serious problem because we
could probably allow such a situation whenever the least common
ancestor of two such selection nodes is a MERGE node.  But these
details would have to be worked out to make this approach work.

\subsection{Cut and Mix Elimination in this Sequent Calculus}\label{ftree-seq-mix-elim}

There is a cut elimination algorithm implemented for the
sequent calculus described in section~\ref{ftree-seq}.
Suppose we have an instance
of the cut rule:

$$\ibnc{\above{\DD_1}{\Gamma_1\rightarrow A,\Delta_1}}{\above{\DD_2}{A,\Gamma_2\rightarrow \Delta_2}}{\Gamma_1,\Gamma_2\rightarrow \Delta_1,\Delta_2}{cut}$$

A cut elimination algorithm should take cut-free derivations $\DD_1$
and $\DD_2$ and return a cut-free derivation $\EE$ of 
$\Gamma_1,\Gamma_2\rightarrow\Delta_1,\Delta_2$.
This is usually done by a recursive algorithm on the two
derivations where at each step either one of the derivations
gets smaller, or the cut formula gets smaller (while the
derivations may get much larger).  Since we have an explicit
$merge$ (contraction) rule, we have the following problematic case.
Suppose $\DD_1$ ends with an application of $merge$:
$$\ianc{\above{\DD_3}{\Gamma_1\rightarrow A,A,\Delta_1}}{\Gamma_1\rightarrow A,\Delta_1}{merge-}$$
where $A$ is the cut formula.  Also, suppose $A$ is the principal formula of $\DD_2$.
The most natural way to handle this case is to first perform cut elimination
on $\DD_3$ and $\DD_2$, giving a cut-free derivation
$$\above{\DD_5}{\Gamma_1,\Gamma_2\rightarrow A,\Delta_1,\Delta_2}$$
Then we could call cut elimination again with $\DD_5$ and $\DD_2$
to obtain a cut-free derivation
$$\above{\DD_6}{\Gamma_1,\Gamma_2,\Gamma_2\rightarrow \Delta_1,\Delta_2,\Delta_2}$$
With some applications of $focus$ and $merge$ to shuffle and merge the formulas,
we would have the cut-free derivation of
$$\above{\DD_7}{\Gamma_1,\Gamma_2\rightarrow \Delta_1,\Delta_2}$$
we desire.  {\it However,} the derivation $\DD_5$ will in general be
bigger than $\DD_3$, while the cut formula $A$ remains the same.
So, this recursive call to cut elimination may not terminate (even with
a first-order derivation). % closed-thm1.prf gives an example

We can get around this problem by performing mix-elimination 
(as did Gentzen ~\cite{Gentzen69}) instead of cut-elimination.
Because we care about the order of formulas in the sequent,
mix elimination is complicated.  The main functions are
\indexfunction{ftree-seq-mix-elim-1} and \indexfunction{ftree-seq-mix-elim-principal}.
To understand mix elimination, 
first consider a few generic examples:


Suppose we have two cut-free derivations of
$$B\rightarrow A,C$$
and
$$D,A\rightarrow E$$
where $A$ is the mix formula, in negative position $0$ in the
first sequent and positive position $1$ in the second sequent.
Mix elimination might return a cut-free derivation of
$$B,D\rightarrow C,E$$
along with two lists of indices
\begin{enumerate}
\item $((t\; . \; 0) \; (nil\; . \; 0))$ (indicating that $B$ corresponds
to the $0$'th positive formula of the first sequent, and
$D$ corresponds to the $0$'th positive formula of the second sequent)
\item $((t\; . \; 1) \; (nil\; . \; 0))$ (indicating that $C$ corresponds
to negative position $1$ in the first sequent and
$E$ corresponds to negative position $0$ in the second sequent)
\end{enumerate}
We say ``might'' because the return value depends, of course,
on the given derivations, not just the sequents.
Other possible outputs include a cut-free derivation of
$$B\rightarrow C,C$$
with two lists of indices
\begin{enumerate}
\item $((t\; . \; 0))$ (again, $B$ corresponds to positive position $0$ in
the first sequent), and
\item $((t\; . \; 1)\; (t\; . \; 1))$ (both occurrences of $C$ correspond
to negative position $1$ in the first sequent).
\end{enumerate}

The point is that we have eliminated the two mix formulas
(two occurrences of $A$) and retain only residues of the other
formulas in the two given sequents.  The other formulas may occur
several times, or not at all.  The lists of indices indicate
where the formulas originally occured.  An ``index'' is a pair
$(<bool>\; . \; <nat>)$ where
\begin{itemize}
\item If $<bool>$ is $t$, the formula is the residue of a formula
in the first sequent.
\item If $<bool>$ is $nil$, the formula is the residue of a formula
in the second sequent.
\item $<nat>$ is a natural number indicating the position of the
formula in the first or second sequent.
\end{itemize}

For another example, given two cut-free derivations of
$$B,C\rightarrow A,A,D$$
and
$$E,A,F\rightarrow G$$
with mix formulas
might return
\begin{itemize}
\item
\begin{itemize}
\item a cut-free derivation of $C,F,E\rightarrow G,D$,
\item positive indices $((t\; .\; 1)\; (nil\; .\; 2)\; (nil \; . \; 0))$,
and
\item negative indices $((nil\; . \;0)\; (t\; . \; 2))$
\end{itemize}
\item or,
\begin{itemize}
\item a cut-free derivation of $B,B\rightarrow G$,
\item positive indices $((t\; . \;0)\; (t\; . \; 0))$, and
\item negative indices $((nil \; . \; 0))$
\end{itemize}
\item or,
\begin{itemize}
\item a cut-free derivation of $F,E\rightarrow D,D$,
\item positive indices $((nil\; . \; 2)\; (nil\; . \; 0))$, and
\item negative indices $((t\; . \; 2)\; (t \; . \; 2))$.
\end{itemize}
\end{itemize}

In general, we start with two derivations
$$\above{\DD_1}{\Gamma_1\rightarrow \Delta_1}$$
and
$$\above{\DD_2}{\Gamma_2\rightarrow \Delta_2}$$
and two lists $(i_1\cdots i_k)$ and $(j_1\cdots j_l)$
of natural numbers.  The natural numbers give us
the positions of the mix formulas in $\Delta_1$ and $\Gamma_2$.
Let $\Delta_1$ be a list
$(A_1 \cdots A_n)$ and $\Gamma_2$ be a list
$(B_1 \cdots B_m)$.  The mix formulas are
$A_{i_r}$ and $B_{j_s}$.  These formulas should have
a common reduct with respect to $\lambda$-reduction,
expanding abbreviations, and expansion equalities
using either extensionality or Leibniz.
Mix elimination returns a cut-free derivation of
$$\Gamma\rightarrow\Delta$$
and two lists of $indl^+$ and $indl^-$
of indices indicating the preimage of
the formulas in $\Gamma$ and $\Delta$.
$\Gamma$ and $indl^+$ are lists of the same length.
For each $B$ in $\Gamma$ there is a corresponding
index $(b . j)$ where either $b$ is $t$
and $j$ is the position of $B$ in $\Gamma_1$
or $b$ is $nil$ and $j\not\in\{j_1,\ldots, j_l\}$ is the position of
$B$ in $\Gamma_2$.
Similarly, $\Delta$ and $indl^-$ have the same length.
For each $A$ in $\Delta$ there is a corresponding index
$(a . i)$ where either $a$ is $t$ and $i\not\in\{i_1,\ldots, i_k\}$ is the position
of $A$ in $\Delta_2$
or $a$ is $nil$ and $i$ is the position of $A$ in $\Delta_2$.

The main function which attempts to eliminate all cuts from
a derivation is \indexfunction{ftree-seq-cut-elim}.
This is a simple recursive algorithm which first eliminates
cuts from premisses, then either imitates the rule
or, if the rule was $cut$, uses mix elimination to obtain
a cut-free derivation from the cut-free derivations of the
premisses.  That is,
given two derivations
$$\above{\DD_1}{\Gamma_1\rightarrow A,\Delta_1}$$
and
$$\above{\DD_2}{A,\Gamma_2\rightarrow \Delta_2}$$
we call mix elimination with these derivations
and two lists of positions $(0)$ and $(0)$.
Mix elimination returns a cut-free derivation of
$\Gamma\rightarrow\Delta$
and two lists of indices.
Using the indices and the $focus$, $weaken$, and $merge$ rules, we can
shuffle the formulas in $\Gamma$ and $\Delta$
to obtain a derivation of
$\Gamma_1,\Gamma_2\rightarrow \Delta_1,\Delta_2$.
This final shuffling is performed by
\indexfunction{ftree-seq-mix-elim-finish}.

\subsection{The Mix Elimination Algorithm}

The mix elimination algorithm works by recursion on the two
given derivations.  Suppose we are given two cut-free derivations
$$\above{\DD_1}{\Gamma_1\rightarrow \Delta_1}$$
and
$$\above{\DD_2}{\Gamma_2\rightarrow \Delta_2}$$
and two lists $nl_1 = (i_1\cdots i_k)$ and $nl_2 = (j_1\cdots j_l)$
indicating the positions of the mix formulas in $\Delta_1$
and $\Gamma_2$.

There are several cases to consider:

\begin{itemize}
\item {\bf Mix Formula Does Not Occur}.  That is, $nl_1 = nil$ or $nl_2 = nil$.
If $nl_1$ is $nil$, then we can return
\begin{itemize}
\item $$\above{\DD_1}{\Gamma_1\rightarrow\Delta_1}$$
\item $((t\; .\; 0)\; \cdots \; (t\; . \; (n - 1)))$ where $n$ is the length of $\Gamma_1$
\item $((t\; .\; 0)\; \cdots \; (t\; . \; (m - 1)))$ where $m$ is the length of $\Delta_1$
\end{itemize}
The indices indicate that every formula is the residual of a formula from
the first sequent.
If $nl_2$ is $nil$, we can return $\DD_2$ and indices which indicate
all the formulas are residuals of the second sequent.
\item {\bf INIT}  Suppose $\DD_1$ is an initial sequent $A\rightarrow A$.
($A$ must be the mix formula, or $nl_1$ must be $nil$.)
Ideally, we would like to return $\DD_2$, and replace the indices
for the mix formula occurrences in $\Gamma_2$ by the index $(t\; .\; 0)$
indicating the positive $A$ from $\DD_1$.
The only problem is that each occurrence of the mix formula in
$\Gamma_2$ is a $B$ equal to $A$ only in the sense that the
two formulas have a common reduct with respect to
$\lambda$-reduction
and expansions of abbreviations and equalities.
The function \indexfunction{ftree-seq-replace-equivwffs} replaces
each such $B$ by $A$ in $\DD_2$.
We can handle $\DD_2$ initial similarly.
\item {\bf FOCUS, WEAKEN, MERGE}  If either $\DD_1$ or $\DD_2$ is
a $focus$, $weaken$, or $merge$ step, we can simply recursively
call with the premiss and the positions shuffled appropriately.
\item {\bf REWRITES}  If either $\DD_1$ or $\DD_2$
is a $\lambda$, $EQUIVWFFS$, $Leibniz=$, or $Ext=$ rewrite,
and the mix formula is the principal formula, we can usually simply
recursively call the algorithm with the premiss and the same positions.
This is possible because we are only maintaining that the mix formulas
are the same up to having a common reduct via such rewriting steps.
The actual rewriting is done in the INIT step.

{\bf Exception:  Nonanalytic Uses of Extensionality}  Suppose one
occurrence of the mix formula is $A[=]$
and another occurrence of the mix formula is 
$A[\forall q\; . \; q \; M_\greeko \;\limplies\; q\; N_\greeko].$
If $A[=]$ is the principal formula of an $Ext=$ rewrite, then 
$$A[\equiv]$$ and $$A[\forall q\; . \; q \; M_\greeko \;\limplies\; q\; N_\greeko]$$
{\it do not} have a common reduct.  The recursion fails at such a step.
In general this problem occurs when one instance of equality
is expanded as Leibniz and another corresponding instance is
expanded as (functional or propositional) extensionality.
The real problem is that the extensionality rules by themselves
without cut
do not give a complete calculus for extensional higher-order logic.
To get a cut-free extensional calculus, one needs to be able to
have initial sequents (i.e., ``mate'') {\em modulo equations}
and then use decomposition rules and extensionality rules to
solve the introduced equations.
Benzmuller's thesis gives rules for handling extensionality
in the context of resolution.
The case for sequent calculi should be described in upcoming technical reports by
Benzmuller, Brown and Kohlhase.
There should also be information in Chad E. Brown's thesis.
\item {\bf NONPRINCIPAL LOGICAL RULE}  If either $\DD_1$ or $\DD_2$
ends with a logical rule and the principal formula is not a mix formula,
then we can recursively call mix elimination on the premisses and
imitate the rule.  For example, if $\DD_1$ is
$$\ibnc{\above{\DD_{11}}{\Gamma_{11}\rightarrow A,\Delta_{11}}}{\above{\DD_{12}}{\Gamma_{12}\rightarrow B,\Delta_{12}}}{\Gamma_{11},\Gamma_{12}\rightarrow A\land B,\Delta_{11},\Delta_{12}}{\land-}$$
the function \indexfunction{ftree-seq-invert-position-list}
uses the positions of the mix formulas in
$$A\land B,\Delta_{11},\Delta_{12}$$
to find the positions of the mix formulas in
$\Delta_{11}$ and $\Delta_{12}$.
Using this we can call mix elimination twice to obtain
$$\above{\DD_3}{\Gamma_3\rightarrow\Delta_3}$$
and
$$\above{\DD_4}{\Gamma_4\rightarrow\Delta_4}$$
The function \indexfunction{ftree-seq-mix-elim-imitate-rule}
finishes this case.  First, \indexfunction{ftree-seq-bring-to-front} uses
structural rules to bring the residuals of $A$ and $B$ to the front of
$\Delta_3$ and $\Delta_4$, so we have
$$\above{\DD'_3}{\Gamma_3\rightarrow A,\Delta'_3}$$
and
$$\above{\DD'_4}{\Gamma_4\rightarrow B,\Delta'_4}$$
At this point, we apply the $\land-$ rule:
$$\ibnc{\above{\DD'_{3}}{\Gamma_3\rightarrow A,\Delta'_3}}{\above{\DD'_4}{\Gamma_4\rightarrow B,\Delta'_4}}{\Gamma_3,\Gamma_4\rightarrow A\land B,\Delta'_3,\Delta'_4}{\land-}$$
and compute the indices of the residuals.
\item {\bf PRINCIPAL} The final, and most important case, is when both mix formulas
are principal.  This is handled by the function
\indexfunction{ftree-seq-mix-elim-principal}.
\end{itemize}

In each of the cases above, one of the derivations gets smaller
in the recursive call.
The case when both mix formulas are principal is complicated
by the need to perform several recursive calls, requiring us
to adjust and compose the indices as we go along.  The cases
for each connective and quantifier are described next.

\begin{itemize}
\item $\top$,$\bot$:  This cannot happen, because the only
way $\top$ can be positive principal in $\DD_2$ is if the
last step is $focus$, $weaken$, or $merge$, which were handled
above.  Similarly, if $\bot$ is negative principal in $\DD_1$,
then it must end with a $focus$, $weaken$, or $merge$.
\item $REW(\equiv)$:  $\DD_1$ is
$$\ianc{\above{\DD_{11}}{\Gamma_1\rightarrow [A\limplies B]\land [B\limplies A],\Delta_1}}{\Gamma_1\rightarrow A\equiv B,\Delta_1}{REW(\equiv)-}$$
$\DD_2$ is
$$\ianc{\above{\DD_{21}}{[A\limplies B]\land [B\limplies A],\Gamma_2\rightarrow \Delta_2}}{A\equiv B,\Gamma_2\rightarrow \Delta_2}{REW(\equiv)+}$$
First we recursively call mix elimination for all the unexpanded formulas $A'\equiv B'$
with $\DD_1$ and $\DD_{21}$ (smaller than $\DD_2$)
giving $\DD_3$:
$$[A\limplies B]\land [B\limplies A],\Gamma_3\rightarrow \Delta_3$$
Next, recursively call mix elimination for all the unexpanded formulas $A'\equiv B'$ with $\DD_{11}$ (smaller than $\DD_1$) and $\DD_2$
giving $\DD_4$:
$$\Gamma_4\rightarrow [A\limplies B]\land [B\limplies A],\Delta_4$$
Finally, we can call mix elimination for the two occurrences of the
``smaller'' mix formula $[A\limplies B]\land [B\limplies A]$
with $\DD_4$ and $\DD_3$ 
giving
$\DD_5$:
$$\Gamma_5\rightarrow \Delta_5$$
Of course, with if we weigh $\equiv$ more than $\land$ and $\limplies$,
we can say $[A\limplies B]\land [B\limplies A]$ is ``smaller'' than $A\equiv B$.
So, this case does not cause a problem with termination.
The hard part is tracing the residuals to return the proper indices.
Each formula in $C$ in $\Gamma_5$ is either the residual of some
$C$ in $\Gamma_4$ or $C$ in $\Gamma_3$.  If the preimage is in $\Gamma_4$,
then $C$ is a residual of a $C$ in $\Gamma_1$ or $\Gamma_2$.  Once
we compute the preimage of the preimage, we have the proper index.

The following diagram is helpful when trying to compute preimages:
$$\ibnc{\ibnc{\DD_{11}}{\DD_2}{\DD_4}{elim}}{\ibnc{\DD_1}{\DD_{21}}{\DD_3}{elim}}{\DD_5}{elim}$$
\item $EXP-,SEL+$ or $SEL-,EXP+$:  Suppose $\DD_1$ is
$$\ianc{\above{\DD_{11}}{\Gamma_1\rightarrow A(t),\Delta_1}}{\Gamma_1\rightarrow \exists x A(x),\Delta_1}{EXP-^t}$$
and $\DD_2$ is
$$\ianc{\above{\DD_{21}}{A(a),\Gamma_2\rightarrow \Delta_2}}{\exists x A(x),\Gamma_2\rightarrow \Delta_2}{SEL+^a}$$
As described in the $REW(\equiv)$ case,
we must first eliminate all the nonprincipal occurrences
of the mix formulas by recursive calls using $\DD_1$ and $\DD_{21}$
to obtain $\DD_3(a)$:
$$A(a),\Gamma_3\rightarrow \Delta_3$$
and with $\DD_{11}$ and $\DD_2$ to obtain $\DD_4$:
$$\Gamma_4\rightarrow A(t),\Delta_4$$
Then we substitute $t$ for $a$ in $\DD_3(a)$ to obtain $\DD_3(t)$,
and recursively call to eliminate the two occurrences of
$A(t)$ using $\DD_4$ and $\DD_3(t)$.

{\bf Remark About Termination:}  Usually $A(t)$ will be ``smaller''
than $\exists x A(x)$.  But, of course, in higher order logic there
are cases where the formula is most certainly not smaller.
The most obvious example is when $A$ is $\exists \; x_\greeko \; x$
and $t$ is also $\exists \; x_\greeko\; x$, so that
$A(t)$ is actually the same as $\exists x A(x)$.
We don't actually know whether the algorithm always
terminates.

As in the $REW(\equiv)$ case,
computing the indices involves computing preimages of preimages
using the following diagram
$$\ibnc{\ibnc{\DD_{11}}{\DD_2}{\DD_4}{elim}}{\ibnc{\DD_1}{\DD_{21}}{\DD_3}{elim}}{\DD_5}{elim}$$
The $SEL-,EXP+$ case is similar.
\item $\lnot$:  This case is relatively simple, we first make two
recursive calls to eliminate the nonprincipal occurrences of the
mix formula $\lnot A$ giving $\DD_3$:
$$\Gamma_3\rightarrow A,\Delta_3$$
and $\DD_4$:
$$A,\Gamma_4\rightarrow \Delta_4$$
Finally, we eliminate the two occurrences of $A$ (smaller than $\lnot A$)
in $\DD_3$ and $\DD_4$ (the order is the opposite of the previous cases).
Again, the indices are preimages of preimages, though we must
be careful to account for the changes in lengths of the two sides
as $\lnot A$ in the final sequents and $A$ in the premisses are on opposite
sides.  The diagram that applies here is
$$\ibnc{\ibnc{\DD_1}{\DD_{21}}{\DD_3}{elim}}{\ibnc{\DD_1}{\DD_{21}}{\DD_4}{elim}}{\DD_5}{elim}$$
\item $\land$:  There are three relevant premisses here.
$\DD_1$ has the form
$$\ibnc{\above{\DD_{11}}{\Gamma_{11}\rightarrow A,\Delta_{11}}}{\above{\DD_{12}}{\Gamma_{12}\rightarrow B,\Delta_{12}}}{\Gamma_{11},\Gamma_{12}\rightarrow A\land B,\Delta_{11},\Delta_{12}}{\land-}$$
and $\DD_2$ has the form
$$\ianc{\above{\DD_{21}}{A,B,\Gamma\rightarrow \Delta}}{A\land B,\Gamma\rightarrow \Delta}{\land+}$$
Recursive calls eliminate the nonprincipal occurrences of the
mix formula.
\begin{itemize}
\item $\DD_3$: ($\DD_1$ mix $\DD_{21}$ to eliminate $A\land B$) 
$\Gamma_3\rightarrow\Delta_3$
where $\Gamma_3$ contains residuals of $A$ and $B$.
\item $\DD_4$: ($\DD_{11}$ mix $\DD_2$ to eliminate $A\land B$) $\Gamma_4\rightarrow \Delta_4$
where $\Delta_4$ contains residuals of $A$
\item $\DD_5$: ($\DD_{12}$ mix $\DD_2$ to eliminate $A\land B$) $\Gamma_5\rightarrow \Delta_5$
where $\Delta_5$ contains residuals of $B$
\item $\DD_6$: ($\DD_5$ mix $\DD_3$ to eliminate residuals of $B$)
$\Gamma_6\rightarrow\Delta_6$
where $\Gamma_6$ contains residuals of $A$
\item $\DD_7$: ($\DD_4$ mix $\DD_6$ to eliminate residuals of $A$)
$\Gamma_7\rightarrow\Delta_7$
where there are no residuals of $A$, $B$, or $A\land B$.
\end{itemize}
Since there were more recursive calls in this case, we must
compute preimages of preimages of preimages in some cases,
and preimages of preimages in other cases, as indicated by the following
diagram:
$$\ibnc{\ibnc{\DD_{11}}{\DD_2}{\DD_4}{}}
{\ibnc{\ibnc{\DD_{12}}{\DD_2}{\DD_5}{}}{\ibnc{\DD_1}{\DD_{21}}{\DD_3}{}}
{\DD_6}{}}{\DD_7}{}$$
Each wff in the sequent proven by $\DD_7$ can be traced back to
a non-mix formula in either $\DD_1$ or $\DD_2$ using this diagram.
\item $\lor$:  This case is similar to the $\land$ case, but $\DD_1$
has one premiss and $\DD_2$ has two premisses.   $\DD_1$ is
$$\ianc{\above{\DD_{11}}{\Gamma\rightarrow A,B,\Delta}}{\Gamma\rightarrow A\lor B,\Delta}{\lor-}$$
and $\DD_2$ is
$$\ibnc{\above{\DD_{21}}{A,\Gamma_1\rightarrow \Delta_1}}{\above{\DD_{22}}{B,\Gamma_2\rightarrow \Delta_2}}{A\lor B,\Gamma_1,\Gamma_2\rightarrow \Delta_1,\Delta_2}{\lor+}$$
The following diagram indicates the order of the recursive calls.
$$\ibnc{\ibnc{\ibnc{\DD_{11}}{\DD_2}{\DD_4}{}}{\ibnc{\DD_1}{\DD_{22}}{\DD_5}{}}
{\DD_6}{}}{\ibnc{\DD_1}{\DD_{21}}{\DD_3}{}}{\DD_7}{}$$
\item $\limplies$:
This is similar to the $\lor$ case.  The following diagram
indicates the order of the recursive calls.
$$\ibnc{\ibnc{\DD_1}{\DD_{21}}{\DD_3}{}}
{\ibnc{\ibnc{\DD_{11}}{\DD_2}{\DD_4}{}}{\ibnc{\DD_1}{\DD_{22}}{\DD_5}{}}{\DD_6}{}}
{\DD_7}{}$$
\end{itemize}

