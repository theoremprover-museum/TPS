\chapter{Events}
\label{events}


% \comment{\chapter{Teaching Records}\label{Teach}}
% {\bf Make revisions to \$progdoc/teach.mss}
% % \comment{EVENT-SIGNAL EVENTS REPORT}

\section{Events in TPS}


The primary purpose of events in {\TPS} is to collect information about
the usage of the system.  That includes support of features such as
automatic grading of exercises, keeping statistics on the application of
inference rules, being informed about bugs in the system,
and recording remarks made by the users of {\TPS}.  Other events are
used to print informative messages to the user during mating search.

Events, once defined and initialized, can be signalled from anywhere in
{\TPS}.  Settings of flags, ordinarily collected into modes, control if,
when, and where signalled events are recorded.

In {\ETPS}, a basic set of events is predefined, and the events are signalled
automatically whenever appropriate.  Whether these events are then recorded
depends on your {\ETPS} profile.

There are some restrictions on events that should be respected, if
you plan to use {\tt REPORT} to extract statistics from the files recording
events.  Most importantly: {\bf No two events should be written to the
same file}.  If you would like to record different things into the
same file, make one event with one template and allow several kinds of
occurrences of the event.  For an example, see the event {\tt PROOF-ACTION}
below.

\subsection{Defining an Event}

If you are using {\ETPS}, it is unlikely that you need to define an event
yourself.  However, a lot of general information about events is given
in the following description.

Events are defined using the {\tt DEFEVENT} macro.
Its format is

\begin{tpsexample}
(defevent {\it name}
  (event-args {\it arg1} ... {\it argn})
  (template {\it list})
  (template-names {\it list})
  (signal-hook {\it hook-function})
  (write-when {\it write-when})
  (write-file {\it file-parameter})
  (write-hook {\it hook-function})
  (mhelp {\it help-string}))
\end{tpsexample}

\begin{description}
\item[{\tt event-args} ]	  list of arguments passed on by {\tt SIGNAL-EVENT} for any event
	of this kind.

\item[{\tt template} ]	  constructs the list to be written.
        It is not assumed that every event is
	time-stamped or has the user-id.  The template
        must only contain globally evaluable forms and the arguments
	of the particular event signalled.  It could be the source of
        subtle bugs, if some variables are not declared special.

\item[{\tt template-names} ]	  names for the individual entries in the template.
These names are used by the {\tt REPORT} facility.  As general conventions,
when the template form is a variable, use the same name for the
template name (e.g. {\tt DPROOF}).  If the template form is {\tt (STATUS {\it statusfn})}
use {\it statusfn} as the template name (e.g. {\tt DATE} for {\tt (STATUS DATE)} or
{\tt USERID} for {\tt (STATUS USERID)}).

\item[{\tt signal-hook} ]	  an optional function to be called whenever the
	the event is signalled.  This should {\bf not} do the writing of
	the information, but may be used to do something else.  If the
        function does a {\tt THROWFAIL}, the calling {\tt SIGNAL-EVENT} will
        return {\tt NIL}, which means failure of the event.  The arguments
        of the function should be the same as {\tt EVENT-ARGS}.

\item[{\tt write-when} ]	  one of {\tt IMMEDIATE}, {\tt NEVER}, or an integer {\it n}, which means
     to write after an implementation dependent period of {\it n}.
     At the moment this will write, whenever the number of inputs = {\it n}
     * {\tt EVENT-CYCLE}, where {\tt EVENT-CYCLE} is a global variable, say 5.

\item[{\tt write-file} ]	  the name of the global {\tt FLAG} with the filename of the
     file for the message to be appended to.

\item[{\tt write-hook} ]	  an optional function to be called whenever a number
	(>0) of events are written.  Its first argument is the file it will
        write to, if the write-hook returns.  Its second argument is the
        list of evaluated templates to be written.  If an event is to be
        written immediately, this will always be a list of length 1.

\item[{\tt mhelp} ]	  The mhelp string for the event.
\end{description}

Remember that an event is ignored, until {\tt (INIT-EVENTS)} or {\tt (INIT-EVENT
{\it event})} has been called.

\subsection{Signalling Events}

{\TPS} provides a function \indexfunction{signal-event}, which takes a variable number
of arguments.  The first argument is the kind of event to be signalled,
the rest of the arguments are the event-args for this particular event.
\indexfunction{signal-event} will return {\tt T} or {\tt NIL}, depending on whether the action
to be taken in case of the event was successful or not.  Note that when
an event is disabled (see below), signalling the event will always be
successful.  There are basically three cases in which an event will be
considered unsuccessful: if the {\tt SIGNAL-HOOK} is specified and does a
{\tt THROWFAIL}, if {\tt WRITE-WHEN} is {\tt IMMEDIATE} and either the {\tt WRITE-HOOK}
(if specified) does a {\tt THROWFAIL}, or if for some reason the writing to
the file fails (if the file does not exists, or is not accessible
because it has the wrong protection, for example).

It is the caller's responsibility to make use of the returned value of
\indexfunction{signal-event}.  For example, the signalling of {\tt DONE-EXERCISE} below.

If {\tt WRITE-WHEN} is a number, the evaluated templates will be collected
into a list {\it event{\tt -LIST}}.  This list is periodically written out and
cleared.  The interval is determined by \indexflag{EVENT-CYCLE}, a global flag
(see description of {\tt WRITE-WHEN} above).  The list is also written out
when the function {\tt EXIT} is called, but not if the user exits {\TPS} with
{\tt \^C}.  Note that if events have been signalled, the writing is done
without considering whether the event is disabled or not.  This ensures
that events signalled are always recorded, except for the {\tt \^C} safety valve.

Events may be disabled, which means that signalling them will always
be successful, but will not lead to a recordable entry.  This is done
by setting or binding the flag {\it event{\tt -ENABLED}} to {\tt NIL} (initially
set to {\tt T}).  For example, the line {\tt (setq error-enabled nil)}
in your {\tt .INI} file will make sure that no Lisp error will be recorded.
For a maintainer using expert mode, this is probably a good idea.

\subsection{Examples}

Here are some examples taken from the file {\tt ETPS-EVENTS}.  Interspersed
is also the code from the places where the events are signalled.

\begin{tpsexample}

(defflag error-file
  (flagtype filespec)
  (default ((tpsrec *) etps error))
  (subjects events)
  (mhelp `The file recording the events of errors.'))

(defevent error
  (event-args error-args)
  (template ((status userid) (status subsys) error-args))
  (template-names (userid subsys error-args))
  (write-when immediate)
  (write-file error-file)    ; a global variable, eg
			     ; `((tpsrec: *) etps error)
  (signal-hook count-errors) ; count errors to avoid infinite loops
  (mhelp `The event of a MacLisp Error.'))

{\rm 
{\tt DT} is used to freeze the daytime upon invocation of {\tt DONE-EXC} so that the code is computed correctly.
The code is computed by {\tt CODE-LIST}, implementing some ``trap-door function''.}

(defvar computed-code 0)

(defvar dt '(0 0 0))

(defflag score-file
  (flagtype filespec)
  (default ((tpsrec *) tps scores))
  (subjects events)
  (mhelp `The file recording completed exercises.'))

(defevent done-exc
  (event-args numberoflines)
  (template ((status userid) dproof numberoflines computed-code
			     (status date) dt))
  (template-names (userid dproof numberoflines computed-code date daytime))
  (signal-hook done-exc-hook)
  (write-when immediate)
  (write-file score-file)
  (mhelp `The event of completing an exercise.'))

(defun done-exc-hook (numberoflines)
  {\tt  The} done-exc-hook will compute the code written to the file.
  (declare (special numberoflines))
  {\tt  because} of the (eval `(list ..)) below.
  (setq dt (status daytime))
  {\tt  Freeze} the time of day right now.
  (setq computed-code 0)
  (setq computed-code (code-list (eval `(list ,j(get 'done-exc 'template))))))

(defflag proof-file
  (flagtype filespec)
  (default ((tpsrec *) tps proof))
  (subjects events)
  (mhelp `The file recording started and completed proofs.'))

(defevent proof-action
  (event-args kind)
  (template ((status userid) kind dproof
			     (status date) (status daytime)))
  (template-names (userid kind dproof date daytime))
  (write-when immediate)
  (write-file proof-file)
  (mhelp `The event of completing any proof.'))

(defflag remarks-file
  (flagtype filespec)
  (default ((tpsrec *) tps remarks))
  (subjects events)
  (mhelp `The file recording remarks.'))

(defevent remark
  (event-args remark-string)
  (template ((status userid) dproof remark-string
			     (status date) (status daytime)))
  (template-names (userid dproof remark-string date time))
  (write-when immediate)
  (write-file remarks-file)
  (mhelp `The event of a remark by the user.'))

\end{tpsexample}

Here is how the {\tt DONE-EXC} and {\tt PROOF-ACTION} are used in the code of
the {\tt DONE} command.  We don't care if the {\tt PROOF-ACTION} was successful
(it will usually be), but it's very important that the user knows
when a {\tt DONE-EXC} was unsuccessful, since it is used for automatic
grading.

\begin{tpsexample}
(defun done ()
  ...
  (when (funcall (get 'exercise 'testfn) dproof)
	{\tt  here} we have an assigned exercise.
	(do ()
	    ((signal-event 'done-exc (length (get dproof 'lines)))
	     (msgf `Score file updated.'))
	  (msgf `Could not write score file.  Trying again... (abort with \^G)')
	  {\tt  wait} for a bit, in case the problem was simultaneous access.
	  (sleep 0.5)))
  (signal-event 'proof-action 'done)
  ...
  )
\end{tpsexample}

\section{More on Events}

Each command (mexpr) may have associated with
it an EVENT-TYPE.  EVENTs could be PRINTING, INFERENCE, SYSTEM, FILEOP,
ADVICE, STARTED-PROOF, DONE-PROOF, and perhaps more.  One may define for
each event, how much information about the event is saved, and when, and
if the operation is legal, if the information could not be saved.
An event could be signalled whenever a command is executed (signal the
associated event), or from within a function (say for an error) with
the (event arg1 ... argn) LEXPR.
For example:
\begin{tpsexample}
(defevent advice
  (append-file advice-file)		;advice-file is a global var.
  (append-when immediate)		;could be IMMEDIATE, NEVER, PERIODICAL.
  (append-failed abort)			;could be ABORT, RETRY-LATER.
					;ABORT means that operation must be
					;aborted if recording of event failed.
  (append-info '(time user exercise))	;a template
  (mhelp `Event that occurs when advice is asked.'))

(defevent inference
  (append-file inference-file)
  (append-when periodical)
  (append-failed retry-later)
  (append-info '(time user exercise legal-p wrong-defaults-count))
					;legal-p and wrong-defaults-count
					; are two args supplied to EVENT
					; inside COMDECODE.
  (mhelp `Event that occurs when an inference rule is applied.'))

(defevent maclisp-error
  (append-file error-file)
  (append-when immediate)
  (append-failed retry-later)
  (append-info '(time current-command err-message))
  (mhelp `An uncaught error condition.'))

\end{tpsexample}
The {\it .proof} and {\it .scores} files, run through REPORT,
can be used to get a distribution of when people did their work, how long
the average proof was etc.

The report called SECURITY checks for lists
with the wrong security code in a source file (tps:etps.rec). To run
it:
\begin{tpsexample}
<\#>lload tn:report
<\#>report
MODE:....[nil]>             ;;hit return
<rep1>security
OUTFIL....[tty:]>
SINCE....[(85 1 1)]>
WHO......[(t *)]>
WHAT.....[(t *)]>           ;;the t says the case fold is on. * matches
                            ;;strings of any length (including zero).
                            ;;? matches a single character.
T
<rep2>on security
<rep3>run
....
<rep4>exitrep
<\#>
\end{tpsexample}
REMARK, GRADE-FILE and EXER-TABLE also work in a similar manner.

Remarks are sent to the file {\it etps.remarks} (in addition to
{\it etps.rec})

\begin{enumerate}
\item A category of EVENT.  Every event has a few properties:
\begin{description}
\item[MHELP]	 the obvious

\item[EVENT-ARGS]	 list of arguments passed on by SIGNAL for any event
	of this kind.

\item[TEMPLATE]	 constructs the list to be written.  Contrary to what
	we had before, we will not assume that every event is
	time-stamped or has the user-id.  The template
        must only contain globally evaluable forms and the arguments
	of the particular event signalled.

\item[WRITE-WHEN]	 one of IMMEDIATE, NEVER, or an integer n, which means
    	write after a period of n inputs.

\item[WRITE-FILE]	 the filename of the file for the message to be
        appended to.

\item[SIGNAL-HOOK]	 an optional function to be called whenever the
	the event is signalled.  This should NOT to the writing of
	the information, but may be used to do something else.

\item[WRITE-HOOK]	 an optional function to be called whenever a number
	(>0) of events are written.
\end{description}

\item A macro or function SIGNAL-EVENT, whose first argument is the
    kind of even to be signalled, the rest of the arguments are the
    event-args for this particular event.  SIGNAL-EVENT will return
    T or NIL, depending on whether the action to be taken in case of
    the even was successful or not.  It is the caller's responsibility
    to act accordingly.  E.g. if (SIGNAL-EVENT COSTLY-ADVICE 'X2106)
    returns NIL, the advice should not be given (of course at the moment
    we don't charge for advice).
\end{enumerate}
Examples:
\begin{tpsexample}
(defevent maclisp-error
  (event-args error-args)
  (template ((status userid) dproof current-command error-args))
  (write-when 5)
  (write-file error-file)    ; a global variable, eg
			     ; `((tpsrec: *) etps error)
  (signal-hook count-errors) ; count errors to avoid infinite loops
  (mhelp `The event of a MacLisp Error.'))

(defevent tps-complain
  (event-args complain-msglist)
  (template ((status userid) complain-msglist))
  (write-when 10)
  (write-file complain-file)
  (mhelp `The event of an error message given by TPS.'))

{\rm The event tps-complain could be ``hard-wired'' into the COMPLAIN macro, so that every time COMPLAIN is
executed, the event is signalled.}

(defevent advice-asked
  (event-args)
  (template ((status userid) dproof)
  (write-when immediate)
  (write-file advice-file)
  (mhelp `Event of user asking for advice.'))

\end{tpsexample}

The definition of EVENTS now includes TEMPLATE-NAMES, which is
a list for the entries in the events.  some general conventions:
(status userid) => userid
(status date)   => date
(status daytime)=> daytime
If an event-arg appears directly in the template, use that same name as
the TEMPLATE-NAME.

% \begin{comment}
% Is there a way of disabling certain events, that is, to stop recording
% the information which an event would otherwise record, without having
% to rebuild the image (and leaving the undesired events undefined)?
% Do we have a variable indicating which events are enabled?
% [This may be another property of a process (instance of a top-level)]
% \end{comment}
\section{The Report Package}

The REPORT package in {\TPS} allows the processing of data
from EVENTS. Each report draws on a single event, reading
its data from the record-file of that event. The execution
of a report begins with its BEGIN-FN being run. Then
the DO-FN is called repetitively on the value of the EVENTARGS
in each record from the record-file of the event, until that
file is exhausted or the special variable DO-STOP is given a non-NIL
value. Finally, the END-FN is called. The arguments
for the report command are given to the BEGIN-FN and END-FN.
The DO-FN can only access these values if they are assigned to
certain PASSED-ARGS, in the BEGIN-FN. Also, all updated values
which need to be used by later iterations of the DO-FN or by
the END-FN should be PASSED-ARGS initialized (if the default NIL
is not acceptable in the BEGIN-FN.

NOTE: The names of PASSED-ARGS should be different from
other arguments (ARGNAMES and EVENTARGS). Also, they should
be different from other variables in those functions where
you use them and from the variables which DEFREPORT2 always
introduces into the function for the report: FILE, INP and DO-STOP.

The definition of the category of REPORTCMD, follows:

\begin{tpsexample}

(defcategory reportcmd
  (define defreport1)	      ; DEFREPORT defines a function and a command
  (properties                 ; (MEXPR), as well as a REPORTCMD.
   (source-event single)
   (eventargs multiple)   ;; selected variables in the var-template of event
   (argnames multiple)
   (argtypes multiple)
   (arghelp multiple)
   (passed-args multiple) ;; values needed by DO-FN (init in BEGIN-FN)
   (defaultfns multiple)
   (begin-fn single)    ;; args = argnames
   (do-fn single)       ;; args = eventargs
   (end-fn single)      ;; args = argnames
   (mhelp single))
  (global-list global-reportlist)
  (mhelp-line `report')
  (mhelp-fn princ-mhelp)
  (cat-help `A task to be done by REPORT.'))

\end{tpsexample}

	The creation of a new report consists of a DEFREPORT statement
and the definition of the BEGIN-FN, DO-FN and END-FN. Any PASSED-ARGS
used in these functions should be declared special. It is suggested
that most of the computation be done by general functions which are more
readily usable by other reports. In keeping with this philosophy,
the report EXER-TABLE uses the general function MAKE-TABLE. The latter
takes three arguments as input:  a list of column-indices, a list of
indexed entries (row-index, column-index, entry) and the maximum printing size
of row-indices. With these, it produces a table of the entries.
EXER-TABLE merely calls this on data it extracts from the record file
for the DONE-EXC event. The definition for EXER-TABLE follows:

\begin{tpsexample}

(defreport exer-table
  (source-event done-exc)
  (eventargs userid dproof numberoflines date)
  (argtypes date)
  (argnames since)
  (passed-args since1 bin exerlis maxnam)
  (begin-fn exertable-beg)
  (do-fn exertable-do)
  (end-fn exertable-end)
  (mhelp `Constructs table of student performance.'))

(defun exertable-beg (since)
  (declare (special since1 maxnam))	; the only passed-args initialized
  (setq since1 since)                   ; to non-Nil values
  (setq maxnam 1))

(defun exertable-do (userid dproof numberoflines date)
  (declare (special since1 bin exerlis maxnam))
  (if (greatdate date since1)
      (progn
       (setq bin (cons (list userid dproof numberoflines) bin))
       (setq exerlis
	     (if (member dproof exerlis) exerlis (cons dproof exerlis)))
       (setq maxnam (max (flatc userid) maxnam)))))

(defun exertable-end (since)
  (declare (special bin exerlis maxnam))
  (if bin
      (progn
       (make-table exerlis bin maxnam)         ;; exerlis --> column headers
       (msg t `On exercises completed since ') ;; bin --> row headers, entries
       (write-date since)                      ;; maxnam =
       (msg `.' t))                            ;; max \{size x : x a row header\}
      (progn
       (msg t `No exercises completed since ')
       (write-date since)                      ;; prints date in English
       (msg `.' t))))

\end{tpsexample}
% \begin{comment}
% \section{More on Report}
% 
% The new efficient REPORT is ready. The passing of arguments is handled
% through the PASSED-ARGS property.
% NOTE: The only arguments to the DO-FN are the EVENTARGS. If you wish
% to use the arguments from the MEXPR (i.e. the ARGNAMES), you must
% make PASSED-ARGS with different names and initialize them
% in the BEGIN-FN. Any PASSED-ARGS used should be declared special.
% Of course, this means that no PASSED-ARGS should bear the same name
% as any of the other arguments (ARGNAMES and EVENTARGS).
% 
% REPORT now uses WITH-OPEN-FILE, which is defined for MacLisp in MACSYS.
% Also, REPORT has a new special variable DO-STOP. If it is set to
% a non-NIL value, it will terminate the reading from the events file.
% 
% There is a command file and a work file which in conjunction makes it
% very easy to evaluate the students progress on TF, while logged in
% on the C.  Here is how to use it:
% 
% \begin{text}
% 
%  jFTP TF
%  ..>LOGIN user-name
%  ..>TAKE TPS:TF.CMD
% \end{text}
% 
%  Next run ETPS and do
% 
% \begin{text}
% 
%  <1>EXECUTE-FILE TPS:TF NO NUL:
%  <2>REPORT
%  <Rep1>
% \end{text}
% 
%  then type a ? for a list of available reports.  Among them is
%  EXER-TABLE and a way to get a list of remarks.
% 
% In the SECURITY report, set
% the code to 0 and include it in the list, then compute the code.
% There is a flag SINCE-DEFAULT for the DEFAULTFN
% of all the reports which take a SINCE argument.  The default date
% is (80 1 1).
% 
% Changed filename HACK to
% EVENT-SIGNAL-UTILS and made it part of a new package EVENT-SIGNAL.
% Both OTLNL and REPORT have the latter as a needed-package.
% 
%    `STATUS ' is now `STATUS-' in file TL:ETPS-EVENTS.Lsp. The 3
% macros (STATUS-USERID), (STATUS-DATE), (STATUS-DAYTIME) are defined in
% file MACSYS-3.
% 
% You changed the file TPS:ETPS-EVENTS.LSP
% and left the file TCL:ETPS-EVENTS.CLISP unchanged.  Therefore the new
% ETPS core image I built turned out to be rotten, since STATUS-USERID (etc)
% were undefined functions!
% Anyway, I changed them to be functions in TPS3 - no reason for the duplication
% of code everywhere it is called.  Also, it returns a string - the old
% function returned a list (eg. (10 30 86)), but that was an easy fix.
% 
% Should have built events into grader.
% 
% ;;; This file contains all the REMARKS sent to the instructor.
% ;;; Format: (userid dproof remark-string date daytime)
% (TK06 X5207 `string?' (86 3 14) (10 22 52))
% (PB99 ***** `just checking the remarks facility on tf.' (86 3 15) (0 26 23))
% (TK06 X5207 `
% 
% \section{Mail Messages}
% 
% I have made a command file and a work file which in conjunction make it
% very easy to evaluate the students progress on TF, while logged in
% on the C.  Here is how to use it:
% 
% \begin{tpsexample}
% jFTP TF
%  ..>LOGIN user-name
%  ..>TAKE TPS:TF.CMD
% \end{tpsexample}
% 
%  Next run ETPS and do
% 
% \begin{tpsexample}
%  <1>EXECUTE-FILE TPS:TF NO NUL:
%  <2>REPORT
%  <Rep1>
% \end{tpsexample}
% 
%  then type a ? for a list of available reports.  Among them is Carl's
%  EXER-TABLE and a way to get a list of remarks.
% 
% \heading{The Old Report Package}
% 
% 	The REPORT package provides a general framework for the running
% of reports. It is accessed through the REPORT command in BTPS. This brings
% the user into the REPORT top-level.
% 	Once in the REPORT top-level, the user establishes the parameters
% for his reports by running:
% 
% SOURCE      :sets the list of input file(s)
% 
% <reportname>:sets the parameters for <reportname>
% 
% RUN         :runs the active reports on the SOURCE file(s)
% 
% 	Reports may be disabled by using the OFF command and enabled using
% the ON command.
% 	The current command strings for each report are saved in their
% Environ properties. These may be labelled at any time using the SETMODE
% command. These modes are accessible during the remainder of the session.
% When the user EXITs REPORT, the modes and the time of the last run of
% each report are saved in the REPORT.INI file on the users account. When
% report is run again, this file is loaded and the modes are available.
% 	The full syntax of the REPORT command is:
% 	REPORT <modename>
% Since SETMODE also allows commands to be added to the ones which establish
% the mode's report-environment, REPORT can be called without any visible
% interaction with its top-level, viz. by adding RUN and EXIT to the mode.
% 
% 
% \end{comment}
re accessible during the remainder of the session.
% When the user EXITs REPORT, the modes and the time of the last run of
% each report are saved in the REPORT.INI file on the users account. When
% report is run again, this file is loaded and the modes are available.
% 	The full syntax of the REPORT command is:
% 	REPORT {\tt\char`\<}modename{\tt\char`\>}
% Since SETMODE also allows commands to be added to the ones which establish
% the mode's report-environment, REPORT can be called without any visible
% interaction with its top-level, viz. by adding RUN and EXIT to the mode.
% 
% 
% \end{comment}
