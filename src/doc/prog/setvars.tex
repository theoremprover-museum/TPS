\chapter{Set Variables}

Inductively, a set type is either the type of propositions
$\greeko$ or a function type $\alpha\beta$ where $\alpha$
is a set type and $\beta$ is any type.  A term of a set type
represents either a proposition, a set, or a relation.

When we refer to set variables, we generally mean
an expansion variable of a set type.  To find
an expansion proof with expansion variables we
may need to instantiate these variables.
Propositional variables are relatively easy to instantiate,
since there are only two possible truth values, $\top$ and $\bot$.
It is far more difficult to instantiate set variables which
are not of propositional type.  Semantically, these set types
may correspond to infinite domains.  Syntactically, we have
logical constants and quantifiers which can be used to create
terms of set types.
In some cases, but certainly not in all cases, the instantiations
can be find using higher-order unification.  For instantiations which require
logical constants or quantifiers, the original method used by
\TPS is that of PRIMSUBS (see section~\ref{setvars:primsubs}).

For references on PRIMSUBS, see ~\cite{Andrews89}, ~\cite{Andrews95b}, and ~\cite{Andrews99}.
Some work regarding instantiating set variables in other
contexts include   % put Felty's paper in logictex.bib
~\cite{Bledsoe77}, ~\cite{Bledsoe79}, ~\cite{Bledsoe83}, ~\cite{Bledsoe93}, 
and ~\cite{Bailin93}.
The SCAN algorithm (see ) % put scan papers in logictex.bib
reduces some second-order formulas
to equivalent first-order formulas, avoiding the need
to instantiate the set variables.

\section{Primitive Substitutions}\label{setvars:primsubs}

Set variables can be instantiated in a pre-processing stage
using Primitive Substitutions.  This depends on the value
of several flags.  There is a subject PRIMSUBS.
The command \verb+LIST PRIMSUBS+ in \TPS will list
the relevant flags.  Some of the main flags that determine
if and how primitive substitutions are generated are
\indexflag{DEFAULT-MS}, \indexflag{DEFAULT-EXPAND},
and \indexflag{PRIMSUB-METHOD}.

Some examples, in principle, might require applying primsubs
beneath primsubs.  An example discussed in ~\cite{Andrews00a}
is the injective version of Cantor's Theorem, {\bf X5309}.

\section{Using Unification to Compute Setsubs}\label{setvars:pr00}

If \indexflag{DEFAULT-MS} is set to \indexother{MS98-1},
\indexflag{MS98-INIT} is set to 2 or 3, and
\indexflag{PRIMSUB-METHOD} is set to \indexother{PR00},
then unification is used to compute instantiations
during pre-processing.
% Explain the difference between setsubs and primsubs + dissolution + example

\section{Set Constraints}\label{setvars:constraints}

An alternative to instantiating set variables in a pre-processing
step is to intertwine instantiating set variables with the mating search.

% Say more here - do mating search on the rigid part, then
% find set variable constraints, then solve some of these, iterate.

Let $v$ be a set variable occurring at the head of some literal.

We write constraints in sequent form.  A constraint is, in practice,
a list of positive and negative literals (or expansion tree nodes).
Usually, positive literals are written on the left of the sequent and negative
literals are written on the right.  In some cases, we write
a positive literal on the right or a negative literal on the left 
and interpret it as the negation of the literal.
A sequent corresponds to a subset of a vertical path on a jform.

Minimal constraints for $v$ are a collection of sequents (these correspond
to subsets of vertical paths in the jform) of the form
$$\Psi | \Gamma(v)\rightarrow [v\, \overline{t}]$$
where $\Gamma(v)$ is a collection of literals 
are not negative literals with $v$ at the head,
and $\Psi$ is a list of selection variables which occur
in the sequent and are banned from occurring in the instantiation
for $v$ (see section~\ref{acyclicity}).
In general, $v$ can occur inside the body of the
literals in $\Gamma$, and this case will be discussed below.
We do not allow $v$ to occur in the argument terms $\overline{t}$.

Maximal constraints can be defined and handled in a dual way. 
We concentrate on minimal constraints for the present.  % say more

It is very easy to see that any collection of minimal constraints
can be simultaneously solved, since $v = \lambda \overline{z} \top$
is a solution.  In interesting cases, this instantiation will fail
other conditions we might need $v$ to satisfy.  What we would prefer
to have is an ``optimal'' solution to the constraints.  In the case of minimal
constraints, ``optimal'' means a minimal solution.

First, consider the case of a single minimal constraint of the form
$$\Psi | \Gamma \rightarrow [v\, \overline{z}].$$
where the arguments $\overline{z}$ are distinct variables
that occur in $\Psi$.  Also, assume that there are no
other variables in $\Psi$.
In this case, we can directly define
the minimal solution as an intersection:
$$\lambda \overline{z} \bigwedge{\Gamma}.$$
The notation $\bigwedge(\Gamma)$ means
$A_1 \wedge\cdots\wedge A_n$
where each $A_j$ is either a positive literal
in $\Gamma$ or $A_j$ is $\lnot B_j$ where
$B_j$ is a negative literal in $\Gamma$.
The case where $\Psi = \overline{z},\overline{w}$
for some extra selected variables $\overline{w}$
is only slightly more complicated.  We can directly
define the minimal solution in this case as
$$\lambda \overline{z} \exists \overline{w} \bigwedge{\Gamma}.$$
This is a legal instantiation for $v$ since all the variables
in $\Psi$ are bound.

Next, consider the more general case in which the minimal constraint is of the form
$$\Psi | \Gamma \rightarrow [v\, \overline{t}]$$
where the arguments $\overline{t}$ need not be distinct 
variables from $\Psi$.  Let $n$ be the length of $\overline{t}$.
In this case, the easiest way to directly define the minimal solution
is
$$\lambda x^1\cdots \lambda x^n \exists \overline{w} . 
x^1 = t^1 \wedge \cdots \wedge x^n = t^n \wedge \bigwedge(\Gamma)$$
for new variables $x^j$ of the same type as $t^j$,
and $\overline{w} = \Psi$.
However, in practice it is easier if we distinguish between
arguments $t^j$ which are actually variables in $\Psi$ and those which are not.
So, let us write $\Psi = \overline{z},\overline{w}$
where each $z\in\overline{t}$.  
For each $j=1,\ldots, n$, 
if $t^j\in\overline{z}$ and
$t^k\neq t^j$ for $k<j$, then let $x^j = t^j$.  Otherwise,
let $x^j$ be a new variable of the same type as $t^j$.
Let $Eqns = \{x^j = t^j | t^j \mbox{ is not the variable } x^j\}$.
In this case, we can directly define the minimal solution as
$$\lambda x^1\cdots \lambda x^n \exists \overline{w} .
   \bigwedge(Eqns) \wedge \bigwedge(\Gamma).$$
Clearly, if $\overline{t}$ actually is the list $\overline{z}$, then this
solution is
$$\lambda \overline{z} \exists \overline{w} . \bigwedge(\Gamma)$$
as before.

Now, consider the case in which there are several minimal constraints of the form
$$\Psi_i | \Gamma_i \rightarrow [v\, \overline{t^i}]$$
for $i=1,\ldots,n$.  Again, we assume $v$ does not occur in $\Gamma_i$,
so we can directly define the minimal solution.
First, let $M_i$ be the minimal solution for each of the
constraints individually defined as above:
$$\lambda \overline{x_i} \exists \overline{w_i} .
   \bigwedge(Eqns_i) \wedge \bigwedge(\Gamma_i).$$
Then we take the union of these
$$\lambda \overline{x} \bigvee_i [M_i \overline{x}]$$
to get the minimal solution for the combined constraints.

Rather than diving into the general case in which $v$ may occur in $\Gamma$,
first consider a familiar example in which we have the two constraints
$$\rightarrow [v\, 0]$$
and
$$w | [v\, w] \rightarrow [v\, [S\, w]].$$
The minimal solution to this is the least set containing
$0$ and closed under the function $S$.  Since we have the
full power of higher-order logic, we can define such a solution
by
$$\lambda x \forall p \, .\, [[p\, 0]\,\land\,[\forall z\, . [v\, z] \limplies [v\, [S\, z]]]]
\,\limplies\, p x.$$

It should be clear at this point that the terms defining these
solutions can become quite large.  Making such instantiations
can be prohibitive in theorem proving, because we may need
to perform unification with the solution in some other part
of the problem.  This suggests it may be simpler to use
the instantiation to prove there is a set satisfying the
conditions we want.  Then using this lemma, the set that
exists is represented by a selected variable (a {\it very}
small term).  Another motivation for using such lemmas
is that we gain more control over what properties of the
sets are included in the lemma.
See section~\ref{exp-pf-lemmas} for more on the implementation
of expansion proofs using lemmas.

Suppose we have a general set of minimal constraints
$$\Psi_i | \Gamma_i(v)\rightarrow [v\, \overline{t_i}].$$
Consider what properties of $v$ an existence lemma should include.
The most obvious is the condition $C_i(v)$:
$$\forall \overline{w_i} . \Gamma_i(v)\rightarrow [v\, \overline{t_i}]$$
where $\Psi_i = \overline{w_i}$.  Of course, as mentioned above,
$C_i(\lambda \overline{x} \top)$, so we certainly need to include more
conditions.  Two other conditions are {\it inversion principles} and
{\it inductive principles}.  An inversion principle for the set of
constraints would be of the form
$$\forall \overline{x} . v \overline{x} \limplies D(v,\overline{x})$$
for some formula $D$.  In words, this principle says that any
element $\overline{x}$ of $v$ must be of a form satisfying $D(v,\overline{x})$.
An induction principle is a statement that the solution really is minimal.
This would be a higher-order statement of the form
$$\forall \overline{p} . \bigwedge_i(C_i'(p))\,\limplies\, v\subseteq p$$
where $C_i'$ is a condition similar to $C_i$ defined above.
The general case below will explain the difference between $C_i'$
and $C_i$.

Consider the minimal constraints
$$y | [A\, y] \rightarrow [v\, y]$$
and
$$z | [B\, z] \rightarrow [v\, z].$$
Here $C_1(u)$ is 
$$\forall y \, . \, [A\, y]\, \limplies [v\, y]$$
and $C_2(u)$ is
$$\forall z \, . \, [B\, z]\, \limplies [v\, z].$$
The inversion principle here would be
$$\forall x\, . [v\, x]\, \limplies\, .\,[A\, x]\, \lor\, [B\, x].$$
The induction principle here would be
$$\forall p\, . C_1(p)\,\land\, C_2(p)\,\limplies\, v\subseteq p.$$
In such cases where none of the minimal constraints are of the
form
$$\Psi | \Gamma(v) \rightarrow [v\, \overline{t}]$$
where $v$ {\it does} occur in $\Gamma$,
the set is fully determined by the conjunction of the
$C_i$ constraints and the inversion principle, since these
amount to an extensional definition of $v$.
In the general case, the induction principle determines the set.
Though the inversion principle follows from the induction principle,
it may be helpful to have the inversion principle when we are
trying to use the lemma to prove the theorem.
The flag \indexflag{INCLUDE-INDUCTION-PRINCIPLE} controls
whether the induction principle is included in set existence lemmas.% at least for now!

Let us return to the familiar example of the constraints
$$\rightarrow [v\, 0]$$
and
$$w | [v\, w] \rightarrow [v\, [S\, w]].$$
In this case, $C_1(v)$ is $[v\, 0]$, and $C_2(v)$ is
$\forall w\,.\, [v\, w]\,\limplies\, [v\, [S\, w]]$.
The inversion principle is the familiar statement
$$\forall x\,.\, [v\, x]\,\limplies\, [[x = 0]\, \lor\,
\exists w\,.\, [x = [S\, w]]\,\land\, [v\, w]].$$
The induction principle is the equally familiar statement
$$\forall p\,.\, [[p\, 0] \land [\forall w\,.\, [p\, w]\,\limplies\, [p\, [S\, w]]]]\,
\limplies \forall w \, .\, [v\, w]\,\limplies\, [p\, w].$$
The inversion principle follows from the induction principle
by instantiating $p$ with
$$\lambda x\,.\, [[x = 0]\, \lor\,
\exists w\,.\, [x = [S\, w]]\,\land\, [v\, w]],$$
but it is clearly easier, if we need the inversion principle,
to have it in the lemma rather than needing to prove it
by instantiating the new set variable $p$.

In the general case in which some constraints
contain a $\Gamma(v)$ with $v$ free in $\Gamma(v)$,
we need machinery to show that there is a solution $v$
satisfying the constraints, the inversion principle,
and the induction principle.  This machinery is provided
by the Knaster-Tarski Fixed Point Theorem.

\subsection{Knaster-Tarski Fixed Point Theorem}\label{knaster-tarski}

The Knaster-Tarski Fixed Point Theorem states that monotone
set functions have fixed points.  There are also versions
showing there are least and greatest fixed points (in fact,
there are a lattice of such fixed points).  % ref Dana Scott & other domain theory sources

{\bf Definitions:}  Suppose $K:\wp(A)\rightarrow \wp(A)$
for a power set $\wp(A)$.  A {\it pre-fixed point} of $K$ is a set $v$ such
that $K(v)\subseteq v$.  A {\it post-fixed point} of $K$ is a set $v$ such
that $v\subseteq K(v)$.  A {\it fixed point} of $K$ is a set $v$ satisfying $K(v) = v$.

{\bf Knaster-Tarski Fixed Point Theorem:}  Suppose $K:\wp(A)\rightarrow \wp(A)$
for a power set $\wp(A)$.  Further suppose $K$ is monotone function in
the sense that for every $v\subseteq w\subseteq A$, $K(v) \subseteq K(w)$.
Then there is a fixed point $u$ of $K$.

{\bf Proof:}  Let $u = \bigcap\{v\in\wp(A) | K(v) \subseteq v\}$.
That is, we define $u$ to be the intersection of all the pre-fixed points of $K$.
We need to show $K(u)\subseteq u$ and $u\subseteq K(u)$.

First, we show $K(u)\subseteq u$.  Suppose $z\in K(u)$.  To show $z\in u$,
we need to show $z\in v$ for every pre-fixed point $v$.  Let $v$ be a
pre-fixed point.  By the definition of $u$, we have $u\subseteq v$.
Since $K$ is monotone, $K(u)\subseteq K(v)$, so $z\in K(v)$.  But
$v$ is a pre-fixed point, so $z\in K(v)\subseteq v$.  Thus, $u$ is
itself a pre-fixed point.  (In fact, it is clearly the least pre-fixed point.)

Since $u$ is a pre-fixed point and $K$ is monotone,
we have $K(K(u)) \subseteq K(u)$.  So, $K(u)$ is a pre-fixed point.
Since $u$ is the least pre-fixed point, we have $u\subseteq K(u)$.
$\Box$

This proof actually shows the following form of the theorem:

{\bf Knaster-Tarski Fixed Point Theorem (Least):}  Suppose $K:\wp(A)\rightarrow \wp(A)$
for a power set $\wp(A)$.  Further suppose $K$ is monotone function in
the sense that for every $v\subseteq w\subseteq A$, $K(v) \subseteq K(w)$.
There is a least pre-fixed point $u$ of $K$ which is also a fixed point of $K$.

A dual proof shows

{\bf Knaster-Tarski Fixed Point Theorem (Greatest):}  Suppose $K:\wp(A)\rightarrow \wp(A)$
for a power set $\wp(A)$.  Further suppose $K$ is monotone function in
the sense that for every $v\subseteq w\subseteq A$, $K(v) \subseteq K(w)$.
There is a greatest post-fixed point $u$ of $K$ which is also a fixed point of $K$.

These statements and proofs have a straightforward representation in
type theory.  The same proof idea works regardless of the arity of the
set type of $u$.  The functions \indexfunction{make-knaster-tarski-lemma},
\indexfunction{make-knaster-tarski-negf}, 
\indexfunction{make-knaster-tarski-leastfp-lemma},
\indexfunction{make-knaster-tarski-leastfp-negf}, 
\indexfunction{make-knaster-tarski-gfp-lemma},
\indexfunction{make-knaster-tarski-gfp-negf},
and others with similar names,
construct (an ftree representation of) an expansion proof of
these different versions of the Knaster-Tarski Theorem
for a given set type.
A proof of the Knaster-Tarski Theorem for the set type of $u$ can be used as a lemma
to show a set existence lemma for $u$ which includes the constraint properties,
the inversion principle, and the induction principle.

{\bf THM2} in \TPS is a version of the Knaster-Tarksi Theorem.

When we apply the Knaster-Tarski theorem, we need to instantiate
$K$ with a set function we know is monotone.  This can be ensured
syntactically.

{\bf Definition:}  We can define a set of terms {\it positive} and
{\it negative} with respect to $u$ by induction.
\begin{itemize}
\item $[u\,\overline{t}]$ is positive with respect to $u$, so long as
$u$ does not occur free in $\overline{t}$.
\item $P_1 \lor P_2$ is positive (negative) with respect to $u$ if both $P_1$ and $P_2$ are.
\item $P_1 \land P_2$ is positive (negative) with respect to $u$ if both $P_1$ and $P_2$ are.
\item $P_1 \limplies P_2$ is positive (negative) with respect to $u$ if $P_1$ 
is negative (positive) with respect to $u$ and $P_2$
is positive (negative) with respect to $u$.
\item $\lnot P$ is positive (negative) with respect to $u$ if
$P$ is negative (positive) with respect to $u$.
\item $\forall x P$ is positive (negative) with respect to $u$ if
$P$ is, or if $x$ is the same variable as $u$.
\item $\exists x P$ is positive (negative) with respect to $u$ if
$P$ is, or if $x$ is the same variable as $u$.
\end{itemize}

{\bf Proposition:} Suppose $P$ is positive with respect to $u$.
Then $\lambda u\lambda \overline{z} P(u,\overline{z})$
represents a monotone function of $u$.

{\bf Proof:}  We can prove this by induction on $P$.
% fill this in
$\Box$

The function \indexfunction{mon-fn-negf} generates an expansion
proof that a particular $\lambda u\lambda \overline{z} P(u,\overline{z})$
is monotone where $P$ is positive with respect to $u$.
This function returns an ftree and pushes new connections
on the special variable \indexother{clist}.

We will solve some constraints by building a monotone function $K$
from the constraints.  Some constraints do not directly give a monotone
function of the set variable.  In these cases, we may want to find
a best approximating.

{\bf Definition:}  Given a function $F:\wp(A)\rightarrow\wp(A)$, let
$K_F:\wp(A)\rightarrow\wp(A)$ and $K^F:\wp(A)\rightarrow\wp(A)$ be
defined by 
$$K_F(u) = \{z | \forall w\, [u\subseteq w\, \limplies \,z\in F(w)]\}$$
and
$$K^F(u) = \{z | \exists w\, [w\subseteq u \,\land \,z\in F(w)]\}.$$

{\bf Proposition:}  $K_F$ and $K^F$ are monotone set functions.
For all $u$, $K_F(u)\subseteq F(u)\subseteq K^F(u)$.  Futhermore,
if $L:\wp(A)\rightarrow\wp(A)$ is a montone set function and
forall $u$, $L(u)\subseteq F(u)$, then forall $u$, $L(u)\subseteq K_F(u)$.
Similarly, if $R:\wp(A)\rightarrow\wp(A)$ is a montone set function and
forall $u$, $F(u)\subseteq R(u)$, then forall $u$, $K^F(u)\subseteq R(u)$.
So, $K_F$ and $K^F$ are the best monotone upper and lower aproximations of $F$.

{\bf Proof:}  Suppose $u\subseteq v$ and $z\in K_F(u)$.  We need to show
that $z\in F(w)$ for every $w\supseteq v$.  Given such a $w$,
apply the definition
of $K_F$ to the set $w$ to obtain $[u\subseteq w]\,\limplies\, z\in F(w)$.
Since $u\subseteq v\subseteq w$, we have $z\in F(v)$.  So, $K_F$ is monotone.  Similarly,
suppose $u\subseteq v$ and $z\in K^F(u)$.  Apply the definition of $K^F(u)$
to obtain a set $w\subseteq u$ with $z\in w$.  Now, since
$w\subseteq u\subseteq v$, this $w$ can be used
to witness that $z\in K^F(v)$.  So, $K^F$ is monotone.

Let $u\in\wp(A)$ be given.  We need to show $K_F(u)\subseteq F(u)$
and $F(u)\subseteq K^F(u)$.  Suppose $z\in K_F(u)$.  Apply the definition
of $K_F$ to $u$.  Since $u\subseteq u$, we have $z\in F(u)$.
Next, suppose $z\in F(u)$.  Then $u$ can witness that $z\in K^F(u)$.

Now suppose $L$ is a monotone function such that for every $u\in\wp(A)$,
$L(u)\subseteq F(u)$.  We need to show for every $u\in\wp(A)$, $L(u)\subseteq K_F(u)$.
Let $u\in\wp(A)$ be given and suppose $z\in L(u)$.  Since $L$ is monotone,
for every $w\supseteq u$, $z\in L(w)$.  So, $z\in F(w)$.  This shows $z\in K_F(u)$
and we are done.

Now suppose $R$ is a monotone function such that for every $u\in\wp(A)$,
$F(u)\subseteq R(u)$.  We need to show for every $u\in\wp(A)$, $K^F(u)\subseteq R(u)$.
Let $u\in\wp(A)$ be given and suppose $z\in K^F(u)$.  So, there is a $w\subseteq u$
with $z\in F(w)$.  So, $z\in R(w)$.  Since $R$ is monotone and $w\subseteq u$,
we have $z\in R(u)$ and we are done.
$\Box$

\subsection{Tracing Through An Example}

Consider again the two minimal constraints
$$\rightarrow [v\, 0]$$
and
$$w | [v\, w] \rightarrow [v\, [S\, w]]$$
where $v$ has type $\greeko\greeki$.
Let us name the literals
\begin{itemize}
\item [L1] $v\, 0$
\item [L2] $v\, w$
\item [L3] $v\, [S\, w]$
\end{itemize}
In the function \indexfunction{ftree-solve-constraint-set},
consider some of the local variables:
\begin{itemize}
\item \verb+f+ is an ftree containing the nodes L1, L2, and L3,
and the expansion variable $v$.
\item \verb+constrs+ has the value ((L1) (L3 L2))
\item \verb+v+ has the value $v$.
\item \verb+vsel+ is a new variable $v^1$ of the same type as $v$.
\item \verb+f3+ is \verb+f+ with $v^1$ (not an expansion variable)
substituted for the expansion variable $v$.
\item \verb+banned-occurs+ ends up having the value (($w$) NIL),
representing $\Psi_1$ and $\Psi_2$.
\item \verb+misc-occurs+ is NIL, assuming $S$ and $0$ are constants.
\item \verb+paths+: ((LF3 LF2) (LF1)) where each LFi
is the ftree node in \verb+f3+ corresponding to Li.
\end{itemize}

Then \indexfunction{make-ftree-setvar-soln} is called with
\begin{itemize}
\item \verb+vsel+: $v^1$
\item \verb+kind+: MIN
\item \verb+paths+: ((LF3 LF2) (LF1))
\item \verb+banned-occurs+: (($w$) NIL)
\item \verb+misc-occurs+: NIL
\item \verb+rec-flag+: T
\end{itemize}
A special dynamic variable \verb+clist+ is used to collect connections
created in the process of building the lemma.
The function \indexfunction{make-min-inv-princ} constructs the inversion principle
$$\forall \,x . \,v^1 \,x \supset \exists \,w [ \,x \,= \,S \,w \land \,v^1 \,w] \lor \,x \,= \,0.$$
Next, \indexfunction{make-min-setvar-lemma-posf} constructs the full lemma \
\begin{center}
\begin{tabular}{r}
$ \exists \,v^2 . [\forall \,w^1 [ \,v^2 \,w^1 \supset \,v^2 . \,S \,w^1 ] \land \,v^2 \,0]$ \\
$\land\,\forall \,x^1 [ \,v^2 \,x^1 \supset \exists \,w [ \,x^1 \,= \,S \,w \land \,v^2 \,w ] \lor \,x^1 \,= \,0 ]$ \\
$\land \forall p\, .\, [\forall \,w^1 [ \,p \,w^1 \supset \,p . \,S \,w^1 ] \land \,p \,0]
\limplies\, \forall x^2\, . \, [v^1\, x^2]\,\limplies\, [p\, x^2].$  \\
\end{tabular}
\end{center}
and the positive ftree that will correspond to how the lemma can be used.
In particular, connections to the nodes LF1, LF2, and LF3 are created
which solve the constraints (i.e., block every verticle path through LF1,
and every vertical path through LF2 and LF3).

In simpler examples which do not require a recursive definition
using the Knaster-Tarski Fixed Point Theorem,
the function \indexfunction{make-min-setvar-lemma-negf}
constructs an ftree proof of the lemma.
In this case, the lemma does require a recursive definition,
so the function \indexfunction{make-clos-setvar-lemma-negf}
is called.  If \indexflag{INCLUDE-INDUCTION-PRINCIPLE} is T,
we need a strong form of the Knaster-Tarski Theorem, in which
we know the set $u$ is the least pre-fixed point.
The function \indexflag{make-knaster-tarski-leastfp-lemma}
is called to construct an ftree proof for this lemma.
Otherwise, \indexflag{make-knaster-tarski-lemma}
gives an ftree proof of the simpler version.
The function \indexflag{make-clos-setvar-lemma-negf-0}
uses the Knaster-Tarski lemma to prove the set existence lemma.

Now, let us examine each of these steps in more detail.

\begin{enumerate}
\item {\bf \indexfunction{make-min-inv-princ}}  Given $v^1$, the paths
(LF1 (LF3 LF2)), and the banned variable information
(($w$) NIL), use the type of $v^1$ to create a formula
$$\forall \,x . \, [v^1\, x] \, \limplies\, InvP_2$$
where $InvP_2$ is formed by \indexfunction{make-min-inv-princ-2}.
$InvP_2$ needs to be positive with respect to $v^1$ so that
we can apply the Knaster-Tarski Theorem later.

\begin{itemize}
\item {\bf \indexfunction{make-min-inv-princ-2}}  Given the bound list
(\verb+zl+) ($x$), the atomic formula (\verb+vzl+) $v x$, as well as
the information above, form a disjunction of inversion principles
corresponding to each path:
$$InvP_4^{(LF3\, LF2)} \, \lor\, InvP_4^{(LF1)}$$
where each $InvP_4$ formula (positive with respect to $v^1$) is constructed by a call to 
\indexfunction{make-min-inv-princ-3}.

\item {\bf \indexfunction{make-min-inv-princ-3}}  Given one of the constraints,
construct equations and a substitution sending some banned variables (those occurring
as an argument of the main literal of the constraint) to the variables
constructed by \indexfunction{make-min-inv-princ} (in this case, $x$).
In this case we have two constraints.  

First consider, (LF1), i.e., $v^1\, 0$.  In this case we only have one literal,
the main literal of the constraint.  Since $0$ is a constant, and not a banned
selected variable, we make an equation $x = 0$ between the argument of $v^1\, x$
and the argument of the main constraint literal $v^1\, 0$.  In general, there
may be several arguments, giving several equations.  Given these equations,
\indexfunction{make-min-inv-princ-4} constructs the formula $InvP_4^{(LF1)}$.

Second consider, (LF3 LF2), i.e., $w|v^1\, w \rightarrow v^1 \, [S\, w]$.
In this case, the main literal is $v^1\, [S\, w]$.  If the argument $[S\, w]$
were $w$, we could replace $w$ by $x$ in the constraint for the purpose
of computing the inversion principle.
Instead, we make an equation $x = [S\, w]$ and call
\indexfunction{make-min-inv-princ-4} to construct the formula $InvP_4^{(LF3\, LF2)}$.

\item {\bf \indexfunction{make-min-inv-princ-4}}  Constructs a formula
existentially binding the remaining banned variables in the constraint.
Since (LF1) has no banned variables, $InvP_4^{(LF1)}$ is $InvP_5^{(LF1)}$
constructed by \indexfunction{make-min-inv-princ-5}.  Since (LF3 LF2)
has the banned variable $w$, $InvP_4^{(LF3\, LF2)}$ is of the form
$$\exists w\, . \, InvP_5^{(LF3\, LF2)}$$
where $InvP_5^{(LF3\, LF2)}$ is constructed by \indexfunction{make-min-inv-princ-5}.

\item {\bf \indexfunction{make-min-inv-princ-5}}  Constructs a conjunct of
the equations generated in \indexfunction{make-min-inv-princ-3} corresponding
to the constraint.  

For the constraint (LF1), $InvP_5^{(LF1)}$ is $x=0$.
Since in this constraint, the constraint has no literals in this
constraint other than the main literal LF1, \indexfunction{make-min-inv-princ-6}
makes no contributation.

For the constraint (LF3 LF2), $InvP_5^{(LF3\, LF2)}$ is
$x=[S\, w]\,\land\, InvP_6^{(LF2)}$
where $InvP_6^{(LF2)}$ is constructed by \indexfunction{make-min-inv-princ-6}
using the extra literals of the constraint, in this case (LF2).

\item {\bf \indexfunction{make-min-inv-princ-6}}  Constructs a conjunct of
formulas for each extra literal of the constraint.  In this case, there is
the one literal LF2, giving $InvP_6^{(LF2)}$ as $v^1\, w$.

In the general case, when a literal LF does not contain the set selected
variable (\verb+vsel+) $v^1$, $InvP_6^{(LF\, .\, <LIST>)}$ will be of
the form $A\,\lor\, InvP_6^{<LIST>}$ where $A$ is the shallow formula
of LF or its negation (if LF is negative).  This will also be
the form if LF is a positive literal only containing $v^1$ at the head.

The more complicated case is when $v^1$ occurs inside the body of the
literal.  Suppose $A(v^1)$ is the shallow formula of such a literal LF
(or its negation if the literal is negative).
Since we will want a formula which is positive with respect to $v^1$,
we let $InvP_6^{(LF\, . \, LEAF)}$ be of a form such as
$$\exists w^i_{\greeko\greeki}\, [[\forall x\, .\, [w^i\, x]\,\limplies\, [v^1_{\greeko\greeki}\, x]]\,\land\, A(w^i)].$$
Of course, in general $\greeko\greeki$ may be any set type and $x$ may be
a list of variables.
See section~\ref{knaster-tarski} for a semantic description of this
as a least monotone upper approximation.
{\bf THM2} is an example where such an approximation is necessary.
\end{itemize}

In the end, we have constructed the inversion principle as follows:
$$\forall \,x . \, [v^1\, x] \, \limplies\, InvP_2$$
$$\forall \,x . \, [v^1\, x] \, \limplies\, .\,InvP_4^{(LF3\, LF2)} \, \lor\, InvP_4^{(LF1)}$$
$$\forall \,x . \, [v^1\, x] \, \limplies\, .\,\exists w\, InvP_5^{(LF3\, LF2)} \, \lor\, InvP_5^{(LF1)}$$
$$\forall \,x . \, [v^1\, x] \, \limplies\, .\,\exists w\, [x=[S\, w]\,\land\, InvP_6^{(LF2)}] \, \lor\, [x=0]$$
$$\forall \,x . \, [v^1\, x] \, \limplies\, .\,\exists w\, [x=[S\, w]\,\land\, [v^1\, w]] \, \lor\, [x=0].$$

\item {\bf \indexfunction{make-min-setvar-lemma-posf}}  Constructs the set
existence lemma \
\begin{center}
\begin{tabular}{r}
$ \exists \,v^2 . \forall \,w^1 [ \,v^2 \,w^1 \supset \,v^2 . \,S \,w^1 ] \land \,v^2 \,0$ \\
$\land \, \forall \,x^1 [ \,v^2 \,x^1 \supset \exists \,w [ \,x^1 \,= \,S \,w \land \,v^2 \,w ] \lor \,x^1 \,= \,0 ]$ \\
$\land \forall p\, .\, [\forall \,w^1 [ \,p \,w^1 \supset \,p . \,S \,w^1 ] \land \,p \,0]
\limplies\, \forall x^2\, . \, [v^2\, x^2]\,\limplies\, [p\, x^2].$ \\
\end{tabular}
\end{center}
and a positive ftree used to solve the constraints.
In general, this would start by universally quantifying
any extra expansion variables and selected variables occurring in the constraints
(\verb+misc-occurs+).  In this case, there are none.  We create a fresh variable
(\verb+v2+) $v^2$ to play the role of the set variable in the formula.
\indexfunction{make-min-setvar-lemma-posf-1} is called to construct
a positive ftree $POSF_1$.  The shallow formula of $POSF_1$ is $LEM_1(v^1)$ the body of
the set existence lemma, using the selected variable $v^1$.
So, we return a selected node $POSF$ with shallow $\exists v^2\,.\, LEM_1(v^2)$
and child $POSF_1$.

\begin{itemize}
\item {\bf \indexfunction{make-min-setvar-lemma-posf-1}}
The main part of the lemma we need to construct is the part that
solves the constraints.  The function \indexfunction{make-min-setvar-lemma-posf-3}
returns a postive ftree $POSF_2$ with shallow formula $Main(v^1)$.  It also
adds connections between literals in $POSF_2$ and the literals LF1, LF2, and LF3
of the constraints.

We have already constructed the inversion principle $InvP$.
If \indexflag{INCLUDE-INDUCTION-PRINCIPLE} is NIL, we construct a positive
ftree for $Main(v^1)\, \land \, InvP$.
If \indexflag{INCLUDE-INDUCTION-PRINCIPLE} is T, we construct a positive
ftree for $Main(v^1)\, \land \, [InvP\,\land\, IndP]$ where $IndP$ is an
induction principle.  In this case, the induction principle is
$$\forall p\, .\, Main(p)\,\limplies\, \forall x^2\, . \, [v^1\, x^2]\,\limplies\, [p\, x^2].$$
The positive ftrees for the inversion principle and the induction principle
are simply constructed by expanding the formula as an ftree, duplicating
the outermost quantifier \indexflag{NUM-OF-DUPS} times.  These are combined
using conjunction nodes with $POSF_2$.

\item {\bf \indexfunction{make-min-setvar-lemma-posf-2}}  For each constraint $P$,
we make a conjunction of $POSF_3^P$ obtained by
calling \indexfunction{make-min-setvar-lemma-posf-3}.  In this case, we have a conjunction of
$POSF_3^{(LF3\, LF2)}$ and $POSF_3^{(LF1)}$.

\item {\bf \indexfunction{make-min-setvar-lemma-posf-3}}  If there are
banned variables in the constraint, we will make an expansion node.
In this case, we will have one child that corresponds expanding using
the banned variables (so we can mate to the literals in the constraints).
In case we will later want to use this part of the lemma elsewhere in the
proof, we also duplicate \indexflag{NUM-OF-DUPS} times.

The constraint (LF1) does not contain any banned variables and has
no extra literals.  So, we simply let $POSF_3^{(LF1)}$ be a positive
leaf with shallow $[v^1\, 0]$ (the same shallow as LF1).
This leaf is connected to the negative node LF1, solving this constraint.

The constraint (LF3 LF2) contains the banned variable $w$.
So, $POSF_3^{(LF3\, LF2)}$ is an expansion node.  
(We create a fresh variable $w^1$ to use as the bound variable in the shallow formula.)
One child of this expansion node
is $POSF_4^{(LF3\, LF2)}$ with expansion term $w$.  This child is constructed
and is used to solve the constraint by calling
\indexfunction{make-min-setvar-lemma-posf-4}.  If \indexflag{NUM-OF-DUPS} is
greater than $0$, we also have \indexflag{NUM-OF-DUPS} many other children
of $POSF_3^{(LF3\, LF2)}$ expanded using expansion variables.  These children
could be used in the proof of the theorem.

\item {\bf \indexfunction{make-min-setvar-lemma-posf-4}}  This function
creates more expansion nodes corresponding to the rest of the banned variables
of the constraint.  Since (LF3 LF2) only has one banned variable, we
skip directly to constructing an implication node $POSF_4^{(LF3\, LF2)}$
where the first child is a negative ftree $POSF_5^{(LF2)}$ and the second
child is a positive leaf with shallow formula $[v^1\, [S\, w]]$ (the
same as LF3).  This is connected to the node LF3.
The negative ftree $POSF_5^{(LF2)}$ is constructed by
\indexfunction{make-min-setvar-lemma-posf-5}
and used to block LF2.

\item {\bf \indexfunction{make-min-setvar-lemma-posf-5}}  This function constructs
a conjunction corresponding to the extra literals in a given constraint.
In this case, we only have the one extra literal LF2.  So, $POSF_5^{(LF2)}$
is $POSF_6^{LF2}$ constructed by \indexfunction{make-min-setvar-lemma-posf-6}.

\item {\bf \indexfunction{make-min-setvar-lemma-posf-6}}  Given a literal in
a constraint, this creates a corresponding leaf and mates it to the literal
in the constraint.  Since LF2 is positive, $POSF_6^{LF2}$ is a negative leaf
with shallow formula $[v^1\, w]$ and this is mated to LF2.
\end{itemize}

So, to sum up, we created a positive ftree for the lemma along with
connections to the nodes in the constraints.  We can follow the construction
by the noting that the construction of the shallow formulas proceeded
as
$$\exists v^2\,.\, LEM_1(v^2)$$
$$Main(v^1)\,\land\, InvP\,\land\, IndP$$
where $Main(v^1)$ was constructed as
$$Sh(POSF_3^{(LF3\, LF2)})\, \land \, Sh(POSF_3^{(LF1)})$$
$$\forall w^1 Sh(POSF_4^{(LF3\, LF2)})\, \land \, [v^1\, 0]$$
$$\forall w^1 [Sh(POSF_5^{(LF2)})\,\limplies\, [v^1\, [S\, w^1]]]\, \land \, [v^1\, 0]$$
$$\forall w^1 [[v^1\, w^1]\,\limplies\, [v^1\, [S\, w^1]]]\, \land \, [v^1\, 0]$$
(where $Sh(N)$ means the shallow formula of the node $N$).

\item {\bf \indexfunction{make-clos-setvar-lemma-negf}}  This produces
a negative ftree with connections giving the proof of the set existence lemma.
Let us assume \indexflag{INCLUDE-INDUCTION-PRINCIPLE} is set to T.
The function \indexfunction{make-knaster-tarski-leastfp-lemma} constructs
an ftree proof $Knaster-Tarski^-$ of the least fixed point version of the
Knaster-Tarski Theorem for the type of the set variable $v$.
In this case, $v$ has type $\greeko\greeki$, so the Knaster Tarski Theorem
generated is \
\begin{center}
\begin{tabular}{r}
$\forall K_{\greeko\greeki(\greeko\greeki)} \, . \,
\forall u_{\greeko\greeki}\forall v_{\greeko\greeki} 
[\forall z \, . \, u\, z \,\limplies\, v\, z]$ \\
$\limplies\, \exists u_{\greeko\greeki} \, . \, \forall z\, [K\, u\, z\, \limplies \, u\, z]
\,\land \,\forall z\, [u\, z\, \limplies \, K \, u\, z]$ \\
$\forall v_{\greeko\greeki}\, . \, \forall z\, [K\, v\, z\, \limplies \, v\, z]
\,\limplies\, \forall z\, .\, u\, z\, \limplies \, v\, z$ \\
\end{tabular}
\end{center}
The function \indexfunction{make-clos-setvar-lemma-negf-0}
does the work of constructing the negative ftree node $NEGF$ giving
the proof of the set existence lemma.  The special variable
\verb+expf+ is set to a positive ftree $Knaster-Tarski^+$ for the Knaster-Tarski
Lemma which is used to prove the set existence lemma.
Then we return the ftree
$$\econj{}{}{Knaster-Tarski^-}{\eimp{}{}{Knaster-Tarski^+}{NEGF_0^-}}$$
The new connections are added to the dynamic variable \verb+clist+.

\begin{itemize}
\item {\bf \indexfunction{make-clos-setvar-lemma-negf-0}}
Our goal is to construct a negative ftree $NEGF_0^-$ for the set existence lemma
\begin{center}
\begin{tabular}{r}
$ \exists \,v^2 . \forall \,w^1 [ \,v^2 \,w^1 \supset \,v^2 . \,S \,w^1 ] \land \,v^2 \,0$ \\
$\land \, \forall \,x^1 [ \,v^2 \,x^1 \supset \exists \,w [ \,x^1 \,= \,S \,w \land \,v^2 \,w ] \lor \,x^1 \,= \,0 ]$ \\
$\land \forall p\, .\, [\forall \,w^1 [ \,p \,w^1 \supset \,p . \,S \,w^1 ] \land \,p \,0]
\limplies\, \forall x^2\, . \, [v^2\, x^2]\,\limplies\, [p\, x^2].$ \\
\end{tabular}
\end{center}
In general, the set existence lemma universally binds the variables
in \verb+misc-occurs+.  In this case there are no such variables.
We also need to construct the positive $Knaster-Tarski^+$ node (\verb+expf+)
and add connections to \verb+clist+ between nodes in these two ftrees.

We proceed to the most important step, instantiating the $K$ in
the Knaster-Tarski Theorem.  The monotone function we want can
be extracted from the inversion principle.  The inversion principle (\verb+inv-princ+)
is
$$\forall x . \,v^1 \,x \supset \exists \,w [ \,x \,= \,S \,w \land \,v^1 \,w] \lor \,x \,= \,0.$$
and we construct the monotone function (\verb+monfn+):
$$\lambda v^1 \lambda x\, . \exists \,w [ \,x \,= \,S \,w \land \,v^1 \,w] \lor \,x \,= \,0.$$
Let us write $\lambda v^1\lambda x\, . \, P(v^1,x)$ for this term.
Note that $P(v^1,x)$ is positive with respect to $v^1$, so the function will be monotone.
We substitute this for $K$ so that $Knaster-Tarski^+$ is an expansion node
$$\uexpnode{}{\mbox{Knaster-Tarksi}}{\lambda v^1\lambda x\, . \, P(v^1,x)}{\erew{}{\lambda}{}{\eimp{}{}{MONNEGF^-}{EUF^+}}}$$
with one child using the monotone function as the expansion term.
The child of this node is a $\lambda$-rewrite passing to the normal form.
Below this is an implication node with two children $MONNEGF^-$ providing
a proof that $\lambda v^1\lambda x\, . \, P(v^1,x)$ is monotone and $EUF^+$ a positive node with shallow formula (\verb+eu+)
\begin{center}
\begin{tabular}{r}
$\exists u_{\greeko\greeki} \, . \, \forall z\, [P(u,z)\, \limplies \, u\, z]
\,\land \,\forall z\, [u\, z\, \limplies \, P(u,z)]$ \\
$\forall v_{\greeko\greeki}\, . \, \forall z\, [P(v,z)\, \limplies \, v\, z]
\,\limplies\, \forall z\, .\, u\, z\, \limplies \, v\, z.$ \\
\end{tabular}
\end{center}
The node $MONNEGF^-$ is constructed by the function \indexfunction{mon-fn-negf}.
This function implements the proof that set functions defined by positive formulas
are monotone.

Let $u^1$ be a new selected variable to use in the selection node $EUF^+$:
$$\sel{}{EU}{u^1}{\econj{}{}{Kuuf^+}{\econj{}{}{uKuf^+}{lff^+}}}$$
where $EU$ is the right side of the top implication of the Knaster-Tarski Theorem.
The node $Kuuf^+$ is has shallow formula $Kuu$:
$$\forall z\, .\, [\exists \,w [ \,z \,= \,S \,w \land \,u^1 \,w] \lor \,z \,= \,0]
\,\limplies\, u^1\, z$$
and is constructed during the process of constructing $NEGF_0^-$ below.
We start with $Kuuf^+$ as a leaf.
The node $uKuf^+$ is a leaf that corresponds directly to
the inversion principle in the set existence lemma
$$\forall z\, [u^1\, z\, \limplies \, \exists \,w [ \,z \,= \,S \,w \land \,u^1 \,w] \lor \,z \,= \,0.$$
Finally, we will construct a node $lff^+$ has shallow formula \
\begin{center}
\begin{tabular}{r}
$\forall p\, \forall z\, [[\exists \,w [ \,z \,= \,S \,w \land \,u^1 \,w] \lor \,z \,= \,0]
\,\limplies\, u^1\, z]$ \\
$\limplies\, \forall z\, .\, u^1\, z\,\limplies\, p\, z$ \\
\end{tabular}
\end{center}
obtained from the fact that $u^1$ is the least pre-fixed point.

The selected variable $u^1$ is used as the expansion term $NEGF_0^-$
we want to construct for the set existence lemma.  So, $NEGF_0^-$
will have the form
$$\uexpnode{}{\mbox{Set Existence}}{u^1}{\econj{}{}{NEGF_1^-}{\econj{}{}{InvF^-}{IndF^-}}}$$
where $NEGF_1^-$ (\verb+f+)
is constructed by \indexfunction{make-clos-setvar-lemma-negf-1},
$InvF^-$ (\verb+l+) is a leaf corresponding to the inversion principle,
and $IndF^-$ (\verb+indf+) constructed by \indexfunction{make-clos-setvar-ind-negf}
corresponds to the induction principle.
The leaves $InvF^-$ and $uKuf^+$ are mated on \verb+clist+.

\item {\bf \indexfunction{make-clos-setvar-lemma-negf-1}}  This function
constructs $NEGF_1^-$ with shallow formula
$$\forall \,w [ \,u^1 \,w \supset \,u^1 . \,S \,w ] \land \,u^2 \,0.$$
Along with this, we will construct $Kuuf^+$ and put connections onto \verb+clist+.
So, we build $NEGF_1^-$ as a conjunction 
$NEGF_2^{(LF3\, LF2)}\land NEGF_2^{(LF1)}$
where $NEGF_2^{(LF3\, LF2)}$ and $NEGF_2^{(LF1)}$
corresponding to the two constraints is constructed by
\indexfunction{make-clos-setvar-lemma-negf-2}.
We also use an integer \verb+n+ to keep up with which constraint we
are considering.

\item {\bf \indexfunction{make-clos-setvar-lemma-negf-2}}
Given a constraint $C$ (\verb+paths+), this function builds
a negative ftree $NEGF_2^C$ and changes $Kuuf^+$ and \verb+clist+.

First, consider the constraint (LF1).  The shallow formula of
$NEGF_2^{(LF1)}$ should be $u^1\, 0$.  Here, we let
\indexfunction{make-clos-setvar-lemma-negf-3} do the work
of constructing an ftree which will be merged with $Kuuf^+$.
Then \indexfunction{make-clos-setvar-lemma-negf-4} creates
the negative leaf $NEGF_2^{(LF1)}$ mated with part of
the ftree generated in \indexfunction{make-clos-setvar-lemma-negf-4}.

The second constraint we need $NEGF_2^{(LF3\, LF2)}$ to have shallow
formula
$$\forall w \, . \, u^1\, w\,\limplies\, u^1\, . \, S\, w.$$
First, we create a new selected variable $w^2$ and let $NEGF_2^{(LF3\, LF2)}$
be 
$$\sel{}{\forall w \, . \, u^1\, w\,\limplies\, u^1\, . \, S\, w}{w^2}{\eimp{}{}{NEGF_3^{(LF2)}}{LEAF3}}$$
where $NEGF_3^{(LF2)}$ is constructed by \indexfunction{make-clos-setvar-lemma-negf-3}
(along with an ftree to be merged with $Kuuf^+$)
and $LEAF3$ is a negative leaf with shallow $u^1\, . \, S\, w^2$
(mated to part of the new part of $Kuuf^+$).

\item {\bf \indexfunction{make-clos-setvar-lemma-negf-3}}  Suppose we are
given a constraint $C$ ($\Gamma\rightarrow A$) and the arguments (\verb+args+) of the main literal of the constraint.
This function constructs three positive ftrees.  The first is an ftree $Kuuf_3^C$
which will be merged with $Kuuf^+$.  The second is $NEGF_3^\Gamma$ where $L$
is the list of extra literals $L$ of the constraint, or NIL if there are no extra
literals.  If $NEGF_3^\Gamma$ is not nil, it is
a positive ftree corresponding to a conjunct of the extra literals $L$ of the constraint.
The third is a leaf inside the first ftree which will be used in the mating.

Recall that $Kuu$ is
$$\forall z\, .\, [\exists \,w [ \,z \,= \,S \,w \land \,u^1 \,w] \lor \,z \,= \,0]
\,\limplies\, u^1\, z,$$
and this will be the shallow formula of $Kuuf_3^C$.
The third ftree returned will be the leaf corresponding to $u^1\, z$
beneath $Kuuf_3^C$.

In the first constraint (LF1), the args are $(0)$.  We use the list of
arguments as expansion terms to create $Kuuf_3^{(LF1)}$.  Here, $Kuuf_3^{(LF1)}$ will be
$$\uexpnode{}{Kuu}{0}{\eimp{}{}{Kuuf_5^{(LF1)}}{LEAF:u^1\, 0^+}}$$
where $LEAF:u^1\, 0^+$ will be the third ftree returned.
$Kuuf_5^{(LF1)}$ is a negative ftree constructed by
\indexfunction{make-clos-setvar-lemma-negf-5}.
In this case there are no extra literals, so the second return value is NIL.

In the second constraint (LF3 LF2), the one argument is $S\, w^2$.
In this case, $Kuuf_3^{(LF3\, LF2)}$ is 
$$\uexpnode{}{Kuu}{S\, w^2}{\eimp{}{}{Kuuf_5^{(LF3\, LF2)}}{u^1\, . \,S\, w^2}}$$
where $Kuuf_5^{(LF3\, LF2)}$ is constructed
along with $NEGF_3^{(LF2)}$
by \indexfunction{make-clos-setvar-lemma-negf-5}.

\item {\bf \indexfunction{make-clos-setvar-lemma-negf-4}}  This simply makes
a negative leaf and mates it to a given positive leaf.

\item {\bf \indexfunction{make-clos-setvar-lemma-negf-5}}  Given a constraint
$C$ ($\Gamma\rightarrow A$),
this function returns two values.  The first is a negative ftree 
$Kuuf_5^C$, and the second is a positive ftree $NEGF_8^\Gamma$ (or NIL if $\Gamma$ is empty)
constructed later.

For the constraint (LF1) $Kuu_5^{(LF1)}$ should have shallow formula
$$[\exists \,w [ \,z \,= \,S \,w \land \,u^1 \,w] \lor \,0 \,= \,0].$$
So, is $Kuu_5^{(LF1)}$ of the form
$$\edisj{}{}{UNUSEDLEAF}{Kuu_6^{(LF1)}}$$
where $Kuu_6^{(LF1)}$ is constructed by \indexfunction{make-clos-setvar-lemma-negf-6}.

For the constraint (LF3 LF2) $Kuu_5^{(LF3\, LF2)}$ should have shallow formula
$$[\exists \,w [ S\, w^2 \,= \,S \,w \land \,u^1 \,w] \lor \,[S\,w^2] \,= \,0].$$
In this case, $Kuu_5^{(LF3\, LF2)}$ is of the form
$$\edisj{}{}{Kuu_6^{(LF3\, LF2)}}{UNUSEDLEAF}$$
where $Kuu_6^{(LF3\, LF2)}$ is constructed by \indexfunction{make-clos-setvar-lemma-negf-6}
(along with $NEGF_8^{(LF2)}$).

\item {\bf \indexfunction{make-clos-setvar-lemma-negf-6}}  This function
constructs $Kuu_6^C$ using the selected variables created in
\indexfunction{make-clos-setvar-lemma-negf-2} as expansion terms,
then calls \indexfunction{make-clos-setvar-lemma-negf-7} to construct
the rest of $Kuu_6^C$ and $NEGF_8^C$.  At this step we distinguish
between arguments (\verb+args2+) that correspond to these selected variables,
and remove one of the arguments corresponding to each such selected variable.
The arguments sent to \indexfunction{make-clos-setvar-lemma-negf-7} correspond
to equations in the formula \verb+wff+ (originating with the inversion principle).

Since the constraint (LF1) has not banned variables, $Kuu_6^{(LF1)}$ 
is constructed as $Kuu_7^{(LF1)}$ 
by \indexfunction{make-clos-setvar-lemma-negf-7}.

The constraint (LF3 LF2) has the single banned variable $w$.
The corresponding selected variable $w^2$ is used to create $Kuu_6^{(LF3\, LF2)}$ as
$$\uexpnode{}{[\exists \,w [ S\, w^2 \,= \,S \,w \land \,u^1 \,w] \lor \,[S\,w^2] \,= \,0]}{w^2}{Kuu_7^{(LF3\, LF2)}}$$
where $Kuu_7^{(LF3\, LF2)}$ by \indexfunction{make-clos-setvar-lemma-negf-7}
(along with $NEGF_8^{(LF2)}$).

\item {\bf \indexfunction{make-clos-setvar-lemma-negf-7}}
The arguments that are sent to this function are those from the main
literal which were not the first argument corresponding to the selected
variables generated in \indexfunction{make-clos-setvar-lemma-negf-2}.
These arguments correspond to a conjunction of reflexive equations.  
If there is more than one literal in the constraint,more
conjuncts are formed when \indexfunction{make-clos-setvar-lemma-negf-8}
is called.

The first constraint (LF1) has no extra literals, so when this function is
called $Kuu_7^{(LF1)}$ with shallow (\verb+wff+) $0=0$
is simply constructed as
$$\erew{}{REFL=}{0 = 0}{\truenode{}}$$

The second constraint (LF3 LF2) does have the extra literal LF2.
When this is called $Kuu_7^{(LF3\, LF2)}$ which shallow (\verb+wff+)
$$[[S\, w^2]\, = \, [S\, w^2]]\,\land\, [u^1\, w^2]$$
is constructed as
$$\econj{}{}{\erew{}{REFL=}{}{\truenode{}}}{Kuu_8^{(LF2)}}$$
where $Kuu_8^{(LF2)}$ is constructed by
\indexfunction{make-clos-setvar-lemma-negf-8}.

\item {\bf \indexfunction{make-clos-setvar-lemma-negf-8}}
Given the left side $\Gamma$ of a constraint, a negative
ftree $Kuu_8^\Gamma$ and positive ftree $NEGF_8^\Gamma$ 
are constructed along with connections between them.
These are constructed as conjuncts for each literal in $\Gamma$
using \indexfunction{make-clos-setvar-lemma-negf-9} to
construct the children of the conjuncts.

In our example, $Kuu_8^{(LF2)}$ is simply $Kuu_9^{LF2}$ since there
is only one literal.  Similarly, $NEGF_8^{(LF2)}$ is $NEGF_9^{LF2}$.

\item {\bf \indexfunction{make-clos-setvar-lemma-negf-9}}
Usually this function is given two wffs which are $\alpha$-equal.
A positive leaf and a negative leaf are created, mated, and returned.
Another case is when set variable occurs embedded in the literal of the constraint.
In this case, the first wff \verb+wff1+ is of the form
$$\exists w^i_{\greeko\greeki}\, [[\forall \overline{x}\, .\, [w^i\, \overline{x}]\,\limplies\, [u^1_{\greeko\greeki}\, \overline{x}]]\,\land\, A(w^i)].$$
where the second wff \verb+wff2+ is $A(u^1)$.  In this case, we create
a positive ftree of the form
$$\uexpnode{}{\exists w^i_{\greeko\greeki}\, [[\forall \overline{x}\, .\, [w^i\, \overline{x}]\,\limplies\, [u^1_{\greeko\greeki}\, \overline{x}]]\,\land\, A(w^i)]}{u^1}{\econj{}{}
{\sel{}{\forall \overline{x}\, .\, [w^i\, \overline{x}]\,\limplies\, [u^1_{\greeko\greeki}\, \overline{x}]}{\overline{a}}{\eimp{}{}{LEAF1:[u^1\, \overline{a}]}{LEAF2:[u^1\, \overline{a}]}}}
{LEAF3:A(u^1)}}$$
and a negative leaf with shallow formula $A(u^1)$ which we mate to $LEAF3$.
Also, $LEAF1$ and $LEAF2$ are mated.
These two ftrees are returned.

In our example, the literal is LF2.  The formulas are both $u^1\, w^2$,
so we create $Kuu_9^{LF2}$ as a positive leaf and $NEGF_9^{LF2}$
as a negative leaf.  The two are mated and returned.

\item {\bf \indexfunction{make-clos-setvar-ind-negf}}  This function
is called with the least pre-fixed point property formula (\verb+lfwff+),
the induction property formula (\verb+indwff+), and the list of constraints (\verb+paths+).
The goal is to prove the induction property from the pre-fixed point
property by constructing a negative ftree $IndF^-$
for the induction property, a positive ftree $lff^+$
for the least pre-fixed point property,
and a complete set of connections between nodes in them.
The dynamic variable \verb+lff+ is set to $lff^+$.
$IndF^-$ is returned.

In our example, the least pre-fixed property is
$$\forall v_{\greeko\greeki}\, . \, \forall z\, [[\exists \,w [ \,x \,= \,S \,w \land \,v \,w] \lor \,x \,= \,0]
\, \limplies \, v\, z]
\,\limplies\, \forall z\, .\, u^1\, z\, \limplies \, v\, z$$
and the induction property is
$$\forall p\, .\, [\forall \,w^1 [ \,p \,w^1 \supset \,p . \,S \,w^1 ] \land \,p \,0]
\limplies\, \forall x^2\, . \, [u^1\, x^2]\,\limplies\, [p\, x^2].$$
The first thing we do is choose a new selected variable $p^1_{\greeko\greeki}$
(\verb+p+).
Let $P_1$ (\verb+lfpre+) be
$$\forall z\, [[\exists \,w [ \,z \,= \,S \,w \land \,p^1 \,w] \lor \,z \,= \,0]
\, \limplies \, p^1\, z],$$
$P_2$ be
$$\forall z\, .\, u^1\, z\, \limplies \, p^1\, z,$$
$I_1$ (\verb+indhyp+) be
$$\forall \,w^1 [ \,p^1 \,w^1 \supset \,p^1 . \,S \,w^1 ] \land \,p^1 \,0,$$
and $I_2$ be
$$\forall x^2\, . \, [u^1\, x^2]\,\limplies\, [p^1\, x^2].$$
We construct the positive ftree $lff^+$ as
$$\uexpnode{}{\mbox{Least Pre-fixed Point}}{p^1}
{\eimp{}{}{lff_1:P_1^-}{LEAF1:P_2^+}}$$
and the negative ftree $IndF^-$ as
$$\sel{}{\mbox{Induction Principle}}{p^1}
{\eimp{}{}{IndF_1:I_1^+}{LEAF2:I_2^-}}$$
where $lff_1$ and $IndF_1$ are constructed by \indexfunction{make-clos-setvar-ind-negf}.
Since $P_2$ and $I_2$ are $\alpha$-equal, we can mate $LEAF1$ and $LEAF2$.

\item {\bf \indexfunction{make-clos-setvar-ind-negf-1}}
We start with two wffs: \verb+wff1+ of the form
$C_1\,\land \cdots \land C_n$
\verb+wff2+ of the form
$$\forall \overline{z} \, . \, [D_1(\overline{z})\,\lor\,\cdots\,\lor\, D_n(\overline{z})]\,\limplies\, [p\, \overline{z}].$$
where $n$ is the number of constraints.
Let $\overline{a}$ be new selected variables and $LEAF1$ be a negative literal
with shallow formula $[p\, \overline{a}]$.  We return two values, 
a negative ftree of the form
$$\sel{}{\forall \overline{z} \, . \, [D_1(\overline{z})\,\lor\,\cdots\,\lor\, D_n(\overline{z})]\,\limplies\, [p\, \overline{z}]}
{\overline{a}}{NF_2}$$
and a positive ftree $PF_2$ with shallow formula \verb+wff1+
where $PF_2$ and $NF_2$ are constructed by
\indexfunction{make-clos-setvar-ind-negf-2}.

In our example, \verb+wff1+ is
$$\forall \,w^1 [ \,p^1 \,w^1 \supset \,p^1 . \,S \,w^1 ] \land \,p^1 \,0,$$
and \verb+wff2+ is
$$\forall z\, [[\exists \,w [ \,z \,= \,S \,w \land \,p^1 \,w] \lor \,z \,= \,0]
\, \limplies \, p^1\, z].$$
We choose a new selected variable $z^1_\greeki$ and create a negative leaf
$LEAF1:p^1\, z^1$.
We call \indexfunction{make-clos-setvar-ind-negf-2} with
$$\forall \,w^1 [ \,p^1 \,w^1 \supset \,p^1 . \,S \,w^1 ] \land \,p^1 \,0,$$
$$\exists \,w [ \,z^1 \,= \,S \,w \land \,p^1 \,w] \lor \,z^1 \,= \,0,$$
and $LEAF1$ to get two positive ftrees, $PF_3^1$ and $PF_3^2$.
We return the positive ftree
$PF_3^1$ and the negative ftree
$$\eimp{}{}{PF_3^2}{LEAF1}$$

\item {\bf \indexfunction{make-clos-setvar-ind-negf-2}}
We start with $n$ constraints (given by \verb+paths+ and \verb+banned-occurs+),
and two wffs, one (\verb+wff1+) of the form
$C_1\,\land \cdots \land C_n$,
and the other (\verb+wff2+) of the form
$$D_1(\overline{z})\,\lor\,\cdots\,\lor\, D_n(\overline{z}).$$
We consider each constraint
$$\Psi_1|\Gamma_i\rightarrow A_i,$$
conjunct $C_i$ and disjunct $D_i$.
We distinguish between variables $z\in\Psi_i$ which
occur as arguments in $A_i$ and those $w\in\Psi_i$ which do not.
When $z\in\Psi_1$ occurs as an argument in $A_i$,
the corresponding argument in the shallow formula of $LEAF1$
is a variable $a$ created in the previous step.
Let $\theta_1$ (\verb+psi-z-assoc+) be the substitution sending $\theta(z)=a$
for each such $z$ and $a$.
Those $w\in\Psi_i$ which do not occur as
arguments are stored on $\Psi^0_i$.  This information is passed
to \indexfunction{make-clos-setvar-ind-negf-3}
which returns two positive ftrees $PF_{D_i}$ and
$PF_{C_i}$ and returns
$$\econj{}{}{PF_{C_1}}{\econj{}{}{PF_{C_2}}{\etra{\cdots}{PF_{C_n}}}}$$
and
$$\edisj{}{}{PF_{D_1}}{\edisj{}{}{PF_{D_2}}{\etra{\cdots}{PF_{D_n}}}}$$

In our example, the two constraints are of the form
$$v^1\, 0$$
and
$$w|v^1\, w\rightarrow v^1\, [S\, w].$$
Neither $0$ nor $[S\, w]$ is a variable in the corresponding $\Psi$'s,
so $\theta_1$ is empty.  $\Psi^0$ is NIL in one case
and ($w$) in the other.

\item {\bf \indexfunction{make-clos-setvar-ind-negf-3}}
Given two wff's $C$ (\verb+wff1+) and $D$ (\verb+wff2+) and $\Psi^0 = \overline{w}$
where \verb+wff2+ is of the form
$\exists \overline{w^1} D'(\overline{w^1})$, we create new selected variables
$\overline{w^2}$.  Let $\theta_2$ (\verb+psi-w-assoc+) send each $w$ to $w^2$.
We call \indexfunction{make-clos-setvar-ind-negf-4}
with $C$ and $D'(\overline{w^2})$ to obtain
two positive ftrees $PF_4^1$ and $PF_4^2$ with shallow formulas
$C$ and $D'(\overline{w^2})$.  We then return $PF_4^1$ and 
$$\sel{}{\exists \overline{w^1} D'(\overline{w^1})}{\overline{w^2}}{PF_4^2}$$

The constraint $\emptyset |\cdot\rightarrow v^1\, 0$ has an empty $\Psi$,
so we proceed directly to \indexfunction{make-clos-setvar-ind-negf-4}
with
$\,p^1 \,0$
and
$\,z^1 \,= \,0$ to get the two positive ftrees.

The constraint $w |v^1\, w\rightarrow v^1\, . \, S\, w$ has the variable $w$
in $\Psi$, and $D$ is
$$\exists w [ \,z^1 \,= \,S \,w \land \,p^1 \,w].$$
We create a new selected variable $w^2$ and call 
\indexfunction{make-clos-setvar-ind-negf-4}
with
$$\forall \,w^1 [ \,p^1 \,w^1 \supset \,p^1 . \,S \,w^1 ]$$
and 
$$z^1 \,= \,S \,w^2 \land \,p^1 \,w^2.$$
The second positive ftree $PF_4^2$ constructed has
shallow $z^1 \,= \,S \,w^2 \land \,p^1 \,w^2$.  We return the
first positive ftree along with
$$\sel{}{\exists w [ \,z^1 \,= \,S \,w \land \,p^1 \,w]}{w^2}{PF_4^2}$$

\item {\bf \indexfunction{make-clos-setvar-ind-negf-4}}
We are given a constraint $\Psi|\Gamma\rightarrow A$ and two wff's $C$ and $D$
where $C$ is of the form
$\forall \overline{y} C' \limplies A'$
(or $\forall \overline{y} A'$ if $\Gamma$ is empty)
where $\Psi=\overline{y'}$.  Define a substitution $\theta$ with $dom(\theta) = \overline{y}$
as follows.
Either $y'\in\overline{y'}$ is a $z\in\Psi$ which occurs as an argument in $A$,
and so is in the domain of $\theta_1$ (\verb+psi-z-assoc+), or
$y$ is a $w\in\Psi$ in the domain of $\theta_2$ (\verb+psi-w-assoc+).
In the first case, let $\theta(y) = \theta_1(y')$.  In the second case,
let $\theta(y) = \theta_2(y')$.
We use these association lists to determine
the expansion term for $y$.  
Then \indexfunction{make-clos-setvar-ind-negf-5}
is called with $\theta(C')$ (or NIL) (\verb+hyp+) and $\theta(A')$ (\verb+conc+).
This constructs two positive ftrees, 
$PF_5^1$ with shallow $\theta(C'\limplies A')$
and a positive ftree $PF_5^2$ with shallow $D$.
We return
$$\uexpnode{}{C}{\theta(\overline(y))}{PF_5^1}$$
and $PF_5^2$.

Again, the constraint $\emptyset |\cdot\rightarrow v^1\, 0$ has an empty $\Psi$,
so we proceed directly to \indexfunction{make-clos-setvar-ind-negf-5}
with
$p^1 \,0$
and
$z^1 \,= \,0$ to get the two positive ftrees.

For the constraint $w |v^1\, w\rightarrow v^1\, . \, S\, w$,
$C$ is
$$\forall \,w^1 [ \,p^1 \,w^1 \supset \,p^1 . \,S \,w^1 ].$$
$\theta(w^1) = \theta(w) = w^2$ by definition.
\indexfunction{make-clos-setvar-ind-negf-5} then constructs
two positive ftrees, $PF_5^1$ with shallow formula
$[ \,p^1 \,w^2 \supset \,p^1 . \,S \,w^2 ]$
and $PF_5^2$ with shallow formula 
$z^1 \,= \,S \,w^2 \land \,p^1 \,w^2$.
We return
$$\uexpnode{}{\forall \,w^1 [ \,p^1 \,w^1 \supset \,p^1 . \,S \,w^1 ]}{w^2}{PF_5^1}$$
and $PF_5^2$.

\item {\bf \indexfunction{make-clos-setvar-ind-negf-5}}
This function starts with the negative leaf $LEAF1$ (\verb+leaf+)
constructed in \indexfunction{make-clos-setvar-ind-negf-1}.
This function compares the arguments in the shallow formula
$p(b_1,\ldots,b_n)$ of $LEAF1$ \verb+leaf+
and the given formula $p(a_1,\ldots,a_n)$ (\verb+conc+).  

We recursively compare the arguments.  Suppose we have a negative
leaf \verb+leaf+ with shallow formula
$p(a_1,\ldots,a_{i-1},b_i,\ldots,b_n)$
If $b_i$ and $a_i$ are syntactically equal,
then these correspond to some $z\in\Psi$.  Otherwise,
there should be an equation at the beginning of \verb+wff2+.
That is, either \verb+wff2+ is $[b_i\, =\,a_i]$ or $[b_i\,=\,a_i]\,\land \,D'.$
The function \indexfunction{make-ftree-subst} creates a positive ftree
with shallow $[b_i\, = \, a_i]$, a negative leaf $LEAF2$ with
shallow formula $p(a_1,\ldots,a_i,b_{i+1},\ldots,b_n)$
(and some new connections to add to the connection list).
Let $EQ_{i_j}$ be these positive ftrees for these equations.

So, after $n$ steps, we have a negative leaf (\verb+leaf+) with shallow formula
$p(a_1,\ldots,a_n)$.  We create a positive leaf $LEAF2$ (\verb+leaf2+) with this shallow
and mate the two leaves.  If $\Gamma$ is empty, we return 
this positive ftree along with the positive ftree
$$\econj{}{}{EQ_{i_1}}{\etra{\cdots}{\econj{}{}{EQ_{i_{m-1}}}{EQ_{i_m}}}}$$
If $\Gamma$ is nonempty, then the rest of $D$ (after removing the equations)
should be $\alpha$-equal to $C$ (\verb+hyp+).  We
create a negative leaf $LEAF3$ and a positive leaf $LEAF4$ with these shallows
and return the positive ftrees
$$\eimp{}{}{LEAF4}{LEAF2}$$
and the positive ftree
$$\econj{}{}{EQ_{i_1}}{\etra{\cdots}{\econj{}{}{EQ_{i_m}}{LEAF3}}}$$

In the constraint (LF1), $\Gamma$ is empty and $D$ is
$z^1 \,= \,0$.  The shallow formula of $LEAF1$ is $p^1 z^1$.
We construct $EQ_1$ as
$$\erew{}{z^1\,=\, 0}{Leibniz=}
{\erew{}{[\lambda x\,\forall y\, .\forall q\,.\,q\,x\,\limplies\,q\,y]\, z^1\, 0}{\lambda}
{\uexpnode{}{\forall q\,.\,q\,z^1\,\limplies\,q\,0}{\lambda x\,.\,\lnot\,[p^1\,x]}
{\erew{}{[\lambda x\,.\,\lnot\,[p^1\,x]]\,z^1\,\limplies\,[\lambda x\,.\,\lnot\,[p^1\,x]]\,0}
{\lambda}
{\eimp{}{}{\eneg{}{}{LEAF3^{+}}}
{\eneg{}{}{{LEAF1'}^{-}}}
}}}}$$
where $LEAF3^+$ mates to $LEAF1^-$.  
We also create a positive leaf $LEAF2$ with shallow formula $p^1\, 0$
and mate this to ${LEAF1'}^-$.  We return $LEAF2$ and $EQ_1$.

In the constraint (LF3 LF2), $D$ is 
$z^1 \,= \,S \,w^2 \land \,p^1 \,w^2$.
As above, a positive ftree $EQ_2$ for the equation $z^1\, = \, S\, w^2$
giving a mate for $LEAF1^-$ and a negative leaf $LEAF3^-$ with shallow
formula $p^1\, .\, S\, w^2$.  We also create a positive leaf $LEAF2^+$
to mate with $LEAF3$.  The ``leftover'' part of $D$, $p^1\, w^2$ corresponds
to $\Gamma$.  We create positive and negative leaves $LEAF4^+$ and $LEAF5^-$
with this shallow formula and mate the two.  We return the two positive ftrees
$$\eimp{}{}{LEAF5^-}{LEAF2^+}$$
and
$$\econj{}{}{EQ_2}{LEAF4^+}.$$

\end{itemize}

\item {\bf \indexfunction{make-knaster-tarski-leastfp-lemma}}

\end{enumerate}