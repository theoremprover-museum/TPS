Here is a summary of what happens after matingsearch has terminated with
a proof. 
The functions involved are located in the files {\it mating-merge.lisp},
{\it mating-merge2.lisp} and {\it mating-merge-eq.lisp}.

\begin{enumerate}
\item Apply Pfenning's Merge algorithm ({\tt etr-merge}), put resulting
      etree and mating in variable \indexother{current-eproof}
  \begin{enumerate}
\item  Get a list of the connections from mating

\item Get a list of substitutions required
      \begin{enumerate}
\item  Extract substitutions from unification tree
        
\item  Replace occurrences of PI and SIGMA by quantifiers
        
\item  Lambda-normalize each substitution
        
\item  Alpha-beta-normalize each substitution
      \end{enumerate}
    
\item  Prune any unused expansions from the tree
    
\item  Make substitutions for variables
    
\item  Carry out merging ({\tt merge-all})
  \end{enumerate}

\item  Replace skolem terms by parameters (if applicable) 
({\tt subst-skol-terms})

\item  If remove-leibniz is T, apply Pfenning's algorithm for removing
      the Leibniz equality nodes for substitution of equality nodes
      ({\tt remove-leibniz-nodes})

\item  Try to replace selected parameters by the actual bound variables
      ({\tt subst-vars-for-params}), not always possible because of restriction
      that a parameter should appear at most once

\item  Raise lambda rewrite nodes, so that in natural deduction the lambda
      normalization occurs as soon as possible. ({\tt raise-lambda-nodes})

\item  Clean up the etree ({\tt cleanup-etree}).  For each expansion term
      in the tree,
   \begin{enumerate}
\item  Lambda-normalize it
      
\item  Minimize the superscripts on bound variables
      
\item Make a new expansion with the new term
      
\item  Deepen the new expansion like the original, but removing 
            unnecessary lambda-norm steps.
      
\item  Remove the original expansion
   \end{enumerate}

 Begin natural deduction proof, using \indexother{current-eproof}
   \begin{itemize}
\item  Set up planned line 
	\begin{enumerate}
\item  Use shallow wff of the current-eproof's etree
          
\item  Give it the tree as value for its NODE property
          
\item  Give it the current-eproof's mating (list of pairs of node names)
                as its MATING property
	\end{enumerate}
   \end{itemize}

 Call {\tt use-tactic} with tactic desired
  \begin{enumerate}
\item  Each line in the proof will correspond to a node in the etree; the 
natural deduction proof is stored in the variable \indexother{dproof}.
    
\item   Here is an important property which should remain invariant during
    the translation process:  It should always be the case that the 
    line-mating of the planned line is a p-acceptable mating for the
    etree that one could construct by making an implication whose antecedent
    is the conjunction of the line-nodes of the supports, and whose 
    consequent is the line-node of the planned line.  This will assure us
    that we have sufficient information to carry out the translation.
  \end{enumerate}
\end{enumerate}

It was observed that when path-focused duplication had been
used, the expansion proof would often have a great deal of redundancy
in the sense that the same expansion term would be used for a given variable
many times. More precisely, if one defines an expansion branch by
looking at sequences of nested expansion nodes, attaching one expansion
term to each expansion node in the sequence, there would be many identical
expansion branches. 

In response to this, {\it mating-merge.lisp} was modified in the following ways:
\begin{itemize}
\item Don't do pruning of unnecessary nodes at the beginning of the merge,
when the tree is its greatest size. 

\item Instead, prune all branches that couldn't possibly have been used; 
they are those that have a zero status. This is probably not necessary,
but certainly makes debugging easier and doesn't cost much.

\item After merging of identical expansions has been done, call the original
pruning function.
\end{itemize}
