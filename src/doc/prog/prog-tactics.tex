


\subsection{Implementation of tactics and tacticals}

\begin{enumerate}
\item Main files (in order of importance): tactics-macros, tacticals,
tacticals-macros, tactics-aux.  These files contain tactic-related functions 
of a general nature.  Most tactics are actually contained in other Lisp packages.

\item When a tactic is executed, two global variables affect its
execution: tacuse (the tactic's use), and tacmode (the current mode).
Tacuse determines for what reason the tactic is being called.  Current
uses are etree-nat (translation of \indexother{eproof} to natural deduction) and
nat-ded (construction of a natural deduction proof without any mating
information).  A single tactic may be defined for more than one use.
Tacmode can have the value of either auto or interactive.  Each tactic
should take this value into account during operation.  In general,
this means that when the value is interactive, the user should be
advised that the tactic is about to be applied and should be allowed
to abort it.  When the value is auto, the tactic should just be
carried out if applicable.  

\item For each use, a number of auxiliary functions needed by the
tacticals must be defined.

\begin{enumerate}
\item get-tac-goal: if a goal has not been specified, get the next one,
e.g., the active planned line.

\item copy-tac-goal: copy the current goal into a new goal, so that subsequent
actions can be performed without destroying the current goal.  Allows
later backtracking if necessary.

\item save-tac-goal: put the current goal into a form suitable for
saving.  This is not actually used by current tactics.

\item restore-tac-goal: backtrack to the previous goal. 
This is not actually used by current tactics.

\item update-tac-goal: given the old (saved) goal, and the new goal on which
some progress has been made, update the old goal to reflect the
progress made.
\end{enumerate}

Tacticals must be independent of the value of tacuse.  
They cannot make any assumptions
about the structure of the goals, etc.

The main function used in applying tactics is apply-tactic.  This
is a function that takes a tactic as argument, and allows keyword
arguments of :goal, :use and :mode.  If not specified, the use and
mode default to the global values of tacuse and tacmode.
If they are specified,
the values given then override the global values of tacuse and tacmode.
apply-tactic and (every tactic) returns four values.  
 The first is a
list of goals, the second a string with some kind of message, the
third a token which indicates the result of the tactic and the fourth
a validation, which, if non-nil, should be a function which specifies
how solutions to the returned goals can be combined to solve the original goal.

apply-tactic works as follows:

\begin{enumerate}
\item  Checks that tactic is a valid tactic.

\item If a goal has not been specified, calls get-tac-goal.

\item If the tactic is an atom:
  \begin{enumerate}
\item     gets the tactic's definition for the use.

\item     if the definition is primitive, and the goal is nil, return
(nil "Goals exhausted." 'complete nil).  If the goal is non-nil, apply the 
tactic to the goal.

\item     if the definition is compound, call apply-tactical on the
definition and the goal.
  \end{enumerate}

\item  If the tactic's definition begins with a tactic, call apply-tactic
recursively,  using those optional arguments.

\item  If the tactic begins with a tactical, call apply-tactical.
\end{enumerate}

 Whenever a tactic begins with a tactical, the function
apply-tactical is used.  It takes two arguments, a goal and a tactic.
It is assumed that the tactic begins with a tactical.  
apply-tactical works as follows:


\begin{enumerate}
\item Get the definition for the tactical.

\item If the definition is primitive (a lambda expression), funcall the
definition on the goal and the remainder (cdr) of the tactic. 

\item If the definition is compound (i.e., is defined in terms of other
tacticals and begins with tac-lambda), expand the definition,
substituting the arguments provided in the tactic's definition for the
dummy  arguments in the tactical's definition.  Then call
apply-tactical recursively.

\item Otherwise abort, returning abort as the token (third value
returned).
\end{enumerate}

\item  Validations:  Though the validation mechanism is in place, no use
is made of them in the current tactic uses, since any changes in the
constructed proofs are made immediately, not saved.  Validations must
be modified as tacticals are executed, since during their execution,
the order of goals may be changed.  For example, a tactical may
repeatedly apply a tactic to a goal, then to all the new goals
created, etc., until it fails on all of them.  When it succeeds on a
successor goal, the validation returned must be integrated into the validation
which was returned for the first application of the tactic on the
original goal.  The function make-validation is used for this purpose.
\end{enumerate}
