\section{Test: Multiple Searches on the Same Problem}\label{testtop}

The \indexcommand{TEST} top level is designed to allow a succession of searches with
varying flag settings to be performed without any interaction from the user.

One of the major uses of this top level is to find settings of the various
flags which will produce a proof of a theorem; this is done by the
\indexcommand{UNIFORM-SEARCH} command, which is explained later in
section \ref{uniform}.

The top level is entered with the \indexcommand{TEST} command; as with
the mate top level, this will prompt for a gwff and ask whether you
want to open a vpwindow.

There are three major groups of TEST commands; those which are for
specifying a list of flags to be varied, those which deal with the
associated library functions and those which actually perform the search.

There is one major new data structure: a searchlist is a list of search items;
each search item contains the name of a flag, a default (initial) value for the flag,
and a range of values over which it should be varied. The user can have
several searchlists in memory at once, and can switch between them
with the \indexcommand{NEW-SEARCHLIST} command. See below for an example of how to
construct a searchlist. Searchlists can be stored in the library with the
\indexcommand{INSERT} command (from the test top level), and retrieved with the
\indexcommand{FETCH} command (also from the test top level). There is an additional
(optional) function attached to each searchlist; this will be called on each iteration of
the search, and may do such things as setting flags which are not in the searchlist (so, for
example, \indexflag{MAX-SEARCH-DEPTH} and \indexflag{MAX-UTREE-DEPTH} can be kept equal to
each other by having one in the searchlist and the other set by the optional function).
The function UNIFORM-SEARCH-FUNCTION is used by the UNIFORM-SEARCH procedures.

The searchlist also contains (invisible to the user) internal flags which determine the current
position in the search; this means that it is possible to interrupt a search,
save the current searchlist in the library, and later on to reload that searchlist
(even into a different core image) and continue the search from the point where it was
interrupted.

Once a searchlist is constructed, the \indexcommand{GO} command will start a search
and the \indexcommand{CONTINUE} command will continue a search after an interruption.
The flags in the subject \indexsubject{test-top} control many of the parameters of the
search; type {\tt list test-top} for more information. The user has the option of opening
a test-window, analogous to the vpwindow, which will show a summary of the search so far.
If the search finds a proof, it will terminate; otherwise, it will increase the time limit
in the flag \indexflag{TEST-INITIAL-TIME-LIMIT} and try again. Once the search terminates,
it will define a new mode which can be saved in the library using \indexcommand{INSERT}
{\it from the test top level}. (The \indexcommand{INSERT} in the library top level is different;
when you try to save a mode, it asks you to specify all the flag settings. The \indexcommand{INSERT}
in the test top level merely asks for the name of the mode.)

\subsection{How to Build A Searchlist Without Any Effort}

The commands \indexcommand{VARY-MODE} and \indexcommand{QUICK-DEFINE} are
useful for defining searchlists quickly. The former takes an existing mode,
and steps through it flag by flag, building a searchlist by offering to
add each flag of the mode and an appropriate range to the new searchlist.
The latter uses a pre-defined list of flags, either those listed in the
\indexflag{TEST-FASTER-*} group or those listed in the \indexflag{TEST-EASIER-*}
group, to create a searchlist in which each of the flags has up to
\indexflag{TEST-MAX-SEARCH-VALUES} possible values.

Given a large searchlist, it is possible to trim it down using the \indexcommand{SCALE-UP}
and \indexcommand{SCALE-DOWN} commands; for each of the flags in the TEST-EASIER-* or TEST-FASTER-*
lists (respectively), these commands compare the range of the flags in the
searchlist with the current values of the flags, and remove all values that would not make the
search easier (in the case of SCALE-UP) or faster (in the case of SCALE-DOWN).

In some cases, it is not necessary to build a searchlist at all. The commands \indexcommand{PRESS-DOWN},
\indexcommand{PUSH-UP} and \indexcommand{FIND-BEST-MODE} all build their own searchlists
using the flags listed in the TEST-FASTER-* flags (for PRESS-DOWN and FIND-BEST-MODE)
and those listed in the TEST-EASIER-* flags (for PUSH-UP and FIND-BEST-MODE). Below are
examples of how to use these commands.

\subsection{Using TEST to Improve a Successful Mode}
\begin{tpsexample}
<18>test sample-theorem !
<test19>mode mode-sample-theorem
{\it mode-sample-theorem is a mode in which sample-theorem can be proven.}
<test20>list test-top

Subject: TEST-TOP
  TEST-FASTER-IF-HIGH:   (MIN-QUICK-DEPTH)
  TEST-FASTER-IF-LOW:    (MAX-SEARCH-DEPTH SEARCH-TIME-LIMIT MAX-UTREE-DEPTH
                          MAX-MATES MAX-SEARCH-LIMIT)
  TEST-FASTER-IF-NIL:    ()
  TEST-FASTER-IF-T:      (MIN-QUANTIFIER-SCOPE MS-SPLIT)
  TEST-FIX-UNIF-DEPTHS:  T
  TEST-INCREASE-TIME:    0
  TEST-INITIAL-TIME-LIMIT: 30
  TEST-MAX-SEARCH-VALUES: 10
  TEST-REDUCE-TIME:      T
{\it Some flags have been omitted. When we do PRESS-DOWN, it will
automatically create a searchlist which will vary each
of the flags listed in the TEST-FASTER-* flags above, and then
start searching with a maximum time of 30 seconds, decreasing the
time as it goes, and fixing the unification depths after the
first successful search.}
<test21>press-down
{\it The search is started. While it works, we can go away for a
cup of coffee (or a two-week vacation to Mexico, depending on how
difficult sample-theorem is).}
\end{tpsexample}

\subsection{Using TEST to Discover a Successful Mode}
\begin{tpsexample}
<18>test sample-theorem !
<test19>mode bad-mode
{\it bad-mode is a mode in which sample-theorem cannot be proven.}
<test20>list test-top

Subject: TEST-TOP
  TEST-EASIER-IF-HIGH:   (MAX-SEARCH-DEPTH SEARCH-TIME-LIMIT NUM-OF-DUPS
                          MAX-UTREE-DEPTH MAX-MATES MAX-SEARCH-LIMIT)
  TEST-EASIER-IF-LOW:    (MIN-QUICK-DEPTH)
  TEST-EASIER-IF-NIL:    ()
  TEST-EASIER-IF-T:      (ETA-RULE MIN-QUANTIFIER-SCOPE MS-SPLIT)
  TEST-INCREASE-TIME:    10
  TEST-INITIAL-TIME-LIMIT: 30
  TEST-MAX-SEARCH-VALUES: 10
{\it Some flags have been omitted. When we do PUSH-UP, it will
automatically create a searchlist which will vary each
of the flags listed in the TEST-EASIER-* flags above, and then
start searching with a maximum time of 30 seconds, increasing the
time by ten percent on each attempt.}
<test21>push-up
{\it The search is started. It will stop as soon as it finds a mode which works.
We can combine both PUSH-UP and PRESS-DOWN (first push-up to find a successful
mode, then press-down to find a better mode) by using the command FIND-BEST-MODE
instead of PUSH-UP.}
\end{tpsexample}

\subsection{Building A Searchlist with TEST}
\begin{tpsexample}
<12>list test-top
Subject: TEST-TOP
  TEST-INITIAL-TIME-LIMIT: 30
  TEST-NEXT-SEARCH-FN:   EXHAUSTIVE-SEARCH
  TEST-REDUCE-TIME:      T
  TEST-VERBOSE:          T
  TESTWIN-HEIGHT:        24
  TESTWIN-WIDTH:         80
{\it These are the flags available. Note the first three; the first says that
each search will be allowed 30 seconds, the second that the function
EXHAUSTIVE-SEARCH will be used to determine which flag settings are used next,
and the third that if a proof is found, the time allowed for the next proof
will be decreased.}
<13>test
GWFF (GWFF0-OR-EPROOF): Gwff or Eproof [No Default]>x5305
DEEPEN (YESNO): Deepen? [Yes]>
WINDOW (YESNO): Open Vpform Window? [No]>yes
File to send copy Vpform output to (`` '' to discard) [``vpwin.vpw'']>``test-top.vpw''

Use CLOSE-MATEVPW when you want to close the vpwindow.
<test14>?
Top Levels:    LEAVE
Mating search: CLOSE-TESTWIN CONTINUE GO OPEN-TESTWIN SEARCH-ORDER
Searchlists:   ADD-FLAG ADD-FLAG* ADD-SUBJECTS NEW-SEARCHLIST REM-FLAG
               REM-FLAG* SEARCHLISTS SHOW-SEARCHLIST
Library:       DELETE FETCH INSERT

<test15>new-searchlist
NAME (SYMBOL): Name of new searchlist [No Default]>x5305-search
{\it We begin to define a new searchlist, containing the flags to be varied.}
<test16>searchlists
X5305-SEARCH

{\it ...this is currently the only searchlist in memory...}
<test17>add-flag*
FLAG (TPSFLAG): Flag to be added [No Default]>max-search-depth
INIT (ANYTHING): Initial value of flag [No Default]>42
RANGE (ANYTHING-LIST): List of possible values to use [No Default]>4 8 10

Add another flag?  [Yes]>

FLAG (TPSFLAG): Flag to be added [No Default]>max-utree-depth
INIT (ANYTHING): Initial value of flag [No Default]>20
RANGE (ANYTHING-LIST): List of possible values to use [No Default]>4 8 10

Add another flag?  [Yes]>no

{\it We have added two flags to our searchlist, with four values for each.
Note also that there are commands to add a single flag (ADD-FLAG) or
every flag in a given list of subjects (ADD-SUBJECTS), as well as
commands to remove a flag or flags (REM-FLAG, REM-FLAG*).}
<test18>show-searchlist
NAME (SYMBOL): Name of searchlist [X5305-SEARCH]>
Searchlist X5305-SEARCH is as follows:
MAX-UTREE-DEPTH = 20, default is 20, range is [(20 4 8 10)]
MAX-SEARCH-DEPTH = 42, default is 42, range is [(42 4 8 10)]

{\it Next, we decide to check how many searches there will be, and in what
order they will be performed.}
<test19>search-order
NAME (SYMBOL): Name of searchlist [X5305-SEARCH]>
There are 16 possible settings of these flags.
Search : (MAX-UTREE-DEPTH = 20) (MAX-SEARCH-DEPTH = 42)
Search : (MAX-UTREE-DEPTH = 10) (MAX-SEARCH-DEPTH = 42)
Search : (MAX-UTREE-DEPTH = 8) (MAX-SEARCH-DEPTH = 42)
Search : (MAX-UTREE-DEPTH = 4) (MAX-SEARCH-DEPTH = 42)
Search : (MAX-UTREE-DEPTH = 20) (MAX-SEARCH-DEPTH = 10)
Search : (MAX-UTREE-DEPTH = 10) (MAX-SEARCH-DEPTH = 10)
Search : (MAX-UTREE-DEPTH = 8) (MAX-SEARCH-DEPTH = 10)
Search : (MAX-UTREE-DEPTH = 4) (MAX-SEARCH-DEPTH = 10)
Search : (MAX-UTREE-DEPTH = 20) (MAX-SEARCH-DEPTH = 8)
Search : (MAX-UTREE-DEPTH = 10) (MAX-SEARCH-DEPTH = 8)
Search : (MAX-UTREE-DEPTH = 8) (MAX-SEARCH-DEPTH = 8)
Search : (MAX-UTREE-DEPTH = 4) (MAX-SEARCH-DEPTH = 8)
Search : (MAX-UTREE-DEPTH = 20) (MAX-SEARCH-DEPTH = 4)
Search : (MAX-UTREE-DEPTH = 10) (MAX-SEARCH-DEPTH = 4)
Search : (MAX-UTREE-DEPTH = 8) (MAX-SEARCH-DEPTH = 4)
Search : (MAX-UTREE-DEPTH = 4) (MAX-SEARCH-DEPTH = 4)

{\it Because TEST-NEXT-SEARCH-FN is EXHAUSTIVE-SEARCH, the search will try
every possible combination of these flags, as shown above.}

<test20>go
MODENAME (SYMBOL): Name for optimal mode [TEST-BESTMODE]>x5305-bestmode
TESTWIN (YESNO): Open a window for test-top summary? [No]>yes
File to send test-top summary to (`` '' to discard) [``info.test'']>

Use CLOSE-TESTWIN when you want to close the testwindow.

Changing flag settings as follows:
(MAX-UTREE-DEPTH = 20) (MAX-SEARCH-DEPTH = 42)

{\it lots of output omitted...}

The time used in each process:
-----------------------------------------------------------------------------
Process Name         | Realtime | Internal-runtime |  GC-time   | I-GC-time
-----------------------------------------------------------------------------
                                (Interrupted)
Mating Search        |       77 |            52.13 |       0.00 |      52.13
                      (1.3 mins)
-----------------------------------------------------------------------------

Changed MAX-UTREE-DEPTH = 20, default is 20, range is [(20 4 8 10)]

Changed MAX-SEARCH-DEPTH = 42, default is 42, range is [(42 4 8 10)]
Search finished.Changing flag settings as follows:
(MAX-UTREE-DEPTH = 20) (MAX-SEARCH-DEPTH = 42)
Finished varying flags; succeeded in proof.
Best mode was ((MAX-SEARCH-DEPTH 42) (MAX-UTREE-DEPTH 8))
Best time was 28.667

Have defined a mode called X5305-BESTMODE; use INSERT to put this into the library.

{\it We ran the search; it terminated and defined a new mode for us.}
<test21>help X5305-BESTMODE
X5305-BESTMODE is a mode.
A mode resulting from a test-search.
Flags are set as follows:
   Flag                     Value in Mode         Current Value
   MAX-SEARCH-DEPTH         42                    42
   MAX-UTREE-DEPTH          8                     20

<test22>close-testwin
Closed test-window file : info.test
Shall I delete the output file info.test?  [No]>yes

\end{tpsexample}

\subsection{Uniform Search: Finding Successful Modes Automatically}\label{uniform}

There are two top-level commands in {\TPS} which search for a successful mode for
proving a theorem, without requiring the user to master the \indexcommand{TEST} top level first.
They are \indexcommand{UNIFORM-SEARCH} and \indexcommand{UNIFORM-SEARCH-L}.

\indexcommand{UNIFORM-SEARCH-L} is analogous to \indexcommand{DIY-L}; it attempts
to find a successful mode which will allow {\TPS} to fill in a subproof during the
interactive construction of a larger theorem. Apart from the fact that it generates
subproofs within a given range of lines, it works in exactly the same way as
\indexcommand{UNIFORM-SEARCH}, and so the rest of this section is devoted entirely to
\indexcommand{UNIFORM-SEARCH}.

\indexcommand{UNIFORM-SEARCH} takes three essential arguments: a gwff which is to be proven,
a mode and a searchlist. By default, the mode is called UNIFORM-SEARCH-MODE and the
searchlist is called UNIFORM-SEARCH-2; if these objects are not present in your library,
you should provide appropriate alternatives.

The idea is that {\TPS} will set all of the flags to the values given in UNIFORM-SEARCH-MODE,
and then vary the flags as prescribed by UNIFORM-SEARCH-2. So UNIFORM-SEARCH-MODE should
be a `neutral' mode which does not constrain the search very much, and which sets all of
the less important flags to reasonable values.

UNIFORM-SEARCH-2 might be defined as follows (for example):
\begin{tpsexample}
DEFAULT-MS = MS91-6, default is MS91-6, range is [(MS91-6 MS91-7)]
MAX-SUBSTS-VAR = 3, default is 3, range is [(3 5 7)]
MAX-MATES = 1, default is 1, range is [(1 2 3 4 6)]
NUM-OF-DUPS = 0, default is 0, range is [(0 1 2)]
REWRITE-EQUALITIES = ALL, default is ALL, range is [(ALL LAZY2 LEIBNIZ)]
REWRITE-DEFNS = (EAGER), default is (EAGER), range is [((EAGER) (LAZY2))]
SEARCH-TIME-LIMIT = 30, default is 30, range is [(30 120 240 920 1840 3600 7200)]
...plus the function UNIFORM-SEARCH-FUNCTION
\end{tpsexample}

As you can see, this varies the most important flags in {\TPS} over a reasonable range of values.
Once you have entered the TEST top level with UNIFORM-SEARCH, all that remains is to type GO,
and wait for a proof. Of course, proofs in UNIFORM-SEARCH usually take longer to be found than proofs in which
the correct mode is already known.

If a proof is found, a new mode will be defined, which can be stored in the library, and by merging the etree
and then calling NAT-ETREE this proof can be translated into a natural deduction proof, exactly
as in the MATE top level.

One final word about UNIFORM-SEARCH: it will also offer to modify the searchlist for you.
This will speed up the search, if possible, by removing those flags in the searchlist
which will have no effect on the proof of the given gwff. In particular, it will remove
unification depths for first-order gwffs, REWRITE-DEFNS for gwffs that contain no definitions,
REWRITE-EQUALITIES for those that contain no equalities, and so on. It will also change the
settings of DEFAULT-MS to MS88 and/or MS90-3 if it determines that there are no primitive
substitutions for the given gwff.

