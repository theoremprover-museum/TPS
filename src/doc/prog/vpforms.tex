\section{Simple MetaWffs in TPS3}

Even though in \TPS the principle metalanguage is of course Lisp, it is
often convenient to be able to use simple notations from the metalanguage
and include them directly in the input format for Wffs.  In \TPS this is
achieved by providing a notation for certain kinds of {\tt WFFOPS} inside
an external specification of a wff.  This method is not perfect, but has
other advantages as well, as we shall see.

\subsection{The Notation}

The motivation behind the notation is an analogy to Lisp: we use the
backquote to introduce some Lisp form which is to be evaluated and
inserted into the Wff.  One restriction is that the wffop must return
a {\it gwff} (or a subtype, like a {\it gvar}).  The other is that \TPS must
have certain pieces of knowledge about the {\it wffop} used, in order to be
able to determine the type of the result of applying the {\it wffop}.

Some examples of external format and what they are parsed to:
\begin{verbatim}
"forall x. `(lcontr [[lambda x. P x x x] [f x]])"
\end{verbatim}
to
$$\forall x.P\; [f\; x]\; [f\; x]\; [f\; x]$$

\begin{verbatim}
"forall x exists y.
  `(lexpd z [f x] `(lexpd z [f x] [Q [f x] [f x] y] `(1)) `t)"
\end{verbatim}
to
$$\forall x\exists y.[\lambda z\; [\lambda z^{1}\; Q\; z^{1}\; z\; y]\; z]\; [f\; x]$$

\begin{verbatim}
"`(substitute-types `((A . (O . I))) [P(OA) subset Q(OA)])"
\end{verbatim}
to
$$P_{\greeko(\greeko\greeki)} \subseteq Q_{\greeko(\greeko\greeki)}.$$
(The latter could have been more easily specified as
\begin{verbatim}
(substitute-types (("A" "OI")) "[P(OA) subset Q]")
\end{verbatim}
but that is no longer possible when the formula is to be embedded
in another.)


Here are the general rules:
\begin{itemize}
\item In an ordinary wff, a backquote may precede something of the form
{\wt ({\it wffop} {\it arg} ... {\it arg})}, where {\it wffop} has all the necessary
type information.  The typecase of {\it wffop} is irrelevant.

\item Among {\wt ({\it arg} ... {\it arg})}, each argument is either a gwff (and may
contain other backquoted expressions) or a Lisp expression, which
is considered a constant.  This is necessary to supply arguments which
are not gwffs to a {\it wffop}.  Notice, that it must be the internal
representation of the argument!\footnote{At some point one could work at
removing this restriction, if types are handled properly.}
\end{itemize}

\section{More about Jforms}

Much of the code for handling jforms is in {\it jforms-labels.lisp}; see
{\tt defflavor jform} in this file for the definition.

In the same file we see:

%\begin{tpsexample}
\begin{verbatim}
(eval-when (load compile eval)
  (defflavor disjunction
    (mhelp "A disjunction label stands for a disjunction of wffs.")
    (inherit-properties jform)
    (include jform (type 'disjunction))
\end{verbatim}
%\end{tpsexample}

This tell us that a jform can be a disjunction.




