\subsection{Customizing {\ETPS} or {\TPS} with your own rules}

Suppose that we wanted to set up a logical system with just one rule,
{\it modus ponens}.  Here's how we would go about it.

First we find the definition of the rule MP, which is located in one
of the files with the extension {\tt .rules}.  Let's make some minor
modifications to it for our new system. This is what it looks like:
\begin{alltt}
(defirule modpon
  (lines (p1 (h) () `A')
	 (d2 (h) () `A implies B')
	 (d3 (h) () `B' (`Modus Ponens' () (p1 d2))))
  (support-transformation (('pp d2 'ss)) ((p1 'ss) ('pp d3 'ss p1)))
  (itemshelp (p1 `Line with Antecedent of Implication')
	     (d2 `Line with Implication')
	     (d3 `Line with Succedent of Implication')
	     (`A(O)' `Antecedent of Implication')
	     (`B(O)' `Succedent of Implication'))
  (mhelp `The rule of Modus Ponens.'))
\end{alltt}

We place the rule definition in a new file, say {\tt mp.rules}.  Now we
need to generate the Lisp functions that carry out the rule.  At the
{\ETPS} top level, we do the following:
\begin{alltt}
;;; Set up flags to read the new rule properly
<0> mode rules
<1>assemble-file
RULE-FILE (FILESPEC): Rule source file to be assembled [`.rules']>mp
PART-OF (SYMBOL): Package the file is part of [OTLSUGGEST]>newpackage
MODPON
Written file /afs/cs.cmu.edu/user/nesmith/tps/mp.lisp.
\end{alltt}

In order to put this new rule into a new {\lisp} package, we add the following
lines to the beginning of the file {\tt mp.lisp}:
\begin{alltt}
(unless (find-package `MY-RULES')
  (make-package `MY-RULES' :use (package-use-list (find-package `ML'))))
(in-package 'my-rules)
\end{alltt}

We also should put the new rule file into a separate {\TPS} package, by
adding the following to the file {\tt defpck.lisp}:
\begin{alltt}
(def-lisp-package my-rules
  (needed-lisp-packages core auto)
  (mhelp `My rules.'))

(defmodule my-rules
  (needed-modules math-logic-2-wffs theorems replace)
  (lisp-pack my-rules)
  (files mp)
  (mhelp `Defines my rules.'))
\end{alltt}

Now we load these changes into {\ETPS}:
\begin{alltt}
<2> qload defpck
Loading stuff from \#<File stream `/afs/cs.cmu.edu/user/nesmith/tps/defpck.lisp'>.
<3> sys-load my-rules
<4> use-package 'my-rules
\end{alltt}

Now the rules in the package my-rules are available for use.  We can have them
loaded each time {\ETPS} starts up by adding the following lines to the
file {\tt etps.patch}:
\begin{alltt}
(qload `defpck')
(sys-load-package 'my-rules)
(use-package 'my-rules)
(unuse-package 'ml)
\end{alltt}

The last form above will make the current rules unavailable to the user.

We could also build a version of {\ETPS} which uses these rules by loading
my-rules instead of math-logic-2-rules.  This change can be made by
modifying the file {\tt etps-build.lisp} in the appropriate place.

Notice that if two rules in different packages (or even different modules)
have the same name, and both are loaded into the same core image, the last one
loaded will be the one that is available. (In particular, in the standard {\TPS},
the modules {\tt math-logic-1} and {\tt math-logic-2} conflict in this way, although
{\tt math-logic-1} is normally not loaded.)

\subsection{Creating Exercises}

In general, you may use \indexfunction{DEFTHEOREM} to define theorems from the book
the student may want to use \indexother{BOOK-THEOREM}, practice exercises \indexother{PRACTICE},
exercises \indexother{EXERCISE} and test problems \indexother{TEST-PROBLEM}. This is indicated
in the value of the \indexother{THM-TYPE} property. There is one other value of \indexother{THM-TYPE},
which is \indexother{LIBRARY}, indicating that the theorem was loaded from the library.

You may specify the amount of advice given and the rules excluded by writing
the appropriate Lisp function and making its name the value of the
\indexother{ALLOWED-CMD-P} property. Some functions are already defined, and are described below.

By a similar device with the \indexother{ALLOWED-LEMMA-P} property,
you may specify which theorems may be asserted legally and used with
{\tt SUBST-WFF} to help in the proof.

The \indexother{ASSERTION} property should be given a wff in quotes, the assertion
of the theorem. There are other properties \indexother{SCORE}, \indexother{REMARKS},
\indexother{FOL} and \indexother{MHELP}. The first two are useless, but are meant to contain
the maximum score for the exercise (GRADER currently ignores this) and any remarks from
the teacher. The third should be T if the theorem is first-order, and the last is the usual
help message.

Here are some examples:

\begin{tpsexample}
(deftheorem X6004
  (assertion
   ` eqp [= x(B)] [= y(A)] ')
  (thm-type exercise)
  (allowed-cmd-p allow-all)
  (allowed-lemma-p allow-no-lemmas))

(deftheorem X5206
  (assertion
   ` \% f(AB) [x union y] = . [\% f x] union [\% f y] ')
  (thm-type exercise)
  (allowed-cmd-p allow-all)
  (required-lemmas (x5200 x5204))
  (allowed-lemma-p only-required-lemmas))
\end{tpsexample}

The functions for \indexother{allowed-cmd-p} are as
follows. Notice that for practice theorems (i.e. those
with THM-TYPE PRACTICE) these generally behave like ALLOW-ALL.

\begin{description}
\item[\indexfunction{ALLOW-ALL} ]	 allows all commands

\item[] \indexfunction{ALLOW-RULEP} 
allows all commands except \indexcommand{ADVICE}.

\item[\indexfunction{DISALLOW-RULEP} ]	 allows all commands except
\indexcommand{ADVICE} and \indexcommand{RULEP}.
\end{description}

Those for \indexother{allowed-lemma-p} are:
\begin{description}
\item[\indexfunction{ALLOW-NO-LEMMAS} ]	 allows no theorems to be asserted
in proving the exercise.

\item[\indexfunction{ONLY-REQUIRED-LEMMAS} ]	 allows only
theorems listed under \indexother{REQUIRED-LEMMAS} to be used.

\item[\indexfunction{THEOREM-NO} {\it nnn} ]	 allows only theorem {\it nnn}
to be asserted.

\item[\indexfunction{ALLOW-LOWER-NOS} ]	 allows any theorem with a lower number
to be asserted. (This obviously requires your theorems to be numbered,
as the default theorems from Andrews' book are.)
\end{description}

The definitions of exercises can be stored in files and loaded into {\TPS}
when needed. In order to use the safeguards of the {\TPS} package loading
functions, these are given a package of their own ({\tt ML} in our system),
and we have defined modules for these files and other files
defining the logical system for {\ETPS} in {\tt defpck}. The main module
for a system will have some mnemonic name, such as {\tt math-logic-2},
and the module for the exercises will have the suffix {\tt -exercises}.
Similarly, the module names for wffs (constants, abbreviations, etc.)
and rules bear the suffices {\tt -wffs} and {\tt -rules}, respectively.

If you just want to put an exercise into {\ETPS} temporarily, add the
necessary information to the file etps.patch as in the following
example:

\begin{tpsexample}
(export '(x8030a))

(context ml2-exercises)

(deftheorem x8030a
  (assertion
"
[g(OO) TRUTH AND g FALSEHOOD] = FORALL x(O) g x
  ")
  (thm-type exercise)
  (allowed-cmd-p allow-all)
  (allowed-lemma-p allow-no-lemmas))


\end{tpsexample}
