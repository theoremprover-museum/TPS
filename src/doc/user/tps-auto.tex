\subsection{An Example Using DIY}

\begin{tpsexample}
{\it Comments are in italics.}
<1>save-work workfile
{\it We create a file called workfile.work which will record the commands used.
This is optional.}
<2>prove theorem-name
{\it We could also use the `exercise' command if the theorem to be proved is
an exercise in ETPS.}
<3>diy
\end{tpsexample}

\subsection{An Example Using MATE and ETREE-NAT}


\begin{tpsexample}
{\it Alternatively, we may proceed as follows:}
<1>save-work workfile
<2>prove theorem-name
<3>mate
<4>go
<5>merge-tree
{\it You will be prompted for this when leaving the mate top level.}
<6>etree-nat
{\it To convert the proof to natural deduction style.}
<7>daterec
{\it To store the timing information in the library.}
{\it You may also want to go into the library to store a mode recording the
current flags.}
<8>texproof
{\it To produce printable output.}
\end{tpsexample}


\subsection{A  Longer Example Using MATE and ETREE-NAT}
\label{MATE-then-ETREE-NAT}

To make the formulas easier to read, we have left off the type information.
Note that `\% f x' denotes the image of the set `x' under the function `f'.

\begin{tpsexample}
<8>exercise x5203
(100)        !  \% f [x INTERSECT y] SUBSET \% f x INTERSECT \% f y           PLAN1

{\it We call mating-search directly.}

<9>mate
GWFF (GWFF0): gwff [No Default]>x5203
POSITIVE (YESNO): Positive mode [No]>!

{\it We call the automatic proof search.}
<Mate1>go
...
Displaying VP diagram ...

|                    LEAF7                     |
|                    x T0                      |
|                                              |
|                    LEAF8                     |
|                    y T0                      |
|                                              |
|                    LEAF6                     |
|                  X0 = f T0                   |
|                                              |
|LEAF12      LEAF13       LEAF15      LEAF16   |
|\(\sim\)x t21 OR \(\sim\)X0 = f t21 OR \(\sim\)y t22 OR \(\sim\)X0 = f t22| 
..*.+1.*.+2.*.+3.*.+4..
Trying to unify mating:(4 3 2 1)
Substitution Stack:

t21   ->   T0
t22   ->   T0..
{\it Return to the main top-level.}
<Mate2>leave
Merging the expansion tree.  Please stand by.
****
T
{\it Begin the translation process using the expansion proof just constructed
in the mating-search top-level.}
<9>etree-nat
PREFIX (SYMBOL): Name of the Proof [X5203]>
NUM (LINE): Line Number for Theorem [100]>
TAC (TACTIC-EXP): Tactic to be Used [COMPLETE-TRANSFORM-TAC]>
MODE (TACTIC-MODE): Tactic Mode [AUTO]>

{\it We elide the output from the translation.}

<0>pall

(1)   1      !  EXISTS t18 .x t18 AND y t18 AND x4 = f t18                   Hyp
(2)   1,2    !  x t18 AND y t18 AND x4 = f t18                       Choose: t18
(3)   1,2    !  x t18                                                   RuleP: 2
(4)   1,2    !  y t18                                                   RuleP: 2
(5)   1,2    !  x4 = f t18                                              RuleP: 2
(88)  1,2    !  x t18 AND x4 = f t18                                  RuleP: 3 5
(89)  1,2    !  EXISTS t19 .x t19 AND x4 = f t19                    EGen: t18 88
(93)  1,2    !  y t18 AND x4 = f t18                                  RuleP: 4 5
(94)  1,2    !  EXISTS t20 .y t20 AND x4 = f t20                    EGen: t18 93
(95)  1,2    !      EXISTS t19 [x t19 AND x4 = f t19]
                 AND EXISTS t20 .y t20 AND x4 = f t20               RuleP: 89 94
(96)  1      !      EXISTS t19 [x t19 AND x4 = f t19]
                 AND EXISTS t20 .y t20 AND x4 = f t20                RuleC: 1 95
(97)         !          EXISTS t18 [x t18 AND y t18 AND x4 = f t18]
                 IMPLIES     EXISTS t19 [x t19 AND x4 = f t19]
                         AND EXISTS t20 .y t20 AND x4 = f t20         Deduct: 96
(98)         !  FORALL x4 .        EXISTS t18 [x t18 AND y t18 AND x4 = f t18]
                            IMPLIES     EXISTS t19 [x t19 AND x4 = f t19]
                                    AND EXISTS t20 .y t20 AND x4 = f t20
                                                                     UGen: x4 97
(99)         !  FORALL x4 .        EXISTS t18 [x t18 AND y t18 AND x4 = f t18]
                            IMPLIES     EXISTS t19 [x t19 AND x4 = f t19]
                                    AND EXISTS t20 .y t20 AND x4 = f t20
                                                                    Equality: 98
(100)        !  \% f [x INTERSECT y] SUBSET \% f x INTERSECT \% f y   EquivWffs: 99
\end{tpsexample}
