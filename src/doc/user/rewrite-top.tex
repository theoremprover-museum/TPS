\section{The Rewriting Top Level} \label{RewritingTopLevel}

In many branches of mathematics and computer science relational theories are an
important matter of study. The formal proof technique which often appears most
natural when dealing with equations or other transitive relations (e.g. order
relations) is known as rewriting. As an example of a proof by rewriting,
consider how one would typically prove the uniqueness of inverses in a group
$\langle G,\cdot\rangle$:

Let $a,a',a''\in G$, $a'$ and $a''$ both inverses of $a$. Then
\[\begin{array}{rcl@{\qquad}r}
a' & = & a'\cdot e & (e\textup{ identity of }\langle G,\cdot\rangle)\\
   & = & a'\cdot(a\cdot a'') & (a''\textup{ inverse of }a)\\
   & = & (a'\cdot a)\cdot a'' & (associative law)\\
   & = & e\cdot a'' & (a'\textup{ inverse of }a)\\
   & = & a'' & (e\textup{ identity of }\langle G,\cdot\rangle)
\end{array}\]

The rewriting top level can be used to create and manipulate
complex rewriting derivations in a convenient way.

\subsection{Interacting with the Main Top Level}

There are two ways of entering the rewriting top level from the main
top level. One is by calling \indexcommand{REWRITING}. In case you had been
working on a rewrite derivation before you left the rewriting top level for
the last time, it will return to this derivation. Otherwise no derivation
will be active, so you may want to start a new one or to restore a saved one
from a file. If you want to use the rewriting top level to justify a line
of a natural deduction proof from a preceding line in a way similar to the
application of \indexcommand{USE-RRULES}, you can enter the top level by
calling \indexcommand{REWRITE} or \indexcommand{REWRITE-IN} (See example in
Subsection \ref{RewritingTopLevelExample}). After you
have found a justification, exit the rewriting top level using
\indexcommand{OK} to modify the natural deduction proof accordingly.

Relations derived in the rewriting top level may also be inserted as lemmas
into the main top level using \indexcommand{ASSERT-TOP}.

It is sometimes useful to be able to access lines of the current natural
deduction proof from the rewriting top level and vice versa. This is possible
by using the commands \indexcommand{TOP} and \indexcommand{REW}. Whenever a
wff needs to be supplied to a command within the rewriting top level, you can
type \texttt{(TOP \textit{linenum})} to access the line
\texttt{\textit{linenum}} of the
current natural deduction proof. In exactly the same way \indexcommand{REW}
allows one to access lines of the current rewriting derivation from the main
top level. Of course, both commands can be combined with \indexcommand{ED}.

\subsection{Rewrite Rules, Theories and Derivations}

For technical reasons, the left-hand side and the right-hand side of a rewrite
rule are not allowed to be identical. Since all rewriting commands work modulo
alphabetic change of bound variables, rules with alpha-equivalent sides are
also prohibited. If you use any rules of the prohibited form, some commands
will not work as expected. To assert reflexivity use the command
\indexcommand{SAME} instead.

Once a relation between two wffs has been established, it is possible to
define a derived rewrite rule from the two wffs using the
\indexcommand{DERIVE-RRULE} command. A rule can be derived in more than one
theory at the same time. Unless one decides to change a theory to include the
derived rule, the rule is not part of any theory in which it was derived.
However, when a derived rule is loaded from the library, it is added to the
run-time representation of those theories in memory in which it was derived.
So, after a derived rule has been loaded, it can be used from any theory which
has been loaded before the rule and in which it is derivable in the same way as
if it was part of the core theory.

Commands of the rewriting top level make certain assumptions about the
structure of rewrite theories.
\begin{enumerate}
\item Rewrite theories describe transitive relations.
\item All subtheories of a theory which share their relation sign with the main
  theory are reflexive iff the main theory is reflexive.
\item All subtheories of a theory are ``compatible'' with the main theory, in
  the following sense: If $R$ is the relation described by the main theory
  and $r$ the one described by a subtheory, then
  \begin{enumerate}
  \item $A~R~B$ and $B~r~C$ imply $A~R~C$,
  \item $A~r~B$ and $B~R~C$ imply $A~R~C$.
  \end{enumerate}
\end{enumerate}

The rewriting top level allows having multiple derivations in memory at the
same time, some of which may use different rewrite theories. To avoid
confusion, every rewrite derivation can be associated with a rewrite theory.
The theory associated with the current rewrite derivation is called the
``current theory''. This notion is to be distinguished from the ``active
theory'', i.e. the theory to be associated with new derivations.

At any point in time there may be at most one active theory. To display
the currently active theory use \indexcommand{ACTIVE-THEORY}. The commands
\indexcommand{USE-THEORY} and \indexcommand{DEACTIVATE-THEORY} can be used
to change the active theory.
When a derivation in the rewriting top level is started by calling
\indexcommand{PROVE} or \indexcommand{DERIVE} from the rewriting top level, or
using the \indexcommand{REWRITE} command from the main top level, and an
active theory is defined, it
becomes associated with the new derivation. Applying any rules which are
not part of the current theory, i.e. the theory associated with the current
derivation, will raise an error. On the other hand, it is possible to use
\indexcommand{APP} and similar commands with rules not in the active theory as
long as they are in the current theory.
To display the current theory use \indexcommand{CURRENT-THEORY}.
Changing the active theory, which can be done at any time both from the main
top level and from the rewriting top level, %by calling
%\indexcommand{USE-THEORY} or \indexcommand{DEACTIVATE-THEORY}
%once a derivation is in progress
will never affect the theory associated with an existing derivation.

If one wants to start a derivation in a theory different from the currently
active theory, one can use \indexcommand{PROVE-IN} or \indexcommand{DERIVE-IN}
from the rewriting top level or \indexcommand{REWRITE-IN} from the main top
level. These commands expect the theory to associate with the newly created
derivation as an additional parameter. If there is a currently active theory,
i.e. if \indexcommand{ACTIVE-THEORY} does not return NIL, it will not be
affected by these commands. If there is no currently active theory, the theory
passed to \indexcommand{PROVE-IN} or \indexcommand{REWRITE-IN} will become
active.

%If a theory makes use of library constants, abbreviations or other objects
%which are not loaded by default when \TPS starts up, \indexcommand{PROVE-IN}
%and \indexcommand{DERIVE-IN} may not work as expected.

\subsection{Automatic Search} \label{RewritingTopLevelAuto}
The command \indexcommand{AUTO} can be used to search for a rewrite sequence
between two specified lines automatically. Dependent on the setting of the
flag \indexflag{REWRITING-AUTO-SEARCH-TYPE}, one of the following search
algorithms is used:
\begin{description}
\item [\texttt{SIMPLE}:] Iterative deepening starting from the source wff (i.e.
  the wff in the lower-numbered line). The procedure uses a hash table, called
  `search table', for cycle and dead-end detection.
\item [\texttt{BIDIR}:] Bidirectional search using iterative deepening,
  starting from the source and the target wff. Two search tables of equal
  maximum size are used.
\item [\texttt{BIDIR-SORTED}:] As \texttt{BIDIR}, but rewriting shorter wffs
  first. This procedure is used by default because it is most likely to find a
  result within reasonable time.
\end{description}
Automatic search uses active rewrite rules from the current theory, or all
active rewrite rules if the current derivation is not associated with any
theory.

See the description of flags starting with \texttt{REWRITING-AUTO} in
Subsection \ref{RewritingTopLevelFlags} to learn how to tune different
parameters of the search.

%\subsection{Rewriting Top Level Commands and Flags}

%The rewriting top level commands are as follows:
%\begin{description}
%\item[]
\subsection{Commands for Entering and Leaving the Rewriting Top Level}
  \begin{description}
  \item[] \indexcommand{REWRITING} Enter the rewriting top level without
    starting a new rewriting derivation.
  \item[] \indexcommand{REWRITE} Enter the rewriting top level to search for a
    rewrite sequence justifying a step in the current natural deduction proof.
    The new derivation will use the currently active theory.
  \item[] \indexcommand{REWRITE-IN} Same as \indexcommand{REWRITE}, but
    prompting for the rewrite theory to use with the new derivation.
  \item[] \indexcommand{ASSERT-TOP} Leave the rewriting top level, inserting
    the derived relation as a lemma into the current natural deduction proof.
    See also \indexcommand{ASSERT} and \indexcommand{ASSERT2}.
  \item[] \indexcommand{BEGIN-PRFW} Enter the \texttt{REW-PRFW} top level to
    open proofwindows.
    Alternatively, one can enter the \texttt{PRFW} top level from the main
    top-level and switch to the rewriting top level afterwards.
  \item[] \indexcommand{END-PRFW} Leave the \texttt{REW-PRFW} top level.
  \item[] \indexcommand{LEAVE} Leave the rewriting top level.
  \item[] \indexcommand{OK} Leave the rewriting top level, passing the proven
    relation as a justification of a rewrite step in the current natural
    deduction proof. This command is only applicable if the current derivation
    was started from the main top level using \indexcommand{REWRITE} or
    \indexcommand{REWRITE-IN}.
  \end{description}

%\item[]
\subsection{Commands for Starting, Finishing and Printing Rewrite Derivations}
  \begin{description}
  \item[] \indexcommand{DERIVE} Begin a new rewrite derivation without a fixed
    target wff, associating with it the currently active theory.
  \item[] \indexcommand{DERIVE-IN} Same as \indexcommand{DERIVE}, but prompting
    for the rewrite theory to use with the new derivation.
  \item[] \indexcommand{DONE} Check whether the current derivation is complete.
  \item[] \indexcommand{PALL} Works the same as in the main top level.
  \item[] \indexcommand{PROOFLIST} Print a list of all rewrite derivations
    currently in memory. For proofs, the corresponding proof assertions are
    printed. For general derivations, the corresponding initial lines are
    printed.
  \item[] \indexcommand{PROVE} Start a new rewriting proof of a given
    relational assertion, using the currently active rewrite theory.
  \item[] \indexcommand{PROVE-IN} Same as \indexcommand{PROVE}, but prompting
    for the rewrite theory to use.
  \item[] \indexcommand{RECONSIDER} Switch the current rewrite derivation.
    Works the same as in the main top level.
  \item[] \indexcommand{RESTOREPROOF} Load a rewriting derivation from a file
    and make it the current derivation. If the derivation was obtained in a
    theory which is not yet in the memory, the command will offer to load the
    theory from the library.
  \item[] \indexcommand{SAVEPROOF} Save the current derivation to a file.
    Works the same as in the main top level.
  \item[] \indexcommand{TEXPROOF} Works the same as in the main top level.
  \end{description}
%\item[] Printing commands:
%\indexcommand{PALL}, \indexcommand{TEXPROOF}. Work
%  the same as in the main top level.

%\item[]
\subsection{Commands for Applying Rewrite Rules}
  \begin{description}
  \item[] \indexcommand{ANY} Try to apply any active rewrite rule from the
    current theory and all its subtheories. If there is no current theory, all
    active rules will be tried.
  \item[] \indexcommand{ANY*}/\indexcommand{UNANY*} Justify a line by a
    sequence of applications of any active rewrite rules from the current
    theory in the forward/backward direction, starting from a preceding line.
    In most cases, this command will apply rewrite rules in the corresponding
    direction as often as possible or until a specified target wff is obtained.
    If the wff after rewriting is specified but the one before rewriting is set
    to NIL, rewrite rules will be applied in the corresponding reverse
    direction, starting from the target formula. The command will add no
    intermediate lines to the derivation outline.\\
    The commands may not terminate if \indexflag{APP*-REWRITE-DEPTH} is set to
    NIL.
  \item[] \indexcommand{ANY*-IN}/\indexcommand{UNANY*-IN} Same as
    \indexcommand{ANY*}/\indexcommand{UNANY*}, but will only try rules from
    the specified subtheory of the current theory.
  \item[] \indexcommand{APP} Apply the specified rewrite rule from the current
    theory or any of its subtheories. The rule need not be active.\\
    Note: Entering the name of a rewrite rule \texttt{\textit{rrule}} at the
    top-level command prompt is considered to be an abbreviation for
    \texttt{APP \textit{rrule}}.
  \item[] \indexcommand{APP*}/\indexcommand{UNAPP*} Same as
    \indexcommand{ANY*}/\indexcommand{UNANY*}, but using the specified single
    rule, which need not be active.
  \item[] \indexcommand{AUTO} Search for a rewrite sequence between two lines
    using any active rewrite rules from the current theory. The exact behaviour
    is affected by following flags: \indexflag{REWRITING-AUTO-DEPTH},
    \indexflag{REWRITING-AUTO-MIN-DEPTH},
    \indexflag{REWRITING-AUTO-TABLE-SIZE},
    \indexflag{REWRITING-AUTO-MAX-WFF-SIZE}, \indexflag{REWRITING-AUTO-SUBSTS}.
    See Subsection \ref{RewritingTopLevelAuto}.
  \item[] \indexcommand{SAME} Use reflexivity of equality. Works almost the
    same as in the main top level, but allows alphabetic change of bound
    variables.
  \end{description}

%\item[]
\subsection{Commands for Rearranging the Derivation}
  \begin{description}
  \item[] \indexcommand{CLEANUP}, \indexcommand{DELETE},
    \indexcommand{INTRODUCE-GAP}, \indexcommand{MOVE}, \indexcommand{SQUEEZE}.
    Work almost the same as in the main top level. \indexcommand{DELETE}
    cannot be used to delete the initial or the target line of a derivation.
    Neither \indexcommand{INTRODUCE-GAP} nor \indexcommand{MOVE} will move the
    initial line of a derivation.
  \item[] \indexcommand{CONNECT} Given two identical lines, delete the lower
    one, rearranging the derivation appropriately. With symmetric relations,
    the command will also rearrange the lines from which the higher-numbered
    line was obtained to follow from the lower-numbered line.

    The main motivation behind \indexcommand{CONNECT} is a situation which is
    quite common when doing equational reasoning. Starting from the source and
    from the target wff, you obtain two derivations which have a line in
    common. By symmetry and transitivity of equality, you can combine the
    two subderivations into a single formal proof. \indexcommand{CONNECT}
    provides an efficient way of doing this transformation.

    More precisely, the behaviour of \indexcommand{CONNECT} can be described
    as follows:\\
    Assume \indexcommand{CONNECT} is applied to the lower-numbered line $p_1$
    and the higher-numbered $p_2$.
    \begin{itemize}
    \item If $p_2$ is the fixed target line of a derivation, or if the first
      line of the rewrite sequence justifying $p_2$ occurs before $p_1$,
      nothing will be done.
    \item If $p_2$ is justified by a rewrite sequence involving $p_1$, or if
      the derivation of $p_2$ involves directed rewrite rules, $p_2$ will be
      deleted. Lines which are justified by $p_2$ will be changed to refer to
      $p_1$ instead.
    \item If $p_1$ is not part of the derivation of $p_2$ and the derivation of
      $p_2$ was obtained using bidirectional rules only, the same as in the
      previous case will happen, but additionally the derivation of $p_2$ will
      be reversed, i.e. the sequence of lines $p_1\ldots d_1\ldots d_n\ldots
      p_2$ will be changed to $p_1\ldots d_n\ldots d_1$. Since in the former
      sequence $d_1$ has to be a planned line, this transformation will result
      in a derivation outline which has one planned line less than the initial
      outline.
    \end{itemize}
  \end{description}

%\item[]
\subsection{Lambda Conversion Commands}
  \begin{description}
  \item[] \indexcommand{BETA-EQ}, \indexcommand{ETA-EQ},
    \indexcommand{LAMBDA-EQ}. Assert the corresponding equivalence between two
    lines.
  \item[] \indexcommand{BETA-NF}, \indexcommand{ETA-NF},
    \indexcommand{LAMBDA-NF}, \indexcommand{LONG-ETA-NF}. Compute the
    corresponding normal form of a line.
  \end{description}

%\item[]
\subsection{Commands Concerned with Rewrite Rules and Theories}
  \begin{description}
  \item[] \indexcommand{CURRENT-THEORY} Show the theory associated with current
    rewrite derivation.
  \item[] \indexcommand{DERIVE-RRULE} Create a derived rewrite rule from two
    provably related lines. If the relation was proven using bidirectional
    rules only, the derived rule may be made bidirectional.
  \item[] \indexcommand{MAKE-RRULE} Create a new rewrite rule in memory.
  \item[] \indexcommand{SAVE-RRULE} Save an existing rewrite rule into the
    library. This is the only way of saving a derived rule such that the rule
    preserves its derived status in the library.
  \end{description}
%\end{description}

\subsection{Applicable Commands from the Main Top Level}
%In addition to the above commands, the following commands from the main
%top level are available:\\
\indexcommand{ACTIVATE-RULES}, \indexcommand{ACTIVE-THEORY},
\indexcommand{DEACTIVATE-RULES}, \indexcommand{DEACTIVATE-THEORY},\\
\indexcommand{DELETE-RRULE}, \indexcommand{LIST-RRULES},
\indexcommand{MAKE-ABBREV-RRULE}, \indexcommand{MAKE-INVERSE-RRULE},
\indexcommand{MAKE-THEORY}, \indexcommand{PERMUTE-RRULES},
\indexcommand{USE-THEORY}, \indexcommand{REVIEW}, \indexcommand{LIB},
\indexcommand{EXIT}.\smallskip

\subsection{Flags} \label{RewritingTopLevelFlags}
%The following flags affect the behaviour of rewriting top level commands:
\begin{description}
\item[] \indexflag{APP*-REWRITE-DEPTH} Determines the maximal rewrite depth of
  an \indexcommand{APP*} application. Used in the same way by
  \indexcommand{UNAPP*}, \indexcommand{ANY*}, \indexcommand{UNANY*},
  \indexcommand{ANY*-IN} and \indexcommand{UNANY*-IN}.
\item[] \indexflag{REWRITING-AUTO-DEPTH} The maximal depth of a search tree
  when applying \indexcommand{AUTO}. For the \texttt{SIMPLE} search procedure,
  the number corresponds to the maximal rewrite depth, whereas for
  \texttt{BIDIR} and \linebreak \texttt{BIDIR-SORTED} the maximal search depth is
  twice the specified number.
\item[] \indexflag{REWRITING-AUTO-GLOBAL-SORT} When NIL, \texttt{BIDIR-SORTED}
  will choose the next wff to be rewritten from the successors of the current
  wff. When T, it will choose the next wff from all unexplored wffs obtained
  so far from the initial or the target wff, respectively. See the flag
  \indexflag{REWRITING-AUTO-SEARCH-TYPE}.
\item[] \indexflag{REWRITING-AUTO-MAX-WFF-SIZE} The maximal size of a wff to be
  rewritten when applying \indexcommand{AUTO}.
\item[] \indexflag{REWRITING-AUTO-MIN-DEPTH} The minimal depth of a search tree
  to be used by AUTO to find a derivation. The value should be less or equal to
  that of REWRITING-AUTO-DEPTH, otherwise no search will be performed.
\item[] \indexflag{REWRITING-AUTO-SEARCH-TYPE} Determines the search algorithm
  used by \indexflag{AUTO}. Currently defined are \texttt{SIMPLE},
  \texttt{BIDIR} and \texttt{BIDIR-SORTED}. \texttt{BIDIR-SORTED} will try to
  rewrite shorter wffs first. When this is not needed, use \texttt{BIDIR}. The
  precise behaviour of \texttt{BIDIR-SORTED} depends on the flag
  \indexflag{REWRITING-AUTO-GLOBAL-SORT}.
\item[] \indexflag{REWRITING-AUTO-SUBSTS} The list of terms to substitute for
  any free variables which may be introduced during rewriting by
  \indexcommand{AUTO}. If NIL, the list will be generated automatically from
  atomic subwffs of the source and the target wff.
\item[] \indexflag{REWRITING-AUTO-TABLE-SIZE} Specifies the maximal size of a
  search table used by \indexcommand{AUTO}. Note that while the \texttt{SIMPLE}
  search procedure uses only one table of the specified size, \texttt{BIDIR}
  and \texttt{BIDIR-SORTED} use two.
\item[] \indexflag{REWRITING-RELATION-SYMBOL} Contains the symbol that is
  printed between lines obtained by rewriting from immediately preceding lines.
  Normally this symbol is the relation sign of the current theory.
  If set explicitly, the symbol will only be used if there is no current
  theory or if the theory has no relation sign associated with it. Also, when
  switching between different rewrite derivations, the value of this flag will
  be changed. The value of the flag is also used by \indexcommand{PROVE} to
  determine the left and the right part of a relational assertion.
\item[] \indexflag{VERBOSE-REWRITE-JUSTIFICATION} When set to T, justification
  of ND proof lines by rewriting in the rewriting top level will indicate the
  rewrite theory used to obtain the justification.
\end{description}

%The rules \indexcommand{USE-THEORY}, \indexcommand{ACTIVE-THEORY} and
%\indexcommand{DEACTIVATE-THEORY} have special

\subsection{Example} \label{RewritingTopLevelExample}

%"FA [LAMBDA x(O).x] = FALSEHOOD"
\begin{tpsexample}
<0>prove "f FALSEHOOD AND f TRUTH IMPLIES f FALSEHOOD AND f TRUTH AND f x"
PREFIX (SYMBOL): Name of the Proof [No Default]>example
NUM (LINE): Line Number for Theorem [100]>
(100) \(\assert\)  f\(\sb{\greeko\greeko}\) \(\bot\) \(\land\) f \(\top\) \(\implies\) f \(\bot\) \(\land\) f \(\top\) \(\land\) f x\(\sb{\greeko}\)                                   PLAN1

<1>deduct !
(1)   1 \(\assert\)  f\(\sb{\greeko\greeko}\) \(\bot\) \(\land\) f \(\top\)                                                      Hyp
(99)  1 \(\assert\)  f\(\sb{\greeko\greeko}\) \(\bot\) \(\land\) f \(\top\) \(\land\) f x\(\sb{\greeko}\)                                             PLAN2

{\it Lines 1 and 99 can be proven equal in the rewrite theory HOL.}

<2>rewrite-in hol
P2 (LINE): Line after rewriting (higher-numbered) [99]>
P1 (LINE): Line before rewriting (lower-numbered) [98]>1
<REWRITING3>pall

(1)      f\(\sb{\greeko\greeko}\) \(\bot\) \(\land\) f \(\top\)
               ...
(100)    f\(\sb{\greeko\greeko}\) \(\bot\) \(\land\) f \(\top\) \(\land\) f x\(\sb{\greeko}\)                                               PLAN1

{\it Let us apply the rule Bin from HOL. It is sufficient to type in the name of the rule.}

<REWRITING4>bin
P1 (LINE): Line before rewriting  (lower-numbered) [1]>
P2 (LINE): Line after rewriting  (higher-numbered) [2]>
B (GWFF-OR-SELECTION): Wff after rewriting
 1)  \(\forall\) f\(\sb{\greeko\greeko}\)

 [1]>
(2)   =  \(\forall\) f\(\sb{\greeko\greeko}\)                                                           Bin: 1

{\it What to do next? We can use ANY to look at some alternatives.}

<REWRITING5>any
P1 (LINE): Line before rewriting  (lower-numbered) [2]>
P2 (LINE): Line after rewriting  (higher-numbered) [3]>
B (GWFF-OR-SELECTION): Wff after rewriting
 1)  [\(\lambda\)\(\sb{\greeko\greeko}\) \(\forall\) u] f\(\sb{\greeko\greeko}\)  (ETA)
 2)  \(\forall\).\(\lambda\)u\(\sb{\greeko}\) f\(\sb{\greeko\greeko}\) u  (ETA)
 3)  \(\forall\) f\(\sb{\greeko\greeko}\) \(\land\) \(\top\)  (AND-ID)
 4)  \(\forall\) f\(\sb{\greeko\greeko}\) \(\lor\) \(\bot\)  (OR-ID)
 5)  f\(\sb{\greeko\greeko}\) = \(\lambda\)u\(\sb{\greeko}\) \(\top\)  (FA-D)
 6)  f\(\sb{\greeko\greeko}\) \(\bot\) \(\land\) f \(\top\)  (BIN)

 [1]>5
(3)   =  f\(\sb{\greeko\greeko}\) = \(\lambda\)u\(\sb{\greeko}\) \(\top\)                                                    Fa-D: 2

<REWRITING6>pall

(1)      f\(\sb{\greeko\greeko}\) \(\bot\) \(\land\) f \(\top\)
(2)   =  \(\forall\) f\(\sb{\greeko\greeko}\)                                                           Bin: 1
(3)   =  f\(\sb{\greeko\greeko}\) = \(\lambda\)u\(\sb{\greeko}\) \(\top\)                                                    Fa-D: 2
               ...
(100)    f\(\sb{\greeko\greeko}\) \(\bot\) \(\land\) f \(\top\) \(\land\) f x\(\sb{\greeko}\)                                               PLAN1

<REWRITING7>and-id
P1 (LINE): Line before rewriting  (lower-numbered) [3]>
P2 (LINE): Line after rewriting  (higher-numbered) [4]>
B (GWFF-OR-SELECTION): Wff after rewriting
 1)  f\(\sb{\greeko\greeko}\) = \(\lambda\)u\(\sb{\greeko}\) \(\top\) \(\land\) \(\top\)
 2)  f\(\sb{\greeko\greeko}\) = \(\lambda\)u\(\sb{\greeko}\).\(\top\) \(\land\) \(\top\)

 [1]>
(4)   =  f\(\sb{\greeko\greeko}\) = \(\lambda\)u\(\sb{\greeko}\) \(\top\) \(\land\) \(\top\)                                              And-Id: 3

{\it Now we want to \(\beta\)-expand the last occurence of \(\top\) into \([\lambda f\sb{\greeko\greeko}.f x].\lambda x\sb{\greeko}.\top\). To save some time, we just specify the desired wff, using AUTO to fill in the gap in the two-step expansion.}

<REWRITING8>auto
P1 (LINE): Line before rewriting  (lower-numbered) [4]>
P2 (LINE): Line after rewriting  (higher-numbered) [100]>6
B (GWFF): Wff after rewriting [No Default]>ed 4
<Ed9>d

\(\top\)
<Ed10>sub "[LAMBDA f.f x]. LAMBDA x(O).TRUTH"

[\(\lambda\)f\(\sb{\greeko\greeko}\) f x\(\sb{\greeko}\)] .\(\lambda\)x \(\top\)
<Ed11>^

f\(\sb{\greeko\greeko}\) = \(\lambda\)u\(\sb{\greeko}\) \(\top\) \(\land\) [\(\lambda\)f f x\(\sb{\greeko}\)] .\(\lambda\)x \(\top\)
<Ed12>ok
Search in progress. Please wait...
Success.
Real time: 0.864435 sec.
Run time: 0.818894 sec.
Space: 6488652 Bytes
GC: 3, GC time: 0.200997 sec.

<REWRITING13>pall

(1)      f\(\sb{\greeko\greeko}\) \(\bot\) \(\land\) f \(\top\)
(2)   =  \(\forall\) f\(\sb{\greeko\greeko}\)                                                           Bin: 1
(3)   =  f\(\sb{\greeko\greeko}\) = \(\lambda\)u\(\sb{\greeko}\) \(\top\)                                                    Fa-D: 2
(4)   =  f\(\sb{\greeko\greeko}\) = \(\lambda\)u\(\sb{\greeko}\) \(\top\) \(\land\) \(\top\)                                              And-Id: 3
(5)   =  f\(\sb{\greeko\greeko}\) = \(\lambda\)u\(\sb{\greeko}\) \(\top\) \(\land\) [\(\lambda\)x\(\sb{\greeko}\) \(\top\)] x                                        Beta: 4
(6)   =  f\(\sb{\greeko\greeko}\) = \(\lambda\)u\(\sb{\greeko}\) \(\top\) \(\land\) [\(\lambda\)f f x\(\sb{\greeko}\)].\(\lambda\)x \(\top\)                                   Beta: 5
               ...
(100)    f\(\sb{\greeko\greeko}\) \(\bot\) \(\land\) f \(\top\) \(\land\) f x\(\sb{\greeko}\)                                               PLAN1

{\it Specifying an appropriate \(\beta\)-expansion was the trickiest part of the derivation. The rest can be done automatically.}

<REWRITING14>auto
P1 (LINE): Line before rewriting  (lower-numbered) [6]>
P2 (LINE): Line after rewriting  (higher-numbered) [100]>
Search in progress. Please wait...
Success.
Real time: 1.682173 sec.
Run time: 1.606185 sec.
Space: 12504376 Bytes
GC: 6, GC time: 0.377588 sec.

<REWRITING15>pall

(1)      f\(\sb{\greeko\greeko}\) \(\bot\) \(\land\) f \(\top\)
(2)   =  \(\forall\) f\(\sb{\greeko\greeko}\)                                                           Bin: 1
(3)   =  f\(\sb{\greeko\greeko}\) = \(\lambda\)u\(\sb{\greeko}\) \(\top\)                                                    Fa-D: 2
(4)   =  f\(\sb{\greeko\greeko}\) = \(\lambda\)u\(\sb{\greeko}\) \(\top\) \(\land\) \(\top\)                                              And-Id: 3
(5)   =  f\(\sb{\greeko\greeko}\) = \(\lambda\)u\(\sb{\greeko}\) \(\top\) \(\land\) [\(\lambda\)x\(\sb{\greeko}\) \(\top\)] x                                        Beta: 4
(6)   =  f\(\sb{\greeko\greeko}\) = \(\lambda\)u\(\sb{\greeko}\) \(\top\) \(\land\) [\(\lambda\)f f x\(\sb{\greeko}\)].\(\lambda\)x \(\top\)                                   Beta: 5
(7)   =  f\(\sb{\greeko\greeko}\) = \(\lambda\)u\(\sb{\greeko}\) \(\top\) \(\land\) [\(\lambda\)f f x\(\sb{\greeko}\)] f                                       Rep: 6
(8)   =  f\(\sb{\greeko\greeko}\) = \(\lambda\)u\(\sb{\greeko}\) \(\top\) \(\land\) f x\(\sb{\greeko}\)                                             Beta: 7
(9)   =  \(\forall\) f\(\sb{\greeko\greeko}\) \(\land\) f x\(\sb{\greeko}\)                                                   Fa-D: 8
(100) =  f\(\sb{\greeko\greeko}\) \(\bot\) \(\land\) f \(\top\) \(\land\) f x\(\sb{\greeko}\)                                              Bin: 9

{\it DONE can be used to check whether a derivation is complete, or to display the reason why the assertion in question has not yet been proved.}

<REWRITING16>done
Derivation complete.

{\it We are done. To exit the rewriting top level, updating the main proof, we use OK.}

<REWRITING17>ok


<18>pall

(1)   1 \(\assert\)  f\(\sb{\greeko\greeko}\) \(\bot\) \(\land\) f \(\top\)                                                      Hyp
(99)  1 \(\assert\)  f\(\sb{\greeko\greeko}\) \(\bot\) \(\land\) f \(\top\) \(\land\) f x\(\sb{\greeko}\)                                   Rewrite(HOL): 1
(100) \(\assert\)  f\(\sb{\greeko\greeko}\) \(\bot\) \(\land\) f \(\top\) \(\implies\) f \(\bot\) \(\land\) f \(\top\) \(\land\) f x\(\sb{\greeko}\)                              Deduct: 99
\end{tpsexample}

\subsection{Semantics of Rewrite Rules}

In order to create own rewrite rules, and to use existing ones efficiently, it
is useful to know how rewrite rules are interpreted by top-level procedures.
Therefore, let us give a short formal account of what it means for a pair of
wffs to be an instance of a rewrite rule.

For the simplicity of presentation, in the following we
consider monomorphic rewrite rules only. Rewrite rules with polymorphic types
can be thought of as being appropriately instantiated before the matching takes
place.

Let the structure of wffs and wff schemata be defined by the following grammar:

\[\begin{array}{rcl}
c & ::= & \top\ |\ \bot\ |\ \land\ |\ \lor\ |\ \ldots\\
Q & ::= & \lambda\ |\ \forall\ |\ \exists \\
A,B,C,D & ::= & x,y\ |\ c\ |\ A\,B\ |\ Qx.A
\end{array}\]

We define the notion of a context to stand for a partial function. The notation
$\Gamma,a{:=}b$ stands for the function
$\lambda x.\textup{if }x=a\textup{ then }b\textup{ else }\Gamma\,x$.

We define the matching relation $\Sigma;\Gamma\vdash B\rhd A;\Sigma'$ between
the initial global context $\Sigma$, the lexical context $\Gamma$, the wff $B$,
the wff schema $A$ and the final global context $\Sigma'$ by induction on the
structure of $A$:
\[
\frac{\Gamma\,x_{\greeka}=y_{\greeka}}
     {\Sigma;\Gamma\vdash y_{\greeka}\rhd x_{\greeka};\Sigma}
\qquad
\frac{x_{\greeka}\notin\textup{dom}\,\Gamma\qquad\Sigma\,x_{\greeka}=A_{\greeka}}
     {\Sigma;\Gamma\vdash A_{\greeka}\rhd x_{\greeka};\Sigma}
\qquad
\frac{x_{\greeka}\notin\textup{dom}\,\Gamma\qquad x_{\greeka}\notin\textup{dom}\,\Sigma}
     {\Sigma;\Gamma\vdash A_{\greeka}\rhd x_{\greeka};\Sigma,x_{\greeka}{:=}A_{\greeka}}
\qquad
\frac{~}{\Sigma;\Gamma\vdash c_{\greeka}\rhd c_{\greeka};\Sigma}
\]\[
\frac{\Sigma;\Gamma\vdash C\rhd A;\Sigma''\qquad
      \Sigma'';\Gamma\vdash D\rhd B;\Sigma'}
     {\Sigma;\Gamma\vdash C\,D\rhd A\,B;\Sigma'}
\qquad
\frac{\Sigma;\Gamma,x_{\greeka}{:=}y_{\greeka}\vdash B\rhd A;\Sigma'}
     {\Sigma;\Gamma\vdash Qy_{\greeka}.B\rhd Qx_{\greeka}.A;\Sigma'}
\]

Two wffs $C,D$ form an instance of a rewrite rule $A\longrightarrow B$
iff there exist global contexts $\Sigma,\Sigma'$ such that
$\varnothing;\varnothing\vdash C\rhd A;\Sigma'$ and
$\Sigma';\varnothing\vdash D\rhd B;\Sigma$.

The applicability test procedure \texttt{S-EQN-AXIOM-APPFN}, always to be used
with the rewriting procedure \texttt{S-EQN-AXIOM-REWFN}, enforces that both
sides of a rewrite rule are interpreted as wffs, not as wff schemata. So,
the only instances accepted by a rewrite rule using
\texttt{S-EQN-AXIOM-APPFN}/\texttt{S-EQN-AXIOM-REWFN} are, modulo alphabetic
change of bound variables, substitution instances of the two wffs forming the
rule. Formally this can be reflected by modifying the second and the third
matching rule as follows:
\[
\frac{x_{\greeka}\notin\textup{dom}\,\Gamma\qquad\Sigma\,x_{\greeka}=A_{\greeka}\qquad
      \textup{FV}\,A_{\greeka}\cup\textup{ran}\,\Gamma=\varnothing}
     {\Sigma;\Gamma\vdash A_{\greeka}\rhd x_{\greeka};\Sigma}
\qquad
\frac{x_{\greeka}\notin\textup{dom}\,\Gamma\qquad x_{\greeka}\notin\textup{dom}\,\Sigma\qquad
      \textup{FV}\,A_{\greeka}\cup\textup{ran}\,\Gamma=\varnothing}
     {\Sigma;\Gamma\vdash A_{\greeka}\rhd x_{\greeka};\Sigma,x_{\greeka}{:=}A_{\greeka}}
\]
$\textup{FV}\,A$ denotes the set of variables which occur free in $A$.

The applicability test \texttt{INFERENCE-SCHEME-APPFN}, to be used with
\texttt{INFERENCE-MATCH-BINDERS-REWFN}, enforces that whenever two binders in
the definition of a rewrite rule bind the same variable, this has to be
reflected in the rule instance. As a counterexample, consider a rule of the
form $\forall x_{\greeka}\forall x_{\greeka}.A\longleftrightarrow
\forall x_{\greeka}.A$. One can easily check that, if no
applicability tests are performed,
$\forall y_{\greeka}\forall x_{\greeka}.f\,x\,y\longleftrightarrow
\forall x_{\greeka}.f\,x\,y$ is an instance of the above rule.

To describe the semantics of
\texttt{INFERENCE-SCHEME-APPFN}/\texttt{INFERENCE-MATCH-BINDERS-REWFN}, we
define a relation $\Delta\vdash B\star A;\Delta'$ between the initial binder
matching context $\Delta$, the wff $B$, the wff schema $A$ and the final
binder matching context $\Delta'$ by induction on the structure of $A$:
\[
\frac{~}{\Delta\vdash A\star x_{\greeka};\Delta}
\qquad
\frac{~}{\Delta\vdash A\star c_{\greeka};\Delta}
\qquad
\frac{\Delta\vdash C\star A;\Delta''\qquad\Delta''\vdash D\star B;\Delta'}
     {\Delta\vdash C\,D\star A\,B;\Delta'}
\]\[
\frac{\Delta\,x_{\greeka}=y_{\greekb}}
     {\Delta\vdash Q'y_{\greekb}.B\star Qx_{\greeka}.A;\Delta}
\qquad
\frac{x_{\greeka}\notin\textup{dom}\,\Delta}
     {\Delta\vdash Q'y_{\greekb}.B\star Qx_{\greeka}.A;
      \Delta,x_{\greeka}{:=}y_{\greekb}}
\]

Two wffs $C,D$ form an instance of a rewrite rule $A\longrightarrow B$, using
the applicability test procedure \linebreak \texttt{INFERENCE-SCHEME-APPFN}
and the rewriting procedure \texttt{INFERENCE-MATCH-BINDERS-REWFN}, iff they
are an instance of $A\longrightarrow B$ and there exist binder matching
contexts $\Delta,\Delta'$ such that $\varnothing\vdash C\star A;\Delta'$ and
$\Delta'\vdash D\star B;\Delta$.

\section{How Rewrite Rules and Theories Are Stored in the Library}
Rewrite rules are stored as library objects of type
rrule, whose description is a dotted pair of gwffs with extra attributes
`typelist' (the list of all polymorphic type symbols in the rule),
`bidirectional' (T if the rule can be applied in both directions),
`appfn' (the name of a function to test applicability of the rule, or NIL),
`function' (an optional extra function which is applied, after the
rewriting, to the subformula that was rewritten),
`variables' (the list of universally quantified free variables; from the
technical point of view, only variables which do not appear in both sides
of the rewrite rule need to appear in this list) and
`derived-in' (the list of theories in which the rule is derivable).

%`appfn', if not NIL, should be a function from gwffs to booleans;
%`function', if not NIL, should be a function from gwffs to gwffs.

% mkaminski -- The above is no longer true.

`appfn', if not NIL, should be a function which expects four arguments.
The first argument will always be the gwff which is to be rewritten. The
following two arguments will contain both sides of the rewrite rule, passed
in the order in which the rule is to be applied. So, for instance, if the rule
is to be applied from right to left, the right-hand side of the rule will be
passed as the second argument, followed by the left-hand side. The fourth
argument is the list of polymorphic types of the rewrite rule.
`appfn' should return a boolean, indicating whether the gwff can be rewritten
by applying the rewrite rule in the specified direction.
`function', if not NIL, should expect the subformula after rewriting as its
first argument, and, just like `appfn', both sides of the rewrite rule as
additional arguments. Unlike `appfn' it is not passed the list of polymorphic
types. `function' should return the modified subformula.

Theories are
stored as the name of the theory with all rewrite rules, subtheories and other
needed objects attached as needed-objects. In addition to this, theories may
have the extra attributes `relation-sign' (a symbol representing the relation
which is assumed to hold between the two sides of a rewrite rule),
`reflexive' (T if the relation represented by the attached rewrite rules is
reflexive, NIL otherwise), `congruent' (T if the relation represented by
the rewrite rules is a congruence on lambda-terms, NIL if the rewrite rules
should only be applicable to a wff as a whole but not to its subwffs),
`derived-appfn' (the value which will be assigned to the `appfn' attribute of
rewrite rules derived from this theory in the rewriting top-level) and
`derived-rewfn' (the same as `derived-appfn' but for the `rewfn' attribute).

