%Note: you will need to \usepackage{alltt} and \usepackage{makeidx} so that alltt can be used below

\makeatletter %one column index
\renewcommand{\theindex}{\if@twocolumn\@restonecolfalse\else \@restonecoltrue\fi%
\onecolumn\section*{\indexname}\@mkboth{\MakeUppercase\indexname} {\MakeUppercase\indexname}%
\thispagestyle{plain}\parindent\z@\parskip\z@\@plus.3\p@\relax\let \item\@idxitem}
\renewcommand{\endtheindex}{\if@restonecol\clearpage\else\twocolumn \fi}
\makeatother

\makeatletter %set the space between numbers and sections in the table of contents
\renewcommand*\l@section{\@dottedtocline{1}{0em}{2.8em}}
\makeatother

\makeatletter %new title page
\renewcommand{\maketitle}{
    \vspace*{5cm}
    \begin{center}
    {\textbf{\LARGE \@title} \par}
    \vskip 2em
    {\large
     \lineskip .75em
      \begin{tabular}[t]{c}
        \@author
      \end{tabular}\par}
      \vskip 1.5em
    {\large \@date \par}
    \end{center}\par
    \vskip 15.5em
    \copyrightandresearchcredit
    \thispagestyle{empty}
    \setcounter{page}{0}
}
\makeatother

\def\research-credit{
This material is based upon work supported by NSF grants 
MCS81-02870, DCR-8402532, CCR-8702699, CCR-9002546, 
CCR-9201893, CCR-9502878, CCR-9624683, CCR-9732312, 
CCR-0097179, and a grant from the Center for Design of 
Educational Computing, Carnegie Mellon University. Any 
opinions, findings, and conclusions or recommendations 
are those of the authors and do not necessarily reflect 
the views of the National Science Foundation.
}

\def\CMU-copyright{
Copyright \copyright Carnegie Mellon University 2005 
}

\def\copyrightandresearchcredit{
Copyright \copyright Carnegie Mellon University 2005.
This material is based upon work supported by NSF grants 
MCS81-02870, DCR-8402532, CCR-8702699, CCR-9002546, 
CCR-9201893, CCR-9502878, CCR-9624683, CCR-9732312, 
CCR-0097179, and a grant from the Center for Design of 
Educational Computing, Carnegie Mellon University. Any 
opinions, findings, and conclusions or recommendations 
are those of the authors and do not necessarily reflect 
the views of the National Science Foundation.
}

%The commands below are copied from 
%/home/theorem/project/doc/prog/manual.tex
\newtheorem{definition}{Definition}[section] % a definition is labelled Definition and is numbered by section
\newtheorem{lemma}[definition]{Lemma} % lemmas are labelled Lemma and use the same numbering as definitions.
\newtheorem{example}[definition]{Example} % etc...
\newtheorem{theorem}[definition]{Theorem}
\newtheorem{algorithm}[definition]{Algorithm}
\newtheorem{proposition}[definition]{Proposition}
\newtheorem{conjecture}[definition]{Conjecture}
%\numberwithin{figure}{chapter} % figures are 3.1, 3.2, 5.1,... rather than 1, 2, 3,...
\newenvironment{notation}{{\sc Notation}\it\ }{\rm}
\newenvironment{remark}{{\sc Remark}\ }{}
\newcommand{\missing}{\par{\bf *****INSERT PROOF HERE*****}\par} % a little reminder that something's missing...
\newcommand{\reml}{[\![}        % Left valuation-bracket, meaning ``remove this''
\newcommand{\remr}{]\!]}        % Right ditto
\newcommand{\uglylabel}[1]{{\hspace{-0.35in}\mbox{\small\bf #1}\hspace*{0.3in}}}
\newcommand{\fixme}{\marginpar[{\bf fix me! $\longrightarrow$}]{{\bf $\longleftarrow$ fix me!}}} % arrows in the margins...
\newcommand{\nchoose}[2]{$($\raisebox{-2pt}{$\stackrel{#1}{\scriptstyle #2}$}$)$} % in-line printing for `n choose k'.
%\newcommand{\cal}{\mathfrak} % calligraphic font is really fraktur.
\newcommand{\foobar}[1]{\makebox[1.5em]{\rule[-.5ex]{0cm}{2ex}#1}} % used to label lattices in the results section
%\newcommand{\comment}[1]{} % mimic the Scribe `comment' command.
\newcommand{\alg}[1]{{\em #1}\index{#1 algorithm}} % how to print the name of an algorithm and index it as well
\newcommand{\mmm}[2]{\alg{Merge}$(#1,#2)$} % \mmm{a}{b} prints Merge(a,b).
\newcommand{\flag}[1]{{\sc #1}\index{#1}} % flags are in small caps, and indexed
\newcommand{\bflag}[1]{{\bsc #1}\index{#1}} % bold flags, for appendix a.
\newcommand{\fval}[1]{{\sc #1}} % flag values in small caps...
\newcommand{\bfval}[1]{{\bsc #1}} % ...or bold ditto, in {description} contexts
\newcommand{\ftyp}[1]{{\sc #1}} % flag types, also in small caps.
\newcommand{\mexpr}[1]{{\tt #1}\index{#1}} % how to index a command (\command is already taken)
\newcommand{\simj}{\sim_{\!_J}} % twiddle-J.
\newcommand{\ssim}{\sim\!\!} % not, but with less space after it.
\newcommand{\nconv}{\stackrel{\sim}{\longrightarrow}}
\newcommand{\bconv}{\stackrel{\scriptscriptstyle\beta}{\displaystyle\longrightarrow}}
\newcommand{\nquiv}{\stackrel{\scriptscriptstyle\beta\sim}{\displaystyle =}}
\newcommand{\bquiv}{\stackrel{\beta}{\displaystyle =}}
\newcommand{\dquiv}{\stackrel{\scriptscriptstyle {\rm def}}{\displaystyle =}}
\newcommand{\rquiv}[1]{\stackrel{\scriptscriptstyle #1}{\displaystyle \longrightarrow}}
\newcommand{\cons}{\!^\frown} % list cons, like the ^ in a^(b)
\newcommand{\mystrut}{\rule[-0.75ex]{0ex}{2.5ex}}
\def\TPS{{\sc Tps}}  % We had a \, to ensure a space, but it looks better if we always put {\TPS}
\def\ETPS{{\sc Etps}} % or {\ETPS} (in curly brackets) and let latex determine the space.
\def\OMEGA{\mbox{$\Omega${\sc mega}}}
\def\LOUI{{\sc L$\Omega$UI}}
\def\SPASS{{\sc Spass}}
\def\OTTER{{\sc Otter}}
\def\PROTEIN{{\sc Protein}}
\def\CLAM{{\rm CL\raise.5ex\hbox{\sc a}M}}

% This is for Scribe compatibility, so that for some
% Scribe commands we can simply change the @ to a \
\def\wt{\sf}
\def\w{\sf}
\def\lisp{{\sc lisp}}
\newcommand{\indexother}[1]{#1\index{#1}}
\newcommand{\indexflag}[1]{#1\index{#1, Flag}}
\newcommand{\indexcommand}[1]{#1\index{#1, Command}}
\newcommand{\indexedop}[1]{#1\index{#1, EdOp}}
\newcommand{\indexmexpr}[1]{#1\index{#1, MExpr}}
\newcommand{\indexparameter}[1]{#1\index{#1, Parameter}}
\newcommand{\indexdata}[1]{#1\index{#1, Data}}
\newcommand{\indexfile}[1]{#1{\it \index{#1, File}}}
\newcommand{\indexfunction}[1]{#1\index{#1, Function}}
\newcommand{\indexstyle}[1]{#1\index{#1, Style}}
\newcommand{\indextypes}[1]{#1\index{#1, Type}}
\newcommand{\indexargtypes}[1]{#1\index{#1, Argument Type}}
\newcommand{\indexSyntax}[1]{#1\index{#1, Syntax}}
\newcommand{\indexProperty}[1]{#1\index{#1, Property}}
\newcommand{\indexData}[1]{#1\index{#1, Data}}
\newenvironment{Example}{ \\}{\\}

%%Those below were developed by pba

\newcommand{\indexsubject}[1]{#1\index{#1, REVIEW subject}}
\newcommand{\indexsyntax}[1]{#1\index{#1, Syntactic Object}}

\newenvironment{tpsexample}{\begin{alltt}}{\end{alltt}}
\newenvironment{lispcode}{\begin{alltt}}{\end{alltt}}
\def\not{\mathord{\sim}}

\def\and{\mathord{\wedge}}
\def\implies{\mathord{\supset}}
\def\union{\mathord{\cup}}
\def\assert{\mathord{\vdash}}
\def\one{\mathord{\overline{1}}}

%\land, \lor are defined correctly in TeX itself.
%the \or command below was found to cause trouble for the
%tabular command, and should not be used.
%Use the tex \lor command instead or \or
%\def\or{\mathord{\vee}}
