%::::::::::::::
%auto-mode.tex
%::::::::::::::

Automatic proof in {\TPS} is based on the `mating-search' paradigm described
in \cite{Andrews2002a}, \cite{Andrews81},  and \cite{Andrews89}, or
enhancements of it described in \cite{Bishop99,Bishop99a,Brown2002,
Brown2004a}.
{\TPS} starts by searching for an expansion proof
\cite{Miller84,Miller87,Andrews89}
or an extensional expansion proof \cite{Brown2004a}, and
then translates this into a natural deduction proof.

There are several ways in which one can use the automatic facilities of \TPS.


The command \indexcommand{DIY} calls the mating-search facility
(see Chapter \ref{ms-guide}), and if that succeeds in finding a proof, applies a
tactic (such as {\tt COMPLETE-TRANSFORM-TAC} or {\tt PFENNING-TAC}) to
translate the expansion tree proof  into a natural
deduction proof. Using  \indexcommand{DIY} is the simplest way to
prove a theorem automatically.

The nature of the search initiated by \indexcommand{DIY} or by
\indexcommand{GO} will be heavily influenced by which setting you have
for the flag \indexflag{DEFAULT-MS}. If you do {\it setflag default-ms} and
then respond with ? when asked for an argument, you'll see the
available options.  Help messages will tell you a little about them, and
there is some more information in Chapter \ref{ms-guide}.


If you want to see more of what is going on in the search for a proof,
you may want to start proving a theorem by entering
the  MATE top-level with a particular wff and
experimenting with the mating-search commands.  To do this, use the
\indexcommand{MATE} command. See section \ref{ms-guide} for information
about this top-level.

After a mating has been found in the mating-search top-level and the
expansion proof constructed, you can apply the proper substitutions to
the expansion tree with the \indexcommand{MERGE-TREE} command, which
also eliminates redundant branches from the tree.  Or just use the
\indexcommand{LEAVE} command, which will apply the merge procedure and
then return you to the main top-level.  To build a natural deduction
proof using the information in the expansion proof you just
constructed, use the command \indexcommand{ETREE-NAT}.  You may also
enter mating-search with the most recent expansion proof created.
This is stored in the variable {\tt \indexother{CURRENT-EPROOF}},
and is the default
value for the {\tt MATE} command.  When you leave mating-search, this
variable is set to the current expansion proof, so you may reenter
with the same proof if desired.  The value of this variable is also
the expansion proof used when translating into a natural deduction
proof with {\tt ETREE-NAT}.  Another variable, {\tt LAST-EPROOF}, stores
the value of the expansion proof created before {\tt CURRENT-EPROOF}.
Either of these symbols may be entered when prompted by the {\tt MATE}
command for a wff or eproof.

Details of how to save output from the mating search are in Chapter
\ref{output}.

Let us discuss here exactly what the effects of the various mating
search and translation commands are.
\begin{itemize}
\item The command \indexcommand{MATE}, when it returns to the top level,
alters the value of
the variable {\tt CURRENT-EPROOF} to be the expansion proof constructed.

\item The command \indexcommand{ETREE-NAT} takes the expansion proof stored
in {\tt CURRENT-EPROOF}, places the information it contains on the specified
line of the natural deduction proof, then calls the specified tactic on the
natural deduction proof.

\item The command \indexcommand{DIY} will construct an expansion proof for
the specified planned line of the current natural deduction proof,
then, like {\tt ETREE-NAT}, will place the information it contains on
the specified line(s) of the natural deduction proof and call the
specified tactic.  The value of {\tt CURRENT-EPROOF} is not changed, however.
\end{itemize}

Note that once the information from the expansion proof has been
placed into the natural deduction proof, as by {\tt DIY} and
{\tt ETREE-NAT}, the value of {\tt CURRENT-EPROOF} is unnecessary to
the translation process, and one can use the command {\tt USE-TACTIC} to
apply translation tactics.
But since the {\tt MATE} command does not transfer the information fromn
{\tt CURRENT-EPROOF} into the natural deduction proof,
the {\tt ETREE-NAT} command must be applied after {\tt MATE} in
order to start the translation process.

It is possible to translate natural deduction proofs to expansion proofs.
This is accomplished by the command {\tt NAT-ETREE}.  The expansion proof
created will be stored in the variable {\tt CURRENT-EPROOF}.  At present,
this procedure
is not complete and will fail on some natural deduction proofs, including:
\begin{itemize}
\item Those which are not cut-free, i.e., do not satisfy the subformula property

\item Those which use substitution of equality rules

\item Those which use the assertion of axioms, like reflexivity of equality
\end{itemize}





