\chapter{Mating Searches}\label{ms-guide}

\section{Expansion trees and how they grow}
See the theses and papers by Miller and Pfenning.

\section{The MATE Top-Level}

Inside the mating-search top-level, you
may examine the expansion tree, apply substitutions for variables,
etc.  If you want to actually search for a mating, use the
\indexcommand{GO} mating-search command (Note that this is a different
{\tt GO} command from the one in the main top-level. This one works only from
the mate top-level.). You can also search for a mating by typing the name
of the mating-search you wish to use.



The algorithm used by {\TPS} to search for matings can be altered by changing
the setting of the flag \indexflag{DEFAULT-MS}. At present, there are eight
possible settings for this flag, not including the matingstree procedures described in
chapter \ref{mtree}:

\begin{description}
\item[] \indexother{MS88} is Andrews' original
search procedure, as detailed in \cite{Andrews81}, which exhausts all paths through
a jform before duplicating the outermost quantifiers and trying again. MS88 will
apply primitive substitutions to a very limited extent.

\item[] \indexother{MS89} is like MS88, but will apply a variety of primitive substitutions
and duplications, and will work on several variants of
the jform simultaneously in the `option tree' style.

\item[] \indexother{MS90-3} uses Issar's `path-focused duplication' procedure to search
a single jform. See \cite{issar90} for further details.

\item[] \indexother{MS90-9} is like MS90-3, but will apply a variety of primitive substitutions
and duplications, and will work on several variants of the jform simultaneously in
the `option tree' style.

\item[] \indexother{MS91-6} is a variant of MS89, using `option sets' rather than `option trees',
which allows the user more control over the
order in which the variant jforms are considered. See section \ref{ms91} for further details.

\item[] \indexother{MS91-7} is a variant of MS90-9, using `option sets' rather than `option trees',
which allows the user more control over the
order in which the variant jforms are considered. See section \ref{ms91} for further details.

\item[] \indexother{MS92-9} is a simulation of MS88, using the code from MS90-3.

\item[] \indexother{MS93-1} is a simulation of MS89, using the code from MS90-9.

\item[] \indexother{MS98-1} is Matt Bishop's Search Procedure, Component Search.
See Matt Bishop's thesis for details.

\end{description}

Each of these flag settings is also a command in its own right, and in the mating-search
top level any of these mating searches may be invoked by simply entering its name.
When using \indexcommand{DIY} or \indexcommand{GO}, however, the mating search that
will be used is that given by \indexflag{DEFAULT-MS}.

More information on (for example) the MS88 procedure may be obtained by entering
{\it help ms88}.
Information comparing the different searches can be found in the help message for
\indexflag{DEFAULT-MS}.

Each of these procedures is governed by a number of other flags; for example, in
some of the procedures the flag \indexflag{NUM-OF-DUPS} controls the maximum number
of duplications allowed. Each of the above flag settings is also a subject heading,
and hence a full list of all the flags associated with (for example) MS88 may be obtained
by typing {\tt list ms88}.

The user may also wish to examine the help message for the flag
\indexflag{QUERY-USER}; this flag allows some interactive control over the
automatic search procedures, by letting the user specify which vpforms to ignore
and which to search on, when to duplicate quantifiers, and so on.

\section{Primitive Substitutions}

There are two distinct procedures relating to primitive substitutions in {\TPS}, and they
interact. The first of these is the actual generation of the substitutions; this is
principally governed by the \indexflag{PRIMSUB-METHOD} flag and its related flags;
the second is the way in which they are chosen and combined in the various mating searches.

Primsub generation is dealt with in the next section; it is done in exactly the same manner
for every mating search, and also for the \indexcommand{NAME-PRIM} commands in the mate and
editor top levels. Once a hashtable of primsubs has been generated, it will remain in memory
until something is done that would change the substitutions available (for example, starting
a new mating search, or changing one of the flags in the subject PRIMSUBS)

Primsub use is considerably more complex, and is dealt with differently by different mating searches.
Given the hashtable of primsubs, which will be the same for all mating searches, it is up to the
individual mating search to decide how to use this information. In general, there are three approaches
to this: do nothing, use option trees, and use option sets.

MS88 does nothing. It has a few very basic primitive substitutions built-in, and uses these instead of
the table of primsubs. MS90-3 also does nothing.

MS89, MS90-9 and MS93-1 all use option trees. In this case, a tree of substitutions is built;
the root is empty, and at each node one branch is added for each possible substitution (it helps at
this point to consider a quantifier duplication as a vacuous substitution). The tree is of course
constructed as required, since it will probably be of infinite depth. The mating search then has a fixed
amount of time to consider the vpform corresponding to each node in the tree, which it does breadth first.
So, for example, it will start by considering the vpform with no substitutions, and then proceed to
the vpform with the first available substitution, then the second, and so on. When it runs out of substitutions,
it may then go on to consider the vpform with two copies of the first substitution, then with one of the first
and one of the second, and so on. This will continue until a proof is found; since there are always more
combinations of substitutions, there is no hope of halting without a proof.

MS91-6 and MS91-7 use option sets, which are multisets of substitutions generated from the original table of primsubs.
Again, since there is no {\it a priori} limit on the size of such a multiset, there are potentially infinitely
many of them, so only a few are generated at any time. In comparison to the option tree method, which gives the
user very little control over
the order in which substitutions are considered, option sets allow the user to specify how to go about
generating the sets, and in what order to consider them. First each substitution, and then each set,
is assigned a `weight'; the search proceeds
similarly to the option-tree searches, except that instead of searching a tree breadth-first, {\TPS} instead
considers the available sets of substitutions in order from `lightest' to `heaviest'.
Although the initial hashtable of substitutions is exactly the same as for all the other mating searches,
by giving certain substitutions or combinations of substitutions `infinite' weight the user can
effectively prevent their ever being considered. Option sets are rather more difficult to use than
option trees, because the user has to specify how this weighting is to be done; section \ref{ms91}
discusses this in some detail.

\subsection{How Primsubs are Generated}

In most mating-search procedures, {\TPS} will attempt to make some substitutions
where appropriate. There are three ways in which these substitutions can be generated;
which is used will depend on the setting of the \indexflag{PRIMSUB-METHOD} (MS88 is an exception
to this general rule; it has an extremely limited set of predefined primitive substitutions,
and does not use \indexflag{PRIMSUB-METHOD}. The results will
be as follows (examples are given for X5305 with \indexflag{NEG-PRIMSUBS} NIL and
\indexflag{PRIM-BDTYPES} (`I')):

\indexother{PR89} is the original method of generating primsubs, first written for MS89.
It generates first basic substitutions: a conjunction of two literals and a disjunction
of two literals. It will also generate a projection, if the types are appropriate, and a
negation, if \indexflag{NEG-PRIMSUB} is T. Finally it generates the simplest possible
quantified substitutions: a universally quantified single literal and and existentially
quantified single literal. The types of the quantified variables, will be determined
by \indexflag{PRIM-BDTYPES}, and two such substitutions will be generated for each
bound type listed in \indexflag{PRIM-BDTYPES}.
The setting of \indexflag{PRIM-BDTYPES} may itself be determined by the setting of
\indexflag{PRIM-BDTYPES-AUTO}; read the help message for this flag for more details. Example:

\begin{tpsexample}
Var:   \(f\sp{{2}}\sb{o\alpha}\)

     PRIM1   LogConst     \(\lambda w \sp{{1}}\sb{\greeka} .f \sp{{1}}\sb{\greeko\greeka}  w \sp{{1}}  \and f \sp{{2}}\sb{\greeko\greeka}  w \sp{{1}}\)
     PRIM2   LogConst     \(\lambda w \sp{{1}}\sb{\greeka} .f \sp{{3}}\sb{\greeko\greeka}  w \sp{{1}}  \lor f \sp{{4}}\sb{\greeko\greeka}  w \sp{{1}}\)
     PRIM3   PrimQuant   \(\lambda w \sp{{1}}\sb{\greeka}   \exists w \sp{{2}}\sb{\greeki}  f \sp{{5}}\sb{\greeko\greeki\greeka}  w \sp{{1}}  w \sp{{2}}\)
     PRIM4   PrimQuant   \(\lambda w \sp{{1}}\sb{\greeka}   \forall w \sp{{2}}\sb{\greeki}  f \sp{{6}}\sb{\greeko\greeki\greeka}  w \sp{{1}}  w \sp{{2}}\)
\end{tpsexample}

\indexother{PR93} extends the above method to more general substitutions.
All of the non-quantified substitutions are generated as before, and then {\TPS}
consults the flags \indexflag{MIN-PRIM-DEPTH} and \indexflag{MAX-PRIM-DEPTH}, which
contain integers, and generates substitutions for each depth in the given range.
At depth 1, quantified substitutions are generated as in PR89.
At depth N>1, a substitution containing (N-1) quantifiers ranging over (N-1) conjunctions
(respectively, disjunctions) of (N-2) disjunctions (respectively, conjunctions). Again, these are generated
at every bound type listed in \indexflag{PRIM-BDTYPES}. Example (\indexflag{MAX-PRIM-DEPTH} 2,
\indexflag{MIN-PRIM-DEPTH} 1):
\begin{tpsexample}
 Var:   \(f\sp{{2}}\sb{\greeko\greeka}\)

     PRIM5   LogConst     \(\lambda w\sp{{3}}\sb{\greeka} .f \sp{{7}}\sb{\greeko\greeka}  w \sp{{3}}  \and f \sp{{8}}\sb{\greeko\greeka}  w \sp{{3}}\)
     PRIM6   LogConst     \(\lambda w \sp{{3}}\sb{\greeka} .f \sp{{9}}\sb{\greeko\greeka}  w \sp{{3}}  \lor f \sp{{10}}\sb{\greeko\greeka}  w \sp{{3}}\)
     PRIM7   PrimQuant   \(\lambda w \sp{{3}}\sb{\greeka}   \exists w \sp{{4}}\sb{\greeki}  f \sp{{11}}\sb{\greeko\greeki\greeka}  w \sp{{3}}  w \sp{{4}}\)
     PRIM8   PrimQuant   \(\lambda w \sp{{3}}\sb{\greeka}   \forall w \sp{{4}}\sb{\greeki}  f \sp{{12}}\sb{\greeko\greeki\greeka}  w \sp{{3}}  w \sp{{4}}\)
     PRIM9   GenSub2   \(\lambda w \sp{{3}}\sb{\greeka}   \exists w \sp{{5}}\sb{\greeki} .f \sp{{13}}\sb{\greeko\greeki\greeka}  w \sp{{3}}  w \sp{{5}}  \lor f \sp{{14}}\sb{\greeko\greeki\greeka}  w \sp{{3}}  w \sp{{5}}\)
     PRIM10  GenSub2   \(\lambda w \sp{{3}}\sb{\greeka}   \exists w \sp{{5}}\sb{\greeki} .f \sp{{15}}\sb{\greeko\greeki\greeka}  w \sp{{3}}  w \sp{{5}}  \and f \sp{{16}}\sb{\greeko\greeki\greeka}  w \sp{{3}}  w \sp{{5}}\)
     PRIM11  GenSub2   \(\lambda w \sp{{3}}\sb{\greeka}   \forall w \sp{{5}}\sb{\greeki} .f \sp{{17}}\sb{\greeko\greeki\greeka}  w \sp{{3}}  w \sp{{5}}  \lor f \sp{{18}}\sb{\greeko\greeki\greeka}  w \sp{{3}}  w \sp{{5}}\)
     PRIM12  GenSub2   \( \lambda w \sp{{3}}\sb{\greeka}   \forall w \sp{{5}}\sb{\greeki} .f \sp{{19}}\sb{\greeko\greeki\greeka}  w \sp{{3}}  w \sp{{5}}  \and f \sp{{20}}\sb{\greeko\greeki\greeka}  w \sp{{3}}  w \sp{{5}}\)
\end{tpsexample}

\indexother{PR95} generates more substitutions than PR93, but is more economical
in terms of the number of literals in the resulting jform. It works as for PR93,
except that at depth N>1 it then examines the flags \indexflag{MIN-PRIM-LITS}
and \indexflag{MAX-PRIM-LITS}; for each M in the range given by these two flags,
it generates all possible arrangements of M literals separated by conjunctions and
disjunctions, and then quantifies each of them with (N-1) quantifiers. The bound
types are determined by \indexflag{PRIM-BDTYPES}. Example (\indexflag{MIN-PRIM-LITS} 2,
\indexflag{MAX-PRIM-LITS} 3):
\begin{tpsexample}
     PRIM13   LogConst     \(\lambda w \sp{{6}}\sb{\greeka} .f \sp{{21}}\sb{\greeko\greeka}  w \sp{{6}}  \and f \sp{{22}}\sb{\greeko\greeka}  w \sp{{6}}\)
     PRIM14   LogConst     \(\lambda w \sp{{6}}\sb{\greeka} .f \sp{{23}}\sb{\greeko\greeka}  w \sp{{6}}  \lor f \sp{{24}}\sb{\greeko\greeka}  w \sp{{6}}\)
     PRIM15   PrimQuant   \(\lambda w \sp{{6}}\sb{\greeka}   \exists w \sp{{7}}\sb{\greeki}  f \sp{{25}}\sb{\greeko\greeki\greeka}  w \sp{{6}}  w \sp{{7}}\)
     PRIM16   PrimQuant   \(\lambda w \sp{{6}}\sb{\greeka}   \forall w \sp{{7}}\sb{\greeki}  f \sp{{26}}\sb{\greeko\greeki\greeka}  w \sp{{6}}  w \sp{{7}}\)
     PRIM17   GenSub2   \(\lambda w \sp{{6}}\sb{\greeka}   \exists w \sp{{8}}\sb{\greeki} .f \sp{{27}}\sb{\greeko\greeki\greeka}  w \sp{{6}}  w \sp{{8}}  \lor f \sp{{28}}\sb{\greeko\greeki\greeka}  w \sp{{6}}  w \sp{{8}}\)
     PRIM18   GenSub2   \(\lambda w \sp{{6}}\sb{\greeka}   \exists w \sp{{8}}\sb{\greeki} .f \sp{{29}}\sb{\greeko\greeki\greeka}  w \sp{{6}}  w \sp{{8}}  \and f \sp{{30}}\sb{\greeko\greeki\greeka}  w \sp{{6}}  w \sp{{8}}\)
     PRIM19   GenSub2   \(\lambda w \sp{{6}}\sb{\greeka}   \exists w \sp{{8}}\sb{\greeki} .f \sp{{31}}\sb{\greeko\greeki\greeka}  w \sp{{6}}  w \sp{{8}}  \lor f \sp{{32}}\sb{\greeko\greeki\greeka}  w \sp{{6}}  w \sp{{8}}  \)
               \(\lor f \sp{{33}}\sb{\greeko\greeki\greeka}  w \sp{{6}}  w \sp{{8}}\)
     PRIM20   GenSub2   \(\lambda w \sp{{6}}\sb{\greeka}   \exists w \sp{{8}}\sb{\greeki} .[f \sp{{34}}\sb{\greeko\greeki\greeka}  w \sp{{6}}  w \sp{{8}}  \lor f \sp{{35}}\sb{\greeko\greeki\greeka}  w \sp{{6}}  w \sp{{8}} ] \)
               \(\land f \sp{{36}}\sb{\greeko\greeki\greeka}  w \sp{{6}}  w \sp{{8}}\)
     PRIM21   GenSub2   \(\lambda w \sp{{6}}\sb{\greeka}   \exists w \sp{{8}}\sb{\greeki} .f \sp{{37}}\sb{\greeko\greeki\greeka}  w \sp{{6}}  w \sp{{8}}  \and f \sp{{38}}\sb{\greeko\greeki\greeka}  w \sp{{6}}  w \sp{{8}}  \)
               \(\lor f \sp{{39}}\sb{\greeko\greeki\greeka}  w \sp{{6}}  w \sp{{8}}\)
     PRIM22   GenSub2   \(\lambda w \sp{{6}}\sb{\greeka}   \exists w \sp{{8}}\sb{\greeki} .f \sp{{40}}\sb{\greeko\greeki\greeka}  w \sp{{6}}  w \sp{{8}}  \and f \sp{{41}}\sb{\greeko\greeki\greeka}  w \sp{{6}}  w \sp{{8}}  \)
               \(\land f \sp{{42}}\sb{\greeko\greeki\greeka}  w \sp{{6}}  w \sp{{8}}\)
     PRIM23   GenSub2   \(\lambda w \sp{{6}}\sb{\greeka}   \forall w \sp{{8}}\sb{\greeki} .f \sp{{43}}\sb{\greeko\greeki\greeka}  w \sp{{6}}  w \sp{{8}}  \lor f \sp{{44}}\sb{\greeko\greeki\greeka}  w \sp{{6}}  w \sp{{8}}\)
     PRIM24   GenSub2   \(\lambda w \sp{{6}}\sb{\greeka}   \forall w \sp{{8}}\sb{\greeki} .f \sp{{45}}\sb{\greeko\greeki\greeka}  w \sp{{6}}  w \sp{{8}}  \and f \sp{{46}}\sb{\greeko\greeki\greeka}  w \sp{{6}}  w \sp{{8}}\)
     PRIM25   GenSub2   \(\lambda w \sp{{6}}\sb{\greeka}   \forall w \sp{{8}}\sb{\greeki} .f \sp{{47}}\sb{\greeko\greeki\greeka}  w \sp{{6}}  w \sp{{8}}  \lor f \sp{{48}}\sb{\greeko\greeki\greeka}  w \sp{{6}}  w \sp{{8}}  \)
               \(\lor f \sp{{49}}\sb{\greeko\greeki\greeka}  w \sp{{6}}  w \sp{{8}}\)
     PRIM26   GenSub2   \(\lambda w \sp{{6}}\sb{\greeka}   \forall w \sp{{8}}\sb{\greeki} .[f \sp{{50}}\sb{\greeko\greeki\greeka}  w \sp{{6}}  w \sp{{8}}  \lor f \sp{{51}}\sb{\greeko\greeki\greeka}  w \sp{{6}}  w \sp{{8}} ] \)
               \(\land f \sp{{52}}\sb{\greeko\greeki\greeka}  w \sp{{6}}  w \sp{{8}}\)
     PRIM27   GenSub2   \(\lambda w \sp{{6}}\sb{\greeka}   \forall w \sp{{8}}\sb{\greeki} .f \sp{{53}}\sb{\greeko\greeki\greeka}  w \sp{{6}}  w \sp{{8}}  \and f \sp{{54}}\sb{\greeko\greeki\greeka}  w \sp{{6}}  w \sp{{8}}  \)
               \(\lor f \sp{{55}}\sb{\greeko\greeki\greeka}  w \sp{{6}}  w \sp{{8}}\)
     PRIM28   GenSub2   \(\lambda w \sp{{6}}\sb{\greeka}   \forall w \sp{{8}}\sb{\greeki} .f \sp{{56}}\sb{\greeko\greeki\greeka}  w \sp{{6}}  w \sp{{8}}  \and f \sp{{57}}\sb{\greeko\greeki\greeka}  w \sp{{6}}  w \sp{{8}}  \)
               \(\land f \sp{{58}}\sb{\greeko\greeki\greeka}  w \sp{{6}}  w \sp{{8}}\)
\end{tpsexample}

\indexother{PR97} usually generates even more substitutions than PR95.
At depth 1, it produces substitutions which are appropriately generalized versions of
the subformulas of the current gwff, each having between \indexflag{MIN-PRIM-LITS}
and \indexflag{MAX-PRIM-LITS} literals.
At depth N>1 it adds (N-1) quantifiers in front of each of the substitutions generated at
depth 1. The types of the quantified variables are determined by \indexflag{PRIM-BDTYPES}.
Example (\indexflag{MIN-PRIM-LITS} 2, \indexflag{MAX-PRIM-LITS} 3, \indexflag{MAX-PRIM-DEPTH} 1):

\begin{tpsexample}
The current gwff is THM15B:
\(\forall f \sb{\greeki\greeki} . \exists g \sb{\greeki\greeki}  [ITERATE+ f g \and  \exists x \sb{\greeki} .g x = x \and  \forall z \sb{\greeki} .g z = z \implies z = x] \)
               \(\implies  \exists y \sb{\greeki} .f y = y\)

     PRIM15   LogConst     \(\lambda w \sp{{70}}\sb{\greeki\greeki} .p \sp{{60}}\sb{\greeko(\greeki\greeki)}  w \sp{{70}}  \and p \sp{{61}}\sb{\greeko(\greeki\greeki)}  w \sp{{70}}\)
     PRIM16   LogConst     \(\lambda w \sp{{70}}\sb{\greeki\greeki} .p \sp{{62}}\sb{\greeko(\greeki\greeki)}  w \sp{{70}}  \lor p \sp{{63}}\sb{\greeko(\greeki\greeki)}  w \sp{{70}}\)
     PRIM17   SubFmSub1   \(\lambda w \sp{{70}}\sb{\greeki\greeki}   \exists w \sp{{71}}\sb{\greeki} .p \sp{{64}}\sb{\greeki\greeki(\greeki\greeki)}  w \sp{{70}}  w \sp{{71}}  = p \sp{{65}}\sb{\greeki\greeki(\greeki\greeki)}  w \sp{{70}}  w \sp{{71}}\)
     PRIM18   SubFmSub1   \(\lambda w \sp{{70}}\sb{\greeki\greeki}   \forall w \sp{{72}}\sb{\greeko\greeki} .p \sp{{66}}\sb{\greeko(\greeko\greeki)(\greeki\greeki)}  w \sp{{70}}  w \sp{{72}}  \)
               \(\implies p \sp{{67}}\sb{\greeko(\greeko\greeki)(\greeki\greeki)}  w \sp{{70}}  w \sp{{72}}\)
     PRIM19   SubFmSub1   \(\lambda w \sp{{70}}\sb{\greeki\greeki} .p \sp{{68}}\sb{\greeki(\greeki\greeki)}  w \sp{{70}}  = p \sp{{69}}\sb{\greeki(\greeki\greeki)}  w \sp{{70}}\)
     PRIM20   SubFmSub1   \(\lambda w \sp{{70}}\sb{\greeki\greeki}   \forall w \sp{{73}}\sb{\greeki\greeki} .p \sp{{70}}\sb{\greeko(\greeki\greeki)(\greeki\greeki)}  w \sp{{70}}  w \sp{{73}}  \)
               \(\implies p \sp{{71}}\sb{\greeko(\greeki\greeki)(\greeki\greeki)}  w \sp{{70}}  w \sp{{73}}\)
     PRIM21   SubFmSub1   \(\lambda w \sp{{70}}\sb{\greeki\greeki} .p \sp{{72}}\sb{\greeko(\greeki\greeki)}  w \sp{{70}}  \and  \forall w \sp{{74}}\sb{\greeki\greeki}  p \sp{{73}}\sb{\greeko(\greeki\greeki)(\greeki\greeki)}  w \sp{{70}}  w \sp{{74}}\)
     PRIM22   SubFmSub1   \(\lambda w \sp{{70}}\sb{\greeki\greeki} .p \sp{{74}}\sb{\greeko(\greeki\greeki)}  w \sp{{70}}  \implies p \sp{{75}}\sb{\greeko(\greeki\greeki)}  w \sp{{70}}\)
     PRIM23   SubFmSub1   \(\lambda w \sp{{70}}\sb{\greeki\greeki}   \forall w \sp{{75}}\sb{\greeko(\greeki\greeki)} .p \sp{{76}}\sb{\greeko(\greeko(\greeki\greeki))(\greeki\greeki)}  w \sp{{70}}  w \sp{{75}}  \)
               \(\implies p \sp{{77}}\sb{\greeko(\greeko(\greeki\greeki))(\greeki\greeki)}  w \sp{{70}}  w \sp{{75}}\)
     PRIM24   SubFmSub1   \(\lambda w \sp{{70}}\sb{\greeki\greeki}  ITERATE+ [p \sp{{78}}\sb{\greeki\greeki(\greeki\greeki)}  w \sp{{70}} ].p \sp{{79}}\sb{\greeki\greeki(\greeki\greeki)}  w \sp{{70}}\)
     PRIM25   SubFmSub1   \(\lambda w \sp{{70}}\sb{\greeki\greeki} .p \sp{{80}}\sb{\greeko(\greeki\greeki)}  w \sp{{70}}  \)
               \(\implies p \sp{{81}}\sb{\greeki(\greeki\greeki)}  w \sp{{70}}  = p \sp{{82}}\sb{\greeki(\greeki\greeki)}  w \sp{{70}}\)
     PRIM26   SubFmSub1   \(\lambda w \sp{{70}}\sb{\greeki\greeki} .p \sp{{83}}\sb{\greeki(\greeki\greeki)}  w \sp{{70}}  = p \sp{{84}}\sb{\greeki(\greeki\greeki)}  w \sp{{70}}  \)
               \(\implies p \sp{{85}}\sb{\greeko(\greeki\greeki)}  w \sp{{70}}\)
     PRIM27   SubFmSub1   \(\lambda w \sp{{70}}\sb{\greeki\greeki}   \forall w \sp{{76}}\sb{\greeki} .p \sp{{86}}\sb{\greeko\greeki(\greeki\greeki)}  w \sp{{70}}  w \sp{{76}}  \)
               \(\implies p \sp{{87}}\sb{\greeki\greeki(\greeki\greeki)}  w \sp{{70}}  w \sp{{76}}  = p \sp{{88}}\sb{\greeki\greeki(\greeki\greeki)}  w \sp{{70}}  w \sp{{76}}\)
     PRIM28   SubFmSub1   \(\lambda w \sp{{70}}\sb{\greeki\greeki}   \forall w \sp{{77}}\sb{\greeki} .p \sp{{89}}\sb{\greeki\greeki(\greeki\greeki)}  w \sp{{70}}  w \sp{{77}}  = p \sp{{90}}\sb{\greeki\greeki(\greeki\greeki)}  w \sp{{70}}  w \sp{{77}}  \)
               \(\implies p \sp{{91}}\sb{\greeko\greeki(\greeki\greeki)}  w \sp{{70}}  w \sp{{77}}\)
     PRIM29   SubFmSub1   \(\lambda w \sp{{70}}\sb{\greeki\greeki} .p \sp{{92}}\sb{\greeki(\greeki\greeki)}  w \sp{{70}}  = p \sp{{93}}\sb{\greeki(\greeki\greeki)}  w \sp{{70}}  \)
               \(\land  \forall w \sp{{78}}\sb{\greeki}  p \sp{{94}}\sb{\greeko\greeki(\greeki\greeki)}  w \sp{{70}}  w \sp{{78}}\)
     PRIM30   SubFmSub1   \(\lambda w \sp{{70}}\sb{\greeki\greeki}   \exists w \sp{{79}}\sb{\greeki} .p \sp{{95}}\sb{\greeki\greeki(\greeki\greeki)}  w \sp{{70}}  w \sp{{79}}  \)
               \(= p \sp{{96}}\sb{\greeki\greeki(\greeki\greeki)}  w \sp{{70}}  w \sp{{79}}  \and  \forall w \sp{{80}}\sb{\greeki}  p \sp{{97}}\sb{\greeko\greeki\greeki(\greeki\greeki)}  w \sp{{70}}  w \sp{{79}}  w \sp{{80}}\)
     PRIM31   SubFmSub1   \(\lambda w \sp{{70}}\sb{\greeki\greeki} .p \sp{{98}}\sb{\greeko(\greeki\greeki)}  w \sp{{70}}  \and  \forall w \sp{{81}}\sb{\greeki\greeki} .p \sp{{99}}\sb{\greeko(\greeki\greeki)(\greeki\greeki)}  w \sp{{70}}  w \sp{{81}}  \)
               \(\implies p \sp{{100}}\sb{\greeko(\greeki\greeki)(\greeki\greeki)}  w \sp{{70}}  w \sp{{81}}\)
     PRIM32   SubFmSub1   \(\lambda w \sp{{70}}\sb{\greeki\greeki} .p \sp{{101}}\sb{\greeko(\greeki\greeki)}  w \sp{{70}}  \and  \forall w \sp{{82}}\sb{\greeki\greeki}  p \sp{{102}}\sb{\greeko(\greeki\greeki)(\greeki\greeki)}  w \sp{{70}}  w \sp{{82}}  \)
               \(\implies p \sp{{103}}\sb{\greeko(\greeki\greeki)}  w \sp{{70}}\)
     PRIM33   SubFmSub1   \(\lambda w \sp{{70}}\sb{\greeki\greeki}   \forall w \sp{{83}}\sb{\greeko(\greeki\greeki)} .p \sp{{104}}\sb{\greeko(\greeko(\greeki\greeki))(\greeki\greeki)}  w \sp{{70}}  w \sp{{83}}  \)
               \(\land  \forall w \sp{{84}}\sb{\greeki\greeki}  p \sp{{105}}\sb{\greeko(\greeki\greeki)(\greeko(\greeki\greeki))(\greeki\greeki)}  w \sp{{70}}  w \sp{{83}}  w \sp{{84}}  \implies p \sp{{106}}\sb{\greeko(\greeko(\greeki\greeki))(\greeki\greeki)}  w \sp{{70}}  w \sp{{83}}\)
     PRIM34   SubFmSub1   \(\lambda w \sp{{70}}\sb{\greeki\greeki} .ITERATE+ [p \sp{{107}}\sb{\greeki\greeki(\greeki\greeki)}  w \sp{{70}} ] [p \sp{{108}}\sb{\greeki\greeki(\greeki\greeki)}  w \sp{{70}} ] \)
               \(\land  \exists w \sp{{85}}\sb{\greeki}  p \sp{{109}}\sb{\greeko\greeki(\greeki\greeki)}  w \sp{{70}}  w \sp{{85}}\)
     PRIM35   SubFmSub1   \(\lambda w \sp{{70}}\sb{\greeki\greeki}   \exists w \sp{{86}}\sb{\greeki\greeki} .ITERATE+ [p \sp{{110}}\sb{\greeki\greeki(\greeki\greeki)(\greeki\greeki)}  w \sp{{70}}  w \sp{{86}} ] \)
               \([p \sp{{111}}\sb{\greeki\greeki(\greeki\greeki)(\greeki\greeki)}  w \sp{{70}}  w \sp{{86}} ] \and  \exists w \sp{{87}}\sb{\greeki}  p \sp{{112}}\sb{\greeko\greeki(\greeki\greeki)(\greeki\greeki)}  w \sp{{70}}  w \sp{{86}}  w \sp{{87}}\)
     PRIM36   SubFmSub1   \(\lambda w \sp{{70}}\sb{\greeki\greeki} . \exists w \sp{{88}}\sb{\greeki\greeki}  p \sp{{113}}\sb{\greeko(\greeki\greeki)(\greeki\greeki)}  w \sp{{70}}  w \sp{{88}}  \)
               \(\implies  \exists w \sp{{89}}\sb{\greeki} .p \sp{{114}}\sb{\greeki\greeki(\greeki\greeki)}  w \sp{{70}}  w \sp{{89}}  = p \sp{{115}}\sb{\greeki\greeki(\greeki\greeki)}  w \sp{{70}}  w \sp{{89}}\)
     PRIM37   SubFmSub1   \(\lambda w \sp{{70}}\sb{\greeki\greeki}   \forall w \sp{{90}}\sb{\greeki\greeki} . \exists w \sp{{91}}\sb{\greeki\greeki}  p \sp{{116}}\sb{\greeko(\greeki\greeki)(\greeki\greeki)(\greeki\greeki)}  w \sp{{70}}  w \sp{{90}}  w \sp{{91}}  \)
               \(\implies  \exists w \sp{{92}}\sb{\greeki} .p \sp{{117}}\sb{\greeki\greeki(\greeki\greeki)(\greeki\greeki)}  w \sp{{70}}  w \sp{{90}}  w \sp{{92}}  = p \sp{{118}}\sb{\greeki\greeki(\greeki\greeki)(\greeki\greeki)}  w \sp{{70}}  w \sp{{90}}  w \sp{{92}}\)
\end{tpsexample}

In the mating-search and editor top levels, the command
\indexcommand{NAME-PRIM} will show you all such substitutions; you can get more
(or fewer) by adjusting the settings of the flags given in the subject \indexother{PRIMSUBS}
(type {\tt list primsubs} for a list of these flags). In the mating-search top level,
you can interactively apply a primitive substitution with one of the commands
\indexcommand{PRIM-ALL}, \indexcommand{PRIM-OUTER}, \indexcommand{PRIM-SINGLE}
or \indexcommand{PRIM-SUB} (these are all slightly different; see the help messages
for more information).

Primitve substitutions are generated for all the head variables of appropriate
type in a given wff.
Substitutions that are generated may or may not respect the
value of \indexflag{PRIM-BDTYPES}, depending on the setting of the
flag \indexflag{PRIM-BDTYPES-AUTO}. Please read the help messages for these
flags for more information.

\subsection{The MS91 Procedures}
\label{ms91}

The two procedures MS91-6 and MS91-7 take the original jform and apply, respectively,
MS88 and MS90-3 to variants of it that are produced by applying primitive substitutions
and duplications. They are very similar to MS89 and MS90-9, but allow the user more
control over the substitutions that will be tried; because of this, they have many more
flags to be set than do any of the other procedures.

The MS91 procedures first generate a list of expansion options
(an expansion option is a quantifier duplication or an
instantiation of a predicate variable), and
then generate combinations (finite sets) of these options. Since there
are an infinite number of these finite sets (because any option may be
used more than once), we generate the option sets a few at a time.
The maximum number of sets generated at any one time is governed
by the flag \indexflag{NEW-OPTION-SET-LIMIT}.

The basic idea of the enumeration scheme is that there is a
weight assigned to each set of options. Starting with the weight
of the initial problem (with the empty set of options), we then
look for variants whose weight is within an acceptable range (given by
\indexflag{MS91-WEIGHT-LIMIT-RANGE}) of some target value. These weights
may correspond to new option sets, or they may correspond to old option
sets that are being reconsidered. If we fail
to find any such variants, we increase the acceptable value and try again.
So our objective is to ensure that the simplest substitutions are assigned
the smallest weights. It is possible for a set to be assigned an infinite
weight, in which case it will never be considered.

The weight is the sum of three
separate weights, which we denote \indexother{weight-a}, \indexother{weight-b}
and \indexother{weight-c}. The user is given control over the
weights by means of the settings of the flags \indexflag{WEIGHT-A-FN},
\indexflag{WEIGHT-B-FN}, \indexflag{WEIGHT-C-FN}, \indexflag{WEIGHT-A-COEFFICIENT},
\indexflag{WEIGHT-B-COEFFICIENT} and \indexflag{WEIGHT-C-COEFFICIENT}.
The total weight of each option set is initially the sum of each of the three
functions evaluated on the set, multiplied by each of the three
coefficients. Asking for help on these flags will give you a list of
the possible functions for each flag; asking for help on MS91-6 or MS91-7
will give you an summary of this section of the manual.

Weight-a should in some way measure
the complexity of an individual option (and so the weight-a of the
option set is the sum of the weight-a's of each of its elements).
So one possible value for \indexflag{WEIGHT-A-FN} is \indexother{EXPANSION-LEVEL-WEIGHT-A},
which assigns to each option its expansion level (i.e. how late in the list of options
it is).

Weight-b should calculate the weight of the entire option set, by
considering the number of duplications and substitutions it contains.
So one possible value for \indexflag{WEIGHT-B-FN} is \indexother{SIMPLE-WEIGHT-B-FN},
which sums up the  various penalties for substitutions and duplications given
in the flags \indexflag{PENALTY-FOR-EACH-PRIMSUB}, \indexflag{PENALTY-FOR-MULTIPLE-PRIMSUBS}
and \indexflag{PENALTY-FOR-MULTIPLE-SUBS}. Another possible value is
\indexother{SIMPLEST-WEIGHT-B-FN}, which adds up the number of substitutions and
duplications and returns that number plus one.
Yet a third possible value is \indexother{ALL-PENALTIES-FN}, which evaluates
\indexother{SIMPLE-WEIGHT-B-FN} and then adds the penalty given by the flag
\indexflag{PENALTY-FOR-ORDINARY-DUP}.

Weight-c should measure the complexity of the resulting jform.
So two possible values for \indexflag{WEIGHT-C-FN} are \indexother{OPTION-SET-NUM-VPATHS}
and \indexother{OPTION-SET-NUM-LEAVES}, which count up (respectively) the number of paths
or leaves in the resulting jform.

The procedures will work on each jform for a limited time (given by the
flag \indexflag{SEARCH-TIME-LIMIT}), before assigning a new weight to the option
set and proceeding to the next jform. This means that the procedure will only
give up searching on a particular jform if it has discovered that there can be no
successful mating, or if the new weight that is assigned is INFINITY.
The user can control the value of the new weight by changing the setting
of the flag \indexflag{RECONSIDER-FN}.

So the structure of the MS91 procedures looks like this:
\begin{enumerate}
\item Work on the original problem for \indexflag{SEARCH-TIME-LIMIT}
seconds using either MS88 or MS90-3 as appropriate.

\item If that fails, change the weight of the old problem (using \indexflag{RECONSIDER-FN}).

\item Generate new options (which will be a list of possible duplications and
primitive substitutions for the expansion variables of the problem).

\item Using the list of options, attempt to generate \indexflag{NEW-OPTION-SET-LIMIT} many
new option sets whose weights lie within \indexflag{MS91-WEIGHT-LIMIT-RANGE} of the
weight of the last option set used. If we fail to do this because no more option
sets can be generated from the current option list, give up and go on to the next step.
If we fail because all the option sets we generate are not in our desired weight range,
increase the lower limit of this range and try again.

\item Apply either MS88 or MS90-3 to each of the new option sets, in order of their
weight (may also reconsider old option sets; an option set expires permanently after \indexflag{MAX-SEARCH-LIMIT}
seconds in total have been spent on it). Each set gets considered for
\indexflag{SEARCH-TIME-LIMIT} seconds at a time.

\item When all of the new option sets have been used we call
the function given by \indexflag{OPTIONS-GENERATE-FN}, which will return T if new options should be
generated. Depending on the setting of this flag, this might be because we have generated all
possible new option sets from the current list of options, or because the current option sets
have become too complicated in some sense. If \indexflag{OPTIONS-GENERATE-FN} returns NIL
then we go back to step 4; otherwise
return to step 3 (note that the problem will usually now contain new expansion variables
because of the primitive substitutions).
\end{enumerate}

\section{Some Important Flags in ms90-3 Search}
\begin{itemize}
\item \indexflag{NUM-OF-DUPS}
The flag determines the maximum number of duplications allowed
on a path. In the extreme case, i.e., when num-of-dup is 0, a
universal jform can still appear several times on different paths
as long as it appears at most once on each path. It is exceedingly
crucial to keep the flag low while you are dealing with higher
problems. This could be achieved by reorganizing jforms, sometimes.

\item \indexflag{MAX-MATES} The flag determines the maximum number of times
a literal could occur in a mating. Please always remember that
literals generated by duplications on a path are regarded as a new
one, which bears no relations with the old ones. Hence, both the new
ones and the old ones can occur in a mating up to
\indexflag{MAX-MATES} times. You may often increment
\indexflag{MAX-MATES} so that you can decrement
\indexflag{NUM-OF-DUPS}, or vice versa. In the higher order case, it
is usually worth augmenting \indexflag{MAX-MATES} to keep
\indexflag{NUM-OF-DUP} low.

\item \indexflag{ORDER-COMPONENTS} The flag gives you a few choices to
reorganize the jform through which you want to search. Setting it to T
often reduces the minimum value of \indexflag{NUM-OF-DUPS} we need to
find a mating. When the jforms under predicate variables are a big
disjunction, it is recommended that you set \indexflag{ORDER-COMPONENTS}
to T so that you might have a chance to set \indexflag{NUM-OF-DUPS} to 0.
This expedites the whole search process dramatically.
\end{itemize}

\section{Helpful Hints for MS91}
\begin{itemize}
\item Setting \indexflag{MS91-WEIGHT-LIMIT-RANGE} to INFINITY means that the next option
set that is generated will always be accepted (unless it has weight INFINITY); this
will spped things up, at the possible expense of generating heavier option sets
before lighter ones.

\item Setting \indexflag{NEW-OPTION-SET-LIMIT} to 1 will minimize the delay between each search.
However, it's quicker to generate a batch of option sets all at once rather than individually,
so this may make the total time for the search slightly longer.

\item Setting \indexflag{RECONSIDER-FN} to INF-WEIGHT will prevent any option set from being
considered more than once.

\item Try not to use everything all at once; there's probably no need to use
weights a, b and c all together. Try setting one or two of their coefficients to 0.

\item Judicious use of INFINITY as a penalty will get you to your goal a lot
faster; so (for example) \indexflag{PENALTY-FOR-ORDINARY-DUPS} INFINITY will ensure
that you never get a vpform with any unused duplications, or
\indexflag{PENALTY-FOR-MULTIPLE-SUBS} INFINITY will ensure that each variable gets
at most one primsub. (However, generating sets with weight INFINITY still
takes time; if it seems to pause for a while before producing a new option
set, it may well be producing and ignoring a batch with weight INFINITY...)

\item Generating new options takes a considerable time (because the jform will have gone from
having one or two expansion variables to having ten or twenty of them).
If your search is likely to reach
the stage of generating new options, then:
\begin{itemize}
\item You could try increasing the number of primsubs produced in the first round of options,
by increasing the value of \indexflag{MAX-PRIM-DEPTH}; this might allow the search to
finish without generating new options. (In general, the set of primsubs produced by having
\indexflag{MAX-PRIM-DEPTH} greater than 1 is a subset of those produced by repeatedly
generating new options with \indexflag{MAX-PRIM-DEPTH} equal to 1.)

\item You should consider making weight-c or weight-a the dominant weight function,
as opposed to weight-b.
(This is because the new options will appear to be single primitive substitutions, just like
the old ones, and thus weight-b will be unable to distinguish them).
\end{itemize}

\item Use the \indexcommand{SEARCH-ORDER} command to see in what order the option sets will
be generated.

\item Compare your flag settings to the four preset modes \indexother{MS91-SIMPLEST}, \indexother{MS91-ORIGINAL},
\indexother{MS91-NODUPS} and \indexother{MS91-DEEP}; these modes are four of the most common
ways in which these flags can usefully be set.
\end{itemize}

% \begin{comment} (This is the old description, prior to  Mar 1994.)
% The two procedures MS91-6 and MS91-7 have a considerable number of flags which
% govern the order in which the various primitive substitutions are applied to the
% jform. The substitutions that are used are limited to those which are shown
% when you type \indexcommand{NAME-PRIM} (in particular, you should check that
% the settings of \indexflag{MAX-PRIM-DEPTH} and \indexflag{MIN-PRIM-DEPTH} are
% appropriate before beginning an automatic search with these procedures).
% 
% An expansion option is a quantifier duplication or an
% instantiation of a predicate variable. In general, we shall take care
% of duplications as part of the search process. Thus, an expansion
% option is essentially a wff to be substituted for a particular
% predicate variable; it may contain additional predicate variables.
% In principle, one may need to use certain options more than once,
% though in simple examples this rarely happens.
% 
% The MS91 procedures first generate a list of expansion options, and
% then generate combinations (finite sets) of these options. Since there
% are an infinite number of these finite sets (because any option may be
% used more than once), we generate the option sets a few at a time.
% The maximum number of sets generated at any one time is governed
% by the flag \indexflag{NEW-OPTION-SET-LIMIT}.
% 
% The basic idea of the enumeration scheme is that there is a
% weight assigned to each set of options. Starting with the weight
% of the initial problem (with the empty set of options), we then
% look for variants whose weight is within an acceptable range (given by
% \indexflag{MS91-WEIGHT-LIMIT-RANGE}) of some target value. These weights
% may correspond to new option sets, or they may correspond to old option
% sets that are being reconsidered. If we fail
% to find any such variants, we increase the acceptable value and try again.
% So our objective is to ensure that the simplest substitutions are assigned
% the smallest weights. It is possible for a set to be assigned an infinite
% weight, in which case it will never be considered.
% 
% The weight is the sum of three
% separate weights, which we denote \indexother{weight-a}, \indexother{weight-b}
% and \indexother{weight-c}. The user is given control over the
% weights by means of the settings of the flags \indexflag{WEIGHT-A-FN},
% \indexflag{WEIGHT-B-FN}, \indexflag{WEIGHT-C-FN}, \indexflag{WEIGHT-A-COEFFICIENT},
% \indexflag{WEIGHT-B-COEFFICIENT} and \indexflag{WEIGHT-C-COEFFICIENT}.
% The total weight of each option set is initially the sum of each of the three
% functions evaluated on the set, multiplied by each of the three
% coefficients. Asking for help on these flags will give you a list of
% the possible functions for each flag; asking for help on MS91-6 or MS91-7
% will give you an summary of this section of the manual.
% 
% Weight-a should in some way measure
% the complexity of an individual option (and so the weight-a of the
% option set is the sum of the weight-a's of each of its elements).
% So one possible value for \indexflag{WEIGHT-A-FN} is \indexother{EXPANSION-LEVEL-WEIGHT-A},
% which assigns to each option its expansion level (i.e. how late in the list of options
% it is).
% 
% Weight-b should calculate the weight of the entire option set, by
% considering the number of duplications and substitutions it contains.
% So one possible value for \indexflag{WEIGHT-B-FN} is \indexother{SIMPLE-WEIGHT-B-FN},
% which sums up the  various penalties for substitutions and duplications given
% in the flags \indexflag{PENALTY-FOR-EACH-PRIMSUB}, \indexflag{PENALTY-FOR-MULTIPLE-PRIMSUBS}
% and \indexflag{PENALTY-FOR-MULTIPLE-SUBS}. Another possible value is
% \indexother{SIMPLEST-WEIGHT-B-FN}, which adds up the number of substitutions and
% duplications and returns that number plus one.
% 
% Weight-c should measure the complexity of the resulting jform.
% So two possible values for \indexflag{WEIGHT-C-FN} are \indexother{OPTION-SET-NUM-VPATHS}
% and \indexother{OPTION-SET-NUM-LEAVES}, which count up (respectively) the number of paths
% or leaves in the resulting jform.
% 
% The procedures will work on each jform for a limited time (given by the
% flag \indexflag{SEARCH-TIME-LIMIT}), before assigning a new weight to the option
% set and proceeding to the next jform. This means that the procedure will only
% give up searching on a particular jform if it has discovered that there can be no
% successful mating, or if the new weight that is assigned is INFINITY.
% The user can control the value of the new weight by changing the setting
% of the flag \indexflag{RECONSIDER-FN}.
% 
% 
% \subsection{Examples}
% 
% To enumerate the variant jforms in as simple a manner as possible, set WEIGHT-A-COEFFICIENT
% and WEIGHT-C-COEFFICIENT to 0, set WEIGHT-B-COEFFICIENT to 10 and set
% WEIGHT-B-FN to SIMPLEST-WEIGHT-B-FN.
% 
% Another setting that works reasonably well is to set all the coefficients to 1,
% WEIGHT-A-FN to EXPANSION-LEVEL-WEIGHT-A, WEIGHT-B-FN to SIMPLE-WEIGHT-B-FN and
% WEIGHT-C-FN to OPTION-SET-NUM-LEAVES, and then set all the various PENALTY- flags
% to 3.
% \end{comment}
\section{The Matingstree Procedure}\label{mtree}

Note: The ideas and resources described in this chapter are
experimental and not well developed. Most users of TPS should
probably ignore them.

\subsection{A Brief Overview}

The matingsearch procedures already described take a vpform, enumerate all possible
paths through it, and attempt to block them all one by one. This is not the way that
a human operator, attempting the same problem, would proceed, and the matingstree
top level is an attempt to approach the problem in a more natural way. For example,
suppose that a vpform contains a conjunct (B OR C) and a single literal A. Then if
A is mated with B on some path, this closes all extensions of that path through B, but
creates an obligation to `do something' about C - which is to say, to block off
all variants of the path that go through C instead of through B. The matingstree
procedure is an attempt to formalize this notion of `obligation' and so to direct the proof
search.


\subsection{A Detailed Plan of the Matingstree Top Level}

	While TPS currently has a number of search procedures, they
all construct matings by searching paths through the formula in a very
systematic way.  This allows TPS to keep track of where it is in its
search process in a very economical way, so that TPS can run for weeks
without running out of memory. However, this very economical use of
space limits our ability to implement various search heuristics within
the context of current search procedures. In particular, we would like
to develop methods of searching for expansion proofs which closely
mimic attempts to construct natural deduction proofs.

	We shall describe a much more general way of organizing the
search process which, at the cost of using additional space, will
enable us to choose links to add to matings in arbitrary order, to
keep track of where we are in the search process, and to
simultaneously build up many matings (thus using a mixture of
breadth-first and depth-first search).  Information will be stored
which will enable TPS to know when it has spanned all paths without
actually searching through them all.  We need to avoid searching all
paths (since there are generally too many), and we need to construct a
mating in such a way that at a certain point we know without further
checking that we have an acceptable mating.  We ignore parts of the
wff (or potentially relevant lemmas which are not part of the wff)
until we explicitly make them part of the search process.

	We would like to avoid backtracking, which is troublesome and
time-consuming. By keeping more information, we can just abandon certain
parts of the search and turn our attention to others.

	We would like expansions of the jform (quantifier duplications
and applications of primitive projections and substitutions, plus
subformula substitutions for predicate variables) to be motivated by
the needs of the matingsearch process. Thus, this procedure should
incorporate some of the key advantages of path-focused duplication
(but will be `literal-focused' duplication rather than `path-focused'
duplication).

	The emphasis will be on being smart rather than fast. We will
try to compute whatever information is necessary to limit the size of
the search space, and give first priority to exploring those parts of
the search space which seem (according to our heuristics) to be most
likely to yield success. However, we will try to do this computation
efficiently. Sometimes a `partial computation' will suffice to make it
clear whether the complete computation is worth doing.

	We will try to manage our use of space effectively. When we
reach a stage where we are ready to discard certain structures, we
will do so in a way that will permit the space to be reclaimed by the
garbage collector. Some structures could be regenerated if necessary.
It would be nice to assign priorities to structures, and have a
mechanism for discarding those of low priority when more space is
needed.


	We shall sometimes speak loosely and call any set of connections
a mating, although officially a mating must have an associated
substitution which makes mated pairs complementary.

	As the search process progresses, we build up a tree of
matings, or matingstree (MST). By making use of the tree structure, we
can hope to store the information economically.  Each node in this
tree represents a step in the search process. In general, a node
represents a mating which was obtained by adding a link to the mating
of its parent node.  Each node has a number of attributes:
\begin{itemize}
\item its mating;

\item a set of obligations (see below);

\item auxiliary information about items such as:

\item free variables,

\item the unification tree for the mating,

\item potential mates for various literals.
\end{itemize}

Definition: A literal L is on some extension of the path P in the
jform W iff no disjunction of W contains L in one of its scopes (left
or right) and some literal which is on P in the other scope (right or
left).

	An obligation consists of a path and a node of the jform on
that path in the jform. When all the obligations for a node of the MST
are fulfilled, the mating will be complete (i.e., span every path).
Thus, we seek to construct a node of the MST which has an empty set of
obligations.


We shall have to consider how obligations can be represented
economically.  It seems that the nodes of the path of an obligation
are all literals in links of the mating associated with the node of
the mst where the obligation first appeared. Perhaps such paths can
be represented implicitly. (For the case where the wff is in cnf and
in fol, compare with the model-elimination procedure.)


	The initial node of the MST has an empty mating, and one
obligation: the node is the entire wff to be refuted, and the path
contains only that node.

	We express the process of constructing a mating in terms of
`blocking' a node N of an expansion tree relative to a path P
containing that node. We imagine an abstract little creature called a
`spoiler' trying to run through the expansion tree (or the jform
corresponding to it), and we must block all its attempts.

A node can be blocked in various ways (and
heuristics will help us decide which tasks to work on first):

\begin{itemize}
\item To block a conjunction, block any of its conjuncts.
One way to do this is to mate two of its conjuncts with each other.

\item To block an expansion node, block any of its expansions.

\item To block a disjunction, block all of its disjuncts.

\item To block a definition or selection node, block its
unique child in the tree.

\item To block a literal N, mate it with some literal of that path.

\item To block a literal N, mate it with some literal L not yet on the
path P but on some extension of it
and establish the set of subtasks
$\{<M, P \union{M}> | M is a literal on some horizontal path through L\}$.
(This amounts to blocking the disjunction node D  containing L
relative to the path $P \union{L}$.) The literal L may be generated
by making a new instantiation of some expansion node.
This might be a quantifier duplication or a primitive substitution
or a sequence of primitive substitutions.

\item A contradictory node (which might arise by instantiating a
predicate variable with FALSEHOOD) is blocked.

\item Another way to block a node is to use equations to mate it
with another node as above, or to reduce it to $ \not{ C = C}$.
Thus we implement equational matings. (Of course, our present
translation command etree-nat won't work on such matings.)

\end{itemize}

We visualize the root of the matingstree (i.e., the empty mating)
as being at the top of the tree.

CGRAPHS

In principle, there should be a connection graph (cgraph)
associated with each node of the matingstree which is obtained from
cgraphs of nodes higher in the matingstree by deleting links which are
not compatible with the mating for that node. Thus, the unifying
substitution (to non-branching depth) associated with the mating
should be used in computing the cgraph.  We don't actually compute
these cgraphs completely; we just put into them information which is
computed as it is needed. Also, instead of computing certain cgraphs
we may just use cgraphs associated with higher nodes in the tree.
In addition, we may associate with each node of the tree
a set of links which are incompatible with the mating at that node.
We call these links `negative'.
Thus, one obtains the cgraph for that node by deleting from the
original cgraph (the cgraph for the root node) all the `negative
links' associated with nodes at or above that node in the matingstree.
(As one adds links to the mating, one also accumulates more negative
links.)

When we try to add a link to a mating and find that it is
incompatible with it, we add that link to the set of negative links
for that node of the matingstree, and thus delete that link from the
positive cgraph for that node.

Eventually, we should compute and represent these cgraphs in as
sophisticated way as possible. Instead of blindly checking all links
which occur in the parent cgraph to see whether they can be deleted,
we should consider which variables are actually constrained by the
relevant links of the mating, and which literals these variables
actually occur in. Review for relevant ideas the paper
Robert Kowalski, `A Proof Procedure Using
 Connection Graphs', Journal of the ACM 22 (1975), 572-595.

We find how we can satisfy obligations by looking in the cgraph.

A node (mating) fails if:
it is found not to be unifiable, or
it is found that there is no way to satisfy one of its obligations.
When a node fails, we delete the link that created it from the
cgraph of the node above it. If this eliminates the last possible
way of fulfilling some obligation of that node, it will also fail.

The search process consists of performing tasks. A task consists of
creating a new node N2 of the MST which is a son of an existing node
N1, and fulfills an obligation of N1. N2 inherits all the obligations
of N1 except for the one it fulfilled, and it may also have additional
obligations.

Each task has a priority (which may change as the search progresses?),
and the priorities determine in what order tasks are performed. If
many processors are available, they each choose the next available task of
of highest priority.

At each stage we can ask questions:
What node (mating) should I work on?
What obligation should I work on?
How shall I try to satisfy that obligation?
Heuristics can be used to answer each of these questions.
If one has even very naive ways to answer all of these questions,
one has an automatic search procedure.

Search heuristics are generally expressed as ways of setting the priorities.
One heuristic: when we do things which create obligations, check quickly
whether those obligations can be simultaneously satisfied.
Another: concentrate on satisfying obligations which can be
satisfied only in a small number of ways (ideally one).

We may need some way to prevent the same mating from being generated on
different nodes of the MST. This is analogous to the subsumption problem
in resolution, and we may have to deal with it explicitly. Perhaps it will help
to order the obligations of a node, have this ordering be inherited by
descendant nodes, and insist that obligations be filled in this order.

Compare this procedure with model-elimination (etc.) when it is applied
to wffs of fol in cnf.
This procedure seems to incorporate the key idea in the resolution
set-of-support strategy.

        This procedure should be designed to deal with the following
problem:

	TPS can't prove THM120F because it doesn't apply apply
unifying substitutions to the jform as it proceeds. Thus projections
for head variables which which could produce contradictions on a path
don't actually get applied during the matingsearch process. Of course,
pure projections eventually get generated and applied as part of
mating-search, but for THM120F we have a literal r[p v $\sim$p]Q, and for
r we substitute [lambda u. lambda v .$\sim$u] in order to deal with other
pairs of literals. If this were actually applied to the jform, it would
yield a contradiction, but it is not applied.

	Note that this is a complicated issue which is not easy to
deal with, since at each stage we have a whole unification tree
associated with the current mating, and we do not have any one
current substitution to apply.

	We need to consider how to economically keeping track of what
we've tried with matingstree procedure. Ideas for possible solutions:
\begin{itemize}
\item Attach heuristic weights to various things we might try, and order
them by weight.

\item Whatever algorithm generates possibilities essentially enumerates
them. Just keep a list of numbers (under this enumeration) of
possibilities already explored.
\end{itemize}

\subsection{How to Use the Mtree Top Level}

The mtree top level is entered with the command \indexmexpr{MTREE}. This is
deliberately designed to be as similar to the {\tt MATE} top level as possible,
in order not to be too confusing. The \indexcommand{LEAVE} command leaves the top level
again, and will prompt the user about merging the expansion tree if it detects that the
proof has been finished. Everything uses the MS88 notation and MS88-style unification.

For brevity, we will refer to the matingstree as `the mtree', and an obligation tree
as `an otree'. There are many otrees (one for each node of the mtree); if we refer to `the otree'
this is to be interpreted as `the obligation tree associated to the current node of the
matingstree'. An obligation which is satisfied is also said to be `closed'; obligations
which are not closed are, naturally, referred to as `open'. A node of the mtree is said
to be `closed' if it has no open obligations left in its otree.

There are a wide variety of printing commands, since we have a lot of structures present.
\indexcommand{PM-NODE} prints the current node of the mtree in full detail; \indexcommand{POB-NODE}
does the same for the current obligation of the otree (there are potentially many open obligations
in an otree; the `current' one will be defined by the setting of the flag \indexflag{DEFAULT-OB}).
\indexcommand{POB} prints the current obligation in minimal detail.

\indexcommand{PMTR}, \indexcommand{PMTR*} and \indexcommand{PMTR-FLAT} print the matingstree, starting from
the current node (by default; they can also all take a node as an optional argument and start from that node instead)
and working downwards towards the leaves, in varying formats and degrees of detail.

\indexcommand{POTR}, \indexcommand{POTR-FLAT} and \indexcommand{POTR*-FLAT} do the same for the otree.

\indexcommand{PPATH} and \indexcommand{PPATH*} print all the obligations from the given (leaf) obligation
to the root of the obligation tree.

\indexcommand{CONNS-ADDED} shows which connections have already been added to this mtree node, if any.
\indexcommand{LIVE-LEAVES} shows which leaves of the tree are not marked as dead (the search procedure will mark a
node as dead if it has open obligations but they cannot be satisfied, or if it has decided to give up on it for
some other reason).
\indexcommand{SHOW-MATING} shows the mating associated with the current mtree node and \indexcommand{SHOW-SUBSTS}
shows the substitution stack at the current node (built up as the connections are unified).

There are also a collection of commands for moving about in the mtree; \indexcommand{UP} and \indexcommand{DOWN}
are fairly clear; \indexcommand{SIB} goes to the next sibling of the current mtree node, and \indexcommand{GOTO}
goes to a node by name. \indexcommand{INIT} starts a new mtree, with a single root node and nothing else.
\indexcommand{KILL} marks a node as dead; \indexcommand{RESURRECT} unmarks it. \indexcommand{PRUNE} removes
all dead nodes below the current node; \indexcommand{REM-NODE} removes a single node.

\indexcommand{ADD-CONN} adds a connection, as in the {\tt MATE} top level; this will generate a new mtree node,
with a new otree; this mtree node will become the current node.

Each time a new connection is added, new copies of all the expansion nodes involved are made, and the connection
is made between these new copies rather than between the original literals. This allows the whole mtree to refer to
a unique master expansion tree, rather than lots of smaller expansion trees. It also means that, at any given node, most
of the master expansion tree will be irrelevant; when a closed
mtree node is reached, the \indexcommand{CHOOSE-BRANCH} command discards all other branches of the tree and
trims the expansion tree down to just those parts that are relevant to the closed node. This allows us to use the same
merging functions as in the {\tt MATE} top level.

\subsection{Automatic Searches with the Mtree Top Level}

First, we have some semi-automatic commands: \indexcommand{QRY} takes a literal and an obligation as arguments,
and returns all possible mates for them. \indexcommand{ADD-ALL-LIT} uses {\tt QRY} to simply add all these possible
mates as sons of the current mtree node. \indexcommand{ADD-ALL-OB} does {\tt ADD-ALL-LIT} for every literal
in a given obligation, and \indexcommand{EXPAND-LEAVES} does {\tt ADD-ALL-OB} at every leaf node of the mtree.

It is clear that, time and space being no object, \indexcommand{EXPAND-LEAVES} is complete, in the sense
that if it possible to arrive at a proof in the mtree top level at all, then repeating {\tt EXPAND-LEAVES} will eventually
get you to a closed mtree node. (Notice that the mtree top level does not currently support primitive substitutions,
so not everything that is provable in {\tt MATE} is provable in {\tt MTREE}.)
The automatic procedure MT94-11 does exactly this. (As in {\tt MATE}, you can call this either directly
with \indexcommand{MT94-11}, or indirectly by setting \indexflag{DEFAULT-MS} to {\tt MT94-11} and calling
\indexcommand{GO}. The same applies to the other searches.)

\indexcommand{MT94-12} is a restricted version of \indexcommand{MT94-11}; for each leaf node of the tree,
using the integer flag
\indexflag{MT94-12-TRIGGER}, it decides whether the current obligation has more or fewer literals than
the current setting of this flag. If more, it just applies \indexcommand{ADD-ALL-LIT} to whichever single
literal has the fewest possible mates. If fewer, it applies \indexcommand{ADD-ALL-OB}.

\indexcommand{MT95-1} is still further restricted; it chooses the mtree node with the fewest open obligations,
and then applies {\tt MT94-12} to that node only, and repeats the process.

\subsection{The Mtree Subsumption Checker}

There are a number of approaches to subsumption checking in the mtree top level; the possible benefits
are very great, since not only the number of leaves in
the mtree but also the size of the expansion tree can be kept down by killing off mtree
nodes that are effectively duplicates of existing nodes.

The subsumption checker is governed by the flag \indexflag{MT-SUBSUMPTION-CHECK}. Setting this flag to {\tt NIL}
will turn off subsumption checking altogether.

The weakest subsumption check is {\tt SAME-TAG}; this uses the flags \indexflag{TAG-CONN-FN} and \indexflag{TAG-LIST-FN}
to generate a number corresponding to the connections in the mating at the current node. A new node is
then rejected if its tag is the same as that of some existing node. This method is very quick, it is also
obviously unsound unless one can guarantee that no two different nodes can get the same tag. Nonetheless, it
will probably be correct almost all the time...

The next strongest is {\tt SAME-CONNS}; this first checks as in {\tt SAME-TAG}, but then if the tags match, it actually
checks the matings to make sure they really are the same. This is sound, but possibly not restrictive enough,
since (e.g.) connection (leaf3 . leaf7.1) may effectively be the same as (leaf3 . leaf7.2), depending on
what other literals are contained in the same expansions as leaf 7.1 or 7.2 (which are copies of each other,
in standard {\TPS} notation), and whether any of them are mated to anything else.
The more accurate subsumption check, which takes this into account, has yet to be written. Meanwhile, {\tt SAME-CONNS}
will never wrongly reject a node, but may accept nodes that could have been rejected.

The strongest of all is {\tt SUBSET-CONNS}; this checks whether the new node contains a subset of the
connections at some other node, and rejects it if it does. This is much too strong for any search except {\tt MT94-11},
because only {\tt MT94-11} generates all possible successors to a given node all at once. (Example: Suppose the
first connection we added was A, and we are now at a stage where
we have (somehow) got a node that contains connections ABC, and we are about to generate one containing just AB, and the
correct mating is in fact ABD. Then rejecting the new node would seem to be wrong; it's only all right in MT94-11
because we know that node AD has already been generated elsewhere in the tree.)

\subsection{An Interactive Session in the Mtree Top Level}

\begin{tpsexample}
<4>mtree x2106 !
{\it We choose a simple example; X2106 says that, for all x, if (Rx implies Px} and ((not Qx) implies Rx),
then for all x either Px or Qx is true.)
<Mtree:5>pob

OB0

|FORALL x^2         |
| |LEAF6     LEAF7| |
| |\(\sim\)R x^2 OR P x^2| |
|                   |
|FORALL x^3         |
| |LEAF11    LEAF10||
| |Q x^3  OR R x^3 ||
|                   |
|      LEAF13       |
|      \(\sim\)P X^1       |
|                   |
|      LEAF14       |
|      \(\sim\)Q X^1       |

{\it This command prints the current obligation at the current node of the
matingstree. Each node of the mtree has an associated obligation tree,
which may have many open obligations; which of these is the `current'
obligation is determined by the flag DEFAULT-OB. At this point, the
mtree has exactly one node, and the otree at that node contains exactly
one obligation, which is the entire formula}

<Mtree:6>add-conn
LITERAL1 (LEAFTYPE):  [No Default]>6
OBLIG1 (SYMBOL-OR-INTEGER):  [OB0]>
LITERAL2 (LEAFTYPE):  [No Default]>10
OBLIG2 (SYMBOL-OR-INTEGER):  [OB0]>
Adding new connection: (LEAF6.1 . LEAF10.1)
MSTREE1

{\it We have added a connection between leaf6 and leaf10, both in obligation
number 0. This has created a new mtree node, MSTREE1. We have also moved
down the tree to this new node, which has become the current node.}

<Mtree:7>potr

Numbers in round brackets are open obligations; those in square
brackets are closed. Branches with *'s denote nodes that are
being omitted for lack of space.


                                     MSTREE1
                                      (OB0)
                                        |
                    /--------------------------------------\
                    |                                      |
                  (OB1)                                  (OB2)
                    |                                      |
          /-------------------\
          |                   |
        (OB3)               [OB4]

{\it Here is the obligation tree at the new node MSTREE1. We started from
the obligation OB0, above. Adding that one connection first broke up the
disjunction of leaf6 and leaf7, to form OB1 and OB2. We then connected leaf6
to leaf10, so the disjunction of leaf10 and leaf11 was also broken up to
create OB3 and OB4 on the path going through leaf6; we satisfied OB4 by
connecting leaf10 to leaf6, but OB3 is still open.}

<Mtree:8>pob

OB3
LEAF11.1
 Q x^5

{\it Our current obligation is OB3. However, this doesn't show what we can
mate it to, so we try a different command.}

<Mtree:9>ppath*

OB3
LEAF11.1
 Q x^5

OB1
LEAF6.1
\(\sim\)R x^4

OB0
|FORALL x^2         |
| |LEAF6     LEAF7| |
| |\(\sim\)R x^2 OR P x^2| |
|                   |
|FORALL x^3         |
| |LEAF11    LEAF10||
| |Q x^3  OR R x^3 ||
|                   |
|      LEAF13       |
|      \(\sim\)P X^1       |
|                   |
|      LEAF14       |
|      \(\sim\)Q X^1       |

{\it This shows the branch of the obligation tree that leads to our current obligation.
It seems clear that we can connect the single literal of OB3 to leaf14 in OB0 to close
off this path, so that's what we'll do.}

<Mtree:10>add-conn
LITERAL1 (LEAFTYPE):  [No Default]>11.1
OBLIG1 (SYMBOL-OR-INTEGER):  [OB3]>
LITERAL2 (LEAFTYPE):  [No Default]>14
OBLIG2 (SYMBOL-OR-INTEGER):  [OB0]>
Adding new connection: (LEAF11.1 . LEAF14)
MSTREE2

{\it Again, we have created a new node of the mtree. Let's look at the
obligation tree again and see what's left to do.}

<Mtree:11>potr
                                     MSTREE2
                                      (OB0)
                                        |
                    /--------------------------------------\
                    |                                      |
                  [OB1]                                  (OB2)
                    |                                      |
          /-------------------\
          |                   |
        [OB3]               [OB4]

{\it Notice that OB3 is now closed, and hence so is OB1. We are left with OB2
as our current, and only, obligation.}

<Mtree:12>ppath*

OB2
LEAF7.1
 P x^4

OB0
|FORALL x^2         |
| |LEAF6     LEAF7| |
| |\(\sim\)R x^2 OR P x^2| |
|                   |
|FORALL x^3         |
| |LEAF11    LEAF10||
| |Q x^3  OR R x^3 ||
|                   |
|      LEAF13       |
|      \(\sim\)P X^1       |
|                   |
|      LEAF14       |
|      \(\sim\)Q X^1       |

<Mtree:13>add-conn 7.1
OBLIG1 (SYMBOL-OR-INTEGER):  [OB2]>
LITERAL2 (LEAFTYPE):  [No Default]>13
OBLIG2 (SYMBOL-OR-INTEGER):  [OB0]>
Adding new connection: (LEAF7.1 . LEAF13)
MSTREE3

<Mtree:14>potr
                                     MSTREE3
                                      [OB0]
                                        |
                    /--------------------------------------\
                    |                                      |
                  [OB1]                                  [OB2]
                    |                                      |
          /-------------------\
          |                   |
        [OB3]               [OB4]

{\it All the obligations are closed. Let's look at the matingstree itself.}

<Mtree:15>pmtr* 0

Numbers in round brackets are open obligations. If the brackets
end in `..', there are too many open obligations to fit
under the mstree label. Leaves underlined with \(\wedge\)'s are
closed matingstrees. Matingstrees enclosed in curly brackets are
marked as dead. Branches with *'s denote nodes that are being
omitted for lack of space.


                                    [MSTREE0]
                                      (OB0)
                                        |
                                        |
                                        |
                                    [MSTREE1]
                                    (3 2 1 0)
                                        |
                                        |
                                        |
                                    [MSTREE2]
                                      (2 0)
                                        |
                                        |
                                        |
                                    [MSTREE3]
                                   \(\wedge\wedge\wedge\wedge\wedge\wedge\wedge\wedge\wedge\)

{\it A very boring matingstree, not branching at all. This is because each literal
had exactly one possible mate; if there had been several mates for a literal, the
tree might have branched. Now we're ready to translate our proof into ND form.}

<Mtree:16>leave
Choose branch? [Yes]>
Merge the expansion tree? [Yes]>

{\it `Choose branch?' allows you to prune the expansion tree to contain only those
expansions which were actually part of the branch leading to the final node. In an
mtree with many branches, this is an essential part of preprocessing before the
standard merge routines are called.}

<17>etree-nat
PREFIX (SYMBOL): Name of the Proof [*****]>x2106
NUM (LINE): Line Number for Theorem [100]>
TAC (TACTIC-EXP): Tactic to be Used [COMPLETE-TRANSFORM-TAC]>
MODE (TACTIC-MODE): Tactic Mode [AUTO]>

{\it Finally, we translate the proof into natural deduction form.}
\end{tpsexample}

