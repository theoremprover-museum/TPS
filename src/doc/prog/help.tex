\chapter{Help and Documentation}
\label{help}

\section{Providing Help}

When the user types the command \indexcommand{HELP object}, \TPS will first
try to determine which category {\tt object} is in (it may be in several, in which case it will 
produce a list of categories and then print the help for each separately).

Recall from the entry about categories (section ~\ref{categories}) that each category has 
{\tt mhelp-line} and {\tt mhelp-fn} properties. The {\tt mhelp-line} is a short phrase that describes 
each object in the category (for example the category {\tt PMPROPSYM} has the mhelp line 
"polymorphic proper symbol"). The mhelp-fn is a function that will print out the help for a specific
object. For many simple categories (e.g. {\tt context}), 
the function {\tt \indexother{princ-mhelp}} is sufficient; this simply 
prints the {\tt mhelp} string attached to the object in question. Other categories need more complex
help (for example {\tt mexpr}), and so have their own specially-defined mhelp functions.

An example of a {\tt mexpr} which has some automatically constructed help information
is the function \indexcommand{EXTRACT-TEST-INFO} in \indexfile{maint.lisp}.

When writing help messages or mhelp functions, keep in mind that the information given 
should contain all the information that a {\it user} would want to know. 
More detailed help for maintainers and programmers should 
be written down and incorporated into this manual, not added into the online documentation.

\subsection{Mhelp and Scribe}
The online documentation can be used to generate a facilities guide, so
it is important that you be aware that the mhelp properties and mhelp functions
you define for new objects or categories will be used to generate Scribe
files.  Take a look at the files {\it mhelp.lisp} and {\it scrdoc.lisp} and see
how this works.  You may need to set things up properly so that the entries you
are introducing are put into the index of such guides.  Look at the
file {\it tpsdoc.mss} in the doc/lib area to see how the indexing is done.

\subsection{Mhelp and \LaTeX }
The online documentation can also be used to generate a facilities guide using \LaTeX. 
Take a look at the files {\it mhelp.lisp} and {\it latexdoc.lisp} and see
how this works. Because of the restrictions of \LaTeX concerning special command characters, such as
``\$ '' or ``\textbackslash '', the help processing may need special care: an alternative to the usual
{\tt \indexother{princ-mhelp}} function is provided to handle such characters in {\it latexdoc.lisp} as
{\tt \indexother{princ-mhelp-latex}}. Many specific \LaTeX commands are defined in the {\it tps.tex} and
{\it tpsdoc.tex} files in the doc/lib area. The User's guide describes how the processing of such manuals
is done.
You may need to set things up properly so that the entries you
are introducing are put into the index of such guides.  Look at the
file {\it latexdoc.lisp} in the lisp area to see how the indexing is done.

\section{The Info Category}

There is a category of objects called \indexother{INFO} which is used solely for providing
help on symbols that would otherwise not have help messages (for example, the various settings
of some of the flags, such as PR97 or APPLY-MATCH-ALL-FRDPAIRS). You can attach a help message
to any symbol {\it foo} with:

%\begin{tpsexample}
\begin{verbatim}
(definfo foo (mhelp "Help text."))
\end{verbatim}
%\end{tpsexample}

\section{Printed Documentation}

The directories with root {\it /home/theorem} mentioned below are
on gtps.

{\it /home/theorem/project/doc/files.dir} contains information about
\TPS documentation files.

{\it /home/theorem/project/doc/etps/tps-cs.mss} describes how to
access \TPS in the cmu cs cell.

{\it /home/theorem/project/doc/etps/etps-andrew.mss} describes how
to access \ETPS in the andrew cell.

{\it /home/theorem/project/doc/<topic>/manual.mss} is the main file
for the manual on <topic> (one of: {\tt char}, {\tt etps}, {\tt facilities}, 
{\tt grader}, {\tt prog}, {\tt teacher} and {\tt user}).

See the \TPS User Manual for additional information.

When new facilities are added to \ETPS, copy the information
about them from the automatically produced {\it facilities.mss} and {\it facilities.tex} into the
appropriate \ETPS mss/tex file.

\section{Indexing in the Manuals}

The basic mechanisms are in {\it /home/theorem/project/doc/lib/index.lib}
and {\it /home/theorem/project/doc/lib/indexcat.mss}. Note the comment on
the use of {\tt @IndexCategory} in the former file.  In the \TeX
version of the Programmer's Guide, there are indexing commands defined
which mimic the role of the corresponding Scribe commands.

{\tt @indexother{DIY-TAC}} in the text on page <pagenumber> puts
{\tt DIY-TAC <pagenumber>} into the index.

{\tt @index*X*(WORD}) in the text on page <pagenumber> puts
{\tt WORD, *Y* <pagenumber>} into the index.

Example:
{\tt @indexcommand{DO-GRADES}} in the text on page <pagenumber> puts
{\tt DO-GRADES, System Command <pagenumber>}
into the index. Here is a partial list of 
of possible values for {\tt *X*} and {\tt *Y*}, where the complete
list is in {\it /home/theorem/project/doc/lib/indexcat.mss}.

\begin{itemize}
\item {\tt *X*} = command gives {\tt *Y*} = System Command
 
\item {\tt *X*} = edop gives {\tt *Y*} = Editor Command
 
\item {\tt *X*} = flag gives {\tt *Y*} = flag
 
\item {\tt *X*} = function gives {\tt *Y*} = function
 
\item {\tt *X*} = style gives {\tt *Y*} = style

\item {\tt *X*} = mexpr gives {\tt *Y*} = mexpr
\end{itemize}

See "quitting" in the index of the ETPS manual to see the 
effect of the following lines in the file
{\it /home/theorem/project/doc/etps/system.mss}:

%\begin{tpsexample}
\begin{verbatim}
@@seealso[Primary="Quitting",Other="@{\tt EXIT}"]
@@seealso[Primary="Quitting",Other="@{\tt END-PRFW}"]
@@seealso[Primary="Quitting",Other="@{\tt OK}"]
@@indexentry[key="Quitting",entry="Quitting"]
\end{verbatim}
%\end{tpsexample}

\section{Other commands in the manuals}

Any other Scribe commands may be used in the manuals; for example
we use the {\tt typewriter} font given by @t for command 
names, and the {\it italic} font given by @i for file names.

In the \TeX versions of the manuals, one uses the corresponding
\TeX commands.

We also have @TPS in Scribe (and \verb+\TPS+ in \TeX)
to print the string "\TPS", and @HTPS in Scribe to do the
same in headers.

\section{Converting Scribe to \LaTeX ~documentation}

The aime of this section is to provide helpful information on how to
program a new documentation device.

\subsection{The {\it latexdoc.lisp} file}

This file was written as an equivalent to {\it scrdoc.lisp}. Functions and macros
are essentially equivalent. Nevertheless, while the Scribe documentation
system contains several calls to special function, such as {\tt scribe-one-fn}, which
are internal properties of special objects (e.g. tactics), {\it latex-doc.lisp} contains
every \LaTeX -specific help-formatting function.

Some of these functions uses the tex style in the \TPS system, which is described in
{\it deftex.lisp}. As the style properties were thought to be used with Scribe, some
of them cannot be easily translated in \LaTeX . This obstacle occurs for instance when
using the Scribe \verb+@Tabset+ function, which has no strict equivalent in \LaTeX , where it
is usually replaced by the \verb+Description+ environment.

\subsection{Special Characters}

The mhelp properties of many \TPS objects present special characters: when Scribe prevents the use
of \@ , a lot of characters have to be escaped, using a prefixed ``\textbackslash '', when generating a \LaTeX ~document. The most common are: \~ , \@ , \# , \$ , \% , and \& . Note that the backslash character in \LaTeX ~is
\verb+\textbackslash+ and the character ``\textasciicircum '' is \verb+\textasciicircum+.

In order to prevent the programmer from editing every mhelp property, a function {\tt princ-mhelp-latex} is used
instead of the regular {\tt princ-mhelp}. This function simply replaces every occurence of a protected character
by the correspondant \LaTeX ~sequence.

\subsection{\LaTeX ~Macros}

To facilitate the conversion from Scribe to \LaTeX ~commands, a great number of new commands and macros for \LaTeX
~are defined in {\it tps.tex} and {\it tpsdoc.tex}. These commands enable us to use commands such as \verb+\greeka+ instead of the regular \verb+\alpha+.